
%%PROBLEM_BEGIN%%
%%<PROBLEM>%%
问题1. 设 $a, b, p, q>0$. 求证:
$$
\frac{a^{p+q}+b^{p+q}}{2} \geqslant\left(\frac{a^p+b^p}{2}\right)\left(\frac{a^q+b^q}{2}\right) .
$$
%%<SOLUTION>%%
原不等式等价于 $\left(a^p-b^p\right)\left(a^q-b^q\right) \geqslant 0$.
而 $a^p-b^p$ 与 $a^q-b^q$ 同号或同为零,故结论成立.
%%PROBLEM_END%%



%%PROBLEM_BEGIN%%
%%<PROBLEM>%%
问题2. 设 $0<a_1 \leqslant a_2 \leqslant \cdots \leqslant a_n ; 0<b_1 \leqslant b_2 \leqslant \cdots \leqslant b_n$, 则
$$
\frac{\sum_{i=1}^n a_i^2 b_i}{\sum_{i=1}^n a_i b_i} \geqslant \frac{\sum_{i=1}^n a_i^2}{\sum_{i=1}^n a_i} .
$$
%%<SOLUTION>%%
原不等式等价于 $\left(\sum a_i^2 b_i\right)\left(\sum a_i\right)-\left(\sum a_i^2\right)\left(\sum a_i b_i\right) \geqslant 0$, 即
$$
\frac{1}{2} \sum \sum\left(a_i-a_j\right)\left(b_i-b_j\right) a_i a_j \geqslant 0 \text {. }
$$
%%PROBLEM_END%%



%%PROBLEM_BEGIN%%
%%<PROBLEM>%%
问题3. 设 $n$ 是给定的正整数, $n \geqslant 3$, 对于 $n$ 个给定的实数 $a_1, a_2, \cdots, a_n$, 记 $m$ 为 $\left|a_i-a_j\right|(1 \leqslant i<j \leqslant n)$ 的最小值.
求在 $a_1^2+a_2^2+\cdots+a_n^2=1$ 的条件下, $m$ 的最大值.
%%<SOLUTION>%%
设 $a_1 \leqslant a_2 \leqslant \cdots \leqslant a_n$, 则
$$
\begin{aligned}
a_j-a_i & =\left(a_j-a_{j-1}\right)+\left(a_{j-1}-a_{j-2}\right)+\cdots+\left(a_{i+1}-a_i\right) \\
& \geqslant(j-i) m .
\end{aligned}
$$
因此 $\sum_{1 \leqslant i<j \leqslant n}\left(a_i-a_j\right)^2 \geqslant \frac{1}{12} n^2(n-1)(n+1) m^2$,
又由于 $\sum_{1 \leqslant i<j \leqslant n}\left(a_i-a_j\right)^2=n\left(a_1^2+a_2^2+\cdots+a_n^2\right)-\left(a_1+a_2+\cdots+a_n\right)^2 \leqslant n$, 故 $m \leqslant \sqrt{\frac{12}{(n-1) n(n+1)}}$, 易见等号成立的条件.
%%PROBLEM_END%%



%%PROBLEM_BEGIN%%
%%<PROBLEM>%%
问题4. 求所有的正整数 $a_1, a_2, \cdots, a_n$, 使得
$$
\frac{99}{100}=\frac{a_0}{a_1}+\frac{a_1}{a_2}+\cdots+\frac{a_{n-1}}{a_n}
$$
其中 $a_0=1$, 且 $\left(a_{k+1}-1\right) a_{k-1} \geqslant a_k^2-1(k=1,2, \cdots, n-1)$.
%%<SOLUTION>%%
显然 $a_k>a_{k-1}$, 且 $a_k \geqslant 2 . k=1,2, \cdots, n-1$.
由条件可得
$$
\frac{a_{k-1}}{a_k} \leqslant \frac{a_{k-1}}{a_k-1}-\frac{a_k}{a_{k+1}-1} .
$$
上面不等式对 $k=i+1, i+2, \cdots, n$ 求和, 得
$$
\frac{a_i}{a_{i+1}}+\frac{a_{i+1}}{a_{i+2}}+\cdots+\frac{a_{n-1}}{a_n}<\frac{a_i}{a_{i+1}-1} .
$$
当 $i=0$ 时, 由题设及上式, $\frac{1}{a_1} \leqslant \frac{99}{100}<\frac{1}{a_1-1}$. 则
$$
\frac{100}{99} \leqslant a_1<\frac{100}{99}+1 \text {, 故 } a_1=2 \text {. }
$$
当 $i=1 、 2 、 3$ 时, 同样可求得 $a_2=5, a_3=56, a_4=78400$.
所以 $\frac{1}{a_5} \leqslant \frac{1}{a_4}\left(\frac{99}{100}-\frac{1}{2}-\frac{2}{5}-\frac{5}{56}-\frac{56}{25-56^2}\right)=0$, 故没有 $a_5$, 因此, 所求的正整数 $a_1=2, a_2=5, a_3=56, a_4=78400$.
%%PROBLEM_END%%



%%PROBLEM_BEGIN%%
%%<PROBLEM>%%
问题5. 设 $a_1, a_2, \cdots, a_n$ 是不全相等的 $n$ 个正数 $(n \geqslant 2)$, 且满足 $\sum_{k=1}^n a_k^{-2 n}=1$, 求证:
$$
\sum_{k=1}^n a_k^{2 n}-n^2 \sum_{1 \leqslant i<j \leqslant n}\left(\frac{a_i}{a_j}-\frac{a_j}{a_i}\right)^2>n^2
$$
%%<SOLUTION>%%
原不等式等价于 $\sum_{k=1}^n a_k^{2 n} \cdot \sum_{k=1}^n \frac{1}{a_k^{2 n}}-n^2 \sum_{i<j}\left(\frac{a_i}{a_j}-\frac{a_j}{a_i}\right)^2>n^2$.
由 Lagrange 恒等式, $\sum_{k=1}^n a_k^{2 n} \cdot \sum_{k=1}^n \frac{1}{a_k^{2 n}}-n^2=\sum_{i<j}\left(\frac{a_i^n}{a_j^n}-\frac{a_j^n}{a_i^n}\right)^2$.
不难验证, 对 $x>0$, 有 $\left(x^n-\frac{1}{x^n}\right)^2 \geqslant n^2\left(x-\frac{1}{x}\right)^2$, 因此 $\sum_{k=1}^n a_k^{2 n} \cdot \sum_{k=1}^n \frac{1}{a_k^{2 n}}- n^2 \geqslant n^2 \sum_{i<j}\left(\frac{a_i}{a_j}-\frac{a_j}{a_i}\right)^2$, 等号不能成立.
%%PROBLEM_END%%



%%PROBLEM_BEGIN%%
%%<PROBLEM>%%
问题6. 给定 $n(\geqslant 2)$ 个实数 $a_1 \leqslant a_2 \leqslant \cdots \leqslant a_n$, 令 $x=\frac{1}{n} \sum_{i=1}^n a_i, y=\frac{1}{n} \sum_{i=1}^n a_i^2$, 求证: $2 \sqrt{y-x^2} \leqslant a_n-a_1 \leqslant \sqrt{2 n\left(y-x^2\right)}$.
%%<SOLUTION>%%
解:$$
\begin{aligned}
n^2\left(y-x^2\right)= & n \sum_{j=1}^n a_j^2-\left(\sum_{j=1}^n a_j\right)^2 \\
= & (n-1) \sum_{j=1}^n a_j^2-2 \sum_{1 \leqslant i<j \leqslant n} a_i a_j \\
= & (n-1)\left(a_1^2+a_n^2\right)-2 a_1 a_n-2 a_1\left(a_2+a_3+\cdots+a_{n-1}\right) \\
& -2 a_n\left(a_2+a_3+\cdots+a_{n-1}\right)+\sum_{2 \leqslant i<j \leqslant n-1}\left(a_i-a_j\right)^2+2\left(a_2^2\right. \\
& \left.+a_3^2+\cdots+a_{n-1}^2\right) \\
= & \sum_{2 \leqslant i<j \leqslant n-1}\left(a_i-a_j\right)^2+2 \sum_{j=2}^{n-1}\left(a_j-\frac{a_1+a_n}{2}\right)^2 \\
& +\frac{n}{2}\left(a_1-a_n\right)^2 \\
\geqslant & \frac{n}{2}\left(a_n-a_1\right)^2
\end{aligned}
$$
故
$$
a_n-a_1 \leqslant \sqrt{2 n\left(y-x^2\right)} .
$$
另一方面, $n^2\left(y-x^2\right)+n \sum_{j=2}^{n-1}\left(a_n-a_j\right)\left(a_j-a_1\right)$
$$
\begin{aligned}
= & (n-1) \sum_{j=1}^n a_j^2-2 \sum_{1 \leqslant i<j \leqslant n} a_i a_j+n a_n \sum_{j=2}^{n-1} a_j+n a_1 \sum_{j=2}^{n-1} a_j-n(n \\
& -2) a_1 a_n-n \sum_{j=2}^{n-1} a_j^2 \\
= & (n-1)\left(a_1^2+a_n^2\right)-\sum_{j=2}^{n-1} a_j^2+(n-2) a_n \sum_{j=2}^{n-1} a_j+(n-2) a_1 \sum_{j=2}^{n-1} a_j \\
& -[n(n-2)+2] a_1 a_n-2 \sum_{2 \leqslant i<j \leqslant n-1} a_i a_j . \\
= & -\left[\left(\sum_{j=2}^{n-1} a_j\right)-\frac{n-2}{2}\left(a_1+a_n\right)\right]^2+\frac{n^2}{4}\left(a_1-a_n\right)^2 .
\end{aligned}
$$
由此即得 $a_n-a_1 \geqslant 2 \sqrt{y-x^2}$,
因此, $2 \sqrt{y-x^2} \leqslant a_n-a_1 \leqslant \sqrt{2 n\left(y-x^2\right)}$.
%%PROBLEM_END%%



%%PROBLEM_BEGIN%%
%%<PROBLEM>%%
问题7. 设不等式
$$
\begin{aligned}
& \left|z_1^{\prime}-z_2^{\prime}\right|+\left|z_2^{\prime}-z_3^{\prime}\right|+\cdots+\left|z_{n-1}^{\prime}-z_n^{\prime}\right| \\
\leqslant & \lambda \cdot\left[\left|z_1-z_2\right|+\left|z_2-z_3\right|+\cdots+\left|z_{n-1}-z_n\right|\right]
\end{aligned}
$$
对一切不全相等的复数 $z_1, z_2, \cdots, z_n$ 都成立.
其中 $k z_k^{\prime}=\sum_{j=1}^k z_j, k=1$, $2, \cdots, n$. 求证: $\lambda \geqslant 1-\frac{1}{n}$. 并问何时取到等号?
%%<SOLUTION>%%
$\sum_{k=1}^{n-1}\left|z_k^{\prime}-z_{k+1}^{\prime}\right|=\sum_{k=1}^{n-1}\left|\frac{1}{k} \sum_{j=1}^k z_j-\frac{1}{k+1} \sum_{j=1}^{k+1} z_j\right|$
$$
\begin{aligned}
& =\sum_{k=1}^{n-1} \frac{1}{k(k+1)}\left|\sum_{j=1}^k j\left(z_j-z_{j+1}\right)\right| \\
& \leqslant \sum_{k=1}^{n-1}\left(\frac{1}{k(k+1)} \cdot \sum_{j=1}^k j\left|z_j-z_{j+1}\right|\right) \\
& =\sum_{j=1}^{n-1} \sum_{k=j}^{n-1}\left(\frac{1}{k(k+1)} j\left|z_j-z_{j+1}\right|\right) \\
& =\sum_{j=1}^{n-1}\left(1-\frac{j}{n}\right)\left|z_j-z_{j+1}\right| \\
& \leqslant \sum_{j=1}^{n-1}\left(1-\frac{1}{n}\right)\left|z_j-z_{j+1}\right| \\
& =\left(1-\frac{1}{n}\right) \sum_{j=1}^{n-1}\left|z_j-z_{j+1}\right| .
\end{aligned}
$$
易见,当 $z_1 \neq z_2, z_2=z_3=\cdots=z_n$ 时等号成立,故命题得证.
%%PROBLEM_END%%



%%PROBLEM_BEGIN%%
%%<PROBLEM>%%
问题8. 设 $n(\geqslant 3)$ 为正整数, $x_1, x_2, \cdots, x_n$ 为实数, 对于 $1 \leqslant i \leqslant n-1$, 有 $x_i< x_{i+1}$, 求证:
$$
\frac{n(n-1)}{2} \cdot \sum_{1 \leqslant i<j \leqslant n} x_i x_j>\left[\sum_{i=1}^{n-1}(n-i) x_i\right]\left[\sum_{j=2}^n(j-1) x_j\right] .
$$
%%<SOLUTION>%%
令 $y_i==\sum_{j=i+1}^n x_j, y=\sum_{j=2}^n(j-1) x_j, z_i=\frac{n(n-1)}{2} y_i-(n-i) y(1 \leqslant i \leqslant n-1)$. 于是
$$
\begin{aligned}
& \frac{n(n-1)}{2} \cdot \sum_{1 \leqslant i<j \leqslant n} x_i x_j-\left[\sum_{i=1}^{n-1}(n-i) x_i\right] \cdot\left[\sum_{j=2}^n(j-1) x_j\right] \\
= & \frac{n(n-1)}{2} \cdot \sum_{i=1}^{n-1} x_i \sum_{j=i+1}^n x_j-\sum_{i=1}^{n-1}(n-i) x_i y \\
= & \sum_{i=1}^{n-1} x_i \cdot\left[\frac{n(n-1)}{2} y_i-(n-i) y\right]=\sum_{i=1}^{n-1} x_i z_i .
\end{aligned}
$$
下面只需证明: $\sum_{i=1}^{n-1} x_i z_i>0$.
注意到 $\sum_{i=1}^{n-1} y_i=y, \sum_{i=1}^{n-1} z_i=0, y=\sum_{j=2}^n(j-1) x_j<\sum_{j=2}^n(j-1) x_n= \frac{n(n-1)}{2} x_n, z_{n-1}=\frac{n(n-1)}{2} y_{n-1}-y=\frac{n(n-1)}{2} x_n-y>0$.
又由于 $\frac{z_{i+1}}{\frac{n(n-1)}{2}(n-i-1)}-\frac{z_i}{\frac{n(n-1)}{2}(n-i)}=\frac{y_{i+1}}{n-i-1}-\frac{y_i}{n-i}=$
$$
\frac{x_{i+2}+x_{i+3}+\cdots+x_n}{n-i-1}-\frac{x_{i+1}+x_{i+2}+\cdots+x_n}{n-i}>0,
$$
有: $\frac{z_1}{n-1}<\frac{z_2}{n-2}<\cdots<\frac{z_{n-2}}{2}<z_{n-1}$.
因此,一定存在一个正整数 $k$, 当 $1 \leqslant i \leqslant k$ 时 $z_i \leqslant 0$; 当 $k+1 \leqslant i \leqslant n$ 时, $z_i>0(k \leqslant n-2)$. 所以 $\left(x_i-x_k\right) z_i \geqslant 0$.
故 $\sum_{i=1}^{n-1} x_i z_i \geqslant x_k \sum_{i=1}^{n-1} z_i=0$, 但等号不能成立.
证毕.
%%PROBLEM_END%%



%%PROBLEM_BEGIN%%
%%<PROBLEM>%%
问题9. 设 $a_1 \geqslant a_2 \geqslant \cdots \geqslant a_n \geqslant a_{n+1}=0$ 是实数序列, 求证:
$$
\sqrt{\sum_{k=1}^n a_k} \leqslant \sum_{k=1}^n \sqrt{k}\left(\sqrt{a_k}-\sqrt{a_{k+1}}\right) .
$$
%%<SOLUTION>%%
$\left[\sum_{k=1}^n(\sqrt{k}-\sqrt{k-1}) \sqrt{a_k}\right]^2=\sum_{i=1}^n(\sqrt{i}-\sqrt{i-1})^2 a_i+2 \sum_{i<j}(\sqrt{i}- \sqrt{i-1})(\sqrt{j}-\sqrt{j-1}) \sqrt{a_i a_j} \geqslant \sum_{i=1}^n(\sqrt{i}-\sqrt{i-1})^2 a_i+2 \sum_{i<j}(\sqrt{i}- \sqrt{i-1})(\sqrt{j}-\sqrt{j-1}) a_j=\sum_{i=1}^n(\sqrt{i}-\sqrt{i-1})^2 a_i+2 \sum_{j=1}^n \sqrt{j-1}(\sqrt{j}- \sqrt{j-1}) a_j=\sum_{i=1}^n a_i$, 因此结论成立.
易见,等号成立时需存在 $m, a_1=a_2=\cdots=a_m$, 而当 $k>m$ 时, $a_k=0$. 注:下面再给出两种证法.
%%PROBLEM_END%%



%%PROBLEM_BEGIN%%
%%<PROBLEM>%%
问题9. 设 $a_1 \geqslant a_2 \geqslant \cdots \geqslant a_n \geqslant a_{n+1}=0$ 是实数序列, 求证:
$$
\sqrt{\sum_{k=1}^n a_k} \leqslant \sum_{k=1}^n \sqrt{k}\left(\sqrt{a_k}-\sqrt{a_{k+1}}\right) .
$$
%%<SOLUTION>%%
证法 2:用 Abel 求和公式能把不等式转化为:
$\sqrt{\sum_{k=1}^n a_k} \leqslant \sum_{k=1}^n \sqrt{a_k}(\sqrt{k}-\sqrt{k-1})=\sum_{k=1}^n \sqrt{a_k c_k}$, 此处 $c_k=\sqrt{k}-\sqrt{k-1}$.
由于 $a_1 \geqslant a_2 \geqslant \cdots \geqslant a_n, c_1 \geqslant c_2 \geqslant \cdots \geqslant c_n$, 诱使我们用 Chebyshev 不等式, 即
$$
\sum_{k=1}^n \sqrt{a_k} \cdot c_k \geqslant \frac{1}{n} \cdot \sum_{k=1}^n \sqrt{a_k} \cdot \sum_{k=1}^n c_k=\frac{1}{\sqrt{n}} \sum_{k=1}^n \sqrt{a_k},
$$
并无多大用处,现将欲证结论改述如下:
$$
\sqrt{\sum_{k=1}^n a_k} \leqslant \sum_{k=1}^{n-1} \sqrt{k}\left(\sqrt{a_k}-\sqrt{a_{k+1}}\right)+\sqrt{n a_n} . \label{(1)}
$$
对 $n$ 用归纳法.
当 $n=1$ 时, (1)显然成立.
假设对某个 $n \geqslant 1$, 每个非增、非负、长为 $n$ 的实数列有 (1) 成立, 考察长为 $n+1$, 满足 $a_1 \geqslant a_2 \geqslant \cdots \geqslant a_{n+1} \geqslant 0$ 的实数列.
由归纳假设, 只需证明: $\sqrt{\sum_{k=1}^{n+1} a_k}- \sqrt{\sum_{k=1}^n a_k} \leqslant-\sqrt{n a_{n+1}}+\sqrt{(n+1) a_{n+1}}$. 不妨设 $a_{n+1}>0, S=\sum_{k=1}^n a_k, m=\frac{S}{a_{n+1}}$, 则上式等价于 $\sqrt{m+1}-\sqrt{m} \leqslant \sqrt{n+1}-\sqrt{n}$, 显然成立.
%%PROBLEM_END%%



%%PROBLEM_BEGIN%%
%%<PROBLEM>%%
问题9. 设 $a_1 \geqslant a_2 \geqslant \cdots \geqslant a_n \geqslant a_{n+1}=0$ 是实数序列, 求证:
$$
\sqrt{\sum_{k=1}^n a_k} \leqslant \sum_{k=1}^n \sqrt{k}\left(\sqrt{a_k}-\sqrt{a_{k+1}}\right) .
$$
%%<SOLUTION>%%
证法 3: 令 $x_i=\sqrt{a_i}-\sqrt{a_{i+1}}, i=1,2, \cdots, n$.
则 $a_i=\left(x_i+x_{i+1}+\cdots+x_n\right)^2, 1 \leqslant i \leqslant n$.
故 $\sum_{k=1}^n a_k=\sum_{k=1}^n k x_k^2+2 \sum_{1 \leqslant k<l \leqslant n} k x_k x_l \leqslant \sum_{k=1}^n k x_k^2+2 \sum_{k<l} \sqrt{k l} x_k x_l= \left(\sum_{k=1}^n \sqrt{k} x_k\right)^2$, 因此结论成立.
%%PROBLEM_END%%



%%PROBLEM_BEGIN%%
%%<PROBLEM>%%
问题10. 证明 Chebyshev 不等式:
设 $a_1 \leqslant a_2 \leqslant \cdots \leqslant a_n, b_1 \leqslant b_2 \leqslant \cdots \leqslant b_n$, 则
$$
n \cdot \sum_{k=1}^n a_k b_k \geqslant\left(\sum_{k=1}^n a_k\right) \cdot\left(\sum_{k=1}^n b_k\right) \geqslant n \sum_{k=1}^n a_k b_{n-k+1} .
$$
%%<SOLUTION>%%
定义 $b_{n+t}=b_n, t=0,1,2, \cdots$. 则 $\sum_{i=1}^n b_{i+t}=\sum_{i=1}^n b_i, \sum_{i=1}^k b_{i+t} \geqslant \sum_{i=1}^k b_i$, $a_k-a_{k+1} \leqslant 0(1 \leqslant k \leqslant n-1)$.
由分部求和公式,
$$
\begin{aligned}
\left(\sum_{k=1}^n a_k\right) \cdot\left(\sum_{k=1}^n b_k\right) & =\sum_{t=0}^{n-1}\left(\sum_{i=1}^n a_i b_{i+t}\right) \\
& =\sum_{t=0}^{n-1}\left[a_n \sum_{i=1}^n b_{i+t}+\sum_{k=1}^{n-1}\left(\sum_{i=1}^k b_{i+t}\right)\left(a_k-a_{k+1}\right)\right] \\
& \leqslant \sum_{t=0}^{n-1}\left[a_n \sum_{i=1}^n b_i+\sum_{k=1}^{n-1}\left(\sum_{i=1}^k b_i\right)\left(a_k-a_{k+1}\right)\right] \\
& =n \cdot \sum_{k=1}^n a_k b_k .
\end{aligned}
$$
同理可证 $\left(\sum_{k=1}^n a_k\right)\left(\sum_{k=1}^n b_k\right) \geqslant n \sum_{k=1}^n a_k b_{n-k+1}$.
%%PROBLEM_END%%



%%PROBLEM_BEGIN%%
%%<PROBLEM>%%
问题11. (W. Janous 不等式推广形式)
设 $a_1, a_2, \cdots, a_n \in \mathbf{R}^{+}, p, q \in \mathbf{R}^{+}$. 记 $S=\left(a_1^p+a_2^p+\cdots+a_n^p\right)^{\frac{1}{p}}$, 则对于 $1,2, \cdots, n$ 的任一排列 $i_1, i_2, \cdots, i_n$, 有: $\sum_{k=1}^n \frac{a_k^q-a_{i_k}^q}{S^p-a_k^p} \geqslant 0$.
%%<SOLUTION>%%
由对称性, 不妨设 $a_1 \geqslant a_2 \geqslant \cdots \geqslant a_n$, 则
$$
\sum_{j=1}^k a_j^q \geqslant \sum_{j=1}^k a_{i_j}^q(1 \leqslant k \leqslant n-1), \sum_{j=1}^n a_j^q=\sum_{j=1}^n a_{i_j}^q,
$$
且 $\left(S^p-a_k^p\right)^{-1} \geqslant\left(S^p-a_{k+1}^p\right)^{-1}(1 \leqslant k \leqslant n-1)$.
由分部求和公式, 有 $\sum_{k=1}^n \frac{a_k^q-a_{k_k}^q}{S^p-a_k^p}=\left(S^p-a_n^p\right)^{-1} \cdot \sum_{j=1}^n\left(a_j^q-a_{i_j}^q\right)+ \sum_{k=1}^{n-1}\left[\sum_{j=1}^k\left(a_j^q-a_{i_j}^q\right)\right] \cdot\left[\left(S^p-a_k^p\right)^{-1}-\left(S^p-a_{k+1}^p\right)^{-1}\right] \geqslant 0$, 故原不等式成立.
%%PROBLEM_END%%



%%PROBLEM_BEGIN%%
%%<PROBLEM>%%
问题12. 设 $a_1, a_2, \cdots$ 是正实数数列, 对所有 $n \geqslant 1$ 满足条件: $\sum_{j=1}^n a_j \geqslant \sqrt{n}$, 求证: 对所有的 $n \geqslant 1$, 有 $\sum_{j=1}^n a_j^2 \geqslant \frac{1}{4}\left(1+\frac{1}{2}+\cdots+\frac{1}{n}\right)$.
%%<SOLUTION>%%
令 $b_k=a_1+a_2+\cdots+a_k-\sqrt{k} \geqslant 0,1 \leqslant k \leqslant n$, 则 $a_k=\left(b_k-b_{k-1}\right)+ (\sqrt{k}-\sqrt{k-1})$. 于是
$$
\begin{aligned}
\sum_{j=1}^n a_j^2= & {\left[b_1^2+\left(b_2-b_1\right)^2+\cdots+\left(b_n-b_{n-1}\right)^2\right]+\left[1^2+(\sqrt{2}-\sqrt{1})^2+\cdots\right.} \\
& \left.+(\sqrt{n}-\sqrt{n-1})^2\right]+\left[2 \cdot \sqrt{1} \cdot b_1+2 \cdot(\sqrt{2}-\sqrt{1}) \cdot\left(b_2-b_1\right)\right. \\
& \left.+\cdots+2(\sqrt{n}-\sqrt{n-1})\left(b_n-b_{n-1}\right)\right] \\
\geqslant & {\left[1^2+(\sqrt{2}-\sqrt{1})^2+\cdots+(\sqrt{n}-\sqrt{n-1})^2\right]+\left[2 \cdot \sqrt{1} \cdot b_1+2(\sqrt{2}-\right.} \\
& \left.\sqrt{1})\left(b_2-b_1\right)+\cdots+2(\sqrt{n}-\sqrt{n-1})\left(b_n-b_{n-1}\right)\right]
\end{aligned}
$$
$$
\begin{aligned}
& \geqslant 1^2+(\sqrt{2}-\sqrt{1})^2+\cdots+(\sqrt{n}-\sqrt{n-1})^2 \\
& >\frac{1}{4} \cdot\left(1+\frac{1}{2}+\cdots+\frac{1}{n}\right) .
\end{aligned}
$$
%%PROBLEM_END%%



%%PROBLEM_BEGIN%%
%%<PROBLEM>%%
问题13. 试证: 对任意实数 $x$, 有 $\sum_{k=1}^n \frac{[k x]}{k} \leqslant[n x]$, 其中 $[x]$ 表示不超过 $x$ 的最大整数.
%%<SOLUTION>%%
令 $A_n=\sum_{k=1}^n \frac{[k x]}{k}$, 下用数学归纳法证明: $A_n \leqslant[n x]$.
当 $n=1$ 时, 显然成立.
假设对 $1 \leqslant k \leqslant n-1$, 有 $A_k \leqslant[k x]$, 则由分部求和公式,
$$
\begin{aligned}
n A_n & =\sum_{k=1}^n k \cdot \frac{[k x]}{k}-\sum_{k=1}^{n-1} A_k(k-(k+1)) \\
& =\sum_{k=1}^n[k x]+\sum_{k=1}^{n-1} A_k \leqslant \sum_{k=1}^n[k x]+\sum_{k=1}^{n-1}[k x] \\
& =[n x]+\sum_{k=1}^{n-1}([(n-k) x]+[k x]) \\
& \left.\leqslant[n x]+\sum_{k=1}^{n-1}[(n-k) x+k x] \text { (因为 }[x]+[y] \leqslant[x+y]\right) \\
& =n[n x],
\end{aligned}
$$
故 $A_n \leqslant[n x]$, 命题得证.
%%PROBLEM_END%%



%%PROBLEM_BEGIN%%
%%<PROBLEM>%%
问题14. 已知 $a_k \geqslant 0, k=1,2, \cdots, n$. 定义 $A_k=\frac{1}{k} \cdot \sum_{i=1}^k a_i$, 求证:
$$
\sum_{k=1}^n A_k^2 \leqslant 4 \sum_{k=1}^n a_k^2
$$
%%<SOLUTION>%%
如果设 $\frac{1}{c} \sum_{k=1}^n A_k^2 \leqslant \sum_{k=1}^n A_k \cdot a_k$, 则有 $\sum_{k=1}^n A_k \cdot a_k \leqslant c \cdot \sum_{k=1}^n a_k^2$, 于是可将问题转为 Abel 方法处理:
$$
\begin{aligned}
\sum_{k=1}^n A_k a_k & =\sum_{k=1}^n A_k\left[k A_k-(k-1) A_{k-1}\right] \\
& =\sum_{k=1}^n k A_k^2-\sum_{k=1}^n(k-1) A_k A_{k-1} \\
& \geqslant \sum_{k=1}^n k A_k^2-\frac{1}{2}\left[\sum_{k=1}^n(k-1) A_k^2+\sum_{k=1}^n(k-1) A_{k-1}^2\right] \\
& =\frac{1}{2} \cdot \sum_{k=1}^n A_k^2+\frac{1}{2} n A_n^2 \geqslant \frac{1}{2} \sum_{k=1}^n A_k^2 .
\end{aligned}
$$
故
$$
\sum_{k=1}^n A_k^2 \leqslant 4 \sum_{k=1}^n a_k^2
$$
%%<REMARK>%%
注:: 由此思想可以证明 Hölder 型不等式: 设 $p>1$, 则 $\sum_{k=1}^n A_k^p \leqslant\left(\frac{p}{p-1}\right)^p$. $\sum_{k=1}^n a_k^p$, 即类似的有: $\frac{1}{c} \sum_{k=1}^n A_k^p \leqslant \sum_{k=1}^n A_k^{p-1} \cdot a_k \leqslant c^{p-1} \sum_{k=1}^n a_k^p$, 然后可证得 $c= \frac{p}{p-1}$ 是可使不等式成立的待定系数.
此题是哈代一道不等式的初等形式.
%%PROBLEM_END%%


