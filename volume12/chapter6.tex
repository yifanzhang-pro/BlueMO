
%%TEXT_BEGIN%%
哈密顿问题.
1856 年, 著名英国数学家哈密顿 (Willian Rowan Hamilton) 提出一个名为 "环游世界" 的游戏.
他用一个正十二面体的二十个顶点代表二十个大城市, 要求沿着棱, 从一个城市出发,经过每个城市恰好一次, 然后回到出发点.
这个游戏曾经风靡一时, 在这个游戏中提到了这样一种链(圈): 它经过图上各顶点一次并且仅仅一次.
这种链 (圈) 称为哈密顿链 (圈),一个图若包含哈密顿圈,则称这个图是哈密顿图.
从表面上看,哈密顿问题与欧拉问题很相似,但实际上有着本质的区别, 它是图论中尚未解决的困难问题之一.
迄今为止还没有找到判断它的充分必要条件, 所以对不同类型的问题, 有不同的判断方法, 下面通过例题作些介绍.
%%TEXT_END%%



%%TEXT_BEGIN%%
对于一个连通图, 是否存在哈密顿链 (圈) 的问题.
虽然直到最后还不知道有什么充要条件, 然而许多第一流的数学家经过一个多世纪的努力, 已经知道一些必要条件和一些充分条件, 下面给出一个简单图具有哈密顿链的充分条件.
定理二设 $G$ 是 $n(n \geqslant 3)$ 阶简单图, 且对每一对顶点 $v, v^{\prime}$ 有
$$
d(v)+d\left(v^{\prime}\right) \geqslant n-1,
$$
则图 $G$ 有哈密顿链.
%%<SOLUTION>%%
证明:先证明 $G$ 是连通图.
若 $G$ 有两个或两个以上的连通部分, 设其中之一有 $n_1$ 个顶点, 另一部分有 $n_2$ 个顶点.
分别从中各取一顶点 $v_1 、 v_2$, 则 $d\left(v_1\right) \leqslant n_1-1, d\left(v_2\right) \leqslant n_2-1$. 故
$$
d\left(v_1\right)+d\left(v_2\right) \leqslant n_1+n_2-2<n-1,
$$
这与题设矛盾, 所以 $G$ 是连通图.
现证明存在哈密顿链.
证明的方法实际上给出一种哈密顿链的构造步骤.
设在 $G$ 中有一条从 $v_1$ 到 $v_p$ 的链: $v_1 v_2 \cdots v_p$. 如果有 $v_1$ 或 $v_p$ 与不在这条链上的一个顶点相邻, 我们可扩展这条链, 使它包含这个顶点.
否则, $v_1$ 和 $v_p$ 都只与这条链上的顶点相邻, 这时存在一个圈包含顶点 $v_1, v_2, \cdots, v_p$. 假设与 $v_1$ 点相邻的顶点集是 $\left\{v_{j_1}, v_{j_2}, \cdots, v_{j_k}\right\}$, 这里 $v_{j_1}, v_{j_2}, \cdots, v_{j_k}$ 都是链 $v_1 v_2 \cdots v_p$ 中的点, 且 $p<n$.
如果 $v_1$ 与 $v_p$ 相邻, 则显然存在一个圈 $v_1 v_2 \cdots v_p v_1$.
如果 $v_1$ 和 $v_p$ 不相邻, 则必然存在一点 $v_l(2 \leqslant l \leqslant p)$ 和 $v_1$ 相邻, 而 $v_{l-1}$ 和 $v_p$ 相邻, 如图(<FilePath:./figures/fig-c6i8.png>) 所示.
因为否则 $v_p$ 最多只和 $p-k-1$ 个顶点相邻, 即排除 $v_{j_1-1} 、 v_{j_2-1} 、 \cdots 、 v_{j_k-1}$ 和 $v_p$ 自身, 这样
$$
d\left(v_1\right)+d\left(v_p\right) \leqslant k+(p-k-1)=p-1<n-1,
$$
这与假设矛盾.
因而存在 $v_1, v_2, \cdots, v_p$ 的圈 $v_1 v_l v_{l+1} \cdots v_p v_{l-1} v_{l-2} \cdots v_2 v_1$.
若 $p=n$, 实际上已存在一个哈密顿圈.
若 $p<n$, 因为 $G$ 是连通的, 所以在 $G$ 中必有一个不属于这个圈的顶点 $v^{\prime}$ 与 $v_1 v_2 \cdots v_p$ 中的某一顶点 $v_k$ 相邻, 如图(<FilePath:./figures/fig-c6i9.png>) 所示.
于是就得到一个包含 $v_1, v_2, \cdots, v_p, v^{\prime}$ 的圈: $v^{\prime} v_k v_{k+1} \cdots v_{l-1} v_p v_{p-1} \cdots v_l v_1 v_2 \cdots v_k v^{\prime}$. 不断重复上面的步骤直到存在一条具有 $n-1$ 条边的链为止.
易知定理二的条件对于图中哈密顿链的存在性只是充分的, 但并不是必要条件.
设 $G$ 是 $n$ 边形, 如图(<FilePath:./figures/fig-c6i10.png>), 其中 $n=6$, 虽然任何两个顶点的度之和是 $4<6-1$, 但在 $G$ 中有一条哈密顿链.
%%TEXT_END%%



%%TEXT_BEGIN%%
定理三(Ore, 1960) $\quad G$ 是 $n(n \geqslant 3)$ 阶简单图, 且对每一对不相邻的顶点 $v 、 v^{\prime}$ 有
$$
d(v)+d\left(v^{\prime}\right) \geqslant n,
$$
那么图 $G$ 有哈密顿圈.
证明当 $n=3$ 时, 由所给条件知 $G$ 一定是完全图 $K_3$, 命题成立.
设 $n \geqslant 4$, 用反证法证明.
设 $G$ 是有 $n$ 个顶点且满足度数条件却没有哈密顿圈的图.
不妨设 $G$ 是具有这种性质的边数最大的图, 也就是说 $G$ 添上一条边就具有哈密顿圈 (否则 $G$ 可以添加一些边, 直到不能再添为止, 加边后顶点的度数条件仍满足), 由此得出在图 $G$ 中有一条包含图中每一个顶点的哈密顿链, 记为 $v_1 v_2 \cdots v_n$. 则 $v_1$ 与 $v_n$ 不相邻, 于是
$$
d\left(v_1\right)+d\left(v_n\right) \geqslant n .
$$
那么在 $v_2, v_3, \cdots, v_{n-1}$ 中必有一点 $v_i$, 使 $v_1$ 与 $v_i$ 相邻, $v_n$ 与 $v_{i-1}$ 相邻, 如图(<FilePath:./figures/fig-c6i11.png>) 所示.
否则, 有 $d\left(v_1\right)=k$ 个点 $v_{i_1}, v_{i_2}, \cdots, v_{i_k} \left(2 \leqslant i_1 \leqslant i_2 \leqslant \cdots \leqslant i_k \leqslant n-1\right)$ 与 $v_1$ 相邻, 而 $v_n$ 与 $v_{i_1-1}, v_{i_2-1}, \cdots,v_{i_k-1}$ 都不相邻, 从而
$$
d\left(v_n\right) \leqslant n-1-k,
$$
则
$$
d\left(v_1\right)+d\left(v_n\right) \leqslant k+n-1-k=n-1<n,
$$
这与条件矛盾.
故 $G$ 存在一条哈密顿圈 $v_1 v_2 \cdots v_{i-1} v_n v_{n-1} \cdots v_i v_1$. 这又与假设矛盾.
从而命题得证.
对于完全图 $K_n(n \geqslant 3)$, 显然有哈密顿圈.
%%TEXT_END%%



%%PROBLEM_BEGIN%%
%%<PROBLEM>%%
例1. 如图(<FilePath:./figures/fig-c6i1.png>) 有无哈密顿链或哈密顿圈?
%%<SOLUTION>%%
解:按照如图(<FilePath:./figures/fig-c6i1.png>) 中所给的编号, 可以看出这样的一个圈是存在的.
这里是采用"直接求解"的方法来解决环游世界的问题, 即从图的某一个顶点出发, 采用一步步试探的方法, 来找出图的哈密顿链 (圈). 如果找到了一条解就出来了, 如果找不到, 就可能没有解.
这种方法一般只用在比较简单的图上, 而且多用在肯定有哈密顿链 (圈) 存在的情况.
%%PROBLEM_END%%



%%PROBLEM_BEGIN%%
%%<PROBLEM>%%
例2. 在一次国际数学家大会上, 7 位来自不同国家的数学家会话能力如下:
$A$ :英语.
$B$ :英语和汉语.
$C$ :英语、意大利语和西班牙语.
$D$ : 汉语和日语.
$E$ : 德语和意大利语.
$F$ : 法语、日语和西班牙语.
$G$ : 法语和德语.
问怎样安排这 7 名数学家围着一个圆桌坐下, 使得每个人都能和他身边的两个人交谈?
%%<SOLUTION>%%
解:设 7 个顶点 $A 、 B 、 C 、 D 、 E 、 F 、 G$ 对应这 7 名数学家, 其中会用同一种语言的人对应的顶点之间连一条边,这样就得到了一个图 $G_1$, 如图(<FilePath:./figures/fig-c6i2.png>) 所示.
于是原来的排座问题就变成了如图(<FilePath:./figures/fig-c6i2.png>) 中找一条哈密顿圈的问题了.
按圈上顶点的顺序来排座位,那么每个人和他相邻的两个人都能交谈.
如图(<FilePath:./figures/fig-c6i2.png>) 中用粗线画出的一个圈, 就是我们所求的解.
也就是说, 如果按照 $A 、 B 、 D 、 F 、 G 、 E 、 C$ 的顺序排座位, 每个人就都可以和他的两个邻座交谈,所采用的语言种类标明如图(<FilePath:./figures/fig-c6i3.png>) 中的对应边上.
%%PROBLEM_END%%



%%PROBLEM_BEGIN%%
%%<PROBLEM>%%
例3. 判断如图(<FilePath:./figures/fig-c6i4.png>) 所示的图 $G$ 有无哈密顿链或哈密顿圈?
%%<SOLUTION>%%
解:我们将图中某个顶点标上 $A$, 例如把点 $a$ 标上 $A$, 所有与 $a$ 相邻的点均标上 $B$, 连续不断地用 $A$ 标记所有与已标上 $B$ 相邻的点, 用 $B$ 标记所有与已标上 $A$ 的相邻的点.
直到图中所有点标记完毕.
如图(<FilePath:./figures/fig-c6i5.png>) 所示.
如果这个图 $G$ 中有一条哈密顿链, 那么它必定交替通过点 $A$ 和点 $B$, 因而点 $A$ 的数目与点 $B$ 的数目相等或相差 1. 但是图(<FilePath:./figures/fig-c6i5.png>) 中点 $A$ 有 9 个, 点 $B$ 有 7 个, 两者相差为 2 ,因此不可能有一条哈密顿链.
一般地, 对于一个偶图 $G=\left(V_1, V_2, E\right)$, 有一个简单的方法可以断定它没有哈密顿链或哈密顿圈.
定理一在偶图 $G=\left(V_1, V_2, E\right)$ 中, 如果 $\left|V_1\right| \neq\left|V_2\right|$, 那么 $G$ 一定无哈密顿圈.
如果 $\left|V_1\right|$ 与 $\left|V_2\right|$ 的差大于 1 , 那么 $G$ 一定无哈密顿链.
可以采用同例 3 完全相同的方法证明.
%%PROBLEM_END%%



%%PROBLEM_BEGIN%%
%%<PROBLEM>%%
例4. 如图(<FilePath:./figures/fig-c6i6.png>) 是半个国际象棋盘,一匹马在右下角,试问: 马能否连续地把棋盘上所有的格都跳到一次并且仅仅一次? 如果去掉了棋盘对角上的两个黑色方格, 又将怎样?
%%<SOLUTION>%%
解:我们考虑这样的图:将棋盘方格对应于图的顶点,如果马从棋盘上的一个方格跳一次后能到另一个方格, 就在这两个方格所对应的顶点之间连上一条边.
于是问题就转化为判断棋盘所对应的这个图是否有一条哈密顿链.
在图中, 顶点是否相邻是由马跳的方式决定的, 也就是说每个顶点只能跟和它组成一个"日"字的对角线上的顶点相邻.
棋盘上组成"日"字对角线的方格所着的颜色正好是相反的,让图上每个顶点涂上它所对应的方格颜色.
这样, 图上每条边所相邻的两个顶点的颜色都是一黑 一白,并且黑、白顶点的个数是相等的, 这就可能有哈密顿链.
用试探的方法可以找到一条链.
如图(<FilePath:./figures/fig-c6i7.png>) 所示的就是一个答案.
\begin{tabular}{|c|c|c|c|c|c|c|c|}
\hline 15 & 18 & 7 & 22 & 11 & 28 & 5 & 24 \\
\hline 8 & 21 & 16 & 27 & 6 & 23 & 2 & 29 \\
\hline 17 & 14 & 19 & 10 & 31 & 12 & 25 & 4 \\
\hline 20 & 9 & 32 & 13 & 26 & 3 & 30 & 1 \\
\hline
\end{tabular}
现在来看问题的第二部分.
仍采用涂色法将问题转化为求对应的图是否存在哈密顿链的问题.
由于黑顶点个数是 14 , 白顶点个数是 16 , 根据定理一, 这个图中没有哈密顿链.
即对于去掉两个黑色方格的半个棋盘来说, 马是无法连续地把每个方格都跳到一次并且仅仅一次.
%%PROBLEM_END%%



%%PROBLEM_BEGIN%%
%%<PROBLEM>%%
例5. $n$ 个人参加一次会议, 在会议期间, 每天都要在一张圆桌上共进晚餐.
如果要求每次晚餐就座时, 每个人相邻就座者都不相同, 问这样的晚餐最多能进行多少次?
%%<SOLUTION>%%
证明:用 $n$ 个点表示 $n$ 个人,作完全图 $K_n$,如图(<FilePath:./figures/fig-c6i12.png>) 所示,则 $K_n$ 中的一个哈密顿圈就是一次晚餐的就座方法.
可见, 晚餐最多能进行的次数就是 $K_n$ 中无公共边的哈密顿圈的个数.
$K_n$ 中有 $\frac{1}{2} n(n-1)$ 条边, 每个哈密顿圈有 $n$ 条边, 因此, 边不相重的哈密顿圈最多有 $\left[\frac{n-1}{2}\right]$ 个.
当 $n=2 k+1$ 时, 将顶点 $0,1,2, \cdots, 2 k$ 排列如图.
先取一个哈密顿圈 $(0,1,2,2 k, 3,2 k-1,4, \cdots$, $k+3, k, k+2, k+1,0)$, 然后绕 0 点依次顺时针旋转 $\frac{\pi}{k}, \frac{2 \pi}{k}, \cdots,(k-1) \frac{\pi}{k}$, 共产生 $k=\left[\frac{n-1}{2}\right]$ 个无公共边的哈密顿圈, 如果 $n=2 k+2$, 那么每次在中间添加一个顶点 $v$, 同样有 $k$ 个哈密顿圈.
由定理三可以推得下面的定理四.
这是 1952 年数学家 Dirac 给出的.
定理四 $G$ 是 $n(n \geqslant 3)$ 阶简单图, 如果每个顶点 $v$ 的度 $d(v) \geqslant \frac{n}{2}$, 则图 $G$ 一定存在哈密顿圈.
%%PROBLEM_END%%



%%PROBLEM_BEGIN%%
%%<PROBLEM>%%
例6. 在 7 天内安排 7 门课的考试, 要使得同一位教师所教的两门课考试不排在接连的两天里.
如果每一位教师教的考试课最多 4 门, 证明这种安排是可能的.
%%<SOLUTION>%%
证明:设 $G$ 是有 7 个顶点的图, 每个顶点对应一门考试课, 如果两个顶点对应的考试课是由不同教师担任的, 则在这两个顶点之间连一条边, 因为这个教师所教的课不超过 4 门, 故每个顶点的度数至少是 3. 任意两个顶点的度数之和至少是 6 , 根据定理二, $G$ 有一条哈密顿链.
由于这条链上任何一条边相邻的两顶点对应的两门课不是同一位教师教的, 所以, 可以按照这条哈密顿链上的顶点顺序安排这 7 门课的考试.
%%PROBLEM_END%%



%%PROBLEM_BEGIN%%
%%<PROBLEM>%%
例7. 某工厂生产由 6 种不同颜色的纱织成的双色布.
已知花布品种中, 每种颜色至少分别和其他 3 种不同的颜色搭配.
试证可以挑出 3 种双色布, 它们恰好含有 6 种不同的颜色.
%%<SOLUTION>%%
证明:用 6 个顶点表示 6 种颜色的纱,若两种颜色的纱能搭配织成一种双色布, 就在相应的顶点之间连一条边, 这样就得到一个图 $G$. 已知条件是, 每种颜色的纱至少和其他三种颜色的纱搭配, 也即对任意顶点 $v_i, d\left(v_i\right) \geqslant 3$. 欲证的是,图 $G$ 中存在三条边, 其中任意两条边都没有公共端点.
因为对图 $G$ 中任一顶点 $v_i, d\left(v_i\right) \geqslant 3$, 根据定理四, $G$ 有哈密顿圈, 记为 $v_1 v_2 v_3 v_4 v_5 v_6 v_1$, 则边 $\left(v_1, v_2\right),\left(v_3, v_4\right),\left(v_5, v_6\right)$ 就是三条两两没有公共端点的边.
我们往往用一个图满足充分条件来肯定这个图是哈密顿图,而用一个图不满足必要条件来否定这个图是哈密顿图.
下面给出一个图是哈密顿图的必要条件.
定理五如果图 $G$ 有哈密顿图, 从 $G$ 中去掉若干个点 $v_1, v_2, \cdots, v_k$ 及与它们关联的边得到图 $G^{\prime}$, 那么 $G^{\prime}$ 的连通分支不超过 $k$ 个.
证明设 $c$ 是图 $G$ 中的哈密顿圈, 将 $k$ 个顶点以及与它们关联的边去掉后, $c$ 最多分为 $k$ 段, 因此 $G^{\prime}$ 的连通分支至多为 $k$ 个.
%%PROBLEM_END%%



%%PROBLEM_BEGIN%%
%%<PROBLEM>%%
例8. 证明如图(<FilePath:./figures/fig-c6i13.png>) 没有哈密顿圈.
%%<SOLUTION>%%
证明:如图(<FilePath:./figures/fig-c6i13.png>) 中将顶点 $v_1, v_2$ 以及与它们关联的边去掉, 得到的 $G^{\prime}$ 有三个连通分支, 不满足定理五的必要条件,所以图 (<FilePath:./figures/fig-c6i13.png>) 没有哈密顿圈.
最后再举一个例子作为本节的结束.
%%PROBLEM_END%%



%%PROBLEM_BEGIN%%
%%<PROBLEM>%%
例9. 若 $A_0 A_1 A_2 \cdots A_{2 n-1}$ 为一个正 $2 n$ 边形,连接它的所有对角线, 这样得到一个图 $G$. 证明: 图 $G$ 的每个哈密顿圈都必定包含两条边, 它们在图中是平行线.
%%<SOLUTION>%%
证明:设 $A_i A_j$ 与 $A_k A_l$ 平行, 由于 $A_i$ 与 $A_l$ 之间和 $A_j$ 与 $A_k$ 之间顶点数相同, 所以 $i-l=k-j$, 于是得到 $A_i A_j$ 与 $A_k A_l$ 平行的充要条件是
$$
i+j \equiv k+l(\bmod 2 n) \text {. }
$$
设 $A_{i_0} A_{i_1} \cdots A_{i_{2 n-1}}$ 是一个哈密顿圈 $\left(i_0, i_1, \cdots, i_{2 n-1}\right.$ 是 $0,1, \cdots, 2 n-1$ 的一个排列), 而其中任意的两条边在图中不是平行线.
因此 $i_0+i_1, i_1+i_2, i_2+ i_3, \cdots, i_{2 n-1}+i_0$ 这 $2 n$ 个数中的任意两个关于模 $2 n$ 都不同余, 即上述这 $2 n$ 个数是模 $2 n$ 的一组完全剩余系.
于是
$$
\begin{aligned}
& \left(i_0+i_1\right)+\left(i_1+i_2\right)+\cdots+\left(i_{2 n-1}+i_0\right) . \\
\equiv & 0+1+2+\cdots+2 n-1=2 n^2-n \\
\equiv & n(\bmod 2 n) .
\end{aligned}
$$
另一方面, 有
$$
\begin{aligned}
& \left(i_0+i_1\right)+\left(i_1+i_2\right)+\cdots+\left(i_{2 n-1}+i_0\right) \\
= & 2\left(i_0+i_1+i_2+\cdots+i_{2 n-1}\right) \\
= & 2(0+1+2+\cdots+2 n-1) \\
= & 2 n(2 n-1) \\
\equiv & 0(\bmod 2 n) .
\end{aligned}
$$
上面得出了两个矛盾的结果, 从而命题得证.
%%PROBLEM_END%%


