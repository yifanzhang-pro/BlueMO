
%%TEXT_BEGIN%%
欧拉问题到
欧拉 (Euler) 问题起源于著名的七桥游戏.
位于欧洲的哥尼斯堡 (Königsberg), 景致迷人,碧波荡漾的普莱格尔 (Pregel) 河横贯其境, 河中有两个岛 $A$ 与 $D$, 河上有七座桥连接这两个岛及河的两岸 $B 、 C$ (如图(<FilePath:./figures/fig-c5i1.png>) 所示).
问:一个旅游者能否通过每座桥一次且仅一次?
这便是著名的哥尼斯堡七桥问题.
天才的欧拉, 以其独具的慧眼, 看出了这个似乎是趣味几何问题的潜在意义.
1736 年, 他发表了题为《哥尼斯堡的七座桥》的论文, 解决了七桥问题.
通常认为这是图论的第一篇论文.
欧拉把图(<FilePath:./figures/fig-c5i1.png>)变成一个图 $G$, 如图(<FilePath:./figures/fig-c5i2.png>) 所示.
岛 $A 、 D$ 及河岸 $B 、 C$ 变成 4 个顶点,图 $G$ 中的 7 条边表示七座桥.
于是七桥游戏就变成了一笔画问题: 能否一笔画出这个图,每条边都无遗漏也无重复地画到? 而一笔画出这个图并未一定要求最后回到原来的出发点,也就是说, 图 $G$ 能否一笔画成(且最后回到原出发点)的问题就等价于这个图是不是一个链(圈).
如果图 $G$ 是一条从 $v_1$ 到 $v_{n+1}$ 的链, 那么每一个不同于 $v_1$ 及 $v_{n+1}$ 的顶点 $v_i(i=2,3, \cdots, n)$ 都是偶顶点.
因为对顶点 $v_i$ 来说, 有一条进人 $v_i$ 的边就有一条从 $v_i$ 引出的边, 而且进、出的边不能重复已走过的边.
所以与 $v_i$ 相邻的边总是成双的.
故图 $G$ 至多有两个奇顶点, 即 $v_1$ 与 $v_{n+1}$. 如果 $G$ 是一个圈, 根据上面的推理, $v_1$ 与 $v_{n+1}$ 也是偶顶点.
因此, 如果图 $G$ 是一个链 (圈), 那么 $G$ 的奇顶点的个数等于 2 (等于 0 ). 这就是图 $G$ 是一条链(圈)的必要条件.
换句话说, 如果图 $G$ 的奇顶点个数大于 2 , 那么图 $G$ 就不是一条链, 从而不能一笔画.
如图(<FilePath:./figures/fig-c5i2.png>) 中, $A 、 B 、 C 、 D$ 都是奇顶点, 因而这个图不是一条链, 所以不能一笔画出.
也就是说一个旅游者要既无重复也无遗漏地走过如图(<FilePath:./figures/fig-c5i1.png>) 中的七座桥是不可能的.
下面给出"一笔画定理".
定理一有限图 $G$ 是一条链或圈 (即可以一笔画成) 的充要条件是: $G$ 是连通的, 并且奇顶点个数等于 0 或 2 . 当且仅当奇顶点的个数等于 0 时, 连通图 $G$ 是一个圈.
证明必要性上面已经证明了,下证充分性.
如果 $G$ 连通,奇顶点的个数为 0 , 则 $G$ 一定是一个圈.
从 $G$ 中任一顶点 $v_0$ 出发, 经关联的边 $e_1$ 进人 $v_1$, 因为 $d\left(v_1\right)$ 是偶数, 由 $v_1$ 再经关联的边 $e_2$ 可进人 $v_2$, 如此继续下去, 每条边仅取一次, 经过若干步后必可回到 $v_1$, 于是得到一个圈 $\mu_1: v_0 v_1 \cdots v_0$.
如果 $\mu_1$ 恰好是图 $G$, 则命题得证.
否则在 $G$ 中去掉 $\mu_1$ 后得子图 $G_1$, 则 $G_1$ 中每个顶点也都是偶顶点.
因图 $G$ 是连通的, 所以在 $G_1$ 中必定存在一个和 $\mu_1$ 公共的顶点 $u$, 在 $G_1$ 中存在一个从 $u$ 出发到 $u$ 的一个圈 $\mu_2$, 于是 $\mu_1$ 和 $\mu_2$ 合起来仍是一个圈.
重复上述过程, 因为 $G$ 中总共只有有限条边, 总有一个时侯, 得到的圈恰好是图 $G$.
如果 $G$ 连通,奇顶点个数为 2. 不妨设 $u 、 v$ 是两个奇顶点, 在 $u 、 v$ 之间连一条边 $e$ 得图 $G^{\prime}$. 于是 $G^{\prime}$ 中奇顶点个数为 0 , 故 $G^{\prime}$ 是一个圈.
从而去掉 $e$ 后, $G$ 便是一条链.
进一步有如下问题,若一个连通图 $G$ 的奇顶点个数不是 0 或 2 , 那么要多少笔才能画成呢? 我们已经知道, 一个图中的奇顶点个数是偶数, 于是有如下结论.
定理二如果连通图 $G$ 有 $2 k$ 个奇顶点,则图 $G$ 可以用 $k$ 笔画成,并且至少要用 $k$ 笔才能画成.
证明把这 $2 k$ 个奇顶点分成 $k$ 对: $v_1, v_1^{\prime} ; v_2, v_2^{\prime} ; \cdots, v_k, v_k^{\prime}$, 在每对点 $v_i, v_i^{\prime}$ 之间添加一条边 $e_i$, 得图 $G^{\prime}$. 图 $G^{\prime}$ 没有奇顶点, 所以 $G^{\prime}$ 是一个圈.
再把这 $k$ 条新添的边去掉, 这个圈至多分为 $k$ 段, 即 $k$ 条链.
这说明图 $G$ 是可以用 $k$ 笔画成的.
设图 $G$ 可以分成 $h$ 条链,由定理一,每条链上至多有两个奇顶点,所以
$$
2 h \geqslant 2 k \text {, }
$$
即 $h \geqslant k$. 图 $G$ 至少要用 $k$ 笔才能画成.
%%TEXT_END%%



%%TEXT_BEGIN%%
如图(<FilePath:./figures/fig-c5i5.png>) 中的图需 5笔才能画成, 如图(<FilePath:./figures/fig-c5i6.png>) 中的图 4 笔可以画成.
%%TEXT_END%%



%%PROBLEM_BEGIN%%
%%<PROBLEM>%%
例1. 如图(<FilePath:./figures/fig-c5i3.png>) 所示是一幢房子的平面图形, 前门进人是一个客厅, 由客厅可通向四个房间.
如果现在你由前门进去, 能否通过所有的门走遍所有的房间和客厅, 然后从后门走出, 而且要求每扇门只能进出一次?
%%<SOLUTION>%%
解:答案是否定的.
把 5 个房间以及前门外面和后门外面作为顶点, 两个地方有门相通就在相应的顶点之间连一条边, 得到图 $G$, 如图(<FilePath:./figures/fig-c5i4.png>) 所示.
在图 $G$ 中, 奇顶点的个数是 4 , 故 $G$ 不是一条链.
所以问题的答案是否定的.
%%PROBLEM_END%%



%%PROBLEM_BEGIN%%
%%<PROBLEM>%%
例3. 如图(<FilePath:./figures/fig-c5i7.png>) 所示,在 $8 \times 8$ 黑白方格的棋盘上跳动一只马, 不论跳动方向如何, 要使这只马跳遍棋盘的每一格且每格只经过一次, 问这是否可能? (一只马跳动一次是指从 $2 \times 3$ 黑白方格组成的长方形的一个对角跳到另一个对角上)
%%<SOLUTION>%%
解:如图(<FilePath:./figures/fig-c5i8.png>) 中给出这个问题的一个解答.
\begin{tabular}{|l|l|l|l|l|l|l|l|}
\hline 56 & 41 & 58 & 35 & 50 & 39 & 60 & 33 \\
\hline 47 & 44 & 55 & 40 & 59 & 34 & 51 & 38 \\
\hline 42 & 57 & 46 & 49 & 36 & 53 & 32 & 61 \\
\hline 45 & 48 & 43 & 54 & 31 & 62 & 37 & 52 \\
\hline 20 & 5 & 30 & 63 & 22 & 11 & 16 & 13 \\
\hline 29 & 64 & 21 & 4 & 17 & 14 & 25 & 10 \\
\hline 6 & 19 & 2 & 27 & 8 & 23 & 12 & 15 \\
\hline 1 & 28 & 7 & 18 & 3 & 26 & 9 & 24 \\
\hline
\end{tabular}
解决这类问题, 常常用以下 4 种方法尝试:
1. 每次将马放到使它能走到的 (尚未走过的)方格为最少的位置, 即先走"出路"少的方格,后走"出路"多的方格.
2. 将棋盘分为几个部分, 在每个部分中找一条哈密顿链 (见第六节), 然后把它们连接起来.
3. 在棋盘上找几个圈,然后将这些圈连接起来.
4. 将一个较小的棋盘镶上边, 产生一个大棋盘上的哈密顿链.
%%PROBLEM_END%%



%%PROBLEM_BEGIN%%
%%<PROBLEM>%%
例4. 如图(<FilePath:./figures/fig-c5i9.png>) 中,甲、乙两只蚂蚁分别处在 $A$ 、 $B$ 两点.
甲蚂蚁向乙蚂蚁提出: "咱俩比赛,看谁先把这个图中的九条边都爬遍后到达 $E$ 点.
乙蚂蚁同意.
假定两只蚂蚁爬的速度相同且同时开始.
问: 究竟哪只蚂蚁最先到达 $E$ 点?
%%<SOLUTION>%%
解:把 $A 、 B 、 C 、 D 、 E$ 作为顶点, 原来的九条线段作为 9 条边, 得图 $G$. 则 $G$ 是连通图, 且奇顶点个数为 2 , 根据定理一, 它是一条链.
由于点 $B$ 是奇顶点, $E$ 也是奇顶点,所以从 $B$ 到 $E$ 存在着一条链,例如 $B C D A C E A B D E$,
对乙蚂蚁来说, 它可以从 $B$ 点出发, 沿着这条链到达 $E$ 点.
但顶点 $A$ 是偶顶点, 从 $A$ 出发到达 $E$ 不可能不重复地走遍图 $G$ 中的所有边, 它至少要重复经过某一条边.
所以蚂蚁乙可以选择一条正确路线比蚂蚁甲先爬到 $E$ 点.
%%PROBLEM_END%%



%%PROBLEM_BEGIN%%
%%<PROBLEM>%%
例5.. 凸 $n$ 边形及 $n-3$ 条在形内不相交的对角线组成的图形称为一个剖分图.
求证: 当且仅当 $3 \mid n$ 时,存在一个剖分图是可以一笔画的圈(即可以从一个顶点出发,经过图中各线段恰一次,最后回到出发点). 
%%<SOLUTION>%%
证明:先用数学归纳法证明充分性.
$n=3$ 时,命题显然成立.
设对任何凸 $3 k$ 边形,存在一个剖分图是可以一笔画的圈.
对一个凸 $3(k+1)=3 k+3$ 边形 $A_1 A_2 A_3 \cdots A_{3 k+3}$, 连接 $A_4 A_{3 k+3}$. 由于 $A_4 A_5 \cdots A_{3 k+3}$ 是凸 $3 k$ 边形, 根据归纳假设, $A_4 A_5 \cdots A_{3 k+3}$ 存在一个剖分图是一个可以一笔画的圈, 作此剖分图,并连接 $A_2 A_4, A_2 A_{3 k+3}$,于是便得到一个凸 $3 k+3$ 边形 $A_1 A_2 A_3 \cdots A_{3 k+3}$ 的剖分图.
因 $A_4 A_5 \cdots A_{3 k+3}$ 的剖分图是一个圈, 故从 $A_{3 k+3}$ 出发, 经过这个剖分图中的每一条边恰一次后可回到 $A_{3 k+3}$, 再经 $A_{3 k+3} A_1$,
$A_1 A_2, A_2 A_3, A_3 A_4, A_4 A_2, A_2 A_{3 k+3}$ 后, 又回到 $A_{3 k+3}$, 这就证明了凸 $3 k+3$ 边形 $A_1 A_2 A_3 \cdots A_{3 k+3}$ 也存在一个剖分图是一个可以一笔画的圈.
于是充分性得证.
再证必要性.
因为一个凸 $n$ 边形存在剖分图是可以一笔画的圈, 则它的每个顶点都是偶顶点.
显然凸四边形和凸五边形不存在每个顶点都是偶顶点的剖分图.
从而当 $3 \leqslant n<6$ 时, 如果凸 $n$ 边形存在每个顶点都是偶顶点的剖分图, 则 $n=3$.
设当 $3 \leqslant n<3 k(k>2)$ 时, 如果凸 $n$ 边形存在每个顶点都是偶顶点的剖分图, 则 $3 \mid n$. 现考虑 $3 k \leqslant n<3(k+1)$ 的情况.
设凸 $n$ 边形 $A_1 A_2 \cdots A_n$ 有一个每个顶点都是偶顶点的剖分图.
易知任意凸 $n(n>3)$ 边形的任何剖分, 都把此凸 $n$ 边形分割成没有公共内部的 $n-2$ 个三角形, 而且这些三角形中至少有两个以这个凸 $n$ 边形两条相邻边为两边.
因此不妨设 $A_1 A_3$ 是凸 $n$ 边形 $A_1 A_2 \cdots A_n$ 剖分图中的一条对角线(如图(<FilePath:./figures/fig-c5i11.png>) 所示). 于是 $A_1 A_3$ 还是剖分图中另一个 $\triangle A_1 A_3 A_i$ 的一边.
由假设 $A_1 A_2 \cdots A_n$ 的剖分图使得每个顶点都是偶顶点, 故 $i \neq 4$, 否则 $A_3$ 是奇顶点.
同样 $i \neq n$, 否则 $A_1$ 为奇顶点.
因此 $4<i<n$. 由 $A_1 A_2 \cdots A_n$ 的这个剖分图分别给出凸 $i-2$ 边形 $A_3 A_4 \cdots A_i$ 和凸 $n-i+2$ 边形 $A_1 A_2 \cdots A_n$ 的剖分图,而且这两个剖分图所对应的凸多边形的每个顶点都是偶顶点.
因此, 根据归纳假设知
$$
3|i-2,3| n-i+2,
$$
所以 $3 \mid n$. 从而必要性得证.
必要性也可用涂色方法来证.
对凸 $n$ 边形的一个剖分图, 可以对其中的三角形涂上两种颜色, 使得有公共边的两个三角形涂有不同的颜色.
这可以这样进行: 有顺序地引出对角线, 每条对角线将多边形的内部划分为两部分, 其中一部分保持原来的颜色, 而另一部分改变颜色, 这个过程直到所需的对角线全部引出,便得到所需的涂色.
因为凸 $n$ 边形有剖分图是一笔画的圈,故每个顶点都是偶顶点.
这样在每个顶点处的三角形个数是奇数.
于是在上述涂色下多边形的所有边属于同色的三角形,不妨设为黑色的 (如图(<FilePath:./figures/fig-c5i12.png>)). 用 $m$ 表示白色三角形的边数,显然 $3 \mid m$, 每个白三角形的边同时也是黑三角形的边, 而多边形的所有边是黑三角形的边, 故黑三角形的边数为 $m+n$. 由 $3 \mid m+n$, 便得 $3 \mid n$.
%%PROBLEM_END%%



%%PROBLEM_BEGIN%%
%%<PROBLEM>%%
例6. 设 $n>3$, 考虑在同一圆周上的 $2 n-1$ 个互不相同的点所成的集合 $E$. 将 $E$ 中一部分点染成黑色, 其余的点不染色.
如果至少有一对黑点, 以它们为端点的两条弧中有一条的内部(不包含端点) 恰含 $E$ 中 $n$ 个点, 则称这种染色方式为"好的". 如果将 $E$ 中 $k$ 个点染黑的每一种染色方式都是好的,求 $k$ 的最小值.
%%<SOLUTION>%%
解:将 $E$ 中的点按逆时针方向依次用顶点 $v_1, v_2, \cdots, v_{2 n-1}$ 表示, 并在顶点 $v_i$ 与 $v_{i+(n+1)}$ 之间连一条边, $i=1,2, \cdots, 2 n-1$. (我们约定 $v_{j+(2 n-1) k}= v_j, k \in \mathbf{Z}$ ). 这样便得一个图 $G, G$ 中每个顶点的度为 2 (即与两个点相邻), 并且 $v_i$ 与 $v_{i+3}$ 与同一个点相邻.
由于 $G$ 中的每个点都是偶顶点, 所以 $G$ 是由一个或几个圈所组成的.
(i) 当 $3 \mid(2 n-1)$ 时, 图 $G$ 由三个圈组成,每个圈的顶点集为
$$
\begin{aligned}
& \left\{v_i \mid i=3 k, k=1,2, \cdots, \frac{2 n-1}{3}\right\}, \\
& \left\{v_i \mid i=3 k+1, k=0,1, \cdots, \frac{2 n-4}{3}\right\}, \\
& \left\{v_i \mid i=3 k+2, k=0,1, \cdots, \frac{2 n-4}{3}\right\} .
\end{aligned}
$$
由于每个圈上的顶点数都是 $\frac{2 n-1}{3}$, 故每个圈上至多可以取出 $\frac{1}{2}\left(\frac{2 n-1}{3}-1\right)=\frac{n-2}{3}$ 个点,两两互不相邻 (注意 $\frac{2 n-1}{3}$ 是奇数). 总共可以取出 $n-2$ 个点互不相邻.
由抽屉原则, 至少要染黑 $n-1$ 个点, 才能保证至少有一对黑点相邻.
(ii) 当 $3 \nmid(2 n-1)$ 时, $v_1, v_2, \cdots, v_{2 n-1}$ 中的每一个点都可以表为 $v_{3 k}$ 的形式, 因此图 $G$ 是一个长为 $(2 n-1)$ 的圈.
在这圈上可以取出 $n-1$ 个互不相邻的点, 而且至多可以取出 $n-1$ 个互不相邻的点.
因而至少要染黑 $n$ 个点, 才能保证至少有一对黑点相邻.
综上所述, 当 $3 \nmid(2 n-1)$ 时, $k$ 的最小值是 $n$, 当 $3 \mid(2 n-1)$ 时, $k$ 的最小值为 $n-1$.
%%PROBLEM_END%%


