
%%PROBLEM_BEGIN%%
%%<PROBLEM>%%
问题1. 如果连通图 $G$ 的顶点数 $\geqslant 2$, 则 $G$ 中至少存在两个顶点, 将这两个顶点及其关联的边去掉后,图仍然连通 (没有顶点的"图"也看作是连通图).
%%<SOLUTION>%%
$G$ 的生成树有两个悬挂点, 去掉这两点后, 图仍然连通.
%%PROBLEM_END%%



%%PROBLEM_BEGIN%%
%%<PROBLEM>%%
问题2. 坐标纸上的 11 条纵线和 11 条横线构成一个图,图的顶点是纵、横线的交点 (格点), 边是格点间的纵、横线段, 问至少应当去掉多少条边才能使每点的度 $<4$ ? 至多可以去掉多少条边还能使图保持连通?
%%<SOLUTION>%%
其中有 $9 \times 9=81$ 个度为 4 的顶点,故至少要去掉 $\left[\frac{81}{2}\right]+1=41$ 条边, 才能使每点的度 $<4$. 至多可以去掉 $2 \times 11 \times 10-120=100$ 条边, 还可以使图保持连通.
%%PROBLEM_END%%



%%PROBLEM_BEGIN%%
%%<PROBLEM>%%
问题3. 若图 $G$ 有 $n$ 个顶点, $n-1$ 条边, 则 $G$ 为树.
这个命题正确吗? 为什么?
%%<SOLUTION>%%
命题不正确, 取 $K_3$ 与一个孤立点 (和图中其他点都不相邻的点) 组成图 $G$. 则 $G$ 有 4 个顶点 3 条边, 它不连通, 显然不是树.
%%PROBLEM_END%%



%%PROBLEM_BEGIN%%
%%<PROBLEM>%%
问题4. 已知一棵树 $T$ 有3个3度顶点,一个2度定点, 其余的都是悬挂点.
1) $T$中有几个悬挂点? (2)试画出两棵满足上述度数要求的不同构的树.
%%<SOLUTION>%%
(1) 设 $T$ 中有 $x$ 个悬挂点, 则 $T$ 的顶点数 $n=3+1+x$, 边数 $e=n- 1=x+3$. $\sum_{i=1}^n d\left(v_i\right)=3 \times 3+2 \times 1+1 \times x=11+x$, 故 $11+x=2(x+3)$, $x=5$.
(2) 如图(<FilePath:./figures/fig-c4a4.png>)所示的两棵树均满足要求,但它们不同构.
%%PROBLEM_END%%



%%PROBLEM_BEGIN%%
%%<PROBLEM>%%
问题5. 一棵树有 $n_i$ 个顶点的度数为 $i, i=1,2, \cdots, k, n_2, \cdots, n_k$ 均为已知数, 问 $n_1$ 应为多少? 若 $n_r(3 \leqslant r \leqslant k)$ 末知, $n_j(j \neq r)$ 均已知, 问 $n_r$ 应为多少?
%%<SOLUTION>%%
设 $T$ 有 $n$ 个顶点, $e$ 条边, 则 $n=\sum_{i=1}^k n_i, e=n-1, \sum_{i=1}^n d\left(v_i\right)=\sum_{i=1}^k \dot{m}_i= 2 e=2 n-2=2 \sum_{i=1}^k n_i-2$, 所以, $n_1=\sum_{i=2}^k(i-2) n_i+2$.
对于 $r \geqslant 3$, 由上面的式子可知 $n_r=\frac{1}{r-2}\left[\sum_{i \neq r}^k(2-i) n_i-2\right]$.
%%PROBLEM_END%%



%%PROBLEM_BEGIN%%
%%<PROBLEM>%%
问题6. 设 $d_1, d_2, \cdots, d_n$ 是 $n$ 个正整数, $n \geqslant 2$, 已知 $\sum_{i=1}^n d_i=2 n-2$. 证明存在一棵顶点度数分别为 $d_1, d_2, \cdots, d_n$ 的树.
%%<SOLUTION>%%
$d_1, d_2, \cdots, d_n$ 中至少有两个为 1 (否则, $\sum_{i=1}^n d_i \geqslant 2 n-1$ ). 对顶点数 $n$ 用数学归纳法.
$n=2$ 时显然成立.
设结论在 $n=k$ 时成立.
当 $n=k+1$ 时在 $d_1, d_2, \cdots, d_k, d_{k+1}$ 中存在为 1 的数, 不妨设 $d_{k+1}=1$. 易知这 $k+1$ 个数中存在大于等于 2 的数, 设为 $d_k$. 考虑 $d_1, d_2, \cdots, d_{k-1},\left(d_k-1\right)$ 这 $k$ 个数, $d_1+\cdots+d_{k-1}+\left(d_k-1\right)=2(k+1)-2-1-1=2 k-2$, 由归纳假设知, 存在树 $T^{\prime}$, 其顶点为 $v_1, \cdots, v_k, \sum_{i=1}^k k\left(v_i\right)=d_1+\cdots+d_{k-1}+\left(d_k-1\right)=2 k-$ 2. 从 $T^{\prime}$ 中顶点 $v_k$ 引出一条边与 $v_{k+1}$ 相邻, 得树 $T$, 则 $\sum_{i=1}^{k+1} d\left(v_i\right)=2 k-2+1+ 1=2(k+1)-2$. 所以 $T$ 即为所求.
%%PROBLEM_END%%



%%PROBLEM_BEGIN%%
%%<PROBLEM>%%
问题7. 平面上有 $n(n \geqslant 3)$ 条线段, 其中任意 3 条都有公共端点, 则这 $n$ 条线段有一个公共端点.
%%<SOLUTION>%%
作图 $G, n$ 条线段的端点为 $G$ 的顶点, 线段为 $G$ 的边, 依题意, $G$ 连通无圈, 故 $G$ 是树, 且最长的链长度为 2 , 故 $G$ 只有一个顶点不是悬挂点, 这点即为 $n$ 条线段的公共端点.
%%PROBLEM_END%%



%%PROBLEM_BEGIN%%
%%<PROBLEM>%%
问题8. 一个 $n$ 行 $n$ 列的数表 (矩阵), 每两行都不完全相同.
证明一定存在某一列, 去掉这一列后, 每两行仍然不完全相同.
%%<SOLUTION>%%
本题参见本节例 7 .
%%PROBLEM_END%%



%%PROBLEM_BEGIN%%
%%<PROBLEM>%%
问题9. 记图 $G$ 的所有顶点集合为 $V$, 所有边的集合为 $E$. 求证: 若 $|E| \geqslant|V|+$ 4 , 则 $G$ 中必有两个无公共边的圈.
(Pösá 定理) 
%%<SOLUTION>%%
假设命题不成立, 则必然存在着反例.
我们考察其中 $|E|+|V|$ 最小的一个反例.
在这个反例中, 一定有 $|E|=|V|+4$ (不然可以把多的边去掉, 这时所得的图仍是一个反例, 而 $|E|+|V|$ 变小, 矛盾!), 则 $|E|>|V|$. 图中必存在一个圈, 则最短圈长至少为 5 . (不然最短圈长不大于 4 , 则把这个圈去掉之后, 仍将有 $|E| \geqslant|V|$, 从而图中仍存在圈.
而这个圈与前一个圈无公共边, 矛盾!) 另外, 图中每一个顶点的度数至少为 3 (不然, 若某点度数为 2 , 则把该点去掉, 它连出的两条边连成一条边, 仍有 $|E|=|V|+4$, 而 $|E|+ |V|$ 变小, 矛盾! 若某点度数为 1 , 则把该点及其连出的边去掉, 仍有 $|E|= |V|+4$, 而 $|E|+|V|$ 变小,矛盾!若存在孤立点, 则把孤立点去掉, $|E|>|V|+4$, 而 $|E|+|V|$ 变小,矛盾!)
取一个最短圈 $C_0$, 其长度至少为 5 , 则圈上至少有 5 个点.
对于 $C_0$ 上各点, 每一点至少与圈外连出一点, 且各连出的点互不相同 (否则将出现长度小于 5 的圈), 这样易知 $|V| \geqslant 2 \times 5=10$. 另一方面, $2|E|=\sum_{v \in V} d(V) \geqslant \sum_{v \in V} 3= 3|V|$, 而 $|E|=|V|+4$, 故 $2|V|+8 \geqslant 3|V|,|V| \leqslant 8$, 矛盾!
因此反例不存在, 原命题得证.
%%<REMARK>%%
注:: 此题的解法就是所谓的"极端原理". 要证明某个命题成立, 用反证法, 反设其不成立, 考察其中某个变量 $V \in N$, 从 $V$ 的最小反例中推出矛盾,多了 $V$ 最小这个条件使证明难度降低了.
题目的结论是最佳的, 当 $|E|=|V|+3$ 时, 可举出反例如图(<FilePath:./figures/fig-c4a9.png>)所示.
%%PROBLEM_END%%



%%PROBLEM_BEGIN%%
%%<PROBLEM>%%
问题10. 某天晚上21个人之间通了电话,有人发现这 21 人共通话102次,且每两人至多通话一次.
他还发现, 存在 $m$ 个人,第 1 个人与第 2 个人通了话, 第 2 个人与第 3 个人通了话, ......., 第 $m-1$ 个人与第 $m$ 个人通了话, 第 $m$ 个人又与第 1 个人通了话,他不肯透露 $m$ 的具体值, 只说 $m$ 是奇数.
求证 21 个人中必存在 3 人,他们两两通了话.
%%<SOLUTION>%%
用 21 个点表示 21 个人,两点之间有 1 条连线当且仅当这两个点代表的人通了电话.
由已知, 存在 1 个长度为 $m$ 的奇圈 (长度为奇数的圈称为奇圈). 设图中长度最短的奇圈为 $C$, 长度为 $2 k+1$.
若 $k=1$, 则 $C$ 为三角形, 其代表的 3 人两两通了电话.
若 $k>1$. 设 $C$ 为 $v_1 v_2 \cdots v_{2 k+1} v_1$, 则 $v_i, v_j$ 之间没有连线 $(1 \leqslant i, j \leqslant 2 k+ 1, i-j \neq \pm 1(\bmod 2 k+1)$. 否则, 设 $v_i, v_j$ 相连, 则圈 $v_1 v_2 \cdots v_i v_j \cdots v_{2 k+1} v_1$ 与圈 $v_i v_{i+1} \cdots v_j v_i$ 长度之和为 $2 k+3$, 故其中必有一个长度小于 $2 k+1$ 的奇数, 这与 $C$ 最短矛盾.
若除 $v_1, v_2, \cdots, v_{2 k+1}$ 之外的 $21-(2 k+1)=20-2 k$ 个点无三角形, 由托兰定理, 它们至多连了 $(10-k)^2$ 条边.
又其中任一点不与 $C$ 的相邻两点连 (否则便有三角形), 所以它至多与 $C$ 中 $k$ 个点相连, 故总的边数
$$
\begin{aligned}
& 2 k+1+k(20-2 k)+(10-k)^2 \\
= & 100+2 k+1-k^2 \\
= & 102-(k-1)^2 \\
\leqslant & 102-(2-1)^2=101,
\end{aligned}
$$
矛盾!
故图中必有三角形存在, 即存在 3 个人, 他们两两通了电话.
%%PROBLEM_END%%



%%PROBLEM_BEGIN%%
%%<PROBLEM>%%
问题11. 某国有若干个城市, 某些城市之间有道路相连, 由每个城市连出 3 条道路.
证明: 存在一个由道路形成的圈,它的长度不能被 3 整除.
%%<SOLUTION>%%
假定存在这样的图, 它的每个顶点的度数都大于 2 , 但该图中的任何一个圈的长度都可被 3 整除.
我们来考察具有这种性质的顶点数目最小的图 G. 显然, 该图中存在着长度最小的圈 $Z$, 该圈上的任意两个不相邻的顶点之间没有边相连, 又因每一顶点的度数都大于 2 , 所以圈 $Z$ 上的每个顶点都有一边与圈外顶点相连, 设圈 $Z$ 依次经过顶点 $A_1, A_2, \cdots, A_{3 k}$. 假定存在连接顶点 $A_m$ 和 $A_n$ 的不包含圈 $Z$ 上的边的路径 $S$. 我们来分别考察由路径 $S$ 和 $Z$ 的 "两半"所组成的圈 $Z_1$ 和 $Z_2$. 由于这两个圈的长度都可被 3 整除, 不难推知路径 $S$ 的长度可被 3 整除.
特别地, 对题目中所给出的图, 可知它的任何一个不在 $Z$ 上的顶点 $X$, 都不可能有边与 $Z$ 的两个不同顶点分别相连.
即, 由圈 $Z$ 上的顶点所连出的不在圈上的边, 应分别连向各不相同的顶点.
我们来作另外一个图 $G_1$, 把图 $G$ 中圈 $Z$ 上的所有顶点 $A_1, A_2, \cdots, A_{3 k}$ 合并为一个顶点 $A$, 保留所有不在圈 $Z$ 上的顶点及它们之间所连的边, 且分别用边将 $A$ 同原来与 $Z$ 上的顶点有边相连的顶点逐一相连, 易知 $A$ 的度数 $\geqslant 3 k$. 于是, 图 $G_1$ 中的顶点数目少于图 $G$, 而每个顶点的度数仍都大于 2. 于是, 按照前面所证的结论, 图 $G_1$ 中的任何一个圈的长度都可被 3 整除.
这样一来, 我们便得出了矛盾: 因为如前所言, 图 $G$ 是具有这种性质的顶点数目最小的图.
这样一来, 在任何所有顶点的度数都大于 2 的图中, 必定存在长度不能被 3 整除的圈, 接下来只需把这一断言应用于我们的题目, 并以城市作为顶点, 以道路作为边即可.
%%PROBLEM_END%%



%%PROBLEM_BEGIN%%
%%<PROBLEM>%%
问题12. 一条河的两岸有一些城市, 城市的总数不少于 3 个.
城市由一些航线连接着, 每条航线将位于两岸的一对城市联系在一起, 每个城市恰好与另一边的 $k$ 个城市连接.
人们可以在任何两座城市之间往来.
证明: 如果航线中有一条被取消,人们还是可以在任何两座城市之间往来.
%%<SOLUTION>%%
不妨称河的两岸分别为北岸与南岸.
北岸的 $n$ 个城市用点 $x_1$, $x_2, \cdots, x_n$ 表示, 其全体记为 $X=\left\{x_1, x_2, \cdots, x_n\right\}$; 南岸的 $m$ 个城市用点 $y_1, y_2, \cdots, y_m$ 表示, 其全体记为 $Y=\left\{y_1, y_2, \cdots, y_m\right\}$. 如果北岸的城市 $x_i$ 与南岸的城市 $y_j$ 之间有航线, 则连接成为边 $\left(x_i, y_i\right)$, 所有的边组成的集合记为 $E$. 这就得到了一个由顶点集 $X 、 Y$ 与边集 $E$ 构成的图, 称为 2 部分图, 也称为偶图, 记为 $G=(X, Y ; E)$. 题中的后两个条件即是: 由任一顶点引出的边都是 $k$ 条; 图 $G$ 是连通的, 即任意两个顶点之间都有由若干条边连接而成的路.
题目的结论是: 从 $E$ 中删去任意一条边 $e$, 图 $G^{\prime}=(X, Y ; E-e)$ 仍然是连通的.
因为每个顶点恰与 $k$ 条边相关联, 所以有
$$
|X| k=|E|=|Y| k,
$$
其中 $|X|,|E|,|Y|$ 表示集合 $X, E, Y$ 中元素的个数.
于是有 $|X|=|Y|$, 即 $n=m$. 又因 $|X|+|Y| \geqslant 3$, 所以 $|X|=|Y| \geqslant 2$.
现去掉 $G$ 的一条边, 得到的图为 $G^{\prime}$. 若 $G^{\prime}$ 不连通, 则 $G^{\prime}$ 由两个连通部分 $G_1, G_2$ 构成.
设
$$
\begin{aligned}
& X=X_1 \cup X_2, X_1 \cap X_2=\varnothing \\
& Y=Y_1 \cup Y_2, Y_1 \cap Y_2=\varnothing . \\
& G_1=\left(X_1, Y_1 ; E_1\right), G_2=\left(X_2, Y_2 ; E_2\right) .
\end{aligned}
$$
去掉的一条边是连接 $X_1$ 与 $Y_2$ 的顶点, 则
$$
\begin{aligned}
& \left|X_1\right| k-1=\left|E_1\right|=\left|Y_1\right| k, \\
& \left|X_2\right| k=\left|E_2\right|=\left|Y_2\right| k-1,
\end{aligned}
$$
从而 $\left(\left|X_1\right|-\left|Y_1\right|\right) k=1$, 得 $k=1$.
又 $G$ 连通, 则 $|X|=1$,与 $|X| \geqslant 2$ 矛盾.
故 $G^{\prime}$ 连通, 从而结论成立.
%%PROBLEM_END%%


