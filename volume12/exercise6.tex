
%%PROBLEM_BEGIN%%
%%<PROBLEM>%%
问题1. $n$ 为何值时, 完全图 $K_n$ 是哈密顿图? 对什么样的 $m, n$, 完全偶图 $K_{m, n}$ 是哈密顿图?
%%<SOLUTION>%%
当 $n \geqslant 3$ 时, $K_n$ 为哈密顿图.
当 $m=n \geqslant 2$ 时, 完全偶图 $K_{m, n}$ 是哈密顿图.
%%PROBLEM_END%%



%%PROBLEM_BEGIN%%
%%<PROBLEM>%%
问题2. 证明: 对于正四面体、正六面体、正八面体及正二十面体所表示的图是哈密顿图.
%%<SOLUTION>%%
请读者自己在图上找.
%%PROBLEM_END%%



%%PROBLEM_BEGIN%%
%%<PROBLEM>%%
问题3. 正二十面体用纸制成, 能否把它剪成两部分, 使每个面也剪成两部分, 而截痕不通过二十面体的顶点? 个 $1 \times 1 \times 1$ 子立方体.
如果它在一个角上开始, 然后依次走向未吃的立方体, 问它吃完时能否恰在立方体的中心?
%%<SOLUTION>%%
如图(<FilePath:./figures/fig-c6a3.png>), 正二十面体是由 20 个全等正三角形围成的.
在每个正三角形的中心设一个顶点, 仅当两个正三角形有公共边时, 在相应的两顶点间连一边, 则构成由 12 个正五边形围成的正十二面体.
从哈密顿周游世界的游戏中已知可从正十二面体上找到一个哈密顿圈, 用剪刀沿此哈密顿圈剪开, 即可把正二十面体剪成两块,且把每个正三角形也剪成两块,截痕不过二十面体顶点.
%%PROBLEM_END%%



%%PROBLEM_BEGIN%%
%%<PROBLEM>%%
问题4. 一只老鼠吃 $3\times 3\times 3$ 立方体的乳酪,其方法是借助于打洞通过所有的 27 个$1\times 1\times 1$ 子立方体.
果它在一个角上开始;然后依次走向未吃的立方体,问它吃完时能否恰在立方体的中心?
%%<SOLUTION>%%
作图 $G$ : 顶点表示 $1 \times 1 \times 1$ 的立方体, 当且仅当两小立方体有公共面时, 对应的两顶点以边相连.
易知 $G$ 是偶图, 设 $G=(X, Y ; E)$, 若角上一小立方体对应点属于 $X$, 则中心的小立方体对应的点属于 $Y$, 又因为 $|X|=14$, $|Y|=13$. 故 $G$ 无哈密顿链.
%%PROBLEM_END%%



%%PROBLEM_BEGIN%%
%%<PROBLEM>%%
问题5. 今要将 6 人分成 3 组 (每组 2 个人)去完成 3 项任务.
已知每个人至少与其余 5 个人中的 3 个人能相互合作.
(1) 能否使得每组的 2 个人都能相互合作?(2)你能给出几种不同的分组方案? 
%%<SOLUTION>%%
(1)用 6 个点 $v_1, v_2, \cdots, v_6$ 代表六个人, 若 $v_i, v_j$ 能相互合作, 则 $v_i$ 与 $v_j$ 相邻.
由已知条件知, $d\left(v_i\right) \geqslant 3, i=1,2, \cdots, 6$. 根据定理四, $G$ 中有一个哈密顿圈 $C=v_{i_1} v_{i_2} \cdots v_{i_3} v_{i_1}$, 在圈中, 相邻的两点代表的两人是能相互合作的.
(2) 将 $v_{i_1}, v_{i_2}$ 分在一组, $v_{i_3}, v_{i_4}$ 分在一组, $v_{i_5}, v_{i_6}$ 分在一组, 也可将 $v_{i_6}$, $v_{i_1}$ 分在一组, $v_{i_2}, v_{i_3}$ 分在一组, $v_{i_4}, v_{i_5}$ 分在一组, 这是两种不同的分组方案.
%%PROBLEM_END%%



%%PROBLEM_BEGIN%%
%%<PROBLEM>%%
问题6. 某国王有 $2n$个大臣,其中某些大臣互相有怨仇,但每个大臣的仇人(限于大臣内部)不超过 $n-1$ 人 (互为仇人), 问能否让他们围圆桌而坐, 使仇人不相邻?
%%<SOLUTION>%%
用 $2 n$ 个顶点表示 $2 n$ 个大臣, 若两人不是仇人, 则相应的顶点相邻, 得图 $G$. 在 $G$ 中, 每个顶点 $v$ 的度 $d(v) \geqslant(2 n-1)-(n-1)=n$, 根据定理四, $G$ 中有哈密顿圈, 按圈上顶点的顺序安排位置即可.
%%PROBLEM_END%%



%%PROBLEM_BEGIN%%
%%<PROBLEM>%%
问题7. 已知在 9 个小孩中,每个小孩至少认识其他 4 个小孩,能否让这 9 个小孩排成一行, 使得每个小孩和与他相邻的小孩都认识? 
%%<SOLUTION>%%
作图 $G: 9$ 个顶点表示 9 名小孩, 两小孩认识, 则相应的顶点相邻.
在 $G$ 中, 任意两个顶点 $v$ 与 $v^{\prime}$, 有 $d(v)+d\left(v^{\prime}\right) \geqslant 8$, 根据定理二, $G$ 有哈密顿链.
按链上顶点顺序把 9 名小孩排成一行即可.
%%PROBLEM_END%%



%%PROBLEM_BEGIN%%
%%<PROBLEM>%%
问题8. 一位厨师用8种原料做菜,每种菜都用2种原料搭配.
已知每种原料都至少用在4种菜里,问:能否从这位厨师做的菜中选出4种,恰好包括了 8 种不同的原料?
%%<SOLUTION>%%
作图 $G$ : 顶点代表原料, 每种菜对应于一条边.
在 $G$ 中, 每个顶点的度 $\geqslant 4$, 根据定理四, $G$ 中有一个哈密顿圈.
%%PROBLEM_END%%



%%PROBLEM_BEGIN%%
%%<PROBLEM>%%
问题9. 一个有限集合的全部子集, 可以如此排列, 使任何相邻的两个子集恰相差一个元素.
%%<SOLUTION>%%
设集合 $A$ 有 $n$ 个元素, 把每个元素编号, 设 $A=\{1,2,3, \cdots, n\}$, 我们用一个长度为 $n$ 的由 0 与 1 构成的序列来表达每个子集, 规则是 $A$ 的元素 $i$ 在该子集中, 则在序列的第 $i$ 位上写 1 , 否则写 0 . 例如空集 $\varnothing=0,0,0, \cdots, 0$. $\{1\}=1,0,0, \cdots, 0,\{n\}=0,0, \cdots, 1,\{2,3\}=0,1,1,0, \cdots, 0$, 则 $A$ 的全部子集共有 $2^n$ 个.
以这 $2^n$ 个子集对应的序列为顶点, 仅当两序列只一个同位数码相异时, 在此二顶点间连一边, 得一个图 $G$, 例如 $n=1$ 时为图(<FilePath:./figures/fig-c6a9-1.png>) 所示的单位线段, $n=2$ 时, $G$ 为图(<FilePath:./figures/fig-c6a9-2.png>) 所示的正方形, 图(<FilePath:./figures/fig-c6a9-2.png>) 是两个图(<FilePath:./figures/fig-c6a9-1.png>) 如下作成的: 在一个的 0,1 码前方都加上 0 , 变成 00,01 , 在另一个的 0,1 码前都加上 1 , 变成 10,11 , 再把一个放在另一个上方, 连上两个"坚边" 作成的一个正方形.
复制两个图 2 , 把一个放在另一个的上方, 把上方的各顶点标志码前方都加一个 0 , 把下方各顶点标志码前方都加一个 1 , 再用 4 条坚线连接上下相对的顶点构成 $n=3$ 时的图 $G, G$ 是一个立方体.
如果 $n=k$ 的图 $G$ 已作好, 则把 $n=k$ 的图及其复制品分别放在上方和下方各一个,再把上方的图的各顶点标码前方都加上一个 0 , 把下方的图的各顶点标码前方都加上一个 1 , 把上、下两方对应的顶点连一条竖直的边, 则得 $n=k+1$ 的图 $G$. 图(<FilePath:./figures/fig-c6a9-3.png>) 是 $n=3$ 的情形, 图(<FilePath:./figures/fig-c6a9-4.png>) 是 $n=4$ 的情形, $n= k$ 的图 $G$ 称为 $n$ 维立方体图.
用数学归纳法容易证明 $n$ 维立方体图有哈密顿圈 $(n \geqslant 2)$. 对于 $n=1$, 显然成立, 因为这时 1 维立方体图是 $K_2$, 是一条哈密顿链.
对于 $n=2$, 是四边形, 显然是哈密顿圈, 对于 $n=k$, 若是哈密顿图, 考虑 $n=k+1$, 把 $G$ 中上方和下方的 $n==k$ 时的哈密顿圈上各删去一条对应边, 再把这两条对应边的对应端点间的两条边选来与上下方的哈密顿链并
成一个 $n=k+1$ 时的哈密顿圈, 如图 4 中的粗线所示.
把 $G$ 中的顶点按哈密顿圈上的顺序放在一个圆周上, 从任一顶点出发, 沿逆时针 (或顺时针) 为序, 则把全部子集排了序, 使得相邻子集恰相差一个元素.
%%PROBLEM_END%%



%%PROBLEM_BEGIN%%
%%<PROBLEM>%%
问题10. 平面上 $n$ 个点和若干条边所成的图不是哈密顿图, 但若任意去掉一点及与之相连的边, 则剩下的图为哈密顿图.
求 $n$ 的最小值.
%%<SOLUTION>%%
首先, 每个点的度至少为 3 , 不然存在一点 $A$ 仅连出至多两边, 则把其中一边去掉后, 剩下的 $A$ 点必不在某个圈上, 这与条件不符.
因此 $n \geqslant 3$.
不难证得 $n \neq 4,5,6$.
若 $n=7$, 则去掉其中度数最大的点 (显然该点度数至少为 3 ), 得到一个长度为 6 的圈.
由于与该点相邻的点在圈上必不相邻 (否则将出现长度为 7 的圈), 于是被去掉的点至多能与圈上三个互不相邻的点相连, 因此该点度数至多为 3 , 从而可知图中各点度数均恰为 3 , 而 $3 \times 7=21$ 是奇数, 而实际上各点度数之和应为偶数,矛盾!
若 $n=8$, 则去掉度数最大的点后, 得到一个长度为 7 的圈.
被去掉的点至多能与圈上三个互不相邻的点相连, 因此它的度数至多为 3 . 由此可知各点度数均为 3. 如图(<FilePath:./figures/fig-c6a10-1.png>), $A, C, F, O$ 的度数已经为 3, 不能再连出任何边.
而 $B, D, E$, $G$ 每点还要各连出一条边.
若 $B$ 与 $G$ 相连, 则 $D$ 与 $E$ 相连 (连有两条边), 不可能.
若 $B$ 与 $D$ 相连, 则 $E$ 与 $G$ 相连, 而此时图中存在长度为 8 的圈, 矛盾! 若 $B$ 与 $E$ 相连, 则 $D$ 与 $G$ 相连, 而此时图中也存在长度为 8 的圈, 矛盾!
若 $n=9$, 由于 $3 \times 9=27$ 不是偶数,因此不可能每点都是 3 度,故至少有一点, 度数至少为 4 . 我们把度数最大的点去掉, 得到一个长度为 8 的圈, 因此被去掉的点至多能与圈上 4 个互不相邻的点相连, 因此图中的点度数最大为 4 , 最小为 3. 如图(<FilePath:./figures/fig-c6a10-2.png>), 则点 $B$ 至少要再连出一条边.
显然 $B$ 不能与 $A, C$ 再连边.
若 $B$ 与 $D$ 连边, 则图中存在长度为 9 的圈, 同理 $B$ 不能与 $H$ 相连.
若 $B$ 与 $F$ 相连, 则图中也将存在长度为 9 的圈, 矛盾! 故 $B$ 只可能与 $E$ 或 $G$ 连边.
由对称性可设 $B$ 与 $E$ 相连.
仿上讨论可知 $H$ 与 $C$ 相连 (若 $H$ 与 $E$ 相连, 则 $E$ 的度数已达到 5 , 矛盾!), $F$ 与 $A$ 相连, $D$ 与 $G$ 相连.
此时图中任何两点不可能再连边, 则去掉点 $A$ 后, 图中应存在长度为 8 的圈.
而事实上, 若图中存在长度为 8 的圈, 则 $B E, B C, H G, H C, F E, F G$ 必在圈上, 而该 6 边已构成一个长度为 6 的圈, 矛盾!
由以上讨论可知, 满足要求的 $n$ 值至少为 10 .
$n=10$ 的例子如图(<FilePath:./figures/fig-c6a10-3.png>), 称 Peterson 图.
%%PROBLEM_END%%



%%PROBLEM_BEGIN%%
%%<PROBLEM>%%
问题11. 求证:围着圆桌至少坐着五个人,那么一定可以调整他们的座位, 使得每人两侧出现新的邻座.
%%<SOLUTION>%%
若恰为五人, 设原来的座次是 $A B C D E A$ : 调整成 $A D B E C A$ 即可.
若超过五人, 以人为顶点, 仅当两人原来不是邻座时, 在此二顶点间连一边, 得图 $G$. 由于每个顶点的度数都是 $|V(G)|-3$, 于是任意两个顶点度数之和为 $2 n-6, n$ 是顶点数.
又 $n>5$, 故 $2 n-6 \geqslant n$, 由定理三, $G$ 中有哈密顿圈, 按圈上的次序请各人人席即可.
%%PROBLEM_END%%


