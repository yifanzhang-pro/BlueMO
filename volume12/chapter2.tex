
%%TEXT_BEGIN%%
图 $G$ 中与顶点 $v$ 关联的边数 (约定环计两次) 称为图 $G$ 中顶点 $v$ 的度 (或次数), 记作 $d_G(v)$. 在不致混淆的时候, 简记为 $d(v)$. 我们用 $\delta(G)$ 与 $\Delta(G)$ 分别表示 $G$ 中顶点的最小度和最大度,分别简记为 $\delta$ 和 $\Delta$.
如图(<FilePath:./figures/fig-c2i1.png>) 中, $d\left(v_1\right)=1, d\left(v_2\right)=3, d\left(v_3\right)=d\left(v_4\right)=2, \delta=1, \Delta=3$.
图 $G$ 中, 若顶点 $v$ 的度是奇数, 则称点 $v$ 为奇顶点; 若顶点 $v$ 的度是偶数, 则称点 $v$ 为偶顶点.
图 $\dot{2}-1$ 中, 点 $v_1, v_2$ 是奇顶点, 点 $v_3, v_4$ 是偶顶点.
在图 $G=(V, E)$ 中,如果对任意的 $v \in V$, 均有 $d(v)=k$, 则称图 $G$ 是 $k$ 正则的.
完全图 $K_n$ 是 $(n-1)$ 正则图.
如图(<FilePath:./figures/fig-c2i2.png>) 中是一个 3 正则图.
关于图 $G$ 中所有顶点的度之和与边数之间有如下结论.
定理一设 $G$ 是 $n$ 阶图, 则 $G$ 中 $n$ 个顶点的度之和等于边数的两倍.
记 $G$ 中 $n$ 个顶点为 $v_1, v_2, \cdots, v_n$, 边数为 $e$, 则
$$
d\left(v_1\right)+d\left(v_2\right)+\cdots+d\left(v_n\right)=2 e .
$$
证明所有顶点的度的和 $d\left(v_1\right)+d\left(v_2\right)+\cdots+d\left(v_n\right)$ 表示以顶点 $v_1$, $v_2, \cdots, v_n$ 中某个顶点为一个端点的边的总数.
由于一条边有两个端点, 因此图 $G$ 的每条边在和 $d\left(v_1\right)+d\left(v_2\right)+\cdots+d\left(v_n\right)$ 中被计人两次.
所以所有顶点的度的和为边数的两倍.
例如如图(<FilePath:./figures/fig-c2i1.png>) 中, $e=4, d\left(v_1\right)+d\left(v_2\right)+d\left(v_3\right)+d\left(v_4\right)=1+3+2+ 2=8=2 e$.
定理一通常称为握手引理, 在二百多年前欧拉就给出了这样一个著名的结论: 如果许多人在见面时握了手, 那么握手的次数为偶数.
进而推得: 握过奇数次手的人有偶数个.
这个推论就是定理二 对于任意的图 $G$, 奇顶点的个数一定是偶数.
证明设 $G$ 中的顶点为 $v_1, v_2, \cdots, v_n$, 且 $v_1, \cdots, v_t$ 是奇顶点, $v_{t+1}$, $\cdots, v_n$ 是偶顶点.
由定理一,
$$
\begin{aligned}
& d\left(v_1\right)+\cdots+d\left(v_t\right)+d\left(v_{t+1}\right)+\cdots+d\left(v_n\right)=2 e, \\
& d\left(v_1\right)+\cdots+d\left(v_t\right)=2 e-d\left(v_{t+1}\right)-\cdots-d\left(v_n\right) .
\end{aligned}
$$
因为 $d\left(v_{t+1}\right), \cdots, d\left(v_n\right)$ 都是偶数, 故上式右边是偶数, 而 $d\left(v_1\right), \cdots, d\left(v_t\right)$ 都是奇数, 要使它们的和为偶数, $t$ 必须是偶数.
即 $G$ 中奇顶点个数为偶数.
%%TEXT_END%%



%%PROBLEM_BEGIN%%
%%<PROBLEM>%%
例1. 是否存在这样的多面体,它有奇数个面,每个面有奇数条棱?
%%<SOLUTION>%%
解:不存在这样的多面体.
事实上, 如果这样的多面体存在, 那么用顶点表示这个多面体的面, 并且仅当 $v_i, v_j$ 所代表的两个面有公共棱时, 在图 $G$ 相应的两顶点之间连一条边, 依题意 $d(v)$ 是奇数, 于是奇数个奇数也是奇数.
与定理相违.
证毕.
%%PROBLEM_END%%



%%PROBLEM_BEGIN%%
%%<PROBLEM>%%
例2. 如图所示, 大三角形的三个顶点分别涂以 $A 、 B 、 C$ 三种颜色.
在大三角形内取若干个点, 将它分为若干个小三角形, 每两个小三角形或者有一条公共边, 或者有一个公共点, 或者完全没有公共点.
将每个小三角形的顶点也分别涂以 $A 、 B 、 C$ 三种颜色之一, 证明不管怎样涂色, 都有一个小三角形, 它的三个顶点的颜色全不相同.
%%<SOLUTION>%%
证明:在大三角形外及小三角形内部各取一点作顶点, 当两个面有一条公共边 $A B$ 时, 就在相应的两个顶点之间连一条边, 得图 $G^{\prime}$, 如图(<FilePath:./figures/fig-c2i3.png>) 所示.
一个具有颜色 $A 、 B 、 C$ 顶点的小三角形对应于 $G^{\prime}$ 中的度为 1 的顶点.
其余的小三角形均对应于 $G$ 中度为 0 或 2 的顶点.
由于大三角形外部的一个顶点 $u$ 的度是 1 , 且奇顶点的个数为偶数, 所以 $G^{\prime}$ 中除了 $u$ 外, 至少还有一个奇顶点 $v$. 这就是说如图(<FilePath:./figures/fig-c2i3.png>) 中至少有一个小三角形, 它的三个顶点分别为 $A 、 B 、 C$ 三种颜色.
%%<REMARK>%%
注:: 本题常见的做法是赋值法, 结合奇偶分析来解决.
这里的解法有图论的独到之处,非常简洁.
%%PROBLEM_END%%



%%PROBLEM_BEGIN%%
%%<PROBLEM>%%
例3. 某地区网球俱乐部的 20 名成员举行 14 场单打比赛, 每人至少上场一次.
证明: 必有六场比赛, 其中 12 个参赛者各不相同.
%%<SOLUTION>%%
证明:用 20 个顶点 $v_1, v_2, \cdots, v_{20}$ 代表 20 名成员,两名选手比赛过, 则在相应的顶点之间连一条边, 得图 $G$.
图 $G$ 中有 14 条边, 设各顶点的度为 $d_i, i=1,2, \cdots, 20$. 由题意知 $d_i \geqslant$ 1. 根据定理一
$$
d_1+d_2+\cdots+d_{20}=2 \times 14=28 .
$$
在每个顶点 $v_i$ 处抹去 $d_i-1$ 条边, 由于一条边可能同时被其两端点抹去, 所以抹去的边数不超过
$$
\left(d_1-1\right)+\left(d_2-1\right)+\cdots+\left(d_{20}-1\right)=28-20=8 .
$$
故抹去了这些边后所得的图 $G^{\prime}$ 中至少还有 $14-8=6$ 条边, 并且 $G^{\prime}$ 中每个顶点的度至多是 1 . 从而这 6 条边所相邻的 12 个顶点是各不相同的.
即这 6 条边所对应的 6 场比赛的参赛者各不相同.
%%PROBLEM_END%%



%%PROBLEM_BEGIN%%
%%<PROBLEM>%%
例4. 设 $S=\left\{x_1, x_2, \cdots, x_n\right\}$ 是平面上的点集, 其中任意两点之间的距离至少是 1 , 证明: 最多有 $3 n$ 对点,每对点的距离恰好是 1 .
%%<SOLUTION>%%
证明:取这 $n$ 个点作为顶点,两顶点相邻当且仅当两点之间的距离为 1 , 得一个图 $G . G$ 中的边数记为 $e$.
显然图 $G$ 中和顶点 $x_i$ 相邻的点是在以 $x_i$ 为圆心, 半径为 1 的圆周上.
由于集 $S$ 中任意两点之间的距离 $\geqslant 1$, 故圆周上至多含有 $S$ 中的 6 个点, 所以 $d\left(x_i\right) \leqslant 6$.
对图 $G$ 用定理一, 有
$$
\begin{gathered}
d\left(x_1\right)+d\left(x_2\right)+\cdots+d\left(x_n\right)=2 e, \\
6 n \geqslant 2 e,
\end{gathered}
$$
即 $e \leqslant 3 n$. 就是说图 $G$ 中的边数 $e$ 不超过 $3 n$. 所以这 $n$ 个点中至多有 $3 n$ 对点,每对点的距离恰好是 1 .
%%PROBLEM_END%%



%%PROBLEM_BEGIN%%
%%<PROBLEM>%%
例5. 有一个团体会议,有 100 人参加.
其中任意四个人都至少有一个人认识其他三人.
问: 该团体中认识其他所有人的成员最少有多少?
%%<SOLUTION>%%
解:先把问题翻译成图论语言.
把该团体的成员视为顶点, 其顶点全体记做 $V$. 对于任意两个顶点 $u, v$ 所代表的成员, 当且仅当彼此认识, 则在 $u, v$
之间连一条边.
得到一个含 100 个顶点的简单图 $G$. 已知条件是, 图 $G$ 中任意四个顶点中都至少有一顶点和其他三个顶点相邻.
要求图 $G$ 中度为 99 的顶点个数的最小值 $m$.
当图 $G$ 是完全图时, 每个顶点的度都是 99 , 所以有 100 个度为 99 的顶点.
当图 $G$ 是非完全图时, 图 $G$ 中必有两个不相邻的顶点 $u$ 和 $v$. 显然 $d(u) \leqslant 98, d(v) \leqslant 98$. 因此图 $G$ 中度为 99 的点的个数 $l \leqslant 98$.
如果 $G$ 中除 $u$ 和 $v$ 外另有两个顶点 $x, y$ 不相邻, 则 $u, v, x$ 和 $y$ 中不存在和其他三个顶点都相邻的顶点, 与题意矛盾 (与图 $G$ 的性质矛盾). 因此 $G$ 中除 $u, v$ 外任意两个顶点相邻.
这说明对 $G$ 中除 $u, v$ 外的任意点 $x$, 均有 $d(x) \geqslant 97$.
如果 $G$ 中除 $u 、 v$ 外的任何 $x$ 都和 $u, v$ 相邻, 则 $d(x)=99$. 此时 $G$ 中度为 99 的顶点个数为 98 .
设 $G$ 中除 $u 、 v$ 外有个顶点 $x$ 和 $u 、 v$ 不都相邻, 则有 $G$ 的性质知, $G$ 中除 $u, v, x$ 外的任意顶点 $y$ 和 $u 、 v 、 x$ 都相邻.
因此 $d(u) \leqslant 98, d(v) \leqslant 98$, $d(x) \leqslant 98, d(y)=99$. 所以 $G$ 中度为 99 的顶点个数为 97 .
这表明含 100 个顶点的简单图 $G$ 中, 如果任意四个顶点中必有一个顶点和其他三个顶点都相邻,那么 $G$ 中至少有 97 个度为 99 的顶点.
回到原问题,即得: 该团体中认识其他所有人的成员最少是 97 个.
%%<REMARK>%%
注:例题中的成员数 100 改为任意的 $n$, 其他条件不变, 则结论为该团体至少有 $n-3$ 人认识其他所有人.
%%PROBLEM_END%%



%%PROBLEM_BEGIN%%
%%<PROBLEM>%%
例6. 国际乒乓球男女混合双打大奖赛有 24 对选手参加, 赛前一些选手屋了手,但同一对选手之间不握手.
赛后某个男选手问每个选手的握手次数, 各人的回答各不相同,问这名男选手的女搭档和多少人握了手?
%%<SOLUTION>%%
解:48 名选手用 48 个顶点 $v, v_0, v_1, \cdots, v_{46}$ 表示, 其中 $v$ 代表那名男选手.
两人握过手就在他们相应的顶点之间连一条边, 得图 $G$. 在 $G$ 中, $d\left(v_i\right) \leqslant 46, i=0,1,2, \cdots, 46$. 并且当 $i \neq j$ 时, $d\left(v_i\right) \neq d\left(v_j\right)$. 所以除顶点 $v$ 外,其他顶点的度分别为
$$
0,1,2, \cdots, 45,46 \text {. }
$$
不妨设 $d\left(v_i\right)=i, i=0,1,2, \cdots, 46$. 对顶点 $v_{46}$ 来说, 它只和顶点 $v_0$ 不相邻, 故 $v_{46}$ 和 $v_0$ 是搭档.
在 $G$ 中去掉顶点 $v_0 、 v_{46}$ 以及与它们相邻的边, 得图 $G_1$, 在 $G_1$ 中除 $v$ 外, 各顶点的度仍然不同, 且度都减小 1 , 同样道理, $v_{45}$ 和 $v_1$ 是搭档.
依次可得 $v_{44}$ 和 $v_2, \cdots, v_{24}$ 和 $v_{22}$ 是搭档.
于是 $v_{23}$ 和 $v$ 是搭档.
所以那个男选手的女搭档握了 23 次手.
%%<REMARK>%%
注:本题证明中, 将 $G$ 的顶点编号, 按度的非降次序 $\left(d_1 \leqslant d_2 \leqslant \cdots \leqslant\right. \left.d_n\right)$ 排列, 得到 $\left(d_1, d_2, \cdots, d_n\right)$ 称为图 $G$ 的度序列.
利用度序列解题是一种重要方法.
%%PROBLEM_END%%



%%PROBLEM_BEGIN%%
%%<PROBLEM>%%
例7. 某俱乐部共有 99 名成员, 每一个成员都声称只愿意和自己认识的人一起打桥牌.
已知每个成员都至少认识 67 名成员.
证明一定有 4 名成员, 他们可以在一起打桥牌.
%%<SOLUTION>%%
证法一,如图(<FilePath:./figures/fig-c2i4.png>), 作一个图 $G$ : 用 99 个点表示 99 名成员,如果两名成员相互认识, 就在相应的两个顶点之间连一条边.
已知条件是: 对任意顶点 $v, d(v) \geqslant$ 67. 欲证 $G$ 中含有一个 4 阶完全图 $K_4$.
在 $G$ 中任取一个顶点 $u$, 由于 $d(u) \geqslant$ 67 , 所以存在顶点 $v$, 使得与 $v$ 相邻且与 $u$ 不相邻的顶点至多为 $(99-1-67=) 31$ 个.
同样,与 $v$ 不相邻且与 $u$ 相邻的顶点也至多 31 个.
于是图 $G$ 中至少有 $(99-31- 31-2=) 35$ 个顶点和 $u, v$ 均相邻.
如图(<FilePath:./figures/fig-c2i4.png>) 所示, 设顶点 $x$ 和顶点 $u, v$ 均相邻.
由于 $d(x) \geqslant 67$, 并且 $G$ 中至多只有 $(31+31+2=) 64$ 个不同时和 $u, v$ 均相邻的顶点, 因此顶点 $x$ 至少还和一个与 $u, v$ 均相邻的顶点 $y$ 相邻.
从而 $u, v, x, y$ 是 4 个两两相邻的顶点.
于是命题得证.
证法二用顶点表示成员, 如果两个人不认识就在相应的顶点之间连一条边, 得图 $G^{\prime}$. 由于每个人认识的人数不少于 67 , 所以对每个顶点 $v$, 都有 $d(v) \leqslant 99-1-67=31$. 要证明的是: $G^{\prime}$ 中存在四个点, 两两之间不相邻.
对于顶点 $u$, 取一个不与它相邻的顶点 $v$, 则剩下的 97 个顶点中与 $u$ 或 $v$ 相邻的顶点个数不超过
$$
d(u)+d(v) \leqslant 31+31=62,
$$
因而存在与 $u, v$ 均不相邻的顶点 $x$, 与顶点 $u, v, x$ 中至少有一个相邻的顶点个数不超过
$$
d(u)+d(v)+d(x) \leqslant 3 \times 31=93,
$$
所以在剩下的 96 个点中, 必有一个点 $y$ 与 $u, v, x$ 均不相邻, 于是 $u, v, x, y$ 所代表的 4 个人是互相认识的, 他们可以在一起打桥牌.
%%<REMARK>%%
注:1: 若将题中的 67 人改为 66 人,则不一定能找出 4 个互相认识的人来.
反例如图(<FilePath:./figures/fig-c2i5.png>)所示.
将顶点集 $V$ 分成三个子集 $\left\{v_1\right.$, $\left.v_2, \cdots, v_{33}\right\},\left\{v_{34}, v_{35}, \cdots, v_{66}\right\},\left\{v_{67}\right.$, $\left.v_{68}, \cdots, v_{99}\right\}$. 同一个子集中任意两顶点均不相邻, 不同子集中的任意两点均相邻.
显然每个顶点的度都是 66 , 任意 4 点中, 至少有 2 点属于同一子集, 从而它们不相邻.
也就是说图中不存在两两相邻的 4 顶点.
注:2: 本题可推广为:
俱乐部有 $n(n \geqslant 4)$ 人, 其中每人都至少认识其中的 $\left[\frac{2 n}{3}\right]+1$ 个人, 则在这 $n$ 个人中必定可以找到 4 个人, 他们是两两认识的.
注:3: 如果 $G$ 是 $n$ 阶简单图, 从完全图 $K_n$ 中把属于 $G$ 的边全部去掉后, 得到的图称为 $G$ 的补图, 通常记为 $\bar{G}$. 证法一中的图 $G$ 与证法二中的图 $G^{\prime}$, 互为补图.
%%PROBLEM_END%%



%%PROBLEM_BEGIN%%
%%<PROBLEM>%%
例8. 设在平面上有 $n$ 个给定的点.
求证其中距离为 1 的点的对数不超过 $\frac{n}{4}+\frac{\sqrt{2}}{2} n^{\frac{3}{2}}$.
%%<SOLUTION>%%
证把 $n$ 个点视为图 $G$ 的顶点, 记 $V=\left\{v_1, v_2, \cdots, v_n\right\}$ 为图 $G$ 的顶点集, 在距离为 1 的两点之间连一边, 则由定理一,
$$
2 e=d\left(v_1\right)+d\left(v_2\right)+\cdots+d\left(v_n\right) .
$$
用 $C_i$ 表示以 $v_i$ 为圆心、半径为 1 的圆, 这 $n$ 个圆两两交点总数不超过 $2 \mathrm{C}_n^2=n(n-1)$ 个.
另一方面, 若 $v_k, v_j$ 与 $v_i$ 相邻, 则 $v_i \in C_k \cap C_j$, 因此, $v_i$ 作为 $C_1, C_2, \cdots$, $C_n$ 中两圆的交点恰好被计数了 $\mathrm{C}_{d\left(v_i\right)}^2$ 次, 故
$$
\mathrm{C}_{d\left(v_1\right)}^2+\mathrm{C}_{d\left(v_2\right)}^2+\cdots+\mathrm{C}_{d\left(v_n\right)}^2 \leqslant 2 \mathrm{C}_n^2=n(n-1) . \label{eq1}
$$
由 Cauchy 不等式,有
$$
\mathrm{C}_{d\left(v_1\right)}^2+\mathrm{C}_{d\left(v_2\right)}^2+\cdots+\mathrm{C}_{d\left(v_n\right)}^2 \geqslant \frac{2}{n} e^2-e . \label{eq2}
$$
由 式\ref{eq1}, \ref{eq2}式得
$$
\frac{2}{n} e^2-e \leqslant n(n-1)
$$
即
$$
2 e^2-n e-n^2(n-1) \leqslant 0 .
$$
解得
$$
e \leqslant \frac{n}{4}+\frac{\sqrt{2}}{2} n^{\frac{3}{2}}
$$
%%PROBLEM_END%%


