
%%TEXT_BEGIN%%
我们经常遇到这样一些现象或问题:
在一群人中,有的两个人之间互相认识, 有的互不相识;
一次足球锦标赛有若干个队参加, 其中有的两个队之间比赛过, 有的没有比赛过;
有若干个大城市, 有的两个城市之间有航线相通, 有的没有航线相通;
平面上的一个点集中, 其中任意两点之间, 有的距离为 1 , 有的距离不为 1 .
在上面这些现象或问题中都包含两方面的内容: 其一是一些 "对象", 如人群、足球队、城市、点等等; 其二是这些对象两两之间的某种特定关系, 如 "互相认识"、"比赛过"、"通航"、"距离为 1 "等.
为了表示这些对象和他们之间的关系, 我们可以用一个点表示一个对象, 称这些点为顶点, 如果两个对象之间有所讨论的关系, 就在相应的两点之间连上一条线, 称这些线为边, 这样就构成了一个图形.
这个用来表示某类对象及它们间特定关系的, 由若干个顶点与连接某些顶点的边构成的图形, 我们直观地称之为图 *.
图论是以图作为研究对象的一个数学分支.
例如图(<FilePath:./figures/fig-c1i1.png>) 中给出了 3 个图 $G_1 、 G_2 、 G_3$, 其中顶点由小圆圈表示.
图的一般数学定义为: 一个图 $G$ 是一个三元组 $(V, E, \phi)$, 其中 $V$ 和 $E$ 是两个不相交的集合, $V$ 非空, $\phi$ 是 $E$ 到 $V$ 的一个映射, $V 、 E 、 \phi$ 分别称为图 $G$ 的顶点集、边集和关联函数.
我们注意到, 在直观地叙述图的定义中, 并没有规定这些顶点的位置以及边的曲直长短, 也没有规定这些顶点、边都要在同一平面中, 不过, 连结两点的边不能通过第三个顶点, 也不能与自己相交.
在图论中, 如果两个图 $G$ 与 $G^{\prime}$ 的顶点之间可以建立起一一对应, 并且 $G$ 中连接顶点 $v_i$ 与 $v_j$ 之间的边数 $k(k=0,1,2, \cdots)$ 与连接 $G^{\prime}$ 中相应的顶点 $v_i^{\prime}$ 与 $v_j^{\prime}$ 的边数相同时, 便称图 $G$ 与 $G^{\prime}$ 是同构的,认为 $G$ 与 $G^{\prime}$ 是相同的图.
例如, 如图(<FilePath:./figures/fig-c1i1.png>) 中的三个图 $G_1 、 G_2 、 G_3$ 是同构的.
如果对图 $G=(V, E)$ 与 $G^{\prime}=\left(V^{\prime}, E^{\prime}\right)$ 有 $V^{\prime} \subseteq V, E^{\prime} \subseteq E$, 即图 $G^{\prime}$ 的顶点都是图 $G$ 的顶点, 图 $G^{\prime}$ 的边也都是图 $G$ 的边, 则称 $G^{\prime}$ 是 $G$ 的子图, 例如图(<FilePath:./figures/fig-c1i2.png>) 中的 $G_1 、 G_2$ 都是 $G$ 的子图.
若在一个图 $G$ 中的两个顶点 $v_i$ 与 $v_j$ 之间有边 $e$ 相连, 则称点 $v_i$ 与 $v_j$ 是相邻的, 否则就称点 $v_i$ 与 $v_j$ 是不相邻的.
如果顶点 $v$ 是边 $e$ 的一个端点, 称点 $v$ 与边 $e$ 是关联的.
如图(<FilePath:./figures/fig-c1i3.png>) 中, 顶点 $v_1$ 与 $v_2$ 是相邻的, 而顶点 $v_2$ 与顶点 $v_5$ 是不相邻的.
顶点 $v_3$ 与边 $e_4$ 是关联的.
有些顶点本身也有边相连, 这样的边称为环.
如图(<FilePath:./figures/fig-c1i3.png>) 所示的边 $e_6$ 是环.
连结两个顶点的边有时可能不止一条, 若两个顶点之间有 $k(k \geqslant 2)$ 条边相连,则称这些边为平行边.
例如图(<FilePath:./figures/fig-c1i3.png>) 中的边 $e_1 、 e_2$ 是平行边.
如果一个图没有环, 并且没有平行边, 这样的图称为简单图.
如图(<FilePath:./figures/fig-c1i1.png>) 中的 $G_1 、 G_2 、 G_3$ 都是简单图, 而如图(<FilePath:./figures/fig-c1i3.png>) 所示的就不是一个简单图.
在简单图中, 连结顶点 $v_i$ 与 $v_j$ 之间的边可用 $\left(v_i, v_j\right)$ 表示.
当然, $\left(v_i, v_j\right)$ 与 $\left(v_j, v_i\right)$ 表示的是同一条边.
如果一个简单图中, 每两个顶点之间都有一条边, 这样的图称为完全图.
通常将有 $n$ 个顶点的完全图记为 $K_n$. 如图(<FilePath:./figures/fig-c1i4.png>) 中是完全图 $K_3 、 K_4 、 K_5$. 完全图 $K_n$ 的边的数目是 $\mathrm{C}_n^2=\frac{1}{2} n(n-1)$.
在图 $G=(V, E)$ 中, 若顶点个数 $|V|(|V|$ 也称为 $G$ 的阶 $)$ 和边数 $|E|$ 都是有限的, 则称图 $G$ 是有限图.
如果 $|V|$ 或 $|E|$ 是无限的, 则称 $G$ 为无限图.
本篇中,除非特别说明,我们所说的图都是指有限简单图.
利用上述的这些基本概念可以帮助我们思考并解决一些问题.
本书中的例题和习题包括图论问题和利用图论方法解决的问题.
%%TEXT_END%%



%%PROBLEM_BEGIN%%
%%<PROBLEM>%%
例1. 某聚会有 605 个人参加, 已知每个人至少和其余的一个人握过手.
证明: 必有一个人至少和其中的两个人握过手.
%%<SOLUTION>%%
证明:将 605 个人用 605 个点 $v_1, v_2, \cdots, v_{605}$ 表示, 如果其中两个人握过手, 就在相应的顶点之间连一条边.
本例要证明: 必有一个人至少和其中的两个人握过手.
倘若不然,则每个人至多和其中一个人握过手,再从题设的每个人至少和其中一个人握过手, 于是便有,每个人恰与其他一个人握过手.
这样就得出,图 $G$ 恰由若干个两点间连一条边的图形构成(如图(<FilePath:./figures/fig-c1i5.png>)).
设图 $G$ 有 $r$ 条边, 则 $G$ 便有 $2 r$ (偶数) 个顶点, 这与 $G$ 的顶点数为 605 (奇数)矛盾.
%%PROBLEM_END%%



%%PROBLEM_BEGIN%%
%%<PROBLEM>%%
例2. 能否让马跳动几次, 将如图(<FilePath:./figures/fig-c1i6.png>) 所示的阵势变为如图(<FilePath:./figures/fig-c1i7.png>) 所示的阵势?("马"按照国际象棋规则跳动)
\begin{tabular}{|l|l|l|}
\hline 1 & 4 & 7 \\
\hline 2 & 5 & 8 \\
\hline 3 & 6 & 9 \\
\hline
\end{tabular}
%%<SOLUTION>%%
解:如图(<FilePath:./figures/fig-c1i8.png>) 所示, 将九个方格编号.
再把每个方格对应为平面上一点.
若马能从一个方格跳往另一个方格, 则在相应两点之间连一条边, 如图(<FilePath:./figures/fig-c1i9.png>).
于是如图(<FilePath:./figures/fig-c1i6.png>) 所示的开始的阵势以及如图(<FilePath:./figures/fig-c1i7.png>) 所示的要求变成的阵势分别变成了如图(<FilePath:./figures/fig-c1i10.png>)、如图(<FilePath:./figures/fig-c1i11.png>) 中的两个图形.
显然,马在一个圆上的前后跟随顺序是不变的,所以不能按要求改变阵势.
%%PROBLEM_END%%



%%PROBLEM_BEGIN%%
%%<PROBLEM>%%
例3. 有 $n(n>3)$ 个人,他们之间有些人互相认识, 有些人互相不认识,而且至少有一个人没有与其他人都认识.
问: 与其他人都认识的人数的最大值是多少?
%%<SOLUTION>%%
解:作图 $G$ : 用 $n$ 个点表示这 $n$ 个人,两顶点相邻当且仅当相应的两个人互相认识.
由于至少有一个人没有与其他人都认识, 所以图 $G$ 中至少有两点不相邻, 设 $v_1, v_2$ 之间没有边 $e=\left(v_1, v_2\right)$. 则图 $G$ 的边数最多时的图形为 $K_n- e$, 即从完全图 $K_n$ 中去掉边 $e$ 后所得的图.
从而与其他顶点都相邻的顶点个数的最大值为 $n-2$. 故与其他人都认识的人数的最大值是 $n-2$.
%%PROBLEM_END%%



%%PROBLEM_BEGIN%%
%%<PROBLEM>%%
例4. 九名数学家在一次国际数学会议上相遇, 发现他们中的任意三个人中, 至少有两个人可以用同一种语言对话.
如果每个数学家至多可说三种语言, 证明至少有三名数学家可以用同一种语言对话.
%%<SOLUTION>%%
证明:,如图(<FilePath:./figures/fig-c1i12.png>)用九个点 $v_1, v_2, \cdots, v_9$ 表示这九名数学家,如果某两个数学家能用第 $i$ 种语言对话, 则在它们相应的顶点之间连一条边并涂以相应的第 $i$种颜色, 这样就得到了一个有九个顶点的简单图 $G$, 它的边涂上了颜色, 每三点之间至少有一条边, 每个顶点引出的边至多有三种不同的颜色.
要证明的是: 图 $G$ 中存在三个点, 它们两两相邻, 且这三条边具有相同的颜色 (这种三角形称为同色三角形).
如果边 $\left(v_i, v_j\right) 、\left(v_i, v_k\right)$ 具有相同的第 $i$ 种颜色, 则按边涂色的意义, 点 $v_j$ 与 $v_k$ 也相邻, 且边 $\left(v_j, v_k\right)$ 也具有第 $i$ 种颜色.
所以对顶点 $v_1$ 来说, 有两种情形:
(1) 点 $v_1$ 与点 $v_2, \cdots, v_9$ 都相邻, 根据抽㞕原理知, 至少有两条边, 不妨设为 $\left(v_1, v_2\right) 、\left(v_1, v_3\right)$, 具有相同的颜色, 从而 $\triangle v_1 v_2 v_3$ 是同色三角形.
(2) 点 $v_1$ 与点 $v_2, \cdots, v_9$ 中的至少一个点不相邻, 不妨设点 $v_1$ 与点 $v_2$
不相邻.
由于每三点之间至少有一条边, 所以从 $v_3$, $v_4, \cdots, v_9$ 发出的, 另一个端点是 $v_1$ 或 $v_2$ 的边至少有 7 条, 由此可知, 点 $v_3, v_4, \cdots, v_9$ 中至少有 4 个点与点 $v_1$ 或 $v_2$ 相邻, 不妨设点 $v_3, v_4, v_5, v_6$ 与点 $v_1$ 相邻, 如图 $1-12$ 所示.
于是边 $\left(v_1, v_3\right) 、\left(v_1, v_4\right) 、\left(v_1, v_5\right) 、\left(v_1\right.$, $\left.v_6\right)$ 中必定有两条具有相同的颜色, 设 $\left(v_1, v_3\right) 、\left(v_1\right.$, $v_4$ ) 同色,则 $\triangle v_1 v_3 v_4$ 是同色三角形.
$v_2 \circ$
%%<REMARK>%%
注:: 若把题中的九改成八, 命题就不成立了.
如图(<FilePath:./figures/fig-c1i13.png>)给出的是一个反例.
$v_1, v_2, \cdots, v_8$ 表示 8 个顶点, $1,2, \cdots, 12$ 表示 12 种颜色, 则图中无同色三角形.
%%PROBLEM_END%%



%%PROBLEM_BEGIN%%
%%<PROBLEM>%%
例5. 有 $n$ 名选手 $A_1, A_2, \cdots, A_n$ 参加数学竞赛, 其中有些选手是互相认识的, 而且任何两个不相识的选手都恰好有两个共同的熟人.
若已知选手 $A_1$ 与 $A_2$ 互相认识,但他俩没有共同的熟人,证明他俩的熟人一样多.
%%<SOLUTION>%%
证明:用 $n$ 个点 $v_1, v_2, \cdots, v_n$ 表示这 $n$ 名选手 $A_1, A_2, \cdots, A_n$, 如果两个选手互相认识, 那么就在相应的两个点之间连一条边, 这样就得到一个简单图 $G$. 图 $G$ 中的顶点满足: 任意两个不相邻的顶点都恰好有两个共同相邻的顶点.
要证明的是相邻的两个顶点 $v_1$ 与 $v_2$ 各自引出的边的条数一样多.
如果记与 $v_1$ 相邻的顶点集合为 $N\left(v_1\right)$, 与 $v_2$ 相邻的顶点集合为 $N\left(v_2\right)$. 若在 $N\left(v_1\right)$ 中除 $v_2$ 外还有点 $v_i$, 则 $v_i \notin N\left(v_2\right)$. 否则 $A_1$ 与 $A_2$ 有共同熟人 $A_i$. 于是 $v_2$ 与 $v_i$ 除 $v_1$ 外还应有一个与它们共同相邻的点 $v_j$, 则 $v_j \in N\left(v_2\right)$. 如图(<FilePath:./figures/fig-c1i14.png>) 所示.
对于 $N\left(v_1\right)$ 中不同于 $v_2$ 的点 $v_i 、 v_k$, 它们不可能与 $N\left(v_2\right)$ 中除 $v_1$ 外的一个点 $v_j$ 都相邻 (否则, 两个不相邻的顶点 $v_1$, $v_j$ 有三个共同相邻的顶点 $v_2, v_i, v_k$ ). 因而, 对于 $N\left(v_1\right)$ 中不同于 $v_i$ 的 $v_k$, 必在 $N\left(v_2\right)$ 中存在不同于 $v_j$ 的相邻点 $v_l$. 由此可得 $N\left(v_1\right)$ 中的顶点个数不大于 $N\left(v_2\right)$ 中的顶点个数.
同样地, $N\left(v_2\right)$ 中的顶点个数不大于 $N\left(v_1\right)$ 中的顶点个数.
于是, 从点 $v_1$ 与 $v_2$ 引出的边的条数是相等的.
%%PROBLEM_END%%



%%PROBLEM_BEGIN%%
%%<PROBLEM>%%
例6. 有 $n$ 个人,已知他们中的任意两人至多通电话一次, 他们中的任意 $n-2$ 个人之间通电话的总次数相等, 都是 $3^m$ 次, 其中 $m$ 是自然数.
求 $n$ 的所有可能值.
%%<SOLUTION>%%
解:显然 $n \geqslant 5$. 记 $n$ 个人为几个点 $A_1, A_2, \cdots, A_n$. 若 $A_i, A_j$ 之间通电话, 则连 $\left(A_i, A_j\right)$. 因此这 $n$ 个点中必有边相连, 不妨设为 $\left(A_1, A_2\right)$.
倘若设 $A_1$ 与 $A_3$ 之间无边,分别考虑 $n-2$ 个点 $A_1, A_4, A_5, \cdots, A_n$ ; $A_2, A_4, A_5, \cdots, A_n$; 及 $A_3, A_4, A_5, \cdots, A_n$. 由题意知 $A_1, A_2, A_3$ 分别与 $A_4, A_5, \cdots, A_n$ 之间所连边的总数相等, 记为 $k$.
将 $A_2$ 加人到 $A_1, A_4, A_5, \cdots, A_n$ 中, 则这 $n-1$ 个点之间边的总数 $S= 3^m+k+1$. 从这 $n-1$ 点中任意去掉一点剩下的 $n-2$ 个点所连边数都是 $3^m$, 故每个点都与其余 $n-2$ 个点连 $k+1$ 条边.
从而
$$
S=\frac{1}{2}(n-1)(k+1) \text {. }
$$
同理, 考虑 $A_3$ 加人 $A_1, A_4, A_5, \cdots, A_n$ 中所得的 $n-1$ 个点的情况可知边的总数为 $t=3^m+k=\frac{1}{2}(n-1) k$.
因为 $S=t+1$, 得
$$
\frac{1}{2}(n-1)(k+1)=\frac{1}{2}(n-1) k+1,
$$
即 $n=3$, 矛盾.
所以 $A_1 、 A_3$ 之间有边.
同理 $A_2, A_3$ 之间也有边.
进而 $A_1, A_2$ 与所有 $A_i(i=3,4, \cdots, n)$ 之间有边.
对于 $A_i, A_j(i \neq j)$, 因为 $A_i$ 与 $A_1$ 之间有边, 可知 $A_i$ 与 $A_j$ 之间有边.
因此这 $n$ 个点构成一个完全图.
所以
$$
3^m=\frac{1}{2}(n-2)(n-3) .
$$
故 $n=5$.
%%PROBLEM_END%%



%%PROBLEM_BEGIN%%
%%<PROBLEM>%%
例7. 设 $n$ 为一正整数,且 $A_1, A_2, \cdots, A_{2 n+1}$ 是某个集合 $B$ 的子集.
设
(1) 每一个 $A_i$ 恰含有 $2 n$ 个元素;
(2) 每一个 $A_i \cap A_j(1 \leqslant i<j \leqslant 2 n+1)$ 恰含有一个元素;
(3) $B$ 的每个元素至少属于 $A_i$ 中的两个.
问对怎样的 $n$, 可以将 $B$ 中元素各标上数 0 或 1 , 使得每个 $A_i$ 恰含有 $n$ 个标上了 0 的元素?
%%<SOLUTION>%%
解:首先, (3) 中的"至少"实际上也可以改成"恰". 因为如果有一个元素 $a_1 \in A_1 \cap A_{2 n} \cap A_{2 n+1}$, 那么剩下的 $2 n-2$ 个子集 $A_2, A_3, \cdots, A_{2 n-1}$ 每个至多含 $A_1$ 中一个元素, 从而 $A_1$ 中至少有一个元素不属于 $A_2 \cup A_3 \cup \cdots \cup A_{2 n-1} \cup A_{2 n} \cup A_{2 n+1}$, 这与(3)矛盾.
于是作完全图 $K_{2 n+1}$, 每一个顶点 $v_i$ 表示一个子集 $A_i$, 每一条边 $\left(v_i\right.$, $\left.v_j\right)=b_{i j}(1 \leqslant i, j \leqslant 2 n+1, i \neq j)$ 表示集 $A_i$ 与 $A_j$ 所共有的那个元素.
于是题目就转化为: 对怎样的 $n$, 可以给 $K_{2 n+1}$ 的每条边贴一个 0 或 1 的标签, 使得从图中任一点 $v_i$ 出发的 $2 n$ 条边中恰有 $n$ 条边贴有 0 的标签.
因为 $K_{2 n+1}$ 有 $n(2 n+1)$ 条边, 如果上述贴标签的要求能够满足, 则贴 0 的边共有 $\frac{1}{2} n(2 n+1)$ 条, 于是 $n$ 必须是偶数.
反之, 若 $n=2 m$ 是偶数, 我们把 $K_{2 n+1}$ 中的边 $\left(v_i, v_{i-m}\right),\left(v_i, v_{i-m+1}\right)$, $\cdots,\left(v_i, v_{i-1}\right),\left(v_i, v_{i+1}\right), \cdots,\left(v_i, v_{i+m}\right), i=1,2, \cdots, 2 n+1$, 全标上 0 , 其余的标上 1 , 则得本题所要求的贴标签方法 (要注意的是, 顶点的下标的加法是按模 $2 n+1$ 进行的, 即 $\left.v_{(2 n+1)+i}=v_i\right)$.
所以,当且仅当 $n$ 为偶数时, 可以满足题目要求.
%%PROBLEM_END%%



%%PROBLEM_BEGIN%%
%%<PROBLEM>%%
例8. 有 $12 k$ 人参加会议, 每人都恰好与 $3 k+6$ 人握过手, 并且对其中任意两人,与这两个人都握过手的人数皆相同.
问有多少人参加会议?
%%<SOLUTION>%%
解:设对任意两人, 与他们都握过手的有 $n$ 人.
考虑某个 $a$, 与 $a$ 握过手的全体记为 $A$, 与 $a$ 没有握过手的全体记为 $B$. 由题设 $|A|=3 k+6,|B|= 9 k-7$.
再考虑 $b \in A$, 与 $a, b$ 都握过手的 $n$ 个人都在 $A$ 中, 因此, $b$ 与 $A$ 中 $n$ 个人握手, 与 $B$ 中 $3 k+5-n$ 人握手.
考虑 $c \in B$, 与 $a, c$ 都握过手的 $n$ 人都在 $A$ 中.
于是 $A$ 与 $B$ 之间握手总数为
$$
\begin{gathered}
(3 k+6)(3 k+5-n)=(9 k-7) n, \\
n=\frac{(3 k+6)(3 k+5)}{12 k-1} . \\
16 n=\frac{(12 k-1+25)(12 k-1+21)}{12 k-1} .
\end{gathered}
$$
从而
$$
16 n=\frac{(12 k-1+25)(12 k-1+21)}{12 k-1}
$$
显然 $(3,12 k-1)=1$, 所以 $(12 k-1) \mid 25 \times 7$. 由于 $12 k-1$ 除以 4 余 3 , 所以 $12 k-1=7,5 \times 7,5^2 \times 7$. 经检验只有 $12 k-1=5 \times 7$ 产生整数解 $k=3, n=6$.
下面构造一个由 36 点组成的图, 图中每点引出 15 条边, 且对每一对点与它们相连的点均为 6 个.
自然地,我们可用 6 个完全图 $K_6$. 把 36 个点分成六组,同组的六人编号, 排成一个 $6 \times 6$ 方阵
$\begin{array}{llllll}1 & 2 & 3 & 4 & 5 & 6 \\ 6 & 1 & 2 & 3 & 4 & 5 \\ 5 & 6 & 1 & 2 & 3 & 4 \\ 4 & 5 & 6 & 1 & 2 & 3 \\ 3 & 4 & 5 & 6 & 1 & 2 \\ 2 & 3 & 4 & 5 & 6 & 1\end{array}$
对方阵中的每个点, 它与同行、同列、同编号的 15 个点相连, 与其余点不相连.
易见,与任意两位代表都握过手的恰好有 6 人.
%%PROBLEM_END%%


