
%%TEXT_BEGIN%%
拉姆赛问题.
通常, 我们把与图的染色、拉姆赛 (Ramsey, 英国逻辑学家) 数、抽庶原则关联的问题称为拉姆赛问题.
我们先从匈牙利的一个竞赛题谈起.
1947 年匈牙利数学奥林匹克中出了这样一道试题:
%%TEXT_END%%



%%TEXT_BEGIN%%
$r_k$ 的存在性是由英国数学家、数理逻辑学家拉姆赛首先证明的, 所以 $r_k$ 叫拉姆赛数,关于 $r_k$ 我们有如下结论.
定理二 (1) 对每个正整数 $k$, 拉姆赛数 $r_k$ 存在, 并且当 $k \geqslant 2$ 时,
$$
r_k \leqslant k\left(r_{k-1}-1\right)+2 \text {; }
$$
(2) 对一切自然数 $k$,
$$
r_k \leqslant 1+1+k+k(k-1)+\cdots+\frac{k !}{2 !}+\frac{k !}{1 !}+k ! .
$$
证明 (1) 对 $k$ 进行归纳.
我们已知 $r_1 、 r_2$ 存在, 且 $r_1=3, r_2=6 \leqslant2\left(r_1-1\right)+2$.
设 $r_k$ 存在, 且 $r_k \leqslant k\left(r_{k-1}-1\right)+2$ 成立.
取 $n=(k+1)\left(r_k-1\right)+2$, 并设 $K_n$ 是 $k+1$ 色完全图, 它们的顶点为 $A_1, A_2, \cdots, A_n$. 任取 $K_n$ 的一个顶点 $A_1$, 从它出发有 $n-1=(k+1)\left(r_k-1\right)+1$ 条边.
这些边共有 $k+1$ 种颜色, 由抽庶原则, 这些边中至少有 $r_k$ 条同色.
不妨设这 $r_k$ 条边是 $A_1 A_2, A_1 A_3, \cdots$, $A_1 A_{r_k+1}$, 且都染 $c_1$ 色.
考虑由顶点 $A_2, A_3, \cdots, A_{r_k+1}$ 构成的 $r_k$ 阶子图 $K_{r_k}$. 若 $K_{r_k}$ 含有 $c_1$ 色边, 例如 $A_2 A_3$, 则 $\triangle A_1 A_2 A_3$ 为同色三角形; 若 $K_{r_k}$ 不含有 $c_1$ 色边, 则 $K_{r_k}$ 的边只有 $k$ 种颜色, 即 $K_{r_k}$ 为 $k$ 色完全图, 按归纳假设, $K_{r_k}$ 含有同色三角形.
总之, $K_n$ 含有同色三角形.
于是知
$M=\{m \mid$ 任何 $m$ 阶 $k+1$ 色完全图都含有同色三角形 $\}$
是自然数集 $N$ 的非空子集, 从而有最小值.
即 $r_{k+1}$ 存在, 且 $r_{k+1} \leqslant n=(k+$ 1) $\left(r_k-1\right)+2$.
(2) 用归纳法.
当 $k=1$ 时, $r_1=3 \leqslant 1+1+1$. 设命题对 $k$ 成立, 则应用 (1) 的结果及归纳假设, 得
$$
\begin{aligned}
r_{k+1} \leqslant & (k+1)\left(r_k-1\right)+2 \\
\leqslant & (k+1)\left[1+k+k(k-1)+\cdots+\frac{k !}{2 !}+\frac{k !}{1 !}+k !\right]+2 \\
= & (k+1)+(k+1) k+(k+1) k(k-1)+\cdots+\frac{(k+1) !}{2 !}+\frac{(k+1) !}{1 !} \\
& \quad+(k+1) !+2 \\
& \quad 1+1+(k+1)+(k+1) k+\cdots+\frac{(k+1) !}{2 !}+\frac{(k+1) !}{1 !}+(k+1) !,
\end{aligned}
$$
所以命题对 $k+1$ 也成立.
如果利用高等数学中关于自然对数的底数 $\mathrm{e}$ 的展开式
$$
\mathrm{e}=1+\frac{1}{1 !}+\frac{1}{2 !}+\cdots+\frac{1}{n !}+\cdots,
$$
可将 (2) 的结果简化成 $r_k \leqslant[k$ ! $\mathrm{e}]+1$. 这里 $[x]$ 表示不超过 $x$ 的最大整数.
定理二虽然证明了 $r_k$ 的存在性,并给出了 $r_k$ 的一个上界.
但是 $r_k$ 的准确值, 只知道三个, 除了前面提及的 $r_1=3, r_2=6$ 外, 还知道一个 $r_3=17$. 事实上, 根据定理二的 (1), 我们有 $r_3 \leqslant 3\left(r_2-1\right)+2=3 \times 5+2=17$. 直接仿照定理二的证明得出的这个结果还被编为 1964 年第六届国际数学奥林匹克试题:
有 17 位科学家, 其中每一个人和其他的所有的人通信, 他们在通信中只讨论 3 个题目,而且每两个科学家之间只讨论一个题目.
求证: 至少有三个科学家相互之间讨论同一个题目.
另一方面,可以给完全图 $K_{16}$ 的边涂上三种颜色,使得图中没有一个同色三角形,如图(<FilePath:./figures/fig-c8i5.png>) 所示,实线表示红边,虚线表示蓝边,未画线的表示黄边.
这就是说, $r_3 \geqslant 17$. 所以 $r_3=17$.
我们来看定理一的另一种推广:
设完全图 $K_n$ 的每条边被染以红、蓝两色, 即 $K_n$ 是染红、蓝两色的完全图,则对固定的自然数 $p, q$, 当 $n$ 充分大时,红、蓝两色完全图 $K_n$ 中, 必然出现红色 $K_p$, 或蓝色 $K_q$. 我们把满足上述性质的最小 $n$ 记为 $r(p, q), r(p, q)$ 也称为拉姆赛数.
用子图、补图的概念, $r(p, q)$ 也可说成使完全图 $K_n$ 的任何 $n$ 阶子图 $G$ 或者包含一个完全子图 $K_p$, 或者它的补图 $G$ 包含一个完全子图 $K_q$ 的 $n$ 的最小值.
由定义及定理一, 我们知 $r(3,3)=r_2=6$, 另外, 还容易发现 $r(1, q)= r(p, 1)=1$.
为了更好地理解后面将要给出的 $r(p, q)$ 的一般结果, 我们证明一个具体的例子: $r(3,4)=9$. 先证 $r(3,4) \leqslant 9$, 它也可以用一个类似于例 1 的形式给出.
%%TEXT_END%%



%%TEXT_BEGIN%%
关于 $r(p, q)$ 我们有如下的结论.
定理三 (1) $r(2, q)=q, r(p, 2)=p$;
(2) $r(p, q)=r(q, p)$;
(3) 在 $p \geqslant 2, q \geqslant 2$ 时,
$$
r(p, q) \leqslant r(p, q-1)+r(p-1, q) .
$$
并且在 $r(p, q-1)$ 与 $r(p-1, q)$ 都是偶数时, 成立严格不等式.
为了叙述方便,我们采用子图、补图的说法给出证明.
(3) 的证明较难, 为了加深理解, 阅读时可参照例 4 的证明.
证明 (1) 设 $G$ 是有 $q$ 个顶点的图.
若 $G$ 中有两个顶点相邻, 这时 $G$ 含有 $K_2$, 否则 $\bar{G}$ 就是 $K_q$, 所以 $r(2, q) \leqslant q$. 又由 $q-1$ 个两两不相邻的点所成的图 $G$ 显然不含 $K_2$, 它的补图 $\bar{G}$ 是 $K_{q-1}$, 不含 $K_q$, 所以 $r(2, q) \geqslant q$.
综上所述, $r(2, q)=q$. 同理可证, $r(p, 2)=p$.
(2) 设图 $G$ 有 $r(p, q)$ 个顶点, 则 $\bar{G}$ 也有 $r(p, q)$ 个顶点.
于是 $\bar{G}$ 中含有 $K_p$, 或者 $G$ 中含有 $K_q$. 换言之, $G$ 中含 $K_q$, 或 $\bar{G}$ 中含 $K_p$, 故 $r(p, q) \geqslant r(q, p)$.
同理 $r(q, p) \geqslant r(p, q)$, 所以 $r(p, q)=r(q, p)$.
(3) 设图 $G$ 有 $r(p, q-1)+r(p-1, q)$ 个顶点, $v_1$ 是 $G$ 的一个顶点.
若 $d\left(v_1\right) \geqslant r(p-1, q)$. 取 $\delta=r(p-1, q)$ 个与 $v_1$ 相邻的顶点 $v_2$, $v_3, \cdots, v_\delta, v_{\delta+1}$. 将 $G$ 中其余的顶点以及关联的边去掉得 $G_1$, 根据 $\delta=r(p- 1, q)$ 的定义, $G_1$ 中含有 $K_{p-1}$ 或 $\bar{G}_1$ 中含有 $K_q$. 如果 $G_1$ 含有 $K_{p-1}$, 那么在 $G$ 中, $v_1$ 与这个 $K_{p-1}$ 组成完全图 $K_p$. 如果 $\bar{G}_1$ 中含有 $K_q$, 则 $\bar{G}$ 中也含有这个 $K_q$.
若与 $v_1$ 相邻的顶点数 $<r(p-1, q)$, 则 $v_1$ 至少与 $r(p, q-1)$ 个顶点不相邻.
设与 $v_1$ 不相邻的顶点为 $v_2, v_3, \cdots, v_\delta, v_{\delta+1}$, 其中 $\varepsilon=r(p, q-1)$. 将 $G$ 中除 $v_2, v_3, \cdots, v_{\delta+1}$ 外的顶点以及关联的边去掉得图 $G_2$. 根据 $\varepsilon=r(p$, $q-1)$ 的定义, $G_2$ 中含有 $K_p$ 或 $\bar{G}_2$ 中含有 $K_{q-1}$. 如果 $G_2$ 中含有 $K_p$, 则 $G$ 中也含有这个 $K_p$. 如果 $\bar{G}_2$ 中含有 $K_{q-1}$, 则 $\bar{G}$ 中, $v_1$ 与这个 $K_{q-1}$ 组成 $K_q$.
综合起来便得
$$
r(p, q) \leqslant r(p, q-1)+r(p-1, q) .
$$
如果 $r(p, q-1)$ 与 $r(p-1, q)$ 都是偶数.
取有 $r(p, q-1)+r(p-1$, $q)-1$ 个顶点的图 $G$. 因奇顶点的个数是偶数, 而 $r(p, q-1)+r(p-1, q)-$ 1 是奇数.
因而 $G$ 中必有一个偶顶点 $v_1$, 对于 $v_1$, 或者 $d\left(v_1\right) \geqslant r(p-1, q)-$ 1 或者 $v_1$ 至少与 $r(p, q-1)$ 个顶点不相邻.
因与 $v_1$ 相邻的顶点数为偶数, 故在前一种情况, $d\left(v_1\right) \geqslant r(p-1, q)$, 照上面完全同样的证法得出
$$
\begin{aligned}
r(p, q) & \leqslant r(p, q-1)+r(p-1, q)-1 \\
& <r(p, q-1)+r(p-1, q) .
\end{aligned}
$$
至此定理三全部得证.
利用定理三的结果, 可以得出一些拉姆赛数 $r(p, q)$ 的上界, 例如
$$
\begin{aligned}
& r(3,3) \leqslant r(3,2)+r(2,3)=3+3=6, \\
& r(3,4) \leqslant r(3,3)+r(2,4)-1 \leqslant 6+4-1=9, \\
& r(3,5) \leqslant r(3,4)+r(2,5) \leqslant 9+5=14, \\
& r(4,4) \leqslant r(4,3)+r(3,4)=9+9=18 .
\end{aligned}
$$
在前面已经证明了 $r(3,3)=6, r(3,4)=9$, 类似地可以证明 $r(3,5)=14, r(4,4)=18$. 由已知不等式, 当然只要证明 $r(3,5)>13, r(4,4)>17$ 就够了.
这里证明前者, 后者的证明从略.
考察如图(<FilePath:./figures/fig-c8i7.png>) 所示的图 $G$, 它不含 $K_3$, 它的补图 $\bar{G}$ 不含 $K_5$, 所以 $r(3,5)>13$.
应用定理三,我们可以得出 $r(p, q)$ 的一个简单的上界.
%%TEXT_END%%



%%TEXT_BEGIN%%
定理四当 $p \geqslant 2, q \geqslant 2$ 时,
$$
r(p, q) \leqslant \mathrm{C}_{p+q-2}^{p-1} .
$$
证明记 $l=p+q$, 我们对 $l$ 进行归纳.
当 $l=4$ 时, $p=q=2$. 左边是 $r(2,2)=2$, 右边是 $\mathrm{C}_{4-2}^1=2$, 命题成立.
设当 $l=k(k \geqslant 4)$ 时, 命题成立.
当 $l=k+1$ 时, 在 $p=k-1, q=2$ 或 $p=2, q=k-1$ 的情况下,
$$
r(k-1,2)=r(2, k-1)=k-1=\mathrm{C}_{k-1}^{k-2}=\mathrm{C}_{k-1}^1 .
$$
命题成立.
当 $p \geqslant 3, q \geqslant 3, p+q=k+1$ 的情况下, 应用定理三的 (3) 及归纳假设,有
$$
\begin{aligned}
r(p, q) & \leqslant r(p-1, q)+r(p, q-1) \\
& \leqslant \mathrm{C}_{p+q-3}^{p-2}+\mathrm{C}_{p+q-3}^{p-1} \\
& =\mathrm{C}_{p+q-2}^{p-1} .
\end{aligned}
$$
命题仍成立.
根据数学归纳法, 定理四得证.
尽管如此, $r(p, q)$ 的准确值却不容易求得.
除 $r(1, q)=r(p, 1)=1$, $r(p, 2)=p, r(2, q)=q$ 外, 目前已经确定的 $r(p, q)$ 仅有少数几个, 它们列在下面的表中.
其中分数的分子表示下界,分母表示上界.
如果把前面两种推广结合起来,可得到如下的推广:
用 $l$ 种颜色 $c_1, c_2, \cdots, c_l$ 去染完全图 $K_n$ 的边, 每边染且只染一种颜色, 得到一个 $l$ 色完全图.
当 $n$ 充分大时, $l$ 色完全图 $K_n$ 中, 或者包含一个 $c_1$ 色的完全子图 $K_{p_1}$, 或者包含一个 $c_2$ 色的完全子图 $K_{p_2}, \cdots$, 或者包含一个 $c_i$ 色的完全子图 $K_{p_l}$. 我们把满足上述性质的最小 $n$ 记为 $r\left(p_1, p_2, \cdots, p_l\right)$, 它也称为拉姆赛数.
如果利用所谓"超图"的概念, 那么还可作推广.
这里就不深人介绍了.
%%TEXT_END%%



%%PROBLEM_BEGIN%%
%%<PROBLEM>%%
例1. 证明: 在任何六个人中, 总可以找到三个相互认识的人或三个相互不认识的人.
%%<SOLUTION>%%
我们用六个顶点表示六个人, 如果某两个人互相认识, 就在相应的两点间连一条边并涂上红色, 某两个人互相不认识, 就在相应的两点间连一条边并涂上蓝色.
要证明的结论就是这个涂了色的 $K_6$ 中一定有一个各边同色的三角形.
无独有偶, 这个变形就是 1953 年美国普特南数学竞赛试题: 空间中的六个点, 任意三点不共线, 任意四点不共面, 成对地连接它们得到十五条线段.
用红色或蓝色染这些线段 (一条线段只染一种颜色), 求证: 无论如何染色, 总存在同色三角形.
证明:设 $A_1, A_2, \cdots, A_6$ 是所给的六点.
考虑由 $A_1$ 出发的 5 条线段 $A_1 A_2, A_1 A_3, \cdots, A_1 A_6$. 因这 5 条线段只有红、蓝两种颜色, 因此至少有 3 条染成同一种颜色.
不妨设这 3 条线段就是 $A_1 A_2, A_1 A_3, A_1 A_4$, 且它们都染成红色 (实线表示红色,虚线表示蓝色). 若 $\triangle A_2 A_3 A_4$ 三边都是蓝色 (如图(<FilePath:./figures/fig-c8i1.png>)), 它即为同色三角形.
若 $\triangle A_2 A_3 A_4$ 至少有一条边, 例如 $A_2 A_3$ 为红色 (如图(<FilePath:./figures/fig-c8i2.png>)), 则 $\triangle A_1 A_2 A_3$ 是同色三角形.
总之, 无论是哪种情况, 都有同色三角形.
从本例我们还易知, 当 $n \geqslant 6$ 时, 给完全图 $K_n$ 的所有边染两种颜色的某一种 (以后简称两色完全图 $K_n$ ), 则总存在同色三角形.
如图(<FilePath:./figures/fig-c8i3.png>) 所示是一个染两色的完全图 $K_5$, 其中没有同色三角形.
之, 不难证明: 没有同色三角形的两色完全图 $K_5$, 必由两个不同色的五边形所组成.
换言之, 在两色完全图 $K_5$ 中, 若既没有蓝色三角形, 又没有蓝色五边形, 则必有红色三角形.
综上所述, 可得如下的结论.
定理一两色完全图 $K_n$ 必存在同色三角形的最小 $n$ 是 6 .
%%PROBLEM_END%%



%%PROBLEM_BEGIN%%
%%<PROBLEM>%%
例2. 证明不能对 $K_{10}$ 的边用四种颜色进行染色, 使得它的任何的 $K_4$ 子图的边包含所有四种颜色.
%%<SOLUTION>%%
证明:我们用反证法.
假设存在一种满足题目要求的染色方式.
若有一点连出 4 条同色的边, 不妨设 $A B 、 A C 、 A D 、 A E$ 全是蓝色的.
$B 、 C 、 D 、 E$ 之间一定有一条边是蓝色的, 不妨设为 $B C$, 则 $A 、 B 、 C 、 D$ 之间有 4 条蓝边,剩下两条边要染上三种颜色, 矛盾.
因此, $A$ 至多连出 3 条同色的边, 而且一定有一种颜色恰好染了三条边.
不妨设 $A B 、 A C 、 A D$ 全是蓝色的.
$A$ 、 $B 、 C 、 D$ 之间有 6 条边, 所以其余的 3 条边一定各染一种颜色, $B C 、 B D 、 C D$ 之中无蓝色边.
考虑余下的 6 个点, 由定理一, 其中必有蓝色三角形或者有某个三角形无蓝色边.
若有三点 $E 、 F 、 G$ 之间无蓝边, 则 $A 、 E 、 F 、 G$ 之间无蓝边, 矛盾.
因此不妨假设三角形 $E F G$ 为蓝色三角形.
因为 $B 、 C 、 D 、 E$ 之间有蓝边, 故只能是 $B E 、 C E 、 D E$ 之一为蓝色, 假设 $B E$ 是蓝色的, 则 $B 、 E 、 F 、 G$ 之间有四条蓝色边,同上矛盾.
综上命题成立.
%%PROBLEM_END%%



%%PROBLEM_BEGIN%%
%%<PROBLEM>%%
例3. 给定空间中 9 个点, 其中任意四个点都不共面.
在每一对点之间都连着一条线段.
试求出最小的 $n$ 值, 使得将其中任意 $n$ 条线段中的每一条任意地染为红、蓝两色之一, 在这 $n$ 条线段的集合中都必然包含有一个各边同色的三角形.
%%<SOLUTION>%%
解:题中"任意四点不共面"只是为了保证 9 点中无 3 点共线, 因此本题仍是一个平面上的图形问题.
于是问题归结为: 平面上有 9 个点, 无三点共线, 每两点连边共有 36 条边.
问至少取多少条边, 才能保证把这些边任意染成红蓝色时,一定出现同色三角形.
构造一个 9 阶 32 边的双色图 $G$. 顶点 $v_1$ 与 $v_2, v_3, v_3, v_9 4$ 点的连线染红色 (实线), $v_1$ 与 $v_4, v_5, v_6, v_7 4$ 点的连线染蓝色 (虚线). 把 $v_1$ 以外的 8 点分成 4 组, I : $\left(v_2, v_3\right), \mathbb{I I}:\left(v_4, v_5\right)$, III: $\left(v_6, v_7\right) ; \mathrm{IV}:\left(v_8, v_9\right)$, 称 I 与 II, $\mathrm{II}$ 与 III, III 与 IV 为相邻的组.
除 $v_1$ 外的两点, 属同一组的不连线, 分属两相邻组的连实线 (红色), 分属两不相邻组的连虚线 (蓝色), 如图(<FilePath:./figures/fig-c8i4.png>), $G$ 有 $\mathrm{C}_9^2-4=32$ 条边, 其中染红色 (实线)与蓝色(虚线)的各 16 条, 易知 $G$ 中不含同色三角形.
$$
n \geqslant 33 \text {. }
$$
下面我们再证明 $n \leqslant 33$.
设给定的 9 个点中有 33 条边染了色, 故此时有 3 边不染色, 不妨设为 $e_1$ 、 $e_2 、 e_3$. 在 $e_1 、 e_2 、 e_3$ 中各取一个端点 $v_1 、 v_2 、 v_3$, 从 $K_9$ 中删除这 3 个点, 其余的 6 点构成 $K_6$, 于是用红蓝两色染色必出现同色三角形.
所以, $n=33$.
为了推广定理一的结果, 我们先只增加染色的数目.
用 $k$ 种颜色 $c_1, c_2, \cdots, c_k$ 去染完全图 $K_n$ 的边, 每条边只染其中一种颜色, 这样得到的完全图 $K_n$ 简称为 $k$ 色完全图 $K_n$. 可以想象, 当阶数 $n$ 充分大时, $k$ 色完全图 $K_n$ 中就必然会出现同色三角形, 使得每一个 $k$ 色完全图 $K_n$ 都含同色三角形的最小 $n$ 记为 $r_k$, 于是定理一即是 $r_2=6$. 至于 $r_1=3$ 是不言自明的.
%%PROBLEM_END%%



%%PROBLEM_BEGIN%%
%%<PROBLEM>%%
例4. 证明: 在任何九个人中, 总可以找到三个人相互认识或四个人相互不认识.
%%<SOLUTION>%%
证明:我们用 9 个点 $A_1, A_2, \cdots, A_9$ 表示 9 个人, 每两点连一条边.
约定: 若 $A_i 、 A_j$ 两人相互认识, 则线段 $A_i A_i$ 染红色, 否则线段 $A_i A_j$ 染蓝色.
要证明的是: 在上述两色完全图 $K_9$ 中, 必存在红色 $K_3$, 或者存在蓝色 $K_4$.
若有一个顶点出发的红边数 $\geqslant 4$, 设为 $A_1 A_2 、 A_1 A_3 、 A_1 A_4 、 A_1 A_5$. 若 $A_2 、 A_3 、 A_4 、 A_5$ 中有两个点的连线是红色的, 例如 $A_2 A_3$, 则 $\triangle A_1 A_2 A_3$ 为红色三角形.
若 $A_2 、 A_3 、 A_4 、 A_5$ 中无两点连线是红色的, 则 $A_2 、 A_3 、 A_4 、 A_5$ 为顶点的子图是蓝色 $K_4$, 命题得证.
若每个顶点出发的红边数 $<4$, 则每个顶点出发的蓝边数都 $\geqslant 5$. 考虑 $A_1$, $A_2, \cdots, A_9$ 为顶点连同所有蓝边构成的图, 奇顶点的个数是偶数, 所以必有一个顶点.
例如 $A_1$ 是偶顶点, 即 $A_1$ 出发的蓝边数是偶数,所以 $A_1$ 出发的蓝边数 $\geqslant 6$. 设 $A_1 A_2, A_1 A_3, \cdots, A_1 A_7$ 为蓝边, 考虑六个顶点 $A_2, A_3, \cdots, A_7$, 它们每两点都有红边或蓝边连接.
按定理一, 或存在红色三角形, 或存在蓝色三角形.
在前一种情况下, 命题已经得证.
对后一种情况, 不妨设 $\triangle A_2 A_3 A_4$ 为蓝色三角形, 则以 $A_1 、 A_2 、 A_3 、 A_4$ 为顶点的完全图 $K_4$ 是蓝色的.
命题也成立.
再考虑一个两色完全图 $K_8$, 如图(<FilePath:./figures/fig-c8i6.png>) 所示.
我们用实线表示染红色的边, 用虚线表示染蓝色的边.
可见存在一种染法, 使 $K_8$ 无红色 $K_3$, 也无蓝色 $K_4$, 说明 $r(3,4)>8$.
综上所述, 即得 $r(3,4)=9$.
%%PROBLEM_END%%



%%PROBLEM_BEGIN%%
%%<PROBLEM>%%
例5. 把数 $1 、 2 、 3 、 4 、 5$ 任意分成两组.
证明: 总可以在某一组中找到这样两个数,它们之差与这一组中的某一个数相同.
%%<SOLUTION>%%
证明:设把数 $1 、 2 、 3 、 4 、 5$ 任意分成了 $A 、 B$ 两组.
取六个点, 并依次编号为 $1 、 2 、 3 、 4 、 5 、 6$. 其中任意两点 $i>j$, 都有 $1 \leqslant i-j \leqslant 5$. 两个点 $i>j$, 如果 $i-j$ 分在 $A$ 组, 则把 $i j$ 边染成红色; 如果 $i-j$ 分在 $B$ 组, 则把 $i j$ 边染成蓝色.
于是, 得到一个 2 色 6 阶完全图 $K_6$. 由本节例 1 知, 这个 $K_6$ 中有一个单色三角形, 设为 $\triangle i j k$, 并且 $i>j>k$. 这表明 $a=i-k, b=i-j, c=j-k$
这三个数分在同一组中, 并且有
$$
a-b=(i-k)-(i-j)=j-k=c .
$$
题中结论得证.
%%<REMARK>%%
注:: 在此例中可能有 $b=c$, 此时有 $a=2 b$. 因此,此题可改写为: 把数 1 、 $2 、 3 、 4 、 5$ 任意分成两组, 证明: 总可以在某一组中找到这样一个数, 它是同一组中某个数的 2 倍,或者是同一组中某两个数之和.
单色三角形的一种变形是所谓的异色三角形, 即三边颜色互不相同的三角形.
题是匈牙利数学奥林匹克试题.
%%PROBLEM_END%%



%%PROBLEM_BEGIN%%
%%<PROBLEM>%%
例6. 某俱乐部有 $3 n+1$ 人,每两人可一起玩下面三种游戏中的某一种: 象棋、围棋、跳棋.
已知每个人都与 $n$ 个人下象棋, 与 $n$ 个人下围棋, 与 $n$ 个人下跳棋.
证明: 这 $3 n+1$ 个人中必有这样三个人, 他们之间有下象棋的, 有下围棋的, 有下跳棋的.
%%<SOLUTION>%%
证明:$3 n+1$ 个人用 $3 n+1$ 个点表示,两人之间下象棋、下围棋、下跳棋, 则对应的两点分别用红线、蓝线、黑线连接.
于是, 得一个 3 色完全图 $K_{3 n+1}$. 本题即是证明: 在这个 3 色完全图 $K_{3 n+1}$ 中, 必存在一个三边不同色的异色三角形.
由一顶点引出的两条边如果不同色, 则称此两条边的夹角为异色角.
- 个三角形是异色三角形当且仅当它的三个内角都是异色角.
每一个顶点引出的 $3 n$ 条边, 其中红边、蓝边、黑边各有 $n$ 条, 因此, 由任一顶点引出的异色角有 $\mathrm{C}_3^2 n^2=3 n^2$ 个, 从而这个 3 色完全图 $K_{3 n+1}$ 中共有 $3 n^2(3 n+1)$ 个异色角.
另一方面, 完全图 $K_{3 n+1}$ 中共有 $\mathrm{C}_{3 n+1}^3=\frac{1}{2} n(3 n+1)(3 n-1)$ 个三角形.
把这些三角形看作"抽庶", 异色角看作"球". 因为 $3 n^2(3 n+-1)>n(3 n+1)(3 n-1)$, 于是, 3 色完全图 $K_{3 n+1}$ 中异色角的数目大于三角形个数的两倍, 由抽屉原理, 必有某个三角形,它有三个异色角,这个三角形即是异色三角形.
数学竞赛中经常出现与拉姆赛问题类同的试题, 我们再举几例作为本节的结束.
%%PROBLEM_END%%



%%PROBLEM_BEGIN%%
%%<PROBLEM>%%
例7. 大厅中会聚了 100 个客人,他们中每人至少认识 67 人, 证明: 在这些客人中一定可以找到 4 人,他们之中任何两人都彼此相识.
%%<SOLUTION>%%
证明:用 $A_1, A_2, \cdots, A_{100}$ 这 100 个顶点表示客人.
连接每两点并染以红、蓝两色, 当且仅当 $A_i$ 与 $A_j$ 互相认识时, 它们之间的边是红色.
本题用图论的语言就是: 红蓝 2 色完全图 $K_{100}$ 中, 如果每个顶点出发的红边至少有 67 条, 则 $K_{100}$ 中含有一个红色完全子图 $K_4$.
任取一顶点 $A_1$, 自它出发的红边至少有 67 条, 故必有一条红边 $A_1 A_2$, 因自 $A_2$ 出发的红边也至少有 67 条,故自 $A_1 、 A_2$ 出发的蓝边至多有 $32 \times 2=64$ 条, 它们涉及 66 个顶点, 因而必有一点, 例如 $A_3$, 使 $A_1 A_3 、 A_2 A_3$ 都是红边.
自 $A_1 、 A_2 、 A_3$ 出发的蓝边至多有 $32 \times 3=96$ 条, 涉及 99 个点, 故必还有一点, 记为 $A_4$, 使 $A_1 A_4 、 A_2 A_4 、 A_3 A_4$ 都是红边.
于是以 $A_1 、 A_2 、 A_3 、 A_4$ 为顶点的完全子图 $K_4$ 是红色的.
%%PROBLEM_END%%



%%PROBLEM_BEGIN%%
%%<PROBLEM>%%
例8. 一棱柱以五边形 $A_1 A_2 A_3 A_4 A_5$ 和 $B_1 B_2 B_3 B_4 B_5$ 为上、下底,这两个多边形的每一条边及每一条线段 $A_i B_j(i, j=1,2, \cdots, 5)$ 均涂上红色或蓝色, 每一个以棱柱顶点为顶点的, 以已涂色的线段为边的三角形都不是同色三角形.
求证:上、下底的 10 条边颜色一定相同.
%%<SOLUTION>%%
证明:先证上底 5 条边同色.
若不然, 则五边形 $A_1 A_2 A_3 A_4 A_5$ 至少有两条边不同色, 从而总有两条相邻边,譬如 $A_1 A_2$ 与 $A_1 A_5$ 不同色.
不妨设 $A_1 A_2$ 为红边, $A_1 A_5$ 为蓝边.
由 $A_1$ 到 $B_1 、 B_2 、 B_3 、 B_4 、 B_5$ 的 5 条边中至少有 3 条边同色, 不妨设 $A_1 B_i 、 A_1 B_j 、 A_1 B_k\left(i, j, k\right.$ 互不相等) 为红边.
因为 $\triangle A_1 B_i B_j$ 不同色, 所以 $B_i B_j$ 为蓝边, 同理 $A_2 B_i$ 也为蓝边, 进而推知 $A_2 B_j$ 为红边.
于是 $\triangle A_1 A_2 B_j$ 为红色三角形,这与题设矛盾.
同理可证下底五条边同色.
若上下底面的边不同色, 不妨设 $A_1 A_2 A_3 A_4 A_5$ 为红色, $B_1 B_2 B_3 B_4 B_5$ 为蓝色,并不妨设 $A_1 B_1$ 为蓝边, 于是由每个三角形不同色的假定, 可知 $A_1 B_5$ 、 $A_1 B_2$ 均为红线段,进而推出 $A_2 B_2 、 A_5 B_5$ 为蓝线段.
同理可知 $A_5 B_1 、 A_5 B_4$ 、 $A_2 B_1 、 A_2 B_3$ 等均为红线段, $A_3 B_3 、 A_4 B_4$ 为蓝线段.
于是 $A_4 B_1$ 和 $A_4 B_2$ 为蓝线段,这样得到蓝色三角形 $A_4 B_1 B_2$, 矛盾.
所以棱柱上、下底面的 10 条边同色.
%%PROBLEM_END%%



%%PROBLEM_BEGIN%%
%%<PROBLEM>%%
例9. 10 个地区之间有甲、乙两个国际航空公司服务,在任意两个地区之间都有且仅有由其中一个公司单独经营的直达航线 (可往返). 证明:两公司中必有某公司, 可以提供两条不经过同一地区的环游旅行线, 而且每条旅行线经过奇数个地区.
%%<SOLUTION>%%
证明:10 个地区用 10 个点 $u_1, u_2, \cdots, u_{10}$ 表示, $u_i$ 与 $u_j$ 两地区的航线若由甲公司经营, 则 $u_i u_j$ 边染红色 (图中用实线表示); 若由乙公司经营, 则 $u_i u_j$ 边染蓝色 (图中用虚线表示). 于是得到一个 2 色 10 阶完全图 $K_{10}$. 为证明结论, 即要证明这个 2 色 10 阶完全图 $K_{10}$ 中, 存在两个没有公共顶点、边数都是奇数并且同色的单色三角形或单色多边形.
2 色 10 阶完全图 $K_{10}$ 中必有单色三角形,设 $\triangle u_8 u_9 u_{10}$ 为单色三角形.
由例 1 又知, 以 $u_1, u_2, \cdots, u_7$ 为顶点的三角形中必有单色三角形, 设 $\triangle u_5 u_6 u_7$ 为单色三角形.
如果 $\triangle u_5 u_6 u_7$ 与 $\triangle u_8 u_9 u_{10}$ 同色, 则已无需再证.
下面设 $\triangle u_5 u_6 u_7$ 为红边三角形, 而 $\triangle u_8 u_9 u_{10}$ 为蓝边三角形.
在 $\left\{u_5, u_6, u_7\right\}$ 与 $\left\{u_8, u_9, u_{10}\right\}$ 两个顶点集之间共有 $3 \times 3=9$ 条边,由抽庶原理知,其中必有 5 条同色,不妨为红色.
这 5 条边由 $\left\{u_8, u_9, u_{10}\right\}$ 引出, 故必有某一顶点, 由该点引出 2 条红边, 设为 $u_8 u_6 、 u_8 u_7$, 如图(<FilePath:./figures/fig-c8i8.png>) 所示.
于是, 又有一个红边三角形 $u_6 u_7 u_8$.
考虑由 $u_1 、 u_2 、 u_3 、 u_4 、 u_5$ 为顶点的 2 色 5 阶完全图 $K_5$. 如果这个 $K_5$ 中有单色三角形, 则不论这个单色三角形是红色边或者是蓝色边的, 再加上红边三角形 $u_6 u_7 u_8$ 或蓝边三角形 $u_8 u_9 u_{10}$, 在 2 色完全图 $K_{10}$ 中都有两个没有公共顶点、同色的单色三角形.
否则, 这个 2 色 5 阶完全图 $K_5$ 中没有单色三角形, 于是易知 $K_5$ 中有两个单色五边形, 一个红色边五边形, 另一个蓝色边五边形.
因此结论成立.
%%<REMARK>%%
注:: 上题中若把 10 个地区改为 9 个地区,则结论不真.
例子如下: 把 9 个地区分为 3 组, 即 $\left\{u_1, u_2, u_3, u_4, u_5\right\}=A,\left\{u_6, u_7, u_8\right\}=B,\left\{u_9\right\}, A$ 中的 5 个地区之间的航线由甲公司经营, $B$ 中的 3 个地区之间的航线由乙公司经营, $A$ 与 $B$ 之间、 $u_9$ 与 $A$ 之间的航线由乙公司经营, $u_9$ 与 $B$ 之间的航线由甲公司经营.
%%PROBLEM_END%%


