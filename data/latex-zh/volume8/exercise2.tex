
%%PROBLEM_BEGIN%%
%%<PROBLEM>%%
问题1. 设 $z$ 是模为 1 的复数, 则函数 $f(z)=z^2+\frac{1}{z^2}$ 的最小值 ( ).
(A) 是 0
(B) 是 -2
(C) 是 2
(D) 不存在
%%<SOLUTION>%%
B.
由 $|z|=1$, 可令 $z=\cos \theta+i \sin \theta$, 则
$$
\begin{aligned}
f(z) & =z^2+\frac{1}{z^2}=z^2+\overline{z^2} \\
& =(\cos 2 \theta+i \sin 2 \theta)+(\cos 2 \theta-i \sin 2 \theta) \\
& =2 \cos 2 \theta .
\end{aligned}
$$
故当 $\cos 2 \theta=-1$ (或 $z=\mathrm{i}$ )时,函数 $f(z)$ 有最小值 -2 .
%%PROBLEM_END%%



%%PROBLEM_BEGIN%%
%%<PROBLEM>%%
问题2. 已知 $\left|z_1\right|=\left|z_2\right|=1, z_1+z_2=\frac{1}{2}+\frac{\sqrt{3}}{2} \mathrm{i}$, 求复数 $z_1, z_2$.
%%<SOLUTION>%%
如图(<FilePath:./figures/fig-c2a2.png>), 设 $\overrightarrow{O A}$ 对应于复数 $z_1, \overrightarrow{O B}$ 对应于复数 $z_2, z_1+z_2$ 对应于向量 $\overrightarrow{O C}$.
由 $z_1+z_2=\frac{1}{2}+\frac{\sqrt{3}}{2} \mathrm{i}$ 知, $\left|z_1+z_2\right|=1$. 所以 $\triangle A O C$ 和 $\triangle B O C$ 都是等边三角形, 于是
$$
\begin{aligned}
& z_1=1, z_2=-\frac{1}{2}+\frac{\sqrt{3}}{2} \mathrm{i} \text { 或 } \\
& \dot{z}_1=-\frac{1}{2}+\frac{\sqrt{3}}{2} \mathrm{i}, z_2=1 .
\end{aligned}
$$
%%PROBLEM_END%%



%%PROBLEM_BEGIN%%
%%<PROBLEM>%%
问题3. 已知非零复数 $z$ 满足 $|z-\mathrm{i}|=1$, 且 $\arg z=\theta$, 求
(1) $\theta$ 的取值范围; (2) 复数 $z$ 的模 (用 $\theta$ 表示);(3)复数 $z^2-z i$ 的辐角.
%%<SOLUTION>%%
(1) 因为 $|z-\mathrm{i}|=1$, 故 $z$ 在复平面上的对应点 $P$ 在以 $(0,1)$ 为圆心, 半径为 1 的圆上 (去除 $(0,0)$ 点), 如图(<FilePath:./figures/fig-c2a3-1.png>), 所以 $\theta$ 的取值范围是 $0< \theta<\pi$.
(2) 如图(<FilePath:./figures/fig-c2a3-2.png>), 在 Rt $\triangle A O P$ 中, 因为 $|O P|=2 \sin \theta$, 故 $|z|=2 \sin \theta$.
(3) 由于 $|z-\mathrm{i}|=1$, 故可令 $z-\mathrm{i}=\cos \varphi+\mathrm{i} \sin \varphi(\varphi \in \mathbf{R})$, 于是
$$
\begin{aligned}
z^2-z \mathrm{i} & =z(z-\mathrm{i})=2 \sin \theta(\cos \theta+\mathrm{i} \sin \theta) \cdot(\cos \varphi+\mathrm{i} \sin \varphi) \\
& =2 \sin \theta[\cos (\theta+\varphi)+\mathrm{i} \cdot \sin (\theta+\varphi)] .
\end{aligned}
$$
又由 $\cos \varphi+\mathrm{i} \sin \varphi=z-\mathrm{i}$
$$
\begin{aligned}
& =2 \sin \theta(\cos \theta+i \sin \theta)-\mathrm{i} \\
& =2 \sin \theta \cos \theta+\mathrm{i}\left(2 \sin ^2 \theta-1\right) .
\end{aligned}
$$
$$
\begin{aligned}
& =\sin 2 \theta-i \cos 2 \theta \\
& =\cos \left(2 \theta-\frac{\pi}{2}\right)+i \sin \left(2 \theta-\frac{\pi}{2}\right) .
\end{aligned}
$$
所以 $\varphi=2 k \pi+2 \theta-\frac{\pi}{2}(k \in \mathbf{Z}), \varphi+\theta=2 k \pi+3 \theta-\frac{\pi}{2}(k \in \mathbf{Z})$.
即 $\operatorname{Arg}\left(z^2-z \mathrm{i}\right)=2 k \pi+3 \theta-\frac{\pi}{2}(k \in \mathbf{Z})$.
%%<REMARK>%%
注:对于已知 $|z|=r(r>0)$ 的有关问题, 可以从以下四个方面去思考:
(1) 令 $z=r(\cos \theta+\mathrm{i} \sin \theta)$;
(2) 令 $z=a+b \mathrm{i}(a, b \in \mathbf{R})$ 且 $a^2+b^2=r^2$;
(3) 由 $|z|^2=r^2$ 得 $z \bar{z}=r^2, z=\frac{r^2}{\bar{z}}, \bar{z}=\frac{r^2}{z}$;
(4) $z$ 在复平面内的对应点在以原点为圆心 $r$ 为半径的圆上.
%%PROBLEM_END%%



%%PROBLEM_BEGIN%%
%%<PROBLEM>%%
问题4. 已知等边 $\triangle A B C$ 的两个顶点坐标是 $A(2,1), B(3,2)$, 求顶点 $C$ 的对应坐标.
%%<SOLUTION>%%
记 $A, B, C$ 的对应复数分别为 $z_A=2+\mathrm{i}, z_B=3+2 \mathrm{i}, z_C$, 则由
$$
z_C=z_A+\left(z_B-z_A\right)\left(\cos 60^{\circ} \pm \operatorname{isin} 60^{\circ}\right),
$$
得
$$
z_C=(2+i)+(1+i)\left(\frac{1}{2} \pm \frac{\sqrt{3}}{2} i\right)=\frac{5 \mp \sqrt{3}}{2}+\frac{3 \pm \sqrt{3}}{2} i,
$$
即 $C$ 点坐标是 $C\left(\frac{5-\sqrt{3}}{2}, \frac{3+\sqrt{3}}{2}\right)$ 或 $C\left(\frac{5+\sqrt{3}}{2}, \frac{3-\sqrt{3}}{2}\right)$.
%%PROBLEM_END%%



%%PROBLEM_BEGIN%%
%%<PROBLEM>%%
问题5. 已知 $z_1 、 z_2 \in \mathbf{C}$, 且 $\left|z_1\right|=2,\left|z_2\right|=9,\left|5 z_1-z_2\right|=9$, 试求 $\mid 5 z_1+ z_2 \mid$ 的值.
%%<SOLUTION>%%
由模引人辐角, 设 $z_1=2(\cos \alpha+i \sin \alpha), z_2=9(\cos \beta+\mathrm{i} \sin \beta)$, 代入 $\left|5 z_1-z_2\right|=9$, 便得
$$
\sqrt{(10 \cos \alpha-9 \cos \beta)^2+(10 \sin \alpha-9 \sin \beta)^2}=9,
$$
即
$$
\begin{gathered}
181-180 \cos (\alpha-\beta)=81, \\
\cos (\alpha-\beta)=\frac{100}{180} .
\end{gathered}
$$
故 $\quad\left|5 z_1+z_2\right|=\sqrt{181+180 \cos (\alpha-\beta)}=\sqrt{281}$.
%%<REMARK>%%
注:用如下恒等式, 可给出此题的另一种解法.
$$
\left|z_1+z_2\right|^2+\left|z_1-z_2\right|^2=2\left(\left|z_1\right|^2+\left|z_2\right|^2\right) \text {. }
$$
%%PROBLEM_END%%



%%PROBLEM_BEGIN%%
%%<PROBLEM>%%
问题6. 已知 $|z|=1, z^{11}+z=1$, 求复数 $z$.
%%<SOLUTION>%%
由 $|z|=1$, 可设 $z=\cos \theta+i \sin \theta$, 且 $0 \leqslant \theta<2 \pi$, 代入 $z^{11}+z=1$,
得
$$
\begin{gathered}
(\cos \theta+i \sin \theta)^{11}+(\cos \theta+i \sin \theta)=1, \\
(\cos 11 \theta+\cos \theta-1)+(\sin 11 \theta+\sin \theta) i=0,
\end{gathered}
$$
所以
$$
\left\{\begin{array}{l}
\cos 11 \theta+\cos \theta-1=0, \\
\sin 11 \theta+\sin \theta=0,
\end{array}\right.
$$
即
$$
\cos 11 \theta=1-\cos \theta, \label{eq1}
$$
且
$$
\sin 11 \theta=-\sin \theta, \label{eq2}
$$
由${ 式\ref{eq1}}^2+{式\ref{eq2}}^2$, 可知
$$
(1-\cos \theta)^2+(-\sin \theta)^2=1,
$$
于是
$$
\begin{gathered}
\cos \theta=\frac{1}{2}, \\
\sin \theta= \pm \frac{\sqrt{3}}{2} .
\end{gathered}
$$
从而经验证知, $z=\frac{1}{2} \pm \frac{\sqrt{3}}{2} \mathrm{i}$ 是原方程的解.
%%<REMARK>%%
注:对 $z^{11}=1-z$ 两边取模, 得 $|z-1|=1$, 并结合 $|z|=1$, 亦可给出简明解法.
%%PROBLEM_END%%



%%PROBLEM_BEGIN%%
%%<PROBLEM>%%
问题7. 已知复数 $z_1 、 z_2$ 满足 $\left|z_1\right|==\left|z_2\right|=1$.
(1) 若 $z_1-z_2=\frac{\sqrt{6}}{3}+\frac{\sqrt{3}}{3} \mathrm{i}$, 求 $z_1 、 z_2$ 的值;
(2) 若 $z_1+z_2=\frac{12}{13}-\frac{5}{13} \mathrm{i}$, 求 $z_1 z_2$ 的值.
%%<SOLUTION>%%
注意到 $\left|z_1+z_2\right| 、\left|z_1-z_2\right|$ 分别是以 $\left|z_1\right| 、\left|z_2\right|$ 为邻边的平行四边形的对角线.
(1) 因为 $\left|z_1-z_2\right|=1=\left|z_1\right|=\left|z_2\right|$, 所以 $\left|z_1\right| 、\left|z_2\right| 、\left|z_1-z_2\right|$ 是正三角形的边.
复数 $z_1 、 z_2$ 分别可以看作 $z_1-z_2$ 按顺时针或逆时针旋转 $\frac{\pi}{3}$ 、 $\frac{2 \pi}{3}$ 得到.
于是
$$
\begin{aligned}
& z_1=\left(\frac{\sqrt{6}}{3}+\frac{\sqrt{3}}{3} \mathrm{i}\right)\left(\frac{1}{2}-\frac{\sqrt{3}}{2} \mathrm{i}\right)=\frac{1}{6}[(\sqrt{6}+3)+(\sqrt{3}-3 \sqrt{2}) \mathrm{i}], \\
& z_2=\left(\frac{\sqrt{6}}{3}+\frac{\sqrt{3}}{3} \mathrm{i}\right)\left(-\frac{1}{2}-\frac{\sqrt{3}}{2} \mathrm{i}\right)=\frac{1}{6}[(-\sqrt{6}+3)+(-\sqrt{3}-3 \sqrt{2}) \mathrm{i}]
\end{aligned}
$$
或 $z_1=\left(\frac{\sqrt{6}}{3}+\frac{\sqrt{3}}{3} \mathrm{i}\right)\left(\frac{1}{2}+\frac{\sqrt{3}}{2} \mathrm{i}\right)=\frac{1}{6}[(\sqrt{6}-3)+(\sqrt{3}+3 \sqrt{2}) \mathrm{i}]$,
$$
z_2=\left(\frac{\sqrt{6}}{3}+\frac{\sqrt{3}}{3} \mathrm{i}\right)\left(-\frac{1}{2}+\frac{\sqrt{3}}{2} \mathrm{i}\right)=\frac{1}{6}[(-\sqrt{6}-3)+(-\sqrt{3}+3 \sqrt{2}) \mathrm{i}] .
$$
(2) 因为 $\left|z_1+z_2\right|=1=\left|z_1\right|=\left|z_2\right|$, 所以 $\left|z_1\right| 、\left|z_2\right| 、\left|z_1-z_2\right|$ 组成顶角为 $\frac{2 \pi}{3}$, 以 $\left|z_1\right| 、\left|z_2\right|$ 为腰长的等腰三角形.
$z_1+z_2$ 按逆(顺) 时针旋转 $\frac{\pi}{3}$ 后, 就可得出 $z_1$ 或 $z_2$ (如图(<FilePath:./figures/fig-c2a7.png>)所示).
设 $z_0=\cos 60^{\circ}+\mathrm{i} \sin 60^{\circ}$, 则当 $z_1=\left(z_1+z_2\right) z_0$ 时, $z_2= \frac{z_1+z_2}{z_0}$; 当 $z_1=\frac{z_1+z_2}{z_0}$ 时, $z_2=\left(z_1+z_2\right) z_0$.
于是 $z_1 z_2=\left(z_1+z_2\right)^2=\left(\frac{12}{13}-\frac{5}{13} \mathrm{i}\right)^2=\frac{119}{169}-\frac{120}{169} \mathrm{i}$.
%%<REMARK>%%
注:对于第(1)小题,应用复数减法的几何意义就可直接求出 $z_1 、 z_2$, 但应注意复数旋转的方向和具有相同的始点.
%%PROBLEM_END%%



%%PROBLEM_BEGIN%%
%%<PROBLEM>%%
问题8. 求证: $\sin (4 \arcsin x)=4 x \cdot \sqrt{1-x^2} \cdot\left(1-2 x^2\right) .(|x| \leqslant 1)$
%%<SOLUTION>%%
因为
$$
(\cos \alpha+i \sin \alpha)^4=1 \cdot(\cos 4 \alpha+i \sin 4 \alpha)=\cos 4 \alpha+i \sin 4 \alpha,
$$
又因为
$$
\begin{aligned}
(\cos \alpha+\mathrm{i} \sin \alpha)^4 & =\left[(\cos \alpha+\mathrm{i} \sin \alpha)^2\right]^2=\left[\left(\cos ^2 \alpha-\sin ^2 \alpha\right)+2 \sin \alpha \cos \alpha \cdot \mathrm{i}\right]^2 \\
& =\left(\cos ^4 \alpha-6 \sin ^2 \alpha \cos ^2 \alpha+\sin ^4 \alpha\right)+4 \sin \alpha \cos \alpha\left(\cos ^2 \alpha-\sin ^2 \alpha\right) \mathrm{i},
\end{aligned}
$$
所以
$$
\cos 4 \alpha+\operatorname{isin} 4 \alpha=\left(\cos ^4 \alpha-6 \sin ^2 \alpha \cos ^2 \alpha+\sin ^4 \alpha\right)+4 \sin \alpha \cdot \cos \alpha\left(\cos ^2 \alpha-\sin ^2 \alpha\right) \text {. }
$$
根据复数相等的定义得
$$
\sin 4 \alpha=4 \sin \alpha \cos \alpha\left(\cos ^2 \alpha-\sin ^2 \alpha\right) .
$$
将 $\alpha=\arcsin x$ 代入上式则有
$$
\begin{aligned}
\sin (4 \arcsin x) & =4 \sin (\arcsin x) \cdot \cos (\arcsin x) \cdot\left[\cos ^2(\arcsin x)-\sin ^2(\arcsin x)\right] \\
& =4 x \cdot \sqrt{1-x^2} \cdot\left[\left(\sqrt{1-x^2}\right)^2-x^2\right] \\
& =4 x \cdot \sqrt{1-x^2} \cdot\left(1-2 x^2\right) .
\end{aligned}
$$
即 $\sin (4 \arcsin x)=4 x \cdot \sqrt{1-x^2} \cdot\left(1-2 x^2\right)$, 证毕.
%%PROBLEM_END%%


