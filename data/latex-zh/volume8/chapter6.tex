
%%TEXT_BEGIN%%
本章中,我们将利用空间向量解决各种立体几何中的计算问题.
%%TEXT_END%%



%%PROBLEM_BEGIN%%
%%<PROBLEM>%%
例1. 在三棱雉 $A-B C D$ 中, 平面 $A B D \perp$ 平面 $B C D, \angle B A D=90^{\circ}$, $A B=A D=3 \sqrt{2}, \angle B D C=90^{\circ}, B C=2 C D$.
(1) 求线段 $A C$ 的长度;
(2)求二面角 $B-A D-C$ 的大小;
(3) 求异面直线 $A C$ 和 $B D$ 所成的角.
%%<SOLUTION>%%
分析:与解如图(<FilePath:./figures/fig-c6i1.png>) 建立空间直角坐标系, 则 $A(0,0,3), B(0,-3,0), D(0,3,0), C(2 \sqrt{3}, 3$, $0), O(0,0,0)$.
(1) $A C=\sqrt{12+9+9}=\sqrt{30}$.
(2) $\overrightarrow{A B}=\{0,-3,-3\}, \overrightarrow{D C}=\{2 \sqrt{3}, 0,0\}$, $\overrightarrow{A D}=\{0,3,-3\}$.
因为 $\overrightarrow{A B} \cdot \overrightarrow{C D}=0, \overrightarrow{D C} \cdot \overrightarrow{A D}=0$, 所以 $C D \perp$ 平面 $A B D$, 故平面 $A C D \perp$ 平面 $A B D$.
所以二面角 $B-A D-C$ 的大小为 $90^{\circ}$.
(3) $\overrightarrow{A C}=\{2 \sqrt{3}, 3,-3\}, \overrightarrow{B D}=\{0,6,0\}, \cos \theta=\frac{\overrightarrow{A C} \cdot \overrightarrow{B D}}{|\overrightarrow{A C}| \cdot|\overrightarrow{B D}|}= \frac{18}{6 \cdot \sqrt{30}}=\frac{\sqrt{30}}{10}$, 所以异面直线 $A C$ 和 $B D$ 所成的角为 $\arccos \frac{\sqrt{30}}{10}$.
%%PROBLEM_END%%



%%PROBLEM_BEGIN%%
%%<PROBLEM>%%
例2. 如图(<FilePath:./figures/fig-c6i2.png>) 所示, 在长方体 $A B C D-A_1 B_1 C_1 D_1$ 中, 点 $E 、 F$ 分别在 $B B_1 、 D D_1$ 上, 且 $A E \perp A_1 B, A F \perp A_1 D$.
(1) 证明: $A_1 C \perp$ 平面 $A E F$;
(2) 若规定两个平面所成的角是这两个平面所组成的二面角中的锐角 (或直角), 在 $A B=4, A D=3, A A_1=5$ 时, 求平面 $A E F$ 与平面 $D_1 B_1 B D$ 所成角的大小 (用反三角函数值表示);
(3) 条件同 (2), 计算 $A_1 D$ 和平面 $A E F$ 所成的角.
%%<SOLUTION>%%
分析:与解 (1) 如图(<FilePath:./figures/fig-c6i2.png>) 所示建立空间直角坐标系.
因为 $A_1 B \perp A E$, 即 $\overrightarrow{A_1 B} \cdot \overrightarrow{A E}=0$, 即 $\{0, a,-c\} \cdot \left\{0, a, h_1\right\}=0$, 所以 $a^2-h_1 c=0$.
因为 $A_1 D \perp A F$, 即 $\overrightarrow{A_1 D} \cdot \overrightarrow{A F}=0$, 即 $\{b, 0, c\} \cdot\left\{-b, 0, h_2\right\}=0$, 所以 $b^2-h_2 c=0$.
因为 $\overrightarrow{A_1 C} \cdot \overrightarrow{A E}=\{-b, a,-c\} \cdot\left\{0, a, h_1\right\}=a^2-h_1 c=0$, 所以 $A_1 C \perp A E$.
因为 $\overrightarrow{A_1 C} \cdot \overrightarrow{A F}=\{-b, a,-c\} \cdot\left\{-b, 0, h_2\right\}=b^2-h_2 c=0$, 所以 $A_1 C \perp A F$.
所以 $A_1 C \perp$ 平面 $A E F$, 证毕.
(2) 在空间中有定理: 若两条直线分别垂直于两个平面, 则这两条直线所成的角与这两个平面所成的角的大小相等.
设 $\vec{a}=\{x, y, z\},|\vec{a}| \neq 0, \vec{a} \perp$ 平面 $D_1 B_1 B D, \overrightarrow{D D_1}=\{0,0,5\}$, $\overrightarrow{D B}=\{3,4,0\}, \overrightarrow{A_1 C}=\{-3,4,-5\}$, 则
$$
\left\{\begin{array} { l } 
{ \vec { a } \cdot \vec { D D _ { 1 } } = 0 , } \\
{ \vec { a } \cdot \vec { D B } = 0 , } \\
{ | \vec { a } | \neq 0 . }
\end{array} \Rightarrow \left\{\begin{array}{l}
5 z=0, \\
3 x+4 y=0, \\
x^2+y^2+z^2 \neq 0 .
\end{array}\right.\right.
$$
所以 $\vec{a}=\left\{x,-\frac{3}{4} x, 0\right\}(x \neq 0)$, 而 $\vec{a} \perp$ 平面 $D_1 B_1 B D$, 由 $A_1 C \perp$ 平面 $A E F$, 设 $\vec{a}$ 和 $\overrightarrow{A_1} \vec{C}$ 所在直线所成的角为 $\theta$, 则
$$
\cos \theta=\left|\frac{\vec{a} \cdot \overrightarrow{A_1 C}}{|\vec{a}| \cdot\left|\overrightarrow{A_1 C}\right|}\right|=\left|\frac{-3 x-3 x}{5 \sqrt{2} \cdot \frac{5}{4} \cdot|x|}\right|=\frac{12 \sqrt{2}}{25} .
$$
所以平面 $A E F$ 与平面 $D_1 B_1 B D$ 所成的角为 $\arccos \frac{12 \sqrt{2}}{25}$.
(3)与平面垂直的向量称为平面的法向量, 要求直线与平面所成角的大小只要求出直线与平面法向量所成的那个不超过 $90^{\circ}$ 的角, 然后求出其余角即可.
特别地当直线与平面的法向量平行时,直线与该平面垂直.
显然 $\overrightarrow{A_1 C}=\{-3,4,-5\}$ 为平面 $A E F$ 的法向量, 且 $\overrightarrow{A D_1}=\{3,0,5\}$, 所以
$$
\cos \varphi=\left|\frac{\overrightarrow{A_1 C} \cdot \overrightarrow{A_1 D}}{\left|\overrightarrow{A_1 C}\right| \cdot\left|\overrightarrow{A_1 D}\right|}\right|=\left|\frac{-9-25}{5 \sqrt{2} \cdot \sqrt{34}}\right|=\frac{\sqrt{17}}{5} .
$$
由 $\frac{\pi}{2}-\arccos \frac{\sqrt{17}}{5}=\arcsin \frac{\sqrt{17}}{5}$ 知, $A_1 D$ 和平面 $A E F$ 所成的角为 $\arcsin \frac{\sqrt{17}}{5}$.
%%PROBLEM_END%%



%%PROBLEM_BEGIN%%
%%<PROBLEM>%%
例3. (1) 直线 $P A$ 交平面 $\alpha$ 于点 $A$, 点 $P$ 在直线 $P A$ 上, $\overrightarrow{n_0}$ 是垂直于平面 $\alpha$ 的单位向量,试叙述 $\left|\overrightarrow{P A} \cdot \vec{n}_0\right|$ 的几何意义;
(2) 在长方体 $A B C D-A_1 B_1 C_1 D_1$ 中, $A B=6, A D=A A_1=4$, 求点 $B_1$ 到平面 $A C D_1$ 的距离;
(3) 第 (2) 小题的条件下, 设 $P 、 Q 、 R$ 分别为 $A_1 B_1 、 B_1 C_1$ 和 $B B_1$ 的中点, 求证平面 $A C D_1$ 平行于平面 $P Q R$.
%%<SOLUTION>%%
分析:与解 (1) <1> $P A \perp$ 平面 $\alpha$ 时, $\overrightarrow{P A}$ 与 $\overrightarrow{n_0}$ 的夹角为 0 或 $\pi, \overrightarrow{P A} \cdot \overrightarrow{n_0}=\pm|\overrightarrow{P A}|$, 所以 $\left|\overrightarrow{P A} \cdot \overrightarrow{n_0}\right|=|\overrightarrow{P A}|$;
<2> $P A$ 不垂直于平面 $\alpha$ 时, 过点 $P$ 作 $P O \perp \alpha$ 于点 $O$, 设向量 $\overrightarrow{P A}$ 与 $\overrightarrow{n_0}$ 的夹角为 $\theta$, 则 $\left|\overrightarrow{P A} \cdot \vec{n}_0\right|=|| \overrightarrow{P A}|\cdot 1 \cdot \cos \theta|=|\overrightarrow{P A}| \cdot|\cos \theta|=|P O|$.
所以由<1>和<2>可知 $\left|\overrightarrow{P A} \cdot \overrightarrow{n_0}\right|$ 为点 $P$ 到平面 $\alpha$ 的距离.
(2) 如图(<FilePath:./figures/fig-c6i3.png>), 建立空间直角坐标系, 则 $A(4,0,0), B_1(4,6,4), C(0$, $6,0), D_1(0,0,4), \overrightarrow{A D_1}=\{-4,0,4\}, \overrightarrow{C D_1}=\{0,-6,4\}, B_1 C=\{-4$, $0,-4\}$, 设 $\vec{n}_0=\{x, y, z\}$ 且 $x^2+y^2+z^2=1$ 为平面 $A C D_1$ 的法向量.
因为
$$
\left\{\begin{array}{l}
\overrightarrow{n_0} \cdot \overrightarrow{A D_1}=0, \\
\overrightarrow{n_0} \cdot \overrightarrow{C D_1}=0, \\
x^2+y^2+z^2=1,
\end{array}\right.
$$
所以
$$
\left\{\begin{array}{l}
-4 x+4 z=0, \\
-6 y+4 z=0, \\
x^2+y^2+z^2=1 .
\end{array}\right.
$$
由此解得取 $\overrightarrow{n_0}=\left\{\frac{3}{\sqrt{2 \overline{2}}}, \frac{2}{\sqrt{22}}, \frac{3}{\sqrt{22}}\right\}, B_1$ 到平面 $A C D_1$ 的距离为
$$
d_1=\left|\overrightarrow{B_1 C} \cdot \overrightarrow{n_0}\right|=\left|-4 \times \frac{3}{\sqrt{22}}-4 \times \frac{3}{\sqrt{22}}\right|=\frac{12 \sqrt{22}}{11} \text {. }
$$
(3) $P(4,3,4), Q(2,6,4), R(4,6,2)$.
设 $\overrightarrow{m_0}=\{x, y, z\}$, 且 $x^2+y^2+z^2=1$ 为平面 $P Q R$ 的法向量, $\overrightarrow{P Q}= \{-2,3,0\}, \overrightarrow{P R}=\{0,3,-2\}$.
取 $\overrightarrow{m_0}=\left\{\frac{3}{\sqrt{22}}, \frac{2}{\sqrt{2} \overline{2}}, \frac{3}{\sqrt{22}}\right\}$.
又因为 $\overrightarrow{n_0}=\left\{\frac{3}{\sqrt{22}}, \frac{2}{\sqrt{22}}, \frac{3}{\sqrt{22}}\right\}$, 显然 $\overrightarrow{m_0} / / \overrightarrow{n_0}$, 所以平面 $A C D_1$ 平行于平面 $P Q R$.
%%PROBLEM_END%%



%%PROBLEM_BEGIN%%
%%<PROBLEM>%%
例4. 如图(<FilePath:./figures/fig-c6i4.png>) 所示,已知正四棱雉 $S-A B C D$ 的底面边长为 6 , 高为 $3, P 、 Q 、 R$ 分别在 $S C 、 S B$ 、 $S D$ 上, 且 $S P: P C=1: 2, S Q: Q B=2: 1, S R: R D=2: 1$.
(1) 求证: $S A / /$ 平面 $P Q R$ 并求出 $S A$ 到平面 $P Q R$ 的距离;
(2) 求点 $P$ 到直线 $B D$ 的距离;
(3) 若 $M 、 N$ 分别是 $B D$ 和 $S C$ 上的动点, 求线段 $M N$ 长度的最小值.
%%<SOLUTION>%%
分析:与解 (1) 如图(<FilePath:./figures/fig-c6i4.png>) 所示, 建立空间直角坐标系, 则 $A(3,-3,0)$, $B(3,3,0), C(-3,3,0), D(-3,-3,0), S(0,0,3), P(-1,1,2)$, $Q(2,2,1), R(-2,-2,1)$, 设 $G$ 是 $S A$ 上任一点.
因为 $\overrightarrow{A G} / / \overrightarrow{A S}$, 所以 $\overrightarrow{A G}=k \overrightarrow{A S}=k\{-3,3,3\}=\{-3 k, 3 k, 3 k\} (k \in \mathbf{R})$.
所以 $\overrightarrow{A G}=\overrightarrow{O G}-\overrightarrow{O A}=\{-3 k, 3 k, 3 k\}, \overrightarrow{O G}=\{3-3 k,-3+3 k, 3 k\}$, $\overrightarrow{G R}=\{3 k-5,-3 k+1,-3 k+1\}$.
设 $\overrightarrow{n_0}$ 为平面 $P Q R$ 的一个单位法向量, 且 $\vec{n}_0=\{x, y, z\}$.
$\overrightarrow{P Q}=\{3,1,-1\}, \overrightarrow{Q R}=\{-4,-4,0\}$, 则
$$
\left\{\begin{array}{l}
\overrightarrow{n_0} \cdot \overrightarrow{P Q}=0, \\
\overrightarrow{n_0} \cdot \overrightarrow{Q R}=0, \\
\left|\overrightarrow{n_0}\right|=1 .
\end{array}\right.
$$
所以
$$
\left\{\begin{array}{l}
3 x+y-z=0, \\
-4 x-4 y=0, \\
x^2+y^2+z^2=1 .
\end{array}\right.
$$
由此解得
$$
\left\{\begin{array} { l } 
{ x = \frac { \sqrt { 6 } } { 6 } , } \\
{ y = - \frac { \sqrt { 6 } } { 6 } , } \\
{ z = \frac { \sqrt { 6 } } { 3 } }
\end{array} \left\{\begin{array}{l}
x=-\frac{\sqrt{6}}{6}, \\
y=\frac{\sqrt{6}}{6}, \\
z=-\frac{\sqrt{6}}{3} .
\end{array}\right.\right.
$$
取 $\overrightarrow{n_0}=\left\{\frac{\sqrt{6}}{6},-\frac{\sqrt{6}}{6}, \frac{\sqrt{6}}{3}\right\}$, 而 $\overrightarrow{A S}=\{-3,3,3\}$, 则 $\overrightarrow{n_0} \cdot \overrightarrow{A S}=-3 \times \frac{\sqrt{6}}{6}- \frac{\sqrt{6}}{6} \times 3+3 \times \frac{\sqrt{6}}{3}=0$.
所以 $A S / /$ 平面 $P Q R$, 又 $\left|\overrightarrow{G R} \cdot \overrightarrow{n_0}\right|=\frac{2 \sqrt{6}}{3}$, 所以 $S A$ 到平面 $P Q R$ 的距离为 $\frac{2 \sqrt{6}}{3}$.
(2) $\overrightarrow{P O}=\{1,-1,-2\}, \overrightarrow{B D}=\{-6,-6,0\}$, 所以 $\overrightarrow{P O} \cdot \overrightarrow{B D}=\{1,-1$, $-2\} \cdot\{-6,-6,0\}=0$, 即 $P O \perp B D$, 点 $P$ 到 $B D$ 的距离即为线段 $P O$ 的长度,故点 $P$ 到 $B D$ 的距离 $P O=\sqrt{1^2+1^2+2^2}=\sqrt{6}$.
(3) 因为 $\overrightarrow{O N}-\overrightarrow{O S}=\overrightarrow{S N}=l\{-3,3,-3\}=\{-3 l, 3 l,-3 l\}$, 所以 $N(-3 l, 3 l, 3-3 l)$.
同理, $M(t, t, 0)(t \in \mathbf{R})$.
所以 $M N^2=(t+3 l)^2+(t-3 l)^2+(3 l-3)^2=2 t^2+27\left(l-\frac{1}{3}\right)^2+6$.
故 $t=0, l=\frac{1}{3}$ 时, $M N_{\text {min }}=\sqrt{6}$.
%%PROBLEM_END%%



%%PROBLEM_BEGIN%%
%%<PROBLEM>%%
例5. 如图(<FilePath:./figures/fig-c6i5.png>) 所示, 平行六面体 $A B C D- A_1 B_1 C_1 D_1$ 的底面是菱形, 且 $\angle C_1 C B=\angle C_1 C D= \angle B C D=60^{\circ}$,
(1) 证明: $C C_1 \perp B D$;
(2)当 $\frac{C D}{C C_1}$ 为何值时 $A_1 C \perp$ 平面 $C_1 B D$ ?
%%<SOLUTION>%%
分析:与解 (1) 设 $\overrightarrow{C D}=\vec{x}, \overrightarrow{C B}=\vec{y}, \overrightarrow{C C_1}= \vec{z}$, 且 $|\vec{x}|=a,|\vec{y}|=a,|\vec{z}|=b$, 则 $\overrightarrow{B D}=\vec{x}-\vec{y}$,
因为 $\overrightarrow{C C_1} \cdot \overrightarrow{B D}=\vec{z} \cdot(\vec{x}-\vec{y})=\vec{z} \cdot \vec{x}-\vec{z} \cdot \vec{y}=a b \cos 60^{\circ}-a b \cos 60^{\circ}=0$,
所以 $C C_1 \perp B D$, 证毕.
(2) $\overrightarrow{C A_1}=\vec{x}+\vec{y}+\vec{z}, \overrightarrow{C_1 D}=\vec{x}-\vec{z}$.
$$
\begin{aligned}
\overrightarrow{C A_1} \cdot \overrightarrow{C_1 D} & =(\vec{x}+\vec{y}+\vec{z}) \cdot(\vec{x}-\vec{z}) \\
& =\vec{x}^2-\vec{x} \cdot \vec{z}+\vec{y} \cdot \vec{x}-\vec{y} \cdot \vec{z}+\vec{z} \cdot \vec{x}-\vec{z}^2 \\
& =a^2+a^2 \cos 60^{\circ}-a b \cos 60^{\circ}-b^2 \\
& =\frac{3}{2} a^2-\frac{1}{2} a b-b^2 \\
& =(a-b)\left(\frac{3}{2} a+b\right) .
\end{aligned}
$$
又因为 $A_1 C \perp$ 平面 $C_1 B D$, 所以 $A_1 C \perp C_1 D$, 即 $\overrightarrow{C A_1} \cdot \overrightarrow{C_1 D}=0$, 且 $a>0$, $b>0$, 所以 $a=b$.
另一方面,
$$
\begin{aligned}
\overrightarrow{C A_1} \cdot \overrightarrow{B D} & =(\vec{x}+\vec{y}+\vec{z}) \cdot(\vec{x}-\vec{y}) \\
& =x^2-\vec{x} \cdot \vec{y}+\vec{y} \cdot \vec{x}-\vec{y}^2+\vec{z} \cdot \vec{x}-\vec{z} \cdot \vec{y} \\
& =a^2-a^2+a b \cos 60^{\circ}-a b \cos 60^{\circ}=0,
\end{aligned}
$$
所以当 $\frac{C D}{C C_1}=1$ 时, $A_1 C \perp C_1 D$ 且 $A_1 C \perp B D$ 即 $A_1 C \perp$ 平面 $C_1 B D$.
%%PROBLEM_END%%



%%PROBLEM_BEGIN%%
%%<PROBLEM>%%
例6. 如图(<FilePath:./figures/fig-c6i6.png>), 长方体 $A B C D-A_1 B_1 C_1 D_1$ 中, $A B= B C=4, A A_1=8, E$ 为 $C C_1$ 的中点, $O$ 为下底面正方形的中心.
求:
(1) 二面角 $C_1-A_1 B_1-O$ 的平面角 $\alpha$ 的大小;
(2) 异面直线 $A_1 B_1$ 和 $E O$ 所成角的大小;
(3) 三棱雉 $O-A_1 B_1 E$ 的体积.
%%<SOLUTION>%%
分析:与解 (1) 如图(<FilePath:./figures/fig-c6i7.png>) 建立图间直角坐标系, 则 $O(2,2,0), C_1(0,4,8), A_1(4,0,8), B_1(4,4,8)$,
$E(0,4,4), \overrightarrow{A_1 B_1}=\{0,4,0\}, \overrightarrow{A_1 O}=\{-2,2,-8\}$.
设 $\overrightarrow{n_0}=\{x, y, z\}$ 是平面 $A_1 B_1 O$ 的单位法向量, 则
$$
\left\{\begin{array}{l}
\overrightarrow{n_0} \cdot \overrightarrow{A_1 B_1}=0, \\
\overrightarrow{n_0} \cdot \overrightarrow{A_1 O}=0, \\
\left|\overrightarrow{n_0}\right|=1 .
\end{array}\right.
$$
所以
$$
\left\{\begin{array}{l}
4 y=0, \\
-2 x+2 y-8 z=0, \\
x^2+y^2+z^2=1
\end{array}\right.
$$
由此解得
$$
\left\{\begin{array} { l } 
{ x = - \frac { 4 } { \sqrt { 1 7 } } , } \\
{ y = 0 , } \\
{ z = - \frac { 1 } { \sqrt { 1 7 } } ; }
\end{array} \text { 或 } \left\{\begin{array}{l}
x=-\frac{4}{\sqrt{17}}, \\
y=0, \\
z=\frac{1}{\sqrt{17}} .
\end{array}\right.\right.
$$
取 $\overrightarrow{n_0}=\left\{\frac{4}{\sqrt{17}}, 0,-\frac{1}{\sqrt{17}}\right\}$.
平面 $A_1 B_1 C_1$ 有一个单位法向量为 $\overrightarrow{m_0}=\{0,0,-1\}$, 显然 $\cos \alpha>0$, 所以
$$
\cos \alpha=\overrightarrow{m_0} \cdot \overrightarrow{n_0}=\frac{1}{\sqrt{17}}
$$
所以二面角 $C_1-A_1 B_1-O$ 的平面角的大小为 $\arccos \frac{1}{\sqrt{17}}$.
(2) $\overrightarrow{A_1 B_1}=\{0,4,0\}, \overrightarrow{E O}=\{2,-2,-4\}, \cos \beta=\frac{\overrightarrow{A_1 B_1} \cdot \overrightarrow{E O}}{\left|\overrightarrow{A_1 B_1}\right| \cdot|\overrightarrow{E O}|}= \frac{-8}{4 \times 2 \sqrt{6}}=-\frac{\sqrt{6}}{6}$
所以异面直线 $A_1 B_1$ 和 $E O$ 所成的角为 $\arccos \frac{\sqrt{6}}{6}$.
(3) 因为 $\overrightarrow{A_1 B_1}=\{0,4,0\}, \overrightarrow{B_1 E}=\{-4,0,-4\}$, 所以 $\overrightarrow{A_1 B_1} \cdot \overrightarrow{B_1 E}= 0,\left|\overrightarrow{A_1 B_1}\right|=4,\left|\overrightarrow{B_1 E}\right|=4 \sqrt{2}$.
所以 $S_{\triangle A_1 B_1 E}=\frac{1}{2} \times 4 \times 4 \sqrt{2}=8 \sqrt{2}$.
设 $\overrightarrow{p_0}$ 是平面 $A_1 B_1 O$ 的单位法向量, 记为 $\overrightarrow{p_0}=\{x, y, z\}$, 则
$$
\left\{\begin{array}{l}
\overrightarrow{p_0} \cdot \overrightarrow{A_1 B_1}=0, \\
\overrightarrow{p_0} \cdot \overrightarrow{B_1 E}=0, \\
\left|\overrightarrow{p_0}\right|=1 .
\end{array}\right.
$$
所以
$$
\left\{\begin{array}{l}
4 y=0 \\
-4 x-4 z=0, \\
x^2+y^2+z^2=1
\end{array}\right.
$$
由此解得
$$
\left\{\begin{array} { l } 
{ x = \frac { \sqrt { 2 } } { 2 } , } \\
{ y = 0 , } \\
{ z = - \frac { \sqrt { 2 } } { 2 } ; }
\end{array} \text { 或 } \left\{\begin{array}{l}
x=-\frac{\sqrt{2}}{2}, \\
y=0, \\
z=\frac{\sqrt{2}}{2} .
\end{array}\right.\right.
$$
取 $\overrightarrow{p_0}=\left\{\frac{\sqrt{2}}{2}, 0,-\frac{\sqrt{2}}{2}\right\}, \overrightarrow{O A_1}=\{2,-2,8\}$, 则 $h=\left|\overrightarrow{O A_1} \cdot p_0\right|= |\sqrt{2}-4 \sqrt{2}|=3 \sqrt{2}$.
所以 $V_{O A_1 B_1 E}=\frac{1}{3} S_{\triangle A_1 B_1 E} \cdot h=\frac{1}{3} \times 8 \sqrt{2} \times 3 \sqrt{2}=16$.
%%PROBLEM_END%%


