
%%PROBLEM_BEGIN%%
%%<PROBLEM>%%
Tsintsifas 的不等式轨迹问题.
欧氏几何中下面的结论是熟知的.
如果 $A 、 B 、 C 、 D$ 是一个平面上的任意四点, 则 $A D \cdot B C 、 B D \cdot C A$ 和 $C D \cdot A B$ 是某一个三角形的三边长; 当 $A 、 B 、 C 、 D$ 四点共圆时, 这个三角形是退化的(在这种情况下,我们有著名的 Ptolemy 定理).
这个结论的证明并不难, 用复数就更为简单.
事实上,设 $A 、 B 、 C 、 D$ 对应的复数分别为 $z_1 、 z_2 、 z_3 、 z_4$, 用下面的恒等式
$$
\left(z_1-z_4\right)\left(z_2-z_3\right)+\left(\dot{z}_2-z_4\right)\left(z_3-z_1\right)+\left(z_3-z_4\right)\left(z_1-z_2\right)=0,
$$
便可立得结论.
Tsintsifas 在 1983 年的 Crux. Math. (9) 上提出问题:
问题设 $T$ 是一个给定的 $\triangle A B C, P$ 是 $T$ 所在平面上的一点,现记以 $a \cdot P A 、 b \cdot P B 、 c \cdot P C$ 为边的三角形为 $T_0$ (可能是退化的). 现设 $R 、 R_0$ 分别是 $T 、 T_0$ 的外接圆半径,求使得不等式
$$
P A \cdot P B \cdot P C \leqslant R R_0
$$
成立的点 $P$ 的轨迹.
%%<SOLUTION>%%
下面的解法 1 关键是应用关于一点 $P$ 的垂足三角形的面积公式.
解法 1 先证引理.
引理设 $P$ 为 $\triangle A_1 A_2 A_3$ 所在平面上一点, 则 $P$ 关于 $\triangle A_1 A_2 A_3$ 的垂足三角形的有向面积为
$$
F=\frac{R^2-\overline{O P^2}}{4 R^2} S_{\triangle A_1 A_2 A_3},
$$
其中 $O$ 为 $\triangle A_1 A_2 A_3$ 的外心.
证明如图(<FilePath:./figures/fig-c9i1.png>), 设垂足三角形为 $\triangle P_1 P_2 P_3$, 又设 $A_2 P$ 交 $\triangle A_1 A_2 A_3$ 的外接圆于 $B_2$, 则
$$
\begin{aligned}
\angle A_2 P A_3 & =\angle P_2 P_1 P_3+\angle A_2 A_1 A_3 \\
& =\angle A_2 B_2 A_3+\angle B_2 A_3 P,
\end{aligned}
$$
注意, 这里的 $\angle$ 表示有向角.
因此
$$
\angle P_2 P_1 P_3=\angle B_2 A_3 P .
$$
这时
$$
\begin{aligned}
F & =\frac{1}{2} \overline{P_1 P_2} \cdot \overline{P_1 P_3} \sin \angle P_2 P_1 P_3 \\
& =\frac{1}{2} \overline{P_1 P_2} \cdot \overline{P_1 P_3} \sin \angle B_2 A_3 P \\
& =\frac{1}{2} \overline{P A_3} \sin \alpha_3 \overline{P A_2} \sin \alpha_2 \sin \angle B_2 A_3 P,
\end{aligned}
$$
又
$$
\frac{\sin \angle B_2 A_3 P}{\sin \angle A_2 B_2 A_3}=\frac{\overline{P B_2}}{\overline{P A_3}},
$$
因此
$$
\begin{aligned}
F & =\frac{1}{2} \overline{P A_2} \cdot \overline{P B_2} \sin \angle A_2 B_2 A_3 \sin \alpha_2 \sin \alpha_3 \\
& =\frac{1}{2}\left(R^2-\overline{O P^2}\right) \sin \alpha_1 \sin \alpha_2 \sin \alpha_3 \\
& =\frac{R^2-\overline{O P}^2}{4 R^2} S_{\triangle A_1 A_2 A_3},
\end{aligned}
$$
引理得证.
下面回到原题.
设 $P A=a^{\prime}, P B=b^{\prime}, P C=c^{\prime}$, 则
$$
R=\frac{a b c}{4 S_T}, R_0=\frac{a b c a^{\prime} b^{\prime} c^{\prime}}{4 S_{T_0}},
$$
因此
$$
R R_0 \geqslant P A \cdot P B \cdot P C
$$
等价于
$$
\begin{aligned}
& \frac{(a b c)^2}{16 S_T S_{T_0}} \geqslant 1, \\
& S_{T_0} \leqslant \frac{(a b c)^2}{16 S_T} . 
\end{aligned} \label{eq1}
$$
如图(<FilePath:./figures/fig-c9i2.png>), 设 $P$ 在 $B C 、 C A 、 A B$ 上的射影分别为 $A^{\prime} 、 B^{\prime} 、 C^{\prime}$, 则
$$
B^{\prime} C^{\prime}=P A \sin A=\frac{a \cdot P A}{2 R},
$$
同理
$$
C^{\prime} A^{\prime}=\frac{b \cdot P B}{2 R}, A^{\prime} B^{\prime}=\frac{c \cdot P C}{2 R},
$$
故 $\triangle A^{\prime} B^{\prime} C^{\prime}$ 与 $T_0$ 相似且相似比为 $2 R$. 因此由式\ref{eq1}知
$$
S_{\triangle A^{\prime} B^{\prime} C^{\prime}}=\frac{1}{4 R^2} S_{T_0} \leqslant \frac{1}{4 R^2} \cdot \frac{a^2 b^2 c^2}{16 S_T}, \label{eq2}
$$
再应用引理得
$$
S_{\triangle A^{\prime} B^{\prime} C^{\prime}}=\frac{\left|R^2-\overline{O P^2}\right|}{4 R^2} S_T . \label{eq3}
$$
因此由从式\ref{eq2}、\ref{eq3}可得
$$
\frac{\left\lfloor R^2-\overline{O P}^2 \mid\right.}{4 R^2} S_T \leqslant \frac{1}{4 R^2} \cdot \frac{a^2 b^2 c^2}{16 S_T} .
$$
这等价于
$$
\left|R^2-\overline{O P}^2\right| \leqslant \frac{a^2 b^2 c^2}{16 S_T^2}=R^2
$$
故
$$
|\overline{O P}| \leqslant \sqrt{2} R .
$$
这说明 $P$ 的轨迹是一个以 $T$ 的外心 $O$ 为圆心, $\sqrt{2} R$ 为半径的圆及内部.
%%PROBLEM_END%%



%%PROBLEM_BEGIN%%
%%<PROBLEM>%%
Tsintsifas 的不等式轨迹问题.
欧氏几何中下面的结论是熟知的.
如果 $A 、 B 、 C 、 D$ 是一个平面上的任意四点, 则 $A D \cdot B C 、 B D \cdot C A$ 和 $C D \cdot A B$ 是某一个三角形的三边长; 当 $A 、 B 、 C 、 D$ 四点共圆时, 这个三角形是退化的(在这种情况下,我们有著名的 Ptolemy 定理).
这个结论的证明并不难, 用复数就更为简单.
事实上,设 $A 、 B 、 C 、 D$ 对应的复数分别为 $z_1 、 z_2 、 z_3 、 z_4$, 用下面的恒等式
$$
\left(z_1-z_4\right)\left(z_2-z_3\right)+\left(\dot{z}_2-z_4\right)\left(z_3-z_1\right)+\left(z_3-z_4\right)\left(z_1-z_2\right)=0,
$$
便可立得结论.
Tsintsifas 在 1983 年的 Crux. Math. (9) 上提出问题:
问题设 $T$ 是一个给定的 $\triangle A B C, P$ 是 $T$ 所在平面上的一点,现记以 $a \cdot P A 、 b \cdot P B 、 c \cdot P C$ 为边的三角形为 $T_0$ (可能是退化的). 现设 $R 、 R_0$ 分别是 $T 、 T_0$ 的外接圆半径,求使得不等式
$$
P A \cdot P B \cdot P C \leqslant R R_0
$$
成立的点 $P$ 的轨迹.
%%<SOLUTION>%%
解法 2 设 $\triangle A B C$ 的外心为 $O$, 现以 $P$ 为反演中心, $\lambda=P A \cdot P B \cdot P C$ 为反演幂做反演变换.
设 $A 、 B 、 C$ 反演后变为 $A^{\prime} 、 B^{\prime} 、 C^{\prime}$, 则
$$
B^{\prime} C^{\prime}=B C \cdot \frac{\lambda}{P B \cdot P C}=a \cdot P A .
$$
同理
$$
C^{\prime} A^{\prime}=b \cdot P B, A^{\prime} B^{\prime}=c \cdot P C .
$$
因此 $\triangle A^{\prime} B^{\prime} C^{\prime}$ 即为以 $a \cdot P A 、 b \cdot P B 、 c \cdot P C$ 为边的三角形.
而 $\odot(O, R)$ 反演后的半径为
$$
R_0=R \frac{\lambda}{|P O|^2-R^2},
$$
因此 $P A \cdot P B \cdot P C \leqslant R R_0$ 等价于
$$
\lambda \leqslant R^2 \cdot \frac{\lambda}{|P O|^2-R^2},
$$
这又等价于
$$
|P O| \leqslant \sqrt{2} R
$$
这说明 $P$ 的轨迹是以 $O$ 为圆心, 半径为 $\sqrt{2} R$ 的圆及内部.
%%<REMARK>%%
注:由上面的解法看出存在 $\triangle A B C$ 的反演像与点 $P$ 的垂足三角形相似.
%%PROBLEM_END%%


