
%%TEXT_BEGIN%%
递推方法.
通过建立递归关系解决问题的方法称之为递推方法.
递推方法是探索数学规律和解题思路的重要方法之一, 它对几乎所有的数学分支都有着重要作用.
随着计算机的广泛应用, 这种方法越来越受到重视.
递推关系是从很多计数问题中产生的, 它也是递推方法的数学描述.
利用递推关系计数的一般步骤是:
(1) 用 $a_n$ 表示与 $n$ 有关的欲计数的对象的个数;
(2) 计算一些初始值 $a_1, a_2, a_3, \cdots$ 等;
(3) 建立 $a_n$ 与 $a_{n-1}, a_{n-2}, \cdots, a_{n-k}$ 之间的递推关系;
(4) 求解递推关系.
%%TEXT_END%%



%%PROBLEM_BEGIN%%
%%<PROBLEM>%%
例1. 设 $n \in \mathbf{N}^*, n>1$. 求 $1,2, \cdots, n$ 的满足下列性质的排列 $\left(a_1\right.$, $\left.a_2, \cdots, a_n\right)$ 的个数: 仅存在一个 $i \in\{1,2, \cdots, n-1\}$, 使得 $a_i>a_{i+1}$.
%%<SOLUTION>%%
解: $p_n$ 表示具有题设性质的排列的个数, $n \geqslant 2$. 易知, $p_2=1$.
对于 $n \geqslant 3$, 若 $a_n=n$, 则这样的排列个数有 $p_{n-1}$ 个;
若 $a_i=n, 1 \leqslant i \leqslant n-1$, 考虑所有这样的排列, 可以从 $n-1$ 个数 1 , $2, \cdots, n-1$ 中选 $i-1$ 个数按从小到大的顺序排列成 $a_1, a_2, \cdots, a_{i-1}$, 其余的按从小到大的顺序排列在剩下的位置, 于是有 $\mathrm{C}_{n-1}^{i-1}$ 种排法, 所以
$$
p_n=p_{n-1}+\sum_{i=1}^{n-1} \mathrm{C}_{n-1}^{i-1}=p_{n-1}+2^{n-1}-1 .
$$
即
$$
\begin{gathered}
p_n-p_{n-1}=2^{n-1}-1, \\
p_{n-1}-p_{n-2}=2^{n-2}-1, \\
\cdots \cdots \\
p_2-p_1=2-1 .
\end{gathered}
$$
把上面这些式子相加, 得
$$
p_n=\left(2^{n-1}-1\right)+\left(2^{n-2}-1\right)+\cdots+(2-1)=2^n-n-1 .
$$
%%<REMARK>%%
注:在解决一些计数问题时, 往往题目并不给出明显表达式, 需通过观察、分析、归纳、猜想、论证等来确定递推关系.
本题中, 将 $p_n$ 分类计数,一类与 $n-1$ 个数的情形相联系, 另一类可直接计数, 从而得到了递推关系.
%%PROBLEM_END%%



%%PROBLEM_BEGIN%%
%%<PROBLEM>%%
例2. 对一个边长互不相等的凸 $n(n \geqslant 3)$ 边形的边染色, 每条边可以染红、黄、蓝三种颜色中的一种, 但是不允许相邻的边有相同的颜色.
问: 共有多少种不同的染色方法?
%%<SOLUTION>%%
解:设不同的染色法有 $p_n$ 种.
易知 $p_3=6$.
当 $n \geqslant 4$ 时,首先, 对于边 $a_1$, 有 3 种不同的染法, 由于边 $a_2$ 的颜色与边 $a_1$ 的颜色不同, 所以,对边 $a_2$ 有 2 种不同的染法, 类似地, 对边 $a_3, \cdots$, 边 $a_{n-1}$ 均有 2 种染法.
对于边 $a_n$, 用与边 $a_{n-1}$ 不同的 2 种颜色染色, 但是, 这样也包括了它与边 $a_1$ 颜色相同的情况, 而边 $a_1$ 与边 $a_n$ 颜色相同的不同染色方法数就是凸 $n-1$ 边形的不同染色方法数的种数 $p_{n-1}$, 于是可得图 $13-1$
$$
\begin{gathered}
p_n=3 \times 2^{n-1}-p_{n-1}, \\
p_n-2^n=-\left(p_{n-1}-2^{n-1}\right) .
\end{gathered}
$$
于是
$$
\begin{gathered}
p_n-2^n=(-1)^{n-3}\left(p_3-2^3\right)=(-1)^{n-2} \cdot 2, \\
p_n=2^n+(-1)^n \cdot 2 .
\end{gathered}
$$
综上所述, 不同的染色方法数为 $p_n=2^n+(-1)^n \cdot 2, n \geqslant 3$.
%%<REMARK>%%
注:本例与前一例略有不同, 是将计算 $p_n$ 时需扣除的量与 $p_{n-1}$ 建立联系.
%%PROBLEM_END%%



%%PROBLEM_BEGIN%%
%%<PROBLEM>%%
例3. 在 $2 \times n$ 的方格表的 $2 n$ 个方格中, 每格用黑、白两色之一染色.
若要求任何两个有公共边的方格不都染黑色,求不同的染色方案的数目.
%%<SOLUTION>%%
解:递推方法.
设满足条件的染色方案共有 $a_n$ 种, 其中在第 $n$ 列中均染白色的染色方案共 $b_n$ 种, 因而第 $n$ 列染不同色的染色方案数为 $a_n-b_n$.
当 $n \geqslant 2$ 时,方格表的前 $n-1$ 列一定是 $2 \times(n-1)$ 方格表的一种满足要求的染色方案.
当第 $n-1$ 列均染白色时, 第 $n$ 列 (从上到下,下同) 对应有"白白"、"黑白"、"白黑" 3 种染色方法; 当第 $n-1$ 列染不同色时, 不妨设为 "黑白", 则第 $n$ 列对应有"白白"、"白黑"2 种染色方法.
因此可建立递推关系:
$$
a_n=3 b_{n-1}+2\left(a_{n-1}-b_{n-1}\right)=2 a_{n-1}+b_{n-1} .
$$
另一方面, $2 \times(n-1)$ 方格表的每一种染色方案与 $2 \times n$ 方格表每一种第 $n$ 列染"白白"的染色方案一一对应,故 $b_n=a_{n-1}$.
因此 $n \geqslant 3$ 时,有 $a_n=2 a_{n-1}+a_{n-2}$, 又验证知 $a_1=3, a_2=7$, 故解得
$$
a_n=\frac{1}{2}\left((1+\sqrt{2})^{n+1}+(1-\sqrt{2})^{n+1}\right) .
$$
%%<REMARK>%%
注一:递推关系具有形式多样性.
比较多的问题中所利用的是单递推关系, 但也有一些是利用多元递推关系.
本题中 $\left\{a_n\right\}$ 的递推关系并不明显, 因而将染色方案分为"第 $n$ 列中均染白色"与 "第 $n$ 列染不同色"两种类型,引人辅助量 $b_n$ 参与递推关系的建立, 最后消去 $b_n$, 即得到 $\left\{a_n\right\}$ 的递推关系.
注二:与之等价的问题有:
设数列 $\left\{a_n\right\},\left\{b_n\right\}$ 每项均为 0 或 1 , 且满足
$$
a_k a_{k+1}=b_k b_{k+1}=a_k b_k=0\left(k \in \mathbf{N}^*\right) .
$$
对给定正整数 $n$, 求 $2 n$ 个数 $a_1, a_2, \cdots, a_n, b_1, b_2, \cdots, b_n$ 的不同的取值方法总数.
%%PROBLEM_END%%



%%PROBLEM_BEGIN%%
%%<PROBLEM>%%
例4. 在 $(1,2, \cdots, n)$ 的一个排列 $\left(a_1, a_2, \cdots, a_n\right)$ 中, 如果 $a_i \neq i(i= 1,2, \cdots, n$ ), 则称这种排列为一个错位排列 (也称更列). 求错位排列的个数 $D_n$.
%%<SOLUTION>%%
解:知 $D_1=0, D_2=1$.
对于 $n \geqslant 3$ 及 $(1,2, \cdots, n)$ 的任意一个错位排列 $\left(a_1, a_2, \cdots, a_n\right)$ 中, $a_1$ 可取除 1 以外的任一其他 $n-1$ 个数,设 $a_1=k(k \neq 1)$, 于是
(1) 如果 $a_k=1$, 这种错位排列数等于 $n-2$ 个元素的错位排列数 $D_{n-2}$;
(2) 如果 $a_k \neq 1$, 则这种错位排列就是元素 $1,2, \cdots, k-1, k+1, \cdots, n$ 在第 2 到第 $n$ 这 $n-1$ 个位置上的一个排列, 其中 1 不在第 $k$ 个位置, 其他元素都不在它自身所标记的位置上, 这种排列相当于 $2,3, \cdots, n$ 这 $n-1$ 个元素的一个错位排列, 所以共有 $D_{n-1}$ 个.
考虑到 $k=2,3, \cdots, n$ 共 $n-1$ 种这样的情况, 我们得到
$$
D_n=(n-1)\left(D_{n-1}+D_{n-2}\right)(n \geqslant 3) . \label{eq1}
$$
令 $E_n=\frac{D_n}{n !}$, 则 式\ref{eq1} 可变形为 $E_n=\left(1-\frac{1}{n}\right) E_{n-1}+\frac{1}{n} E_{n-2}$, 所以
$$
E_n-E_{n-1}=-\frac{1}{n} \cdot\left(E_{n-1}-E_{n-2}\right) . \label{eq2}
$$
反复利用 式\ref{eq2}, 并注意到 $E_1=0, E_2=\frac{1}{2}$, 对 $n \geqslant 2$, 有
$$
\begin{aligned}
E_n-E_{n-1} & =\left(-\frac{1}{n}\right) \cdot\left(-\frac{1}{n-1}\right) \cdot\left(E_{n-2}-E_{n-3}\right)=\cdots \\
& =\left(-\frac{1}{n}\right) \cdot\left(-\frac{1}{n-1}\right) \cdots\left(-\frac{1}{3}\right) \cdot\left(E_2-E_1\right)=\frac{(-1)^n}{n !} .
\end{aligned}
$$
由此可得
$$
E_n=\frac{(-1)^n}{n !}+\frac{(-1)^{n-1}}{(n-1) !}+\cdots+\frac{(-1)^2}{2 !},
$$
所以
$$
D_n=n ! \cdot\left(1-\frac{1}{1 !}+\frac{1}{2 !}-\cdots+(-1)^n \frac{1}{n !}\right) .
$$
%%<REMARK>%%
注:这是著名的"错位排列"计数问题的一种递推解法.
在建立递推关系时, 对应原理是必不可少的, 但有时候需要变通地找出对应关系, 例如在上述解法讨论 $a_1=k, a_k \neq 1$ 的情况时, 把元素 1 暂时视作 $k$, 即对应到 $2,3, \cdots, n$ 这 $n-1$ 个元素的一个错位排列.
%%PROBLEM_END%%



%%PROBLEM_BEGIN%%
%%<PROBLEM>%%
例5. 一种密码锁的密码设置是在正 $n$ 边形 $A_1 A_2 \cdots A_n$ 的每个顶点处赋值 0 和 1 两个数中的一个, 同时在每个顶点处染红、蓝两种颜色之一, 使得任意相邻的两个顶点的数字或颜色中至少有一个相同.
问: 该种密码锁共有多少种不同的密码设置方案?
%%<SOLUTION>%%
解:设满足条件的密码设置方案共 $a_n$ 种, 其中, 顶点 $A_1, A_n$ 赋值与染色全同的方案有 $b_n$ 种, 其全体构成集合 $B_n ; A_1, A_n$ 赋值与染色恰有一项相同的方案有 $c_n$ 种, 其全体构成集合 $C_n$, 则
$$
a_n=b_n+c_n . \label{eq1}
$$
此外, 若不考虑相邻顶点 $A_1, A_n$ 间的赋值与染色是否兼容, 首先设置 $A_1$ 位置, 有 4 种方式, 再依次设置 $A_2, \cdots, A_n$, 各有 3 种方式, 根据乘法原理, 共 $4 \times 3^{n-1}$ 种方式, 因此, 若将 $A_1, A_n$ 赋值与染色完全不同, 但其余任意两个相邻顶点赋值与染色至少有一项相同的设置方式数记为 $d_n$, 则
$$
a_n+d_n=4 \times 3^{n-1} . \label{eq2}
$$
对每种密码设置方案, 考虑顶点 $A_1, A_{n-1}$ 的赋值与染色情况.
(1)若赋值与染色情况全同, 这样的情况数为 $b_{n-1}$, 每种情况恰可对应 $B_n$ 中的一个元素及 $C_n$ 中的两个元素 (例如, 若 $A_1, A_{n-1}$ 均为 "0 红", 则 $A_n$ 可为" 0 红"、" 0 蓝"或" 1 红", 前者对应 $B_n$ 中的方案, 后两者对应 $C_n$ 中的方案).
(2)若赋值与染色恰有一项相同, 这样的情况数为 $c_{n-1}$, 每种情况恰可对应 $B_n$ 中的一个元素及 $C_n$ 中的一个元素 (例如, 若 $A_1$ 为 "0 红", $A_{n-1}$ 为 "1 红", 则 $A_n$ 可为" 0 红"、" 1 红", 前者对应 $B_n$ 中的方案, 后者对应 $C_n$ 中的方案).
(3)若赋值与染色均不同, 这样的情况数为 $d_{n-1}$, 每种情况恰可对应 $C_n$
中两个元素 (例如, 若 $A_1$ 为" 0 红", $A_{n-1}$ 为" 1 蓝", 则 $A_n$ 可为" 0 蓝"、" 1 红").
由此,可建立递推关系
$$
\begin{gathered}
b_n=b_{n-1}+c_{n-1} ; \label{eq3} \\
c_n=2 b_{n-1}+c_{n-1}+2 d_{n-1} . \label{eq4}
\end{gathered}
$$
由 式\ref{eq1} 和 \ref{eq3} 知 $b_n=a_{n-1}, c_n=a_n-a_{n-1}$, 结合 式\ref{eq2} 与 \ref{eq4} 可整理得 $a_n=a_{n-2}+ 8 \times 3^{n-2}$, 又枚举得 $a_2=12, a_3=28$, 所以
$$
a_n=\left\{\begin{array}{l}
3^n+1, n=3,5,7, \cdots \\
3^n+3, n=4,6,8, \cdots .
\end{array}\right.
$$
%%<REMARK>%%
注:本题为 2010 年全国高中数学联赛加试最后一题, 标准解法是直接计数, 其中涉及组合式的化简.
上述解法为递推方法, 虽不算很简洁, 但体现了递推的思维特点: 为了计算 $a_n$, 可根据解题的实际需要引人一系列辅助量 $b_n$, $c_n, d_n$, 通过计数原理清楚地列出它们之间的等量关系 (如果是 $k$ 个辅助量, 等量关系通常应列出 $k+1$ 个), 最后消去辅助量便可得到 $\left\{a_n\right\}$ 的递推关系.
下面的一些例子说明递推思想的运用不纯然得借助于递推式,而是可以有灵活多样的形式.
%%PROBLEM_END%%



%%PROBLEM_BEGIN%%
%%<PROBLEM>%%
例6. 设非负实数数列 $\left\{a_n\right\},\left\{b_n\right\}$ 满足: 对任意 $i, j \in \mathbf{N}^*,|i-j| \in \{2011,2012\}$, 都有 $a_i+b_j \leqslant a_i b_j$. 求证: $\left\{a_n\right\}$ 与 $\left\{b_n\right\}$ 中的所有项不是全大于 1 , 就是全等于 0 .
%%<SOLUTION>%%
证明:果对任意 $n \in \mathbf{N}^*$, 有 $a_n, b_n>1$, 则命题已成立.
以下考虑 $\left\{a_n\right\}$, $\left\{b_n\right\}$ 中存在某项 (不妨设为 $a_k$ ) 小于 1 的情况.
因不等式 $a_i+b_j \leqslant a_i b_j$ 等价于 $\left(1-a_i\right)\left(1-b_j\right) \geqslant 1$, 若 $0 \leqslant a_i<1$, 则
$$
1 \geqslant 1-b_j \geqslant \frac{1}{1-a_i} \geqslant 1 \text {, }
$$
可见不等号均为等号, 所以 $a_i=b_j=0$.
由于 $0 \leqslant a_k<1$, 故令 $(i, j)=(k, k+2011),(k, k+2012)$, 得
$$
a_k=b_{k+2011}=b_{k+2012}=0 .
$$
再令 $(i, j)=(k-1, k+2011),(k+1, k+2012)$ (若已有 $k=1$ 则省略前者), 得
$$
a_{k-1}=a_{k+1}=0 \text {. }
$$
以此类推得
$$
a_{k-2}=a_{k-3}=\cdots=a_1=0, a_{k+2}=a_{k+3}=\cdots=0 \text {, }
$$
即 $\left\{a_n\right\}$ 中每项都是 0 . 又 $\left\{b_n\right\}$ 中已有一项是 0 , 同样递推得 $\left\{b_n\right\}$ 中每项都是 0 . 命题成立.
%%<REMARK>%%
注:本题的大致求解思路是: 若存在某项小于 1 , 先确定它等于 0 , 再充分利用条件证明它的前后相邻项都等于 0 , 再往正向和逆向递推, 就说明了所有项都是 0 . 对另一个数列同理可验证所有项为 0 .
在同样的求解思路下,本题可将条件中的"2011,2012"推广至更一般的情形: 当正整数 $p, q$ 互素,且不均为奇数时,命题仍能成立.
%%PROBLEM_END%%



%%PROBLEM_BEGIN%%
%%<PROBLEM>%%
例7. 设 $n$ 为大于 1 的奇数, $\alpha$ 是 $P(x)=(x-1)^n-x^2$ 的零点, 证明 $\alpha> 2+\frac{1}{n}$.
%%<SOLUTION>%%
证明: $\alpha<1$, 由于 $n$ 为奇数,则 $P(\alpha)=(\alpha-1)^n-\alpha^2<0$,矛盾.
所以 $\alpha \geqslant 1$.
若 $1 \leqslant \alpha<2$, 则 $(\alpha-1)^n<1 \leqslant \alpha^2$, 故 $P(\alpha)<0$, 矛盾.
所以 $\alpha \geqslant 2$.
若 $2 \leqslant \alpha \leqslant 2+\frac{1}{n}$, 则 $(\alpha-1)^n \leqslant\left(1+\frac{1}{n}\right)^n<\mathrm{e}<4 \leqslant \alpha^2$, 故 $P(\alpha)<0$, 矛盾.
所以 $\alpha>2+\frac{1}{n}$.
%%<REMARK>%%
注:本题是个关于多项式零点分布的问题, 但若简单用导数研究 $P(x)$ 的单调性并不足以解决问题, 因此我们通过三个步骤来递推讨论零点的范围: 先选取第一个范围 $\alpha<1$, 此时非常容易证明 $\alpha$ 不是零点, 这样就将所需考虑的范围缩小为 $\alpha \geqslant 1$, 故而为第二步讨论创设了有利条件 $\alpha^2 \geqslant 1$; 对第二个范围 $1 \leqslant \alpha<2$ 的选取也是根据同样的策略,这样为第三步创设了条件 $\alpha^2 \geqslant 4$; 到第三步时再发起总攻, 把问题完全解决.
这种步步为营、层层推进的做法有利于降低推理难度,有时候反而能更快且干净地解决问题.
在解决某些数学问题时,我们还可采取 "递归"的思想方法, 粗略地讲, 其中包括"回溯"和"递推"两个过程.
例如对某种规模为 $n$ 的问题, 将其降解成若干个规模小于 $n$ 的问题,依次降解直到问题规模可求; 再求出低阶规模的解, 逐次代入高阶问题中, 直至求出规模为 $n$ 的问题的解.
%%PROBLEM_END%%



%%PROBLEM_BEGIN%%
%%<PROBLEM>%%
例8. 一开始桌上放着 3 堆火柴, 其中一堆的根数是另两堆之和.
两人依次轮流做取火柴游戏: 游戏者每次任意取走其中一堆,并把余下两堆中的任意一堆分成非空的两堆.
谁无法这样做, 就算输了.
证明或否定: 先取火柴的一方有必胜策略,并说明理由.
%%<SOLUTION>%%
解:若 3 堆火柴根数分别为 $2^{n_1} a_1, 2^{n_2} a_2, 2^{n_3} a_3$ (其中 $n_1, n_2, n_3 \in \mathbf{N}$, $a_1, a_2, a_3$ 为奇数), 那么当 $n_1, n_2, n_3$ 不全相等时, 定义这个状态为 $W$; 当 $n_1=n_2=n_3$ 时, 定义这个状态为 $L$.
首先说明: 在状态 $W$ 下总可以进行游戏的下一步, 并且能通过适当的策略得到某个状态 $L$. 事实上, 不妨设 $n_1 \leqslant n_2 \leqslant n_3$, 根据状态 $W$ 的定义, 有 $n_1< n_3$, 注意此时第三堆火柴多于一根, 必能分堆, 并且可这样操作: 取走第二堆火柴, 再将第三堆分成根数为 $2^{n_1}, 2^{n_1}\left(2^{n_3-n_1} a_3-1\right)$ 的两堆, 那么三堆火柴根数可以写成 $2^{n_1} a_1, 2^{n_1}, 2^{n_1}\left(2^{n_3-n_1} a_3-1\right)$, 其中 $a_1, 1,2^{n_3-n_1} a_3-1$ 均为奇数, 这是一个状态 $L$.
其次说明, 状态 $L$ 下若能进行操作, 则操作后只能得到状态 $W$. 事实上, 在状态 $L$ 下三堆火柴根数可写成 $2^n a_1, 2^n a_2, 2^n a_3\left(n \in \mathbf{N}, a_1, a_2, a_3\right.$ 为奇数) 的形式.
假设操作后仍得到一个状态 $L$, 不妨设操作中第三堆火柴未动, 那么新的三堆火柴根数必是 $2^n b_1, 2^n b_2, 2^n a_3$ ( $b_1, b_2, a_3$ 为奇数) 的形式, 其中 $b_1+b_2$ 等于 $a_1$ 或 $a_2$, 但这与 $a_1, a_2, b_1, b_2$ 为奇数相矛盾.
故状态 $L$ 操作后必变为状态 $W$.
最后说明游戏的初始状态为 $W$. 事实上, 假设此时 $2^{n_1} a_1+2^{n_2} a_2=2^{n_3} a_3$ 且为状态 $L$, 即 $n_1=n_2=n_3=n$, 则 $a_1+a_2=a_3$, 与 $a_1, a_2, a_3$ 为奇数矛盾.
考虑到每步操作都将使火柴数减少, 故游戏必在有限步内结束, 那么综上可知, 本题的结论是肯定的, 即先取火柴的一方有必胜策略.
%%<REMARK>%%
注:数学竞赛中常常出现这样的双人博弯问题: 博弯是有限的、零和的, 对局双方依照规则轮流进行操作, 直至判定胜负.
当双方均采取最佳方案时, 须确定必胜的一方 (有时候是要确定不败的一方). 此类博恋问题的一般解决步骤为:
(1) 手推小数据,对"胜状态" (即可以采取适当操作使最终获胜的局面)和 "负状态" (即无论进行何种操作, 对方总能适当操作取胜的局面)作出合理猜测;
(2)验证从每个胜状态确实可以适当操作得到一个负状态;
(3)验证从每个负状态无论进行何种操作均只能得到胜状态.
这样就能确定每种初始状态下的胜负结果.
本题改编自 1994 年俄罗斯数学奥林匹克试题.
还可以进一步考虑这样的问题: 如果初始状态是 4 堆火柴, 其中一堆的根数是另外某两堆之和, 那么按照司样规则进行游戏, 先取火柴的一方是否仍有必胜策略.
%%PROBLEM_END%%


