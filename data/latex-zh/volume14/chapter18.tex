
%%TEXT_BEGIN%%
构造法.
解题过程中, 由于某种需要, 要么把题设条件中的关系构造出来, 要么将这些关系设想在某个模型上得到实现,要么把题设条件经过适当地逻辑组合而构造出一种新的形式, 从而使数学问题获得解决.
在这个过程中, 思维的创造活动的特点是 "构造", 我们不妨称之为构造性思维, 运用构造性思维解题的方法, 称为构造法.
构造法解题需要我们有比较全面的知识以及敏锐的直觉, 能多角度多渠道地进行联想,将代数、三角、几何、数论等知识相互渗透有机结合.
%%TEXT_END%%



%%PROBLEM_BEGIN%%
%%<PROBLEM>%%
例1. 证明:不定方程 $x^2+y^2=z^6$ 有无穷多组正整数解.
%%<SOLUTION>%%
证明:有一组正整数组 $(a, b, c)$, 使 $a^2+b^2=c^2$ (称为勾股数组), 则两边同时乘以 $c^4$, 可得 $\left(a c^2\right)^2+\left(b c^2\right)^2=c^6$, 即 $x=a c^2, y=b c^2, z=c$ 为原方程的一组正整数解.
熟知勾股数组 $(a, b, c)$ 有无穷多组, 且 $c$ 的值各不相同,所以方程 $x^2+y^2=z^6$ 的无穷多组正整数解.
%%<REMARK>%%
注:考虑到原方程与不定方程 $x^2+y^2=z^2$ 的结构相似性, 很自然会考虑有没有一组勾股数 $(a, b, c)$, 使 $a^2+b^2=c^2$ 且 $c$ 为完全立方数 (如 8,27 等). 虽然这样最终可以找到解, 不过数字大毕竟稍微麻烦了点, 而且距离得到无穷多解尚缺实质性进展.
干脆我们换条思路: 对代数式 $a^2+b^2=c^2$ 进行简单的恒等变形就能造出需要的解了.
甚至我们完全不需要用到勾股数组的无穷性:在等式 $a^2+b^2=c^2$ 的两边乘以 $c^{6 k+4}(k \in \mathbf{N})$ 就有 $\left(a c^{3 k+2}\right)^2+\left(b c^{3 k+2}\right)^2=\left(c^{k+1}\right)^6$ 了, 这里 $(a, b, c)$ 可以只取某一组勾股数组, 比如 $(3,4,5)$, 当 $k$ 取遍一切自然数时得到无穷组不同的解.
对证明不定方程存在无穷组解的问题, 往往需要我们构造一个与问题密切相关的恒等式,一举达到写出无穷多组解的目的.
但此类问题难度差异比较大, 例如, 要证明不定方程 $x^3+y^3+1=z^3$ 有无穷多组正整数解, 所构造的恒等式可以是 $\left(9 k^3-1\right)^3+\left(9 k^4-3 k\right)^3+1=\left(9 k^4\right)^3$, 但就不太容易想得到了.
总之解这类问题需要对各种代数式结构的一些理解和直觉.
虽然解题思路会因人而异, 构造过程也有曲折反复, 但始终应围绕原方程的结构特征进行构造, 一计不成再生一计, 只要头脑中不断地有好的想法, 就容易成功.
%%PROBLEM_END%%



%%PROBLEM_BEGIN%%
%%<PROBLEM>%%
例2. 实数 $a 、 b 、 c$ 满足 $(a+c)(a+b+c)<0$, 证明:
$$
(b-c)^2>4 a(a+b+c) \text {. }
$$
%%<SOLUTION>%%
分析:证明 $(b-c)^2>4 a(a+b+c)$, 即证明 $(b-c)^2-4 a(a+b+-c)>$ 0 , 联想到一元二次方程根的判别式, 进而构造符合条件的二次函数, 通过对函数图象和性质的研究, 使得问题得以解决.
解若 $a=0$, 则 $c(b+c)<0$, 从而 $b \neq c$ (否则, $\left.2 b^2<0\right)$, 于是 $(b-c)^2>$ 0 , 命题成立.
若 $a \neq 0$, 设二次函数 $y=a x^2+(b-c) x+(a+b+c)$.
令 $x_1=0$, 得函数值 $y_1=a+b+c$, 令 $x_2=-1$, 得函数值 $y_2=2(a+c)$.
因为 $(a+c)(a+b+c)<0$, 所以 $y_1 y_2<0$, 这说明二次函数 $y=a x^2+ (b-c) x+(a+b+c)$ 上两点 $\left(x_1, y_1\right)$ 和 $\left(x_2, y_2\right)$ 分别在 $x$ 轴的两侧, 由此可见抛物线与 $x$ 轴有两个不同的交点, 即方程 $a x^2+(b-c) x+(a+b+c)=0$ 有两个不相等的实数根.
因此 $\Delta=(b-c)^2-4 a(a+b+c)>0$, 即
$$
(b-c)^2>4 a(a+b+c) .
$$
%%PROBLEM_END%%



%%PROBLEM_BEGIN%%
%%<PROBLEM>%%
例3. 设 $a, b, c$ 是绝对值小于 1 的实数, 证明: $a b+b c+c a+1>0$.
%%<SOLUTION>%%
证明:造函数 $f(x)=(b+c) x+b c+1$, 它的图象是一条直线, 若能证明函数值 $f(-1), f(1)$ 都大于 0 , 则以点 $(-1, f(-1))$ 和 $(1, f(1))$ 为端点的线段上的每一点的函数值都大于 0 , 即对满足 $-1<x<1$ 的每一个 $x$, 都有 $f(x)>0$, 从而命题得证.
因为
$$
\begin{gathered}
f(-1)=-(b+c)+b c+1=(b-1)(c-1)>0, \\
f(1)=(b+c)+b c+1=(b+1)(c+1)>0,
\end{gathered}
$$
所以
$$
f(a)=a(b+c)+b c+1>0 .
$$
%%PROBLEM_END%%



%%PROBLEM_BEGIN%%
%%<PROBLEM>%%
例4. 设 $a \geqslant c, b \geqslant c, c>0$, 证明不等式
$$
\sqrt{c(a-c)}+\sqrt{c(b-c)} \leqslant \sqrt{a b} .
$$
%%<SOLUTION>%%
证明: $a=c$ 或 $b=c$ 时, 不等式显然成立.
当 $a \neq c$ 且 $b \neq c$ 时, 讨论如下: 两个正数 $x 、 y$ 的乘积 $x y$ 可以看成边长为 $x$ 和 $y$ 的矩形的面积, 也可以看成直角边为 $x$ 和 $y$ 的直角三角形面积的两倍, 于是构造图形,如图(<FilePath:./figures/fig-c18i1.png>).
$$
\begin{gathered}
A B=D E=\sqrt{c}, B C=\sqrt{a-c}, \\
C D=\sqrt{b-c}, \\
\angle A B C=\angle C D E=\frac{\pi}{2} .
\end{gathered}
$$
设 $\angle A C E=\alpha$, 则
$$
\begin{aligned}
S_{A B D E} & =\sqrt{c}(\sqrt{a-c}+\sqrt{b-c})=S_{\triangle A B C}+S_{\triangle C D E}+S_{\triangle A C E} \\
& =\frac{1}{2} \sqrt{c} \sqrt{a-c}+\frac{1}{2} \sqrt{c} \sqrt{b-c}+\frac{1}{2} \sqrt{a} \sqrt{b} \sin \alpha \\
& \leqslant \frac{1}{2} \sqrt{c}(\sqrt{a-c}+\sqrt{b-c})+\frac{1}{2} \sqrt{a b},
\end{aligned}
$$
所以
$$
\sqrt{c(a-c)}+\sqrt{c(b-c)} \leqslant \sqrt{a b},
$$
于是命题得证.
%%PROBLEM_END%%



%%PROBLEM_BEGIN%%
%%<PROBLEM>%%
例5. 设实数 $x, y, z$ 满足 $0<x<y<z<\frac{\pi}{2}$, 证明:
$$
\frac{\pi}{2}+2 \sin x \cos y+2 \sin y \cos z>\sin 2 x+\sin 2 y+\sin 2 z .
$$
%%<SOLUTION>%%
证明:直角坐标平面上以原点为圆心作单位 圆.
考虑第一象限, 在单位圆上取点 $A_1, A_2, A_3$, 使得 $\angle A_1 O x=x, \angle A_2 O x=y, \angle A_3 O x=z($ 如图(<FilePath:./figures/fig-c18i2.png>) 所示).
由于三个矩形面积之和 $S_1+S_2+S_3$ 小于 $\frac{1}{4}$ 单位圆的面积, 此即
$$
\sin x(\cos x-\cos y)+\sin y(\cos y-\cos z)+\sin z \cos z<\frac{\pi}{4},
$$
整理后便得
$$
\frac{\pi}{2}+2 \sin x \cos y+2 \sin y \cos z>\sin 2 x+\sin 2 y+\sin 2 z .
$$
%%<REMARK>%%
注:例 4 和例 5 条件中的数量关系能以某种方式与几何图形建立联系或具有明显的几何意义, 从而构造图形, 将题设条件及数量关系直接在图形中得到实现,然后在所构造的图形中寻求所证的结论.
%%PROBLEM_END%%



%%PROBLEM_BEGIN%%
%%<PROBLEM>%%
例6. 对任意给定正整数 $n$, 试找出 $2 n+1$ 个正整数 $a_i(1 \leqslant i \leqslant 2 n+1)$, 使它们成等差数列, 且它们的积为完全平方数.
%%<SOLUTION>%%
解: $a_i=i k(1 \leqslant i \leqslant 2 n+1)$, 则 $a_1 a_2 \cdots a_{2 n+1}=(2 n+1) ! k^{2 n+1}$, 取 $k= (2 n+1)$ ! 即可.
%%<REMARK>%%
注:构造法解题的过程往往蕴含在思维中, 书写却可能很简短, 不一定能反映出考虑问题的过程.
这个问题可以怎样考虑呢? 例如, 对任意数列 $\left\{a_n\right\}$, 若 $a_1 a_2 \cdots a_{2 n+1}=N$, 则我们不难说明 $\left(N a_1\right)\left(N a_2\right) \cdots\left(N a_{2 n+1}\right)=N^{2 n+2}=\left(N^{n+1}\right)^2$ 是完全平方数, 而每项乘以一个常数并不改变等差数列的属性, 这步调整完全可以放到最后一个环节来完成.
这就拟定了解题框架, 余下只要写一个明确无误的过程就行了 (读者不妨进一步构造一个正整数等差数列 $\left\{a_n\right\}$, 使 $a_1 a_2 \cdots a_{2 n+1}$ 等于某个 $N^{2^m}\left(N \in \mathbf{N}^*\right)$ 的形式, 其中 $m$ 为给定正整数).
先构造出满足部分性质的一个对象再予以调整, 这是一种典型的构造思想, 在面对一些大型的构造性问题时, 这样做往往可以将大问题分解成几个小问题逐步完成, 降低构造的难度.
在本节最后, 我们特意设置了取材于 2009 年中国数学奥林匹克第 6 题的一个习题 (见本节习题 11), 其中每个小题正是原问题构造性解答中的一个环节.
%%PROBLEM_END%%



%%PROBLEM_BEGIN%%
%%<PROBLEM>%%
例7. 求所有的正整数 $n$, 使得存在非零整数 $x_1, x_2, \cdots, x_n, y$, 满足
$$
\left\{\begin{array}{l}
x_1+\cdots+x_n=0, \\
x_1^2+\cdots+x_n^2=n y^2 .
\end{array}\right.
$$
%%<SOLUTION>%%
解:然 $n \neq 1$.
当 $n=2 k\left(k \in \mathbf{N}^*\right)$ 时, 令 $x_{2 i-1}=1, x_{2 i}=-1, i=1,2, \cdots, k, y=1$, 则满足条件.
当 $n=3+2 k\left(k \in \mathbf{N}^*\right)$ 时, 令 $y=2, x_1=4, x_2=x_3:=x_4=x_5=-1$,
$$
x_{2 i}=2, x_{2 i+1}=-2, i=3,4, \cdots, k+1,
$$
则满足条件.
当 $n=3$ 时,若存在非零整数 $x_1, x_2, x_3$, 使得
$$
\left\{\begin{array}{l}
x_1+x_2+x_3=0, \\
x_1^2+x_2^2+x_3^2=3 y^2,
\end{array}\right.
$$
则消去 $x_3$ 得
$$
2\left(x_1^2+x_2^2+x_1 x_2\right)=3 y^2,
$$
不妨设 $\left(x_1, x_2\right)=1$, 则 $x_1, x_2$ 都是奇数或者一奇一偶, 从而, $x_1^2+x_2^2+x_1 x_2$ 是奇数, 另一方面, $2 \mid y$, 故 $3 y^2 \equiv 0(\bmod 4)$, 而 $2\left(x_1^2+x_2^2+x_1 x_2\right) \equiv 2(\bmod 4)$, 矛盾.
综上所述,满足条件的正整数 $n$ 是除了 1 和 3 外的一切正整数.
%%<REMARK>%%
注:本题中 $n=1$ 只需平凡考虑, 对 $n=3$ 的排除也不算困难.
关键在于如何证明其余的正整数 $n$ 满足题意.
此时需要设法将解逐一构造出来.
注意到只要 $\left|x_1\right|=\left|x_2\right|=\cdots=\left|x_n\right|=y$, 则 $x_1^2+\cdots+x_n^2=n y^2$ 成立, 其中若 $n$ 为偶数, 那么只要各个 $x_i$ 中一半取正数,一半取负数, 就构造成功了.
而在 $n \geqslant 5$ 且为奇数的情形下, 我们进行小范围的试探,仍能给出一般的构造方法.
本题中所运用的这种分类构造策略,既简洁完整地做到了对所有情形的验证, 又避免了统一构造欠缺灵活的一面.
%%PROBLEM_END%%



%%PROBLEM_BEGIN%%
%%<PROBLEM>%%
例8. 正整数 $a_1, a_2, \cdots, a_{2006}$ (可以有相同的)使得 $\frac{a_1}{a_2}, \frac{a_2}{a_3}, \cdots, \frac{a_{2005}}{a_{2006}}$ 两不相等.
问: $a_1, a_2, \cdots, a_{2006}$ 中最少有多少个不同的数? 
%%<SOLUTION>%%
解: $a_1, a_2, \cdots, a_{2006}$ 中出现 $n$ 个互不相同的数, 那么这些正整数两两间所产生的不同比值不多于 $n(n-1)+1$ 个(其中, 相等的两数产生比值 1), 为保证条件满足,必有 $n(n-1)+1 \geqslant 2005$, 可知 $n>45$.
下面构造一个例子, 说明 $n=46$ 可以取到.
设 $p_1, p_2, \cdots, p_{46}$ 为 46 个互不相同的素数,构造 $a_1, a_2, \cdots, a_{2006}$ 如下:
$$
\begin{aligned}
& p_1, p_1, \\
& p_2, p_1, \\
& p_3, p_2, p_3, p_1, \\
& p_4, p_3, p_4, p_2, p_4, p_1, \\
& \cdots \ldots \\
& p_k, p_{k-1}, p_k, p_{k-2}, p_k, \cdots, p_k, p_2, p_k, p_1, \\
& \cdots \cdots \\
& p_{45}, p_{44}, p_{45}, p_{43}, p_{45}, \cdots, p_{45}, p_2, p_{45}, p_1, \\
& p_{46}, p_{45}, p_{46}, p_{44}, p_{46}, \cdots, p_{46}, p_{34},
\end{aligned}
$$
这 2006 个正整数满足要求.
所以 $a_1, a_2, \cdots, a_{2006}$ 中最少有 46 个互不相同的数.
%%<REMARK>%%
注:本题是一个典型的离散最值问题, 解答分为两个部分, 一是估计出 $n$ 的范围 $n>45$, 二是通过构造具体例子表明范围中最极端的值 $n=46$ 可以实现.
数学竞赛中的最值问题覆盖代数、几何、组合等方方面面, 其中有些变量的上、下界并不能用经典的求最值方法求得, 此时, 构造一个例子或一个反例本身就解决了题目的一半, 这样的情况在组合最值问题及离散最值问题中是极为多见的.
甚至有时候一个本质的例子对解决另一半问题还具有启发性.
构造法的价值由此可见一斑.
%%PROBLEM_END%%



%%PROBLEM_BEGIN%%
%%<PROBLEM>%%
例9. 求证: 对每个正整数 $m$, 平面内存在一个有限非空点集 $S$, 具有如下性质: 对于任意一点 $A \in S$, 在 $S$ 中与点 $A$ 距离为 1 的点恰有 $m$ 个.
%%<SOLUTION>%%
证明: $m$ 用数学归纳法.
当 $m=1$ 时,取长为 1 的线段的两个端点构成点集 $S$ 即可.
假设 $m=k$ 时命题成立, 即存在点集 $S_k$, 对任意 $A \in S_k$, 恰有 $S_k$ 中 $k$ 个点到点 $A$ 距离为 1 .
以 $S_k$ 中的每个点为圆心作半径为 1 的圆, 这些圆两两之间的交点是有限个, 设它们构成集合 $T_k$, 那么 $S_k \cup T_k$ 中任意两点的连线的方向只有有限个.
任取一个方向 $\vec{d}$ 不属于这有限个方向, 将 $S_k$ 沿 $\vec{d}$ 平移一个单位得到点集 $S_k^{\prime}$.
由 $\vec{d}$ 的取法不难验证: 一方面 $S_k \cap S_k^{\prime}=\varnothing$; 另一方面, 两点 $A \in S_k$ 和 $A^{\prime} \in S_k^{\prime}$ 之间距离为 1 当且仅当 $A^{\prime}$ 是由 $A$ 平移所得.
当 $m=k+1$ 时, 令 $S_{k+1}=S_k \cup S_k^{\prime}$. 对任意 $A \in S_{k+1}$, 不失一般性, 设 $A \in S_k$, 根据归纳假设, 恰有 $S_k$ 中的 $k$ 个点与它距离为 1 , 又 $S_k^{\prime}$ 中恰有一点与它距离为 1 , 故 $S_{k+1}$ 中恰好有 $k+1$ 个点与 $A$ 的距离为 1 . 因此 $m=k+1$ 时命题成立.
由数学归纳法知, 对任意正整数 $m$, 平面内存在满足题意的点集.
%%<REMARK>%%
注:本题用的是归纳构造方法.
由于 $m$ 不是具体数值, 而且点集不易直接构造, 那么先把最简单的 $m=1$ 的情形构造出来, 再通过适当平移点集后取并集(有点像把自己和自己的影子合起来)的手法实现归纳过渡.
一般地, 对一个与正整数 $n$ 有关的命题, 可以从 $n$ 等于某个 $n_0$ 开始, 先构造出满足题意的对象 $P\left(n_0\right)$, 然后假设 $P(n)$ 已经构造出来, 证明 $P(n+1)$ 也能被构造出来, 当然在具体问题中, 这个证明可以是构造性的, 也可以是非构造性的.
%%PROBLEM_END%%



%%PROBLEM_BEGIN%%
%%<PROBLEM>%%
例10. 对任何正整数 $n$, 证明恒等式: $\sum_{k=0}^n 2^k \mathrm{C}_n^k \mathrm{C}_{n-k}^{\left[\frac{n-k}{2}\right]}=\mathrm{C}_{2 n+1}^n$. 
%%<SOLUTION>%%
证法一(构造函数方法)
考虑函数 $f(x)=(1+x)^{2 n+1}$. 一方面,需证等式右边显然是 $f(x)$ 的 $n$ 次项系数.
另一方面,
$$
\begin{aligned}
f(x) & =(1+x)\left(1+2 x+x^2\right)^n=(1+x) \sum_{k=0}^n \mathrm{C}_n^k\left(1+x^2\right)^{n-k}(2 x)^k \\
& =\sum_{k=0}^n 2^k \mathrm{C}_n^k(1+x)\left(1+x^2\right)^{n-k} x^k,
\end{aligned}
$$
当 $n-k$ 为偶数时, $(1+x)\left(1+x^2\right)^{n-k}$ 中 $x^{n-k}$ 的系数为 $\mathrm{C}_{n-k}^{\frac{n-k}{2}}$;
当 $n-k$ 为奇数时, $(1+x)\left(1+x^2\right)^{n-k}$ 中 $x^{n-k}$ 的系数为 $\mathrm{C}_{n-k}^{\frac{n-k-1}{2}}$.
所以对 $k=0,1, \cdots, n$, 在 $2^k \mathrm{C}_n^k(1+x)\left(1+x^2\right)^{n-k} x^k$ 中 $x^n$ 的系数总是 $2^k \mathrm{C}_n^k \mathrm{C}_{n-k}^{\left[\frac{n-k}{2}\right]}$.
从而 $f(x)$ 的 $n$ 次项系数为 $\sum_{k=0}^n 2^k \mathrm{C}_n^k \mathrm{C}_{n-k}^{\left[\frac{n-k}{2}\right]}$.
综合两方面可得: $\sum_{k=0}^n 2^k \mathrm{C}_n^k \mathrm{C}_{n-k}^{\left[\frac{n-k}{2}\right]}=\mathrm{C}_{2 n+1}^n$.
%%PROBLEM_END%%



%%PROBLEM_BEGIN%%
%%<PROBLEM>%%
例10. 对任何正整数 $n$, 证明恒等式: $\sum_{k=0}^n 2^k \mathrm{C}_n^k \mathrm{C}_{n-k}^{\left[\frac{n-k}{2}\right]}=\mathrm{C}_{2 n+1}^n$. 
%%<SOLUTION>%%
证法二(构造计数模型方法)
设有 $n$ 对夫妻, 一个导游, 共 $2 n+1$ 人到甲乙两地游玩, 其中 $n$ 人去甲地, $n+1$ 人去乙地.
一方面, 分配方案显然为 $\mathrm{C}_{2 n+1}^n$ 种.
另一方面, 我们将所有分配方案按 $n$ 对夫妻中恰有 $k(k=0,1, \cdots, n)$ 对分开游玩进行分类计数.
对给定的 $k(0 \leqslant k \leqslant n)$, 先从 $n$ 对夫妻中选取 $k$ 对分开游玩的, 其中每对中谁去甲地各有 2 种可能,所以安排这 $2 k$ 个人的方案有 $2^k \mathrm{C}_n^k$ 种.
对其余 $n-k$ 对夫妻和一个导游, 若 $n-k$ 为偶数, 则必有 $\frac{n-k}{2}$ 对夫妻去甲地; 若 $n-k$ 为奇数, 则必有 $\frac{n-k-1}{2}$ 对夫妻外加导游去甲地, 选择方案总是 $\mathrm{C}_{n^{n-k}}^{\left[\frac{n-k}{2}\right]}$ 种.
由乘法原理可知: $n$ 对夫妻中恰有 $k$ 对分开游玩的情况总数为 $2^k \mathrm{C}_n^k \mathrm{C}_{n-k}^{\left[\frac{n-k}{2}\right]}$. 对 $k=0,1, \cdots, n$ 求和即知 $\sum_{k=0}^n 2^k \mathrm{C}_n^k \mathrm{C}_{n-k}^{\left[\frac{n-k}{2}\right]}=\mathrm{C}_{2 n+1}^n$.
%%<REMARK>%%
注:构造函数与构造模型是数学中经常使用的方法, 前者主要是通过构造函数,把原问题的求解转化为对所作函数性质的研究; 后者则是将原问题中的条件、数量关系在某个模型上实现并得到一种解释, 从而转化为对模型上相应问题的考察, 其中, 所构造的计数模型常常取材于一些通俗易懂的虚构情景( 例如与日常生活有关).
运用构造法解题, 能使代数、三角、几何等各种知识相互渗透, 有利于提高分析问题和解决问题的能力.
%%PROBLEM_END%%


