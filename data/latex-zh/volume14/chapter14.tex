
%%TEXT_BEGIN%%
染色法.
有一些数学问题, 例如操作问题、逻辑推理问题等, 不便用通常的数学方法来解; 还有一些实际问题, 研究的是事物的某种状态或性质, 其本身与数量无关, 也不便用通常的数学方法来解.
这些非常规数学问题可能需要用特定的方法来解决, 例如染色法, 以及在下一节中将要介绍的赋值法等.
染色法是对问题所研究的对象进行分类的一种形象化的方法.
借助于染色手段,能使比较抽象的组合问题转化为一个具体的染色问题,有利于我们观察、分析对象之间的关系,再通过对染色图形的处理达到对原问题的解决.
常见的染色方式有:点染色、边染色、小方格染色和区域染色等.
下面通过一些例题来说明染色法的应用技巧.
%%TEXT_END%%



%%PROBLEM_BEGIN%%
%%<PROBLEM>%%
例1. 证明: 任意 6 人中, 或者有 3 人互相认识, 或者有 3 人互不相识.
%%<SOLUTION>%%
证明: 6 个人视为 6 个点 $A_1, A_2, \cdots, A_6$, 两人相识就在对应的顶点间连一条红边, 否则就连一条蓝边.
这样构成一个图.
问题转化为证明: 该图中必有同色三角形.
考虑 $A_1$ 引出的 5 条边 $A_1 A_2, A_1 A_3, \cdots, A_1 A_6$, 根据抽屉原理, 必有 3 条边染有同一种颜色.
不妨设 $A_1 A_2, A_1 A_3, A_1 A_4$ 都染了红色.
若 $\triangle A_2 A_3 A_4$ 三边均为蓝色, 则结论已成立; 若 $\triangle A_2 A_3 A_4$ 三边中有一条红色, 不妨设 $A_2 A_3$ 为红色,则 $\triangle A_1 A_2 A_3$ 为红色三角形, 结论仍成立.
%%<REMARK>%%
注:这是染色问题中一个经典例子, 它存在各种各样的变形.
通过对边进行染色,我们把问题转化为证明图论中一个著名的拉姆赛 (Ramsey) 型命题: 2 色完全图 $K_6$ 中必存在同色三角形.
此外,每条边既已染色,我们还可以从"同色角"(即由两条同色边组成的角) 的计数出发, 给出另一种证明如下:
设同色三角形个数为 $n$. 考虑图 $K_6$ 中的同色角个数 $S$.
一方面,每个同色三角形有 3 个同色角,非同色三角形有一个同色角, 所以 $S=3 n+\left(\mathrm{C}_6^3-n\right)=2 n+20$; 另一方面, 如果一个顶点引出 $k$ 条红边, 则它引出 $5-k$ 条蓝边, 故以该点为顶点的同色角个数为 $\mathrm{C}_k^2+\mathrm{C}_{5-k}^2$, 其最小值为 4 ,
从而 $S \geqslant 6 \times 4=24$. 综合两方面可知 $n \geqslant 2$, 即有更强的结论成立: 2 色完全图 $K_6$ 中至少有两个同色三角形.
以上的两种证明方法均可以视作证明图的染色问题的典型方法.
%%PROBLEM_END%%



%%PROBLEM_BEGIN%%
%%<PROBLEM>%%
例2. 一只内空尺寸是 $6 \times 6 \times 6$ 的木箱最多能装多少件尺寸为 $1 \times 2 \times 4$ 的长方体货物?
%%<SOLUTION>%%
解:如图(<FilePath:./figures/fig-c14i1.png>),将 $6 \times 6 \times 6$ 的木箱分隔成 27 个 $2 \times 2 \times 2$ 的"中正方体",并进行黑白相间染色, 这样共有 1.4 个黑色的、13 个白色的中正方体.
再将每个中正方体分隔成 8 个小正方体,共 216 个小正方体,其中黑色小正方体 112 个,白色小正方体 104 个.
设最多能放 $k$ 件货物.
显然 $k \leqslant \frac{216}{8}=27$.
假定 $k=27$, 则木箱必被填满而不留空隙, 此时每件货物恰好占据 4 个黑色小正方体和 4 个白色小正方体的空间, 但黑、白小正方体的总个数不同, 故得矛盾.
因此 $k \leqslant 26$.
另一方面又可以放人 26 件货物, 只需注意到图中 27 个中正方体可以配成 13 个两两相邻的对(剩下一个黑色的中正方体),每对提供了 $2 \times 2 \times 4$ 的空间,可容下两件货物.
故 $k=26$.
%%<REMARK>%%
注:本题本身与染色无关, 但若就 $1 \times 2 \times 4$ 长方体本身的形状作讨论, 则难以取得进展.
我们通过染色将木箱的空间进行分类, 说明了有一类空间不可能被填满, 从而找到了解决问题的捷径.
%%PROBLEM_END%%



%%PROBLEM_BEGIN%%
%%<PROBLEM>%%
例3. 有一张 $4 \times 8$ 的方格棋盘.
求证: …只 "马"不能从一格出发, 遍历这张棋盘的每一方格恰好一次,最后回到原出发点.
%%<SOLUTION>%%
证明:假设存在一条满足题目所述条件的"马" 的路径, 那么不妨将左上角 $A$ 格定为它的出发点, 最后它又回到 $A$ 格.
先将棋盘按黑白相间方式染色(如图(<FilePath:./figures/fig-c14i2.png>), $A$ 在白格). 每步"马"一定从一种颜色的格子跳人另一种颜色的格子, 因此 "马" 奇数步走遍一切黑格, 偶数步走遍一切白格.
另一方面, 若将第 $1 、 4$ 行染为白色, 第 $2 、 3$ 行染为黑色 (如图(<FilePath:./figures/fig-c14i3.png>)), 由于"马"从白格只能跳人黑格, 因此为了遍历每一格后恰好回到出发点, "马" 每次在黑格时也必跳向白格, 故"马" 奇数步走遍一切黑格, 偶数步走遍一切白格.
显然两种染色意义下的黑格全体不同, 故假设不成立.
从而命题得证.
%%<REMARK>%%
注:本例使用了两种染色方法.
第一种染色方法与国际象棋棋盘相似, 这种黑白相间的染色方法往往称作 "自然染色", 在解题中常与配对、奇偶性等有内在关系, 可谓一种常规的染色方法; 后一种染色方法则体现了求解具体问题的灵活性, 可谓非常规方法.
在本题中, 我们充分考虑棋盘与棋子的特性, 将常规方法与非常规方法配套使用, 每种方法各揭示了棋子遍历棋盘过程中的一种不变规律,再相互对比导致矛盾,使命题获证.
%%PROBLEM_END%%



%%PROBLEM_BEGIN%%
%%<PROBLEM>%%
例4. 甲、乙两人在一张无限大的方格棋盘上一人一步轮流下棋, 甲先走, 乙后走, 每步棋可将一枚棋子放人任意一个尚未棋子的方格中.
谁先在棋盘上横着或坚着连出 5 枚自己所放的棋子, 判谁获胜.
问: 甲是否有必胜策略?
%%<SOLUTION>%%
解:我们证明乙存在一种使甲无法获胜的策略, 因而甲没有必胜策略.
如图(<FilePath:./figures/fig-c14i4.png>),将棋盘按 $2 \times 2$ 为一"大格"划分, 然后将这些大格黑白相间染色.
当甲在黑格中放一枚棋子时,乙就在横向相邻的黑格中也放一枚棋子; 当甲在白格中放一枚棋子时,乙就在纵向相邻的白格中也放一枚棋子.
在这种策略下, 显然当甲每步放棋子后, 乙所需放棋子的位置必然空着, 故乙的策略可以一直执行下去.
考虑到任何横向连出的 5 枚棋子中必有两枚在相邻黑格中, 任何纵向连出的 5 枚棋子中必有两枚在相邻白格中, 但这两种情形都被乙破坏, 故甲无法获胜.
%%<REMARK>%%
注:本题类似于"五子棋"游戏, 只是斜向连出 5 枚棋子不算获胜, 根据下五子棋的经验, 不妨大胆预测甲没有必胜策略.
余下的问题是乙究竟怎样阻止甲获胜.
不妨考虑采用配对思想, 例如将棋盘自然染色, 并把相邻的黑格与白格两两配对, 甲每下一子, 乙就在旁边配对的位置放一子进行 "防守". 此时, 为了兼顾横向和纵向的防守, 须适当调整配对的秩序, 为此引人 "大格" 的概念, 对大格进行自然染色, 在黑格与白格中分别进行"横向配对"与 "纵向配对",在此基础上便能构造出一种使乙不败的策略.
%%PROBLEM_END%%



%%PROBLEM_BEGIN%%
%%<PROBLEM>%%
例5. 在一个 $10 \times 10$ 的方格表中有一个由 $4 n$ 个 $1 \times 1$ 的小方格组成的图形, 它既可被 $n$ 个"图(<FilePath:./figures/fig-c14e1.png>)"型的图形覆盖, 也可被 $n$ 个"图(<FilePath:./figures/fig-c14e2.png>)"或"图(<FilePath:./figures/fig-c14e3.png>)"型(可以旋转) 的图形覆盖.
求正整数 $n$ 的最小值.
%%<SOLUTION>%%
解:首先论证 $n$ 是偶数.
用如图(<FilePath:./figures/fig-c14i5.png>) 所示方法将平面网格染色.
记图(<FilePath:./figures/fig-c14e1.png>)的图形为A形, 图(<FilePath:./figures/fig-c14e2.png>)和图(<FilePath:./figures/fig-c14e3.png>)可以旋转的图形为 $B$ 形.
无论 $A$ 形覆盖哪 4 个方格, 其中黑格数必是偶数,而对于 $B$ 形则是奇数.
如果 $n$ 是奇数, $n$ 个 $A$ 形所覆盖的黑方格数必是偶数; 而 $n$ 个 $B$ 形所覆盖的黑方格数必是奇数, 矛盾.
所以 $n$ 必是偶数.
如果 $n=2$, 由 2 个 $A$ 形拼成的图形只有如图(<FilePath:./figures/fig-c14i6.png>) 所示的两种情形,但是它们都不能由 2 个 $B$ 形拼成.
所以, $n \geqslant 4$. 如图(<FilePath:./figures/fig-c14i7.png>) 是 $n=4$ 时的拼法.
%%<REMARK>%%
注:本题采用的染色方法与上一题相同, 但从不同的角度考察染色的特性, 因而解决了一个全然不同的问题.
%%PROBLEM_END%%



%%PROBLEM_BEGIN%%
%%<PROBLEM>%%
例6. 平面上有一无穷大的方格棋盘, 每格只允许放一枚棋子.
两枚棋子称为 "相邻", 是指它们所在的格子有至少一个公共的顶点.
一开始棋子正好摆成一个 $9 \times 9$ 的正方形.
每一步可选择下述两种规则之一进行操作:
(1)如果一枚棋子 $X$ 在水平或垂直方向有相邻的棋子 $Y$, 并且 $X$ 关于 $Y$ 对称的格子是空的, 则可在那个空格放一枚棋子, 同时取出 $X$ 和 $Y$ 这两枚棋子;
(2)如果一枚棋子 $X$ 没有相邻的棋子, 那么可将它取出, 并在所有相邻的 8 个格子中同时放上棋子.
问 : 能否通过有限步操作使棋盘上只剩一枚棋子?
%%<SOLUTION>%%
解:结论是否定的.
将棋盘按如图(<FilePath:./figures/fig-c14i8.png>)中方式分别染上 $A, B, C$ 三种颜色.
对于每步操作, 若是按规则 (1) 操作, 则两种颜色的方格所含的棋子数减少 1 , 另一种颜色的方格所含的棋子数增加 1 ; 若是按规则 (2) 操作, 不妨设这枚取出的棋子所在格为 $A$ 色, 由于 $A$ 色格周围 8 格中恰有 2 个 $A$ 色格, 3 个 $B$ 色格, 3 个 $C$ 色格, 因此操作后 $A$ 色格所含的棋子数增加 $1, B$ 色格与 $C$ 色格所含的棋子数各增加 3. 可见无论哪种操作规则, 每步操作后每种颜色的方格所含棋子数的奇偶性都同时改变.
开始时在 $A, B, C$ 三种颜色的方格内各有 27 枚棋子, 其奇偶性相同.
假如若干步后棋盘上只剩一枚棋子, 则三种颜色方格内的棋子数肯定分别为 $1,0,0$, 其奇偶性不同, 这不可能.
因此无法通过有限步操作使棋盘上只剩一枚棋子.
%%<REMARK>%%
注:本题中运用"三染色"方法是解题的关键, 这样便使操作的不变规律 (即三类方格内棋子数总是同时改变奇偶性) 直观地显示出来, 使分析和处理问题变得明朗.
下面是关于本题的两点补充说明:
(1)当题目中 $9 \times 9$ 改成一般的 $n \times n$ 时,可以证明当且仅当 $n$ 不是 3 的倍数时, 可通过有限步操作使棋盘上只剩一枚棋子(相应的操作步骤可以归纳构造).
(2)从所采用的染色方法中不难发现,即便放宽规则 (1), 允许被操作的棋子 $X$ 与 $Y$ 之间互为"左上一一右下"的相邻关系, 上述证明同样奏效.
又根据对称性,若规则 (1)放宽为允许互为 "左下一一右上"的相邻关系, 那么可将染色方法改为镜像,使证明仍有效.
但若将规则 (1) 同时作上述两种放宽, 那么本题结论将变为肯定的,请读者自行验证.
%%PROBLEM_END%%



%%PROBLEM_BEGIN%%
%%<PROBLEM>%%
例7. $15 \times 15$ 的方格表中有一条非自交闭折线,该折线由若干条连接相邻小方格 (两个有公共边的小方格称为相邻小方格) 的中心的线段组成, 且它关于方格表的某条对角线对称.
证明:这条闭折线的长度不大于 200 .
%%<SOLUTION>%%
解:显然,折线与对角线相交.
令 $A$ 是一个这样的交点.
我们沿着折线运动, 设 $B$ 是第一个再次与对角线相交的点.
由对称性, 如果我们沿着折线按另一方向运动, $B$ 仍然是第一个与对角线的交点.
这样折线在 $A$ 与 $B$ 之间已经封闭起来.
这表明折线与该对角线有且只有两个交点.
现在将方格表中的小方格用黑白相间的方式染色,使得对角线上的小方格全为黑色.
注意到沿折线运动时, 黑白格交替经过.
因此, 经过的黑白格数目相等.
表中黑格比白格多一个.
由于对角线上都是黑格, 折线与其中的 13 个不交,故折线至少与 12 个白格不交.
由此,折线的长度不超过 $15^2-13-12=$ 200.
%%PROBLEM_END%%



%%PROBLEM_BEGIN%%
%%<PROBLEM>%%
例8. 已知圆周上有 $3 k\left(k \in \mathbf{N}^*\right)$ 个分点, 它们把圆周分成 $3 k$ 段弧, 其中长度为 $1 、 2 、 3$ 的弧各有 $k$ 条.
求证: 这 $3 k$ 个分点中必有两点为对径点 (即这两点的连线为圆的直径). 
%%<SOLUTION>%%
证明:将给定的 $3 k$ 个分点都染为红色,再将所有长度为 2 的弧段的中点和长度为 3 的弧段的三等分点都染成蓝色.
显然蓝点也是 $3 k$ 个.
于是, 问题转化为证明: 有一对红点为对径点.
假设结论不成立, 则红点的对径点都是蓝点.
又由于红点与蓝点一样多, 故蓝点的对径点必为红点.
在 $3 k$ 段弧中, 任取一段长度为 2 的弧 $\overparen{A C}$, 则 $A, C$ 为红点, $\overparen{A C}$ 的中点 $B^{\prime}$ 是蓝点, 因此 $B^{\prime}$ 的对径点 $B$ 是红点.
考察长度为 $3 k-1$ 的弧 $\overparen{A B}$, 设其上长度为 $i$ 的弧段数目是 $n_i(i=1,2$, 3 , 则有 $n_1+2 n_2+3 n_3=3 k-1$. 因为长度为 1 的弧段的两个端点都是红点, 所以它们的对径点都是蓝点.
这两个相邻的蓝点恰好对应了弧 $\overparen{B C}$ 上一条包含它们的长度为 3 的弧, 并且这是个一一对应.
从而弧 $\overparen{B C}$ 上长为 3 的弧段数等于 $n_1$. 另一方面, 弧 $\overparen{A B}$ 和弧 $\overparen{B C}$ 上一共有 $k$ 段长度为 3 的弧, 所以 $n_3+n_1=k$. 从而
$$
2 n_2+2 n_3=\left(n_1+2 n_2+3 n_3\right)-\left(n_3+n_1\right)=(3 k-1)-k=2 k-1 .
$$
上式左端为偶数而右端为奇数, 矛盾.
%%<REMARK>%%
注:本题借助点的染色使条件和问题转化, 有利于我们观察、分析对象间的关系, 使证明过程直观具体.
在证明中还涉及了对应原理和奇偶性原理.
%%PROBLEM_END%%


