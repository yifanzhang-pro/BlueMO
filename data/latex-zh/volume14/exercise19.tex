
%%PROBLEM_BEGIN%%
%%<PROBLEM>%%
问题1. 已知三个数 $5,12,18$. 每一次操作是从这三个数中任意选出两个 $a, b$, 并用 $\frac{\sqrt{2}}{2}(a+b), \frac{\sqrt{2}}{2}(a-b)$ 代替它们.
问: 是否能进行有限次操作, 使得得到的三个数为 $3,13,20$ ?
%%<SOLUTION>%%
因为
$$
\left[\frac{\sqrt{2}}{2}(a+b)\right]^2+\left[\frac{\sqrt{2}}{2}(a-b)\right]^2=a^2+b^2,
$$
所以, 每一次操作后, 这三个数的平方和保持不变 (即三个数的平方和是不变量). 而 $5^2+12^2+18^2=493,3^2+13^2+20^2=578$, 于是, 无论如何操作, 数 $5,12,18$ 不会变为数 $3,13,20$.
%%PROBLEM_END%%



%%PROBLEM_BEGIN%%
%%<PROBLEM>%%
问题2. 对于黑板上的 100 个数 $1, \frac{1}{2}, \cdots, \frac{1}{100}$, 每次任意选定两数 $a, b$ 擦去,同时补上 $a+b+a b$, 共操作 99 次, 求最后剩下的那个数.
%%<SOLUTION>%%
考虑到 $(1+a)(1+b)=1+a+b+a b$, 因此操作前后 $\prod_{i=1}^n\left(1+a_i\right)$ 保持不变, 其中 $n$ 为黑板上数的个数, $a_1, a_2, \cdots, a_n$ 为这 $n$ 个数的值.
设最后剩下的数为 $x$, 则
$$
x+1=\prod_{i=1}^{100}\left(1+\frac{1}{i}\right)=\prod_{i=1}^{100} \frac{i+1}{i}=101,
$$
所以 $x=100$.
%%PROBLEM_END%%



%%PROBLEM_BEGIN%%
%%<PROBLEM>%%
问题3. 在平面上画一个 $4 \times 4$ 的方格表, 在这些小方格的每一个中都任意地填人 1 或 -1 . 下面的一种改变填人数字的方式称为一次操作: 对任意一个小方格, 凡是与此小方格有一条公共边的所有小方格(不包括此小方格本身) 中的数作连乘积, 于是每取一格, 就算出了一个数.
在所有小方格都取遍后, 再将这些算出的数放人相应的小方格中.
问: 对任意的一种初始填表方式, 是否能经过有限次操作后, 使得所有小方格中的数都变为 1 ?
%%<SOLUTION>%%
答案是否定的.
对于如图(<FilePath:./figures/fig-c19a3.png>)所示的一个 $4 \times 4$ 的数表, 每经过一次操作, 仍然是它本身, 即这个数表在上述操作中是不变量.
所以, 这个数表不可能经过有限次操作后, 使得所有的小方格中的数都变为 1 .
\begin{tabular}{|c|c|c|c|}
\hline 1 & -1 & -1 & 1 \\
\hline-1 & 1 & 1 & -1 \\
\hline-1 & 1 & 1 & -1 \\
\hline 1 & -1 & -1 & 1 \\
\hline
\end{tabular}
%%PROBLEM_END%%



%%PROBLEM_BEGIN%%
%%<PROBLEM>%%
问题4. 某岛上生活着 45 条变色龙, 其中有 13 条灰色的, 15 条褐色的, 17 条紫色的.
每当两条颜色不同的变色龙相遇时, 它们就一起都变为第三种颜色.
问:能否经过一段时间, 45 条变色龙全都变为同一颜色?
%%<SOLUTION>%%
每一次变化, 都有两条不同颜色的变色龙消失, 而增加了两条第三种颜色的变色龙.
用三元数组 $(a, b, c)$ 来表示变色龙的条数, 其中 $a, b, c$ 分别表示灰色,褐色,紫色的变色龙的条数.
在一次变化后, $(a, b$, c) 变为 $(a-1, b-1, c+2)$, 或者变为 $(a-1, b+2, c-1)$, 或者变为 $(a+2$, $b-1, c-1)$.
由于灰色和褐色变色龙的数目之差的变化只能是 $0,-3$ 和 3 , 也就是说, 这个差被 3 除所得的余数不变, 这是一个不变量! 开始时, $a-b=13-15=$ -2 , 如果变为同一颜色, 则有 $a-b \equiv 0(\bmod 3)$, 不可能.
%%PROBLEM_END%%



%%PROBLEM_BEGIN%%
%%<PROBLEM>%%
问题5. 在直角坐标系中, 有 4 个筹码放在整点上, 一个筹码按照如下规则移到新位置: 它的原位置和新位置的中点必须有另外一个筹码.
问 : 是否存在确定的移动步骤, 使放在 $(0,0),(0,1),(1,0),(1,1)$ 的筹码移到 $(0,0)$, $(1,1),(3,0),(2,-1)$ 上?
%%<SOLUTION>%%
对于每次有效的移动, 按照规则存在唯一的反向移动, 因此只需证明不能从最终位置按照规则移动到开始位置.
在最终的情形下, 4 个筹码的坐标 $(x, y)=(0,0),(1,1),(3,0),(2$, $-1)$ 均满足 $x \equiv y(\bmod 3)$, 根据筹码的移动规则可知, 无论多少次移动, 4 个筹码的坐标仍满足 $x \equiv y(\bmod 3)$, 因此不可能移到最初情形 $(0,0),(0,1)$, $(1,0),(1,1)$. 从而命题成立.
%%PROBLEM_END%%



%%PROBLEM_BEGIN%%
%%<PROBLEM>%%
问题6. 有一个 $n \times n$ 的方格表, 先允许从中任意选择 $n-1$ 个方格染成黑色, 再逐步地将那些至少与两个已染黑的方格相邻的方格也染为黑色.
证明: 不论怎样选择最初的 $n-1$ 个方格, 都不能按这样的法则染黑所有的方格.
%%<SOLUTION>%%
考虑黑色区域周长之和 (注意两个相邻黑格的公共边界并不计人周长).
一开始任选 $n-1$ 个方格染成黑色, 此时黑色区域周长至多为 $4(n-1)$. 在此后逐步将一些方格染黑的过程中, 我们证明黑色区域的周长之和是个不增的量.
事实上, 如图(<FilePath:./figures/fig-c19a6.png>), 若在某一时刻与未染色的 $P$ 格相邻的四格 $A, B, C, D$ 中已有 $k$ 个为黑色,那么添上 $P$ 后黑色区域少去 $k$ 条边界, 同时增加 $4-k$ 条边界, 但 $P$ 能染色的前提是 $k \geqslant 2$, 从而 $4-k \leqslant k$,这样 $P$ 格染色后黑色区域周长之和不增.
假设最终所有方格都染黑, 那么黑色区域周长应为 $4 n> 4(n-1)$,矛盾!所以必定无法染黑所有的方格.
%%PROBLEM_END%%



%%PROBLEM_BEGIN%%
%%<PROBLEM>%%
问题7. 在一块已写有一个正整数的黑板上进行如下操作: 若 $x$ 已写在黑板上, 则可以在黑板上写上数 $2 x+1$ 或 $\frac{x}{x+2}$. 已知在某个时刻黑板上写有数 2008. 证明: 黑板上原有的数是 2008. 
%%<SOLUTION>%%
显然黑板上只能出现正有理数.
设某一时刻黑板上有有理数 $x=\frac{p}{q}$, $(p, q)=1$, 则由 $x$ 可得到 $2 x+1=\frac{2 p+q}{q}$ 或 $\frac{x}{x+2}=\frac{p}{p+2 q}$.
注意到
$$
(2 p+q, q)=(2 p, q) \leqslant 2(p, q)=2,(p, p+2 q)=(p, 2 q) \leqslant 2(p, q)=2,
$$
而
$$
(2 p+q)+q=p+(p+2 q)=2(p+q),
$$
故由 $x$ 得到的最简分数的分子与分母之和为 $p+q$ 或 $2(p+q)$, 即分子分母之和或者不变或者变为原来的两倍.
由于后来出现的 2008 的分子分母之和为 2009 是一个奇数,故从初始数到 2008 的变化过程中没有出现过加倍的情况.
因此初始数的分子分母之和就是 2009. 再考虑到初始数是一个正整数, 故它为 2008 .
%%PROBLEM_END%%



%%PROBLEM_BEGIN%%
%%<PROBLEM>%%
问题8. 桌上有 2009 枚硬币, 将其一面染上白色, 另一面染上黑色.
起初, 将所有硬币排成一排,其中一枚硬币黑面朝上,其余 2008 枚硬币均白面朝上.
按如下规则操作: 每一次操作中, 选择一枚黑面朝上的硬币, 并翻转与其相邻的两枚硬币; 若所选的黑面朝上的硬币在这一排硬币的两端,则只需翻转与端点相邻的那一枚硬币.
请找出初始状态时那枚黑面朝上的硬币的所有可能位置, 使得经过若干次操作后, 所有的硬币均黑面朝上.
%%<SOLUTION>%%
将给定的这一排硬币从左至右依次编号为 $1,2, \cdots, 2009$.
在任何一步操作之前, 设所有黑面朝上的硬币所在位置从左至右分别对应编号 $a_1, a_2, \cdots, a_k$ (其中 $k$ 亦可能随操作而变化).
构造一个量 $S=\sum_{i=1}^k(-1)^{i+1} a_i$.
设此时对编号 $a_j$ 的硬币进行操作.
考虑操作后相应于 $S$ 的量 $S^{\prime}$.
(1)硬币 $a_j$ 不是第一枚或最后一枚, 即 $a_j \neq 1,2009$. 我们来验证 $S^{\prime}=$ S:
若与 $a_j$ 相邻的两枚硬币同色面朝上, 可不妨假定操作前第 $a_j-1, a_j+1$ 枚硬币均白面朝上,则操作后它们黑面朝上,其余硬币无变化, 故
$$
S^{\prime}-S=\left((-1)^{j+1}\left(a_j-1\right)+(-1)^{j+2} a_j+(-1)^{j+3}\left(a_j+1\right)\right)-(-1)^{j+1} a_j=0 .
$$
若第 $a_j-1, a_j+1$ 枚硬币异色面朝上,可不妨假定操作前第 $a_j-1$ 枚硬币白面朝上,第 $a_j+1$ 枚硬币黑面朝上 (即 $a_{j+1}=a_j+1$ ), 类似地有
$$
S^{\prime}-S=\left((-1)^{j+1} a_j+(-1)^{j+2}\left(a_j+1\right)\right)-\left((-1)^{j+1}\left(a_j-1\right)+(-1)^{j+2} a_j\right)=0 .
$$
(2)硬币 $a_j$ 是最后一枚, 即 $j=k, a_k=2009$.
我们引进编号为 $a_{k+1}=2010$ 的"虚拟硬币",操作前它黑面朝上.
记
$$
S_1=\sum_{i=1}^{k+1}(-1)^{i+1} a_i=S+(-1)^{k+2} 2010 .
$$
由前一种情形的讨论知 $S_1^{\prime}=S_1$, 而操作后第 2010 枚硬币白面朝上,所以
$$
S^{\prime}=S_1^{\prime}=S_1=S+(-1)^{k+2} 2010 .
$$
(3)硬币 $a_j$ 就是第一枚,即 $j=1, a_1=1$.
若操作前 $a_2=2$, 则
$$
S^{\prime}=1-\sum_{i=3}^k(-1)^{i+1} a_i=-a_1+a_2-\sum_{i=3}^k(-1)^{i+1} a_i=-S ;
$$
若操作前 $a_2>2$, 则
$$
S^{\prime}=1-2+\sum_{i=3}^{k+1}(-1)^{i+1} a_{i-1}=-a_1+\sum_{i=2}^k(-1)^{i+2} a_i=-S .
$$
综合 (1)、(2)、(3) 可知, 操作前后 $S, S^{\prime}$ 总满足 $S^{\prime} \equiv \pm S(\bmod 2010)$. 若要使所有硬币均黑面朝上, 则最终的 $S=\sum_{i=1}^{2009}(-1)^{i+1} i=1005$, 故仅有一种可能的初始状态, 即起初第 1005 个位置上的硬币黑面朝上.
最后构造一种操作方式使上述初始状态最终达到所有硬币黑面朝上的状态.
记第 1005 个位置为 $m$. 假如黑面朝上的硬币刚好出现在 $m-k$ 到 $m+k$ 位置上 $(k \in \mathbf{N})$, 若 $k=2 l(l \in \mathbf{N})$, 则依次对如下位置的硬币进行操作:
$$
m-2 l, m+2 l, m-(2 l-2), m+(2 l-2), \cdots, m-2, m+2, m ;
$$
若 $k=2 l+1(l \in \mathbf{N})$, 则依次对如下位置的硬币进行操作 :
$$
m-(2 l-1), m+(2 l-1), m-(2 l-3), m+(2 l-3), \cdots, m-1, m+1 \text {. }
$$
两种情况下均可使黑面朝上的硬币刚好出现在 $m-k-1$ 到 $m+k+1$ 位置上.
由于初始状态就是 $k=0$ 的情形, 因此根据数学归纳法可知 $k=1004$ 的情形可以通过适当操作而得到, 此时所有硬币均黑面朝上.
从而满足题意的位置就是第 1005 个位置.
%%PROBLEM_END%%


