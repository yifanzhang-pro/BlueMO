
%%TEXT_BEGIN%%
集合的分划一个集合可以写成若干个集合的并集,例如集合 $\{1,2, 3, 4, 5\}$, 可以写成两个集合 $A=\{1,2,3\}, B=\{3,4,5\}$ 的并集 $A \cup B$; 也可以写成三个集合 $C=\{1,2\}, D=\{3,5\}, E=\{4\}$ 的并集 $C \cup D \cup E$, 等等.
下面我们来研究将一个集合表示成若干个集合的并集的一种特殊情形.
定义把一个集合 $M$ 分成 $n$ 个非空的子集: $A_1, A_2, \cdots, A_n$, 如果:
(1) $A_i \cap A_j=\varnothing(1 \leqslant i, j \leqslant n, i \neq j)$;
(2) $\bigcup_{i=1}^n A_i=M$,
那么,这些子集的全体叫做集合 $M$ 的一个 $n$-分划.
由集合分划的定义, 容易证明有限集的一个非常有用的性质:
加法原理设 $A_1, A_2, \cdots, A_n$ 是有限集 $M$ 的一个 $n$-分划, 则
$$
|M|=\sum_{i=1}^n\left|A_i\right|
$$
这是一个基本的计数公式.
集合的分划引出了大量有趣的数学问题.
%%TEXT_END%%



%%PROBLEM_BEGIN%%
%%<PROBLEM>%%
例1. 试将集合 $\{1,2, \cdots, 1989\}$ 分为 117 个互不相交的子集 $A_i ( i=1, 2, \cdots, 117)$, 使得:
(1) 每个 $A_i$ 都含有 17 个元素;
(2) 所有 $A_i$ 中各元素之和都相同.
%%<SOLUTION>%%
分析:因为 $1989=117 \times 17$, 故可将 $\{1,2, \cdots, 1989\}$ 顺次分成 17 段, 每段含 117 个数.
显然, 只要把每段的 117 个数适当地分别放人 $A_1, A_2, \cdots, A_{117}$ 中以使条件 (2)满足,问题就解决了.
解将集合 $\{1,2, \cdots, 1989\}$ 中的数从小到大顺次分成 17 段,每段含 117 个数.
从第 4 段数开始, 将偶数段的数从小到大依次放人 $A_1, A_2, \cdots, A_{117}$ 中, 并将奇数段的数从大到小依次放人这 117 个子集中.
易见,所有集合中的 14个数之和都相等.
于是问题归结为如何将前三段数 $\{1,2, \cdots, 351\}$ 每 3 个一组分别放人每个集中, 且使每组 3 数之和都相等.
把这些数中 3 的倍数抽出来从大到小排好: $\{351,348,345, \cdots, 6,3\}$ , 共 117 个数,依次放人 $A_1, A_2, \cdots, A_{117}$ 中.
其余的 234 个数从小到大排列并分成两段, 每段 117 个数, 即 $\{1,2,4,5,7, \cdots, 173 , 175\}$ 和 $\{176,178,179, \cdots, 349,350\}$. 将这两段数分别顺次放人 $A_1, A_2, \cdots, A_{117}$ 之中便满足要求.
事实上, 若将这两段数中的数顺次相加, 则其和为 $\{177,180,183,186, \cdots, 522,525\}$. 由此可见, 放人每个 $A_i$ 的 3 数之和都是 528 .
说明上述解法是通过具体地构造 $A_i(i=1,2, \cdots, 117)$ 完成的.
由此不难看出, 这种构造方式不是惟一的, 有兴趣者不妨一试.
%%PROBLEM_END%%



%%PROBLEM_BEGIN%%
%%<PROBLEM>%%
例2. 集合 $H=\{1,2, \cdots, 9\}$ 的分拆 $p$ 是将 $H$ 表示为两两不相交的子集的并.
对于 $n \in H$ 和分拆 $p$, 将包含 $n$ 的子集中元素的数目记为 $p(n)$. 例如, 若 $p:\{1,4,5\} \cup\{2\} \cup\{3,6,7,8,9\}$, 则 $p(6)=5$. 证明: 对 $H$ 的任意两个分拆 $p_1 、 p_2$, 存在 $H$ 的两个不同的元素 $m 、 n$, 使得
$$
p_1(m)=p_1(n), p_2(m)=p_2(n) .
$$
%%<SOLUTION>%%
分析:因为 $H$ 只有 9 个元素, 对于一个确定的分拆 $p, p(i)(i=1$, $2, \cdots, 9)$ 只有三种不同的取值, 这是因为若有四种不同的取值, 则至少需要 $1+2+3+4=10$ 个元素.
这就给我们打开了一条通过对 $p(i)$ 可能的取值个数的研究解决问题的思路.
解用反证法.
假设可以找到 $H$ 的两个分拆 $p_1 、 p_2$, 使不存在 $H$ 的两个不同的元素 $m 、 n$, 满足
$$
p_1(m)=p_1(n), p_2(m)=p_2(n) .
$$
对于确定的 $p_1$, 若 $p_1(i)(i=1,2, \cdots, 9)$ 有四种不同的取值, 则至少需要 $1+2+3+4=10$ 个元素, 而 $|H|=9$, 矛盾.
所以, $p_1(i)$ 至多有三种不同的取值.
若同时有四个元素的 $p_1(i)$ 取值相等, 由于 $p_i(i)$ 至多有三个不同取值, 所以,必有四个中的两个元素 $m 、 n$,使得 $p_2(m)=p_2(n)$ ,与假设矛盾.
若 $p_1(i)$ 至多有两种不同的取值, 由抽屉原理知, 至少有 $H$ 的四个不同元素的 $p_1(i)$ 值相同.
这说明对于任意确定的 $p_1, p_1(i)$ 恰有三种不同取值, 且每种取值有三个元素取到.
也就是说对于分拆 $p_1, H$ 的每一个子集的元素个数不超过 3 . 不妨设
$$
\begin{aligned}
& p_1(1)=p_1(2)=p_1(3)=1, \\
& p_1(4)=p_1(5)=p_1(6)=2,
\end{aligned}
$$
$$
p_1(7)=p_1(8)=p_1(9)=3 .
$$
但 $p_1(4)=p_1(5)=p_1(6)=2$ 是不可能的.
这就否定了假设.
%%PROBLEM_END%%



%%PROBLEM_BEGIN%%
%%<PROBLEM>%%
例3. 对一个由非负整数组成的集合 $S$, 定义 $r_s(n)$ 为满足下述条件的有序对 $\left(s_1, s_2\right)$ 的对数:
$$
s_1 \in S, s_2 \in S, s_1 \neq s_2, \text { 且 } s_1+s_2=n .
$$
问: 是否能将非负整数集分划为两个集合 $A$ 和 $B$, 使得对任意 $n$, 均有 $r_A(n)=r_B(n)$ ?
%%<SOLUTION>%%
分析:整数有多种表示形式, 其中二进制表示的每位数字只有 0 和 1 这两种选择.
由于是将 $S$ 分划为两个集合 $A 、 B$, 对每个固定的 $n$, 满足 $s_1+s_2=n$ 的非负整数对 $\left(s_1, s_2\right)$ 是有限的, 用二进制数来讨论 $\left(s_1, s_2\right)$ 在 $A$ 和 $B$ 中的分配情况似乎较有利.
解存在上述的分划.
将所有二进制表示下数码 1 出现偶数个的非负整数归人集合 $A$, 其余的非负整数归人 $B$, 则 $A 、 B$ 是非负整数集 $N$ 的分划.
注意到, 对 $A$ 中满足 $a_1+a_2=n, a_1 \neq a_2, a_1, a_2 \in A$ 的数对 $\left(a_1, a_2\right)$, 由于 $a_1 \neq a_2$, 因此在二进制表示下 $a_1$ 与 $a_2$ 必有一位上的数码不同, 从右到左看, 第 1 个不同数码的数位上, 改变 $a_1 、 a_2$ 在该位上的数码, 分别得到 $b_1 、 b_2$, 则 $b_1 、 b_2 \in B$, 且 $b_1 \neq b_2, b_1+b_2=n$. 这个将 $\left(a_1, a_2\right)$ 对应到 $\left(b_1, b_2\right)$ 的映射是一一对应, 因此 $r_A(n)=r_B(n)$.
说明这是一个存在性问题.
我们是利用二进制数构造 $S$ 的 $2-$ 分划 $A$ 、 $B$, 然后通过建立 $A$ 中有序对集 $\{\left(a_1, a_2\right) \mid a_1, a_2 \in A, a_1 \neq a_2, a_1+a_2=n\}$ 与 $B$ 中有序对集 $\left\{\left(b_1, b_2\right) \mid b_1, b_2 \in B, b_1 \neq b_2, b_1+b_2=n\right\}$ 的一一对应来解决的.
利用一一对应解决计数问题的方法就是所谓的配对原理.
%%PROBLEM_END%%



%%PROBLEM_BEGIN%%
%%<PROBLEM>%%
例4. 设集合 $A=\{1,2, \cdots, m\}$. 求最小的正整数 $m$, 使得对 $A$ 的任意一个 14 -分划 $A_1, A_2, \cdots, A_{14}$, 一定存在某个集合 $A_i(1 \leqslant i \leqslant 14)$, 在 $A_i$ 中有两个元素 $a 、 b$, 满足 $b<a \leqslant \frac{4}{3} b$.
%%<SOLUTION>%%
分析:由于要考虑的是一种极端情况, 我们来作一张元素、集合从属关系的表: 从 1 开始, 由小到大每 14 个数为一组, 依次填人表中的每一列中 (如表 4-1). 填满 4 列后, 观察发现: 去掉右下角的数 56 后, 子集 $A_1, A_2, \cdots$, $A_{13}$ 中每一个都有 4 个元素, 而 $A_{14}$ 有 3 个元素, 这时 $A_1, A_2, \cdots, A_{14}$ 任何一个中都不存在两个元素满足题中的不等式.
故 $m \geqslant 56$.
表 4-1
$\begin{array}{llllll}A_1 & 1 & 15 & 29 & 43 & \cdots\end{array}$
$\begin{array}{llllll}A_2 & 2 & 16 & 30 & 44 & \cdots\end{array}$
$\begin{array}{llllll}A_3 & 3 & 17 & 31 & 45 & \cdots\end{array}$
$\begin{array}{cccccc}\vdots & \vdots & \vdots & \vdots & \vdots & \vdots \\ A_{12} & 12 & 26 & 40 & 54 & \cdots \\ A_{13} & 13 & 27 & 41 & 55 & \cdots \\ A_{14} & 14 & 28 & 42 & 56 & \cdots\end{array}$
表 4-2
$\begin{array}{lllll}A_1 & 1 & 15 & 29 & 43\end{array}$
$\begin{array}{lllll}A_2 & 2 & 16 & 30 & 44\end{array}$
$\begin{array}{lllll}A_3 & 3 & 17 & 31 & 45\end{array}$
$\begin{array}{ccccc}\vdots & \vdots & \vdots & \vdots & \vdots \\ A_{12} & 12 & 26 & 40 & 54 \\ A_{13} & 13 & 27 & 41 & 55 \\ A_{14} & 14 & 28 & 42 & \end{array}$
解如表 4-2, 第 $i$ 行的数即为子集 $A_i$ 的元素.
这时 $\left|A_i\right|=4(i=1,2, \cdots, 13),\left|A_{14}\right|=3$. 显然, 14 个子集每一个都不存在两个元素满足题中不等式.
所以, $m \geqslant 56$.
另一方面, 若 $m=56$, 则对 $A$ 的任意分划 $A_1, A_2, \cdots, A_{14}$, 数 $42 , 43, \cdots, 56$ 中, 必有两个数属于同一个 $A_i$, 取此二数为 $a 、 b$, 则
$$
42 \leqslant a<b \leqslant 56=\frac{4}{3} \cdot 42 \leqslant \frac{4}{3} a .
$$
综上所述, 所求 $m$ 的最小正整数值为 56 .
另解若 $m<56$, 令 $A_i=\{a \mid a \equiv i(\bmod 14), a \in A\}$, 则对任意 $a, b \in A_i(i=1,2, \cdots, 14), b<a$, 均有 $56>a>b$, 且 $a-b \geqslant 14$. 故 $b<a- 14<42$. 于是
$$
\frac{a}{b}=1+\frac{a-b}{b} \geqslant 1+\frac{14}{b}>1+\frac{14}{42}=\frac{4}{3} .
$$
所以, $m \geqslant 56$.
后同前解.
%%PROBLEM_END%%



%%PROBLEM_BEGIN%%
%%<PROBLEM>%%
例5. 证明: 可以把自然数集分划为 100 个非空子集,使得对任何 3 个满足关系式 $a+99 b=c$ 的自然数 $a 、 b 、 c$, 都可以从中找出两个数属于同一子集.
%%<SOLUTION>%%
分析:当然, 只要能具体地构造一个满足条件的 100 -分划即可.
在构造之前, 有必要对关系式 $a+99 b=c$ 进行讨论.
有两点是显然的: $a(\bmod 99) \equiv c(\bmod 99) ; a 、 b 、 c$ 中偶数的个数为奇数.
我们的证明由此人手.
证明按如下法则构造自然数集的子集: 在第 $i$ 个子集 $(1 \leqslant i \leqslant 99)$ 中放人所有被 99 除余 $i-1$ 的偶数,而在第 100 个子集中放人所有的奇数.
显然, 这是一个自然数集的 $100-$ 分划.
在任何满足方程 $a+99 b=c$ 的自然数 $a 、 b 、 c$ 中, 偶数的个数为奇数, 且
$a(\bmod 99) \equiv c(\bmod 99)$.
如果 $a 、 b 、 c$ 中有两个为奇数,则此二奇数同属第 100 个子集; 否则, 它们全为偶数,且 $a$ 和 $c$ 被 99 除同余,故 $a$ 与 $c$ 仍属于同一子集.
%%PROBLEM_END%%



%%PROBLEM_BEGIN%%
%%<PROBLEM>%%
例6. 设集合 $A_1, A_2, \cdots, A_n$ 和 $B_1, B_2, \cdots, B_n$ 是集合 $M$ 的两个 $n$ 一分划, 已知对任意两个交集为空集的集合 $A_i, B_j(1 \leqslant i, j \leqslant n)$, 均有 $\left|A_i \cup B_j\right| \geqslant n$. 求证: $|M| \geqslant \frac{n^2}{2}$.
%%<SOLUTION>%%
分析:由 $A_i 、 B_j$ 的交集为空集,有 $\left|A_i \cup B_j\right|=\left|A_i\right|+\left|B_j\right| \geqslant n$. 当每一个 $\left|A_i\right| \geqslant \frac{n}{2}$ 时, 结论显然成立.
当某个 $\left|A_i\right|$, 不妨设为 $\left|A_1\right|$ 小于 $\frac{n}{2}$ 时, 设 $\left|A_1\right|=k$, 这时与 $A_1$ 相交的 $B_j$ 至多有 $k$ 个; 而至少有 $n-k$ 个集合与 $A_1$ 不相交, 它们每一个的元素个数不小于 $n-k$. 假如 $k$ 是所有 $\left|A_i\right| 、\left|B_j\right|$ 中最小的, 则有 $|M| \geqslant k \cdot k+(n-k)(n-k) \geqslant \frac{n^2}{2}$.
证明设 $k=\min \left\{\left|A_i\right|,\left|B_j\right|, 1 \leqslant i, j \leqslant n\right\}$, 不妨设 $\left|A_1\right|=k$.
若 $k \geqslant \frac{n}{2}$, 则
$$
|M|=\sum_{i=1}^n\left|A_i\right| \geqslant n k \geqslant \frac{n^2}{2}
$$
若 $k<\frac{n}{2}$, 由于 $B_1, B_2, \cdots, B_n$ 两两不相交, 故 $B_1, B_2, \cdots, B_n$ 中至多有 $k$ 个集合与 $A_1$ 的交集不空, 从而另外的 $n-k$ 个集合均与 $A_1$ 的交集为空集, 且这些集合的元素个数不小于 $n-k$. 由 $n>2 k$, 得 $n-k>k$. 于是我们有
$$
\begin{aligned}
|M| & =-\sum_{i=1}^n\left|B_i\right| \geqslant k \cdot k+(n-k) \cdot(n-k) \\
& =k^2+(n-k)^2 \\
& \geqslant \frac{1}{2}(k+(n-k))^2=\frac{n^2}{2} .
\end{aligned}
$$
综上所述,命题成立.
说明本例的解答应用了最小数原理.
关于最小数原理的应用, 我们将在后面作专门的介绍.
%%PROBLEM_END%%



%%PROBLEM_BEGIN%%
%%<PROBLEM>%%
例7. 设自然数集分划成 $r$ 个互不相交的子集: $\mathbf{N}=A_1 \cup A_2 \cup \cdots \cup A_r$. 求证其中必有某个子集 $A$, 它具有如下性质 $P$ : 存在 $m \in \mathbf{N}$, 使对任何正整数 $k$, 都能找到 $a_1, a_2, \cdots, a_k \in A$, 满足
$$
1 \leqslant a_{j+1}-a_j \leqslant m, j=1,2, \cdots, k-1 .
$$
%%<SOLUTION>%%
分析:显然具有性质 $P$ 的子集 $A$, 不可能是 $\mathbf{N}$ 的 $r$-分划中的有限集.
不妨设 $\mathbf{N}$ 的 $r$-分划中的无限集为 $A_1, A_2, \cdots, A_{r^{\prime}}$, 令 $B=A_2 \cup A_3 \cup \cdots \cup A_{r^{\prime}}$. 设 $b$ 是集合 $A_{r^{\prime}+1} \cup \cdots \cup A_{r-1} \cup A_r$ 中的最大自然数, 则 $b$ 以后的自然数都在 $N^{\prime}= A_1 \cup B$ 中, 即 $N^{\prime}$ 中存在任意有限长度的相继自然数段.
只需证明: 若 $A_1$ 不具有性质 $P$, 则 $B$ 必具有性质 $P$.
证明先证下面的引理:
引理设 $\mathbf{N}=A_1 \cup A_2 \cup \cdots \cup A_r$, 且 $A_1, A_2, \cdots, A_r$ 两两不相交.
若 $A_i \cup A_{i+1} \cup \cdots \cup A_r$ 包含任意有限长度的相继自然数段.
而 $A_i$ 不具有性质 $P$, 则 $A_{i+1} \cup \cdots \cup A_r$ 中必定含有任意有限长度的相继自然数段.
引理的证明若 $A_i$ 不具有性质 $P$, 则对于任给的 $m \in \mathbf{N}$, 存在 $k(m) \in \mathbf{N}$, 使得对于 $A_i$ 的任何 $k(m)$ 个数 $a_1<a_2<\cdots<a_{k(m)}$, 都可找到下标 $j \in\{1,2, \cdots, k(m)-1\}$, 数 $a_j$ 与 $a_{j+1}$ 之间至少有 $m$ 个相继自然数都不属于 $A_i$.
在 $A_i \cup A_{i+1} \cup \cdots \cup A_r$ 中选取一个长度为 $L=k(m) m$ 的相继自然数段.
若该段数中有 $k(m)$ 个数属于 $A_i$, 则因 $A_i$ 不具有性质 $P$, 故在这 $k(m)$ 个数中, 存在两个数 $a_j$ 与 $a_{j+1}$, 它们之间有 $m$ 个相继自然数都不属于 $A_i$, 当然就都属于 $A_{i+1} \cup \cdots \cup A_r$. 若选出的长度为 $L$ 的相继自然数段中属于 $A_i$ 的数少于 $k(m)$ 个, 则当把这 $L$ 个相继自然数依次分成 $k(m)$ 段, 每段恰有 $m$ 个数时, 由抽庶原理知其中必有一段 $m$ 个数中不含 $A_i$ 中的数, 当然都属于 $A_{i+1} \cup \cdots \cup A_r$. 由 $m \in \mathbf{N}$ 的任意性知引理成立.
回到原题的证明.
若 $A_1$ 具有性质 $P$, 则结论成立; 若 $A_1$ 不具有性质 $P$, 则由引理知 $A_2 \cup A_3 \cup \cdots \cup A_r$ 满足引理的条件.
若 $A_2$ 具有性质 $P$, 则结论成立; 若 $A_2$ 不具有性质 $P$, 则 $A_3 \cup \cdots \cup A_r$ 又满足引理的条件.
这样继续下去, 或者在某一步得出 $A_{i_0}$ 具有性质 $P$, 或者进行到最后, 得到 $A_r$ 含有任意有限长度的自然数段, 当然具有性质 $P$.
说明由上面的证明可以看出, 本例可作如下的加强:设 $M \subset \mathbf{N}, M$ 中存在任意有限长度的相继自然数段, 作 $M$ 的 $r$-分划: $A_1, A_2, \cdots, A_r$, 则在这些子集中必存在某个子集 $A$ 具有性质 $P$. 可以对 $r$ 进行归纳证明.
%%PROBLEM_END%%



%%PROBLEM_BEGIN%%
%%<PROBLEM>%%
例8. 将正整数集拆分为两个不相交的子集 $A 、 B$, 满足条件:
(1) $1 \in A$;
(2) $A$ 中没有两个不同的元素, 使它们的和形如 $2^k+2(k=0,1,2, \cdots)$;
(3)B 中也没有两个不同的元素, 其和具有上述形式.
证明: 这种拆分可以以唯一的方式实现, 并确定 1987, 1988, 1989 所属的子集.
%%<SOLUTION>%%
分析:对任意的自然数 $n$, 总存在非负整数 $h$, 使 $2^h \leqslant n<2^{h+1}$. 若 $m<n$, 则存在 $n+m=2^h+2$ 或 $n+m=2^{h+1}+2$ 两种可能, 只要将 $n$ 与 $m$ 置于不同的集合即可.
证明因为 $1+2=2^0+2$, 所以 $2 \in B$. 设对小于 $n$ 的数均有惟一的归属, 且满足条件 (1)、(2)、(3). 考虑 $n(n \geqslant 3)$, 总有自然数 $h$, 使
$$
2^h \leqslant n<2^{h+1} .
$$
若 $n=2^h, h>1$, 因 $2 \in B$, 故 $n \in A$. 这时, 对 $A$ 中任一元素 $m<n$,有
$$
2^h<n+m<2^{h+1} .
$$
而 $m \neq 2$, 所以 $n+m$ 不能写成 $2^h+2$ 的形式.
条件(1)、(2)、(3)成立.
若 $n=2^h+1$, 则 $1+n=2^h+2$, 故 $n \in B$. 这时, 对 $B$ 中任一元素 $m<n$,
$$
2^h+2<n+m<2^{h+1}+2 .
$$
条件(1)、(2)、(3)成立.
若 $n>2^h+1$, 则 $2^{h+1}+2-n<n$, 所以必须令 $n$ 与 $2^{n+1}+2-n$ 在不同集合中.
这时, 设 $m<n$, 且与 $n$ 在同一集合中, 则
$$
2^h+2<n+m<2^{h+2}+2,
$$
而 $n+m \neq 2^{h+1}+2$. 所以, 条件 (1)、(2)、(3)仍然成立.
这说明所说的拆分可以惟一地实现.
由于 $1987=2^{11}+2-63$, 而 $63=2^6+2-3,3=2^1+2-1 \in B$, 所以 $1987 \in B$.
同理可知 $1988 \in A, 1989 \in B$.
%%PROBLEM_END%%



%%PROBLEM_BEGIN%%
%%<PROBLEM>%%
例9. 平面上横纵坐标都为有理数的点称为有理点.
求证: 平面上的全体有理点可分成 3 个两两不相交的集合,满足条件:
(i)在以每个有理点为圆心的任一圆内一定包含 3 个点分属这 3 个集合;
(ii)在任何一条直线上都不可能有 3 个点分别属于这 3 个集合.
%%<SOLUTION>%%
分析:由有理数的稠密性知, 以坐标平面上任何点 $D$ 为圆心, 任何正数 $r$ 为半径的圆内都有无数多个有理点.
关键是怎样使这些点分属三个不同的集合, 这似乎比较容易办到.
如果直线 $a x+b y+c=0$ 上有 1 个以上的有理点, 则直线方程化简后的系数必皆为有理数, 这时直线上有无数多个有理点, 如果 3 -分划能使同一直线上的有理点至多属于分划的两个子集, 则问题获解.
证明显然, 任一有理点均可惟一地写成 $\left(\frac{u}{w}, \frac{v}{w}\right)$ 的形式, 其中 $u 、 v 、 w$ 都是整数, $w>0$ 且 $(u, v, w)=1$.
令
$$
\begin{aligned}
& A=\left\{\left(\frac{u}{w}, \frac{v}{w}\right) \mid 2 \nmid u\right\}, \\
& B=\left\{\left(\frac{u}{w}, \frac{v}{w}\right)|2| u, 2 \nmid v\right\}, \\
& C=\left\{\left(\frac{u}{w}, \frac{v}{w}\right)|2| u, 2 \mid v\right\} .
\end{aligned}
$$
让我们来验证这 3 个集合满足条件(i)和(ii).
设平面上的直线方程为
$$
a x+b y+c=0 .
$$
如果其上有两个不同的有理点 $\left(x_1, y_1\right)$ 和 $\left(x_2, y_2\right)$, 则有
$$
\left\{\begin{array}{l}
a x_1+b y_1+c=0, \\
a x_2+b y_2+c=0 .
\end{array}\right.
$$
如果 $c=0$, 则可取 $a 、 b$ 为有理数.
如果 $c \neq 0$, 不妨设 $c=1$, 于是, 从上面的联立方程中可解得 $a$ 和 $b$ 的值, 当然都是有理数.
再通分即知, 可以使 $a 、 b 、 c$ 都是整数且满足 $(a, b, c)=1$.
设有理点 $\left(\frac{u}{w}, \frac{v}{w}\right)$ 在直线 $a x+b y+c=0$ 上, 于是, 有
$$
L: a u+b v+c w=0 .
$$
(1) 先证集合 $A 、 B 、 C$ 满足条件(ii). 分三种情形.
(a) $2 \nmid c$. 若 $2|u, 2| v$, 则由 (1) 知 $2 \mid c w$, 从而 $2 \mid w$, 此与 $(u, v, w)=$ 1 矛盾.
所以,集合 $C$ 中的点都不能在直线 $L$ 上.
(b) $2 \mid c, 2 \nmid b$. 若 $2 \nmid v$, 则 $2 \nmid a u$, 从而 $2 \nmid u$. 因此, 集合 $B$ 中的点都不能在直线 $L$ 上.
(c) $2|c, 2| b$. 由(1)得 $2 \mid a u$. 又因 $(a, b, c)=1$, 故 $2 \nmid a$. 所以 $2 \mid u$. 这表明集合 $A$ 中的点都不在直线 $L$ 上.
综上可知, $A 、 B 、 C$ 这 3 个集合满足条件(ii).
(2) 再证满足条件 (i).
设 $D$ 是以有理点 $\left(\frac{u_0}{w_0}, \frac{w_0}{w_0}\right)$ 为圆心, 以 $r$ 为半径的圆.
取正整数 $k$, 使得
$$
2^k>\max \left\{w_0, \frac{1}{r}\left(\left|u_0\right|+\left|v_0\right|+1\right)\right\} .
$$
于是易验证, 下列 3 个有理点
$$
\begin{gathered}
\left(\frac{u_0 2^k+1}{w_0 2^k}, \frac{v_0 2^k}{w_0 2^k}\right) \in A,\left(\frac{u_0 2^k}{w_0 2^k}, \frac{v_0 2^k+1}{w_0 2^k}\right) \in B, \\
\left(\frac{u_0 2^k}{w_0\left(2^k+1\right)}, \frac{v_0 2^k}{w_0\left(2^k+1\right)}\right) \in C
\end{gathered}
$$
都在 $\odot D$ 的内部.
注意, 在上述 3 点中, $u_0 、 v_0 、 w_0$ 不一定互质.
但由于 $2^k> w_0$, 故约分之后不改变分子的奇偶性.
这表明条件(i)成立.
最后,我们来看一个非常特殊的集合分划的例子.
%%PROBLEM_END%%



%%PROBLEM_BEGIN%%
%%<PROBLEM>%%
例10. 设 $A=\{1,2, \cdots, 2002\}, M=\{1001,2003,3005\}$. 对 $A$ 的任一非空子集 $B$, 当 $B$ 中任意两数之和不属于 $M$ 时, 称 $B$ 为 $M$-自由集.
如果 $A= A_1 \cup A_2, A_1 \cap A_2=\varnothing$, 且 $A_1 、 A_2$ 均为 $M$-自由集, 那么称有序对 $\left(A_1, A_2\right)$ 为 $A$ 的一个 $M$-划分.
试求 $A$ 的所有 $M$-划分的个数.
%%<SOLUTION>%%
解:对 $m, n \in A$, 若 $m+n=1001$ 或 2003 或 3005 , 则称 $m$ 与 $n$ “有关”.
易知, 与 1 有关的数仅有 1000 和 2002 ,与 1000 和 2002 有关的都是 1 和 1003 , 与 1003 有关的为 1000 和 2002 .
将 $1,1003,1000,2002$ 分为两组 $\{1,1003\},\{1000,2002\}$, 其中一组中的数仅与另一组中的数有关, 我们将这样的两组叫做一个 “组对”. 同样可划分其他各组对 $\{2,1004\},\{999,2001\} ;\{3,1005\},\{998,2000\} ; \cdots ;\{500,1502\},\{501,1503\} ;\{1001\},\{1002\}$.
这样 $A$ 中的 2002 个数被分划成 501 个组对,共 1002 组.
由于任意数与且只与对应另一组有关, 所以, 若一组对中一组在 $A_1$ 中, 另一组必在 $A_2$ 中.
反之亦然, 且 $A_1$ 与 $A_2$ 中不再有有关的数.
故 $A$ 的 $M$ 一划分的个数为 $2^{501}$.
%%PROBLEM_END%%


