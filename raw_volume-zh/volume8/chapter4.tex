
%%TEXT_BEGIN%%
向量的加减法.
一、向量的有关概念.
1. 向量: 既有大小又有方向的量叫做向量.
记作 $\overrightarrow{A B}$, 其中 $A$ 是向量的起点, $B$ 是向量的终点.
也可以记作 $\vec{a}$.
2. 向量的模: 向量 $\overrightarrow{A B}$ 的大小亦即线段 $A B$ 的长度叫做向量的模, 记作 $|\overrightarrow{A B}|$ (向量 $\vec{a}$ 的模记作 $|\vec{a}|$ ). 向量的模又叫做向量的长度.
3. 单位向量: 模为 1 的向量叫做单位向量.
4. 零向量: 模为 0 的向量叫做零向量, 记作 $\overrightarrow{0}$. 零向量的方向任意, 所有的零向量都相等.
5. 平行向量: 方向相同或相反的向量叫做平行向量.
向量 $\vec{a}$ 和 $\vec{b}$ 平行记作 $\vec{a} / / \vec{b}$. 我们规定 $\overrightarrow{0}$ 与任一向量平行.
平行向量又叫做共线向量.
6. 相等向量: 模相等且方向相同的向量叫做相等向量.
向量 $\vec{a}$ 和 $\vec{b}$ 相等记作 $\vec{a}=\vec{b}$. 零向量与零向量相等.
任意两个相等的非零向量, 都可用同一条有向线段来表示,并且与有向线段的起点无关.
7. 相反向量: 与 $\vec{a}$ 模相等, 方向相反的向量, 叫做 $\vec{a}$ 的相反向量, 记作 $-\vec{a}$. $\vec{a}$ 和一 $\vec{a}$ 互为相反向量.
我们规定 $\overrightarrow{0}$ 的相反向量仍是 $\overrightarrow{0}$. 于是任一向量与它的相反向量之和是零向量, 即 $\vec{a}+(-\vec{a})=\overrightarrow{0}$.
8. 向量的夹角: 已知两个非零向量 $\vec{a}$ 和 $\vec{b}$, 作 $\overrightarrow{O A}=\vec{a}, \overrightarrow{O B}=\vec{b}$, 则 $\angle A O B=\theta\left(0^{\circ} \leqslant \theta \leqslant 180^{\circ}\right)$ 叫做向量 $\vec{a}$ 和 $\vec{b}$ 的夹角.
向量 $\vec{a}$ 和 $\vec{b}$ 的夹角也记作 $\langle\vec{a}, \vec{b}\rangle$.
二、向量的运算.
1. 向量的加法: 已知向量 $\vec{a}, \vec{b}$, 在平面内任取一点 $A$, 作 $\overrightarrow{A B}=\vec{a}, \overrightarrow{B C}= \vec{b}$, 则向量 $\overrightarrow{A C}$ 叫做向量 $\vec{a}$ 与 $\vec{b}$ 的和, 记作 $\vec{a}+\vec{b}$, 即 $\vec{a}+\vec{b}=\overrightarrow{A B}+\overrightarrow{B C}=\overrightarrow{A C}$.
求两个向量和的运算, 叫做向量的加法.
对于零向量和任一向量 $\vec{a}$, 有 $\vec{a}+\overrightarrow{0}=\overrightarrow{0}+\vec{a}=\vec{a}$.
以同一点 $A$ 为起点的两个已知向量 $\vec{a} 、 \vec{b}$ 为邻边作平行四边形 $A B C D$, 则以 $A$ 为起点的对角线 $\overrightarrow{A C}$ 就是 $\vec{a}$ 与 $\vec{b}$ 的和, 我们把这种作两个向量和的方法叫做向量加法的平行四边形法则.
而前面根据向量加法的定义得出的求向量和的方法, 称为向量加法的三角形法则.
这个法则可以推广到多个向量的求和一一多边形法则.
2. 向量的减法: 向量 $\vec{a}$ 加上 $\vec{b}$ 的相反向量, 叫做 $\vec{a}$ 与 $\vec{b}$ 的差.
即
$$
\vec{a}-\vec{b}=\vec{a}+(-\vec{b}) .
$$
求两个向量差的运算, 叫做向量的减法.
因为 $(\vec{a}-\vec{b})+\vec{b}=\vec{a}+(-\vec{b})+\vec{b}=\vec{a}+\overrightarrow{0}=\vec{a}$, 所以求 $\vec{a}-\vec{b}$ 就是求这样一个量, 它与 $\vec{b}$ 的和等于 $\vec{a}$. 因此可得如下求 $\vec{a}-\vec{b}$ 的作图方法.
已知 $\vec{a}$ 和 $\vec{b}$, 在平面内任取一点 $O$, 作 $\overrightarrow{O A}=\vec{a}, \overrightarrow{O B}=\vec{b}$, 则 $\overrightarrow{B A}=\vec{a}-\vec{b}$. 即 $\vec{a}-\vec{b}$ 可以表示为从向量 $\vec{b}$ 的终点指向向量 $\vec{a}$ 的终点的向量.
3. 实数与向量的积: 实数 $\lambda$ 与向量 $\vec{a}$ 的积是一个向量, 记作 $\lambda \vec{a}$, 它的模与方向规定如下:
(1) $|\lambda \vec{a}|=|\lambda||\vec{a}|$;
(2)当 $\lambda>0$ 时, $\lambda \vec{a}$ 的方向与 $\vec{a}$ 相同; 当 $\lambda<0$ 时, $\lambda \vec{a}$ 的方向与 $\vec{a}$ 相反; 当 $\lambda=0$ 时, $\lambda \vec{a}=\overrightarrow{0}$.
4. 向量的数量积: 已知两个非零向量 $\vec{a}$ 和 $\vec{b}$, 它们的夹角为 $\theta$, 我们把数量 $|\vec{a}||\vec{b}| \cos \theta$ 叫做 $\vec{a}$ 与 $\vec{b}$ 的数量积, 记作 $\vec{a} \cdot \vec{b}$, 即
$$
\vec{a} \cdot \vec{b}=|\vec{a}||\vec{b}| \cos \theta=|\vec{a}||\vec{b}| \cos \langle\vec{a}, \vec{b}\rangle .
$$
并且规定,零向量与任一向量的数量积为 0 . 向量的数量积又叫做内积.
设 $\overrightarrow{O A}=\vec{a}, \overrightarrow{O B}=\vec{b}$, 过点 $B$ 作 $B B_1$ 垂直于直线 $O A$, 垂足为 $B_1$, 则
$$
O B_1=|\vec{b}| \cos \theta
$$
$|\vec{b}| \cos \theta$ 叫做向量 $\vec{b}$ 在 $\vec{a}$ 方向上的投影, 当 $\theta$ 为锐角时, 它是正值; 当 $\theta$ 为钝角时, 它是负值; 当 $\theta$ 为直角时, 它是 0 . 当 $\theta=0^{\circ}$ 时, 它是 $|\vec{b}|$; 当 $\theta=180^{\circ}$ 时, 它是 $-|\vec{b}|$.
因此, 我们得到 $\vec{a} \cdot \vec{b}$ 的几何意义:数量积 $\vec{a} \cdot \vec{b}$ 等于 $\vec{a}$ 的长度 $|\vec{a}|$ 与 $\vec{b}$ 在 $\vec{a}$ 方向上的投影 $|\vec{b}| \cos \theta$ 的乘积.
三、向量的运算法则.
1. 加法的交换律: $\vec{a}+\vec{b}=\vec{b}+\vec{a}$;
加法的结合律: $(\vec{a}+\vec{b})+\vec{c}=\vec{a}+(\vec{b}+\vec{c})$.
2. $\lambda(\mu \vec{a})=(\lambda \mu) \vec{a}$,
分配律: $(\lambda+\mu) \vec{a}=\lambda \vec{a}+\mu \vec{a}$;
分配律: $\lambda(\vec{a}+\vec{b})=\lambda \vec{a}+\lambda \vec{b}$.
3. 数量积的交换律: $\vec{a} \cdot \vec{b}=\vec{b} \cdot \vec{a}, \vec{a} \cdot(\lambda \vec{b})=\lambda(\vec{a} \cdot \vec{b})$.
分配律: $(\vec{a}+\vec{b}) \cdot \vec{c}=\vec{a} \cdot \vec{c}+\vec{b} \cdot \vec{c}$.
4. 平方公式:
$$
\begin{aligned}
& (\vec{a}+\vec{b})^2=\vec{a}^2+2 \vec{a} \cdot \vec{b}+\vec{b}^2, \\
& (\vec{a}-\vec{b})^2=\vec{a}^2-2 \vec{a} \cdot \vec{b}+\vec{b}^2 .
\end{aligned}
$$
5. 平方差公式:
$$
(\vec{a}+\vec{b}) \cdot(\vec{a}-\vec{b})=\vec{a}^2-\vec{b}^2 .
$$
四、向量的共线与垂直.
1. 不共线的四点 $A 、 B 、 C 、 D$ 组成平行四边形的充要条件是 $\overrightarrow{A B}=\overrightarrow{C D}$ 或 $\overrightarrow{A B}=\overrightarrow{D C}$.
2. 向量 $\vec{b}$ 与非零向量 $\vec{a}$ 共线的充要条件是有且仅有一个实数 $\lambda$, 使得 $\vec{b}=\lambda \vec{a}$.
3. 两个非零向量 $\vec{a} 、 \vec{b}$ 垂直的充要条件是 $\vec{a} \cdot \vec{b}=0$.
4. 对于共线三点 $P_1 、 P_2 、 P$ 一定存在实数 $\lambda$, 使得 $\overrightarrow{P_1 P}=\lambda \vec{P} \overrightarrow{P_2}$, 若 $P_1$ 、 $P_2$ 是已知点, 则点 $P$ 位置由 $\lambda$ 确定, $\lambda>0$ 时, $P$ 为 $\overrightarrow{P_1 P_2}$ 内分点; $\lambda<0$ 时, $P$ 为 $\overrightarrow{P_1 P_2}$ 外分点; $|\lambda|=\frac{\left|\overrightarrow{P_1 P}\right|}{\left|\overrightarrow{P_2}\right|}$, 称 $\lambda$ 为 $P$ 分 $\overrightarrow{P_1 P_2}$ 所成的比.
并且有
$$
\overrightarrow{O P}=\frac{1}{1+\lambda} \overrightarrow{O P_1}+\frac{\lambda}{1+\lambda} \overrightarrow{O P_2}
$$
%%TEXT_END%%



%%PROBLEM_BEGIN%%
%%<PROBLEM>%%
例1. 已知 $\vec{a}=\{1,2\}, \vec{b}=\{-3,2\}$, 求实数 $k$ 使 $k \vec{a}+\vec{b}$ 与 $\vec{a}-3 \vec{b}$ 同方向或反方向.
%%<SOLUTION>%%
分析:与解 $k \vec{a}+\vec{b}=\{k-3,2 k+2\}, \vec{a}-3 \vec{b}=\{10,-4\}$.
由题意得 $k \vec{a}+\vec{b} / / \vec{a}-3 \vec{b}$, 所以 $\frac{k-3}{10}=\frac{2 k+2}{-4}$, 解得 $k=-\frac{1}{3}$.
%%PROBLEM_END%%



%%PROBLEM_BEGIN%%
%%<PROBLEM>%%
例2. 如图(<FilePath:./figures/fig-c4i1.png>), $P$ 点在 $\triangle A B C$ 所在平面上, $\overrightarrow{A P}=m \overrightarrow{A B}+n \overrightarrow{A C}$. 求证: $P$ 点在直线 $B C$ 上的充要条件是 $m+n=1$.
%%<SOLUTION>%%
分析:与解 $\overrightarrow{B P}=\overrightarrow{B A}+\overrightarrow{A P}=\overrightarrow{A P}-\overrightarrow{A B}= (m-1) \overrightarrow{A B}+n \overrightarrow{A C}$,
$$
\overrightarrow{B C}=\overrightarrow{B A}+\overrightarrow{A C}=-\overrightarrow{A B}+\overrightarrow{A C}
$$
(1) 若 $m+n=1$, 则 $\overrightarrow{B P}=-n \overrightarrow{A B}+n \overrightarrow{A C}=n \overrightarrow{B C}$, 故 $B 、 P 、 C$ 共线;
(2) 若 $B 、 P 、 C$ 共线, 则存在实数 $t$, 使 $\overrightarrow{B P}=t \overrightarrow{B C}$.
即 $(m-1) \overrightarrow{A B}+n \overrightarrow{A C}=t(-\overrightarrow{A B}+\overrightarrow{A C})$, 所以 $m-1=-t, n=t$.
从而 $(m-1)+n=0$, 即 $m+n=1$, 证毕.
%%PROBLEM_END%%



%%PROBLEM_BEGIN%%
%%<PROBLEM>%%
例3. $ \triangle A B C$ 中, 点 $O$ 为外心, $H$ 为垂心, 求证: $\overrightarrow{O H}=\overrightarrow{O A}+\overrightarrow{O B}+\overrightarrow{O C}$.
%%<SOLUTION>%%
分析:与解作直径 $\overrightarrow{B D}$, 连接 $D A 、 D C$, 有 $\overrightarrow{O B}=-\overrightarrow{O D}, D A \perp A B, D C \perp B C, A H \perp B C, C H \perp A B$.
故 $C H / / D A, A H / / D C$, 得 $A H C D$ 是平行四边形, 进而 $\overrightarrow{A H}=\overrightarrow{D C}$.
又 $\overrightarrow{D C}=\overrightarrow{O C}-\overrightarrow{O D}=\overrightarrow{O C}+\overrightarrow{O B}$, 得 $\overrightarrow{O H}=\overrightarrow{O A}+\overrightarrow{A H}=\overrightarrow{O A}+\overrightarrow{D C}=\overrightarrow{O A}+ \overrightarrow{O B}+\overrightarrow{O C}$, 证毕.
%%PROBLEM_END%%



%%PROBLEM_BEGIN%%
%%<PROBLEM>%%
例4. 设直线 $l: y=k x+m$ (其中 $k 、 m$ 为整数) 与椭圆 $\frac{x^2}{16}+\frac{y^2}{12}=1$ 交于不同两点 $A 、 B$, 与双曲线 $\frac{x^2}{4}-\frac{y^2}{12}=1$ 交于不同两点 $C 、 D$, 问是否存在直线 $l$, 使得向量 $\overrightarrow{A C}+\overrightarrow{B D}=\overrightarrow{0}$, 若存在, 指出这样的直线有多少条? 若不存在, 请说明理由.
%%<SOLUTION>%%
分析:与解由 $\left\{\begin{array}{l}y=k x+m, \\ \frac{x^2}{16}+\frac{y^2}{12}=1,\end{array}\right.$ 消去 $y$ 化简整理得
$$
\left(3+4 k^2\right) x^2+8 k m x+4 m^2-48=0 . 
$$
设 $A\left(x_1, y_1\right) 、 B\left(x_2, y_2\right)$, 则 $x_1+x_2=-\frac{8 k m}{3+4 k^2}$.
$$
\Delta_1=(8 k m)^2-4\left(3+4 k^2\right)\left(4 m^2-48\right)>0 . \label{eq1}
$$
由 $\left\{\begin{array}{l}y=k x+m, \\ \frac{x^2}{4}-\frac{y^2}{12}=1,\end{array}\right.$ 消去 $y$ 化简整理得
$$
\left(3-k^2\right) x^2-2 k m x-m^2-12=0 .
$$
设 $C\left(x_3, y_3\right) 、 D\left(x_4, y_4\right)$, 则 $x_3+x_4=\frac{2 k m}{3-k^2}$.
$$
\Delta_2=(-2 k m)^2+4\left(3-k^2\right)\left(m^2+12\right)>0 . \label{eq2}
$$
因为 $\overrightarrow{A C}+\overrightarrow{B D}=\overrightarrow{0}$, 所以 $\left(x_4-x_2\right)+\left(x_3-x_1\right)=0$, 此时 $\left(y_4-y_2\right)+ \left(y_3-y_1\right)==0$. 由 $x_1+x_2=x_3+x_4$ 得
$$
-\frac{8 k m}{3+4 k^2}=\frac{2 k m}{3-k^2} \text {. }
$$
所以 $2 k m=0$ 或 $-\frac{4}{3+4 k^2}=\frac{1}{3-k^2}$. 由上式解得 $k=0$ 或 $m=0$. 当 $k=0$ 时, 由 式\ref{eq1} 和 \ref{eq2} 得 $-2 \sqrt{3}<m<2 \sqrt{3}$. 因 $m$ 是整数,所以 $m$ 的值为 $-3,-2$, $-1,0,1,2,3$. 当 $m=0$ 时, 由 式\ref{eq1} 和 \ref{eq2} 得 $-\sqrt{3}<k<\sqrt{3}$. 因 $k$ 是整数,所以 $k$ 的值为 $-1,0,1$. 于是满足条件的直线共有 9 条.
%%PROBLEM_END%%



%%PROBLEM_BEGIN%%
%%<PROBLEM>%%
例5. 是否存在 4 个平面向量, 两两不共线, 其中任意两个向量之和与其余两个向量之和垂直?
%%<SOLUTION>%%
分析:与解在正 $\triangle A B C$ 中, $O$ 为内心, $P$ 为内切圆周上一点, 满足 $\overrightarrow{P A}$ 、 $\overrightarrow{P B} 、 \overrightarrow{P C} 、 \overrightarrow{P O}$ 两两不共线, 则
$$
\begin{aligned}
(\overrightarrow{P A}+\overrightarrow{P B}) \cdot(\overrightarrow{P C}+\overrightarrow{P O}) & =(\overrightarrow{P O}+\overrightarrow{O A}+\overrightarrow{P O}+\overrightarrow{O B}) \cdot(\overrightarrow{P O}+\overrightarrow{O C}+\overrightarrow{P O}) \\
& =(2 \overrightarrow{P O}+\overrightarrow{O A}+\overrightarrow{O B}) \cdot(2 \overrightarrow{P O}+\overrightarrow{O C}) \\
& =(2 \overrightarrow{P O}-\overrightarrow{O C}) \cdot(2 \overrightarrow{P O}+\overrightarrow{O C}) \\
& =4 \overrightarrow{P O}^2-\overrightarrow{O C}^2=4|P O|^2-|O C|^2=0,
\end{aligned}
$$
即 $(\overrightarrow{P A}+\overrightarrow{P B}) \perp(\overrightarrow{P C}+\overrightarrow{P O})$.
同理可证其他情况, 从而 $\overrightarrow{P A} 、 \overrightarrow{P B} 、 \overrightarrow{P C} 、 \overrightarrow{P O}$ 符合题意.
%%<REMARK>%%
注:本题属于构造性问题,利用向量和的定义将一个向量拆成多个向量和的技巧,望读者切实掌握.
%%PROBLEM_END%%



%%PROBLEM_BEGIN%%
%%<PROBLEM>%%
例6. 如图(<FilePath:./figures/fig-c4i2.png>), 在 $\triangle A B C$ 的内部任选点 $O$, 证明: $S_A \cdot \overrightarrow{O A}+S_B \cdot \overrightarrow{O B}+S_C \cdot \overrightarrow{O C}=\overrightarrow{0}$, 其中 $S_A$ 、 $S_B 、 S_C$ 分别为 $\triangle B C O 、 \triangle C A O 、 \triangle A B O$ 的面积.
%%<SOLUTION>%%
分析:与解如图(<FilePath:./figures/fig-c4i2.png>), 设 $\overrightarrow{O A} 、 \overrightarrow{O B} 、 \overrightarrow{O C}$ 上的单位向量分别为 $\overrightarrow{e_1} 、 \overrightarrow{e_2} 、 \overrightarrow{e_3}$, 作 $\triangle P Q R$, 使 $P Q / / O A, Q R / / O B$, $P R / / O C$, (如图(<FilePath:./figures/fig-c4i3.png>)) 则 $\sin \angle R=\sin \alpha, \sin \angle P= \sin \beta, \sin \angle Q=\sin \gamma$.
因为 $\overrightarrow{Q P}+\overrightarrow{P R}+\overrightarrow{R Q}=\overrightarrow{Q Q}=\overrightarrow{0}$, 所以 $|\overrightarrow{Q P}| \overrightarrow{e_1}+ |\overrightarrow{P R}| \overrightarrow{e_2}+|\overrightarrow{R Q}| \overrightarrow{e_3}=0$.
设 $R$ 为 $\triangle P Q R$ 的外接圆半径, 则
$2 R \sin \alpha \cdot \overrightarrow{e_1}+2 R \sin \beta \cdot \overrightarrow{e_2}+2 R \sin \gamma \cdot \overrightarrow{e_3}=\overrightarrow{0}$, $\sin \alpha \cdot \overrightarrow{e_1}+\sin \beta \cdot \overrightarrow{e_2}+\sin \gamma \cdot \overrightarrow{e_3}=\overrightarrow{0}$.
$$
\begin{gathered}
\frac{1}{2}|O A| \cdot|O B| \cdot|O C| \cdot \sin \alpha \overrightarrow{e_1}+\frac{1}{2}|O A| \cdot|O B| \cdot|O C| \cdot \sin \beta \overrightarrow{e_2} \\
+\frac{1}{2}|O A| \cdot|O B| \cdot|O C| \cdot \sin \gamma \overrightarrow{e_3}=\overrightarrow{0}
\end{gathered}
$$
$$
\begin{gathered}
\left(\frac{1}{2}|O B| \cdot|O C| \sin \alpha\right)\left(|O A| \cdot \overrightarrow{e_1}\right)+\left(\frac{1}{2}|O A| \cdot|O C| \sin \beta\right)\left(|O B| \cdot \overrightarrow{e_2}\right) \\
+\left(\frac{1}{2}|O A| \cdot|O B| \sin \gamma\right)\left(|O C| \cdot \overrightarrow{e_3}\right)=\overrightarrow{0}
\end{gathered}
$$
所以 $S_A \cdot \overrightarrow{O A}+S_B \cdot \overrightarrow{O B}+S_C \cdot \overrightarrow{O C}=\overrightarrow{0}$, 证毕.
%%PROBLEM_END%%



%%PROBLEM_BEGIN%%
%%<PROBLEM>%%
例7. 设 $O$ 是 $\triangle A B C$ 内部一点.
证明: 存在正整数 $p 、 q 、 r$, 使得
$$
|p \cdot \overrightarrow{O A}+q \cdot \overrightarrow{O B}+r \cdot \overrightarrow{O C}|<\frac{1}{2007} \text {. }
$$
%%<SOLUTION>%%
分析:与解由条件可知存在正实数 $\beta 、 \gamma$ 使得 $\overrightarrow{O A}+\beta \overrightarrow{O B}+\gamma \overrightarrow{O C}=\overrightarrow{0}$, 于是对任意正整数 $k$, 都有 $k \overrightarrow{O A}+k \beta \overrightarrow{O B}+k \gamma \overrightarrow{O C}=\overrightarrow{0}$, 记 $m(k)=[k \beta], n(k)= [k \gamma]$, 这里 $[x]$ 表示不超过实数 $x$ 的最大整数, $\{x\}=x-[x]$.
利用 $\beta 、 \gamma$ 都是正实数可知 $m(k T)$ 和 $n(k T)$ 都是关于正整数 $k$ 的严格递增数列, 这里 $T$ 是某个大于 $\max \left\{\frac{1}{\beta}, \frac{1}{\gamma}\right\}$ 的正整数.
因此
$$
\begin{aligned}
|k T \cdot \overrightarrow{O A}+m(k T) \cdot \overrightarrow{O B}+n(k T) \cdot \overrightarrow{O C}| & =|-\{k T \beta\} \overrightarrow{O B}-\{k T \gamma\} \overrightarrow{O C}| \\
& \leqslant\{k T \beta\}|\overrightarrow{O B}|+\{k T \gamma\}|\overrightarrow{O C}| \\
& \leqslant|\overrightarrow{O B}|+|\overrightarrow{O C}| .
\end{aligned}
$$
这表明有无穷多个向量 $k T \cdot \overrightarrow{O A}+m(k T) \cdot \overrightarrow{O B}+n(k T) \cdot \overrightarrow{O C}$ 的终点落在一个以 $O$ 为圆心, $|\overrightarrow{O B}|+|\overrightarrow{O C}|$ 为半径的圆内, 因此, 其中必有两个向量的终点之间的距离小于 $\frac{1}{2007}$, 也就是说, 这两个向量的差的模长小于 $\frac{1}{2007}$. 即存在正整数 $k_1<k_2$, 使得
$$
\begin{gathered}
\mid\left(k_2 T \cdot \overrightarrow{O A}+m\left(k_2 T\right) \cdot \overrightarrow{O B}+n\left(k_2 T\right) \cdot \overrightarrow{O C}\right)- \\
\left(k_1 T \cdot \overrightarrow{O A}+m\left(k_1 T\right) \cdot \overrightarrow{O B}+n\left(k_1 T\right) \cdot \overrightarrow{O C}\right) \mid<\frac{1}{2007} .
\end{gathered}
$$
于是, 令 $p=\left(k_2-k_1\right) T, q=m\left(k_2 T\right)-m\left(k_1 T\right), r=n\left(k_2 T\right)-n\left(k_1 T\right)$, 结合 $T$ 与 $m(k T) 、 n(k T)$ 的单调性可知 $p 、 q 、 r$ 都是正整数,证毕.
%%PROBLEM_END%%


