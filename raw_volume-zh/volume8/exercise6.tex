
%%PROBLEM_BEGIN%%
%%<PROBLEM>%%
问题1. 在长方体 $O A B C-O_1 A_1 B_1 C_1$ 中, $|O A|=2,|A B|=3,\left|A A_1\right|=2, E$
是 $B C$ 中点.
(1) 求异面直线 $A O_1$ 和 $B_1 E$ 所成的角;
(2) 作 $O_1 D \perp A C$ 于点 $D$, 求向量 $\overrightarrow{O_1 D}$.
%%<SOLUTION>%%
如图(<FilePath:./figures/fig-c6a1.png>), 建立空间直角坐标系, 则 $A(2,0$, $0), O_1(0,0,2), B(2,3,0), C(0,3,0), B_1 (2,3,2), E(1,3,0)$.
(1) $\overrightarrow{A O_1}=\{-2,0,2\}, \overrightarrow{B_1 E}=\{-1,0$, $-2\}$.
$$
\cos \theta=\frac{\overrightarrow{A O_1} \cdot \overrightarrow{B_1 E}}{\left|\overrightarrow{A O_1}\right| \cdot\left|\overrightarrow{B_1 E}\right|}=\frac{2-4}{2 \sqrt{2} \times \sqrt{5}}=
$$
$-\frac{\sqrt{10}}{10}$, 所以异面直线 $A O_1$ 和 $B_1 E$ 所成的角为
$\arccos \frac{\sqrt{10}}{10}$
(2) $\overrightarrow{A C}=\{-2,3,0\}$, 在 $x O y$ 坐标系中, 直线 $A C$ 的方程为 $\frac{x}{2}+\frac{y}{3}=1$, 则 $D\left(t, \frac{6-3 t}{2}, 0\right), t \in \mathbf{R}$, 所以 $\overrightarrow{O_1 D}=\left\{t, \frac{6-3 t}{2},-2\right\}$.
又因为 $O_1 D \perp A C$, 所以 $\overrightarrow{O_1 D} \cdot \overrightarrow{A C}=0$, 故 $-2 t+\frac{18-9 t}{2}=0$, 即 $t=\frac{18}{13}$.
所以 $\overrightarrow{O_1 D}=\left\{\frac{18}{13}, \frac{12}{13},-2\right\}$.
%%PROBLEM_END%%



%%PROBLEM_BEGIN%%
%%<PROBLEM>%%
问题2. 已知非零向量 $\vec{a}$ 和 $\vec{b}$ 不平行, 且 $\vec{m}=3 \vec{a}-2 \vec{b}, \vec{n}=2 \vec{a}+k \vec{b}$, 问是否存在实数 $k$, 使 $\vec{m}$ 和 $\vec{n}$ 互相平行? 若存在求出 $k$ 的值, 若不存在, 说明理由.
%%<SOLUTION>%%
因为 $\vec{a}$ 和 $\vec{b}$ 不平行, 且 $|\vec{a}| \neq 0,|\vec{b}| \neq 0$, 所以 $\vec{a}$ 和 $\vec{b}$ 可以确定平面的一个坐标系 (有可能不是直角坐标系), 且分别以 $\vec{a}$ 和 $\vec{b}$ 的方向作为两坐标轴的正方向, 以 $|\vec{a}|$ 和 $|\vec{b}|$ 作为两坐标轴的单位长度, 则
$$
\vec{m}=\{3,-2\}, \vec{n}=\{2, k\} .
$$
如果 $\vec{m}$ 和 $\vec{n}$ 互相平行, 则 $\frac{2}{3}=\frac{k}{-2}$, 即 $k=-\frac{4}{3}$.
%%PROBLEM_END%%



%%PROBLEM_BEGIN%%
%%<PROBLEM>%%
问题3. 如图(<FilePath:./figures/fig-c6p3.png>), 直四棱柱 $A B C D-A_1 B_1 C_1 D_1$ 中, $A D / / B C$, 且 $\angle C B A=90^{\circ}, B C=2, A D=6, A B= 2, A A_1=4$, 且 $C E=C_1 E, C G=D G, A F= \frac{1}{2} F D$, 求:
(1) $E F$ 和 $D_1 G$ 所成角的大小;
(2) $E F$ 与平面 $A_1 C_1$ 所成角的大小;
(3) 二面角 $E-F G-D_1$ 的平面角的大小.
%%<SOLUTION>%%
如图(<FilePath:./figures/fig-c6a3.png>), 建立空间直角坐标系, 则 $E(2,2$, 2), $F(0,2,0), D_1(0,6,4), G(1,4,0)$.
(1) $\overrightarrow{E F}=\{-2,0,-2\}, \overrightarrow{D_1 G}=\{1,-2$, $-4\}, \cos \alpha=\frac{\overrightarrow{E F} \cdot \overrightarrow{D_1 G}}{|\overrightarrow{E F}| \cdot\left|\overrightarrow{D_1 G}\right|}=\frac{-2+8}{2 \sqrt{2} \cdot \sqrt{21}}= \frac{\sqrt{21}}{14}$
所以 $E F$ 和 $D_1 G$ 所成的角为 $\arccos \frac{\sqrt{21}}{14}$.
(2)显然 $\overrightarrow{p_0}=\{0,0,-1\}$ 是平面 $A_1 C_1$ 的一个单位法向量.
所以 $\cos \beta=\frac{\overrightarrow{E F} \cdot \overrightarrow{p_0}}{|\overrightarrow{E F}|}=\frac{2}{2 \sqrt{2}}=\frac{\sqrt{2}}{2}$, 故 $\frac{\pi}{2}-\beta=\frac{\pi}{4}$.
所以 $E F$ 与平面 $A_1 C_1$ 所成的角为 $\frac{\pi}{4}$.
(3) 因为平面 $x O y$ 内 $F G$ 的直线方程为 $2 x-y+2=0$, 设 $P 、 Q$ 在直线 $F G$ 上, 且 $E P \perp F G, D_1 Q \perp F G$, 则可设 $P\left(t_1, 2 t_1+2,0\right), Q\left(t_2, 2 t_2+2\right.$, $0)\left(t_1 、 t_2 \in \mathbf{R}\right)$.
因为 $\left\{\begin{array}{l}\overrightarrow{E P} \cdot \overrightarrow{F G}=0, \\ \overrightarrow{D_1 Q} \cdot \overrightarrow{F G}=0,\end{array}\right.$ 所以 $\left\{\begin{array}{l}\left\{t_1-2,2 t_1,-2\right\} \cdot\{1,2,0\}=0, \\ \left\{t_2, 2 t_2-4,-4\right\} \cdot\{1,2,0\}=0 .\end{array}\right.$
即 $\left\{\begin{array}{l}t_1-2+4 t_1=0, \\ t_2+4 t_2-8=0 .\end{array}\right.$ 解方程, 得 $\left\{\begin{array}{l}t_1=\frac{2}{5}, \\ t_2=\frac{8}{5} .\end{array}\right.$
所以 $\overrightarrow{E P}=\left\{-\frac{8}{5}, \frac{4}{5},-2\right\}, \overrightarrow{D_1 Q}=\left\{\frac{8}{5},-\frac{4}{5},-4\right\}$, 故
$$
\cos \theta=\frac{\overrightarrow{E P} \cdot \overrightarrow{D_1 Q}}{|\overrightarrow{E P}| \cdot\left|\overrightarrow{D_1 Q}\right|}=\frac{-\frac{64}{25}-\frac{16}{25}+8}{\frac{3 \sqrt{20}}{5} \cdot \frac{4 \sqrt{30}}{5}}=\frac{\sqrt{6}}{6} \text {. }
$$
所以二面角 $E-F G-D_1$ 的大小为 $\arccos \frac{\sqrt{6}}{6}$.
%%PROBLEM_END%%



%%PROBLEM_BEGIN%%
%%<PROBLEM>%%
问题4. 对于 $n$ 个向量 $\overrightarrow{a_1}, \overrightarrow{a_2}, \cdots, \overrightarrow{a_n}$, 如存在不全为零的 $n$ 个实数 $k_1, k_2, \cdots, k_n$, 使 $k_1 \overrightarrow{a_1}+k_2 \overrightarrow{a_2}+\cdots+k_n \overrightarrow{a_n}=\overrightarrow{0}$ 成立, 则称这 $n$ 个向量 $\overrightarrow{a_1}, \overrightarrow{a_2}, \cdots, \overrightarrow{a_n}$ 线性相关; 反之如果 $k_1 \overrightarrow{a_1}+k_2 \overrightarrow{a_2}+\cdots+k_n \overrightarrow{a_n}=\overrightarrow{0}$ 当且仅当 $k_1=k_2=\cdots= k_n=0$ 时成立,则称这 $n$ 个向量 $\overrightarrow{a_1}, \overrightarrow{a_2}, \cdots, \overrightarrow{a_n}$ 线性无关.
在棱长为 1 的立方体 $A B C D-A_1 B_1 C_1 D_1$ 中, $N$ 为正方形 $A_1 B_1 C_1 D_1$ 的中心.
(1) 判断 $\overrightarrow{A A_1}$ 和 $\overrightarrow{B C} ; \overrightarrow{A_1 N}$ 和 $\overrightarrow{C A} ; \overrightarrow{A_1 N}$ 和 $\overrightarrow{B_1 D_1}$ 是否线性相关?
(2) 判断 $\overrightarrow{A A_1} 、 \overrightarrow{A B}$ 和 $\overrightarrow{B C} ; \overrightarrow{A_1 D_1} 、 \overrightarrow{A_1 B_1}$ 和 $\overrightarrow{A C}$ 这两组向量是否线性相关?
(3) 说明 $\vec{a} 、 \vec{b}$ 线性相关和 $\vec{a} 、 \vec{b} 、 \vec{c}$ 线性相关的几何意义.
%%<SOLUTION>%%
如图(<FilePath:./figures/fig-c6a4.png>), 建立空间直角坐标系, 则 $A(0,0,0)$, $A_1(0,0,1), B(1,0,0), C(1,1,0), N\left(\frac{1}{2}, \frac{1}{2}, 1\right)$, $D(0,1,0), D_1(\theta, 1,1), B_1(1,0,1)$.
(1) $\overrightarrow{A A_1}=\{0,0,1\}, \overrightarrow{B C}=\{0,1,0\}$.
设 $k_1 、 k_2 \in \mathbf{R}$, 有 $k_1 \overrightarrow{A A_1}+k_2 \overrightarrow{B C}=\overrightarrow{0}$, 即 $\left\{0, k_2\right.$, $\left.k_1\right\}=\overrightarrow{0}$.
所以 $k_1=k_2=0$,所以 $\overrightarrow{A A_1}$ 和 $\overrightarrow{B C}$ 线性无关, $\overrightarrow{A A_1}$
和 $\overrightarrow{B C}$ 不平行.
$$
\overrightarrow{A_1} \vec{N}=\left\{\frac{1}{2}, \frac{1}{2}, 0\right\}, \overrightarrow{C A}=\{-1,-1,0\}
$$
设 $k_1 、 k_2 \in \mathbf{R}$, 有 $k_1 \overrightarrow{A_1} \vec{N}+k_2 \overrightarrow{C A}=\overrightarrow{0}$, 即 $\left\{\frac{1}{2} k_1-k_2, \frac{1}{2} k_1-k_2, 0\right\}=\overrightarrow{0}$. 此时存在不全为零的实数 $k_1=2, k_2=1$ 有 $k_1 \overrightarrow{A_1 N}+k_2 \overrightarrow{C A}=\overrightarrow{0}$, 所以 $\overrightarrow{A_1 N}$ 和 $\overrightarrow{C A}$ 线性相关, $\overrightarrow{A_1 N}$ 和 $\overrightarrow{C A}$ 平行.
$$
\overrightarrow{A_1 N}=\left\{\frac{1}{2}, \frac{1}{2}, 0\right\}, \overrightarrow{B_1 D_1}=\{-1,1,0\} .
$$
设 $k_1 、 k_2 \in \mathbf{R}$, 有 $k_1 \overrightarrow{A_1 N}+k_2 \overrightarrow{B_1 D_1}=\overrightarrow{0}$, 即 $\left\{\frac{k_1}{2}-k_2, \frac{k_1}{2}+k_2, 0\right\}=\overrightarrow{0}$.
所以 $\left\{\begin{array}{l}\frac{k_1}{2}-k_2=0, \\ \frac{k_1}{2}+k_2=0 .\end{array}\right.$ 即 $k_1=k_2=0$.
所以 $\overrightarrow{A_1 N}$ 和 $\overrightarrow{B_1 D_1}$ 线性无关, $\overrightarrow{A_1 N}$ 和 $\overrightarrow{B_1 D_1}$ 不平行.
(2) $\overrightarrow{A A_1}=\{0,0,1\}, \overrightarrow{A B}=\{1,0,0\}, \overrightarrow{B C}=\{0,1,0\}$.
设 $k_1 、 k_2 、 k_3 \in \mathbf{R}$, 有 $k_1 \overrightarrow{A A_1}+k_2 \overrightarrow{A B}+k_3 \overrightarrow{B C}=\overrightarrow{0}$, 即 $\left\{k_2, k_3, k_1\right\}=\overrightarrow{0}$.
所以 $k_1=k_2=k_3=0$, 所以 $\overrightarrow{A A_1} 、 \overrightarrow{A B} 、 \overrightarrow{B C}$ 线性无关, $\overrightarrow{A A_1} 、 \overrightarrow{A B} 、 \overrightarrow{B C}$ 不共面.
$$
\overrightarrow{A_1 D_1}=\{0,1,0\}, \overrightarrow{A_1 B_1}=\{1,0,0\}, \overrightarrow{A C}=\{1,1,0\} .
$$
设 $k_1 、 k_2 、 k_3 \in \mathbf{R}$, 有 $k_1 \overrightarrow{A_1 D_1}+k_2 \overrightarrow{A_1 B_1}+k_3 \overrightarrow{A C}=\overrightarrow{0}$, 即 $\left\{k_2+k_3, k_1+\right. \left.k_3, 0\right\}=\overrightarrow{0}$.
所以 $\left\{\begin{array}{l}k_2+k_3=0, \\ k_1+k_3=0 .\end{array}\right.$ 可以取 $k_1=k_2=1, k_3=-1$.
所以存在不全为零的系数 $k_1=k_2=1, k_3=-1$ 有 $k_1 \overrightarrow{A_1 D_1}+k_2 \overrightarrow{A_1 B_1}+ k_3 \overrightarrow{A C}=\overrightarrow{0}$.
所以 $\overrightarrow{A_1 D_1} 、 \overrightarrow{A_1 B_1} 、 \overrightarrow{A C}$ 线性相关, $\overrightarrow{A_1 D_1} 、 \overrightarrow{A_1 B_1} 、 \overrightarrow{A C}$ 可以共面.
(3) $\vec{a} 、 \vec{b}$ 线性相关当且仅当 $\vec{a}$ 和 $\vec{b}$ 平行, $\vec{a} 、 \vec{b} 、 \vec{c}$ 线性相关当且仅当 $\vec{a} 、 \vec{b} 、 \vec{c}$ 共面.
%%PROBLEM_END%%


