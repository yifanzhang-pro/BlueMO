
%%TEXT_BEGIN%%
数学归纳法.
与正整数 $n$ 有关的命题常常用到数学归纳法.
(1)数学归纳法的基本形式(第一数学归纳法)是:
设 $P(n)$ 是一个含正整数 $n$ 的命题, 如果
( I ) $P(1)$ 成立;
(II )在 $P(k)$ 成立的假设下, 可证明 $P(k+1)$ 成立,
那么 $P(n)$ 对任意正整数 $n$ 成立.
通常我们将步骤 ( I) 称为归纳奠基, 将步骤 (II) 称为归纳过渡.
两者不可缺其一.
显然,第一数学归纳法可以推广为:
设 $p(n)$ 是一个含有正整数 $n$ 的命题, 如果
( I ) $p(n)$, 当 $n=n_0$ 时成立;
(II )在 $p(k)\left(k \geqslant n_0\right)$ 成立的假定下, 可以证明 $p(k+1)$ 成立, 那么 $p(n)$ 对一切大于或等于 $n_0$ 的正整数 $n$ 都成立.
(2) 第二数学归纳法: 设 $p(n)$ 是一个含正整数 $n$ 的命题,如果
( I ) $P(1)$ 成立;
(II) 在 $p(m)$ 对于所有适合 $m \leqslant k$ 的正整数 $m$ 成立的假定下, 可以证明 $p(k+1)$ 成立,
那么 $p(n)$ 对任意正整数 $n$ 都成立.
第二数学归纳法也有类似的推广, 即使命题 $p(n)$ 成立的起点可用某个正整数 $n_0$ 代替.
(3)反向数学归纳法, 又称倒推数学归纳法, 是法国著名数学家柯西首先使用的.
柯西利用反向数学归纳法证明了: $n$ 个正数的算术平均大于或等于这 $n$ 个正数的几何平均.
下面给出反向数学归纳法:
( I ) $p(n)$ 对无限多个正整数 $n$ 成立;
(II)假设 $p(k+1)$ 成立, 可推出 $p(k)$ 也成立,
那么 $p(n)$ 对一切正整数 $n$ 都成立.
反向数学归纳法也可以推广为:
设 $p(n)$ 是一个含有正整数 $n$ 的命题, 如果
( I ) $p(n)$ 对某个正整数 $m_0\left(m_0 \geqslant 1\right)$ 成立;
(II)假设 $p(k+1)$ 成立, 可推出 $p(k)$ 也成立,
那么 $p(n)$ 对一一切不大于 $m_0$ 的正整数 $n$ 都成立.
(4) 双参数归纳法.
在证明与两个独立的正整数有关的命题 $p(n, m)$ 时, 可以用如下形式进行:
( I ) 证明 $p(1, m)$ 对任意正整数 $m$ 成立, $p(n, 1)$ 对任意正整数 $n$ 成立;
(II)假设 $p(n+1, m)$ 和 $p(n, m+1)$ 成立, 由此推出 $p(n+1, m+1)$ 成立, 则对所有的正整数 $n, m, p(n, m)$ 成立.
数学归纳法的应用十分广泛, 而在很多情况下, 数学归纳法是以一些常见的"变体"实施的, 另外也涉及一些技巧, 如主动加强命题、灵活选取起点、 灵活选取跨度等等.
%%TEXT_END%%



%%TEXT_BEGIN%%
归纳构造实际上是借助数学归纳法的原理, 将整个构造过程逐层分解为一些比较简单又具有统一规律的构造环节, 这是一种典型的构造实现形式.
很多人用"推倒一系列多米诺骨牌"来形象地比喻"证明与正整数 $n$ 有关的命题"的过程.
例如在第一数学归纳法中, 将 "归纳奠基"看成推倒第一块骨牌, 将"归纳过渡"比作某种使之后的每块骨牌依次被前一块骨牌所推倒的连锁反应.
最后每块骨牌都能被推倒代表命题证完.
这里不妨借此进一步比喻数学归纳法的一些变通形式的证明流程:
第二数学归纳法: 先推倒第一块骨牌, 再依次对每个 $n$, 借助前 $n$ 块骨牌的效应推倒第 $n+1$ 块骨牌,直至所有骨牌被推倒;
反向归纳法:一要保证每块骨牌能反过来推倒所有在它之前的骨牌, 二要说明可以推倒无穷块骨牌,从而无论多大号码的骨牌总能被推倒;
双参数归纳法或螺旋归纳法: 在需要推倒的骨牌边放一系列辅助骨牌, 先推倒起点处的骨牌, 再把所有骨牌和辅助骨牌全部推倒 (例如每块原有的骨牌推倒旁边的辅助骨牌,而辅助骨牌推倒下面的骨牌);
后置起点: 先独立推倒前 $k$ 块骨牌, 再说明第 $k$ 块骨牌可以用来推倒后面所有的骨牌;
前置起点: 放一些骨牌作"引子" 来推倒一切骨牌 (应当注意, 在用数学归纳法证明一个关于确定数 $n_0$ 的命题时, 可以理解成设置了长为 $n_0-1$ 的 "引子");
选取跨度 $k\left(k \geqslant 2, k \in \mathbf{N}^*\right)$ : 把骨牌按模 $k$ 分类, 1 排变 $k$ 排, 每排中推倒第一块骨牌, 再说明该骨牌可以推倒这一排中的一切骨牌;
加强命题: 适当 "加重" 每块骨牌分量, 稍稍费力地推倒第一块骨牌, 再利用每块骨牌所加重的分量给予后面骨牌更强的作用, 使全部骨牌推倒.
归根结底我们的目的是推倒所有骨牌, 若不能独立推倒, 那么总应当按一定秩序推, 而不应无所适从乱做一气.
就此而言, 各种形式的数学归纳法恰是为证明命题而设计的若干实用的秩序.
%%TEXT_END%%



%%PROBLEM_BEGIN%%
%%<PROBLEM>%%
例1. 设 $\left\{x_n\right\}$ 是一实数列, 且对任一非负整数 $n$, 满足
$$
x_0^3+x_1^3+\cdots+x_n^3=\left(x_0+x_1+\cdots+x_n\right)^2 .
$$
求证: 对所有非负整数 $n$, 存在整数 $m$, 使得 $x_0+x_1+\cdots+x_n= \frac{m(m+1)}{2}$.
%%<SOLUTION>%%
证明:我们用数学归纳法来证明结论.
当 $n=0$ 时, 由题设, $x_0^3=x_0^2$, 得 $x_0=0$ 或 $1$. 这时可取 $m=0$ 或 $1$ , 从而当 $n=0$ 时结论成立.
设 $n=k$ 时结论成立, 即当 $x_0^3+x_1^3+\cdots+x_k^3=\left(x_0+x_1+\cdots+x_k\right)^2$ 时, 存在整数 $m$, 使得 $x_0+x_1+\cdots+x_k=\frac{m(m+1)}{2}$.
为方便书写, 记 $\frac{m(m+1)}{2}=c$, 则 $x_0^3+x_1^3+\cdots+x_k^3=c^2$.
对于 $x_{k+1}$, 如果 $x_0^3-+x_1^3+\cdots+x_k^3+x_{k+1}^3=\left(x_0+x_1+\cdots+x_k+x_{k+1}\right)^2$, 那么 $c^2+x_{k+1}^3=\left(c+x_{k+1}\right)^2$, 所以 $x_{k+1}\left(x_{k+1}^2-x_{k+1}-m(m+1)\right)=0$, 解得 $x_{k+1}=0,-m, m+1$.
当 $x_{k+1}=0$ 时, $x_0+x_1+\cdots+x_k+x_{k+1}=\frac{m(m+1)}{2}$;
当 $x_{k+1}=-m$ 时, $x_0+x_1+\cdots+x_k+x_{k+1}=\frac{m(m+1)}{2}-m=\frac{m(m-1)}{2}$;
当 $x_{k+1}=m+1$ 时, $x_0+x_1+\cdots+x_k+x_{k+1}=\frac{m(m+1)}{2}+m+1=$
$$
\frac{(m+1)(m+2)}{2}
$$
即结论在 $n=k+1$ 时也成立, 从而对一切非负整数 $n$, 结论成立.
%%<REMARK>%%
注:运用数学归纳法时, 证明的核心与难点多数情况下在于如何实现归纳过渡.
此时应设法使 $P(k)$ 与 $P(k+1)$ 的关系充分显露出来, 根据题目的特点, 或是从 $P(k)$ 出发, 过渡到 $P(k+1)$, 或是先从 $P(k+1)$ 人手, 从中分离出 $P(k)$ 的形式, 某些时候还需对一些代数式和命题进行反复变形转化.
总之应当创设条件,使归纳假设得以充分利用.
%%PROBLEM_END%%



%%PROBLEM_BEGIN%%
%%<PROBLEM>%%
例2. 设有 $2^n$ 个球分成了许多堆, 我们可以任意选取甲、乙两堆按如下规则挪动: 若甲堆中的球数 $p$ 不小于乙堆中的球数 $q$, 则从甲堆中拿出 $q$ 球放人乙堆, 这算是挪动一次.
证明可以经过有限次挪动把所有球并成一堆.
%%<SOLUTION>%%
证明:对 $n$ 用数学归纳法.
当 $n=1$ 时, 只有两个球, 至多挪动一次即可,结论成立.
设 $n=k$ 时结论成立.
在 $n=k+1$ 的情形下, 首先注意到总球数为偶数, 所以球数为奇数的堆必有偶数个, 先将它们两两配对, 并在每对的两堆球之间进行一次挪动, 其中对每对球堆, 设球数分别为奇数 $p$ 和 $q(p \geqslant q)$, 那么挪动后两堆球的个数分别为偶数 $p-q, 2 q$. 因此一系列挪动后可使所有这些堆中的球数都变为偶数 (球数变为 0 个的球堆就消失了). 这时将每堆中的球两两捆绑起来视为一个"球", 于是总 "球" 数变为 $2^k$, 由归纳假设知可以在有限步挪动后全部并为一堆,所以 $n=k+1$ 的情形结论也成立.
由数学归纳法得: 对一切正整数 $n$ 结论成立.
%%<REMARK>%%
注:本题中 $P(k)$ 与 $P(k+1)$ 的关系是球数的 2 倍关系.
如果我们能在 $n =k+1$ 的情形下找到某些球堆, 它们的总球数为 $2^k$, 那么剩下的球堆的总球数也是 $2^k$, 归纳过渡自然能实现, 遗憾的是这种有利情况并不必然发生.
如何利用好归纳假设实现过渡呢? 上述解法中, 我们是先通过操作调整使所有球堆都含有偶数个球, 然后只需将 $2^{k+1}$ 个球两两捆绑看成一些新的"球", 这样 "球" 数就回到了 $2^k$ 个, 从而找到了 $P(k)$ 和 $P(k+1)$ 的过渡.
特别注意本题中命题 $P(k)$ 所涉及的对象是球, 而归纳假设则是作用在新的载体一一捆绑以后的 $2^k$ 个"球"上.
%%PROBLEM_END%%



%%PROBLEM_BEGIN%%
%%<PROBLEM>%%
例3. 求证:第 $n$ 个素数 (将素数从小到大编上序号, 2 算作第一个素数) $p_n$ 小于 $2^{2^n}$.
%%<SOLUTION>%%
解: $n=1$ 时, $p_1=2<2^{2^1}$, 结论成立.
设当 $n \leqslant k$ 时, 结论成立, 即 $p_i<2^{2^i}(i=1,2, \cdots, k)$, 于是将这 $k$ 个不等式两边分别相乘, 得
$$
p_1 p_2 \cdots p_k<2^{2^1+2^2+\cdots+2^k},
$$
所以 $p_1 p_2 \cdots p_k+1 \leqslant 2^{2^1+2^2+\cdots+2^k}=2^{2^{k+1}-2}<2^{2^{k+1}}$.
因为 $p_1, p_2, \cdots, p_k$ 都不能整除 $p_i p_2 \cdots p_k+1$, 所以 $p_1 p_2 \cdots p_k+1$ 的素因数 $q$ 不可能是 $p_1, p_2, \cdots, p_k$, 而只能大于或等于 $p_{k+1}$.
于是, 我们有 $p_{k+1} \leqslant q \leqslant p_1 p_2 \cdots p_k+1<2^{2^{k+1}}$.
这就是说, 当 $n=k+1$ 时, 结论也成立.
根据数学归纳原理, 对于任意正整数 $n$, 都有 $p_n<2^{2^n}$.
%%<REMARK>%%
注:在运用数学归纳法证明本题的过程中, 我们将 "假设 $n=k$ 时结论成立" 改为更有力的"假设 $n \leqslant k$ 时结论成立". 在这样的变通之下, 很容易对第 $k+1$ 个素数给出所需的上界估计.
%%PROBLEM_END%%



%%PROBLEM_BEGIN%%
%%<PROBLEM>%%
例4. 设 $0<a<1, x_0=1, x_{n+1}=\frac{1}{x_n}+a(n \in \mathbf{N})$. 证明: 对一切 $n \in \mathbf{N}^*$, 有 $x_n>1$.
%%<SOLUTION>%%
证法 1 我们用数学归纳法证明结论.
当 $n=1,2$ 时, $x_1=1+a>1, x_2=\frac{1}{1+a}+a=\frac{1+a+a^2}{1+a}>1$, 命题成立.
设 $n=k$ 时命题成立, 则 $n=k+2$ 时, 根据归纳假设和递推关系得
$$
0<x_{k+1}=\frac{1}{x_k}+a<1+a,
$$
继而有
$$
x_{k+2}=\frac{1}{x_{k+1}}+a>\frac{1}{1+a}+a=\frac{1+a+a^2}{1+a}>1,
$$
所以 $n=k+2$ 时命题成立.
因此对任意 $n \in \mathbf{N}^*$, 有 $x_n>1$.
%%PROBLEM_END%%



%%PROBLEM_BEGIN%%
%%<PROBLEM>%%
例4. 设 $0<a<1, x_0=1, x_{n+1}=\frac{1}{x_n}+a(n \in \mathbf{N})$. 证明: 对一切 $n \in \mathbf{N}^*$, 有 $x_n>1$.
%%<SOLUTION>%%
证法 2 我们主动加强命题, 证明 $1<x_n \leqslant 1+a$.
当 $n=1$ 时, $1<x_1=1+a$ 显然成立.
设 $n=k$ 时命题成立, 则 $n=k+1$ 时, 由归纳假设得
$$
\frac{1}{1+a}+a \leqslant \frac{1}{x_k}+a<1+a,
$$
即 $1<\frac{1+a+a^2}{1+a} \leqslant x_{k+1}<1+a$, 所以 $n=k+1$ 时命题也成立.
这样就证得了加强命题.
特别地, $x_n>1$ 成立.
%%<REMARK>%%
注:本题若直接用 $x_k>1$ 来证明 $x_{k+1}>1$ 是不可能的.
上述两个证明展现了归纳法的两种典型的变通方式.
第一个证明是设置跨度 2 来证明一般的 $n$ 成立, 注意此时我们要验证的起点也应变为两个.
另有些问题虽可以用归纳法的基本形式给出证明,但若将设置适当跨度可使论证显著简化, 此时只要确保起点都能逐一验证即可.
第二个证明是主动给 $x_k$ 设置一个上限, 根据递推关系, 下一项 $x_{k+1}$ 就自动得到一个下限.
第一步的验证固然要增多一点工作量, 但实施归纳时, 从归纳假设中获得的信息更强,这点有利于结论的证明.
应当注意的是, 在加强命题之后, 应对 $n=k+1$ 的情况也推出加强后的结论, 例如本题中不应当仅推出 $x_{k+1}>1$ 完事, 而应推出 $1<x_{k+1} \leqslant 1+a$, 这样才能说明这个加强的性质具有继承性.
%%PROBLEM_END%%



%%PROBLEM_BEGIN%%
%%<PROBLEM>%%
例5. 设 $a_n=1+\frac{1}{2}+\cdots+\frac{1}{n}$, 求证: 当 $n \geqslant 2$ 时, 有
$$
a_n^2>2\left(\frac{a_2}{2}+\frac{a_3}{3}+\cdots+\frac{a_n}{n}\right) .
$$
%%<SOLUTION>%%
证明:把命题加强为当 $n \geqslant 2$ 时, 有
$$
a_n^2>2\left(\frac{a_2}{2}+\frac{a_3}{3}+\cdots+\frac{a_n}{n}\right)+\frac{1}{n} .
$$
当 $n=2$ 时, 左边 $=a_2^2=\left(1+\frac{1}{2}\right)^2=\frac{9}{4}$, 右边 $=a_2+\frac{1}{2}=2$, 命题成立.
假设命题在 $n=k$ 时成立, 即
$$
a_k^2>2\left(\frac{a_2}{2}+\frac{a_3}{3}+\cdots+\frac{a_k}{k}\right)+\frac{1}{k} .
$$
当 $n=k+1$ 时,有
$$
\begin{aligned}
a_{k+1}^2 & =\left(a_k+\frac{1}{k+1}\right)^2=a_k^2+\frac{2 a_k}{k+1}+\frac{1}{(k+1)^2} \\
& >2\left(\frac{a_2}{2}+\frac{a_3}{3}+\cdots+\frac{a_k}{k}\right)+\frac{1}{k}+\frac{2}{k+1}\left(a_{k+1}-\frac{1}{k+1}\right)+\frac{1}{(k+1)^2} \\
& =2\left(\frac{a_2}{2}+\cdots+\frac{a_k}{k}+\frac{a_{k+1}}{k+1}\right)+\frac{1}{k}-\frac{1}{(k+1)^2} \\
& >2\left(\frac{a_2}{2}+\cdots+\frac{a_k}{k}+\frac{a_{k+1}}{k+1}\right)+\frac{1}{k+1},
\end{aligned}
$$
即 $n=k+1$ 时命题也成立.
所以, 由数学归纳法知, 加强的命题成立, 从而原命题成立.
%%PROBLEM_END%%



%%PROBLEM_BEGIN%%
%%<PROBLEM>%%
例6. 有 2005 个青年围坐在一个大圆桌旁, 其中男孩不多于 668 人.
如果一个女孩 $G$ 从两个方向之一数到任何一个人 (从自己的下一个人开始数起), 女孩数目总是大于男孩数目, 称 $G$ 是在一个 "好位置"上.
证明: 无论怎样安排座位, 总存在一个女孩在一个好位置上.
%%<SOLUTION>%%
证明:我们用数学归纳法证明一个更强的且具有一般形式的命题:
若 $n$ 个男孩与 $2 n+1$ 个女孩围坐在一个大圆桌旁,则一定存在一个女孩, 从她开始逆时针数到任何一个人时,女孩总比男孩多.
将圆桌旁的座位按顺时针编号为 1 至 $3 n+1$.
当 $n=1$ 时,不妨设 $1 、 2 、 3$ 号为女孩, 4 号为男孩,则 3 号女孩满足命题.
假设当 $n=k$ 时命题成立.
当 $n=k+1$ 时, 由于女孩多于男孩, 必有两个女孩相邻.
以这两个女孩为起点逆时针数到第一个男孩, 不妨设在 1 号位置, 则 $2 、 3$ 号都是女孩.
当所有人坐好后, 撤掉 $1 、 2 、 3$ 号三人,由归纳假设知剩下的人中必有一个女孩,不妨记为 $A$,她处在 $m$ 号位置, 从她开始逆时针数,女孩总比男孩多.
再让 $1 、 2 、 3$ 号三人坐回原处.
此时, 根据 $A$ 的取法, $A$ 从 $m-1$ 号数到 4 号时知女孩一直比男孩多; 当数到 $3 、 2 、 1$ 号时, 由于先数到两个女孩, 因此仍是女孩比男孩多; 再数下去时,根据 $A$ 的取法,仍保持女孩比男孩多.
所以 $n=k+1$ 时命题成立.
故加强命题成立.
在原问题中取 $n=668$ 即得证(若男孩人数少于 668 , 则表明即便将某些女孩当做男孩计人,仍能满足命题,因此结论必定成立).
%%<REMARK>%%
注:求解本题时, 先观察出已知条件中的数字 2005 与 668 实际上是 $3 n+1$ 和 $n$ 的内在关系, 不妨将问题一般化, 便于抓住本质予以证明.
在试图归纳证明时, 又发现, 当考虑 $n=k+1$ 的情形时, 若撤掉相邻 3 人后只是利用 $n=k$ 时"存在一个女孩, 她可以从两个方向之一数到任何一个人,女孩总比男孩多"的归纳假设,那么归纳过渡会发生困难.
因此, 为了获得更强的归纳假设, 我们大胆加强命题,证明"存在一个女孩, 当她逆时针数时,女孩总比男孩多",并且这种性质具有继承性, 从而问题解决.
本题可谓加强命题并辅以结论一般化的典型.
有时候, 我们把一个命题的结论加强或者一般化, 表面上看起来命题由于结论加强了, 问题更加难于处理, 但是, 恰恰相反, 解起来反而比特殊的、具体的情况轻松容易.
这在用数学归纳法时尤其常见.
苏联数学家辛钦曾说过: "在数学归纳法的证明中, 假设命题当 $n-1$ 时成立, 再来证明它当 $n$ 时也成立, 因此, 命题越强, 在 $n-1$ 的情况下所给的条件也越多, 而对数 $n$, 要证明的东西也越多, 但是在许多问题中, 条件较多显得更为重要.
%%PROBLEM_END%%



%%PROBLEM_BEGIN%%
%%<PROBLEM>%%
例7. 证明: 对一切正整数 $n$, 不定方程 $x^2+y^2=z^n$ 都有正整数解.
%%<SOLUTION>%%
证明:当 $n=1$ 时,取 $x=y=1, z=2$; 当 $n=2$ 时,取 $x=3, y=4$, $z=5$, 即可使它们满足方程, 故知命题在 $n=1$ 和 2 时成立.
假定当 $n=k$ 时, $x=x_0, y=y_0, z=z_0$ 是一组正整数解;那么当 $n= k+2$ 时, 只要取 $x=x_0 z_0, y=y_0 z_0, z=z_0$, 就有
$$
\left(x_0 z_0\right)^2+\left(y_0 z_0\right)^2=z_0^2\left(x_0^2+y_0^2\right)=z_0^{k+2},
$$
知它们恰为方程的一组正整数解.
所以当 $n=k+2$ 时,命题也成立.
由于我们采用了两个起点, 所以可以用跨度 2 跳跃.
这表明对一切正整数 $n$, 不定方程都有正整数解,证毕.
%%PROBLEM_END%%



%%PROBLEM_BEGIN%%
%%<PROBLEM>%%
例8. 对怎样的正整数 $n$, 集合 $\{1,2, \cdots, n\}$ 可以分成 5 个互不相交的子集, 每个子集的元素和相等.
%%<SOLUTION>%%
解:先找一个必要条件:如果 $\{1,2, \cdots, n\}$ 能分成 5 个互不相交的子集, 各个子集的元素和相等,那么
$$
1+2+\cdots+n=\frac{1}{2} n(n+1)
$$
能被 5 整除.
所以 $n=5 k$ 或 $n=5 k-1$,其中 $k \in \mathbf{N}^*$.
显然, $k=1$ 时, 上述条件不是充分的.
下用数学归纳法证明 $k \geqslant 2$ 时, 条件是充分的.
当 $k=2$, 即 $n=9,10$ 时, 我们把集合 $\{1,2, \cdots, 9\}$ 和 $\{1,2, \cdots, 10\}$ 作如下分拆:
$$
\begin{aligned}
& \{1,8\},\{2,7\},\{3,6\},\{4,5\},\{9\} ; \\
& \{1,10\},\{2,9\},\{3,8\},\{4,7\},\{5,6\} ;
\end{aligned}
$$
当 $k=3$ 时, 即 $n=14,15$ 时,有
$$
\begin{aligned}
& \{1,2,3,4,5,6\},\{7,14\},\{8,13\},\{9,12\},\{10,11\} ; \\
& \{1,2,3,5,6,7\},\{4,8,12\},\{9,15\},\{10,14\},\{11,13\} .
\end{aligned}
$$
因为若集合 $\{1,2, \cdots, n\}$ 能分成 5 个互不相交的子集, 并且它们的元素和相等, 那么 $\{1,2, \cdots, n, n+1, \cdots, n+10\}$ 也能分成 5 个元素和相等但互不相交的子集.
事实上, 如果
$$
\{1,2, \cdots, n\}=A_1 \cup A_2 \cup A_3 \cup A_4 \cup A_5,
$$
令 $B_1=A_1 \cup\{n+1, n+10\}, B_2=A_2 \bigcup\{n+2, n+9\}, B_3=A_3 \bigcup\{n+ 3, n+8\}, B_4=A_4 \cup\{n+4, n+7\}, B_5=A_5 \bigcup\{n+5, n+6\}$, 那么
$\{1,2, \cdots, n, n+1, \cdots, n+10\}=B_1 \cup B_2 \cup B_3 \cup B_4 \cup B_5$,
并且 $B_i \cap B_j=\varnothing, 1 \leqslant i<j \leqslant 5,\left|B_1\right|=\left|B_2\right|=\cdots=\left|B_5\right|$.
假设命题对于 $5 k-1,5 k$ 成立.
由上面讨论知, 命题对于 $5(k+2)-1$, $5(k+2)$ 也成立.
从而证明了对于 $k \geqslant 2$, 当 $n=5 k-1,5 k$ 时,集 $\{1,2, \cdots, n\}$ 可以分成 5 个元素和相等的互不相交的子集.
%%<REMARK>%%
注:本例中, 我们选取起点 $k=3$ 来实施数学归纳法证明, 并且是由 $P(k)$ 进到 $P(k+2)$, 即以步长为 2 前进.
有时候, 步子可能更大, 视具体情况而定.
%%PROBLEM_END%%



%%PROBLEM_BEGIN%%
%%<PROBLEM>%%
例9. 设 $f(x)$ 是定义在非负实数集上的函数, $f(0)=0$, 且对任意 $x \geqslant y \geqslant 0$, 有 $|f(x)-f(y)| \leqslant(x-y) f(x)$. 求 $f(x)$.
%%<SOLUTION>%%
解:们用数学归纳法证明: 对任意正整数 $n$, 当 $\frac{n-1}{2} \leqslant x<\frac{n}{2}$ 时, 有 $f(x)=0$.
当 $n=1$ 时,在条件中取 $x, y$ 使 $0 \leqslant x<\frac{1}{2}, y=0$, 得
$$
|f(x)| \leqslant x f(x) \leqslant x|f(x)| \leqslant \frac{1}{2}|f(x)|,
$$
故 $|f(x)|=0$, 即 $f(x)=0$, 命题成立.
设 $n=k$ 时命题成立, 则当 $n=k+1$ 时, 取 $x, y$ 使 $\frac{k}{2} \leqslant x<\frac{k+1}{2}, y=x- \frac{1}{2}$, 此时 $\frac{k-1}{2} \leqslant y<\frac{k}{2}$, 由归纳假设知 $f(y)=0$, 代入已知条件得 $|f(x)| \leqslant \frac{1}{2} f(x)$, 类似前面的讨论可知 $f(x)=0$, 故 $n=k+1$ 时命题成立.
注意到当 $n$ 取遍所有正整数时, $x \in\left[\frac{n-1}{2}, \frac{n}{2}\right)$ 取遍一切非负实数, 从而由数学归纳法得: $f(x)$ 恒等于 0 .
%%<REMARK>%%
注:这是在实数情形下使用数学归纳法的一个例子.
与证明关于正整数的命题不同, 由于正实数集是 "不可列" 的, 换言之就是不存在一个数列取遍一切正实数值 (严格证明需要用到一点高等数学的知识), 因此我们势必对无穷多个起点进行验证.
本题中, 我们的处理方式是将 $[0,+\infty)$ 拆成 "可列" 区间 $\left[\frac{n-1}{2}, \frac{n}{2}\right)$, 通过对第一个区间的讨论统一完成了无穷个起点的验证过程, 再设置"步长"为 $\frac{1}{2}$ 进行证明.
%%PROBLEM_END%%



%%PROBLEM_BEGIN%%
%%<PROBLEM>%%
例10. 设 $f(m, n)$ 满足 $f(1, n)=f(m, 1)=1\left(m, n \in \mathbf{N}^*\right)$, 且当 $m$, $n \geqslant 2$ 时有
$$
f(m, n) \leqslant f(m, n-1)+f(m-1, n) .
$$
求证: $f(m, n) \leqslant \mathrm{C}_{m+n-2}^{m-1}$.
%%<SOLUTION>%%
证明:将命题 $f(m, n) \leqslant \mathrm{C}_{m+n-2}^{m-1}$ 记为 $P(m, n)$. 下用数学归纳法证明 $P(m, n)$.
因为对 $m, n \in \mathbf{N}^*$, 有 $f(1, n)=1=\mathrm{C}_{1+n-2}^{1-1}, f(m, 1)=1=\mathrm{C}_{m+1-2}^{m-1}$, 故 $P(1, n)$ 与 $P(m, 1)$ 成立.
当 $m, n \geqslant 2$ 时,假设 $P(m, n-1)$ 与 $P(m-1, n)$ 成立, 即
$$
f(m, n-1) \leqslant \mathrm{C}_{m+n-3}^{m-1}, f(m-1, n) \leqslant \mathrm{C}_{m+n-3}^{m-2},
$$
则结合已知条件得
$$
f(m, n) \leqslant f(m, n-1)+f(m-1, n) \leqslant \mathrm{C}_{m+n-3}^{m-1}+\mathrm{C}_{m \nmid n-3}^{m-2}=\mathrm{C}_{m \nmid n-2}^{m-1},
$$
即命题 $P(m, n)$ 成立.
由双参数数学归纳法知, 对任意 $m, n \in \mathbf{N}^*, P(m, n)$ 成立.
%%<REMARK>%%
注:本例给出的是双参数数学归纳法的一种基本证题模式.
双参数归纳法往往用来证明某个与正整数 $m, n$ 有关的命题 $P(m, n)$. 对双参数归纳法, 也有各种灵活多样的形式.
%%PROBLEM_END%%



%%PROBLEM_BEGIN%%
%%<PROBLEM>%%
例11. 设 $k$ 是正整数, 证明: 可以将集合 $\left\{0,1,2,3, \cdots, 2^{k+1}-1\right\}$ 分成两个没有公共元素的子集 $\left\{x_1, x_2, \cdots, x_{2^k}\right\}$ 和 $\left\{y_1, y_2, \cdots, y_{2^k}\right\}$, 使得 $\sum_{i=1}^{2^k} x_i^m= \sum_{i=1}^{2^k} y_i^m$ 对任何 $m \in\{1,2, \cdots, k\}$ 都成立.
%%<SOLUTION>%%
证明:对 $k$ 用数学归纳法.
当 $k=1$ 时, 令 $x_1=0, x_2=3, y_1=1, y_2=2$ 即可,命题成立.
假设命题在 $k$ 时成立, 考虑 $k+1$ 时的情况.
由归纳假设知, 可将 $\left\{0,1,2, \cdots, 2^{k+1}-1\right\}$ 分成满足条件的子集 $\left\{x_1\right.$, $\left.x_2, \cdots, x_{2^k}\right\}$ 和 $\left\{y_1, y_2, \cdots, y_{2^k}\right\}$. 此时令
$$
\begin{aligned}
& A=\left\{x_1, x_2, \cdots, x_{2^k}, 2^{k+1}+y_1, 2^{k+1}+y_2, \cdots, 2^{k+1}+y_{2^k}\right\}, \\
& B=\left\{y_1, y_2, \cdots, y_{2^k}, 2^{k+1}+x_1, 2^{k+1}+x_2, \cdots, 2^{k+1}+x_{2^k}\right\},
\end{aligned}
$$
则 $A \cup B=\left\{0,1,2, \cdots, 2^{k+2}-1\right\}, A \cap B=\varnothing$.
为证明 $k+1$ 时命题成立, 需要验证的是: 对 $m \in\{1,2, \cdots, k+1\}$, 有
$$
\sum_{i=1}^{2^k} x_i^m+\sum_{i=1}^{2^k}\left(2^{k+1}+y_i\right)^m=\sum_{i=1}^{2^k} y_i^m+\sum_{i=1}^{2^k}\left(2^{k+1}+x_i\right)^m,
$$
显然该式可依次等价变形为
$$
\begin{gathered}
\sum_{i=1}^{2^k} x_i^m+\sum_{i=1}^{2^k}\left(y_i^m+\sum_{j=0}^{m-1} \mathrm{C}_m^j\left(2^{k+1}\right)^{m-j} y_i^j\right)=\sum_{i=1}^{2^k} y_i^m+\sum_{i=1}^{2^k}\left(x_i^m+\sum_{j=0}^{m-1} \mathrm{C}_m^j\left(2^{k+1}\right)^{m-j} x_i^j\right) \\
\sum_{i=1}^{2^k} x_i^m+\sum_{i=1}^{2^k} y_i^m+\sum_{j=0}^{m-1} \mathrm{C}_m^j\left(2^{k+1}\right)^{m-j} \sum_{i=1}^{2^k} y_i^j=\sum_{i=1}^{2^k} y_i^m+\sum_{i=1}^{2^k} x_i^m+\sum_{j=0}^{m-1} \mathrm{C}_m^j\left(2^{k+1}\right)^{m-j} \sum_{i=1}^{2^k} x_i^j ; \\
\sum_{j=0}^{m-1} \mathrm{C}_m^j\left(2^{k+1}\right)^{m-j}\left(\sum_{i=1}^{2^k} x_i^j-\sum_{i=1}^{2^k} y_i^j\right)=0 \label{eq1}.
\end{gathered}
$$
对任意给定的 $m \in\{1,2, \cdots, k+1\}$, 由于 $\sum_{i=1}^{2^k} x_i^0=2^k=\sum_{i=1}^{2^k} y_i^0$, 且根据归纳假设知, 当 $j \in\{1,2, \cdots, k\}$ 时有 $\sum_{i=1}^{2^k} x_i^j=\sum_{i=1}^{2^k} y_i^j$, 故对 $j=0,1, \cdots$, $m-1$, 有
$$
\sum_{i=1}^{2^k} x_i^j-\sum_{i=1}^{2^k} y_i^j=0 .
$$
故式\ref{eq1}成立, 进而命题在 $k+1$ 时成立.
由数学归纳法知, 对任意正整数 $k$, 存在满足条件的集合分法.
%%<REMARK>%%
注:本例中, 对满足条件的集合分法的存在性, 很难给出非构造性的证明, 然而直接构造又显得困难重重, 因此从 $k=1$ 的简单情形着手, 通过数学归纳法构造出在每个正整数 $k$ 时集合的具体分法.
%%PROBLEM_END%%


