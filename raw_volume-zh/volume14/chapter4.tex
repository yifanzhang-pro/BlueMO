
%%TEXT_BEGIN%%
抽屉原理又称鸽巢原理, 它是组合数学中的一个基本原理, 最先是由德国数学家狄里克利明确地提出来的,因此,也称为狄里克利原理.
把 10 个苹果放到 9 个抽屉里,一定有一个抽屉里至少有 2 个苹果; 把 10 个苹果放到 3 个抽屉里,一定有一个抽屉里至少有 4 个苹果; 把 9 个苹果放到 3 个抽屉里,一定有一个抽屉里至少有 3 个苹果.
这些看似简单的道理, 却是我们解决存在性问题的一个非常有用的方法.
抽屈原理的常用形式为:
抽屉原理 1 : 如果把 $n+1$ 件东西任意放人 $n$ 个抽屉, 那么必定有一个抽屉里至少有两件东西.
抽屉原理 2 : 如果把 $m$ 件东西任意放人 $n$ 个抽屉, 那么必定有一个抽屉里至少有 $\left[\frac{m-1}{n}\right]+1$ 件东西, 也必定有一个抽屉里至多有 $\left[\frac{m}{n}\right]$ 件东西, 其中 $[x]$ 表示不超过 $x$ 的最大整数.
抽屉原理 3 : 如果把无穷多件东西放人 $n$ 个抽屉, 那么必定至少有一个抽屉里有无穷多件东西.
其实, 抽屉原理 1 是抽屉原理 2 的特殊情况.
用抽屉原理解题, 关键是设计 "抽屉", 抽屉设计得好, 题目就容易解决, 设计得不好, 反而使问题复杂化, 甚至无法解决.
究竟如何设计, 没有统一的套路, 需对具体问题作具体分析.
运用抽屉原理解题时, 有时用等价的 "平均数原理", 在几何上则用 "重叠原理".
当处理存在性问题时, 我们常常会用到抽屉原理.
%%TEXT_END%%



%%PROBLEM_BEGIN%%
%%<PROBLEM>%%
例1-1. 已知集合 $S=\{1,2, \cdots, 100\}$. 试证明: 若 $A$ 是 $S$ 的 51 元子集,则必存在 $i, j \in A$,使 $i-j=50$
%%<SOLUTION>%%
证明:令 $A_k=\{k, k+50\}, k=1,2, \cdots, 50$.
由于 $\bigcup_{k=1}^{50} A_k=S$, 根据抽屉原理, 51 元集合 $A$ 中必存在两个元素属于同一个 $A_k$. 取 $i=k+50, j=k$, 则 $i-j=50$. 故结论成立.
%%PROBLEM_END%%



%%PROBLEM_BEGIN%%
%%<PROBLEM>%%
例1-2. 已知集合 $S=\{1,2, \cdots, 100\}$. 试证明:
(2) 若 $B$ 是 $S$ 的 $n$ 元子集 $(2 \leqslant n \leqslant 100)$, 则必存在 $i, j \in B$, 使得
$$
0<i-j<\frac{100}{n-1}
$$
%%<SOLUTION>%%
证明:记 $\frac{99}{n-1}=\alpha$, 令 $B_k=\{x \in S \mid 1+(k-1) \alpha \leqslant x \leqslant 1+k \alpha\}, k=1$, $2, \cdots, n-1$.
由于 $\bigcup_{k=1}^{n-1} B_k=S$, 根据抽屉原理, $n$ 元集合 $B$ 中必存在两个元素 $i, j(i>$ j) 属于同一个 $B_k$, 从而 $0<i-j \leqslant \alpha<\frac{100}{n-1}$. 故结论成立.
%%PROBLEM_END%%



%%PROBLEM_BEGIN%%
%%<PROBLEM>%%
例1-3. 已知集合 $S=\{1,2, \cdots, 100\}$. 试证明:
若 $C$ 是 $S$ 的 51 元子集,则必存在 $i, j \in C, i<j$, 使得 $i \mid j$
%%<SOLUTION>%%
证明:令 $C_k=\left\{x \in S \mid x=2^p(2 k-1), p \in \mathbf{N}\right\}, k=1,2, \cdots, 50$.
由于 $\bigcup_{k=1}^{50} C_k=S$, 根据抽屉原理, 51 元集合 $C$ 中必存在两个元素 $i, j(i< j$ ), 它们属于同一个 $C_k$, 由 $C_k$ 的构造方法可得 $i \mid j$. 故结论成立.
%%PROBLEM_END%%



%%PROBLEM_BEGIN%%
%%<PROBLEM>%%
例1-4. 已知集合 $S=\{1,2, \cdots, 100\}$. 试证明:
若 $D$ 是 $S$ 的 75 元子集, 则 $D$ 中必存在 4 个元素两两互素.
%%<SOLUTION>%%
证明:令 $D_1=\{1,2,3,5,7,11, \cdots, 89,97\}, D_2=\{2 \times 47,3 \times 31$, $5 \times 19,7 \times 13\}, D_3=\{2 \times 43,3 \times 29,5 \times 17,7 \times 11\}$,
$D_4=\left\{2 \times 41,3 \times 23,5 \times 13,7^2\right\}$, 其中 $D_1$ 包含 1 及所有 25 个不超过 100 的素数.
因 $S$ 中除 $D_0=\bigcup_{i=1}^4 D_i$ 中的元素外剩下 62 个元素, 故 75 元子集 $D$ 中必有 13 个元素属于 $D_0$, 根据抽屟原理, 必有 4 个元素属于某个 $D_i, i \in\{1,2,3$, $4\}$, 显然它们两两互素.
故结论成立.
%%<REMARK>%%
注:应用抽屉原理的关键就是找出适当的分类规则, 用于证明存在性等结论, 通俗地讲就是 "制造抽屉". 这组题目涉及到抽屉原理的简单应用, 例如第 (1)问是直接简单分组, 第 (2) 问通过划分区间制造抽屉, 这是最基本的手法; 第 (3) 问是按最大奇因数来分类; 第 (4) 问在素数上做文章, 有较高的技巧,但这也是常用的手法,并且体现了抽屉原理解决问题的灵活性.
%%PROBLEM_END%%



%%PROBLEM_BEGIN%%
%%<PROBLEM>%%
例2-1. 把 $1,2, \cdots, 10$ 按任意次序排成一个圆圈.
证明:一定可以找到三个相邻的数,它们的和不小于 18.
%%<SOLUTION>%%
证明:设这 10 个数在圆周上排列为 $1, a_1, a_2, \cdots, a_9$ (如图(<FilePath:./figures/fig-c4i1.png>)). 由于
$$
\left(a_1+a_2+a_3\right)+\left(a_4+a_5+a_6\right)+\left(a_7+a_8+a_9\right)=2+3+\cdots+10=54,
$$
所以 $a_1+a_2+a_3, a_4+a_5+a_6, a_7+a_8+a_9$ 这三个数中一定有一个数不小于 $\frac{54}{3}=18$.
%%PROBLEM_END%%



%%PROBLEM_BEGIN%%
%%<PROBLEM>%%
例2-2. 把 $1,2, \cdots, 10$ 按任意次序排成一个圆圈.
证明:一定可以找到三个相邻的数,它们的和不大于 15 .
%%<SOLUTION>%%
证明:设这 10 个数在圆周上排列为 $10, b_1, b_2 \cdots, b_9$ (如图(<FilePath:./figures/fig-c4i2.png>)). 由于
$$
\left(b_1+b_2+b_3\right)+\left(b_4+b_5+b_6\right)+\left(b_7+b_8+b_9\right)=1+2+\cdots+9=45 \text {, }
$$
所以, $b_1+b_2+b_3, b_4+b_5+b_6, b_7+b_8+b_9$ 这三个数中一定有一个数不大于 $\frac{45}{3}=15$.
%%<REMARK>%%
注:本例运用的是与抽屉原理本质相同的"平均数原理".
%%PROBLEM_END%%



%%PROBLEM_BEGIN%%
%%<PROBLEM>%%
例3. 设 $m$ 为给定正整数.
证明: 若集合 $A_1, A_2, \cdots, A_m$ 两两交集为空, 且满足 $\bigcup_{i=1}^m A_i=\mathbf{N}^*$, 则必存在一个集合 $A_i(1 \leqslant i \leqslant m)$, 它含有任意正整数的倍数.
%%<SOLUTION>%%
证明:我们考虑无穷数列 $\{n !\}$, 其中 $n ! \in \mathbf{N}^*$, 即 $n ! \in \bigcup_{i=1}^m A_i$. 根据抽屉原理, 在有限个集合 $A_1, A_2, \cdots, A_m$ 中, 必存在一个集合, 不妨设为 $A_1$, 它含有上述数列中的无穷项.
此时, 对任意正整数 $N$, 必存在一个正整数 $N_1 \geqslant N$, 使 $N_{1} ! \in A_1$, 因而集合 $A_1$ 含有 $N$ 的倍数 $N_1$ !. 由 $N$ 的任意性知结论成立.
%%<REMARK>%%
注:本例所使用的是着眼于无限形式的抽屉原理 3. 注意题目中的条件 "两两交集为空"可以去掉而不影响结果.
%%PROBLEM_END%%



%%PROBLEM_BEGIN%%
%%<PROBLEM>%%
例4. 从 $1,2,3, \cdots, 16$ 这 16 个数中, 最多能选出多少个数, 使得被选出的数中, 任意三个数都不是两两互素的.
%%<SOLUTION>%%
解:首先, 取出 $1,2, \cdots, 16$ 中所有 2 或 3 的倍数, 有
$$
2,3,4,6,8,9,10,12,14,15,16 .
$$
这 11 个数要么是 2 的倍数, 要么是 3 的倍数.
由抽屉原理知, 这 11 个数中的任意三个数,都必有两个数同为 2 或 3 的倍数, 它们的最大公约数大于 1 , 也就是说这三个数是不两两互素的.
所以, 从 $1,2, \cdots, 16$ 中可以选出 11
个数满足要求.
下面证明从 $1,2, \cdots, 16$ 中任取 12 (或 12 以上) 个数, 其中一定有 3 个数两两互素.
事实上, 令数组 $A=\{1,2,3,5,7,11,13\}$. 数组 $A$ 中有 7 个数, 而且这 7 个数是两两互素的.
从 $1,2, \cdots, 16$ 中任取 12 个数, 由于 $A$ 以外只有 9 个数, 故 $A$ 中至少有 3 个数被选出, 这三个数是两两互素的.
综上可知,最多可以选出 11 个满足题设的数.
%%PROBLEM_END%%



%%PROBLEM_BEGIN%%
%%<PROBLEM>%%
例5. 设 $T$ 是由 $60^{100}$ 的所有正因数组成的集合.
$S$ 是 $T$ 的一个子集, 其中没有一个数是另一个数的倍数.
求 $S$ 的元素个数的最大值.
%%<SOLUTION>%%
解:约定下文中的 $a, b, c$ 都是非负整数.
因为 $60=2^2 \cdot 3 \cdot 5$, 所以
$$
\begin{gathered}
T=\left\{2^a 3^b 5^c \mid 0 \leqslant a \leqslant 200,0 \leqslant b, c \leqslant 100\right\} . \\
S=\left\{2^{200-b-c} 3^b 5^c \mid 0 \leqslant b, c \leqslant 100\right\},
\end{gathered}
$$
对任意 $0 \leqslant b, c \leqslant 100$, 有 $0 \leqslant 200-b-c \leqslant 200$, 所以 $S$ 是 $T$ 的一个子集且含 $101^2$ 个元素.
下面我们证明 $S$ 中没有一个数是另一个数的倍数, 并且元素个数超过 $101^2$ 的子集都不满足这个条件.
假设 $2^{200-b-c} 3^b 5^c$ 是 $2^{200-i-j} 3^i 5^j$ 的倍数, 且 $(b, c) \neq(i, j)$, 则
$$
200-b-c \geqslant 200-i-j, b \geqslant i, c \geqslant j,
$$
第一个不等式表明 $b+c \leqslant i+j$, 与后两个不等式联立得 $b=i, c=j$. 矛盾.
所以 $S$ 中没有一个元素是另一个的倍数.
设 $U$ 是 $T$ 的一个超过 $101^2$ 个元素的子集.
因为只有 $101^2$ 对互异的 $(b$, $c$ ), 由抽屉原理, $U$ 中必有两个元素 $u_1=2^{a_1} 3^{b_1} 5^{c_1}, u_2=2^{a_2} 3^{b_2} 5^{c_2}$, 其中 $b_1= b_2, c_1=c_2$, 而 $a_1 \neq a_2$. 若 $a_1>a_2$, 则 $u_1$ 是 $u_2$ 的倍数; 若 $a_1<a_2$, 则 $u_2$ 是 $u_1$ 的倍数.
因此 $U$ 不满足题设条件.
所以 $T$ 的满足题设条件的子集最多可以含有 $101^2=10201$ 个元素.
%%PROBLEM_END%%



%%PROBLEM_BEGIN%%
%%<PROBLEM>%%
例6. 正实数 $a_1, a_2, \cdots, a_{2011}$ 满足 $a_1<a_2<\cdots<a_{2011}$. 证明: 其中一定存在两个数 $a_i, a_j(i<j)$, 使得
$$
a_j-a_i<\frac{\left(1+a_i\right)\left(1+a_j\right)}{2010} .
$$
%%<SOLUTION>%%
证明:令 $x_i=\frac{2010}{1+a_i}, i=1,2, \cdots, 2011$. 则 $0<x_{2011}<x_{2010}<\cdots< x_1<2010$, 即 2011 个数 $x_{2011}, x_{2010}, \cdots, x_1$ 在如下 2010 个区间里:
$$
(0,1],(1,2],(2,3], \cdots,(2008,2009],(2009,2010) \text {, }
$$
所以, 由抽屉原理知, 其中一定有两个数在同一个区间里, 故一定存在 $1 \leqslant k \leqslant 2010$, 使得 $x_k-x_{k+1}<1$, 从而
$$
\frac{2010}{1+a_k}-\frac{2010}{1+a_{k+1}}<1
$$
即
$$
a_{k+-1}-a_k<\frac{\left(1+a_k\right)\left(1+a_{k+1}\right)}{2010},
$$
故命题得证.
%%PROBLEM_END%%



%%PROBLEM_BEGIN%%
%%<PROBLEM>%%
例7. 一位象棋大师为参加一次比赛将进行 77 天的练习, 他准备每天至少下一局棋,而每周至多下 12 局棋.
证明存在一个正整数 $n$, 使得他在这 77 天里有连续的 $n$ 天共下了 21 局棋.
%%<SOLUTION>%%
证明:设 $a_i$ 是这位大师从第 1 天到第 $i$ 天下棋的总局数, $i=1,2, \cdots$, 77. 因为他每天至少下一局棋, 所以
$$
1 \leqslant a_1<a_2<\cdots<a_{77}
$$
又因为每周至多下 12 局棋, 所以 77 天中下棋的总局数 $a_{77} \leqslant 12 \times \frac{77}{7}=$ 132.
考虑数列 $a_1+21, a_2+21, \cdots, a_{77}+21$, 有
$$
22 \leqslant a_1+21<a_2+21<\cdots<a_{77}+21 \leqslant 132+21=153 .
$$
考察数列 $a_1, a_2, \cdots, a_{77}, a_1+21, a_2+21, \cdots, a_{77}+21$. 该数列有 154 个项, 每个数都是小于等于 153 的正整数.
由抽屉原理, 必定存在 $i, j$ 使得
$$
a_i=a_j+21 .
$$
令 $n=i-j$, 那么该大师在第 $j+1, j+2, \cdots, j+21(=i)$ 的连续 $n$ 天中共下了 21 局棋.
%%<REMARK>%%
注:本问题中, 假如考虑第 $i$ 天下棋的局数 $b_i$, 则很难设计适当的 "抽屉", 保证一个抽屉中存在和为 21 且下标连续的若干个 $b_i$. 转而考察第 1 天到第 $i$ 天下棋的总局数 $a_i$, 那么任意两个 $a_i$ 之差总是连续若干天下棋的局数, 因此对 $a_i$ 就比较容易设计"抽屉"解决问题.
此外, 本题也可直接把 $\{1,2, \cdots, 1.32\}$ 分成 63 个二元集 $A_{p q}$ 及 6 个一-元集 $A_r$ 的并集,其中
$$
\begin{gathered}
A_{p q}=\{42 p+q, 42 p+q+21\}, p=0,1,2, q=1,2, \cdots, 21, \\
A_r=\{r\}, r=127,128, \cdots, 132 .
\end{gathered}
$$
根据抽屉原理可知 $a_1, a_2, \cdots, a_{77}$ 中必有两数在同一个 $A_{p q}$ 中.
%%PROBLEM_END%%



%%PROBLEM_BEGIN%%
%%<PROBLEM>%%
例8. 设 100 个非负实数的和为 1 , 证明: 可将它们适当排列在圆周上,使得将每两个相邻数相乘后, 所得的 100 个乘积之和不超过 0.01 .
%%<SOLUTION>%%
证明:记这 100 个数字为 $x_1, x_2, \cdots, x_{100}$, 它们所能构成的圆排列共 99! 种, 记这 99! 种情况所对应的相邻数乘积之和分别为 $S_1, S_2, \cdots, S_n$, 其中 $n=99$ !.
由于圆周上使 $x_i, x_j$ 相邻可有两种次序, 对每种次序, 相应的 $x_1$, $x_2, \cdots, x_{100}$ 的圆排列个数等于其余 98 个数在直线上的全排列数, 即 $98 !$ 个, 从而使 $x_i, x_j$ 相邻的圆排列共 $2 \cdot 98$ ! 种.
考察式子 $M=\sum_{i=1}^n S_i$.
根据上述讨论可知, 每个乘积 $x_i x_j(1 \leqslant i<j \leqslant 100)$ 在该式右端出现的次数为 $2 \cdot 98$ !, 因此结合柯西不等式得
$$
\begin{aligned}
M & =2 \cdot 98 ! \sum_{1 \leqslant i<j \leqslant 100} x_i x_j=98 ! \cdot\left[\left(\sum_{i=1}^{100} x_i\right]^2-\sum_{i=1}^{100} x_i^2\right) \\
& \leqslant 98 ! \cdot\left[\left(\sum_{i=1}^{100} x_i\right]^2-\frac{1}{100}\left[\sum_{i=1}^{100} x_i\right]^2\right]=\frac{99 !}{100} .
\end{aligned}
$$
由平均值原理知, 存在一个 $i(1 \leqslant i \leqslant 99 !)$ 使 $S_i \leqslant \frac{M}{99 !}=\frac{1}{100}$, 命题得证.
%%<REMARK>%%
注:平均数原理就其证明方法而言, 和抽屈原理几乎一样,但它可以进一步适用于非正整数的场合.
本题中对所有可能的排列所组成的整体应用平均数原理, 说明了 100 个乘积之和"平均而言"不超过 0.01 , 因而结论成立.
平均数原理的应用往往也反映了从整体考虑问题的思想方法.
%%PROBLEM_END%%



%%PROBLEM_BEGIN%%
%%<PROBLEM>%%
例9. 给定一个 $n^2+1$ 项的实数列
$$
a_1, a_2, a_3, \cdots, a_{n^2+1}, \label{eq1}
$$
证明该数列中一定存在一个由 $n+1$ 数组成的子数列, 这个子数列是递增的或递减的.
%%<SOLUTION>%%
证明:从数列 式\ref{eq1} 中的每一项 $a_i$, 向后可选出若干个单调增子数列, 其中有一个项数最多的单调增子数列, 设其项数为 $l_i, i=1,2, \cdots, n^2+1$. 于是得到另一个整数数列
$$
l_1, l_2, l_3, \cdots, l_{n^2+1} . \label{eq2}
$$
如果 式\ref{eq2} 中有一个数 $l_k \geqslant n+1$, 那么命题已经获证, 否则, 即不存在项数超过 $n$ 的单调增子数列, 也就是 $0<l_i \leqslant n, i=1,2, \cdots, n^2+1$.
根据抽屉原理, 在 $l_1, l_2, l_3, \cdots, l_n{ }^2+1$ 中有 $\left[\frac{n^2+1-1}{n}\right]+1=n+1$ 个数是相等的, 不妨设为 $l_{k_1}=l_{k_2}=\cdots=l_{k_{n+1}}=l$, 其中 $k_1<k_2<\cdots<k_{n+1}$.
由于所有 $a_i$ 是不同的实数, 所以对应的 $a_{k_1}, a_{k_2}, \cdots, a_{k_{n+1}}$ 必然满足
$$
a_{k_1}>a_{k_2}>\cdots>a_{k_{n+1}} . \label{eq3}
$$
否则, 设 $k_i<k_j$, 有 $a_{k_i}<a_{k_j}$, 那么把 $a_{k_i}$ 加到从 $a_{k_j}$ 开始的长度为 $l$ 的单调数列的前面, 构成了从 $a_{k_i}$ 开始的长度为 $l+1$ 的单调增数列, 这和 $l$ 的最大性矛盾.
由于数列 式\ref{eq3} 本身是一个单调减子数列, 这就证明了如果不存在 $n+1$ 个项的单调增子数列, 便一定存在有 $n+1$ 项的单调减子数列, 同样可证,如果 (1)不存在项数为 $n+1$ 的单调减子数列, 那么必定存在项数为 $n+1$ 的单调增子数列.
%%<REMARK>%%
注:本例是由数学家 P. Erdös 和 A. Szekeres 发现和证明的一个定理, 其简单推广形式为: 给定一个 $m n+1$ 项的实数列 $a_1, a_2, a_3, \cdots, a_{m n+1}$, 若该数列的每个单调增子数列至多含 $m$ 项,则一定存在一个 $n+1$ 项的单调减子数列.
%%PROBLEM_END%%



%%PROBLEM_BEGIN%%
%%<PROBLEM>%%
例10. 给出斐波那契数列 $\left\{F_n\right\}$ 如下: $1,1,2,3,5,8, \cdots$ (自第 3 项起, 每项等于前两项之和). 试问, 在数列的前一亿项中, 是否会有某项是 10000 的倍数?
%%<SOLUTION>%%
解:结论是肯定的.
为方便起见,约定 $F_0=0$.
设 $f_n$ 是 $F_n$ 除以 10000 所得的余数 $(n \in \mathbf{N})$, 则 $0 \leqslant f_n \leqslant 9999$, 因此根据抽屉原理, 在 $10000^2+1=10^8+1$ 对整数对 $\left(f_n, f_{n+1}\right)\left(0 \leqslant n \leqslant 10^8\right)$ 中必有两对相同, 不妨设 $\left(f_p, f_{p+1}\right)=\left(f_q, f_{q+1}\right)$, 其中 $p, q \in \mathbf{N}, 0 \leqslant p<q \leqslant 10^8$, 此时 $F_p \equiv F_q, F_{p+1} \equiv F_{q+1}(\bmod 10000)$.
若 $p \geqslant 1$, 则根据递推关系得
$$
F_{q-1}=F_{q+1}-F_q \equiv F_{p+1}-F_p=F_{p-1}(\bmod 10000),
$$
以下依次有
$$
F_{q-2} \equiv F_{p-2}, \cdots, F_{q-p+1} \equiv F_1, F_{q-p} \equiv F_0=0(\bmod 10000) .
$$
若 $p=0$, 则 $F_{q-p} \equiv F_0=0(\bmod 10000)$ 仍成立.
显然 $F_{q-p}>0$, 故 $F_{q-p}$ 是 10000 的倍数, 且 $1 \leqslant q-p \leqslant 10^8$, 说明 $F_{q-p}$ 的确是前一亿项中的一项.
结论成立.
%%<REMARK>%%
注:本例若直接递推或构造出某项是 10000 的倍数固然直截了当, 但显然困难重重, 而作为一个存在性命题, 抽屉原理也许是解决的办法之一.
本例中用的是通过剩余类制造抽屉的方法, 然而若单纯地证明存在两项关于模 10000 同余并无多大用处, 关键在于如何使同余关系不断"前置", 最终得到某项与 $F_0=0$ 同余.
考虑到有 $F_{n-1}=F_{n+1}-F_n$, 我们只需说明存在两个数组 $\left(F_n, F_{n+1}\right)$, 它们每个分量对应同余即可, 而这点仍可用抽屉原理给出证明.
%%PROBLEM_END%%



%%PROBLEM_BEGIN%%
%%<PROBLEM>%%
例11. 某次考试有 5 道选择题, 每题都有 4 个不同的答案供选择, 每人每题恰选 1 个答案.
在 2000 份答卷中发现存在一个 $n$, 使得任何 $n$ 份答卷中都存在 4 份, 其中每两份的答案都至多有 3 道题相同.
求 $n$ 的最小可能值.
%%<SOLUTION>%%
解:将每题的四个选项依次记为 $1,2,3,4$, 每份答案记为 $(a, b, c, d$, $e)$, 其中 $a, b, c, d, e \in\{1,2,3,4\}$.
对给定数组 $(a, b, c, d), a, b, c, d \in\{1,2,3,4\}$, 将 2000 份答卷中答案为 $(a, b, c, d, 1),(a, b, c, d, 2),(a, b, c, d, 3),(a, b, c, d, 4)$ 的答卷归为一类, 这样共 $4^4=256$ 类.
由抽屉原理, 至少有 $\left\lceil\frac{2000}{256}\right\rceil=8$ 份答卷属于一类.
不妨取其中 8 份答卷作为 $A$ 组答卷; 在剩下的答卷中, 同理至少可取出 $\left\lceil\frac{1992}{256}\right\rceil=8$ 份, 作为 $B$ 组答卷; 剩下的答卷中仍至少可取出 $\left\lceil\frac{1984}{256}\right\rceil=8$ 份, 作为 $C$ 组答卷.
在 $A, B, C$ 组这 24 份答卷中, 根据抽屉原理, 任意 4 份答卷必有两份在一组, 则它们至少有 4 题答案相同.
所以 $n \geqslant 25$.
另一方面,当 $n=25$ 时,构造如下 2000 份答卷:
取 250 组不同的数组 $(a, b, c, d), a, b, c, d \in\{1,2,3,4\}$, 对其中每个数组, 取 $e \in\{1,2,3,4\}$ 使得 $a+b+c+d+e$ 是 4 的倍数, 对这样一组 ( $a$, $b, c, d, e)$, 令 8 份答卷写有这样的答案, 共 2000 份答卷.
任取其中 25 份答卷, 必有 4 份答案两两不同, 再根据数组 $(a, b, c, d, e)$ 的取法可知, 其中任何两份答卷至多只有 3 题答案相同.
综上可知, $n$ 的最小可能值为 25 .
%%<REMARK>%%
注:本题的表述比较复杂, 应先仔细分析题目要求.
方面我们要构造 2000 份答卷, 尽可能使比较小的 $n$ 满足题意; 另一方面又要找到尽可能大的 $k$, 使得任意 2000 份答卷都不满足题意, 即存在 $k$ 份答卷, 使得其中任何 4 份答卷中, 总有两份的答案至少有 4 题一致.
在上述两方面的讨论中, "任何 $n$ 份答卷"、"任意 2000 份答卷"、"任何 4 份答卷"这些表述均涉及到答卷的任意性, 而要处理的又是存在性问题, 于是抽屉原理便有了用武之地(注意在解答最后一步验证"25 份答卷中必有 4 份答案两两不同"时,也用了抽屉原理: 假设 25 份答卷中只出现 3 种不同的答案, 那么必有一种答案出现超过 8 次).
抽屉原理是解决存在性问题的有力工具.
抽屉原理本身并不难,难在如何运用它去解决问题.
在具体问题中如何制造抽屉, 希望本节的例题对读者能有所启示.
%%PROBLEM_END%%


