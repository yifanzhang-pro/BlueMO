
%%PROBLEM_BEGIN%%
%%<PROBLEM>%%
问题1. 在银行的某个窗口前有 12 个人在排队, 当该窗口因故关闭时, 这 12 个人都要到另一个窗口重新排队.
要使得这 12 个人中每个人的位置与原来所排的次序相差都不超过 1, 问有多少种不同的排队方法?
%%<SOLUTION>%%
用 $a_n$ 表示有 $n$ 人排队时,满足条件的排队方法数.
易验证 $a_1=1, a_2=2$.
当 $n \geqslant 3$ 时, 考虑第 $n$ 个人重新排队后的位置, 共有两种可能:
若第 $n$ 个人仍排在第 $n$ 个位置, 则前 $n-1$ 人的排队方式数恰好为 $a_{n-1}$;
若第 $n$ 个人排在第 $n-1$ 个位置, 则第 $n-1$ 人必排在第 $n$ 个位置, 而前 $n-2$ 人的排队方式数恰好为 $a_{n-2}$.
因此 $a_n=a_{n-1}+a_{n-2}(n \geqslant 3)$.
由初值与递推式即得 $a_{12}=233$, 故共有 233 种不同的排队方法.
%%PROBLEM_END%%



%%PROBLEM_BEGIN%%
%%<PROBLEM>%%
问题2. $A, B$ 两人轮流掷一个骰子, 第一次由 $A$ 先掷, 若 $A$ 掷到一点, 下次仍由 $A$ 掷.
若 $A$ 掷不到一点, 下次换 $B$ 掷, 对 $B$ 同样适用规则.
如此依次投掷, 记第 $n$ 次由 $A$ 掷的概率为 $A_n$. 求 $A_n$.
%%<SOLUTION>%%
我们建立 $A_{n+1}$ 与 $A_n$ 之间的关系.
第 $n$ 次可能由 $A$ 来掷, 也可能由 $B$ 来掷, 第 $n$ 次若由 $A$ 来掷, 则第 $n+1$ 次仍由 $A$ 来掷, 这样的事件发生的概率是 $\frac{1}{6} A_n$, 若第 $n$ 次由 $B$ 来掷 (这样的概率是 $1-A_n$ ), 则第 $n+1$ 次由 $A$ 来掷, 这样的事件发生的概率是 $\frac{5}{6}\left(1-A_n\right)$.
故 $A_{n+1}=\frac{1}{6} A_n+\frac{5}{6}\left(1-A_n\right)=\frac{5}{6}-\frac{2}{3} A_n$, 且 $A_1=1$. 易得
$$
A_n=\frac{1}{2}\left(1+\left(-\frac{2}{3}\right)^{n-1}\right) \text {. }
$$
%%PROBLEM_END%%



%%PROBLEM_BEGIN%%
%%<PROBLEM>%%
问题3. 已知 $n$ 位数的各位数字只能取集合 $\{1,2,3,4,5\}$ 中的元素, 设含有数字 5 且在 5 的前面不含 3 的 $n$ 位数个数为 $f(n)$, 求 $f(n)$.
%%<SOLUTION>%%
这样的 $n$ 位数可由如下两种情况生成:
(1) 在不含数字 5 的 $n-1$ 位数前面加上 5 , 有 $4^{n-1}$ 个;
(2) 在含有数字 5 且 5 的前面不出现 3 的 $n-1$ 位数之前加上 $1,2,4,5$, 这样的数有 $4 f(n-1)$ 个.
故
$$
\begin{aligned}
& f(n)=4(n-1)+4^{n-1}, \text { 且 } f(1)=1, \\
& \frac{f(n)}{4^n}=\frac{f(n-1)}{4^{n-1}}+\frac{1}{4}, \\
& \frac{f(n)}{4^n}=\frac{1}{4}+\frac{1}{4}(n-1)=\frac{1}{4} n,
\end{aligned}
$$
从而 $f(n)=n \cdot 4^{n-1}$.
%%<REMARK>%%
注:本题若不用递推方法亦很方便求解.
%%PROBLEM_END%%



%%PROBLEM_BEGIN%%
%%<PROBLEM>%%
问题4. 试求将如下表达式去括号和合并同类项之后所得的多项式中 $x^2$ 的系数: $\left(\left(\cdots\left(\left((x-2)^2-2\right)^2-2\right)^2-\cdots-2\right)^2-2\right)^2$ (共 $n$ 重括号).
%%<SOLUTION>%%
记 $P_k(x)=\left(\left(\cdots\left(\left((x-2)^2-2\right)^2-2\right)^2-\cdots-2\right)^2-2\right)^2$ (共 $k$ 重括号), 其常数项等于 $P(0)=\left(\left(\cdots\left(\left((0-2)^2-2\right)^2-2\right)^2-\cdots-2\right)^2-2\right)^2=4$.
再以 $A_k$ 和 $B_k$ 表示 $P_k(x)$ 的一次项和二次项系数, 其中由于 $P_1(x)= (x-2)^2$, 故 $A_1=-4, B_1=1$.
下面寻找 $A_k$ 和 $B_k$ 的递推关系.
一方面, $P_k(x)=4+A_k x+B_k x^2+O(x)$ (其中 $O(x)$ 代表一个多项式, 它不含小于等于二次的项,下同); 另一方面, 当 $k \geqslant 2$ 时,
$$
\begin{aligned}
P_k(x) & =\left(P_{k-1}(x)-2\right)^2=\left(4+A_{k-1} x+B_{k-1} x^2+O(x)-2\right)^2 \\
& =4+4 A_{k-1} x+\left(4 B_{k-1}+A_{k-1}^2\right) x^2+O(x),
\end{aligned}
$$
比较两式中 $P_k(x)$ 的 $x, x^2$ 项系数可得:
$$
A_k=4 A_{k-1}, B_k=4 B_{k-1}+A_{k-1}^2(k \geqslant 2) .
$$
由此可先推出 $A_n=4^{n-1} A_1=-4^n$. 再将后一个递推式变形为 $\frac{B_k}{4^k}=\frac{B_{k-1}}{4^{k-1}}+\frac{A_{k-1}^2}{4^k}$, 故
$$
\begin{aligned}
\frac{B_n}{4^n} & =\frac{B_1}{4}+\frac{A_{n-1}^2}{4^n}+\frac{A_{n-2}^2}{4^{n-1}}+\cdots+\frac{A_1^2}{4^2} \\
& =\frac{1}{4}+4^{n-2}+4^{n-3}+\cdots+4^0=\frac{1}{4} \cdot \frac{4^n-1}{3} .
\end{aligned}
$$
因此 $P_n(x)$ 中 $x^2$ 的系数 $B_n=\frac{4^{n-1}\left(4^n-1\right)}{3}$.
%%PROBLEM_END%%



%%PROBLEM_BEGIN%%
%%<PROBLEM>%%
问题5. 一副纸牌共 52 张,其中"方块"、"梅花"、"红心"、"黑桃"每种花色的牌各 13 张, 标号依次是 $2,3, \cdots, 10, J, Q, K, A$, 其中相同花色、相邻标号的两张牌称为 "同花顺牌", 并且 $A$ 与 2 也算是顺牌 (即 $A$ 可以当成 1 使用). 试确定, 从这副牌中取出 13 张牌, 使每种标号的牌都出现, 并且不含 "同花顺牌"的取牌方法数.
%%<SOLUTION>%%
先一般化为下述问题: 设 $n \geqslant 3$, 从 $A=\left(a_1\right.$, $\left.a_2, \cdots, a_n\right), B=\left(b_1, b_2, \cdots, b_n\right), C=\left(c_1, c_2, \cdots, c_n\right)$, $D=\left(d_1, d_2, \cdots, d_n\right)$ 这四个数列中选取 $n$ 个项, 且满足:
(1) $1,2, \cdots, n$ 每个下标都出现;
(2) 下标相邻的任意两项不在同一个数列中(下标 $n$ 与 1 视为相邻).
设选取方法数为 $x_n$. 今确定 $x_n$ 的表达式: 将一个圆盘分成 $n$ 个扇形格, 顺次编号为 $1,2, \cdots, n$, 并将数列 $A, B, C, D$ 各对应一种颜色, 对于任意一个选项方案, 如果下标为 $i$ 的项取自某颜色数列, 则将第 $i$ 号扇形格染上该颜色.
于是 $x_n$ 就成为将圆盘的 $n$ 个扇形格染四色, 使相邻格不同色的染色方法数, 易知,
$$
x_1=4, x_2=12, x_n+x_{n-1}=4 \cdot 3^{n-1}(n \geqslant 3), \label{eq1}
$$
将 式\ref{eq1} 写作
$$
(-1)^n x_n-(-1)^{n-1} x_{n-1}=-4 \cdot(-3)^{n-1} \text {. }
$$
因此
$$
\begin{gathered}
(-1)^{n-1} x_{n-1}-(-1)^{n-2} x_{n-2}=-4 \cdot(-3)^{n-2} ; \\
\cdots \cdots \\
(-1)^3 x_3-(-1)^2 x_2=-4 \cdot(-3)^2 ; \\
(-1)^2 x_2=-4 \cdot(-3) .
\end{gathered}
$$
相加得, $(-1)^n x_n=(-3)^n+3$, 于是 $x_n=3^n+3 \cdot(-1)^n(n \geqslant 2)$.
因此 $x_{13}=3^{13}-3$. 这就是所求的取牌方法数.
%%<REMARK>%%
注:若将 4 种颜色推广到 $m$ 种颜色, 那么将圆盘的 $n$ 个扇形格染 $m$ 种颜色,使相邻格不同色的染色方法数为 $(m-1)^n+(-1)^n(m-1)$.
%%PROBLEM_END%%



%%PROBLEM_BEGIN%%
%%<PROBLEM>%%
问题6. 设非空集合 $S \subseteq \mathbf{N}^*$ 满足: 若 $a, b \in S$ ( $a, b$ 可以相同), 则 $a b+3 \in S$. 证明: $S$ 中含有无穷多个 7 的倍数.
%%<SOLUTION>%%
显然 $S$ 中不存在最大元素, 故 $S$ 是无限集.
记 $A_k=\left\{x \in \mathbf{N}^* \mid x \equiv k(\bmod 7)\right\}(k=0,1, \cdots, 6)$, 并将命题 " $S \cap A_k$ 为无限集"记为 $P_k$. 显然 $P_0$. 即为题目需证之结论.
先证明一个引理: 若 $P_k$ 成立, 则 $P_{k^2+3}$ 成立 (下标在模 7 的意义下理解).
证明: 若 $P_k$ 成立, 则存在一列正整数 $a_1<a_2<\cdots<a_n<\cdots$, 满足 $a_1 \equiv a_2 \equiv \cdots \equiv k(\bmod 7)$, 则
$$
a_1 a_i+3 \equiv k^2+3(\bmod 7), a_1 a_i+3<a_1 a_{i+1}+3\left(i \in \mathbf{N}^*\right),
$$
故 $S \cap A_{k^2+3}$ 含有无限多个元素, 即 $P_{k^2+3}$ 成立.
在原题中, 由于 $S$ 是无限集, 根据抽屉原理, 命题 $P_0, P_1, \cdots, P_6$ 中必有一个成立.
反复利用引理可得: $P_1 \Rightarrow P_4 \Rightarrow P_5 \Rightarrow P_0$, 而 $P_2 \Rightarrow P_0, P_3 \Rightarrow P_5$, $P_6 \Rightarrow P_4$, 故 $P_0$ 总成立, 即 $S$ 中必含有无穷多个 7 的倍数.
%%PROBLEM_END%%



%%PROBLEM_BEGIN%%
%%<PROBLEM>%%
问题7. 一个 $m \times n$ 的矩形, 每个单位正方形要么被染为黑色, 要么被染为白色.
对一个黑色的单位正方形, 若在其所在行中, 左边有某个单位正方形是白色的,且在其所在列中,上边有某个单位正方形是白色的,则称这个黑色的单位正方形为 "搁浅的". 图(<FilePath:./figures/fig-c13p7.png>)中给出的是一个 $4 \times 5$ 的矩形, 且没有搁浅的黑格.
求没有搁浅的黑格的 $2 \times n$ 的矩形的数目.
%%<SOLUTION>%%
记有搁浅黑格的 $2 \times n$ 矩形的数目为 $a_n$, 无搁浅黑格的 $2 \times n$ 矩形的数目为 $b_n$, 则 $b_n=2^{2 n}-a_n$, 且 $a_1=0$.
由题意知, 搁浅的黑格只可能位于第 2 行.
(1) 若第 2 行前 $n-1$ 个方格中已有搁浅的黑格, 则第 $n$ 列无论怎样染色, 均使 $2 \times n$ 矩形中存在搁浅的黑格, 此时共有 $2^2 a_{n-1}=4 a_{n-1}$ 种情况.
(2) 若第 2 行前 $n-1$ 个方格中无搁浅的黑格,则第 2 行第 $n$ 列的方格必为 $2 \times n$ 矩形中唯一的挌浅的黑格,所以它上面的方格确定为白格, 且第 2 行前 $n-1$ 个方格中存在白格, 而在第 2 行前 $n-1$ 个方格均为黑格的情况下,第一行任意一种染色方法均可使前 $n-1$ 列无搁浅的黑格, 所以这里 $2^{n-1}$ 种情况需要排除,故共有 $b_{n-1}-2^{n-1}=2^{2(n-1)}-a_{n-1}-2^{n-1}$ 种情况.
$$
\text { 综合 (1)、(2) 知 } a_n=4 a_{n-1}+2^{2(n-1)}-a_{n-1}-2^{n-1}=3 a_{n-1}+4^{n-1}-2^{n-1} \text {. }
$$
即
$$
\frac{a_n}{3^{n-1}}=\frac{a_{n-1}}{3^{n-2}}+\left(\frac{4}{3}\right)^{n-1}-\left(\frac{2}{3}\right)^{n-1}
$$
将该式中的 $n$ 分别代换成 $2,3, \cdots, n$ 并累加, 注意 $a_1=0$, 有
$$
\begin{aligned}
\frac{a_n}{3^{n-1}} & =\sum_{k=2}^n\left(\frac{4}{3}\right)^{n-1}-\sum_{k=2}^n\left(\frac{2}{3}\right)^{n-1} \\
& =4 \times\left(\left(\frac{4}{3}\right)^{n-1}-1\right)-2 \times\left(1-\left(\frac{2}{3}\right)^{n-1}\right)=\frac{4^n+2^n}{3^{n-1}}-6,
\end{aligned}
$$
即 $a_n=4^n+2^n-6 \times 3^{n-1}=4^n+2^n-2 \times 3^n$.
从而 $b_n=2^{2 n}-a_n=2 \times 3^n-2^n$.
%%PROBLEM_END%%



%%PROBLEM_BEGIN%%
%%<PROBLEM>%%
问题8. 桌上放着 4 堆火柴,两人依次轮流做取火柴游戏: 游戏者每次任意取走其中一堆, 并把余下两堆中的任意一堆分成非空的两堆.
谁无法这样做, 就算输了.
假如一开始 4 堆火柴中有 3 堆是 2 根, 另一堆是 $n$ 根, 求正整数 $n$ 的所有可能值,使先取火柴的一方有必胜策略.
%%<SOLUTION>%%
首先可验证 $n=1,2,5,6, \cdots$ 时不满足题意, $n=3,4,7,8, \cdots$ 时满足题意, 故猜想当且仅当 $n \equiv 0,3(\bmod 4)$ 时, 先取方有必胜策略.
用四元正整数组表示 4 堆火柴的根数 (若仅是顺序不同, 则认为是同一数组), 并将游戏的胜状态和负状态集合分别记为 $W$ 和 $L$.
我们先证引理: 当正整数组被写成 $P=\left(2^{n_1} a_1, 2^{n_2} a_2, 2^{n_3} a_3, 2^{n_4} a_4\right)$
的形式(其中 $n_i \in \mathbf{N}, a_i$ 为奇数, $i=1,2,3,4$ ) 时, 若存在两两不同的 $i$, $j, k$, 使 $n_i=n_j<n_k$ (情形 1 ), 则 $P \in W$; 若 $n_1=n_2=n_3=n_4$ (情形 2 ), 则 $P \in L$.
事实上, 在情形 1 下, 可取走非 $i, j, k$ 的那堆, 并将第 $k$ 堆拆成根数分别为 $2^{n_i}, 2^{n_i}\left(2^{n_k-n_i} a_k-1\right)$ 的两堆, 此时 $a_i, a_j, 1,2^{n_k-n_i} a_k-1$ 均为奇数,故产生的数组属于情形 2 ; 在情形 2 下, 设 $n_1=n_2=n_3=n_4=n$, 不妨设取走第 1 堆, 并将第 2 堆拆成根数为 $2^{m_1} b_1, 2^{m_2} b_2$ 的两堆 ( $m_1, m_2 \in \mathbf{N}, b_1, b_2$ 为奇数), 考虑到 $a_2$ 是奇数且 $2^{m_1} b_1+2^{m_2} b_2=2^n a_2$, 故要么 $m_1=m_2<n$, 要么 $m_1=n< m_2$ 或 $m_2=n<m_1$, 无论如何总变为情形 1 .
由于情形 1 下总能进行操作且游戏必在有限步内结束, 递归可知引理成立.
下面再证明: 对任意 $k, l \in \mathbf{N}^*$, 若 $P=(2,2,4 k, l),(2,2,4 k-1$, l) (情形 3 ), 则 $P \in W$; 若 $P=(2,2,2,4 k-2),(2,2,2,4 k-3$ ) (情形 4), 则 $P \in L$.
事实上, 在情形 3 下, 可取走第 4 堆, 并将第 3 堆拆成根数分别为 $2,4 k-$ 2 或 $2,4 k-3$ 的两堆, 化为情形 4 . 而在情形 4 下, 我们证明任何操作不是得到一个胜状态就是回到情形 3 : 若是将前三堆中的某堆拆开, 则产生数组 $(1,1$, $2, m)$, 据引理知 $(1,1,2, m) \in W$. 若在前三堆中取走某一堆, 而拆开第 4 堆, 那么产生数组 $(2,2, c, d)$. 此时若 $c, d$ 中有某个模 4 余 3 , 则回到情形 3 ; 若 $c$ 和 $d$ 模 4 均余 1 或 2 , 由于 $c+d \equiv 1,2(\bmod 4)$, 只可能是 $c, d \equiv 1(\bmod 4)$ 的情况, 由引理知 $(2,2, c, d) \in W$.
根据递归可知结论成立.
从该结论易看出, 当且仅当 $n \equiv 0,3(\bmod 4)$ 时, 先取方有必胜策略.
%%PROBLEM_END%%


