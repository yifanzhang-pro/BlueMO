
%%PROBLEM_BEGIN%%
%%<PROBLEM>%%
问题1. 设非负实数 $x_1,x_2,\cdots,x_n$ 满足 $x_{1}+x_{2}+\cdots+x_{n}\leq{\frac{1}{2}}$ ,求$(1-x_1)(1-x_2)\cdots(1-x_{n})$ 的最小值
%%<SOLUTION>%%
先固定 $x_1, x_2, \cdots, x_{n-2}$, 注意到 $\left(1-x_{n-1}\right)\left(1-x_n\right)=1-x_{n-1}-x_n+ 222 x_{n-1} x_n$, 其中 $x_{n-1}+x_n$ 为定值, 所以 $x_{n-1} x_n$ 的值越小则原式值越小, 故令
$$
x_i^{\prime}=x_i, i==1,2, \cdots, n-2, x_{n-1}^{\prime}=x_{n-1}+x_n, x_n^{\prime}=0,
$$
此时 $x_{n-1}^{\prime} x_n^{\prime}=0 \leqslant x_{n-1} x_n$, 所以
$$
\left(1-x_1^{\prime}\right)\left(1-x_2^{\prime}\right) \cdots\left(1-x_{n-1}^{\prime}\right) \leqslant\left(1-x_1\right)\left(1-x_2\right) \cdots\left(1-x_n\right),
$$
其中 $x_1^{\prime}+x_2^{\prime}+\cdots+x_{n-1}^{\prime}=x_1+x_2+\cdots+x_n \leqslant \frac{1}{2}$.
再进行 $n-2$ 次类似的调整过程可知
$$
\left(1-x_1\right)\left(1-x_2\right) \cdots\left(1-x_n\right) \geqslant 1-\left(x_1+x_2+\cdots+x_n\right) \geqslant \frac{1}{2},
$$
等号当 $x_1=\frac{1}{2}, x_2=x_3=\cdots=x_n=0$ 时可取到, 所以所求最小值为 $\frac{1}{2}$.
%%PROBLEM_END%%



%%PROBLEM_BEGIN%%
%%<PROBLEM>%%
问题2. 设 $x_1, x_2, \cdots, x_{10}$ 是正整数,且满足
$$
x_1+x_2+\cdots+x_{10}=49,
$$
求 $x_1^2+x_2^2+\cdots+x_{10}^2$ 的最大值和最小值.
%%<SOLUTION>%%
由于把 49 写成 10 个正整数的和, 写法只有有限种, 所以一定有一种使得 $x_1^2+x_2^2+\cdots+x_{10}^2$ 达到最大值, 也一定有一种使得 $x_1^2+x_2^2+\cdots+x_{10}^2$ 达到最小值.
假设 $x_1 \leqslant x_2 \leqslant \cdots \leqslant x_{10}$ 满足 $x_1+x_2+\cdots+x_{10}=49$, 且使得 $x_1^2+x_2^2+\cdots +x_{10}^2$ 达到最大值, 若 $x_1>1$, 取 $y_1=x_1-1, y_2=x_2+1, y_k=x_k, k=3,4$,
$\cdots, 10$, 则
$$
y_1+y_2+\cdots+y_{10}=49,
$$
且 $y_1^2+y_2^2=\left(x_1-1\right)^2+\left(x_2+1\right)^2=x_1^2+x_2^2+2\left(x_2-x_1\right)+2>x_1^2+x_2^2$, 从而 $y_1^2+y_2^2+\cdots+y_{10}^2>x_1^2+x_2^2+\cdots+x_{10}^2$, 矛盾.
所以 $x_1=1$, 进而 $x_2=1, \cdots, x_9=1$, 所以 $x_{10}=40$. 于是 $x_1^2+ x_2^2+\cdots+x_{10}^2$ 的最大值为 $9+40^2=1609$.
假设 $x_1 \leqslant x_2 \leqslant \cdots \leqslant x_{10}$ 是满足 $x_1+x_2+\cdots+x_{10}=49$, 且使得 $x_1^2+ x_2^2+\cdots+x_{10}^2$ 达到最小值, 则 $x_1, x_2, \cdots, x_{10}$ 中任意两个数的差的绝对值不超过 1 .
事实上, 若存在 $x_i, x_j, 1 \leqslant i \leqslant j \leqslant 10$, 有 $x_j-x_i \geqslant 2$, 令
$$
y_i=x_i+1, y_j=x_j-1, x_k=y_k, k \neq i, j
$$
则 $y_i^2+y_j^2=\left(x_i+1\right)^2+\left(x_j-1\right)^2=x_i^2+x_j^2-2\left(x_j-x_i\right)+2<x_i^2+x_j^2$, 矛盾.
所以,当 $x_1=4, x_2==x_3=\cdots=x_{10}=5$ 时, $x_1^2+x_2^2+\cdots+x_{10}^2$ 取得最小值 241.
%%PROBLEM_END%%



%%PROBLEM_BEGIN%%
%%<PROBLEM>%%
问题3. 把 2006 分成若干个互不相等的正整数的和, 且使得这正整数的乘积最大,求出该乘积.
%%<SOLUTION>%%
由于把 2006 分成若干个互不相等的正整数的和的分法只有有限种, 因而一定存在一种分法,使得这些正整数的乘积最大.
若把 1 作为因子, 乘积显然不会最大 (只需将 1 加到另一个因子上去即可). 把 2006 分成若干个互不相等的正整数的和, 因子个数越多, 乘积越大, 为了使因子个数尽可能地多, 我们把 2006 分成 $2+3+4+\cdots+n$ 直到和不小于 2006.
如果和比 2006 大 1 , 这时, 因子个数至少减少 1 个, 为了使乘积最大, 应去掉最小的 2 , 并将最后一个数 (最大) 加上 1 .
如果和比 2006 大 $m(m \neq 1)$, 那么去掉等于 $m$ 的那个数, 便可使乘积最大.
令
$$
\begin{gathered}
2+3+4+\cdots+n \geqslant 2006, \\
\frac{n(n+1)}{2}-1 \geqslant 2006, \\
n^2+n-4014 \geqslant 0 . \label{eq1}
\end{gathered}
$$
由于 $n$ 是满足不等式 \ref{eq1} 的最小正整数, 所以 $n=63$ (因 $62^2+62-4014= \left.-108<0,63^2+63-4014=18>0\right)$. 这时
$$
\begin{gathered}
2+3+4+\cdots+63=2015, \\
2015-2006=9 .
\end{gathered}
$$
所以,把 2006 分成
$$
(2+3+\cdots+8)+(10+11+\cdots+63),
$$
这一形式时, 这些数的乘积最大, 其积为
$$
2 \times 3 \times \cdots \times 8 \times 10 \times \cdots \times 63=\frac{63 !}{9} .
$$
%%PROBLEM_END%%



%%PROBLEM_BEGIN%%
%%<PROBLEM>%%
问题4. 设 $\pi(n)$ 表示不大于 $n$ 的素数的个数.
求证: 对任意正整数 $n$ 和非负整数 $k \leqslant \pi(n)$, 总存在 $n$ 个连续正整数, 其中恰含有 $k$ 个素数.
%%<SOLUTION>%%
对正整数 $m$, 定义 $f(m)$ 为 $m, m+1, \cdots, m+n-1$ 这 $n$ 个连续正整数中素数的个数.
显然 $f(1)=\pi(n)$.
取 $(n+1) !+2,(n+1) !+3, \cdots,(n+1) !+(n+1)$ 这 $n$ 个连续正整数, 它们都是合数, 即 $f((n+1) !+2)=0$. 所以
$$
f((n+1) !+2) \leqslant k \leqslant \pi(n)=f(1) . \label{eq1}
$$
另一方面,考虑 $f(m+1)$ 与 $f(m)$ 的关系:
当 $m, m+n$ 都为素数或都为合数时,有 $f(m+1)=f(m)$;
当 $m$ 为素数且 $m+n$ 为合数时,有 $f(m+1)=f(m)-1$;
当 $m$ 为合数且 $m+n$ 为素数时,有 $f(m+1)=f(m)+1$.
总之 $|f(m+1)-f(m)| \leqslant 1$, 结合 \ref{eq1} 式知必有某个 $i \in\{1,2, \cdots,(n+1) !+2\}$ 使得 $f(i)=k$, 即连续 $n$ 个正整数 $i, i+1, \cdots, i+n-1$ 中恰含有 $k$ 个素数.
%%PROBLEM_END%%



%%PROBLEM_BEGIN%%
%%<PROBLEM>%%
问题5. 已知条件组 $\left\{\begin{array}{l}x_1+x_2+\cdots+x_n=m, \\ x_1^2+x_2^2+\cdots+x_n^2-\frac{m^2}{n}<2 .\end{array}\right.$
(1) 求所有正整数 $n$, 使条件组对一切正整数 $m$ 都有整数解 $\left(x_1, x_2, \cdots\right.$, $\left.x_n\right)$;
(2) 求一切正整数组 $(m, n)$, 使得条件组存在整数解 $\left(x_1, x_2, \cdots, x_n\right)$.
%%<SOLUTION>%%
引理: 令 $S=\sum_{i=1}^n x_i^2-\frac{m^2}{n}$. 当 $S$ 取最小值时, 对任意 $i, j$, 都有 $\left|x_i-x_j\right| \leqslant 1$.
证明: 假定存在情形 $x_1 \geqslant x_2+2$. 令 $x_1^{\prime}=x_1-1, x_2^{\prime}=x_2+1, x_3^{\prime}=x_3$,
$\cdots, x_n^{\prime}=x_n, m^{\prime}=\sum_{i=1}^n x_i^{\prime}, S^{\prime}=\sum_{i=1}^n x_i^{\prime 2}-\frac{m^{\prime 2}}{n}$, 则 $m^{\prime}=m, S^{\prime}-S=\sum_{i=1}^n\left(x_i^{\prime 2}-\right. \left.x_i^2\right)=x_1^{\prime 2}-x_1^2+x_2^{\prime 2}-x_2^2=\left(x_1-1\right)^2-x_1^2+\left(x_2+1\right)^2-x_2^2=2-2 x_1+ 2 x_2<0$. 故在 $m$ 不变的前提下, 通过调整可使 $S$ 取到更小的 $S^{\prime}$, 又这样的调整只能做有限次, 故 $S$ 可取最小值, 且此时任何 $x_i, x_j$ 之差不超过 1 . 引理证毕.
下面对 (1)、(2)一并解决.
设 $m=n k-t, k \in \mathbf{N}^*, t \in\{0,1, \cdots, n-1\}$.
设 $x_1=x_2=\cdots=x_t=k-1, x_{t+1}=x_{t+2}=\cdots=x_n=k$, 它们的和恰为 $m$, 且由引理知, 已使 $S$ 尽可能小 (由于其他形式的一切解 $\left(x_1, x_2, \cdots, x_n\right)$ 都对应一组这样的调整后的解, 所以此设法不失一般性). 此时
$$
\begin{gathered}
S==\sum_{i=1}^n x_i^2-\frac{m^2}{n}=\sum_{i=1}^n x_i^2-\frac{1}{n} \sum_{i=1}^n x_i^2-\frac{2}{n} \sum_{1 \leqslant i<j \leqslant n} x_i x_j=\frac{1}{n} \sum_{1 \leqslant i<j \leqslant n}\left(x_i-x_j\right)^2, \\
t(n-t)=\sum_{1 \leqslant i<j \leqslant n}\left(x_i-x_j\right)^2=n S<2 n,
\end{gathered}
$$
故 $f(t)=t^2-n t+2 n>0$.
当 $n \leqslant 7$ 时, 对一切 $t$, 都成立 $f(t)=\left(t-\frac{n}{2}\right)^2+\frac{n(8-n)}{4}>0$, 即对一切正整数 $m$, 条件组都存在整数解.
当 $n \geqslant 8$ 时, $f(4)=16-2 n \leqslant 0$, 故对 $m=n k-4$, 无整数解满足.
这样,第 (1) 问的结果为 $n=1,2, \cdots, 7$.
此外, 对 $n \geqslant 8$ 寻求别的正整数组 $(m, n)$.
经算得: $f(0)=2 n>0, f(1)=f(n-1)=n+1>0, f(2)=f(n-$ 2) $=4>0$.
当 $n=8$ 时, $f(3)=f(5)=1>0$, 故可取 $t=0,1,2,3,5,6,7$.
当 $n \geqslant 9$ 时, $f(3)=f(n-3)=9-n \leqslant 0$, 由二次函数性质知, 当 $t \in(3$, $n-3)$ 时, $f(t)<\max \{f(3), f(n-3)\} \leqslant 0$, 故恰可取 $t=0,1,2, n-2$, $n-1$. 结合所设的 $m=n k-t, k=1,2, \cdots, t \in\{0,1, \cdots, n-1\}$ 可知, 第 (2) 问的一切正整数组 $(m, n)$ 有如下 3 类:
(1) $(a, 1),(a, 2), \cdots,(a, 7)$, 其中 $a$ 为任意正整数;
(2) $(b k-t, b)$, 其中 $b \geqslant 8, k$ 为任意正整数, $t=0,1,2, b-2, b-1$;
(3) $(8 k-3,8),(8 k-5,8)$, 其中 $k$ 为任意正整数.
而 " $x_1=x_2=\cdots=x_t=k-1, x_{t+1}=x_{t+2}=\cdots=x_n=k$ " 保证了每组 $(m, n)$ 都对应了一组具体的解 $\left(x_1, x_2, \cdots, x_n\right)$.
%%PROBLEM_END%%



%%PROBLEM_BEGIN%%
%%<PROBLEM>%%
问题6. 空间中有 1989 个点, 其中任意三点不共线, 把它们分成点数互不相同的 30 组,在任何三个不同的组中各取一点为顶点作三角形,要使这种三角形的总数最大,各组的点数应为多少?
%%<SOLUTION>%%
设这 1989 个点分成的 30 组中, 每组的点数分别为 $n_1, n_2, \cdots, n_{30}$, 且不妨设 $n_1<n_2<\cdots<n_{30}$. 按题设组成的三角形总数为
$$
S=\sum_{1 \leqslant i<j<k \leqslant 30} n_i n_j n_k
$$
本题是要在 $\sum_{i=1}^{30} n_i=30$, 且 $n_1, n_2, \cdots, n_{30}$ 互不相同的约束条件下, 求 $S$的最大值.
由于把 1989 个点分成 30 组的不同分法只有有限种, 故必有一种分法使得 $S$ 取到最大, 即 $S$ 的最大值是存在的.
下面我们用逐步调整方法来求得 $S$的最大值.
(1) 欲使 $S$ 最大, 相邻两个点组的点数之差 $n_{i+1}-n_i(i=1,2, \cdots, 29)$ 均不超过 2. 如果有 $i_0$, 使得 $n_{i_0+1}-n_{i_0} \geqslant 3$, 不妨设 $i_0=1$. 于是
$$
S=n_1 n_2 \sum_{k=3}^{30} n_k+\left(n_1+n_2\right) \sum_{3 \leqslant j<k \leqslant 30} n_j n_k+\sum_{3 \leqslant i<j<k \leqslant 30} n_i n_j n_k .
$$
令 $n_1^{\prime}=n_1+1, n_2^{\prime}=n_2-1$, 因为
$$
n_1^{\prime} n_2^{\prime}=\left(n_1+1\right)\left(n_2-1\right)=n_1 n_2+\left(n_2-n_1-1\right)>n_1 n_2,
$$
所以用 $n_1^{\prime}, n_2^{\prime}$ 代替 $n_1, n_2$ 时, $S$ 中的第一项 $n_1 n_2 \sum_{k=3}^{30} n_k$ 变大, 而后面两项不变, 从而 $S$ 的值增大,矛盾.
所以, $n_{i+1}-n_i \leqslant 2(i=1,2, \cdots, 29)$.
(2) 欲使 $S$ 最大, $n_{i+1}-n_i=2$ 的 $i$ 值至多只有一个, 若不然, 设 $n_{i+1}- n_i=2, n_{j_0+1}-n_{j_0}=2$. 那么作如下调整: 用 $n_{i_0}^{\prime}=n_{i_0}+1$ 及 $n_{j_0+1}^{\prime}=n_{j_0+1}-1$ 代替 $n_{i_0} 、 n_{j_0+1}, S$ 便会增大.
所以使得 $n_{i+1}-n_i=2$ 的下标 $i$ 最多只有一个.
(3) 如果 $n_1, n_2, \cdots, n_{30}$ 组成公差为 1 的等差数列, 那么
$$
n_1+n_2+\cdots+n_{30}=15 \times\left(2 n_1+29\right)=1989,
$$
而 15 不整除 1989 , 故上式不成立, 因此 $n_1, n_2, \cdots, n_{30}$ 中, 必有两项的差为 2 . 现设各组点数为
$$
n_1, n_1+1, \cdots, n_1+i_0-1, n_1+i_0+1, \cdots, n_1+30 .
$$
其中 $1 \leqslant i_0 \leqslant 29$, 它们的总和为 $1989=\left(2 n_1+30\right) \times 31-\left(n_1+i_0\right)$, 所以
$$
30 n_1-i_0=1524 \text {. }
$$
易知 $n_1=51, i_0=6$. 所以, 当 30 组点数依次为 $51,52, \cdots, 56,58$, $59, \cdots, 81$ 时, $S$ 最大.
%%PROBLEM_END%%



%%PROBLEM_BEGIN%%
%%<PROBLEM>%%
问题7. 设正实数 $a, b, c, d$ 满足 $a b c d=1$, 求证:
$$
\frac{1}{a}+\frac{1}{b}+\frac{1}{c}+\frac{1}{d}+\frac{9}{a+b+c+d} \geqslant \frac{25}{4} .
$$
%%<SOLUTION>%%
不妨设 $a \leqslant b \leqslant c \leqslant d$, 并记
$$
f(a, b, c, d)=\frac{1}{a}+\frac{1}{b}+\frac{1}{c}+\frac{1}{d}+\frac{9}{a+b+c+d} .
$$
先往证: $\quad f(a, b, c, d) \geqslant f(\sqrt{a c}, b, \sqrt{a c}, d)$. \label{eq1}
事实上,上式等价于
$$
\begin{aligned}
& \frac{1}{a}+\frac{1}{c}+\frac{9}{a+b+c+d} \geqslant \frac{1}{\sqrt{a c}}+\frac{1}{\sqrt{a c}}+\frac{9}{2 \sqrt{a c}+b+d} \\
\Leftrightarrow & \frac{(\sqrt{a}-\sqrt{c})^2}{a c} \geqslant \frac{9(\sqrt{a}-\sqrt{c})^2}{(a+b+c+d)(2 \sqrt{a c}+b+d)} \\
\Leftarrow & \left.(a+b+c+d)(2 \sqrt{a c}+b+d) \geqslant 9 a c \text { (因为 }(\sqrt{a}-\sqrt{c})^2 \geqslant 0\right) \\
\Leftarrow & \left(a+c+\frac{2}{\sqrt{a c}}\right)\left(2 \sqrt{a c}+\frac{2}{\sqrt{a c}}\right) \geqslant 9 a c \text { (因为 } b+d \geqslant 2 \sqrt{b d}=\frac{2}{\sqrt{a c}} \text { ) } \label{eq2}
\end{aligned}
$$
而 $1=a b c d \geqslant a \cdot a \cdot c \cdot c \Rightarrow a c \leqslant 1 \Rightarrow \frac{2}{\sqrt{a c}} \geqslant 2 \sqrt{a c}$. 且 $a+c \geqslant 2 \sqrt{a c}$, 故
式\ref{eq2} 左边 $\geqslant\left(2 \sqrt{a c}+\frac{2}{\sqrt{a c}}\right)\left(2 \sqrt{a c}+\frac{2}{\sqrt{a c}}\right) \geqslant 4 \sqrt{a c} \cdot 4 \sqrt{a c}= 16 a c>9 a c=$ 式\ref{eq2} 右边.
所以 式\ref{eq1} 成立.
式\ref{eq1} 说明, $f(a, b, c, d)$ (其中 $a \leqslant b \leqslant c \leqslant d$ ) 的最小值 (或极小值) 总是在 $a=c$, 即 $a=b=c$ 时取得.
欲得到该四元函数的下界, 我们就可不妨设 $(a, b$, $c, d)=\left(\frac{1}{t}, \frac{1}{t}, \frac{1}{t}, t^3\right)$, 这里 $t \geqslant 1$; 这也说明了只需证明对 $\forall t \geqslant 1$, 总有
$$
f\left(\frac{1}{t}, \frac{1}{t}, \frac{1}{t}, t^3\right) \geqslant \frac{25}{4}, \label{eq3}
$$
就证明了原不等式成立.
$$
\begin{aligned}
& f\left(\frac{1}{t}, \frac{1}{t}, \frac{1}{t}, t^3\right) \geqslant \frac{25}{4} \\
& \Leftrightarrow 3 t+\frac{1}{t^3}+\frac{9}{t^3+\frac{3}{t}} \geqslant \frac{25}{4} \\
& \Leftrightarrow 12 t^8-25 t^7+76 t^4-75 t^3+12 \geqslant 0 \\
& \Leftrightarrow(t-1)^2\left(12 t^6-t^5-14 t^4-27 t^3+36 t^2+24 t+12\right) \geqslant 0 \\
& \Leftrightarrow 12 t^6-t^5-14 t^4-27 t^3+36 t^2+24 t+12 \geqslant 0 \\
& \Leftrightarrow(t-1)\left(12 t^5+11 t^4-3 t^3-30 t^2+6 t+30\right)+42 \geqslant 0 . \label{eq4} \\ 
& \text { 而 } t \geqslant 1,12 t^5+6 t \geqslant 2 \sqrt{12 t^5 \cdot 6 t}=12 \sqrt{2} t^3>3 t^3, \\
& \qquad 11 t^4+30 \geqslant 2 \sqrt{11 t^4 \cdot 30}=2 \sqrt{33} \overline{0} t^2>30 t^2 .
\end{aligned}
$$
而 $t \geqslant 1,12 t^5+6 t \geqslant 2 \sqrt{12 t^5 \cdot 6 t}=12 \sqrt{2} t^3>3 t^3$,
$$
11 t^4+30 \geqslant 2 \sqrt{11 t^4 \cdot 30}=2 \sqrt{33} \overline{0} t^2>30 t^2 .
$$
故 $(t-1)\left(12 t^5+11 t^4-3 t^3-30 t^2+6 t+30\right)+42>0$, 式\ref{eq4} 成立.
至此, 式\ref{eq3} 成立, 原不等式得证.
%%PROBLEM_END%%


