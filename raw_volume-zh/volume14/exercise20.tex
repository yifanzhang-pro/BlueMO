
%%PROBLEM_BEGIN%%
%%<PROBLEM>%%
问题1. 证明: 任何一群人中, 至少有两个人, 它们的朋友数目相同.
%%<SOLUTION>%%
设任意给定的一群人有 $n$ 个.
用顶点表示这 $n$ 个人.
当且仅当顶点 $u$, $v$ 表示的两个人是朋友时令 $u, v$ 相邻,得到 $n$ 个顶点的简单图 $G$.
$G$ 中任意顶点 $x$ 的度 $d(x)$ 满足 $0 \leqslant d(x) \leqslant n-1$. 如果图 $G$ 的顶点的度都不相同, 则图 $G$ 具有 0 度顶点 $u$ 和 $n-1$ 度顶点 $v . n-1$ 度顶点和 $G$ 中其他顶点都相邻, 特别地和 0 度顶点 $u$ 相邻, 矛盾.
这就证明了 $G$ 中必定有两个顶点, 它们的度相同.
也就是说, 这群人中必有两个人, 他们的朋友一样多.
%%PROBLEM_END%%



%%PROBLEM_BEGIN%%
%%<PROBLEM>%%
问题2. 㩊是否存在这样的多面体, 它有奇数个面, 每个面有奇数条棱?
%%<SOLUTION>%%
不存在这样的多面体.
事实上, 如果这样的多面体存在, 那么用顶点表示这个多面体的面, 并且仅当 $v_i, v_j$. 所代表的两个面有公共棱时, 在图 $G$ 相应的两顶点之间连一条边, 依题意, 每个 $d(v)$ 是奇数, 于是奇数个顶点度数之和也是奇数,这与定理相违.
证毕.
%%PROBLEM_END%%



%%PROBLEM_BEGIN%%
%%<PROBLEM>%%
问题3. 有一个团体会议,有 100 人参加.
其中任意四个人都至少有一个人认识三人.
问: 该团体中认识其他所有人的成员最少有多少?
%%<SOLUTION>%%
把该团体的成员视为顶点,其顶点全体记做 $V$. 对于任意两个顶点 $u$, $v$ 所代表的成员, 当且仅当彼此认识, 则在 $u, v$ 之间连一条边, 得到一个含 100 个顶点的简单图 $G$. 已知条件是, 图 $G$ 中任意四个顶点中都至少有一顶点和其他三个顶点相邻.
要求图 $G$ 中度为 99 的顶点个数的最小值 $m$.
当图 $G$ 是完全图时, 每个顶点的度都是 99 , 所以有 100 个度为 99 的顶点.
当图 $G$ 是非完全图时, 图 $G$ 中必有两个不相邻的顶点 $u$ 和 $v$. 如果除 $u$ 和 $v$ 外另有两个顶点 $x, y$ 不相邻, 则 $u, v, x$ 和 $y$ 中不存在和其他三个顶点都相邻的顶点, 与题意矛盾 (与图 $G$ 的性质矛盾). 因此 $G$ 中除 $u, v$ 外任意两个顶点相邻.
此时, 如果 $G$ 中除 $u 、 v$ 外的任何 $x$ 都和 $u, v$ 相邻, 则 $d(x)=99$, 即 $G$ 中度为 99 的顶点个数为 98 . 设 $G$ 中除 $u$ 、 $v$ 外有个顶点 $x$ 和 $u 、 v$ 不都相邻, 则由 $G$ 的性质知, $G$ 中除 $u, v, x$ 外的任意顶点 $y$ 和 $u 、 v 、 x$ 都相邻.
因此 $d(u) \leqslant 98, d(v) \leqslant 98, d(x) \leqslant 98, d(y)=99$. 所以 $G$ 中度为 99 的顶点个数为 97 .
这表明图 $G$ 中度为 99 的顶点个数的最小值为 97 .
回到原问题,即得: 该团体中认识其他所有人的成员最少是 97 个.
%%<REMARK>%%
注:例题中的成员数 100 改为任意的 $n(n \geqslant 4)$, 其他条件不变,则结论为该团体至少有 $n-3$ 人认识其他所有人.
%%PROBLEM_END%%



%%PROBLEM_BEGIN%%
%%<PROBLEM>%%
问题4. 证明: 在任何 5 个无理数中, 总可以选出 3 个数, 使它们两两之和为无理数.
%%<SOLUTION>%%
将 5 个数看成 5 个顶点, 当且仅当两数之和为有理数时, 令相应的两个顶点相邻, 得一个简单图 $G$, 只需证明 $G$ 中存在 3 个顶点两两不相邻即可.
首先说明图中不存在三角形 (即长为 3 的圈). 事实上, 若顶点 $x, y, z$ 两两相邻, 那么 $x+y, y+z, z+x$ 都是有理数, 从而 $x, y, z$ 都是有理数, 与已知矛盾.
同理可知图中无五边形 (即长为 5 的圈).
若某个顶点 $x$ 与至少三个顶点 $y, z, u$ 相邻, 则 $y, z, u$ 两两不相邻 (否则将产生以 $x$ 为一个顶点的三角形), 这三个顶点即为所求.
若某个顶点 $x$ 与至多一个顶点 $v$ 相邻, 那么由于其余三个顶点 $y, z, u$ 不构成三角形, 所以必有两点, 例如 $y, z$ 不相邻, 则 $x, y, z$ 即为所求.
假如以上两种情况都不满足, 那么每个顶点恰好是两条边的端点, 由一笔画理论易知这个图是一个长为 5 的圈, 矛盾! 故这种情况不会发生.
综上所述,必有三个顶点两两不相邻,故命题得证.
%%<REMARK>%%
注:实际上, 这里相当于证明了: 对 5 阶完全图 $K_5$ 所有的边染 $A, B$ 两种颜色之一, 若不存在颜色 $A$ 的五边形, 则必存在同色三角形.
%%PROBLEM_END%%



%%PROBLEM_BEGIN%%
%%<PROBLEM>%%
问题5. 一个给定圆周上有 13 个点.
能否用数字 $1,2, \cdots, 13$ 给它们编号, 使得相邻两点标上的数之差的绝对值至多是 5 , 至少是 3 ?
%%<SOLUTION>%%
作一个图 $G$ : 将 $1,2, \cdots, 13$ 看成 13 个顶点, 其中对满足 $3 \leqslant|i-j| \leqslant$ 5 的正整数 $i, j$, 令顶点 $i, j$ 相邻.
如果可以按照题目要求进行编号, 则图 $G$ 必有一个哈密顿圈 $C$ (经过图中每个顶点恰好一次的圈).
将 $G$ 的顶点分成如下两个集合
$$
A=\{1,2,3,11,12,13\}, B=\{4,5,6,7,8,9,10\} .
$$
由于每个顶点必是哈密顿圈 $C$ 中两条边的端点, 但 $A$ 中任意两个顶点不相邻, 因此 $A, B$ 之间共有 $C$ 中的 12 条边, 因此 $C$ 中恰有一条边是连接 $B$ 中两个顶点的, 不妨记为 $e$.
对于 $B$ 的顶点 $4, A$ 中只有一个顶点 1 与之相邻, 所以 4 必是边 $e$ 的一个端点.
同理 10 也是边 $e$ 的端点.
但 4 与 10 不相邻.
矛盾.
所以满足题意的编号方式不存在.
%%PROBLEM_END%%



%%PROBLEM_BEGIN%%
%%<PROBLEM>%%
问题6. 如果凸 $n$ 边形的任意 3 条对角线都不交于一点, 试问该凸 $n$ 边形被它的对角线分成多少部分?
%%<SOLUTION>%%
以凸 $n$ 边形的顶点以及所有对角线的交点为顶点, 以顶点间已连有的线段为边构成一个平面图 $G$. 设 $G$ 的顶点数为 $V$, 边数为 $E$, 面数为 $F$ (其中, 凸多边形的外部也算作一个面,故所求的结果即为 $F-1$ ).
由于凸 $n$ 边形的任意 4 个顶点唯一对应形内的一个交点, 从而形内的交点总数为 $\mathrm{C}_n^4$, 故 $G$ 的顶点数 $V=n+\mathrm{C}_n^4$.
又由于 $G$ 中作为凸 $n$ 边形顶点的顶点具有度数 $n-1$, 而作为凸 $n$ 边形对角线交点的顶点具有度数 4 , 所以边数 $E=\frac{1}{2}\left(n(n-1)+4 \mathrm{C}_n^4\right)$.
将以上两式代入平面图的欧拉公式 $V-E+F=2$ 得
$$
\begin{aligned}
F-1 & =E-V+1=\frac{1}{2}\left(n(n-1)+4 \mathrm{C}_n^4\right)-\left(n+\mathrm{C}_n^4\right)+1 \\
& =\frac{1}{24}(n-1)(n-2)\left(n^2-3 n+12\right),
\end{aligned}
$$
即该凸 $n$ 边形被它的对角线分成 $\frac{1}{24}(n-1)(n-2)\left(n^2-3 n+12\right)$ 个部分.
%%PROBLEM_END%%



%%PROBLEM_BEGIN%%
%%<PROBLEM>%%
问题7. 某地区网球俱乐部的 20 名成员举行 14 场单打比赛,每人至少上场一次.
证明: 必有 6 场比赛,其中 12 个参赛者各不相同.
%%<SOLUTION>%%
用 20 个顶点 $v_1, v_2, \cdots, v_{20}$ 代表 20 名成员, 两名选手比赛过, 则在相应顶点间连一条边, 得图 $G$.
设各顶点度数为 $d_i, i=1,2, \cdots, 20$, 由题意得 $d_i \geqslant 1$, 又图 $G$ 中有 14 条边, 因此有 $d_1+d_2+\cdots+d_{20}=2 \times 14=28$.
在每个顶点 $v_i$ 处抹去 $d_i-1$ 条边, 至多抹去
$$
\left(d_1-1\right)+\left(d_2-1\right)+\cdots+\left(d_{20}-1\right)=8
$$
条边.
此时所得的图 $G^{\prime}$ 中至少还有 $14-8=6$ 条边, 并且 $G^{\prime}$ 中每个顶点的度数至多是 1 . 因此这 6 条边的 12 个端点各不相同, 从而相应的 6 场比赛的 12 名参赛者各不相同.
%%PROBLEM_END%%



%%PROBLEM_BEGIN%%
%%<PROBLEM>%%
问题8. $n$ 项正整数列 $x_1, x_2, \cdots, x_n$ 的各项之和为 2009 , 如果这 $n$ 个数既可分为和相等的 41 个组, 又可分为和相等的 49 个组, 求 $n$ 的最小值.
%%<SOLUTION>%%
设数列的 $n$ 项可分成每组之和为 49 的 41 个组 $u_1, u_2, \cdots, u_{41}$, 又可分成每组之和为 41 的 49 个组 $v_1, v_2, \cdots, v_{49}$. 把 $u_i(1 \leqslant i \leqslant 41)$ 和 $v_j(1 \leqslant j \leqslant 49)$ 看成顶点, 如果存在 $x_k(1 \leqslant k \leqslant n)$ 同时属于 $u_i$ 和 $v_j$, 就在 $u_i, v_j$ 间连一条边, 得简单图 $G$. 下面证明 $G$ 是连通图.
考虑 $G$ 的最大连通分支 $G^{\prime}$. 设 $G^{\prime}$ 含有 $a$ 个顶点 $u_{i_s}, s=1,2, \cdots, a$ 和 $b$ 个顶点 $v_{j_t}, t=1,2, \cdots, b$, 则所有出现在某个 $u_{i_s}(1 \leqslant s \leqslant a)$ 中的项 $x_k$ 必同时出现在某个 $v_{j_t}(1 \leqslant t \leqslant b)$ 中 (否则 $G$ 中将有比 $G^{\prime}$ 更大的连通分支), 反之亦然.
这些数的总和一方面等于 $49 a$,一方面又等于 $41 b$, 故 $49 a=41 b$, 所以 $a+b \geqslant 41+49=90$, 只能 $G^{\prime}=G$. 这样 $G$ 至少有 89 条边, 又数列的项数不少于 $G$ 的边数, 故 $n \geqslant 89$.
另一方面, 当 $n=89$ 时, 若数列满足
$$
x_1=x_2=\cdots=x_{41}=41, x_{42}=x_{43}=\cdots=x_{81}=8, x_{82}=x_{83}=\cdots=x_{89}=1 \text {, }
$$
则容易验证满足题意.
因此 $n$ 的最小值为 89 .
%%PROBLEM_END%%



%%PROBLEM_BEGIN%%
%%<PROBLEM>%%
问题9. 今有 10 个互不相同的非零数, 它们之中任意两个数的和与积中, 至少有一个是有理数.
证明: 每个数的平方都是有理数.
%%<SOLUTION>%%
首先注意到, 只要存在某个 $x \in \mathbf{Q}$, 则对其余任意数 $t$, 由 $x+t$ 与 $x t$ 之一为有理数即可推知 $t \in \mathbf{Q}$ (注意这些数均非零), 从而 10 个数均为有理数.
作一个 2 色完全图 $K_{10}$, 在它的 10 个顶点上分别放上这 10 个数.
如果某两数的和为有理数, 就在相应的顶点间连一条蓝边, 否则, 这两数的积必为有理数,那么连一条红边.
由 Ramsey 定理得, 该图中必有同色三角形.
(1)若存在蓝色三角形,则表明存在 3 个数 $x, y, z$ 两两之和为有理数, 故 $x, y, z \in \mathbf{Q}$, 从而 10 个数均为有理数.
(2)若存在红色三角形,则表明存在 3 个数 $x, y, z$ 两两之积为有理数, 因而 $x^2=\frac{x y \cdot x z}{y z} \in \mathbf{Q}$. 设 $x=m \sqrt{a}$, 其中 $a \in \mathbf{Q}, m= \pm 1$. 由于 $x y= m \sqrt{a} y=b \in \mathbf{Q}$, 所以 $y=\frac{b \sqrt{a}}{m a}=c \sqrt{a}$, 其中 $c \in \mathbf{Q}, c \neq m$. 对其余任意数 $t$, 如果 $x t$ 或 $y t$ 为有理数, 那么经过类似的讨论, 可知 $t=d \sqrt{a}$, 其中 $d \in \mathbf{Q}$, 因而 $t^2 \in \mathbf{Q}$. 反之, 如果 $x t, y t \notin \mathbf{Q}$, 则 $x+t, y+t \in \mathbf{Q}$, 但 $(x+t)-(y+t)= (m-c) \sqrt{a} \notin \mathbf{Q}$,矛盾.
综上, 我们证明了或者每个数都是有理数, 或者每个数的平方都是有理数, 这正是所要证明的.
%%PROBLEM_END%%



%%PROBLEM_BEGIN%%
%%<PROBLEM>%%
问题10. 已知 $X$ 为 $n$ 元集合.
设 $X$ 的每个有序元素对 $(x, y)(x, y \in X)$ 对应一个数 $f(x, y)=0$ 或 1 , 且对一切 $x, y \in X, x \neq y$, 恒有 $f(x, y) \neq f(y$, $x$ ). 求证:下述两种情况恰有一种出现:
(1) $X$ 是两个不交的非空集 $A, B$ 之并, 且对一切 $x \in A, y \in B$, 有 $f(x$, $y)=1$;
(2) $X$ 的元素可排成序列 $x_1, x_2, \cdots, x_n$, 使得对 $i=1,2, \cdots, n$, 有 $f\left(x_i, x_{i+1}\right)=1$, 其中 $x_{n+1}=x_1$.
%%<SOLUTION>%%
将 $X$ 的 $n$ 个元素视作顶点, 对任意 $x, y \in X, x \neq y$, 若 $f(x, y)=$ 1 , 则作一条有向边 $x \rightarrow y$, 得有向图 $G$. 由已知条件得, 图 $G$ 的任意两个顶点 $x, y$ 之间恰有一条有向边相连 (这样的图 $G$ 是 $n$ 阶有向完全图 $\overline{K_n}$, 称作"竞赛图").
情况 (1)表明不存在从顶点 $y \in B$ 到顶点 $x \in A$ 的有向边,情况(2)表明可以沿有向边遍历 $X$ 各点恰好一次, 即 $G$ 存在有向哈密顿圈.
显然两者不能同时成立.
假设上述情况 (1)不成立, 我们证明情况 (2)必成立.
任取一点 $x \in X$, 记 $P=\{a \in X \mid x \rightarrow a\}, Q=\{a \in X \mid a \rightarrow x\}$. 因情况 (1) 不成立, 故 $P, Q$ 均为非空集, 且存在 $y \in P, z \in Q$, 使 $y \rightarrow z$, 这表明图 $G$ 中存在圈 $x \rightarrow y \rightarrow z \rightarrow x$.
设 $x_1 \rightarrow x_2 \rightarrow \cdots \rightarrow x_k \rightarrow x_1$ 是图 $G$ 中的圈.
下面证明, 若 $k<n$, 则 $G$ 中存在更长的圈, 从而 $G$ 中必存在长为 $n$ 的圈, 即哈密顿圈.
设 $k<n$. 记 $X_0=X-\left\{x_1, x_2, \cdots, x_k\right\}$, 则 $X_0$ 为非空集.
若有某个 $u \in X_0$, 使得存在边 $x_i \rightarrow u$ 及 $u \rightarrow x_j$, 其中 $i, j \in\{1,2, \cdots$, $k\}$, 不失一般地可设 $i=1$, 则必存在这样的 $s \in\{1,2, \cdots, j-1\}$, 使 $x_s \rightarrow u \rightarrow x_{s+1}$, 这表明 $G$ 中有更长的圈
$$
x_1 \rightarrow x_2 \rightarrow \cdots \rightarrow x_s \rightarrow u \rightarrow x_{s+1} \rightarrow \cdots \rightarrow x_k \rightarrow x_1 .
$$
若不出现这样的 $u \in X_0$, 记 $P=\left\{a \in X_0 \mid x_1 \rightarrow a\right\}, Q=\left\{a \in X_0 \mid a \rightarrow\right. \left.x_1\right\}$, 则对任意 $y \in P$ (如果 $P$ 非空), 必有 $x_i \rightarrow y, i=1,2, \cdots, k$; 对任意 $z \in Q$ (如果 $Q$ 非空), 必有 $z \rightarrow x_i, i=1,2, \cdots, k$. 因情况 (1) 不成立, 故 $P, Q$ 均为非空集, 且存在 $y \in P, z \in \mathbf{Q}$, 使 $y \rightarrow z$, 这时 $G$ 中仍有更长的圈
$$
x_1 \rightarrow y \rightarrow z \rightarrow x_2 \rightarrow \cdots \rightarrow x_k \rightarrow x_1 .
$$
从而 $G$ 中必存在哈密顿圈 $x_1 \rightarrow x_2 \rightarrow \cdots \rightarrow x_n \rightarrow x_1$, 即情况 (2) 成立.
证毕.
%%PROBLEM_END%%


