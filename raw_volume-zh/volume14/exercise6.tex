
%%PROBLEM_BEGIN%%
%%<PROBLEM>%%
问题1. 在一次乒乓球循环赛中, $n(n \geqslant 3)$ 名选手中没有全胜的.
证明:一定可以从中找出三名选手 $A, B, C$, 有 $A$ 胜 $B, B$ 胜 $C, C$ 胜 $A$.
%%<SOLUTION>%%
取 $n$ 个选手中所赢局数最多的一位选手作 $A$ (取极端!), 由于 $A$ 未全胜, 所以存在选手 $C$ 胜 $A$. 考虑 $A$ 击败的选手全体, 知其中必有选手 $B$ 胜 $C$ (否则, $A$ 的败将也都是 $C$ 的败将, 从而 $C$ 赢的局数超过了 $A$ ). 于是所述的 $A, B, C$ 即为所求.
%%PROBLEM_END%%



%%PROBLEM_BEGIN%%
%%<PROBLEM>%%
问题2. 平面上给定 100 个点, 已知其中任意两点的距离不超过 1 , 且任意三点所成的三角形是钝角三角形.
证明: 这 100 个点被一个半径为 $\frac{1}{2}$-的圆覆盖.
%%<SOLUTION>%%
在这 100 个点中, 每两点有一个距离, 这有限个数 $\left(\frac{100 \times 99}{2} 个 \right)$ 中, 一定有一个最大数, 不妨设 $A, B$ 两点间的距离最大, 以线段 $A B$ 为直径作一个圆, 由题设, $A B \leqslant 1$, 所以, 此圆的半径 $\leqslant \frac{1}{2}$.
在其余的 98 个点中任取一点 $C$, 因为 $A B \geqslant A C$, $A B \geqslant B C$, 又由题设知 $\triangle A B C$ 是钝角三角形, 如图(<FilePath:./figures/fig-c6a2.png>)所示, 所以 $\angle C$ 一定是钝角, 于是点 $C$ 位于以 $A B$ 为直径的圆内, 也就是说这 100 个点都被这个圆所覆盖.
%%PROBLEM_END%%



%%PROBLEM_BEGIN%%
%%<PROBLEM>%%
问题3. 空间中给出了 8 个点,其中任意 4 个点都不在同一平面上,在它们之间连以 17 条线段.
证明:这些线段至少形成了一个三角形.
%%<SOLUTION>%%
每一个点都连出了若干条线段 (至多为 7 条), 不妨设连出线段数目最多的点为 $A$, 它共连出了 $n$ 条线段.
如果所有 17 条线段都没有形成三角形, 那么与 $A$ 相连的 $n$ 个点之间彼此都没有线段相连, 而其余的 $(7-n)$ 个点中, 每一点所连出的线段条数不多于 $n$ 条,因此,线段的总数目不超过
$$
\begin{aligned}
n+(7-n) n & =-n^2+8 n \\
& =-(n-4)^2+16 \leqslant 16,
\end{aligned}
$$
这与已知的有 17 条线段矛盾.
从而命题成立.
%%<REMARK>%%
注:其实本题的结论可加强为 "三角形的数目不少于 4 个", 这个问题较难, 留给有兴趣的读者思考.
%%PROBLEM_END%%



%%PROBLEM_BEGIN%%
%%<PROBLEM>%%
问题4. 若干个人聚会, 其中某些人彼此认识, 已知如果某两人在聚会者中有相同数目的熟人,那么他俩便没有共同的熟人.
证明: 若聚会者中有人至少有 2012 个熟人,则必然也有人恰好有 2012 个熟人.
%%<SOLUTION>%%
我们考虑 (聚会者中) 熟人最多的某个人(如果这样的人不止一个, 那么任取其中一个), 记为 $A$, 设 $A$ 共认识 $n$ 个人, 这些熟人依次记为 $B_1$, $B_2, \cdots, B_n$.
由于 $B_1, B_2, \cdots, B_n$ 中任意两个人 $B_i$ 与 $B_j$ 都认识 $A$, 即是他俩的共同熟人, 因此由题设推出了 $B_i$ 与 $B_j$ 的熟人数目不等.
此外, $B_1, B_2, \cdots, B_n$ 的熟人数目均不会超过 $n$ (这里用到了 $n$ 的 "最大性"!), 于是他们的熟人数目恰好是
$$
1,2, \cdots, n \text {. }
$$
现在已知有人至少认识 2012 人, 这意味着 $n \geqslant 2012$, 所以数 2012 在上述数列中出现, 于是 $B_1, B_2, \cdots, B_n$ 中恰好有人有 2012 个熟人.
%%PROBLEM_END%%



%%PROBLEM_BEGIN%%
%%<PROBLEM>%%
问题5. 由凸多边形内任意一点向它的各边引垂线, 证明: 至少有一个垂足在多边形的某条边内 (不含端点).
%%<SOLUTION>%%
由于凸多边形内任意一点 $P$ 向各边所引的垂线只有有限条, 故必存在一条垂线段是其中最短的, 设为 $P Q, P Q$ 与边 $A B$ 垂直于 $Q$. 我们证明 $Q$ 必在线段 $A B$ 内.
假设 $Q$ 不在线段 $A B$ 内, 由多边形的凸性知 $Q$ 在线段 $A B$ 端点或在多边形外,无论哪种情况,线段 $P Q$ 必与多边形的某条不是 $A B$ 的边有公共点.
不妨设 $P Q$ 与边 $M N$ 有公共点 $R$, 此时 $P R$ 与 $M N$ 不垂直, 故 $P$ 到直线 $M N$ 的距离 $d<P R \leqslant P Q$, 这与先前最短垂线段的选取相矛盾! 故假设不成立, 因此垂足 $Q$ 必在凸多边形的边 $A B$ 内.
%%PROBLEM_END%%



%%PROBLEM_BEGIN%%
%%<PROBLEM>%%
问题6. 在一个由若干个城市组成的国家中, 其中某些城市之间有道路相连, 满足: 所有道路互不相交; 对任意两个城市都可以从一个城市出发沿道路走到另一个城市 (中间可能通过其他城市). 在每个城市中都设置了一个里程表, 写有从这个城市出发开车途经所有城市所走过路程的最小值(同一城市可能经过几次). 证明: 任意两个城市里程表上的数字的比不超过 $\frac{3}{2}$. 
%%<SOLUTION>%%
考虑一条途经所有城市的最短的线路 $l$. 设 $l$ 的长度为 $N$, 起点和终点分别在 $A$ 和 $B$, 则在城市 $A$ 和 $B$ 的里程表上的数字均为 $N$. 对任何一座城市 $C, C$ 必在 $l$ 上,故 $C$ 沿 $l$ 到 $A$ 和 $B$ 之一的长度不大于 $\frac{N}{2}$, 不妨设到 $A$ 的长度不大于 $\frac{N}{2}$, 则从 $C$ 出发沿 $l$ 到 $A$ 然后沿 $l$ 到 $B$ 总长度不大于 $\frac{3 N}{2}$ 且通过所有城市, 这表明任何一座城市 $C$ 的里程表上的数字不大于 $\frac{3 N}{2}$, 也不小于 $N$ (因为 $l$ 是最短线路), 得证.
%%PROBLEM_END%%



%%PROBLEM_BEGIN%%
%%<PROBLEM>%%
问题7. 平面直角坐标系内, 证明不存在整点正 $n(n \geqslant 7)$ 边形.
%%<SOLUTION>%%
假设存在, 由于整点多边形的面积一定是正整数或正整数的一半, 故设其中面积最小的整点正 $n$ 边形之一为 $A_1 A_2 \cdots A_n$, 其边长为 $a$, 外接圆半径为 $R$.
因为当 $n \geqslant 7$ 时, $\frac{2 \pi}{n}<\frac{\pi}{3}$, 故 $a<R$.
在坐标平面内作 $\overrightarrow{A_i A_{i+1}}$ 的位置向量 $\overrightarrow{O B_i}(i=1,2, \cdots, n)$, 则 $B_1$, $B_2, \cdots, B_n$ 也是整点, 且 $\overrightarrow{O B_i}, \overrightarrow{O B_{i+1}}$ 的夹角相等(约定 $A_{n+1}=A_l, B_{n+1}= \left.B_1\right)$, 故 $B_1 B_2 \cdots B_n$ 也为整点正 $n$ 边形, 它的外接圆半径为 $a$, 小于 $A_1 A_2 \cdots A_n$ 的外接圆半径 $R$, 故 $B_1 B_2 \cdots B_n$ 是面积比 $A_1 A_2 \cdots A_n$ 更小的整点正 $n$ 边形, 矛盾.
所以不存在整点正 $n(n \geqslant 7)$ 边形.
%%<REMARK>%%
注:本题运用无穷递降思想求解, 即在假设存在的前提下, 构造一个比原先更小的整点正 $n$ 边形,但整点多边形面积不能无限小, 从而导出矛盾.
最后用极端原理进行表达.
%%PROBLEM_END%%



%%PROBLEM_BEGIN%%
%%<PROBLEM>%%
问题8. 将 2007 个整数放在一个圆周上, 使得任意相邻的 5 个数中有三个的和等于另两个数的和的 2 倍.
证明: 这 2007 个数都是 0. 
%%<SOLUTION>%%
假设存在不全为 0 的 2007 个整数满足条件,不妨取出其中一组, 它们按逆时针顺序依次为 $x_1, x_2, \cdots, x_{2007}$, 且使 $\left|x_1\right|+\left|x_2\right|+\cdots+\left|x_{2007}\right|$ 达到最小可能正值 (正整数集合中必存在最小数).
根据条件可得: 圆周上任意相邻 5 个数之和是其中两个数之和的 3 倍, 因此对 $i=1,2, \cdots, 2007$, 都有
$$
x_i+x_{i+1}+x_{i+2}+x_{i+3}+x_{i+4} \equiv 0(\bmod 3),
$$
从而可知 $x_i \equiv x_{i+5}(\bmod 3)$ (约定当 $j \equiv k(\bmod 2007)$ 时, 有 $\left.x_j=x_k\right)$. 又 5 和 2007 互素, 故对一切 $i=1,2, \cdots, 2007$, 有 $x_i \equiv x_1(\bmod 3)$. 此时由于
$$
x_i+x_{i+1}+x_{i+2}+x_{i+3}+x_{i+4} \equiv 5 x_1 \equiv 0(\bmod 3),
$$
故 $x_1 \equiv 0(\bmod 3)$, 因而 $x_1, x_2, \cdots, x_{2007}$ 都是 3 的倍数.
设 $y_i \equiv \frac{x_i}{3}, i=1,2, \cdots, 2007$, 则 $y_1, y_2, \cdots, y_{2007}$ 也是一组满足条件的整数, 但 $\left|y_1\right|+\left|y_2\right|+\cdots+\left|y_{2007}\right|<\left|x_1\right|+\left|x_2\right|+\cdots+\left|x_{2007}\right|$, 这与 $x_1$, $x_2, \cdots, x_{2007}$ 的取法矛盾.
故假设不成立.
所以这 2007 个数都是 0 .
%%PROBLEM_END%%



%%PROBLEM_BEGIN%%
%%<PROBLEM>%%
问题9. 已知正整数 $n \geqslant 2$, 且对于整数 $k, 0 \leqslant k \leqslant \sqrt{\frac{n}{3}}, k^2+k+n$ 都是素数.
求证: 对于整数 $k, 0 \leqslant k \leqslant n-2, k^2+k+n$ 也是素数.
%%<SOLUTION>%%
设 $m$ 是使 $m^2+m+n$ 为合数的最小正整数, 由已知得, $m>\sqrt{\frac{n}{3}}$.
若 $m \leqslant n-2$, 则 $m^2+m+n \leqslant(n-2)(n-1)+n<n^2$.
令 $p$ 是 $m^2+m+n$ 的最小素因子, 则 $p \leqslant \sqrt{m^2+m+n}<n$.
(1) 若 $m \geqslant p$, 则 $p \mid(m-p)^2+(m-p)+n$, 又 $(m-p)^2+(m-p)+ n \geqslant n>p$, 这与 $m$ 是使 $m^2+m+n$ 为合数的最小正整数矛盾.
(2) 若 $m \leqslant p-1$, 则
$$
(p-1-m)^2+(p-1-m)+n=p^2-2 p(m+1)+m^2+m+n
$$
被 $p$ 整除, 且 $(p-1-m)^2+(p-1-m)+n$ 大于 $p$, 故为合数.
根据 $m$ 的最小性可知 $p-1-m \geqslant m$, 所以
$$
2 m+1 \leqslant p \leqslant \sqrt{m^2+m+n},
$$
平方整理得 $3 m^2+3 m+1 \leqslant n$, 但 $m>\sqrt{\frac{n}{3}}$, 故 $3 m^2+3 m+1>3 m^2>n$, 矛盾!
所以对所有整数 $k, 0 \leqslant k \leqslant n-2, k^2+k+n$ 是素数.
%%<REMARK>%%
注:在反证法假设下, 找到最小的 $m$. 和此时 $m^2+m+n$ 的最小素因子 $p$, 通过调整得到更小的,因而矛盾!
%%PROBLEM_END%%


