
%%TEXT_BEGIN%%
不变量与恒增(减).
我们经常会遇到这样一些量, 它们经过运动、操作、变换后保持不变, 这样的量称为不变量.
俗话说 "万变不离其宗". 变化的是现象, 不变的是本质.
不变量方法就是通过寻找某种不变的本质来解决问题.
在具体解题过程中, 不变量可以在和、差、积、商、平方和、倒数和等运算结构中寻找, 也可以考虑正负、奇偶性、同余等一些方面.
此外, 很多染色与赋值的问题实际上也是通过对某些事物分类或赋予一些数值, 从中发现某种不变规律(参考第 14 节和第 15 节).
先看 3 个简单的例子.
%%TEXT_END%%



%%PROBLEM_BEGIN%%
%%<PROBLEM>%%
例1. 设 $a_1, a_2, \cdots, a_9$ 是非零实数.
证明: 行列式
$$
\left|\begin{array}{lll}
a_1 & a_2 & a_3 \\
a_4 & a_5 & a_6 \\
a_7 & a_8 & a_9
\end{array}\right|=a_1 a_5 a_9+a_2 a_6 a_7+a_3 a_4 a_8-a_1 a_6 a_8-a_2 a_4 a_9-a_3 a_5 a_7
$$
这六项中, 至少有一项是负数,且至少有一项是正数.
%%<SOLUTION>%%
证明:六项 $a_1 a_5 a_9, a_2 a_6 a_7, a_3 a_4 a_8,-a_1 a_6 a_8,-a_2 a_4 a_9,-a_3 a_5 a_7$ 的乘积为
$$
-\left(a_1 a_2 \cdots a_9\right)^2<0 .
$$
所以其中必有奇数个负项.
从而结论成立.
%%<REMARK>%%
注:在例 1 中, 关键的一点就是: 无论 $a_1, a_2, \cdots, a_9$ 中有多少个正数,多少个负数,所给的六项的乘积总是负数.
这就是一个不变量.
%%PROBLEM_END%%



%%PROBLEM_BEGIN%%
%%<PROBLEM>%%
例2. 在黑板上写有 97 个数: $\frac{48}{k}(k=1,2, \cdots, 97)$, 每次可任意选择两个数 $a, b$, 将它们擦去, 用 $2 a b-a-b+1$ 代替, 这样经过 96 次操作后, 黑板上还剩下一个数, 问: 这个数是多少?
%%<SOLUTION>%%
解:为
$$
2 a b-a-b+1=2\left(a-\frac{1}{2}\right)\left(b-\frac{1}{2}\right)+\frac{1}{2},
$$
所以, 当 $a$ 或 $b$ 为 $\frac{1}{2}$ 时, 得到的数还是 $\frac{1}{2}$, 即黑板上这个 $\frac{1}{2}$ 是不会消失的 (在操作过程中的不变量). 由于 $\frac{48}{96}=\frac{1}{2}$, 所以,最后剩下的数是 $\frac{1}{2}$.
在例 2 中, $\frac{1}{2}$ 是不会消失的, 是在操作过程中的不变量.
%%PROBLEM_END%%



%%PROBLEM_BEGIN%%
%%<PROBLEM>%%
例3. 若干名立场不坚定的人对某一问题有两种相反的意见.
他们在辩论桌上围坐一圈, 按逆时针次序轮流发言表明立场.
如果一个人发现他的朋友中大多数与他意见相反, 他就改变立场, 转而发言支持相反的意见, 否则他就表明原来的立场.
求证: 经过若干轮发言之后, 这些人均不再改变立场.
(注:朋友关系是相互的,且不改变.)
%%<SOLUTION>%%
证明:察所有意见相反的朋友对数目.
每当一名立场不坚定的人改变立场时, 他的意见便与大多数朋友相同, 因此意见相反的朋友对数目严格减小.
由于开始时意见相反的朋友对的数目是有限的, 所以不可能有某个人无限次改变立场, 从而若干轮发言之后, 这些人均不再改变立场.
在例 3 中,我们所考虑的是一种广义的"不变量"
"恒增(减) 量". 在不变量方法中,所谓"不变量"并不仅限于恒定的量, 更广泛地说, 也可以指变化着的量, 只要这种"变化" 具有某种"不变的规律". 本题中, "意见相反的朋友对数目"这个量具有单调性, 就是一个"不变量".
利用"恒增(减)量",乃至一般地,利用某种不变的规律解题是十分常见的, 同时也是技巧性较高的一个方面.
%%PROBLEM_END%%



%%PROBLEM_BEGIN%%
%%<PROBLEM>%%
例4. 设直线上一开始从左到右依次写有 $1,2, \cdots, 2011$ 这 2011 个数.
可以对相邻位置的三个数 $a, b, c$ 进行这样的操作: 把 $(a, b, c)$ 换成 $(b, c, a)$. 证明: 无论怎样操作, 不可能使直线上的数从左到右依次是 $2011,2010, \cdots, 2,1$.
%%<SOLUTION>%%
证明: $(1,2, \cdots, 2011)$ 的每一个排列 $P=\left(a_1, a_2, \cdots, a_{2011}\right)$, 称 $\left(a_i\right.$, $\left.a_j\right)$ 是 $P$ 的一个 "逆序对", 如果有 $i<j, a_i>a_j$. 将 $P$ 的逆序对个数称为逆序数, 记作 $I(P)$.
根据该定义,若在排列 $P$ 中选取相邻三个数 $a, b, c$ 进行操作, 得到另一个排列 $P^{\prime}$, 则必有 $I\left(P^{\prime}\right) \equiv I(P)(\bmod 2)$. 事实上, 对
$$
P=(\cdots, a, b, c, \cdots), P^{\prime}=(\cdots, b, c, a, \cdots),
$$
若 $(a, b),(a, c)$ 中有 $k$ 个是 $P$ 的逆序对, 则 $(b, a),(c, a)$ 中有 $2-k$ 个是 $P^{\prime}$ 的逆序对.
其余的数对是 $P$ 中的逆序对当且仅当是 $P^{\prime}$ 中的逆序对, 因此
$$
I\left(P^{\prime}\right)-I(P)=k-(2-k) \equiv 0(\bmod 2) .
$$
由于对 $P_0=(1,2, \cdots, 2011)$ 与 $P_1=(2011,2010, \cdots, 1)$, 分别有
$$
I\left(P_0\right)=0, I\left(P_1\right)=\frac{2011 \times 2010}{2} \equiv 1(\bmod 2),
$$
所以不能通过操作使 $P_0$ 变为 $P_1$. 证毕.
%%<REMARK>%%
注:"逆序对" 和 "逆序数" 是排列问题中常会涉及的概念, 考虑逆序数的变化规律 (如单调性、奇偶性、增量具有何种上界等) 往往成为解题的关键.
本题中我们用到的是逆序数的奇偶性这个不变量.
关于逆序数,一个基本且重要的结论是: 若 $n$ 个实数 $a_1, a_2, \cdots, a_n$ 两两不等,则任取其中两个数对换位置后, 排列的逆序数奇偶性改变.
特别地, 只需注意依次作 $(a, b) \rightarrow(b, a)$ 和 $(a, c) \rightarrow(c, a)$ 这两个对换恰好可将 $(a, b, c)$ 换成 $(b, c, a)$, 本题中逆序数的奇偶不变性就能轻易获得.
%%PROBLEM_END%%



%%PROBLEM_BEGIN%%
%%<PROBLEM>%%
例5. 数列 $\left\{a_n\right\}$ 和 $\left\{b_n\right\}$ 满足 $a_1=1, b_1=2$,
$$
a_{n+1}=\frac{1+a_n+a_n b_n}{b_n}, b_{n+1}=\frac{1+b_n+a_n b_n}{a_n},
$$
求证: $a_{2008}<5$. 
%%<SOLUTION>%%
证明:已知得
$$
1+a_{n+1}=\frac{\left(1+a_n\right)\left(1+b_n\right)}{b_n}, 1+b_{n+1}=\frac{\left(1+a_n\right)\left(1+b_n\right)}{a_n},
$$
且根据递推关系显然有 $a_n, b_n>0$, 故当 $n \in \mathbf{N}^*$ 时,
$$
\begin{aligned}
& \frac{1}{1+a_{n+1}}-\frac{1}{1+b_{n+1}} \\
= & \frac{b_n-a_n}{\left(1+a_n\right)\left(1+b_n\right)} \\
= & \frac{\left(1+b_n\right)-\left(1+a_n\right)}{\left(1+a_n\right)\left(1+b_n\right)} \\
= & \frac{1}{1+a_n}-\frac{1}{1+b_n},
\end{aligned}
$$
这表明 $\frac{1}{1+a_n}-\frac{1}{1+b_n}$ 是不变量.
以下有
$$
\frac{1}{1+a_{2008}}>\frac{1}{1+a_{2008}}-\frac{1}{1+b_{2008}}=\frac{1}{1+a_1}-\frac{1}{1+b_1}=\frac{1}{2}-\frac{1}{3}=\frac{1}{6},
$$
所以 $a_{2008}<5$.
%%<REMARK>%%
注:具体问题中, 数量关系及变化形式可能多种多样, 要从中寻找出微妙的不变性并加以利用, 须对问题进行透彻的分析, 同时也离不开一定的解题经验.
例如在本题中, 有一定解题经验的读者首先会想到在递推式两边加 1 , 而此后拟定计划时就有几个选择, 要做进一步的盘算.
如能注意到取倒数并作差后所出现的裂项的结构特征, 便能发现不变量 $\frac{1}{1+a_n}-\frac{1}{1+b_n}$. 一旦抓住这个不变量并加以利用, 问题就迎刃而解了.
%%PROBLEM_END%%



%%PROBLEM_BEGIN%%
%%<PROBLEM>%%
例6. 设二次三项式 $f(x)$ 可以用 $x^2 f\left(\frac{1}{x}+1\right)$ 或者 $(x-1)^2 f\left(\frac{1}{x-1}\right)$ 来代换.
问: 能否利用上述的代换, 把设二次三项式 $x^2+4 x+3$ 变为 $x^2+ 10 x+9$ ?
%%<SOLUTION>%%
解:案是否定的.
设 $f(x)=a x^2+b x+c$, 其判别式 $\Delta=b^2-4 a c$.
对于第一种代换: $x^2 f\left(\frac{1}{x}+1\right)=(a+b+c) x^2+(b+2 a) x+a$, 其判别式
$$
\Delta_1=(b+2 a)^2-4(a+b+c) a=b^2-4 a c ;
$$
对于第二种代换: $(x-1)^2 f\left(\frac{1}{x-1}\right)=c x^2+(b-2 c) x+(a-b+c)$, 其判别式
$$
\Delta_2=(b-2 c)^2-4 c(a-b+c)=b^2-4 a c,
$$
从而, 这两个代换不改变二次三项式的判别式 (即判别式是不变量!).
因为 $x^2+4 x+3$ 的判别式为 $4^2-4 \times 3=4$, 而 $x^2+10 x+9$ 的判别式为 $10^2-4 \times 9=64$, 故通过上述的代换, 不能把二次三项式 $x^2+4 x+3$ 变为 $x^2+10 x+9$.
%%PROBLEM_END%%



%%PROBLEM_BEGIN%%
%%<PROBLEM>%%
例7. 已知数列
$$
1,0,1,0,1,0,3, \cdots
$$
中,每一项等于它前面 6 项之和的末位数字.
证明: 在这个数列中不存在连续的 6 项依次是 $0,1,0,1,0,1$.
%%<SOLUTION>%%
证明:这个数列中每连续的 6 项 $x, y, z, u, v, w$ 对应于数
$$
2 x+4 y+6 z+8 u+10 v+12 w
$$
的个位数字.
例如对于开始的六项 $1,0,1,0,1,0$, 对应的数是
$$
2 \times 1+4 \times 0+6 \times 1+8 \times 0+10 \times 1+12 \times 0=18
$$
的个位数字 8 .
如果 $x, y, z, u, v, w, r$ 是顺次的 7 项, 那么 $y, z, u, v, w, r$ 所对应的数 $B$ 与 $x, y, z, u, v, w$ 所对应的数 $A$ 应满足:
$$
\begin{aligned}
B-A & =(2 y+4 z+6 u+8 v+10 w+12 r)-(2 x+4 y+6 z+8 u+10 v+12 w) \\
& =12 r-2(x+y+z+u+v+w) \equiv 12 r-2 r \equiv 0(\bmod 10) .
\end{aligned}
$$
所以, 在每一项换成它后面一项时, 连续 6 项所对应的数保持不变, 即对于这个数列, 每连续 6 项所对应的数永远是 8 .
假如 $0,1,0,1,0,1$ 能作为这个数列中的连续 6 项, 这时对应的数为
$$
2 \times 0+1 \times 1+6 \times 0+8 \times 1+10 \times 0+12 \times 1=24
$$
的个位数字 4 , 不等于 8 . 所以数列中不存在连续 6 项依次是 $0,1,0,1,0,1$.
%%<REMARK>%%
注:本题递推阶数较高, 不宜从数列的周期性着手, 又若从模周期性 (例如奇偶性)考虑, 并不能推出矛盾.
因此不妨设出一个具有 $a x+b y+c z+d u+ e v+f w$ 形式的不变量, 其中 $a$ 至 $f$ 为待定的整数.
注意到若 $r \equiv x+y+z+u+v+w(\bmod 10)$, 则有
$$
\begin{gathered}
(a y+b z+c u+d v+e w+f r)-(a x+b y+c z+d u+e v+f w) \\
\equiv(f-a) x+(a+f-b) y+(b+f-c) z \\
+(c+f-d) u+(d+f-e) v+e w(\bmod 10),
\end{gathered}
$$
因此,适当地取整数 $a$ 至 $f$, 使 $x, y, z, u, v, w$ 前的系数均被 10 整除, 即可构造出这样的不变量.
本题便是通过这样的不变量证明了结论.
%%PROBLEM_END%%



%%PROBLEM_BEGIN%%
%%<PROBLEM>%%
例8. 如图(<FilePath:./figures/fig-c19i1.png>) 所示, 圆形的水池被分割为 $2 n(n \geqslant 5)$个 "格子". 我们把有公共隔墙 (公共边或公共弧) 的"格子"称为相邻的, 从而每个"格子"都有三个邻格.
水池中一共跳人了 $4 n+1$ 只青蛙,青蛙难于安静共处,只要某个"格子"中有不少于 3 只青蛙,那么迟早一定会有其中 3 只分别同时跳往三个不同邻格.
证明: 只要经过一段时间之后, 青蛙便会在水池中大致分布均匀.
所谓大致分布均匀, 就是任取其中一个"格子", 或者它里面有青蛙, 或者它的三个邻格里都有青蛙.
%%<SOLUTION>%%
证明:们把一个格子中出现一次 3 只青蛙同时分别跳向三个邻格的事件称为该格子发生一次 "爆发". 而把一个格子或者是它里面有青蛙, 或者是它的三个相邻的格子里面都有青蛙, 称为该格子处于"平衡状态".
容易看出,一个格子一旦有青蛙跳入, 那么它就一直处于 "平衡状态". 事实上,只要不 "爆发", 那么该格子中的青蛙不会动, 它当然处于 "平衡状态"; 而如果发生 "爆发", 那么它的三个邻格中就都有青蛙, 并且只要三个邻格都不"爆发", 那么它就一直处于"平衡状态"; 而不论哪个邻格发生"爆发", 都会有青蛙跳到它里面, 它里面就一定有青蛙, 所以它一直处于"平衡状态".
这样一来, 为证明题中断言, 我们就只要证明: 任何一个格子都迟早会有青蛙跳入.
任取一个格子,把它称为 $A$ 格, 把它所在的扇形称为 1 号扇形, 把该扇形中的另一个格子称为 $B$ 格, 我们要证明 $A$ 格迟早会有青蛙跳入.
按顺时针方向依次将其余扇形接着编为 2 至 $n$ 号.
首先证明 1 号扇形迟早会有青蛙跳人.
假设 1 号扇形中永无青蛙到来,那么就不会有青蛙越过 1 号扇形与 $n$ 号扇形之间的隔墙.
我们来考察青蛙所在的扇形编号的平方和.
由于没有青蛙进人 1 号扇形 (尤其没有青蛙越过 1 号扇形与 $n$ 号扇形之间的隔墙), 所以只能是有 3 只青蛙由某个 $k(3 \leqslant k \leqslant n-1)$ 号扇形分别跳入 $k-1$, $k$ 和 $k+1$ 号扇形各一只,因此平方和的变化量为
$$
(k-1)^2+k^2+(k+1)^2-3 k^2=2,
$$
即增加 2. 一方面, 由于青蛙的跳动不会停止 (因为总有一个格子里有不少于 3 只青蛙), 所以平方和的增加趋势不会停止; 但是另一方面, 青蛙所在扇形编号的平方和不可能永无止境地增加下去 (不会大于 $(4 n+1) n^2$ ), 由此产生矛盾, 所以迟早会有青蛙越过 1 号扇形与 $n$ 号扇形之间的隔墙,进人 1 号扇形.
我们再来证明 1 号扇形迟早会有 3 只青蛙跳人.
如果 1 号扇形中至多有两只青蛙跳人, 那么它们都不会跳走, 并且自始至终上述平方和至多有两次变小 (只能在两只青蛙越过 1 号扇形与 $n$ 号扇形之间的隔墙时变小), 以后便一直持续不断地上升, 从而又重蹈刚才的矛盾.
所以 1 号扇形迟早会有 3 只青蛙跳人.
如果这 3 只青蛙中有位于 $A$ 格的, 那么 $A$ 格中已经有青蛙跳人; 如果这 3 只青蛙全都位于 $B$ 格, 那么 $B$ 格迟早会发生 "爆发", 从而有青蛙跳人 $A$ 格.
%%<REMARK>%%
注:除了不变量方法, 本例中同时运用了化归、抽屉原理(用于保证青蛙的跳动不会停止)等解题思想方法.
%%PROBLEM_END%%


