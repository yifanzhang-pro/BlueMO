
%%PROBLEM_BEGIN%%
%%<PROBLEM>%%
问题1. 设图 $G=(V, E), V=\left\{v_1, v_2, \cdots, v_5\right\}, E=\left\{\left(v_1, v_2\right),\left(v_2, v_4\right),\left(v_3\right.\right.$, $\left.\left.v_4\right),\left(v_4, v_5\right),\left(v_1, v_3\right)\right\}$. 画出 $G$ 的图形.
%%<SOLUTION>%%
$G$ 的图形如图(<FilePath:./figures/fig-c1a1.png>)所示.
%%PROBLEM_END%%



%%PROBLEM_BEGIN%%
%%<PROBLEM>%%
问题2. 设图 $G=(V,\,E)$ 是简单图, $\vert V\vert=n,\vert E\vert=e,$ ,证明 $e\leq{\frac{n(n-1)}{2}}.$ 
%%<SOLUTION>%%
简单图每条边与两个不同顶点关联,且每两个顶点间至多有一条边连接, 现有 $n$ 个顶点, 故边数至多为 $\mathrm{C}_n^2=\frac{n(n-1)}{2}$, 由于简单图不一定是完全图, 所以 $e \leqslant \frac{n(n-1)}{2}$.
%%PROBLEM_END%%



%%PROBLEM_BEGIN%%
%%<PROBLEM>%%
问题3. 说明下面两个图(<FilePath:./figures/fig-c1p3-1.png>)和(<FilePath:./figures/fig-c1p3-2.png>) 是同构的.
%%<SOLUTION>%%
如图(<FilePath:./figures/fig-c1a3-1.png>)和如图(<FilePath:./figures/fig-c1a3-2.png>)所示建立点的对应关系为: $v_1 \longleftrightarrow u_1, v_2 \longleftrightarrow u_2, v_3 \longleftrightarrow u_3, v_4 \longleftrightarrow u_4, v_5 \leftrightarrow u_5$, 边的对应关系为 $e_1 \longleftrightarrow e_1^{\prime}, e_2 \longleftrightarrow e_2^{\prime}, e_3 \leftrightarrow e_3^{\prime}, \cdots, e_8 \leftrightarrow e_8^{\prime}$. 这两个图的顶点数相等, 边数相等, 且顶点和边相互对应, 因此这两图是同构的.
%%PROBLEM_END%%



%%PROBLEM_BEGIN%%
%%<PROBLEM>%%
问题4. 有 $n$ 个药箱,每两个药箱里有一种相同的药,每种药恰好在两个药箱里出现, 问有多少种药?
%%<SOLUTION>%%
按下法作一个图: 每一个顶点表示一个药箱, 每一条边 $\left(v_i, v_j\right)$ 表示两个药箱 $v_i$ 与 $v_j$ 所共有的那一种药.
据题设, 此图是 $n$ 个顶点的完全图 $K_n$. 药的种数即 $K_n$ 的边数 $\frac{1}{2} n(n-1)$.
%%PROBLEM_END%%



%%PROBLEM_BEGIN%%
%%<PROBLEM>%%
问题5. 一次会议有 $n$ 名教授 $A_1, A_2, \cdots, A_n$ 参加, 证明可以将这 $n$ 个人分为两组, 使得每一个人 $A_i$ 在另一组中认识的人数 $d_i$ 不少于他在同一组中认识的人数 $d_i^{\prime}(i=1,2, \cdots, n)$.
%%<SOLUTION>%%
用 $n$ 个点 $v_1, v_2, \cdots, v_n$ 表示这 $n$ 名教授, 并在相互认识的人之间连一条边, 将这 $n$ 个点任意分成两组, 只有有限多种分法.
考虑在两组之间的连线条数 $S$, 其中必存在一种分法, 使 $S$ 达到最大值.
此时定有 $d_i \geqslant d_i^{\prime}$ ( $i= 1,2, \cdots, n)$. 若不然, 设对 $v_1$, 有 $d_1<d_1^{\prime}$, 则将 $v_1$ 从这组换到另一组, $S$ 增加了$d_1^{\prime}-d_1>0$. 这与 $S$ 已达到最大值矛盾.
%%PROBLEM_END%%



%%PROBLEM_BEGIN%%
%%<PROBLEM>%%
问题6. 18 个队进行比赛, 每一轮中每个队与另一个队比赛一场, 并且在其他轮比赛中这两个已赛过的队彼此不再比赛, 现在比赛已进行完 8 轮.
证明一定有三个队在前 8 轮比赛中, 彼此之间尚未比赛过.
%%<SOLUTION>%%
设 $A$ 队经过 8 轮之后与 8 个队赛过, 而与其他 9 个队没有赛过, 若这 9 个队在前 8 轮中相互之间都赛过, 由于每队只赛了 8 场, 所以这 9 个队与其他各队都没有赛过.
但这 9 个队中第一轮比赛只能赛 4 场, 所以必有一个队要与其他队比赛, 矛盾.
所以在这 9 个队中必有两个队 $B, C$, 它们之间没有赛过, 这样 $A, B, C$ 三队之间便彼此没有赛过了.
%%PROBLEM_END%%



%%PROBLEM_BEGIN%%
%%<PROBLEM>%%
问题7. 某次会议有 $n$ 名代表出席, 已知任意的四名代表中都有一个人与其余的三个人握过手, 证明任意的四名代表中必有一个人与其余的 $n-1$ 名代表都握过手.
%%<SOLUTION>%%
$n$ 名代表用 $n$ 个点表示, 如果两名代表没有握过手, 就在相应的顶点之间连一条边, 得图 $G$. 如果 $G$ 中任意 4 点 $v_1, v_2, v_3, v_4$ 中, 每一点都有与之相邻的点, 分别设为 $v_1^{\prime}, v_2^{\prime}, v_3^{\prime}, v_4^{\prime}$. 由已知条件知, $v_1, v_2, v_3, v_4$ 中有一点, 不妨设为 $v_1$, 与其余三点 $v_2, v_3, v_4$ 均不相邻.
所以 $v_1^{\prime} \neq v_2, v_3, v_4$. 如果 $v_2^{\prime} \neq v_1^{\prime}$, 则在 $v_1, v_2, v_1^{\prime}, v_2^{\prime}$ 这 4 个点中, 没有一个点与其余的三点均不相邻, 所以 $v_2^{\prime}=v_1^{\prime}$. 同理 $v_3^{\prime}=v_1^{\prime}$. 于是在 $v_1, v_2, v_3, v_1^{\prime}$ 这 4 个点中, 又没有一个点与其余的三个点均不相邻.
所以在任意 4 个顶点中必有一个点与其余的 $n-1$ 个点均不相邻.
%%PROBLEM_END%%



%%PROBLEM_BEGIN%%
%%<PROBLEM>%%
问题8. 有三所中学, 每所有学生 $n$ 名.
每名学生都认识其他两所中学的 $n+1$ 名学生.
证明: 从每所中学可以选出一名学生, 使选出来的 3 名学生互相认识.
%%<SOLUTION>%%
如图(<FilePath:./figures/fig-c1a8.png>), 用 $3 n$ 个顶点表示这些学生,三所中学的学生组成的三个顶点集合分别记为 $X, Y$ 和 $Z$. 若 $u$ 和 $v$ 是不同学校的学生, 而且是互相认识的, 则在 $u$ 与 $v$ 之间连一条边, 这样便得图 $G$. 设 $x \in X, Y$ 和 $Z$ 中和 $x$ 相邻的顶点数记作 $k$ 和 $l$, 则 $k+l=n+1$. $k$ 与 $l$ 中大的记作 $m(x)$, 让 $x$ 跑遍 $X, m(x)$ 的最大值记作 $m_X, m_Y$ 与 $m_Z$ 作同样理解.
数 $m_X, m_Y, m_Z$ 中的最大值记作 $m$, 不妨设 $m=m_X$, 并且 $x_0 \in X$,
使得 $Y$ 中和 $x_0$ 相邻的顶点集合 $Y_1$ 中顶点数 $\left|Y_1\right|=m$. 于是 $Z$ 中与 $x_0$ 相邻的顶点数为 $n+1-m \geqslant 1$. 设 $z_0 \in Z$ 与 $x_0$ 相邻.
如果有 $y_0 \in Y_1$ 与 $z_0$ 相邻, 则 $\triangle x_0 y_0 z_0$ 是 $G$ 中的一个三角形.
若 $Y_1$ 中每一个 $y$ 与 $z_0$ 都不相邻, 则 $Y$ 中与 $z_0$ 相邻的顶点数 $\leqslant n-m$. 因此 $z_0$ 与 $X$ 中相邻的顶点数 $\geqslant n+1-(n-m)= m+1$,与 $m$ 的最大性矛盾.
于是证得 $G$ 中必有 $\triangle x_0 y_0 z_0$.
%%PROBLEM_END%%



%%PROBLEM_BEGIN%%
%%<PROBLEM>%%
问题9. 一个很大的棋盘上有 $2 n$ 个红色的方格, 对任何两个红色方格可从其中一个出发, 每步横或坚走到相邻的红色方格而到另一个方格中.
证明: 所有的红色方格可以分为 $n$ 个长方形.
%%<SOLUTION>%%
当 $n=1$ 时,有 2 个红格相邻, 显然为一个矩形.
设当 $n \leqslant k$ 时成立, 即可以将 $2 k$ 个连通方格分成 $k$ 个矩形.
当 $n=k+1$ 时,
(i) 对于 $2 k+2$ 个方格中, 若去掉一对相邻的红格后有一个图仍为连通图, 则由归纳假设结论成立.
(ii) 若去掉一对相邻的红格被分成若干个连通图, 而每个图的红格个数为偶数时, 由归纳假设结论成立.
(iii) 若去掉任何一对相邻的红格被分成若干个连通图, 而其中存在连通图的红格个数为奇数时:
当 $n=2$ 时,有"T"形图 $1 \times 3$ 和 $1 \times 1$ 的两个矩形满足要求.
当 $n \neq 2$ 时,观察所有方格中左上角的"T"形图,去掉这两个方格至多形成两个连通图.
若去掉左上角的两个后所形成的两个连通图的红格的个数均为奇数,则去掉 $1 \times 3$ 和 $1 \times 1$ 的两个矩形后, 易知仍为两个连通图, 而红格的个数为偶数.
综上所述, $n=k+1$ 也成立,故结论成立.
%%PROBLEM_END%%



%%PROBLEM_BEGIN%%
%%<PROBLEM>%%
问题10. 某参观团有 2000 个人, 其中任意 4 个人中一定有某一个人认识其他三人.
问: 认识该参观团所有成员的人数最少是多少?
%%<SOLUTION>%%
如果 2000 个成员都彼此认识, 则认识参观团所有成员的人数为 2000. 因此,不妨设有某两个成员 $u$ 与 $v$ 互不认识.
下面分三个步骤.
(i)除 $u, v$ 外的任意两个成员必彼此认识.
设 $a, b$ 是另外两个成员, 由题设,在 $a, b, u, v$ 这 4 个成员中, 必有一人认识其余三人, 这个人只能是 $a$ 或 $b$,这表明 $a, b$ 互相认识.
(ii)如果 $u, v$ 两人都认识其他的 1998 个人中的每一个,则该参观团有 1998 个人认识所有其他成员.
设 $a$ 是除 $u, v$ 外的任意一个成员, 由假设知 $a$ 认识 $u, v$. 设 $b$ 是另一成员, 由前所证 $a$ 与 $b$ 必彼此认识.
由 $b$ 的任意性知 $a$ 认识参观团的所有其他成员; 又由 $a$ 的任意性知该参观团除 $u, v$ 外的 1998 个人认识所有其他成员.
(iii)如果 $u, v$ 中某一个不全认识其他 1998 个成员, 则该参观团有 1997 个人认识所有其他成员, 不妨设除 $v$ 外, $u$ 不认识另一个成员 $w$. 设 $a$ 是 $u, v$, $w$ 外的 1997 个成员中的任意一个,由题设,在 $a, u, v, w$ 中认识另外三人的人只能是 $a$, 这表明 $u, v, w$ 这三个人的每一个都认识该参观团中的其余 1997 人.
综上,认识该参观团所有成员的人数最少是 1997 个.
%%PROBLEM_END%%



%%PROBLEM_BEGIN%%
%%<PROBLEM>%%
问题11. 在一个车相里, 任何 $m(m \geqslant 3)$ 个旅客都有唯一的公共朋友 (当甲是乙的朋友时, 乙也是甲的朋友.
任何人不作为他自己的朋友). 问: 在这个车厢里,有多少人?
%%<SOLUTION>%%
根据条件,每个人都有朋友.
如果 $k(k \leqslant m)$ 个人彼此是朋友, 由于他们有一个公共的朋友, 所以 $k+1$ 个人彼此是朋友.
依此类推, 导出有 $m+1$ 个人 $A_1, A_2, \cdots, A_{m+1}$ 彼此是朋友.
下面证车厢中除 $A_1, A_2, \cdots, A_{m+1}$ 外,别无他人.
若 $B$ 是这 $m+1$ 个人以外的人, 并且 $B$ 至少与 $A_1, A_2, \cdots, A_{m+1}$ 中两个人是朋友.
设 $B$ 与 $A_1 、 A_2$ 是朋友, 则 $B, A_3, A_4, \cdots, A_{m+1}$ 这 $m$ 个人有两个公共的朋友 $A_1 、 A_2$, 与已知矛盾.
因此 $A_1, A_2, \cdots, A_{m+1}$ 之外的人 $B$ 至多与 $A_1, A_2, \cdots, A_{m+1}$ 中一个人是朋友.
故不妨设除 $A_1$ 外, $A_2, A_3, \cdots, A_{m+1}$ 都不是 $B$ 的朋友.
于是 $m$ 个旅客 $B$, $A_1, A_2, \cdots, A_{m-1}$ 的公共朋友 $C$, 当然不是 $A_2, A_3, \cdots, A_{m+1}$, 也不是 $A_1$. 由于
$m \geqslant 3, C$ 与 $A_1, A_2, \cdots, A_{m-1}$ 中 $m-1 \geqslant 2$ 个人是朋友.
这与上面已证 $C$ 至多与 $A_1, \cdots, A_{m+1}$ 中一个人是朋友矛盾.
于是车相中只有 $A_1, A_2, \cdots, A_{m+1}$ 这 $m+1$ 个人, 每个人的朋友恰好是 $m$ 个.
%%PROBLEM_END%%



%%PROBLEM_BEGIN%%
%%<PROBLEM>%%
问题12. 平面上给定五点 $A 、 B 、 C 、 D 、 E$, 其中任三点不共线, 试证: 任意用线段连接某些点 (这些线段称为边), 若所得图形中不出现以这五点中任三点为顶点的三角形, 则此图不可能有 7 条或更多的边.
%%<SOLUTION>%%
$K_5$ 有 $\mathrm{C}_5^2=10$ 条边, $\mathrm{C}_5^3=10$ 个三角形,与每条边有关的三角形恰有 3 个.
若图中有不少于 7 条边, 则即从 $K_5$ 中至多去掉 3 边, 于是至多去掉 $3 \times 3=9$ 个三角形,所以仍有 1 个三角形,与题设矛盾.
所以此图不可能有 7 条或更多的边.
%%PROBLEM_END%%


