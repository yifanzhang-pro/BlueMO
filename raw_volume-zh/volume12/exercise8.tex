
%%PROBLEM_BEGIN%%
%%<PROBLEM>%%
问题1. 设 $A_1, A_2, A_3, A_4, A_5, A_6$ 是平面上的 6 点, 其中任意三点不共线.
如果这些点之间任意连接 13 条线段,证明: 必存在 4 点, 它们每两点之间都有线段连接.
%%<SOLUTION>%%
如图(<FilePath:./figures/fig-c8a1.png>), 将已连结的 13 条线段全染成红色, 还未连上的两条用蓝线连上 (因为所有两点连一线段时应该共有 15 条). 于是必有一个同色三角形,现在的蓝色线只有两条, 所以同色三角形必为红色的.
不妨设 $\triangle A_1 A_2 A_3$ 是红色的.
从 $A_4 、 A_5 、 A_6$ 引向 $\triangle A_1 A_2 A_3$ 顶点各有 3 条, 这 9 条线段中最多只有 2 条蓝色, 起码有 7 条是红色的, 因此, 或者是 $A_4$, 或者是 $A_5$, 或者是 $A_6$, 引向
$\triangle A_1 A_2 A_3$ 顶点的线段全是红色.
比如说, $A_4 A_1 、 A_4 A_2 、 A_4 A_3$ 全是红色, 那么 4 点 $A_1 、 A_2 、 A_3 、 A_4$ 的每 2 点连线全是红色的, 命题得证.
%%PROBLEM_END%%



%%PROBLEM_BEGIN%%
%%<PROBLEM>%%
问题2. 空间八个点两两连线, 并二染色其边, 证明 : 必存在三条无公共端点的同色线段.
%%<SOLUTION>%%
如图(<FilePath:./figures/fig-c8a2.png>), 若不存在三条无公共端点的同色线段, 如图, 不妨设 $A_1 A_2$ 为红色.
由假设 $A_3 A_4, A_5 A_8, A_6 A_7$ 三线段不可能都是蓝色, 不妨设 $A_3 A_4$ 为红色.
由于 $A_1 A_2$, $A_3 A_4$ 为红色,得 $A_5, A_6, A_7, A_8$ 四点两两连线皆蓝色,同样可得 $A_1, A_2, A_3, A_4$ 四点两两连线皆红色.
不妨设 $A_1 A_6$ 为蓝色, 则 $A_3 A_8$ 必为红色, 此时不论对 $A_4 A_5$ 染什么颜色,均与假设矛盾.
%%PROBLEM_END%%



%%PROBLEM_BEGIN%%
%%<PROBLEM>%%
问题3. 平面上有六点, 任何三点都是不等边三角形的顶点.
证明 : 这些三角形中有一个的最短边同时是另一个三角形的最长边.
%%<SOLUTION>%%
把每个三角形的最短边染成红色, 剩下的边染成蓝色.
由于 $r_2=6$, 必出现同色三角形, 而这个同色三角形必定是红色的,所以它的最长边也是另一个三角形的最短边.
%%PROBLEM_END%%



%%PROBLEM_BEGIN%%
%%<PROBLEM>%%
问题4. 空间六条直线, 其中每三条直线都不共面.
证明必存在三条直线, 满足下列条件之一: (i)两两异面; (ii) 互相平行; (iii)交于同一点.
%%<SOLUTION>%%
用六个点表示六条直线, 若两直线异面, 则对应两点连线染红色; 若两直线共面, 则对应两点连线染蓝色.
得两色完全图 $K_6$. 故其中必有同色三角形.
若是红色三角形, 则三顶点对应的三直线两两异面, 若为蓝色三角形, 则三顶点对应的三直线两两共面, 由于题设三直线不共于一面, 故这三直线所在的三个平面两两相交于这三条直线, 由此得这三直线互相平行或交于一点.
%%PROBLEM_END%%



%%PROBLEM_BEGIN%%
%%<PROBLEM>%%
问题5. 设 $n$ 个新生中, 任意 3 个人中有两个人互相认识, 任意四个人中有两个人互不认识.
试求 $n$ 的最大值.
%%<SOLUTION>%%
所求 $n$ 的最大值为 8 .
当 $n=8$ 时, 如图(<FilePath:./figures/fig-c8a5.png>)所示的例子满足要求, 其中 $A_1$, $A_2, \cdots, A_8$ 表示 8 个学生, $A_i$ 与 $A_j$ 连线表示 $A_i$ 与 $A_j$ 认识,否则不认识.
下设 $n$ 个学生满足题设要求,我们来证明 $n \leqslant 8$. 为此, 我们先来证明如下两种情况不可能出现:
(1) 若某人 $A$ 至少认识 6 个人, 设为 $B_1, \cdots, B_6$, 由
Ramsey 定理, 这 6 个人中存在 3 个互不相识 (这与已知任 3 个人中有 2 个相识矛盾); 或存在 3 个人互相认识, 这时 $A$ 与这 3 个人共 4 人两两互相认识, 亦与已知矛盾.
(2) 若某人 $A$ 至多认识 $n-5$ 个人,则剩下至少 4 个人均与 $A$ 不相识, 从而这 4 个人两两相识,矛盾.
其次, 当 $n \geqslant 10$ 时,(1) 与 (2) 必有一种情况出现, 故此时 $n$ 不满足要求; 当 $n=9$ 时, 要使 (1) 与 (2) 均不出现, 则此时每个人恰好认识其他 5 个人, 于是这时 9 个人产生的朋友对 (相互认识的对子) 的数目为 $\frac{9 \times 5}{2} \notin \mathbf{N}^*$, 矛盾!
%%PROBLEM_END%%



%%PROBLEM_BEGIN%%
%%<PROBLEM>%%
问题6. 把连接圆周上 9 个不同点的 36 条边染成红色或蓝色.
假定由 9 个点中每 3 个点所确定的三角形都含有红边, 证明有 4 个点, 其中每两点连的都是红边.
%%<SOLUTION>%%
取两色完全图 $K_9$ 的顶点 $A$, 若 $A$ 连有 4 条蓝边 $A A_1, A A_2, A A_3$, $A A_4$, 则 $K_9$ 中以 $A_1, A_2, A_3, A_4$ 为顶点的完全子图 $K_4$ 不含蓝边.
如果 $A$ 连有 6 条红边 $A A_1, A A_2, \cdots, A A_6$, 则 $K_9$ 中以 $A_1, \cdots, A_6$ 为顶点的完全子图含有同色三角形 $\triangle A_i A_j A_k(1 \leqslant i, j, k \leqslant 6)$. 由于 $K_9$ 不含蓝色三角形, 故 $\triangle A_i A_j A_k$ 是红色三角形.
则 $A, A_i, A_j, A_k$ 为顶点的完全子图 $K_4$ 是红色的.
如果 $K_9$ 中每一顶点都恰好连有 5 条红边, 则 $K_9$ 中红边数为 $\frac{5 \times 9}{2}$, 不可能.
%%PROBLEM_END%%



%%PROBLEM_BEGIN%%
%%<PROBLEM>%%
问题7. 证明: 在任何 19 个人中, 总有 3 个人互相认识或者 6 个人互相不认识.
%%<SOLUTION>%%
由定理三, $r(3,6) \leqslant r(3,5)+r(2,6)-1=14+6-1=19$.
%%PROBLEM_END%%



%%PROBLEM_BEGIN%%
%%<PROBLEM>%%
问题8. 证明: 在任何 18 个人中, 总有 4 个人相互认识, 或者相互不认识.
%%<SOLUTION>%%
由定理三, $r(4,4) \leqslant r(4,3)+r(3,4)=9+9=18$.
%%PROBLEM_END%%



%%PROBLEM_BEGIN%%
%%<PROBLEM>%%
问题9.求证: 将自然数 $1,2, \cdots, N$ 分到 $n$ 个类中, 则在 $N$ 充分大时,一定有一个类同时含有数 $x 、 y$ 及这两个数的差 $|x-y|$. (许尔(Schur) 定理)
%%<SOLUTION>%%
考虑 $n$ 色完全图 $K_{r_n}$, 染色方法是当且仅当 $|x-y|$ 在第 $i$ 类时,将 $(x$, $y$ ) 染上第 $i$ 种颜色.
由定理二, $K_{r_n}$ 中一定有一个同色的三角形.
设这三角形的三边均为第 $j$ 种颜色, 则在 $1,2, \cdots, r_n$ 中有三个自然数 $a>b>c$, 使 $x= a-c, y=a-b, z=b-c=x-y$ 均在第 $j$ 类中.
%%PROBLEM_END%%



%%PROBLEM_BEGIN%%
%%<PROBLEM>%%
问题10. 证明: 在两色完全图 $K_7$ 中, 必有两个无公共边的同色三角形.
%%<SOLUTION>%%
在 $A_1, A_2, \cdots, A_7$ 中, 前 6 个点构成的三角形必有两个是同色的.
它们如果没有公共边, 则命题已成立.
如果有公共边, 不妨设为 $\triangle A_1 A_2 A_3$ 、 $\triangle A_1 A_2 A_4$, 现除去 $A_1$ 点, 加进 $A_7$ 点, 则又存在两个同色三角形, 在这两个同色三角形中必有一个不同于 $\triangle A_2 A_3 A_4$, 这个三角形与 $\triangle A_1 A_2 A_3$ 或 $\triangle A_1 A_2 A_4$ 中的一个无公共边.
%%PROBLEM_END%%



%%PROBLEM_BEGIN%%
%%<PROBLEM>%%
问题11. 求最小正整数 $n$, 使得在任意给定的 $n$ 个无理数中, 总存在这样的三个无理数,其中任意两个数之和仍是无理数.
但没有公共边的两个单色三角形.
%%<SOLUTION>%%
取由四个无理数组成的集合 $\{\sqrt{2},-\sqrt{2}, \sqrt{3},-\sqrt{3}\}$. 从这个集合中任意取出三个数, 必然是: 或者 $\sqrt{2},-\sqrt{2}$ 都被取出, 此时 $\sqrt{2}+(-\sqrt{2})=0$ 是有理数; 或者 $\sqrt{3},-\sqrt{3}$ 都被取出, 此时 $\sqrt{3}+(-\sqrt{3})=0$ 是有理数.
可见当 $n=4$ 时题中结论不成立.
因此, 满足题中要求的 $n$, 必有 $n \geqslant 5$. 下面证明: 任意给定五个无理数, 总可以从中找到这样的三个, 其中任意两个无理数之和仍是无理数.
设 $\{x, y, z, u, v\}$ 是任意给定的由五个无理数组成的集合.
把这五个数当成五个点, 如果两个数之和为无理数, 则对应的两点之间联结一条红色边;
如果两个数之和为有理数,则对应的两点之间联结一条蓝色边.
于是得一个 2 色 5 阶完全图 $K_5$. 先证明这个 2 色 5 阶完全图 $K_5$ 中不含蓝边三角形.
否则, 设有蓝边三角形 $x y z$, 即 $x+y, y+z, z+x$ 都是有理数,则
$$
x=\frac{1}{2}[(x+y)+(z+x)-(y+z)]
$$
也应是有理数,与 $x$ 是无理数矛盾.
再证明这个 2 色完全图 $K_5$ 中不含蓝边五边形.
否则, 设有蓝边五边形 $x y z u v$, 即 $x+y, y+z, z+u, u+v, v+x$ 都是有理数, 则
$$
x=\frac{1}{2}[(x+y)+(z+u)+(v+x)-(y+z)-(u+v)]
$$
也应是有理数,与 $x$ 是无理数矛盾.
这个 $K_5$ 中既不含蓝边五边形, 也不含蓝边三角形, 故由图 8-3 后的 * 得知, 必含红边三角形.
设 $\triangle x y z$ 是红边三角形,则有 $x+y, y+z, z+x$ 都是无理数.
%%PROBLEM_END%%



%%PROBLEM_BEGIN%%
%%<PROBLEM>%%
问题12. 求最小正整数 $n$, 使当以任意方式将 $\mathrm{K}_{n}$ 二染色时,总存在具有相同颜色但没有公共边的两个单色三角形
%%<SOLUTION>%%
如图(<FilePath:./figures/fig-c8a12-1.png>),  对 $K_7$ 二染色, 红色边以实线、蓝色边以虚线表示.
有 4 个红色三角形: $\triangle A_1 A_4 A_6 、 \triangle A_2 A_4 A_6 、 \triangle A_3 A_4 A_6 、 \triangle A_7 A_4 A_6$, 有 4 个蓝色三角形: $\triangle A_1 A_2 A_3 、 \triangle A_2 A_3 A_7 、 \triangle A_1 A_3 A_7 、 \triangle A_1 A_2 A_7$, 易见任何两个相同颜色的单色三角形都有一条公共边,所以 $n \geqslant 8$.
下面证明 $n=8$ 时命题一定成立.
用反证法证之.
先证明一个引理: 倘若命题不成立, 则必存在一个红色三角形和一个蓝色三角形, 二者恰有一个公共点.
首先, 二染色 $K_8$ 必存在一个单色三角形.
不妨设其为蓝色 $\triangle A_1 A_2 A_3$. 这时 $A_3 A_4 A_5 A_6 A_7 A_8$ 中必存在一个单色三角形, 显然不能是蓝色三角形.
若这个红色三角形含顶点 $A_3$, 则引理成立.
不然设 $\triangle A_4 A_5 A_6$ 为红色三角形, 则 $\triangle A_1 A_2 A_3$ 与 $\triangle A_4 A_5 A_6$ 间有 9 条连线,至少有 5 条是同色的.
不妨设为红色, 则 $A_1, A_2, A_3$ 发出至少 5 条红边, 其中有一点至少发出 2 条红边, 由该点形.
成一个红色三角形且与 $\triangle A_1 A_2 A_3$ 有一个公共点.
故引理必成立.
下面证本命题: 倘若命题不成立, 由引理可设 $\triangle A_1 A_2 A_3$ 为蓝色三角形, $\triangle A_3 A_4 A_5$ 为红色三角形.
考虑 $A_1 A_4 A_6 A_7 A_8$ 间的连线,显然其内不能再有单色三角形,故该 $K_5$ 由一个蓝五圈和一个红五圈组成.
见如图(<FilePath:./figures/fig-c8a12-2.png>), 红色边用实线,蓝色边用虚线,不妨设 $A_1 A_4 A_6 A_7 A_8$ 为蓝五圈, $A_1 A_7 A_4 A_8 A_6$ 为红五圈,下面讨论 $A_3 A_7$ 的颜色.
若 $A_3 A_7$ 为蓝色, 则 $A_3 A_8 、 A_3 A_6$ 必为红色 (不然 $\triangle A_3 A_7 A_8$ 或 $A_3 A_6 A_7$ 为蓝色, 与 $\triangle A_1 A_2 A_3$ 发生矛盾). 而此时 $\triangle A_3 A_6 A_8$ 为红色, 又与 $\triangle A_3 A_4 A_5$ 发生矛盾!
若 $A_3 A_7$ 为红色.
再讨论 $A_3 A_8$ 的颜色.
若 $A_3 A_8$ 为蓝色, 则 $A_2 A_4$ 必须为红色, $A_2 A_8$ 必须为蓝色, $A_2 A_7$ 必须为红色.
故 $\triangle A_2 A_4 A_7$ 为红色三角形, 与 $\triangle A_3 A_4 A_5$ 发生矛盾!
若 $A_3 A_8$ 也是红色, 则 $A_5 A_7$ 必须是蓝色, $A_5 A_8$ 必须是蓝色.
故 $\triangle A_5 A_7 A_8$ 是蓝色, 与 $\triangle A_1 A_2 A_3$ 发生矛盾!
综上所述, $n=8$ 时命题必成立,故所求最小自然数是 8 .
%%PROBLEM_END%%



%%PROBLEM_BEGIN%%
%%<PROBLEM>%%
问题13. 在一个足球联赛里有 20 支足球队.
第一轮它们分成 10 对互相比赛, 第二轮也分成 10 对互相比赛 (注: 每支球队两轮比赛的对手不一定不同). 求证: 在第三轮开赛之前,一定可以找出 10 支球队, 它们两两没有赛过.
%%<SOLUTION>%%
用 20 个点代表 20 支球队.
在第一轮互相比赛过的球队之间连红线, 在第二轮互相比赛过的球队之间连蓝线, 则每一个点都连出一红一蓝两条边, 从而整个图必然由若干个偶圈组成.
在每个偶圈中可以选出半数顶点, 任两个顶点不相邻,则共选出了 10 支球队, 两两未赛过.
原命题得证.
%%PROBLEM_END%%


