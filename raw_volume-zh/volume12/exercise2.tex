
%%PROBLEM_BEGIN%%
%%<PROBLEM>%%
问题1. 设图 $G=(V, E),|V|=n,|E|=e$, 证明 $: \delta \leqslant \frac{2 e}{n} \leqslant \Delta$.
%%<SOLUTION>%%
因为 $\sum_{i=1}^n d\left(v_i\right)=2 e, n \delta \leqslant 2 e \leqslant n \Delta$, 所以 $\delta \leqslant \frac{2 e}{n} \leqslant \Delta$.
%%PROBLEM_END%%



%%PROBLEM_BEGIN%%
%%<PROBLEM>%%
问题2. 证明任何一群人 (多于两个人)中,至少有两个人,他们的朋友数目相同.
%%<SOLUTION>%%
设任意给定的一群人有 $n(n>2)$ 个.
用顶点表示这 $n$ 个人.
当且仅当顶点 $u, v$ 表示的两个人是朋友时令 $u, v$ 相邻, 得到 $n$ 个顶点的简单图 $G$.
对 $G$ 中任意 $x$, 由于它至多只能和其他 $n^{--}$个顶点相邻, 所以顶点 $x$ 的度 $d(x)$ 满足 $0 \leqslant d(x) \leqslant n-1$, 即图 $G$ 的顶点度只能是 $n-1$ 个非负数 $0,1, \cdots, n-1$ 中的一个.
如果图 $G$ 的顶点的度都不相同, 则图 $G$ 具有 0 度顶点 $u$ 和 $n-1$ 度顶点 $v . n-1$ 度顶点和 $G$ 中其他顶点都相邻, 特别地和顶点 $u$ 相邻.
但 0 度顶点 $u$ 和 $G$ 中任何顶点都不相邻, 矛盾.
这就证明了 $G$ 中必定有两个顶点, 它们的度相同.
也就是说, 这群人中必有两个人, 他们的朋友一样多.
%%PROBLEM_END%%



%%PROBLEM_BEGIN%%
%%<PROBLEM>%%
问题3. $n$ 名选手进行对抗赛, 每名选手至少赛一场, 每场两名选手参加, 已赛完 $n+1$ 场.
证明:至少有一名选手赛过三次.
%%<SOLUTION>%%
把选手看成顶点.
当且仅当 $v_i, v_j$ 所代表的两名选手比赛过时, 令 $v_i$, $v_j$ 相邻, 得到含 $n$ 个顶点的简单图 $G$. 由于总共赛过 $n+1$ 场, 所以图 $G$ 的边数是 $n+1$.
由定理, $d\left(v_1\right)+d\left(v_2\right)+\cdots+d\left(v_n\right)=2(n+1)$,
如果图 $G$ 中所有顶点的度都不超过 2 , 则由上式得到
$$
2(n+1)=d\left(v_1\right)+d\left(v_2\right)+\cdots+d\left(v_n\right) \leqslant 2 n,
$$
这不可能.
因此图 $G$ 中至少有一个顶点 $x$, 它的度至少是 3. 于是, 顶点 $x$ 所表示的选手至少赛过三次.
%%PROBLEM_END%%



%%PROBLEM_BEGIN%%
%%<PROBLEM>%%
问题4. 有一个参观团,其中任意四个成员中总有一名成员原先见过其他三名成员.
证明: 在任意四名成员中, 总有一名成员原先见过所有成员.
%%<SOLUTION>%%
如图(<FilePath:./figures/fig-c2a4.png>), 用图论语言表示即: 图 $G$ 的任意四点中至少有一个顶点和其他三个顶点相邻.
证明图 $G$ 任意四个顶点中至少一个顶点和 $G$ 中其他所有顶点都相邻.
反证法.
如果命题不成立, 则 $G$ 中有四个点 $x$, $y, z, w$, 它们和图 $G$ 中的其他所有顶点不都相邻.
于是存在四个顶点 $x^{\prime}, y^{\prime}, z^{\prime}, w^{\prime}$, 它们依次与 $x, y, z, w$ 都相邻.
不妨设这个点为 $x$. 如图.
所以 $x^{\prime}$ 不是 $y, z, w$ 中的一个, 且 $y^{\prime}$ 与 $x$ 是两个不同的顶点.
如果 $y^{\prime}$ 与 $x^{\prime}$ 不同,则 $x, y, x^{\prime}, y^{\prime}$ 中没有一个顶点和其他三个顶点都相邻, 和已知矛盾.
所以 $y^{\prime}$ 和 $x^{\prime}$ 重合.
同理可证, $z^{\prime}$ 和 $x^{\prime}$ 重合.
于是 $x^{\prime}$ 和 $y z w$ 都不相邻, 和已知矛盾.
这就证明了图 $G$ 中任意四个顶点中至少有一个顶点和 $G$ 的其他所有顶点都相邻.
%%<REMARK>%%
注:本题和例题 4 关系密切, 两个命题实际是等价的.
%%PROBLEM_END%%



%%PROBLEM_BEGIN%%
%%<PROBLEM>%%
问题5. 小城共有 15 部电话, 能否用电线连接它们, 使得每部电话恰好与 5 部别的电话相连?
%%<SOLUTION>%%
不能.
由定理二即得.
%%PROBLEM_END%%



%%PROBLEM_BEGIN%%
%%<PROBLEM>%%
问题6. 参加某次学术讨论会共有 123 个人, 已知每个人至少和 5 位与会者讨论过问题.
证明: 至少有一个人和 5 位以上的与会者讨论过问题.
%%<SOLUTION>%%
作一个图 $G: 123$ 个顶点 $v_1, v_2, \cdots, v_{123}$ 代表 123 个人, 如果两人讨论过问题, 则相应的顶点相邻.
于是图 $G$ 中的每一个顶点的度 $\geqslant 5$.
如果 $G$ 中没有一个顶点的度大于 5 , 那么 $G$ 中的每个顶点的度就都等于 5 , 这就得出 $G$ 中的奇顶点个数是奇数, 这是不可能的.
从而图 $G$ 中至少有一个顶点的度大于 5 .
%%PROBLEM_END%%



%%PROBLEM_BEGIN%%
%%<PROBLEM>%%
问题7. 在一次会议中,已知每个议员至多与三人不相识, 证明: 一定可以把所有议员分为两组,使每一组中, 每个议员至多与一人不相识.
个人中总可以找出 4 个人来, 这 4 个人可以围着圆桌坐下, 使得每个人旁边都是认识的人 $(n \geqslant 2)$.
%%<SOLUTION>%%
作图 $G: n$ 个点表示 $n$ 名议员, 若两人不认识, 则连一条边在图 $G$ 中, 于是对任意的点 $v_i$, 均有 $d\left(v_i\right) \leqslant 3$. 现把图 $G$ 任意分成两部分 $G_1$ 和 $G_2$. 在同一部分中的两点, 若原来有边相连, 则这条边仍留在这部分中.
在不同部分中的两点, 若原来有边相连, 则这条边就不再存在了.
把这些去掉的边放在一起构成集 $E$. 在两部分中, 如果有一点的度大于 1 , 不妨设在 $G_1$ 中有点 $v_1$, $d\left(v_1\right) \geqslant 2$, 把这个点移到 $G_2$ 中去.
此时 $G_1$ 中至少要消失两条边; 由于 $d\left(v_1\right) \leqslant$ 3 , 因而 $G_2$ 中至多增加一条边, 于是 $E$ 中至少增加一条边.
反复进行这一步骤.
这样 $E$ 中的边不断增加, 但总边数是有限的, 所以到某一步, 这两部分中再也没有度大于 1 的点.
于是命题得证.
%%PROBLEM_END%%



%%PROBLEM_BEGIN%%
%%<PROBLEM>%%
问题8. 有$2n$个人在一起聚会;其中每个人至少同其中的$n$个人认识,证明这 $2n$个人中总可以找出4个人来,这4个人可以围着圆桌坐下,使得每个人旁边都是认识的人 $(n\geq2)$ 
%%<SOLUTION>%%
问题可转化为: 在一个 $2 n$ 阶图 $G$ 中, 每个顶点的度 $\geqslant n$, 证明 $G$ 中有一个四边形.
若 $G=K_{2 n}$, 结论显然.
当 $G \neq K_{2 n}$ 时, 则存在点 $v_1, v_2$ 不相邻, 由于 $d\left(v_1\right)+d\left(v_2\right) \geqslant 2 n$, 根据抽屉原理, 其余的 $2 n-2$ 个点中必有两个点, 设为 $v_3, v_4$, 与 $v_1, v_2$ 都相邻, 于是这 4 点就构成一个四边形.
%%PROBLEM_END%%



%%PROBLEM_BEGIN%%
%%<PROBLEM>%%
问题9. 已知 9 个人 $v_1, v_2, \cdots, v_9$ 中 $v_1$ 和 2 个人握过手, $v_2 、 v_3$ 各和 4 个人握过手, $v_4 、 v_5 、 v_6 、 v_7$ 各和 5 个人握过手, $v_8 、 v_9$ 各和 6 个人握过手, 证明这 9 个人中一定可以找出 3 个人,他们互相握过手.
%%<SOLUTION>%%
作图 $G$ : 用 9 个点表示 9 个人,两顶点相邻当且仅当这两人握过手.
因为 $d\left(v_9\right)=6$, 所以存在 $v_k \neq v_1, v_2, v_3$ 与 $v_9$ 相邻.
显然 $d\left(v_k\right) \geqslant 5$. 在与 $v_9$ 相邻的其余五个点中一定有一个点 $v_h$ 与 $v_k$ 相邻 (否则 $d\left(v_k\right) \leqslant 9-5-1=3$ ), 于是 $v_9, v_k, v_h$ 即为所求.
%%PROBLEM_END%%



%%PROBLEM_BEGIN%%
%%<PROBLEM>%%
问题10. 一个旅行团中共14人;在山上休息时,他们想打桥牌,而其中每个人都曾和其中的 5 个人合作过.
现规定只有 4 个人中任两个人都未合作过, 才能在一起打一局牌.
这样, 打了三局就没法再打下去了.
这时, 来了另一位旅游者, 他当然没有与该旅行团中的任何人合作过.
如果他也参加打牌, 证明一定可以再打一局桥牌.
%%<SOLUTION>%%
用 14 个点 $v_1, v_2, \cdots, v_{14}$ 表示 14 个人, 两顶点 $v_i$ 与 $v_j$ 相邻当且仅当这两人未合作过, 得图 $G$. 在 $G$ 中每个顶点的度都是 8 . 打三局要去掉 6 条边, 因此至少有 2 个顶点的度数保持为 8 , 设其中之一为 $v_1$. 在与 $v_1$ 相邻的 8
个顶点中至少有一个 $v_2$ 的度数不小于 7 , 可以知道 $v_2$ 与和 $v_1$ 相邻的其余 7 个点之一设为 $v_3$ 相邻.
这样, $v_1, v_2, v_3$ 与新来的 $v$ 组成 $K_4$.
%%PROBLEM_END%%



%%PROBLEM_BEGIN%%
%%<PROBLEM>%%
问题11. 对于平面上任意 $n$ 个点构成的点集 $P$, 如果其中任意两点之间的距离均已确定, 那么就称这个点集是"稳定的". 求证:在 $n(n \geqslant 4)$ 个点的平面点集 $P$ 中, 无三点共线, 且其中的 $\frac{1}{2} n(n-3)+4$ 个两点之间的距离已被确定,那么点集 $P$ 就是"稳定的".
%%<SOLUTION>%%
先约定,确定距离的两点用边相连.
我们用数学归纳法来证明本题.
当 $n=4$ 时, $\frac{1}{2} n(n-3)+4=6$, 四点之间只有 $\left(\mathrm{C}_4^2=\right) 6$ 个距离, 它们均已确定,故命题成立.
设 $n=k(k \geqslant 4)$ 时命题成立.
当 $n=k+1$ 时,点集共连了
$$
\frac{1}{2}(k+1)(k-2)+4
$$
条边.
设 $A_{k+1}$ 是这个点集中"度" (即自该点出发的边数) 最小的点, 则其度
$$
\begin{aligned}
d\left(A_{k+1}\right) & \leqslant \frac{2\left[\frac{1}{2}(k+1)(k-2)+4\right]}{k+1} \\
& =k-2+\frac{8}{k+1} \\
& \leqslant k-2+\frac{8}{5}<k,
\end{aligned}
$$
所以 $d\left(A_{k+1}\right) \leqslant k-1$.
于是,剩下 $k$ 个点 $A_1, A_2, A_3, \cdots, A_k$ 之间至少连了
$$
\frac{1}{2}(k+1)(k-2)+4-(k-1)=\frac{1}{2} k(k-3)+4
$$
条边.
按归纳假设这 $k$ 个点的集合是稳定的.
又 $d\left(A_{k+1}\right) \geqslant \frac{1}{2}(k+1)(k-2)+4-\mathrm{C}_k^2=3$, 故 $A_{k+1}$ 至少与 $A_1, A_2, \cdots$, $A_k$ 中的 3 点相连.
不妨设 $A_{k+1}$ 与 $A_1, A_2, A_3$ 相连, 且 $A_{k+1} A_1=x, A_{k+1} A_2= y, A_{k+1} A_3=z$. 易证 $A_{k+1}$ 是唯一确定的.
若不然, 设 $A_{k+1}^{\prime}$ 是另外一点, 也有 $A_{k+1}^{\prime} A_1=x, A_{k+1}^{\prime} A_2=y, A_{k+1}^{\prime} A_3=z$, 则 $A_1, A_2, A_3$ 将都在 $A_{k+1} A_{k+1}^{\prime}$ 的垂直平分线上, 这与无三点共线的假定矛盾.
于是 $A_{k+1} A_4, \cdots, A_{k+1} A_k$ 都可确定,点集 $\left\{A_1, A_2, \cdots, A_{k+1}\right\}$ 是稳定的,即当 $n-k+1$ 时命题也成立.
综上所述, 命题得证.
%%PROBLEM_END%%



%%PROBLEM_BEGIN%%
%%<PROBLEM>%%
问题12. 棱长为 $n$ (自然数) 的立方体被平行于它的侧面的平面切成 $n^3$ 个单位立方体,其中有多少对公共顶点不多于 2 的单位立方体?
%%<SOLUTION>%%
以单位立方体为顶点, 当且仅当二单位立方体有公共侧面时, 在此二顶点间加一边, 构成一个图 $G, G$ 的补图 $\bar{G}$ 的边数即为所求.
易知 $G$ 的边数为 $3 n^2(n-1), K_{n^2}$ 的边数为 $\frac{1}{2} n^3\left(n^3-1\right)$, 故 $\bar{G}$ 的边数为 $\frac{1}{2} n^3\left(n^3-1\right)-3 n^2(n-1)=\frac{1}{2} n^6-\frac{7}{2} n^3+3 n^2$. 即公共顶点不多于 2 的单位立方体共有 $\frac{1}{2} n^6-\frac{7}{2} n^3+3 n^2$ 对.
%%PROBLEM_END%%



%%PROBLEM_BEGIN%%
%%<PROBLEM>%%
问题13. 某国首都有 21 条航线连接其他城市, 而 $A$ 城只有一条航线与某个城市连接,其他各城市中的每个城市都有 20 条航线连接别处.
证明: 由首都可以飞抵 $A$ 城.
%%<SOLUTION>%%
我们来观察包括首都在内的航线图中的连通成分, 构成一个连通图, 只要证明在这个连通图中也包括 $A$ 市.
假设不真.
于是由这个图的一个顶点 (首都)连出了 21 条边, 而从其他每个顶点都连出 20 条边.
这意味着这个图中恰有一个度数为奇数的顶点, 此为不可能!
%%PROBLEM_END%%


