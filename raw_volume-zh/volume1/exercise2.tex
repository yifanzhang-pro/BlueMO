
%%PROBLEM_BEGIN%%
%%<PROBLEM>%%
问题1 已知集合 $M=\{2,|a|\}$ 是全集 $U=\left\{2,3, a^2+2 a+2\right\}$ 的一个子集, 且 $\complement_U M=\{5\}$, 则实数 $a$ 的值等于
%%<SOLUTION>%%
-3 . $|a|=3$, 且 $a^2+2 a+2=5$. 解得 $a=-3$.
%%PROBLEM_END%%



%%PROBLEM_BEGIN%%
%%<PROBLEM>%%
问题2 已知全集 $I=\mathbf{Z}, M=\{x \mid x=2 n, n \in \mathbf{Z}\}, S=\{x \mid x=3 n, n \in \mathbf{Z}\}$, 则 $M \cap \complement_{\mathrm{z}} S=$
%%<SOLUTION>%%
$\{x \mid x=6 n \pm 2, n \in \mathbf{Z}\} . M=\{x \mid x=6 n, 6 n+2,6 n+4, n \in \mathbf{Z}\}$, $S=\{x \mid x=6 n, 6 n+3, n \in \mathbf{Z}\}$, 于是 $M \cap \complement_{\mathbf{Z}} S=\{x \mid x=6 n \pm 2, n \in \mathbf{Z}\}$.
%%PROBLEM_END%%



%%PROBLEM_BEGIN%%
%%<PROBLEM>%%
问题3 已知 $M=\left\{(x, y) \mid y=\sqrt{16-x^2}, y \neq 0\right\}, N=\{(x, y) \mid y=x-a\}$, 如果 $M \cap N \neq \varnothing$, 那么 $a$ 的范围是
%%<SOLUTION>%%
$[-4 \sqrt{2}, 4) . M \cap N \neq \varnothing \Leftrightarrow y=\sqrt{16-x^2}=x-a \neq 0$ 有解.
从而有 $-4<x<4, a<4$, 且 $2 x^2-2 a x+a^2-16=0$ 有实数解.
%%PROBLEM_END%%



%%PROBLEM_BEGIN%%
%%<PROBLEM>%%
问题4 $A=\left\{x \mid x^2+(p+2) x+1=0, x \in \mathbf{R}\right\}$, 且 $\{x \mid x>0\} \cap A=\varnothing$, 则实数 $p$ 的取值范围是
%%<SOLUTION>%%
$(-4,+\infty)$. 依题意, $x^2+(p+2) x+1=0$ 无解或有两个小于 0 的解.
所以, $\Delta<0$ 或 $\Delta \geqslant 0$ 且 $-(p+2)<0$.
%%PROBLEM_END%%



%%PROBLEM_BEGIN%%
%%<PROBLEM>%%
问题5 若 $M=\left\{(x, y)|| \tan \pi y \mid+\sin ^2 \pi x=0\right\}, N=\left\{(x, y) \mid x^2+y^2 \leqslant 2\right\}$, 则 $M \cap N$ 的元素个数是
%%<SOLUTION>%%
9. $M=\{(x, y) \mid x=k, y=l, k, l \in \mathbf{Z}\}$.
%%PROBLEM_END%%



%%PROBLEM_BEGIN%%
%%<PROBLEM>%%
问题6 已知两个复数集合 $M=\left\{z \mid z=\cos \alpha+\left(4-\cos ^2 \alpha\right) \mathrm{i}, \alpha \in \mathbf{R}\right\}, N= \{z \mid z=\cos \beta+(\lambda+\sin \beta) i, \beta \in \mathbf{R}\}$. 当 $M \cap N \neq \varnothing$ 时, 实数 $\lambda$ 的取值范围是
%%<SOLUTION>%%
$\left[\frac{11}{4}, 5\right]$. 设 $\cos \alpha+\left(4-\cos ^2 \alpha\right) \mathrm{i}=\cos \beta+(\lambda+\sin \beta) \mathrm{i}$, 则有 $\cos \alpha= \cos \beta, 4-\cos ^2 \alpha=\lambda+\sin \beta$. 消去 $\alpha$, 得 $\lambda=\frac{11}{4}+\left(\sin \beta-\frac{1}{2}\right)^2$, 所以 $\frac{11}{4} \leqslant \lambda \leqslant 5$.
%%PROBLEM_END%%



%%PROBLEM_BEGIN%%
%%<PROBLEM>%%
问题7 若集合 $A=\{x \mid-2 \leqslant x \leqslant 5\}, B=\{x \mid m+1 \leqslant x \leqslant 2 m-1\}$, 且 $A \cap B=B$, 则实数 $m$ 的取值范围是
%%<SOLUTION>%%
$m \leqslant 3$. 当 $B=\varnothing$ 时, $m+1>2 m-1$, 得 $m<2$; 当 $B \neq \varnothing$ 时, 须 $-2 \leqslant m+1$, $2 m-1 \leqslant 5, m+1 \leqslant 2 m-1$ 同时满足, 解得 $2 \leqslant m \leqslant 3$.
%%PROBLEM_END%%



%%PROBLEM_BEGIN%%
%%<PROBLEM>%%
问题8 设全集 $U=\left\{x \mid x=2 n-1, n \in \mathbf{N}^*, n \leqslant 7\right\}, A \cap\left(\complement_U B\right)=\{3,7\}$, $\left(\complement_U A\right) \cap B=\{9,13\},\left(\complement_U A\right) \cap\left(\complement_U B\right)=\{1,11\}$. 则 $A=? B=?$
%%<SOLUTION>%%
如图(<FilePath:./figures/fig-c2p8.png>)所示,
$A=\{3,5,7\}, B=\{5,9,13\}$.
%%PROBLEM_END%%



%%PROBLEM_BEGIN%%
%%<PROBLEM>%%
问题9 设集合 $A=\left\{x \mid x^2-a x+a^2-19=0\right\}, B=\left\{x \mid x^2-5 x+6=0\right\}$, $C=\left\{x \mid x^2+2 x-8=0\right\}$, 且 $A \cap B \neq \varnothing, A \cap C=\varnothing$. 则实数 $a=?$
%%<SOLUTION>%%
$a=-2 . B=\{2,3\}, C=\{2,-4\}$. 由第 8 题图题设知 $3 \in A, 2 \notin A,-4 \notin A$. 将 3 代入方程 $x^2-a x+a^2-19=0$, 得 $a=-2$ 或 5 . 然后逐一检验.
%%PROBLEM_END%%



%%PROBLEM_BEGIN%%
%%<PROBLEM>%%
问题10 集合 $A=\left\{(x, y) \mid \sin (3 x+5 y)>0\right.$, 且 $\left.x^2+y^2 \leqslant \pi^2\right\}$ 所构成的平面图形的面积是
%%<SOLUTION>%%
$\frac{1}{2} \pi^3 . A=\{(x, y) \mid 2 k \pi<3 x+ 5 y<2 k \pi+\pi, k \in \mathbf{Z}, \text{且} x^2+y^2 \leqslant \pi^2\}, A$ 所成图形为图(<FilePath:./figures/fig-c2p10.png>)中阴影部分.
%%PROBLEM_END%%



%%PROBLEM_BEGIN%%
%%<PROBLEM>%%
问题11 设 $m, n \in \mathbf{N}^*$, 且 $m>n$, 集合 $A=\{1,2, \cdots, m\}, B=\{1,2, \cdots, n\}$, 又 $C \subset A, B \cap C \neq \varnothing$. 则这样的集合 $C$ 的个数是
%%<SOLUTION>%%
$2^{m-n}\left(2^n-1\right)$. 由于 $A$ 的子集中只有由自然数 $n+1, n+2, \cdots, m$ 中任取若干个数组成的集合 $C^{\prime}$,才能使 $B \cap C^{\prime}=\varnothing$, 而这样的集合 $C^{\prime}$ 有 $2^{m-n}$ 个.
所以满足 $B \cap C \neq \varnothing$ 的集合 $C$ 的个数是 $2^m-2^{m-n}$.
%%PROBLEM_END%%



%%PROBLEM_BEGIN%%
%%<PROBLEM>%%
问题12 设 $\left\{a_n\right\}$ 为等差数列, $d$ 为公差, 且 $a_1$ 和 $d$ 均为实数, $d \neq 0$, 它的前 $n$ 项和记作 $S_n$. 设集合 $A=\left\{\left(a_n, \frac{S_n}{n}\right) \mid n \in \mathbf{N}^*\right\}, B=\{(x, y) \mid \frac{1}{4} x^2-y^2= 1, x, y \in \mathbf{R}\}$. 下列结论是否正确? 如果正确, 请给出证明; 如果不正确, 请举一个例子说明:
(1) 以集合 $A$ 中的元素为坐标的点都在同一直线上;
(2) $A \cap B$ 至多有一个元素;
(3) $a_1 \neq 0$ 时,一定有 $A \cap B \neq \varnothing \varnothing$.
%%<SOLUTION>%%
(1) 正确.
因 $S_n=\frac{\left(a_1+a_n\right)^n}{2}$, 所以 $\frac{S_n}{n}=\frac{a_1+a_n}{2}$. 这说明点 $\left(a_n, \frac{S_n}{n}\right)$ 在直线 $y=\frac{a_1+x}{2}$ 上.
(2) 正确.
设 $(x, y) \in A \cap B$, 则 $y=\frac{a_1+x}{2}$, 且 $\frac{1}{4} x^2-y^2=1$. 消去 $y$, 得 $2 a_1 x+a_1^2=-4$. 当 $a_1=0$ 时, 方程组无解; 当 $a_1 \neq 0$ 时, $x=\frac{-4-a_1^2}{2 a_1}$, 方程组恰有一个解.
故 $A \cap B$ 至多有一个元素.
(3) 不正确.
举例如下: 取 $a_1=1, d=1$, 此时若有 $A \cap B \neq \varnothing$, 则存在 $(x, y) \in A \cap B$. 由 (2) 有 $x=\frac{-4-a_1^2}{2 a_1}=-\frac{5}{2}<0$. 另一方面, 对一切 $n \in \mathbf{N}^*, a_n=n>0$, 所以 $(x, y) \notin A$. 这与 $(x, y) \in A \cap B$ 矛盾.
%%PROBLEM_END%%



%%PROBLEM_BEGIN%%
%%<PROBLEM>%%
问题13 已知 $A=\{(x, y) \mid x=n, y=n a+b, n \in \mathbf{Z}\}, B=\{(x, y) \mid x=m, y=3 m^2+15, m \in \mathbf{Z} \}, C=\left\{(x, y) \mid x^2+y^2 \leqslant 144\right\}$ 是坐标平面内的三个点集.
试问, 是否存在实数 $a 、 b$, 使得 $A \cap B \neq \varnothing$ 、点 $(a, b) \in C$ 同时成立? 若存在, 请求出 $a 、 b$ 的值; 若不存在, 则说明理由.
%%<SOLUTION>%%
假设存在实数 $a 、 b$, 同时满足题中的两个条件, 则必存在整数 $n$, 使 $3 n^2-a n+(15-b)=0$, 于是它的判别式 $\Delta=(-a)^2-12(15-b) \geqslant 0$, 即 $a^2 \geqslant 12(15-b)$. 又由 $a^2+b^2 \leqslant 144$ 得 $a^2 \leqslant 144-b^2$, 由此便得 $12(15- b) \leqslant 144-b^2$, 即 $(b-6)^2 \leqslant 0$, 故 $b=6$. 将 $b=6$ 代入上述的 $a^2 \geqslant 12(15-b)$ 及 $a^2 \leqslant 144-b^2$ 得 $a^2=108$, 所以 $a= \pm 6 \sqrt{3}$. 将 $a= \pm 6 \sqrt{3} 、 b=6$ 代入方程 $3 n^2-a n+(15-b)=0$, 求得 $n= \pm \sqrt{3} \notin \mathbf{Z}$. 这说明满足已知两个条件的实数 $a 、 b$ 是不存在的.
%%PROBLEM_END%%



%%PROBLEM_BEGIN%%
%%<PROBLEM>%%
问题14 给定自然数 $a \geqslant 2$, 集合 $A=\left\{y \mid y=a^x, x \in \mathbf{N}\right\}, B=\{y \mid y=(a+$ 1) $x+b, x \in \mathbf{N}\}$. 在区间 $[1, a]$ 上是否存在 $b$, 使得 $C=A \cap B \neq \varnothing$ ? 如果存在, 试求 $b$ 的一切可能值及相应的集合 $C$; 如果不存在, 试说明理由.
%%<SOLUTION>%%
因 $a \geqslant 2, a \in \mathbf{N}, x \in \mathbf{N}$, 所以 $a^x \in \mathbf{N}$, 且 $(a+1) x \in \mathbf{N}$. 又因为 $A \cap B \neq \varnothing$, 所以 $b \in \mathbf{N}$. 只需求 $b \in \mathbf{N}$ 的 $b$ 值, 使得满足 $(a+1) x_2+b=a^{x_1}$, 即 $x_2=\frac{a^{x_1}-b}{a+1}, x_1, x_2 \in \mathbf{N}$. 当 $x_1$ 为正偶数时, $x_2=-\frac{1-a^{x_1}}{1+a}-\frac{b-1}{1+a}$. 因为 $-\frac{1-a^{x_1}}{1+a} \in \mathbf{N}, 1 \leqslant b \leqslant a$, 所以 $0 \leqslant \frac{b-1}{1+a}<1$. 因 $x_2 \in \mathbf{N}$, 故 $b=1$. 当 $x_1$ 为正奇数时, $x_2=\frac{a^{x_1}+1}{a+1}-\frac{b+1}{a+1}$. 因为 $\frac{a^{x_1}+1}{a+1}$ 是大于 1 的自然数, $0 \leqslant \frac{b+1}{a+1} \leqslant 1, x_2 \in \mathbf{N}$, 所以 $b=a$. 综上知, 在 $[1, a]$ 上存在 $b=1$ 或 $b=a$, 使得 $A \cap B \neq \varnothing$. 当 $b=1$ 时, $A \cap B=\left\{y \mid y=a^{2 x}, x \in \mathbf{N}\right\}$; 当 $b=a$ 时, $A \cap B=\{y \mid y=a^{2 x+1}, x \in \mathbf{N}\}$.
%%PROBLEM_END%%



%%PROBLEM_BEGIN%%
%%<PROBLEM>%%
问题15 设 $Z$ 表示所有整数的集合.
对于固定的 $A, B, C \in \mathbf{Z}$, 令
$$
\begin{aligned}
& M_1=\left\{x^2+A x+B \mid x \in \mathbf{Z}\right\}, \\
& M_2=\left\{2 x^2+2 x+C \mid x \in \mathbf{Z}\right\},
\end{aligned}
$$
求证: 对任何 $A, B \in \mathbf{Z}$, 都可选取 $C \in \mathbf{Z}$, 使得集合 $M_1$ 与 $M_2$ 互不相交.
%%<SOLUTION>%%
如果 $A$ 为奇数, 则有 $x(x+A)+B \equiv B(\bmod 2)$, 这表明 $M_1$ 中的所有数都与 $B$ 奇偶性相同.
对于 $M_2$ 中的数, 有 $2 x(x+1)+C \equiv C(\bmod 2)$. 可见, 为使 $M_1 \cap M_2=\varnothing$, 只须取 $C=B+1$ 即可.
如果 $A$ 为偶数, 则有 $2 x(x+1)+C \equiv C(\bmod 4)$. 又因 $\left(x+\frac{A}{2}\right)^2$ 作为完全平方数模 4 时只能为 0 或 1 , 故由 $x^2+A x+B=\left(x+\frac{A}{2}\right)^2-\left(\frac{A}{2}\right)^2+B$ 知 $M_1$ 中元素模 4 时只能与 $B 、 B+1$ 或 $B+3$ 同余.
因而, 当取 $C=B+2$ 时, $M_1 \cap M_2=\varnothing$.
%%PROBLEM_END%%



%%PROBLEM_BEGIN%%
%%<PROBLEM>%%
问题16 设集合 $S_n=\{1,2, \cdots, n\}$, 若 $Z$ 是 $S_n$ 的子集, 把 $Z$ 中的所有数的和称为 $Z$ 的“容量” (规定空集的容量为 0 ). 若 $Z$ 的容量为奇 (偶) 数, 则称 $Z$ 为 $S_n$ 的奇 (偶)子集.
(1) 求证: $S_n$ 的奇子集与偶子集个数相等;
(2)求证: 当 $n \geqslant 3$ 时, $S_n$ 的所有奇子集的容量之和与所有偶子集的容量之和相等;
(3)当 $n \geqslant 3$ 时, 求 $S_n$ 的所有奇子集的容量之和.
%%<SOLUTION>%%
设 $S$ 为 $S_n$ 的奇子集, 令 $T=\left\{\begin{array}{l}S \cup\{1\}, \text { 若 } 1 \notin S, \\ S \backslash\{1\} \text {, 若 } 1 \in S .\end{array}\right.$ 则 $T$ 是偶子集, $S \rightarrow T$ 是奇子集到偶子集的一一对应, 而且对每个偶子集 $T$, 恰有一个奇子集 $S=\left\{\begin{array}{l}T \cup\{1\}, \text { 若 } 1 \notin T, \\ T \backslash\{1\}, \text { 若 } 1 \in T .\end{array}\right.$ 与之对应, 所以(1) 的结论成立.
对任一 $i(1 \leqslant i \leqslant n)$, 含 $i$ 的子集共 $2^{n-1}$ 个, 用上面的对应方法可知当 $i \neq 1$ 时, 这 $2^{n-1}$ 个集中有一半是奇子集.
当 $i=1$ 时, 由于 $n \geqslant 3$, 将上边的 1 换成 3 , 同样可得其中有一半是奇子集.
于是在计算奇子集容量之和时, 元素 $i$ 的贡献是 $2^{n-2} \cdot i$. 奇子集容量之和是 $\sum_{i=1}^n 2^{n-2} i=n(n+1) \cdot 2^{n-3}$. 由上可知, 这也是偶子集的容量之和, 两者相等.
%%PROBLEM_END%%



%%PROBLEM_BEGIN%%
%%<PROBLEM>%%
问题17 已知一族集合 $A_1, A_2, \cdots, A_n$ 具有性质:
(1) 每个 $A_i$ 含 30 个元素;
(2) 对每一对 $i 、 j, 1 \leqslant i<j \leqslant n, A_i \cap A_j$ 恰含有一个元素;
(3) $A_1 \cap A_2 \cap \cdots \cap A_n=\varnothing$.
求使这些集合存在的最大的正整数 $n$.
%%<SOLUTION>%%
最大的 $n=871$. 若 $n \geqslant 872$, 则 $A_1$ 中必有一个元素 $a$ 至少属于除 $A_1$外的 30 个集合 (因 $29 \times 30+1=871<n$ ). 设 $a \notin A_i$, 每个含 $a$ 的集与 $A_i$ 有一个公共元, 故 $A_i$ 至少有 31 个元, 矛盾.
如下 871 个集满足题设: $A=\left\{a_0, a_1, \cdots, a_{29}\right\} ; B_i=\left\{a_0, a_{i, 1}\right.$, $\left.a_{i, 2}, \cdots, a_{i, 29}\right\}, 1 \leqslant i \leqslant 29 ; A_{i, j}=\left\{a_i\right\} \cup\left\{a_{k, j+(k-1)(i-1)}, k=1,2, \cdots, 29\right\}$, $1 \leqslant i, j \leqslant 29$, 其中 $a_{k, s}$ 与 $a_{k, s+29}$ 是同一个元素.
容易验证 $A \cap B_i, A \cap A_{i, j}$, $B_i \cap B_j(i \neq j), B_s \cap A_{i, j}, A_{i, j} \cap A_{i, t}(j \neq t), A_{i, j} \cap A_{s, j}(i \neq s)$ 为单元素集.
而 $A_{i, j} \cap A_{s, t}=\left\{a_{h, j+(h-1)(i-1)}\right\}, i \neq s, j \neq \bar{t}$, 其中 $h$ 是 $(x-1)(i-s) \equiv t-j(\bmod 29)$ 的惟一解.
%%PROBLEM_END%%



%%PROBLEM_BEGIN%%
%%<PROBLEM>%%
问题18 设 $M=\{1,2, \cdots, 20\}$, 对于 $M$ 的任一 9 元子集 $S$, 函数 $f(S)$ 取 $1 \sim 20$ 之间的整数值.
求证: 不论 $f$ 是怎样的一个函数, 总存在 $M$ 的一个 10 元子集 $T$, 使得对所有的 $k \in T$, 都有
$$
f(T-\{k\}) \neq k(T-\{k\} \text { 为 } T \text { 对 }\{k\} \text{的差集}).
$$
%%<SOLUTION>%%
如果一个 10 元子集 $T$ 具有性质: 对任何 $k \in T$, 均有 $f(T-\{k\}) \neq k$, 我们就称 $T$ 为 “好集”. 不是 “好集” 的 10 元子集称为 “坏集”, 也就是说, 如果 $T$ 为“坏集”, 则在 $T$ 中必有一 $k_0$, 使 $f\left(T-\left\{k_0\right\}\right)=k_0$. 若令 $S=T-\left\{k_0\right\}$, 这是一个 9 元子集, 则一方面 $f(S)=k_0$, 另一方面 $T=S \cup\left\{k_0\right\}$, 即 $T=S \cup \{f(S)\}$. 上式表示了 “坏集” 的结构, 它可由某一个 9 元子集 $S$ 生成, 即 $S$ 与 $\{f(S)\}$ 的并集构成了“坏集”.
如果 $f(S) \in S$, 那么 $S \cup\{f(S)\}$ 是一个 9 元子集, 而不是 10 元子集, 即不构成“坏集”. 因此任一 9 元子集按上式至多能生成一个“坏集”(由函数的定义,对于给定的 $S, f(S)$ 是惟一的). 于是, “坏集” 的个数 $\leqslant 9$ 元子集的个数 $<10$ 元子集的个数.
最后一个不等式成立是因为, 9 元子集的个数是 $\mathrm{C}_{20}^9$, 而 10 元子集的个数是 $\mathrm{C}_{20}^{10}=\frac{11}{10} \mathrm{C}_{20}^9>\mathrm{C}_{20}^9$. 由此可知, “好集”是存在的.
%%PROBLEM_END%%



%%PROBLEM_BEGIN%%
%%<PROBLEM>%%
问题19 设 $k \geqslant 6$ 为自然数, $n=2 k-1$. $T$ 为所有 $n$ 元数组 $\left(x_1, x_2, \cdots, x_n\right)$ 的集合, 其中 $x_i \in\{0,1\}, i=1,2, \cdots, n$. 对于 $x=\left(x_1, x_2, \cdots, x_n\right), y= \left(y_1, y_2, \cdots, y_n\right) \in T$, 定义
$$
d(x, y)=\sum_{i=1}^n\left|x_i-y_i\right| .
$$
特别地有 $d(x, x)=0$. 设有一个由 $T$ 的 $2^k$ 个元素组成的子集 $S$, 使对任何 $x \in T$, 都存在惟一的 $y \in S$, 使得 $d(x, y) \leqslant 3$. 求证: $n=23$.
%%<SOLUTION>%%
由于 $d(x, x)=0<3$, 所以 $S$ 中任何两个元素的距离都大于 3 . 因而当将 $S$ 中的元素 $x$ 的 $n$ 个分量改变 1 个、2 个或 3 个 (1 变为 0 或 0 变为 1 ) 时, 所得的元素都在 $T-S$ 中, 且 $S$ 中的不同元素按上述办法所得的元素也互不相同.
按假设知, 对每个 $x \in T$, 都有惟一的 $y \in S$, 使得 $d(x, y) \leqslant 3$, 所以 $T- S$ 中的每个元素都可由 $S$ 中的元素按上述办法生成.
从而有 $2^n=2^k\left(\mathrm{C}_n^0+\mathrm{C}_n^1+\right. \left.\mathrm{C}_n^2+\mathrm{C}_n^3\right)$. 由于 $n=2 k-1$, 故有 $3 \cdot 2^{k-2}=k\left(2 k^2-3 k+4\right)$. (1) 若 $3 \times k$, 则 $k= 2^m$. 由于 $k \geqslant 6$, 所以 $m \geqslant 3$. 于是 $2 k^2-3 k+4$ 是 4 的倍数但不是 8 的倍数.
从而由 (1) 可得 $2 k^2-3 k+4=12$, 这个方程无解, 所以 $k$ 为 3 的倍数.
记 $k= 3 h=3 \cdot 2^q, q \geqslant 1$, 于是 (1) 式化为 $2^{3 h-2}=h\left(18 h^2-9 h+4\right)$. (2) 当 $q \geqslant 3$ 时, $18 h^2-9 h+4$ 是 4 的倍数不是 8 的倍数.
由 (2) 知 $18 h^2-9 h+4=4$, 无解.
从而 $q=1$ 或 $2, h=2$ 或 4 . 代入 (2) 式知 $h=2$ 不是根而 $h=4$ 是根.
所以得到 $n=2 k-1==6 h-1=23$.
%%PROBLEM_END%%



%%PROBLEM_BEGIN%%
%%<PROBLEM>%%
问题20 设 $n \in \mathbf{N}^*$, 而 $A_1, A_2, \cdots, A_{2 n+1}$ 是集合 $B$ 的一族子集且满足条件:
(1) 每个 $A_i$ 中恰含有 $2 n$ 个元素;
(2) $A_i \cap A_j(1 \leqslant i<j \leqslant 2 n+1)$ 恰含有一个元素;
(3) $B$ 中每个元素至少属干两个子集 $A_{i_1}$ 和 $A_{i_2}, 1 \leqslant i_1<i_2 \leqslant 2 n+1$.
试问:对怎样的 $n \in \mathbf{N}^*$, 可以将 $B$ 中的每一个元素贴上一张写有 0 或 1 的标签, 使得每个 $A_i$ 中恰好有 $n$ 个元素贴有标签 0 ? 试说明理由.
%%<SOLUTION>%%
首先, 由条件(1)-(3) 可以导出更强的条件: (3') $B$ 中每个元素恰好属于 $A_1, A_2, \cdots, A_{2 n+1}$ 中的两个.
若不然, 不妨设有 $b \in B$, 使得 $b \in A_1 \cap A_2 \cap A_3$. 因为对任何 $i \neq j, A_i \cap A_j$ 恰有一个元素, 故知 $A_1 \cap\left(\bigcup_{j=2}^{2 n+1} A_j\right)= \left[A_1 \cap\left(A_2 \cup A_3\right)\right] \cup\left[A_1 \cap\left(\bigcup_{j=4}^{2 n+1} A_j\right)\right]$ 中至多有 $2 n-1$ 个元素.
另一方面, 又由 (3) 知 $A_1$ 中每个元素至少属于另外的某个 $A_i(i \neq 1)$, 所以 $A_1 \cap\left(\bigcup_{j=2}^{2 n+1} A_j\right)$ 中又应有 $2 n$ 个元素, 矛盾.
由 $\left(3^{\prime}\right)$ 知, 对于每个 $a \in B, a$ 都对应于由两个不同正整数组成的一个数对.
具体地说, 若 $a \in A_i \cap A_j$, 则令 $a$ 与 $\{i, j\}$ 相对应.
显然, 这个对应是个双射.
若能按要求为 $B$ 中的数标上 0 和 1 , 则在上述数组中恰有一半与 0 对应.
这样的数组的个数为 $\frac{1}{2} \mathrm{C}_{2 n+1}^2=\frac{1}{2} n(2 n+1)$. 可见, $n$ 必为偶数.
用构造法证明, 把一个圆周用 $2 n+1$ 个点均分成 $2 n+1$ 等分.
在这些点依逆时针顺序标上 $1,2, \cdots, 2 n+1$. 对于任何 $1 \leqslant i<j \leqslant 2 n+1$, 看 $i$ 与 $j$ 在圆周上的劣弧, 若 $i$ 与 $j$ 之间有奇数段弧, 则给 $\{i, j\}$ 所对应的 $a \in B$ 标上数 1 ; 若有偶数段弧, 则标上数 0 . 由于 $n$ 为偶数, 因此不论 $i$ 为何值, $A_i$ 中的元素都恰有一半标有数 0 而另一半标有数 1 .
%%PROBLEM_END%%


