
%%TEXT_BEGIN%%
我们可以将某些集合取来作为元素构成一个新的集合, 如 $A^*=\{\{1\},\{0,1\},\{0\}, \varnothing\}$ 就是一个含有 4 个元素 $\{1\} 、\{0,1\} 、\{0\} 、 \varnothing$ 的集合.
特别地, 将集合 $M$ 的若干子集作为元素构成的集合 $M^*$ 叫做原集合的一个子集族.
例如前面的 $A^*$ 就是二元集 $A=\{0,1\}$ 的全部子集所构成的子集族.
子集族中所含原来集合的子集的数目叫做该子集族的阶.
例如子集族 $A^*$ 的阶为 4 , 即 $\left|A^*\right|=4$.
一、C 族最简单的子集族是由有限集 $M$ 的全体子集所构成的子集族, 简称为 $C$ 族.
$C$ 族有如下基本的性质:
性质设 $|M|=n$, 则集合 $M$ 的全部子集构成的集合 $M^*$ 的阶为 $2^n$, 即
$$
\left|M^*\right|=\mathrm{C}_n^0+\mathrm{C}_n^1+\cdots+\mathrm{C}_n^n=2^n .
$$
%%TEXT_END%%



%%TEXT_BEGIN%%
二、求解子集族求解子集族的问题主要有两类: 求子集族的阶, 或确定集合的满足特定条件的子集族中的每个集合.
%%TEXT_END%%



%%PROBLEM_BEGIN%%
%%<PROBLEM>%%
例1. 试证: 任一有限集的全部子集可以排定次序, 使得任何相邻的两个子集都相差一个元素.
%%<SOLUTION>%%
分析:不妨设有限集 $A=\{1,2,3, \cdots, n\}$. 先来看一些简单情形:
当 $n=1$ 时,显然可以排成: $\varnothing,\{1\}$;
当 $n=2$ 时,共有 $2^2=4$ 个子集,可排成: $\varnothing,\{1\},\{1,2\},\{2\}$;
当 $n=3$ 时,共有 $2^3=8$ 个子集,可排成: $\varnothing,\{1\},\{1,2\},\{2\},\{2,3\},\{1,2,3\},\{1,3\},\{3\}$.
显然符合条件的排序方式不是惟一的.
请注意 $n=3$ 时的上述排法: 所有子集可分为两组, 前 4 个子集都不含元素 3 ; 后 4 个均含元素 3, 且去掉 3 后恰是前 4 个子集排列的逆序.
事实上, $n=2$ 时也如此.
这说明我们可以考虑用数学归纳法来证明.
证明设有限集为 $M_n=\left\{w_1, w_2, \cdots, w_n\right\}$, 我们对 $n$ 进行归纳.
当 $n=1$ 时, $M_1=\left\{w_1\right\}$,将它的两个子集排列成 $\varnothing,\left\{w_1\right\}$ 即可.
假设当 $n=k$ 时,命题成立.
当 $n=k+1$ 时,
$$
M_{k+1}=\left\{w_1, w_2, \cdots, w_k, w_{k+1}\right\},
$$
它是由集合 $M_k=\left\{w_1, w_2, \cdots, w_k\right\}$ 添加元素 $w_{k+1}$ 而形成的.
$M_k$ 的子集个数为 $2^k$. 由归纳假设知, 可将 $M_k$ 的全体子集排成满足题设要求的一列, 不妨设
$$
A_1, A_2, A_3, \cdots, A_{2^k}\left(A_i \subseteq M_k, i=1,2,3, \cdots, 2^k\right)
$$
就是这样的一个排列.
我们来看排列
$$
A_1, A_2, A_3, \cdots, A_{2^k}, A_{2^k} \bigcup\left\{w_{k+1}\right\}, A_{2^k-1} \bigcup\left\{w_{k+1}\right\}, \cdots, A_1 \bigcup\left\{w_{k+1}\right\},
$$
它恰好由 $M_{k+1}$ 的 $2^{k+1}$ 个不同子集排成, 且任意两个相邻集合的元素都仅相差 1 个.
可见当 $n=k+1$ 时, 命题也成立.
所以,对任意的 $n \in \mathbf{N}^*$, 所述命题成立.
说明一个复杂的问题,也许一时找不到解题的突破口, 这时可考虑“以退求进”的策略.
先解决一些简单的或特殊的情形, 从中发现规律和方法, 从而找到解决一般问题的办法.
这也就是从特殊到一般的思维方法.
%%PROBLEM_END%%



%%PROBLEM_BEGIN%%
%%<PROBLEM>%%
例2. 在某次竞选中各政党作出 $n$ 种不同的诺言 $(n>0)$, 有些政党可以作某些相同的诺言.
现知其中每两个政党都至少作了一个相同的诺言, 但没有两个政党的诺言完全相同.
求证: 政党个数 $\leqslant 2^{n-1}$.
%%<SOLUTION>%%
证明:设有 $m$ 个政党.
以 $A$ 记所有诺言的集合, $A_i$ 记第 $i$ 个政党的诺言的集合 $(i=1,2, \cdots, m)$. 由题设知
$$
|A|=n, A_i \cap A_j \neq \varnothing, A_i \neq A_j, 1 \leqslant i<j \leqslant m .
$$
因 $\left(\complement_A A_i\right) \cap A_i=\varnothing$, 故 ${ }_A A_i \neq A_j(i, j=1,2, \cdots, m)$, 即 $\complement_A A_i$ 不同于任何一个政党的诺言的集合.
所以
$$
A_1, A_2, \cdots, A_m, \complement_A A_1, \complement_A A_2, \cdots, \complement_A A_m
$$
各不相同, 而它们的个数不超过集合 $A$ 的所有子集的数目 $2^n$, 即 $2 m \leqslant 2^n$, 所以
$$
m \leqslant 2^{n-1} .
$$
说明上述解法的特点是将一个趣味问题转化为集合问题, 然后借助集合的知识和方法来解决.
%%PROBLEM_END%%



%%PROBLEM_BEGIN%%
%%<PROBLEM>%%
例3. 设正整数 $n \geqslant 5, n$ 个不同的正整数 $a_1, a_2, \cdots, a_n$ 有下列性质: 对集合 $S=\left\{a_1, a_2, \cdots, a_n\right\}$ 的任何两个不同的非空子集 $A$ 和 $B, A$ 中所有数的和与 $B$ 中所有数的和都不会相等.
在上述条件下,求
$$
\frac{1}{a_1}+\frac{1}{a_2}+\cdots+\frac{1}{a_n}
$$
的最大值.
%%<SOLUTION>%%
分析:因为 $S$ 的任何两个不同的非空子集的各自元素之和不相等, 由集合元素的互异性及正整数二进制表示的惟一性的启示, 似乎集合 $S$ 中的数应是形如 $2^r(r \in \mathbf{N})$ 的数.
下面的工作就是由此展开的.
解不妨设 $a_1<a_2<\cdots<a_n$.
先证明对任意自然数 $k \leqslant n$, 都有
$$
\sum_{i=1}^k a_k \geqslant 2^k-1
$$
用反证法.
若 $\sum_{i=1}^k a_k<2^k-1$, 则 $\left\{a_1, a_2, \cdots, a_k\right\}$ 的每个非空子集的元素和不超过 $2^k-2$. 但 $\left\{a_1, a_2, \cdots, a_k\right\}$ 有 $2^k-1$ 个非空子集, 根据抽屉原则, 必有两个非空子集的元素和相等, 这与题设矛盾.
故所证结论(1)成立.
接着证明:
$$
\frac{1}{a_1}+\frac{1}{a_2}+\cdots+\frac{1}{a_n} \leqslant 1+\frac{1}{2}+\cdots+\frac{1}{2^{n-1}}=2-\frac{1}{2^{n-1}} .
$$
事实上,
$$
\begin{gathered}
1+\frac{1}{2}+\cdots+\frac{1}{2^{n-1}}-\left(\frac{1}{a_1}+\frac{1}{a_2}+\cdots+\frac{1}{a_n}\right) \\
=\frac{a_1-1}{a_1}+\frac{a_2-2}{2 a_2}+\cdots+\frac{a_n-2^{n-1}}{2^{n-1} a_n} . \\
\text { 令 } C_i=\frac{1}{2^{i-1} a_i}, d_i=a_i-2^{i-1}, D_k=\sum_{i=1}^k d_i \text { 显然 } C_1>C_2>\cdots>C_n, \text { 且 } \\
D_k=\sum_{i=1}^k a_i-\left(1+2+\cdots+2^{k-1}\right)=\sum_{i=1}^k a_i-\left(2^k-1\right) \geqslant 0 .
\end{gathered}
$$
于是我们有
$$
\begin{aligned}
& 1+\frac{1}{2}+\cdots+\frac{1}{2^{n-1}}-\left(\frac{1}{a_1}+\frac{1}{a_2}+\cdots+\frac{1}{a_n}\right) \\
= & \sum_{i=1}^n C_i d_1 \\
= & C_1 D_1+C_2\left(D_2-D_1\right)+\cdots+C_n\left(D_n-D_{n-1}\right) \\
= & \left(C_1-C_2\right) D_1+\left(C_2-C_3\right) D_2+\cdots+\left(C_{n-1}-C_n\right) D_{n-1}+C_n D_n \\
\geqslant & 0,
\end{aligned}
$$
故(2)式得证.
注意到, 当 $S=\left\{1,2,2^2, \cdots, 2^{n-1}\right\}$ 时, 题设条件成立.
此时有
$$
\frac{1}{a_1}+\frac{1}{a_2}+\cdots+\frac{1}{a_n}=1+\frac{1}{2}+\cdots+\frac{1}{2^{n-1}}=2-\frac{1}{2^{n-1}} \text {. }
$$
因此, 所求的最大值是 $2-\frac{1}{2^{n-1}}$.
%%PROBLEM_END%%



%%PROBLEM_BEGIN%%
%%<PROBLEM>%%
例4. 已知集合 $A=\{1,2, \cdots, 10\}$. 求集合 $A$ 的具有下列性质的子集个数: 每个子集至少含有 2 个元素, 且每个子集中任何两个元素的差的绝对值大于 1 .
%%<SOLUTION>%%
分析:集合 $A$ 有 $2^{10}-1$ 个非空子集,逐一考察的工作只有交给计算机.
像例 1 一样, 我们先来看看比 $A$ 的元素少一些的集合的情形.
记集合 $A_i$ 符合条件的子集族为 $A_i^*,\left|A_i^*\right|=a_i$.
$$
\begin{aligned}
A_1 & =\{1\}, A_1^*=\varnothing, a_1=0 ; \\
A_2 & =\{1,2\}, A_2^*=\varnothing, a_2=0 ; \\
A_3 & =\{1,2,3\}, A_3^*=\{\{1,3\}\}, a_3=1 ; \\
A_4 & =\{1,2,3,4\}, A_4^*=\{\{1,3\},\{1,4\},\{2,4\}\}, a_4=3 ; \\
A_5 & =\{1,2,3,4,5\}, A_5^*=\{\{1,3\},\{1,4\},\{2,4\},\{1,3,5\}, \{1,5\},\{2,5\},\{3,5\}\}, a_5=7 . &
\end{aligned}
$$
我们来考察写出 $A_5^*$ 的过程, 这可以分作两步: 第一步写出 $A_4^*$ 的全部元素, 它们都不含元素 5 ; 第二步写出含 5 的子集, 它们是在 $A_3^*$ 的元素中添 5 所成, 或者是含 5 的二元子集, 即 $a_5=a_4+a_3+3$. 其实对 $A_4^* 、 A_3^*$ 有类似的结论: $a_4=a_3+a_2+2, a_3=a_2+a_1+1$. 我们可以将这个作法推广到一般。
解设 $a_n$ 是集合 $\{1,2, \cdots, n\}$ 的具有题设性质的子集个数.
对于集合 $\{1,2, \cdots, n, n+1, n+2\}$, 具有题设性质的子集可分为两类: 第一类子集不包含 $n+2$, 它们是集合 $\{1,2, \cdots, n, n+1\}$ 的全部具有题设性质的子集, 共有 $a_{n-1}$ 个; 第二类子集包含 $n+2$, 它们是集合 $\{1,2, \cdots, n\}$ 的每个具有题设性质的子集与 $\{n+2\}$ 的并集, 以及二元子集 $\{1, n+2\},\{2$, $n+2\}, \cdots,\{n, n+2\}$, 共有 $a_n+n$ 个.
于是, 我们有
$$
a_{n+2}=a_{n+1}+a_n+n .
$$
易知, $a_1=a_2=0$, 因此 $a_3=1, a_4=3, a_5=7, a_6=14, a_7=26$, $a_8=46, a_9=79, a_{10}=133$.
所以,所求子集的个数为 133 .
说明上述解法的特点是将问题一般化,一般问题解决了, 特殊问题当然就解决了.
这里用到了递推方法, 递推也是解决组合问题的常用方法之一.
与上例相反,我们来看一个已知子集族求恰好包含这些子集的集合的阶的问题.
%%PROBLEM_END%%



%%PROBLEM_BEGIN%%
%%<PROBLEM>%%
例5. 对于整数 $n(n \geqslant 2)$, 如果存在集合 $\{1,2, \cdots, n\}$ 的子集族 $A_1$, $A_2, \cdots, A_n$ 满足;
(a) $i \notin A_i, i=1,2, \cdots, n$;
(b) 若 $i \neq j, i, j \in\{1,2, \cdots, n\}$, 则 $i \in A_j$, 当且仅当 $j \notin A_i$;
(c) 任意 $i, j \in\{1,2, \cdots, n\}, A_i \cap A_j \neq \varnothing$.
则称 $n$ 是 “好数”.
证明: (1) 7 是好数;
(2)当且仅当 $n \geqslant 7$ 时, $n$ 是好数.
%%<SOLUTION>%%
分析:对于 $n=7$, 可以作出满足条件的子集族来验证; 当 $n \geqslant 7$ 时, 可考虑用数学归纳法证明.
证明 (1) 当 $n=7$ 时, 取
$$
\begin{aligned}
& A_1=\{2,3,4\}, A_2=\{3,5,6\}, A_3=\{4,5,7\}, \\
& A_4=\{2,6,7\}, A_5=\{1,4,6\}, A_6=\{1,3,7\}, \\
& A_7=\{1,2,5\}
\end{aligned}
$$
即可.
(2) 先证当 $n \geqslant 7$ 时, $n$ 是好数.
对 $n$ 进行归纳.
由 (1) 知, 当 $n=7$ 时, 结论成立.
假设 $n(n \geqslant 7)$ 是好数, 则存在子集族 $A_1, A_2, \cdots, A_n$ 满足条件.
对于 $n+$ 1 , 取子集族 $B_1=A_1, B_2=A_2, \cdots, B_n=A_n, B_{n+1}=\{1,2, \cdots, n\}$. 由归纳假设易知, 它们也是满足条件的.
下面证明每一个好数 $n$ 都至少为 7 .
如果 $A_1, A_2, \cdots, A_n$ 是一个 $n$ 为好数的集合的子集族,那么, 每一个 $A_i$ 至少有三个元素.
事实上,若 $A_i \subset\{j, k\}$, 则
$$
A_i \cap A_j=\{k\}, A_i \cap A_k=\{j\} .
$$
所以, $k \in A_j, j \in A_k$. 矛盾.
考虑一个由元素 $0 、 1$ 构成的 $n \times n$ 阶正方形表格,当且仅当 $j \in A_i$ 其第 $i$ 行第 $j$ 列的元素为 1 . 表中对角线上的元素为 0 , 对于余下的元素,因为 $i \neq j$, 当且仅当 $a_{j i}=1$ 时 $a_{i j}=0$, 所以 0 的个数等于 1 的个数.
因此, 表中元素的和为 $\frac{n^2-n}{2}$. 又每行元素的和大于等于 3 , 所以 $n^2-n \geqslant 6 n$, 故 $n \geqslant 7$.
%%PROBLEM_END%%



%%PROBLEM_BEGIN%%
%%<PROBLEM>%%
例6. 集合 $X=\{1,2, \cdots, 6 k\}, k \in \mathbf{N}^*$. 试作出 $X$ 的三元子集族 $\mathscr{A}$, 满足:
(1) $X$ 的任一二元子集至少被族 $\mathscr{A}$ 中的一个三元子集包含;
(2) $|\mathscr{A}|=6 k^2$.
%%<SOLUTION>%%
解:先证明下面的引理:
引理对 $n \in \mathbf{N}^*$, 集合 $X_1=\{1,2, \cdots, 2 n\}$ 的全部二元子集可分成 $2 n-1$ 组, 且每组是 $X_1$ 的一个分划.
引理的证明: 如图(<FilePath:./figures/fig-c5e6.png>),将 $1,2, \cdots, 2 n-1$ 这 $2 n-1$ 个数按顺时针方向放到一个正 $2 n-1$ 边形的顶点上,数 $2 n$ 放在外接圆圆心.
连结 $2 n$ 与 1 , 作 $n-1$ 条以 $2 n-1$ 边形顶点为端点且垂直于 1 与 $2 n$ 连线的线段,便得到 $X_1$ 的 $n$ 个二元子集构成 $X_1$ 的一个分划.
将 $2 n$ 与 1 的连线依次顺时针旋转 $\frac{2 \pi}{2 n-1}, \frac{4 \pi}{2 n-1}, \cdots, \frac{(4 n-4) \pi}{2 n-1}$, 作出相应的图及
$X_1$ 的 $n$ 个二元子集.
这样, $X_1$ 的全部 $\mathrm{C}_{2 n}^2=n(2 n-1)$ 个二元子集被分成 $2 n-1$ 组, 且每组 $n$ 个集合构成 $X_1$ 的一个分划.
下面来作满足题设的子集族:
$$
\text { 令 } A=\{1,2, \cdots, 2 k\}, B=\{2 k+1,2 k+2, \cdots, 4 k\}, C=\{4 k+1 ,4 k+2, \cdots, 6 k\}
$$. 
由引理, $A$ 的全部二元子集可分成 $2 k-1$ 组, 每组是 $A$ 的一个分划.
将其中一组重复一次, 得到 $A$ 的 $2 k$ 个分划, 让其中每个分划与 $B$ 的一个元素搭配作出 $k$ 个 $X$ 的三元子集.
类似地,作出 $B$ 的 $2 k$ 个二元子集构成的分划, 包含 $B$ 的全部二元子集, 让其中每个分划与 $C$ 的一个元素搭配作出 $k$ 个 $X$ 的三元子集; 作出 $C$ 的 $2 k$ 个二元子集构成的分划, 包含 $C$ 的全部二元子集, 让其中每个分划与 $A$ 的一个元素搭配作出 $k$ 个 $X$ 的三元子集.
上面得到的 $k \times 2 k \times 3=6 k^2$ 个 $X$ 的三元子集组成的族 $\mathscr{A}$ 满足题设要求.
说明 $X$ 的二元子集有 $\mathrm{C}_{6 k}^2=3 k(6 k-1)=18 k^2-3 k$ 个.
而所作的三元子集族中的每个集合 (子集族的元素) 都包含 3 个二元子集, 子集族共可生成二元子集 $3 \times 6 k^2=18 k^2$ 个.
这说明有 $3 k$ 个(次)二元子集在子集族中被重复生成.
那么, 满足条件 (1) 的 $|\mathcal{A}|$ 的最小值是 $6 k^2$ 吗?
三、有关子集族的最值问题有关集合子集族的最值主要有三类:(1)求子集族阶的最值; (2) 求子集族中的集合阶的最值; (3) 求符合特定条件的集合元素的最值.
%%PROBLEM_END%%



%%PROBLEM_BEGIN%%
%%<PROBLEM>%%
例7. 集合 $A=\{0,1,2, \cdots, 9\},\left\{B_1, B_2, \cdots, B_k\right\}$ 是 $A$ 的一族非空子集, 当 $i \neq j$ 时, $B_i \cap B_j$ 至多有两个元素.
求 $k$ 的最大值.
%%<SOLUTION>%%
分析:集合 $A$ 的一元、二元、三元子集显然符合要求.
而 $A$ 的任一多于三元的子集 $B^{\prime}$ 必包含了.
$A$ 的三元子集, 故 $B^{\prime}$ 与其包含的三元子集不能同在题中的子集族内.
解首先至多含 3 个元素的 $A$ 的非空子集有
$$
\mathrm{C}_{10}^1+\mathrm{C}_{10}^2+\mathrm{C}_{10}^3=10+\frac{10 \times 9}{2}+\frac{10 \times 9 \times 8}{6}=175 \text { (个). }
$$
这些集合的交集至多有两个元素, 否则两集合相等, 矛盾.
因此 $k_{\max } \geqslant 175$.
下面证明 $k_{\max } \leqslant 175$.
设 $\mathscr{b}$ 为满足题设的子集族.
若 $B \in \mathscr{C}$, 且 $|B| \geqslant 4$, 设 $b \in B$, 则 $B$ 与 $B- \{b\}$ 不能同时含于 $\mathscr{C}$, 以 $B-\{b\}$ 代 $B$, 则 $\mathscr{C}$ 中元素数目不变.
仿此对 $\mathscr{C}$ 中所有元素数目多于 4 的集合 $B$ 作相应替代, 替代后子集族 $\mathscr{C}$ 中的每个集合都是元素数目不多于 3 的非空集合.
故 $k_{\max } \leqslant 175$.
所以, $k$ 的最大值为 175 .
说明上述解答采用了“两边夹”的策略: 先得出 $k$ 的最大值不小于 175 , 然后指出 $k$ 不大于 175 , 从而得出 $k_{\max }=175$.
%%PROBLEM_END%%



%%PROBLEM_BEGIN%%
%%<PROBLEM>%%
例8. 设 $A \subseteq\{0,1,2, \cdots, 29\}$ 满足:对任何整数 $k$ 及 $A$ 中任意数 $a 、 b$ ( $a 、 b$ 可以相同), $a+b+30 k$ 均不是两个相邻整数之积.
试定出所含元素个数最多的 $A$.
%%<SOLUTION>%%
分析:因为当 $b=a$ 时, $2 a+30 k$ 均不是两个相邻整数之积, 故我们只需考察 $2 a$ 被 30 除的余数.
解所求 $A$ 为 $\{3 l+2 \mid 0 \leqslant l \leqslant 9\}$.
设 $A$ 满足题中条件且 $|A|$ 最大.
因为两个相邻整数之积被 30 除, 余数为 $0,2,6,12,20,26$. 则对任一 $a \in A$, 有 $2 a \neq 0,2,6,12,20,26(\bmod 30)$ ,
即 $a \neq 0,1,3,6,10,13,15,16,18,21,25,28$, 因此, $A \subseteq\{2,4,5,7,8,9,11,12,14,17,19,20,22,23,24,26,27,29\}$, 后一集合可分拆成下列 10 个子集的并, 其中每一个子集至多有一个元素包含在 $A$ 中: $\{2,4\}, \{5,7\},\{8,12\},\{9,11\},\{14,22\},\{17,19\},\{20\},\{23,27\},\{24,26\},\{29\}$, 故 $|A| \leqslant 10$.
若 $|A|=10$, 则每个子集恰好有一个元素包含在 $A$ 中, 因此, $20 \in A, 29 \in A$.
由 $20 \in A$ 知 $12 \notin A$, 从而 $8 \in A$, 这样 $4 \notin A, 22 \notin A, 24 \notin A$. 因此 $2 \in A, 14 \in A, 26 \in A$.
由 $29 \in A$ 知 $7 \notin A, 27 \notin A$, 从而 $5 \in A, 23 \in A$, 这样 $9 \notin A, 19 \notin A$, 因此 $11 \in A, 17 \in A$.
综上所述,有 $A=\{2,5,8,11,14,17,20,23,26,29\}$, 此集合 $A$ 确实满足要求.
%%PROBLEM_END%%



%%PROBLEM_BEGIN%%
%%<PROBLEM>%%
例9. 设 $n$ 为正整数, 在数集
$$
\{-n,-n+1,-n+2, \cdots,-1,0,1, \cdots, n-1, n\}
$$
中最多选取多少个数, 可使任意三个数的和均不为 0 (三个数可以相同)?
%%<SOLUTION>%%
分析:显然, 当选取的数的绝对值充分大时, 可使任意三个数的和均不为 0 .
解设从题中数集中最多选取 $k$ 个数, 可使任意三个数的和均不为 0 . 考察子集
$$
\left\{-n, \cdots,-\left[\frac{n}{2}\right]-1,\left[\frac{n}{2}\right]+1, \cdots, n\right\},
$$
其中 $[x]$ 表示不超 $x$ 的最大整数.
知当 $n$ 为偶数时, $k \geqslant n$; 当 $n$ 为奇数时, $k \geqslant n+1$.
设 $A=\left\{a_1, a_2, \cdots, a_m\right\}, B=\left\{b_1, b_2, \cdots, b_l\right\}$ 都是元素为整数的非空集合.
定义集合
$$
A+B=\{a+b \mid a \in A, b \in B\},
$$
可以证明 $A+B$ 至少有 $m+l-1$ 个元素.
事实上, 不妨设 $a_1<a_2<\cdots<a_n, b_1<b_2<\cdots<b_l$, 则
$$
a_1+b_1, a_1+b_2, \cdots, a_1+b_l, a_2+b_l, \cdots, a_m+b_l
$$
是一个有 $m+l-1$ 项的严格递增的数列, 其中每一个数都是集合 $A+B$ 的元素.
假设 $S$ 是一个满足题设的子集.
显然 $0 \notin S$. 取
$$
\begin{aligned}
& A=S \cap\{-n,-n+1, \cdots,-1\}, \\
& B=S \cap\{1,2, \cdots, n\} .
\end{aligned}
$$
于是, $A+B$ 和 $-S=\{-s \mid s \in S\}$ 是集合 $\{-n,-n+1, \cdots, n\}$ 的两个不相交的子集.
由前证知
$$
\begin{aligned}
2 n+1 & \geqslant|A+B|+|-S| \\
& \geqslant|A|+|B|-1+|S| \\
& =2|S|-1,
\end{aligned}
$$
即 $|S| \leqslant n+1$.
当 $n$ 为奇数时, 就证明了 $k=n+1$.
当 $n$ 为偶数时, 还需要证明 $|S|=n+1$ 是不可能的.
由于 $A+B \subseteq\{-n+1,-n+2, \cdots, n-1\}$, 若有
$$
|A+B|+|-S|=2 n+1 \text {, }
$$
则必有 $-n, n \in-S$, 即 $-n, n \in S$. 于是
$$
\{1, n-1\}, \cdots,\left\{\frac{n}{2}-1, \frac{n}{2}+1\right\},\left\{\frac{n}{2}\right\}
$$
每个集合中至多有一个元素在 $B$ 中.
因此,
$$
|B| \leqslant \frac{n}{2} .
$$
同理,
$$
|A| \leqslant \frac{n}{2} .
$$
由 $A 、 B$ 的定义, 知
$$
|S|=|A|+|B| \leqslant n .
$$
与 $|S|=n+1$ 矛盾.
因此,当 $n$ 为偶数时, $k=n$.
%%PROBLEM_END%%



%%PROBLEM_BEGIN%%
%%<PROBLEM>%%
例10. 集合 $A=\{1,2, \cdots, 1997\}$, 对 $A$ 的任意一个 999 元子集 $X$, 若存在 $x, y \in X$, 使得 $x<y$ 且 $x \mid y$, 则称 $X$ 集为好集.
求最大自然数 $a(a \in A)$, 使任一含有 $a$ 的 999 元子集都为好集.
%%<SOLUTION>%%
分析:抓住 $A$ 的 999 元子集 $X_0=\{999,1000, \cdots, 1997\}$ 是关键.
因为 $999 \times 2=1998>1997$, 所以 $a<999$. 考虑集合 $A$ 的这样的元素 $b: 2 b \in X_0$, $3 b \notin X_0$. 易知 $b=666+i, i=0,1, \cdots, 332$. 由 $B_i=\{666+i\} \cup X_0 \backslash \{2(666+i)\}, i=0,1, \cdots, 332,\left|B_i\right|=999$, 知 $a \leqslant 665$.
解我们证明 $\max a=665$.
先证 $a \leqslant 665$. 显然 $A$ 的 999 元子集 $X_0=\{999,1000,1001, \cdots, 1997\}$ 中不存在 $x, y \in X_0$, 使得 $x<y$ 且 $x \mid y$. 事实上, $X_0$ 的最小元素为 999 , 它的最小倍数除本身外为 $2 \times 999=1998>1997$, 即比 $X_0$ 的最大元素还大.
这样, $a$ 就不能为 $999,1000,1001, \cdots, 1997$ 中的任一个数.
构造集合
$$
B_i=\{666+i\} \bigcup X_0 \backslash\{2(666+i)\}, i=0,1, \cdots, 332 .
$$
对 $B_i$ 来说, $(666+i) \times 3 \geqslant 1998$, 而 $(666+i) \times 2 \notin B_i$, 故 $666+i$ 除本身外其他倍数都不在 $B_i$ 中.
上面已证 $X_0$ 的任一非本身的倍数都不在 $X_0$ 中; 而 $666+i<999(i=0,1,2, \cdots, 332)$, 故 $X_0$ 中任 $\cdots$ 元素的倍数不可能为 $666+i(i=0,1, \cdots, 332)$. 这样 $B_i$ 中仍不存在两元素满足 $x<y$ 且 $x \mid y$. 而 $B_i$ 中 $(i=0,1, \cdots, 332)$ 包含了 $666,667, \cdots, 998$, 故 $a \neq 666,667, \cdots, 998$. 所以 $a \leqslant 665$.
下证 665 是可取的.
反设存在一个含 665 的 999 元子集 $X$,不存在这样的 $x, y \in X, x<y$ 使 $x \mid y$, 则 $665 \times 2 、 665 \times 3 \notin X$.
构造如下 997 个抽庶, 它包含了 $A$ 中除 $665 、 665 \times 2 、 665 \times 3$ 外的所有元素,且每个元素只出现一次
$$
\begin{aligned}
& \left\{1,1 \times 2,1 \times 2^2, \cdots, 1 \times 2^{10}\right\}, \\
& \left\{3,3 \times 2,3 \times 2^2, \cdots, 3 \times 2^9\right\}, \\
& \left\{5,5 \times 2,5 \times 2^2, \cdots, 5 \times 2^8\right\}, \\
& \cdots \ldots \\
& \{663,663 \times 2,663 \times 3\}, \\
& \{667,667 \times 2\}, \\
& \{669,669 \times 2\}, \\
& \cdots \ldots . \\
& \{1991\},\{1993\},\{1997\} .
\end{aligned}
$$
$X$ 中除 665 外的其他 998 个元素归人这 997 个抽屉里, 定有两个在同一抽屉, 而同一抽屉里的数互为倍数关系, 矛盾.
证毕.
%%PROBLEM_END%%


