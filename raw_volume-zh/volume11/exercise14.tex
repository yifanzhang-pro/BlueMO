
%%PROBLEM_BEGIN%%
%%<PROBLEM>%%
问题1. 求一切正整数 $n \geqslant 5$, 使存在一种染色方法至多用 6 种颜色给 $n$ 边形的顶点染色 (每个顶点染一种颜色), 满足任意连续 5 个顶点互不同色.
%%<SOLUTION>%%
设 6 种颜色是 $a, b, c, d, e, f$, 并记 $S_1=(a, b, c, d, e), S_2= (a, b, c, d, e, f)$ 为两个序列.
若 $n$ 可表示成为 $5 x+6 y(x, y$ 为非负整数) 的形式, 且 $n \geqslant 5$, 则我们将 $n$ 边形的 $n$ 个顶点按 $x$ 个 $S_1$ 序列, $y$ 个 $S_2$ 序列着色, 就满足了题目要求.
令 $y=0,1,2,3,4$ 就得到 $n$ 可以等于形如 $5 x, 5 x+ 6,5 x+12,5 x+18,5 x+24$ ( $x$ 为非负整数) 的正整数.
而大于 4 不具有上述形式的数只有 $7,8,9,13,14$ 和 19 等 6 个数.
下面我们证明这 6 个数都不满足要求.
假设 $n$ 边形存在满足题目要求的染色方法, 那么存在正整数 $k$ 使 $6 k<n \leqslant 6(k+1)$. 由抽屉原理, $n$ 边形至少有 $\left[\frac{n-1}{6}\right]+1=k+1$ 个顶点同色, 这些同色顶点中每两个顶点之间至少有其他 4 个顶点 (因连续 5 个顶点不同色)故至少有 $5 k+5$ 个顶点, 所以 $n \geqslant 5 k+5$, 然而这个不等式与不等式 $6 k<n \leqslant 6(k+1)$ 不可能对 $n=7,8,9,13,14$ 和 19 中任何一个同时成立, 故所求 $n$ 为不小于 5 且不等于 $7,8,9,13,14,19$ 中任何一个的任意正整数.
%%PROBLEM_END%%



%%PROBLEM_BEGIN%%
%%<PROBLEM>%%
问题2. $100 \times 100$ 方格表中每一个小方格被染成 4 种颜色之一,使得每行每列恰有每种颜色的小方格各 25 个.
证明: 可以在表中找到 2 行和 2 列, 它们所交成的 4 个小方格分别染成了 4 种颜色.
%%<SOLUTION>%%
用 $A 、 B 、 C 、 D$ 表示 4 种颜色, 我们称位于同一行且异色的一对小方格为 "异色对", 则每行有 $C_4^2 \cdot 25^2=6 \cdot 25^2$ 个"异色对", 于是各行一共有 $100 \cdot 6 \cdot 25^2$ 个"异色对". 另一方面,每行中每对"异色对"位于一对不同的列中, 因共有 $\mathrm{C}_{100}^2=99 \cdot 50$ 对不同的列, 故存在两列, 其中至少有 $\left[\frac{100 \cdot 6 \cdot 25^2-1}{99 \cdot 50}\right]+1=76$ 对"异色格".下面我们不问其他的行和列, 总可以假定在这两列 76 对异色格组成的 $76 \times 2$ 的长方形表中每小格被染成 4 色之一, 每列中每种颜色小方格至多 25 个, 且同一行的两个小方格不同色.
若 $\{A, B\},\{C, D\}$ 每对颜色分别出现在两行中, 则结论成立.
类似地, 若 $\{A$, $C\},\{B, D\}$ 或 $\{A, D\},\{B, C\}$ 分别出现在 2 行中结论也成立.
因此我们假设这三组中, 每组至多只有一对出现在某行中, 不妨设含颜色 $A$ 的"异色对"出现得最多.
这本质上只有两种情形(1) $\{A, B\},\{A, C\},\{A, D\}$; (2) $\{A, B\}$, $\{A, C\},\{B, C\}$. 在情形(1)下, 每行都有一个 $A$ 色小方格, 于是必有一列中至少在 $\left[\frac{76-1}{2}\right]+1=38$ 个 $A$ 色方格, 这与每列至多 25 个 $A$ 色方格矛盾.
在情形(2)下,每列至多含 $A 、 B 、 C$ 这 3 种颜色,故至少有一列有 $\left[\frac{76-1}{3}\right]+1=26$ 个方格同色,矛盾.
于是, 命题得证.
%%PROBLEM_END%%



%%PROBLEM_BEGIN%%
%%<PROBLEM>%%
问题3. 某国有1001个城市,每两个城市之间都有单向行车的道路相连,每个城市刚好有 500 条出城道路,也都刚好有 500 条人城的道路.
由该国分裂出一个独立国家,它拥有 668 个城市.
证明: 由这个新独立国家的每个城市都可以到达它的其他任何一个城市, 而无需越出自己的边界.
%%<SOLUTION>%%
假设题中断言不成立.
例如, 由该新独立国家的城市 $X$ 不能通过该国内部的通道到达另一个城市 $Y$. 我们将由城市 $X$ 可以沿该国内通道到达的该国的所有城市的集合记作 $A$ (其中包括城市 $X$ 本身),而该国其余城市的集合记作 $B$ (显然有 $Y \in B \neq \varnothing$ ). 不难看出, 连接这两个集合中城市的所有道路都是由 $B$ 中城市驶往 $A$ 中城市的 (否则存在一条从 $X$ 到达 $A$ 中某城市 $X_1$ 再到 $B$ 中某城市 $Y_1$ 的单向通道, 这与从 $X$ 不能到达 $B$ 中城市的假设矛盾). 分别记 $A, B$ 中城市的个数为 $a$ 和 $b$. 于是 $a+b=668$, 若 $a \geqslant b$, 则 $a \geqslant 334 \geqslant b$. 因 $B$ 中城市间共有 $\mathrm{C}_b^2$ 条单向通道, 所以 $B$ 中存在一个城市 $Z$, 由它出发至少有 $\frac{1}{b} \mathrm{C}_b^2=\frac{1}{2}(b-1)$ 条单向道路通向 $B$ 中其他城市, 并且由 $Z$ 出发有 $a$ 条道路通向 $A$ 中城市.
这样一来 $Z$ 的出城道路不少于 $a+\frac{b-1}{2}= \frac{a+(a+b)-1}{2} \geqslant \frac{334+668-1}{2}>500$, 导致与已知条件矛盾.
若 $a<b$, 则 $a<334<b$. 因 $A$ 中城市间共有 $\mathrm{C}_a^2$ 条单向人城通道, 故必存在 $A$ 中一座城市 $W$, 它的人城通道至少有 $\frac{1}{a} \mathrm{C}_a^2=\frac{1}{2}(a-1)$ 条.
又从 $B$ 中城市进人 $W$ 的人城通道有 $b$ 条.
故 $W$ 的人城通道至少有 $\frac{1}{2}(a-1)+b=\frac{b+(a+b)-1}{2}> \frac{334+668-1}{2}=500$, 也导致与已知条件矛盾,故题中结论成立.
%%PROBLEM_END%%



%%PROBLEM_BEGIN%%
%%<PROBLEM>%%
问题4. 在一所学校中有 $n$ 名男生和 $n$ 名女生 $(n>2000)$. 规定每名学生参加社团的数目不能超过 100 个.
已知任意两名异性学生至少参加了一个共同的社团, 证明: 存在一个社团至少有 11 名男生和 11 名女生.
%%<SOLUTION>%%
假设每个社团要么最多有 10 名男生, 要么最多有 10 名女生.
下面用算二次方法导出矛盾.
设 $m$ 是形如 $(b, g, c)$ 的三元组的数目, 其中 $b$ 表示一名男生, $g$ 表示一名女生, $c$ 表示一个他们都参加的社团.
由于任意两名异性学生至少参加了一个共同的社团, 故 $m \geqslant n^2$. 另一方面, 设 $X$ 是最多有 10 名男生的社团的集合, $Y$ 是最少有 11 名男生的社团的集合(这样的社团最多有 10 名女生). 于是, 对 $c \in X$, 三元组 $(b, g, c)$ 的数目不超过 $n \times 10 \times 100= 1000 n$, 其中, 对于 $b$ 最多有 10 种选择, 对于 $g$ 有 $n$ 种选择, 对于 $c$ 最多有 100 种选择 (每名学生最多参加 100 个社团). 用同样的方法可得对于 $c \in Y$, 三元组 $(b, g, c)$ 至多有 $1000 n$ 个, 故 $m \leqslant 1000 n+1000 n=2000 n$, 从而 $n^2 \leqslant 2000 n$, $n \leqslant 2000$, 这与已知条件 $n>2000$ 矛盾, 这就证明了题中结论成立.
%%PROBLEM_END%%



%%PROBLEM_BEGIN%%
%%<PROBLEM>%%
问题5. 一次高难度数学竞赛试题由初试、复试两部分组成,共 28 个题目,每名竞赛者恰好解出其中 7 道题目, 每对试题恰有两人解出.
证明: 必有一名参赛者,他至少解出了 4 道初试题或没有解出初试题.
%%<SOLUTION>%%
设共有 $n$ 名参赛者, $m$ 道初试题, 将每个人与他解出的二道题组成一个 "三元组", 这种三元组的集合为 $S$, 则由已知条件可得 $|S|=n \cdot \mathrm{C}_7^2=2$ ・ $\mathrm{C}_{28}^2$, 于是 $n=36$. 
其次任取一道题目 $A$, 假设它被 $r$ 个人 $a_1, a_2, \cdots, a_r$ 解出这 $r$ 个人每人还解出了其他 6 道试题, 于是 $S$ 中包含 $A$ 的三元组有 $6 r$ 个.
另一方面将 $A$ 与其他 27 题中每个题配对, 每对题恰有 2 人解出 (因这两人解出了题目 $A$,故他们必是 $a_1, a_2, \cdots, a_r$ 中两人) 从而可形成 $2 \times 27$ 个含 $A$ 的三元组, 所以 $6 r=2 \times 27$, 即 $r=9$, 也就是说每道题恰有 9 人解出.
如果结论不成立, 那么, 每人解出的初试题目只可能为 $1,2,3$. 设解出 $1,2,3$ 道初试题的人数分别为 $x, y, z$, 则 $x+y+z=36 \cdots$ (1). 
将每个人与他解出的一道初试题配对, 这种对子个数为 $x+2 y+3 z$, 又为 $9 m$ (因每道题恰有 9 人解出), 所以 $x+2 y+3 z=9 m \cdots$ (2). 
又通过计算 $S$ 中恰含 2 道初试题的 "三元组"可得 $\mathrm{C}_2^2 y+ \mathrm{C}_3^2 z=2 \mathrm{C}_m^2$, 即 $y+3 z=m(m-1)$ (3). 
由(1), (2), (3)可解出 $x=m^2-19 m+$ 108, $y=-2 m^2+29 m-108, z=m^2-10 m+36$, 其中 $y=-2\left(m-\frac{29}{4}\right)^2- \frac{23}{8}<0$, 这与 $y$ 为非负整数矛盾.
于是题目结论成立.
%%PROBLEM_END%%



%%PROBLEM_BEGIN%%
%%<PROBLEM>%%
问题6. 设空间给定 $n$ 个点,其中任何 4 点不共面, 它们之间连有 $q$ 条线段,则这些线段至少构成 $\frac{4 q}{3 n}\left(q-\frac{n^2}{4}\right)$ 个不同的三角形.
%%<SOLUTION>%%
设图中 $n$ 个点为 $A_1, A_2, \cdots, A_n$, 若 $A_i$ 与 $A_j$ 连有线段, 则将该线段染红色, 若 $A_i$ 与 $A_j$ 没有连线, 则将 $A_i$ 与 $A_j$ 连一条蓝色线段, 得到一个 2 色完全图 $K_n$. 由已知条件知图 $K_n$ 有 $q$ 条红边.
要证明图中至少有 $\frac{4 q}{3 n}\left(q-\frac{n^2}{4}\right)$ 个红色三角形 (三边为红色的三角形). 
设从 $A_i$ 出发有 $d_i$ 条红线, $n-1-d_i$ 条蓝线 $(i=1,2, \cdots, n)$ 于是 $\sum_{i=1}^n d_i=2 q \cdots$ (1). 
设图中有 $\alpha$ 个红色三角形, $\beta$ 个两边红一边蓝的三角形, $\gamma$ 个二边蓝一边红的三角形, 并称从一点出发的两条红线组成的角叫做红色角, 从一点出发的一条红线和一条蓝线组成的角叫做异色角, 于是红色角的个数为 $3 \alpha+\beta=\sum_{i=1}^n \mathrm{C}_{d_i}^2 \cdots$ (2), 
异色角的个数为 $2(\beta+\gamma)=\sum_{i=1}^n d_i\left(n-1-d_i\right) \cdots$ (3). 
由(3)得 $\beta \leqslant \beta+\gamma=\frac{1}{2} \sum_{i=1}^n d_i(n- 1-d_i$ ), 
代入 (2) 并利用柯西不等式及 (1) 得 $\alpha=\frac{1}{3}\left[\sum_{i=1}^n \mathrm{C}_{d_i}^2-\beta\right] \geqslant \frac{1}{3}\left[\sum_{i=1}^n \frac{1}{2} d_i\left(d_i-1\right)-\frac{1}{2} \sum_{i=1}^n d_i\left(n-1-d_i\right)\right]=\frac{1}{3}\left(\sum_{i=1}^n d_i^2-\frac{n}{2} \sum_{i=1}^n d_i\right) \geqslant \frac{1}{3}\left[\frac{1}{n}\left(\sum_{i=1}^n d_i\right)^2-\frac{n}{2} \sum_{i=1}^n d_i\right]=\frac{1}{3}\left[\frac{1}{n}(2 q)^2-\frac{n}{2}(2 q)\right]=\frac{4 q}{3 n}\left(q-\frac{n^2}{4}\right)$, 于是原命题得证.
%%PROBLEM_END%%



%%PROBLEM_BEGIN%%
%%<PROBLEM>%%
问题7. 由 $n$ 个点和这些点之间的 $l$ 条线组成一个空间图形,其中 $n=q^2+q+1$, $l \geqslant \frac{1}{2} q(q+1)^2+1, q \geqslant 2, q \in \mathbf{N}_{+}$. 已知图形中任意四点不共面, 每点至少连一条线段,且存在一点至少连有 $q+2$ 条线段.
证明: 图中必存在一个空间四边形 (即由四点 $A 、 B 、 C 、 D$ 和四条线段 $A B 、 B C 、 C D 、 D A$ 组成的图形).
%%<SOLUTION>%%
设 $n$ 个点的集合为 $V=\left\{A_0, A_1, \cdots, A_{n-1}\right\}$, 记 $A_i$ 的所有邻点(即与 $A_i$ 连有线段的点) 的集合为 $B_i, B_i$ 中点的个数为 $\left|B_i\right|=b_i$. 鼠然 $\sum_{i=0}^{n-1} b_i=2 l$, 且 $b_i \leqslant n-1(i=0,1,2, \cdots, n-1)$. 
(1) 若存在 $b_i=n-1$, 不妨设 $b_0= n-1$, 于是 $B_0$ 中 $n-1$ 个点之间的连线数 $l-b_0 \geqslant \frac{1}{2} q(q+1)^2+1-(n-1)=\frac{1}{2}(q+1)(n-1)+1-(n-1) \geqslant \frac{3}{2}(n-1)+1-(n-1)=\frac{n-1}{2}+1 \geqslant\left[\frac{n-1}{2}\right]+1$. 故 $B_0$ 中必存在一点 $A_i$ 与 $B_0$ 中另两点 $A_j$ 与 $A_k$ 都连有线段.
于是存在四边形 $A_0 A_j A_i A_k$, 结论成立.
因此, 下面是讨论 $b_i \leqslant n-2(i= 0,1,2, \cdots, n-1)$ 的情形.
(2) 不妨设 $q+2 \leqslant b_0 \leqslant n-2$ 及 $b_i \leqslant n- 2(i=1,2, \cdots, n-1)$. 若图中不存在四边形, 则当 $i \neq j$ 时, $\left|B_i \cap B_j\right| \leqslant 1(0 \leqslant i<j \leqslant n-1)$ 记 $\bar{B}_0=\complement_V B_0$. 则 $\left|B_i \cap \bar{B}_0\right| \geqslant b_i-1(i=1,2, \cdots$, $n-1$ ). 并且当 $1 \leqslant i<j \leqslant n$ 时 $B_i \cap \bar{B}_0$ 与 $B_j \cap \bar{B}_0$ 没有公共的点对(否则存在四边形 $)$, 所以 $\bar{B}_0$ 中点对的个数 $\geqslant \sum_{i=1}^{n-1}\left\{\left(B_i \cap \bar{B}_0\right)\right.$ 中点对的个数 $\}$, 即 $\mathrm{C}_{n-b_0}^2 \geqslant \sum_{i=1}^{n-1} \mathrm{C}_{b_i-1}^2$ (当 $b_i=1$ 或 2 时 $\left.\mathrm{C}_{b_i-1}^2=0\right)$, 也就是 $\frac{1}{2}\left(n-b_0\right)\left(n-b_0-1\right) \geqslant$
$$
\begin{aligned}
& \frac{1}{2} \sum_{l=1}^{n-1}\left(b_i^2-3 b_i+2\right) \geqslant \frac{1}{2}\left[\frac{1}{n-1}\left(\sum_{l=1}^{n-1} b_i\right)^2-3 \sum_{l=1}^{n-1} b_i+2(n-1)\right]= \\
& \frac{1}{2}\left[\frac{1}{n-1}\left(2 l-b_0\right)^2-3\left(2 l-b_0\right)+2(n-1)\right]=\frac{1}{2(n-1)}\left(2 l-b_0-n+1\right)\left(2 l-b_0-2 n+2\right) \geqslant \frac{1}{2(n-1)}\left[(n-1)(q+1)+2-b_0-n+1\right][(n-1) \left.(q+1)+2-b_0-2 n+2\right]=\frac{1}{2(n-1)}\left(n q-q+2-b_0\right)\left(n q-q-n+3-b_0\right)
\end{aligned}
$$
, 故 $(n-1)\left(n-b_0\right)\left(n-b_0-1\right) \geqslant\left(n q-q+2-b_0\right)\left(n q-q-n+3-b_0\right)$, 即 $q(q+1)\left(n-b_0\right)\left(n-b_0-1\right) \geqslant\left(n q-q+2-b_0\right)\left(n q-q-n+3-b_0\right) \cdots \label{eq1}$, 
但 $\left(n q-q-n+3-b_0\right)-q\left(n-b_0-1\right)=(q-1) b_0-n+3 \geqslant(q-1)(q+ 2)-n+3=0 \cdots \label{eq2}$ .
$\left(n q-q+2-b_0\right)-(q+1)\left(n-b_0\right)=q b_0-q-n+2 \geqslant q(q+2)-q-n+2=1>0 \cdots \label{eq3}$ . 
因 $\left(n-b_0\right)(q+1),\left(n-b_0-1\right) q$ 皆为正数.
因式\ref{eq2},\ref{eq3}联合起来与式\ref{eq1}矛盾.
故原命题成立.
%%PROBLEM_END%%



%%PROBLEM_BEGIN%%
%%<PROBLEM>%%
问题8. 凸 $n$ 边形 $p$ 中每条边和每条对角线被染为 $n$ 种颜色中的一种.
问: 对怎样的 $n$, 存在一种染色方式, 使得对于这 $n$ 种颜色中的任何三种不同颜色, 都能找到一个三角形,其顶点为多边形 $p$ 的顶点,且它的三边被染为这三种颜色?
%%<SOLUTION>%%
当 $n(\geqslant 3)$ 为奇数时, 存在符合要求的染法; 当 $n$ 为偶数时,不存在所述的染法.
因为每 3 个顶点确定一个三角形, 一共确定 $\mathrm{C}_n^3$ 个三角形, 而 $n$ 种颜色的三三搭配也刚好有 $\mathrm{C}_n^3$ 种.
所以,本题只要证明不同的三角形对应于不同的颜色组合, 即形成一一对应.
以下将多边形的边和对角线都称为线段.
对于每一种颜色, 其他 $n-1$ 种颜色形成 $\mathrm{C}_{n-1}^2$ 种不同的搭配, 每种颜色的线段 (边或对角线) 都应出现在 $\mathrm{C}_{n-1}^2$ 个三角形中, 而每一条线段都是 $n-2$ 个三角形的公共边, 因此, 在满足要求的染法中, 每种颜色的线段都应有 $\frac{C_{n-1}^2}{n-2}=\frac{n-1}{2}$ (条).
当 $n$ 为偶数时, $\frac{n-1}{2}$ 不是整数,因此不可能存在满足条件的染法.
下面设 $n=2 m+1$ 为奇数, 我们给出一种染法, 并证明它满足题中条件.
将凸 $n$ 边形的顶点依次记为 $A_1, A_2, \cdots, A_{2 m+1}$, 对于整数 $i \notin\{1,2, \cdots, 2 m+1\}$, 若 $i \equiv j(\bmod 2 m+1)$ 且 $j \in\{1,2, \cdots, 2 m+1\}$, 则认为 $A_i$ 就是 $A_j$. 再将 $2 m+1$ 种颜色分别记为颜色 $1,2, \cdots, 2 m+1$. 
现将边 $A_i A_{i+1}$ 染颜色 $i (i=1,2,3, \cdots, 2 m+1)$, 再对每个 $i$, 将对角线 $A_{i-k} A_{i+1+k}(k=1,2, \cdots$, $m-1$ ) 也染颜色 $i$, 于是每种颜色的线段 (边或对角线) 都恰有 $m$ 条.
值得注意的是: 在规定的染色方法下, 当且仅当 $i_1+j_1 \equiv i_2+j_2(\bmod 2 m+$ 1) $\cdots \label{eq1}$  时, 线段 $A_{i_1} A_{j_1}$ 与 $A_{i_2} A_{j_2}$ 同色.
([注] 如果 $n$ 为奇数且凸 $n$ 边形为正 $n$ 边形, 那么当且仅当 $A_{i_1} A_{j_1} / / A_{i_2} A_{j_2}$ 时, $A_{i_1} A_{j_1}$ 与 $A_{i_2} A_{j_2}$ 同色.
) 因此, 对任何 $i \not j j(\bmod 2 m+1), k \not \equiv 0(\bmod 2 m+1)$, 线段 $A_i A_j$ 不与 $A_{i+k} A_{j+k}$ 同色, 即如果 $i_1-j_1 \equiv i_2-j_2(\bmod 2 m+1) \cdots \label{eq2}$ , 
那么线段 $A_{i_1} A_{j_1}$ 必不与 $A_{i_2} A_{j_2}$ 同色.
任取两个三角形: $\triangle A_{i_1} A_{j_1} A_{k_1}$ 和 $\triangle A_{i_2} A_{j_2} A_{k_2}$, 如果它们中至多只有一条对应线段同色, 当然它们不会含有相同的颜色组合.
如果它们有 2 条对应线段同色,我们证明: 它们的第 3 条线段必不同色.
为确定起见, 不妨设 $A_{i_1} A_{j_1}$ 与 $A_{i_2} A_{j_2}$ 同色.
下面分两种情形讨论: 
(1) 若 $A_{j_1} A_{k_1}$ 与 $A_{j_2} A_{k_2}$ 同色, 则由式 \ref{eq1} 知 $i_1+j_1 \equiv i_2+j_2(\bmod 2 m+1), j_1+k_1 \equiv j_2+k_2 (\bmod 2 m+1)$, 两式相减得 $i_1-k_1 \equiv i_2-k_2(\bmod 2 m+1)$, 
故由式 \ref{eq2} 知 $A_{k_1} A_{i_1}$ 与 $A_{k_2} A_{i_2}$ 不同色; 
(2) 若 $A_{i_1} A_{k_1}$ 与 $A_{i_2} A_{k_2}$ 同色, 则由式 \ref{eq1} 得 $i_1+j_1 \equiv i_2+j_2(\bmod 2 m+1), i_1+k_1 \equiv i_2+k_2(\bmod 2 m+1)$, 两式相减得 $j_1-k_1 \equiv j_2-k_2(\bmod 2 m+1)$, 
故由式 \ref{eq2} 知 $A_{j_1} A_{k_1}$ 与 $A_{j_2} A_{k_2}$ 不同色.
总之, $\triangle A_{i_1} A_{j_1} A_{k_1}$ 与 $\triangle A_{i_2} A_{j_2} A_{k_2}$ 对应不同的颜色组合.
这就证明了当且仅当 $n$ 为奇数时, 存在满足题目要求的染色方法.
%%PROBLEM_END%%


