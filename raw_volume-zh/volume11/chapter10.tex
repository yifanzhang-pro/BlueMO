
%%TEXT_BEGIN%%
一、反证法.
当我们直接证明一个命题的结论成立感到困难时, 可考虑用反证法.
即从结论的否定出发经过推理导致矛盾, 从而推出结论成立.
二、利用极端性原理利用极端原理解题就是从极端元素 (最大数或最小数, 最大距离或最小距离, 最大面积或最小面积, 获胜场次最多的队 (员) 或获胜场次最少的队 (员),等等)出发,经过推理得出结论,或从结论的否定出发,利用极端元素导致矛盾, 从而推出结论成立.
%%TEXT_END%%



%%PROBLEM_BEGIN%%
%%<PROBLEM>%%
例1. 在直角坐标平面内给定凸五边形 $A B C D E$, 它的顶点都是整点 (横坐标及纵坐标都为整数的点). 证明:其对角线相交成的凸五边形 $A_1 B_1 C_1 D_1 E_1$ (图 10-1) 的内部或周界上至少有一个整点.
%%<SOLUTION>%%
证明:为了说话简便, 将"在内部或在边界上"统称为"在……中", 用反证法.
假设结论不成立, 考虑不满足题目结论且具有最小面积 $S$ 的凸五边形 $A B C D E$
(因整点五边形的面积的 2 倍是正整数,故其中必存在面积最小的) 并称凸五边形 $A_1 B_1 C_1 D_1 E_1$ 为 $A B C D E$ 的内五边形, 不妨设 $\triangle A B C, \triangle B C D, \triangle C D E$, $\triangle D E A, \triangle E A B$ 中以 $\triangle A B C$ 的面积为最小.
为了导致矛盾, 我们只需证明以 $A B 、 B C$ 为邻边的平行四边形的另一个顶点 $O$ (必为整点) 在内五边形中, 为此, 我们先证明 $\triangle A C_1 D_1$ 中的整点除 $A$ 外, 只可能在线段 $C_1 D_1$ 上.
事实上若除了 $A$ 或 $C_1 D_1$ 上的点外, $\triangle A C_1 D_1$ 中还有一点 $K$ 为整点, 则凸五边形 $K B C D E$ 的面积小于凸五边形 $A B C D E$ 的面积, 且 $K B C D E$ 的内五边形包含在 $A_1 B_1 C_1 D_1 E_1$ 中, 故 $K B C D E$ 的内五边形中也没有整点,这与 $S$ 的最小性假设矛盾.
其次, 因 $S_{\triangle A B C} \leqslant S_{\triangle B C D}$, 故 $D$ 到 $B C$ 的距离 $\geqslant A$ 到 $B C$ 的距离, 又 $S_{\triangle A B C} \leqslant S_{\triangle A B E}$, 故 $E$ 到 $A B$ 的距离 $\geqslant C$ 到 $A B$ 的距离, 所以, 以 $A B 、 B C$ 为邻边的平行四边形的另一顶点 $O$ 必在 $\triangle A C B_1$ 内, 且由 $A, B, C$ 为整点知 $O$ 为整点, 且 $O \neq A, O \neq C$. 
若 $O$ 在 $\triangle A C_1 D_1$ 中则由前面证明知 $O$ 只能在 $C_1 D_1$ 上.
同理,若 $O$ 在 $\triangle C A_1 E_1$ 中, 则 $O$ 只能在 $A_1 E_1$ 上, 即 $O$ 只能在内五边形
$A_1 B_1 C_1 D_1 E_1$ 中, 这与 $A_1 B_1 C_1 D_1 E_1$ 中无整点的假设矛盾.
于是命题结论成立.
%%<REMARK>%%
注:用反证法证明涉及整点多边形的结论时, 常常要取满足假设条件而且具有最小面积的整点多边形作为证明的出发点.
%%PROBLEM_END%%



%%PROBLEM_BEGIN%%
%%<PROBLEM>%%
例2. 将 100 个互异实数分别放置在圆周上的不同地方.
求证:一定存在相邻的 4 个数使得两端的两数之和大于中间的两数之和.
%%<SOLUTION>%%
证明:假设结论不成立, 设圆周上 100 个实数为 $a_1, a_2, \cdots, a_{100}$ 且 $a_{n+100}=a_n$, 则
$$
a_n+a_{n+3} \leqslant a_{n+1}+a_{n+2},
$$
即
$$
a_{n+3}-a_{n+2} \leqslant a_{n+1}-a_n, n=1,2, \cdots, 100 .
$$
由此可得
$$
a_{100}-a_{99} \leqslant a_{98}-a_{97} \leqslant \cdots \leqslant a_2-a_1 \leqslant a_{100}-a_{99} .
$$
因此 $a_{2 k}-a_{2 k-1}=m(m$ 为常数, $k=1,2, \cdots, 50)$. 同理可得 $a_{2 k+1}-a_{2 k}=l(l$ 为常数, $k=1,2, \cdots, 50)$. 将所有这些等式相加得 $0=50 m+50 l$, 于是 $m= -l$. 这推出
$$
a_3-a_2=l=-m=a_1-a_2,
$$
即 $a_1=a_3$, 矛盾.
故题中结论成立.
%%PROBLEM_END%%



%%PROBLEM_BEGIN%%
%%<PROBLEM>%%
例3. 证明: 在任意 $n(\geqslant 4)$ 个人中都存在 2 人 $A$ 和 $B$ 使得其余 $n-2$ 人中至少有 $\left[\frac{n}{2}\right]-1$ 人满足: 他们中每个人或者同 $A, B$ 都互相认识或者同 $A$, $B$ 都互相不认识.
%%<SOLUTION>%%
证法一用反证法.
假设结论不成立, 那么对 $n$ 个人中任意 2 个人 $A$ 和 $B$, 在其余 $n-2$ 人中同时与 $A, B$ 互相认识以及同时与 $A, B$ 互相不认识的人一共至多只有 $\left[\frac{n}{2}\right]-2$ 个.
再设在其余 $n-2$ 人中恰与 $A, B$ 中一人互相认识的有 $k$ 人.
则
$$
n-2-k \leqslant\left[\frac{n}{2}\right]-2,
$$
所以
$$
k \geqslant n-\left[\frac{n}{2}\right] \geqslant n-\frac{n}{2}=\frac{n}{2},
$$
即对任意两人 $A, B$, 恰与 $A, B$ 中一个人互相认识的人至少有 $\frac{n}{2}$ 个.
若 $C$ 恰与 $A, B$ 中一个人互相认识, 则将 $(A, B, C)$ 组成一个三元组.
设这种三元组的个数为 $S$.
因为对任意两人 $A, B$,含 $A, B$ 的三元组至少有 $\frac{n}{2}$ 个, 而 $(A, B)$ 对有 $\mathrm{C}_n^2$ 种取法,故
$$
S \geqslant \frac{n}{2} C_n^2=\frac{n^2(n-1)}{4} . \label{eq1}
$$
另一方面, 对 $n$ 个人中任何 1 人 $C$, 设其余 $n-1$ 个人中有 $h$ 个同 $C$ 互相认识, $n-1-h$ 个人同 $C$ 互相不认识.
故含 $C$ 的三元组 $(A, B, C)$ 的个数为 $h(n-1-h) \leqslant\left[\frac{h+(n-1-h)}{2}\right]^2=\frac{(n-1)^2}{4}$. 而 $C$ 有 $n$ 种不同的取法,故
$$
S \leqslant \frac{n(n-1)^2}{4} . \label{eq2}
$$
式\ref{eq2}与\ref{eq1}矛盾.
于是题中结论成立.
%%PROBLEM_END%%



%%PROBLEM_BEGIN%%
%%<PROBLEM>%%
例3. 证明: 在任意 $n(\geqslant 4)$ 个人中都存在 2 人 $A$ 和 $B$ 使得其余 $n-2$ 人中至少有 $\left[\frac{n}{2}\right]-1$ 人满足: 他们中每个人或者同 $A, B$ 都互相认识或者同 $A$, $B$ 都互相不认识.
%%<SOLUTION>%%
证法二用平面内 $n$ 个点 $A_1, A_2, \cdots, A_n$ 表示 $n$ 个人(其中任意三点不共线). 若两人互相认识 (互相不认识), 则对应两点的连线染红 (蓝)色.
如果某点 $C$ 与另两点 $A, B$ 的连线同色, 那么称 $\angle A C B$ 为从 $C$ 点引出的同色角, 也叫做点 $C$ 对 $(A, B)$ 所张的同色角,简称同色角.
因此原题结论等价于证明: 存在两点 $A, B$, 使其余 $n-2$ 点中至少有 $\left[\frac{n}{2}\right]-1$ 个点对 $(A, B)$ 张有同色角.
如果结论不成立, 那么对任意两点 $A, B$, 对 $(A, B)$ 张有同色角的点至多有 $\left[\frac{n}{2}\right]-2$ 个.
又 $(A, B)$ 有 $C_n^2$ 种不同取法, 故图中同色角的个数 $S$ 满足
$$
S \leqslant\left(\left[\frac{n}{2}\right]-2\right) \mathrm{C}_n^2 \leqslant\left(\frac{n}{2}-2\right) \cdot \frac{n(n-1)}{2}=\frac{n(n-1)(n-4)}{4} . \label{eq3}
$$
另一方面, 设从 $A_i$ 出发有 $x_i$ 条红线, $n-1-x_i$ 条蓝线, 那么从 $A_i$ 引出的同色角的个数为
$$
\begin{aligned}
\mathrm{C}_{x_i}^2+\mathrm{C}_{n-1-x_i}^2 & =\frac{1}{2} x_i\left(x_i-1\right)+\frac{1}{2}\left(n-1-x_i\right)\left(n-2-x_i\right) \\
& =x_i^2-(n-1) x_i+\frac{(n-1)(n-2)}{2}
\end{aligned}
$$
$$
=\left(x_i-\frac{n-1}{2}\right)^2+\frac{(n-1)(n-3)}{4} \geqslant \frac{(n-1)(n-3)}{4} .
$$
所以图中同色角的个数 $S$ 满足
$$
S=\sum_{i=1}^n\left(\mathrm{C}_{x_i}^2+\mathrm{C}_{n-1-x_i}^2\right) \geqslant \frac{n(n-1)(n-3)}{4}, \label{eq4}
$$
式\ref{eq4}与\ref{eq3}矛盾,故题中结论成立.
%%PROBLEM_END%%



%%PROBLEM_BEGIN%%
%%<PROBLEM>%%
例4. 设 $A_1, A_2, \cdots, A_{2010}$ 是圆周上依次排列的 2010 个点, 最初 $A_1$ 上标的数为 $0, A_2, A_3, \cdots, A_{2010}$ 上标的数为 1 . 允许进行如下操作: 任取一点 $A_j$, 若 $A_j$ 上所标的数为 1 , 则同时将 $A_{j-1}, A_j, A_{j+1}$ 上标的数 $a, b, c$ 分别改为 $1-a, 1-b, 1-c$. (这时 $A_0=A_{2010}, A_{2011}=A_1$ ) 问能否经过有限次这样的操作将所有点上标的数都变为 0 ?
%%<SOLUTION>%%
解:不能.
现用反证法证明如下: 若可以经过有限次 ( $m$ 次) 操作, 使所有点上标的数都变为 0 , 则由于 $(1-a)+(1-b)+(1-c)=3-(a+b+c)$ 与 $a+b+c$ 的奇偶性相反.
故每经过一次操作, 标的 2010 个数之和的奇偶性都要改变一次,而最初 2010 个数之和为奇数 2009 ,所以 $m$ 是奇数.
其次, 设以 $A_j$ 为出发点的操作次数为 $x_j(1 \leqslant j \leqslant 2010)$, 在 $A_j$ 上标的数改变的次数为 $y_j$, 则 $\sum_{j=1}^{2010} x_j=m$, 并且 $y_1$ 为偶数, 当 $j \in\{2,3, \cdots, 2010\}$ 时 $y_j$ 为奇数.
此外还有 $y_j=x_{j-1}+x_j+x_{j+1}(1 \leqslant j \leqslant 2010)$, 这里 $x_0=x_{2010}$, $x_{2011}=x_1$, 所以应有 $m=y_2+y_5+y_8+\cdots+y_{2009}$. 而 $y_2, y_5, \cdots, y_{2009}$ 一共是 670 个奇数, 它们之和 $m$ 应为偶数, 这与 $m$ 为奇数矛盾, 故不可能经过有限次操作使所有点上标的数都变为 0 .
%%<REMARK>%%
注:由上述证明即知,把 2010 改为 $6 n\left(n \in \mathbf{N}_{+}\right)$, 其结论仍成立.
%%PROBLEM_END%%



%%PROBLEM_BEGIN%%
%%<PROBLEM>%%
例5. 能否将全体整数分拆为 3 个不相交的集合, 使得对任意整数 $n$, 数 $n, n-50, n+2011$ 分别属于所分成的 3 个不同的子集?
%%<SOLUTION>%%
解:不可能, 现用反证法证明如下, 假设存在符合题目要求的分拆方法.
我们用记号 $m \sim k$ 表示 $m$ 与 $k$ 属于同一子集.
若 3 个数 $p, q, r$ 分别属于 3 个不同的子集则记为 $(p, q, r) \in \mathcal{M}$. 于是, 邑 $(n, n-50, n+2011) \in \mathscr{M}$ 出发, 分别将 $n$ 用 $n-50$ 和 $n+2011$ 代替得
$$
\begin{aligned}
& (n-50, n-100, n+1961) \in \mathscr{M}, \\
& (n+2011, n+1961, n+2 \cdot 2011) \in \mathcal{M} .
\end{aligned}
$$
于是 $n+1961 \not \sim n-50, n+1961 \not \chi n+2011$, 故只有 $n+1961 \sim n$. 故由 $(n-50, n-100, n+1961) \in \mathcal{M}$ 得 $(n-50, n-100, n) \in \mathcal{M}$. 再将其中 $n$
用 $n-50$ 代替得 $(n-100, n-150, n-50) \in \mathcal{M}$, 于是由 $n \Varangle n-50$ 和 $n \nsucc n-100$ 得 $n \sim n-150$. 再从 $n \sim n+1961$ 和 $n \sim n-150$ 出发可推得
$$
\begin{aligned}
& 0 \sim 1961 \sim 2 \times 1961 \sim \cdots \sim 50 \times 1961 \\
= & 654 \cdot 150-50 \sim 653 \cdot 150-50 \sim \cdots \sim 150-50 \sim-50 .
\end{aligned}
$$
这与 $n \Varangle n-50$ 从而 $0 \nsucc-50$ 矛盾.
故满足题目要求的分拆子集方法是不存在的.
%%PROBLEM_END%%



%%PROBLEM_BEGIN%%
%%<PROBLEM>%%
例6. 在 $2 \times n$ 方格表的每个 $1 \times 1$ 小方格内写上一个正实数,使得每列中两个数之和等于 1 . 证明: 可以从每列中删去一个数, 使得每行中剩下的各数之和都不超过 $\frac{n+1}{4}$. 
%%<SOLUTION>%%
证明:假设第一行中的 $n$ 个数从左到右依次为 $a_1, a_2, \cdots, a_n$, 必要时交换列的位置可使得 $a_1 \leqslant a_2 \leqslant \cdots \leqslant a_n$. 此时第二行中所写的数依次为 $b_1= 1-a_1, b_2=1-a_2, \cdots, b_n=1-a_n$, 于是 $b_1 \geqslant b_2 \geqslant \cdots \geqslant b_n$. 如果 $a_1+a_2+\cdots+ a_n \leqslant \frac{n+1}{4}$, 那么就删去第二行中所有各数即可达到目的.
否则存在使 $a_1+ a_2+\cdots+a_k>\frac{n+1}{4}$ 成立的最小正整数 $k$, 这时我们只要删去 $b_1, b_2, \cdots, b_{k-1}$ 及 $a_k, a_{k+1}, \cdots, a_n$, 则由 $k$ 的取法有 $a_1+a_2+\cdots+a_{k-1} \leqslant \frac{n+1}{4}$, 并且由于
$$
a_k \geqslant \frac{a_1+a_2+\cdots+a_k}{k}>\frac{n+1}{4 k},
$$
所以
$$
\begin{gathered}
b_k+b_{k+1}+\cdots+b_n \leqslant(n+1-k) b_k=(n+1-k)\left(1-a_k\right) \\
<(n+1-k)\left(1-\frac{n+1}{4 k}\right)=\frac{5}{4}(n+1)-\left[\frac{(n+1)^2+(2 k)^2}{4 k}\right] \\
\leqslant \frac{5}{4}(n+1)-\frac{2(n+1)(2 k)}{4 k}=\frac{n+1}{4},
\end{gathered}
$$
从而证明了题中结论成立.
%%PROBLEM_END%%



%%PROBLEM_BEGIN%%
%%<PROBLEM>%%
例7. 有男女青年共 1000 人围成一个圆圈.
如果有一个女孩 $G$, 她沿任意方向依次数到任何人 (包括她自己以及最后数到的人),女孩的数目总是大于男孩的数目, 那么称 $G$ 在一个好位置上, 试问: 要保证在任何情形下都至少有一个女孩在好位置上,女孩的人数的最小值是多少?
%%<SOLUTION>%%
解:因为如果一个女孩的某一侧是一个男孩, 那么这个女孩一定不在好位置上.
所以只要将圆圈上 1000 个位置分成 334 组, 前 333 组都是 2 个女孩中间站 1 个男孩, 最后一组只 1 个男孩, 于是一共有 666 个女孩和 334 个男孩,但没有一个女孩站在好位置上.
故所求女孩人数的最小值不小于 667 .
下面我们证明: 如果 1000 个男女青年中至少有 667 个女孩,那么其中必有 1 个女孩站在好位置上.
实际上, 我们可以证明下列一般性结论: 如果 $3 n+1\left(n \in \mathbf{N}_{+}\right)$个男女青年站在一个圆圈上, 并且其中至少有 $2 n+1$ 个女孩, 那么必有一个女孩站在好位置上.
当 $n=1$ 时,一共有 $3 \times 1+1=4$ 个男女青年,其中至少有 $2 \times 1+1=3$ 个女孩, 从而至多有 1 个男孩,站成一个圆圈时, 必有一个女孩 $G$, 她的两侧都为女孩,故 $G$ 在好位置上.
设 $n=k$ 时结论成立, 则当 $n=k+1$ 时,一共有 $3(k+1)+1=3 k+4$ 个男女青年.
其中女孩至少有 $2(k+1)+1=2 k+3$ 个.
任取一个男孩 $A$, 并在 $A$ 的两侧各找一个到 $A$ 的距离最近的女孩 $B$ 和 $C$. 先证 $A 、 B 、 C$ 离开圆圈, 则圆圈上共有 $3 k+1$ 个男女青年.
其中女孩至少有 $2 k+1$ 个.
由归纳假设知其中必有 1 个女孩 $G$ 站在好位置上, 并且由好位置的定义知 $G$ 的两侧皆为女孩, 然后让 $A, B, C$ 站回去, 则由 $B 、 C$ 的取法知 $G$ 不在 $A, B, C$ 所在的圆弧上, 从 $G$ 出发沿任意方向计算男、女孩总人数时, 必然是先数到 $B$ (或 $C$ ) 后才数到 $A$ (如果要数到 $A$ 的话). 由归纳假设知, 数到的女孩总人数必定多于男孩的总人数, 即 $G$ 仍在好位置上, 于是 $n=k+1$ 时结论成立.
故一般性命题得证.
特别 $n=333$ 时结论成立.
由此可知原题中所求女孩人数的最小值为 667 .
%%PROBLEM_END%%



%%PROBLEM_BEGIN%%
%%<PROBLEM>%%
例8. 在学校足球冠军赛中, 要求每一个队都必须同其余各队进行一场比赛, 每场比赛胜队得 2 分, 平局各得 1 分, 负队得 0 分.
已知有一队得分最多 (其余每队得分都比这队少), 但它胜的场次比任何一队都少, 问最少有多少队参赛?
%%<SOLUTION>%%
解:设 $A$ 队是得分最多的队, 它胜 $n$ 场平 $m$ 场, 则 $A$ 队的总分为 $2 n+ m$. 由已知, 其余每队至少要胜 $n+1$ 场, 得分不少于 $2(n+1)$ 分, 于是
$$
2 n+m>2(n+1), m \geqslant 3 .
$$
即 $A$ 至少平 3 场, 故可找到一个队, 它和 $A$ 踢成平局, 这个队得分至少为
$2(n+1)+1$ 分, 从而有
$$
2 n+m>2(n+1)+1, m \geqslant 4 .
$$
设共有 $S$ 个队参加比赛, 则 $A$ 队至少胜一场, 否则 $A$ 队的得分不超过 $S-1$, 而任何其他队的得分都严格少于 $S-1$, 这样所有参赛队的得分严格少于 $S(S-1)$ 这与 $S$ 个队参赛所得的总分为 $2 \mathrm{C}_S^2=S(S-1)$ 矛盾.
于是 $m \geqslant 4, n \geqslant 1$, 即 $A$ 队至少要比赛 5 场, 所以参加比赛的队不少于 6 个.
\begin{tabular}{|c|c|c|c|c|c|c|c|}
\hline & $A$ & $B$ & $C$ & $D$ & $E$ & $F$ & 得分 \\
\hline$A$ & & 1 & 1 & 1 & 1 & 2 & 6 \\
\hline$B$ & 1 & & 2 & 0 & 0 & 2 & 5 \\
\hline$C$ & 1 & 0 & & 0 & 2 & 2 & 5 \\
\hline$D$ & 1 & 2 & 2 & & 0 & 0 & 5 \\
\hline$E$ & 1 & 2 & 0 & 2 & & 0 & 5 \\
\hline$F$ & 0 & 0 & 0 & 2 & 2 & & 4 \\
\hline
\end{tabular}
另一方面, 如表中所示 $A 、 B 、 C 、 D 、 E 、 F$ 六个队满足题目要求: $A$ 队胜的场次最少,而得分最多.
综上可知,最少应有 6 个队参加比赛.
%%PROBLEM_END%%


