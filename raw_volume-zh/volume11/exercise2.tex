
%%PROBLEM_BEGIN%%
%%<PROBLEM>%%
问题1. 有 17 位科学家,其中每位科学家都同其他所有人通信,他们在通信时只讨论了三个题目,且每两位科学家之间只讨论一个题目, 证明至少有三位科学家,他们互相之间讨论的是同一个题目.
%%<SOLUTION>%%
在圆周上任取 17 个点代表 17 位科学家,互相通信讨论的 3 个题目分别用对应两点连线染成红、蓝、黄 3 种颜色来表示, 因为从其中一点 $A_1$ 出发的 16 条线段被染成了 3 种颜色,由抽屈原理知其中至少有 $\left[\frac{16-1}{3}\right]+1=6$ 条同色, 不妨设 $A_1 A_2, A_1 A_3, A_1 A_4, A_1 A_5, A_1 A_6, A_1 A_7$ 同为红色.
若 $A_2$, $A_3, \cdots, A_7$ 这 6 点中有 2 点 $A_i, A_j(2 \leqslant i<j \leqslant 7)$ 的连线染红色, 则存在三边同为红色的三角形 $A_1 A_i A_j$, 否则 $A_2, A_3, \cdots, A_7$ 这 6 点间的连线只染成了蓝、黄两种颜色, 由 Ramsey 定理知也存在三边同色的三角形.
这都表示对应的 3 位科学家,他们之间讨论的是同一个题目.
%%PROBLEM_END%%



%%PROBLEM_BEGIN%%
%%<PROBLEM>%%
问题2. 是否存在 (1) 4 个; (2) 5 个不同的正整数,它们中任意 3 个数之和是素数? 
%%<SOLUTION>%%
(1) $1,3,7,9$ 满足要求; (2) 考虑任意 5 个数被 3 除的余数,若 0,1 ,
2 这三种余数都出现, 则这 3 个数之和被 3 整除, 不是素数; 若 $0,1,2$ 这三种余数中至多出现 2 种, 则由抽屈原理知至少有 $\left[\frac{5-1}{2}\right]+1=3$ 个数被 3 除的余数相等, 从而这三个数之和也是 3 的倍数, 不是素数.
故不存在 5 个不同的正整数,它们中任意三个数之和为素数.
%%PROBLEM_END%%



%%PROBLEM_BEGIN%%
%%<PROBLEM>%%
问题3. 在面积为 1 的 $\triangle A B C$ 内任意放人 7 个点, 其中任意 3 点不共线.
证明 : 这 7 个点中必有 3 个点, 以它们为顶点的三角形的面积不大于 $\frac{1}{4}$.
%%<SOLUTION>%%
如图(<FilePath:./figures/fig-c2a3.png>), 将 $\triangle A B C$ 分成一个平行四边形 $B D E F$ 和两个三角形: $\triangle C D E, \triangle A F E$, (其中 $D 、 E 、 F$ 分别为三边 $B C 、 A C 、 A B$ 的中点) 由抽屈原理知至少在 $\left[\frac{7-1}{3}\right]+1=3$ 个点落在同一个图形内, 以这 3 点为顶点的三角形面积不大于 $\frac{1}{4}$.
%%PROBLEM_END%%



%%PROBLEM_BEGIN%%
%%<PROBLEM>%%
问题4. 设 $S=\{1,2,3, \cdots, 2011\}$, 问从 $S$ 中最多能选出多少个数,使得其中任何两数之和都不能被它们的差整除?
%%<SOLUTION>%%
令 $M=\{3 k+1 \mid k=0,1,2, \cdots, 670\}$, 则 $M \varsubsetneqq S$ 且 $|M|=671$, $M$ 中任意两数之差均是 3 的倍数但两数之和则不是 3 的倍数,故 $M$ 中任意两数之和都不被它们的差整除, 所以, 所求的最大值不小于 671 . 其次, 将 $S$ 中的数分为下列 671 个子集: $\{1,2,3\},\{4,5,6\},\{7,8,9\}, \cdots,\{2008,2009$, $2010\},\{2011\}$,从 $M$ 中任取 672 个数,其中必有 $\left[\frac{672-1}{671}\right]+1=2$ 个数属于同一组, 这两个数之差不大于 2 , 设这两个数为 $a$ 和 $b$ 且 $a>b$. 若 $a-b=1$, 则显然有 $a+b$ 被 $a-b$ 整除; 若 $a-b=2$, 则 $a$ 与 $b$ 同为奇数或同为偶数,即 $a+b$ 必为偶数, 从而 $a+b$ 也被 $a-b$ 整除.
综上可知从 $S$ 中最多可选出 671 个数,使其中任意两数之和都不能被它们的差整除.
%%PROBLEM_END%%



%%PROBLEM_BEGIN%%
%%<PROBLEM>%%
问题5. 设 $S=\{1,2,3, \cdots, 2000\}, M$ 是 $S$ 的一个子集且 $M$ 中任意两数之差都不等于 5 或 8 , 问 $M$ 中最多有多少个元素?
%%<SOLUTION>%%
首先证明 $S$ 中任意 13 个连续正整数中最多有 6 个属于 $M$. 以 $T= \{1,2, \cdots, 13\}$ 为例进行证明.
考虑 $T$ 的下列 13 个子集: $\{1,6\},\{2,7\}$, $\{3,8\},\{4,9\},\{5,10\},\{6,11\},\{7,12\},\{8,13\},\{1,9\},\{2,10\}$, $\{3,11\},\{4,12\},\{5,13\}$ 且 $T$ 中每个数恰属于 2 个子集.
任取 $T$ 中 7 个元素, 它们属于上述 13 个子集中 14 个子集, 由抽屉原理知其中必有 2 个元素属于同一个子集,它们之差等于 5 或 8 . 因此 $T$ 中任何 7 个元素都不能同时属于 $M$. 另一方面 $T^{\prime}=\{1,2,4,5,8,11\}$ 中任何两个数之差不等于 5 或 8 , 它们可以同时属于 $M$, 故 $T$ 中最多有 6 个数属于 $M$. 因为 $2000=13 \times 154-2$, 故 $S$ 中最多有 $154 \times 6=924$ 个数属于 $M$. 又因为集合 $\{1,2,4,5$, $8,11,14,15,17,18,21,24\}$ 中任何两个数之差不等于 5 或 8 , 故集合 $\{13 n+k \mid k=1,2,4,5,8,11, n=0,1,2, \cdots, 153\}$ 中任何两个数之差不等于 5 或 8 ,并且它有 924 个数.
综上可知 $M$ 内最多有 924 个数.
%%PROBLEM_END%%



%%PROBLEM_BEGIN%%
%%<PROBLEM>%%
问题6. 有 $n(n>12)$ 个人参加某次数学邀请赛, 试卷由 15 个填空题组成, 每答对 1 题得 1 分, 不答或答错得 0 分.
分析每一种可能的得分情况, 发现: 只要其中任意 12 个人得分之和不少于 36 分, 则这 $n$ 个人中至少有 3 个人答对了至少 3 个同样的题,求 $n$ 的最小可能值.
%%<SOLUTION>%%
注意到 $\mathrm{C}_{15}^3=455$. 若有 $2 \mathrm{C}_{15}^3=910$ 名学生参赛, 将这 910 名学生分为 455 组,每组 2 人,并且每组恰答对同样的 3 个题并且不同的组答对的 3 个题不全相同, 此时不满足题设条件,故所求 $n$ 的最小值不小于 911 . 另一方面若至少有 911 名学生参赛, 则只有下列两种情形: (1) 当每个学生至少答对 3 个题时, 而每个学生答对 3 个题的不同情况只有 $\mathrm{C}_{15}^3=455$ 种, 故由抽屉原理知至少有 $\left[\frac{911-1}{455}\right]+1=3$ 名学生答对了同样的 3 个问题; (2) 当有一名学生 $A$ 答对的题数不多于 2 时, 其他学生中答对不超过 3 个题的学生人数不超过 10 人 (否则他们中 11 人与第一个学生 $A$ 共 12 人的得分总和不多于 $2 \times 1+3 \times 11=35<36$, 这与已知条件矛盾). 故至少有 $911-11=900$ 名学生, 他们中每一个至少答对了 4 个问题, 由于 $\mathrm{C}_4^3=4$, 故由抽居原理知这时至少有 $\left[\frac{4 \times 900-1}{455}\right]+1=8$ 名学生答对了同样的 3 个问题.
总之, 当参赛学生人数不少于 911 人中, 其中至少有 3 人答对了至少 3 个同样的问题.
综上可得所求 $n$ 的最小值为 911 .
%%PROBLEM_END%%



%%PROBLEM_BEGIN%%
%%<PROBLEM>%%
问题7. 10 人到书客去买书,已知:
(1) 每人都买了 3 种书;
(2)任何两人买的书中,都至少有一种相同.
问购买人数最多的一种书最少有几人购买? 说明理由.
%%<SOLUTION>%%
设 $A$ 是 10 人之一, 由已知 $A$ 共买了 3 种书, 且其余 9 人所买的书中都至少有一种与 $A$ 买的书相同, 于是由抽屉原理知, 9 人中至少有 $\left[\frac{9-1}{3}\right]+ 1=3$ 人, 加上 $A$ 共 4 人买了同一种书, 因而所求最小值不小于 4 . 若购买人数最多的一种书共有 4 人购买, 则可以证明每种书恰有 4 人购买.
设 10 人共买下了 $n$ 种不同的书,则有 $4 n=30$, 但因 4 不整除 30 , 此不可能,故知所求的最小值不小于 5 . 另一方面, 我们用数字表示书种的号码, 并使 10 人分别买书如下: $\{1,2,3\},\{1,2,3\},\{1,4,5\},\{1,6,7\},\{2,4,6\},\{2,4,6\},\{2$, $5,7\},\{3,4,7\},\{3,5,6\},\{3,5,6\}$. 容易验证,他们买的书满足题中要求且购买人数最多的一种书有 5 人购买, 故知所求最小值等于 5 .
%%PROBLEM_END%%



%%PROBLEM_BEGIN%%
%%<PROBLEM>%%
问题8. 的面平面上每个点都以红、蓝色之一着色.
证明: (1) 对任意实数 $a$, 存在边长为 $a, \sqrt{3} a, 2 a$ 且三个顶点同色的直角三角形; (2) 存在两个相似三角形, 它们的相似比为 1995 , 并且每个三角形的三个顶点同色.
%%<SOLUTION>%%
(1) 首先证明: 对任意实数 $a$, 存在距离等于 $2 a$ 且同色的两点.
事实上, 取一个红点 $A$, 以 $A$ 为中心, $2 a$ 为半径作圆.
若该圆上有一红点 $B$, 则结论成立, 否则该圆上所有的点全为蓝点.
于是该圆的内接正六边形的一边的两个端点的距离等于 $2 a$ 且全为蓝点,结论也成立.
其次设 $A B=2 a$ 且 $A$ 与 $B$ 同色,不妨设 $A$ 与 $B$ 同为红色, 以 $A B$ 为直径作圆, 并且设该圆的六等分点依次为 $A, C, D$, $B, E, F$. 若 $C 、 D 、 E 、 F$ 中有一点为红点, 例如 $C$ 为红点, 则直角 $\triangle A B C$ 的 3 个顶点同为红色且 $B C=a, C A=\sqrt{3} a, A B=2 a$. 结论得证, 否则直角 $\triangle C D E$ 的 3 个顶点同为蓝色且 $C D=a, D E=\sqrt{3} a, E C=2 a$, 结论也得证.
(2) (证法一) 在 (1) 中分别取 $a=1, a=1995$ 所得两个直角三角形即为所求.
(证法二) 以任意一点 $O$ 为圆心, 分别以 1 和 1995 为半径作两个同心圆.
在内圆上任取 9 个点, 由抽庶原理知其中至少有 $\left[\frac{9-1}{2}\right]+1=5$ 个点同色, 不妨设内圆上的 5 个点 $A_1, A_2, A_3, A_4, A_5$ 同色,作射线 $O A_i$ 交外圆于 $B_i(i=1,2,3,4,5)$, 再由抽庶原理知 $B_1, B_2, B_3, B_4, B_5$ 中至少有 3 个点 $B_i, B_j, B_k(1 \leqslant i<j<k \leqslant 5)$ 同色, 于是 $\triangle B_i B_j B_k$ 与 $\triangle A_i A_j A_k$ 是相似比为 1995 的两个相似三角形并且每个三角形的三个顶点同色.
%%PROBLEM_END%%



%%PROBLEM_BEGIN%%
%%<PROBLEM>%%
问题9. 若凸多面体有 6 个顶点和 12 条棱,证明: 它的每一个面都是三角形.
%%<SOLUTION>%%
设凸多面体有 $F$ 个面, 又已知它的顶点数为 $V=6$, 棱数为 $E=12$, 代入 Euler 公式 $V+F-E=2$, 得 $F=E+2-V=12+2-6=8$. 设第 $i$ 个面上有 $x_i$ 条棱 $(i=1,2, \cdots, 8)$, 于是 $x_i \geqslant 3$ 且 $x_1+x_2+\cdots+x_8= 2 E=24$, 平均值 $\frac{1}{8} \sum_{i=1}^8 x_i=3$. 若有某个 $x_{i_0}>3$, 则 $\frac{1}{8} \sum_{i=1}^8 x_i>3$, 矛盾.
所以 $x_1=x_2=\cdots=x_8=3$, 即每个面均为三角形.
%%PROBLEM_END%%



%%PROBLEM_BEGIN%%
%%<PROBLEM>%%
问题10. 平面有限点集叫做稳定的, 如果它内部任意两点间的距离是确定的.
给定含 $n \geqslant 4$ 个点的平面点集 $M_n$, 其中任意 3 点不共线, 若 $M_n$ 中有 $\frac{1}{2} n(n-$ 3 ) +4 对点之间的距离是确定的.
证明: $M_n$ 是稳定的.
%%<SOLUTION>%%
对 $n$ 用归纳法.
$n=4$ 时, 已知这 4 点中有 $\frac{1}{2} \times 4 \times(4-3)+4=$ 6 对点之间的距离是确定的,但 4 个点一共只能形成 $\mathrm{C}_4^2=6$ 对点, 所以这 4 点组成的点集 $M_4$ 是稳定的.
假设 $n=k(k \geqslant 4)$ 时结论成立, 即当 $M_k$ 中已有 $\frac{1}{2} k(k-3)+4$ 对点间的距离是确定的时, $M_k$ 是稳定的.
那么 $n=k+1$ 时, 假设 $M_{k+1}$ 中已有 $\frac{1}{2}(k+1)(k-2)+4$ 对点之间的距离是确定的.
那么从这 $k+1$ 个点出发有确定距离的点对数之和为 $(k+1)(k-2)+8$, 由平均值原理知, 其中必有一个点 $A$, 从它出发的有确定距离的点对数 $l \leqslant \frac{1}{k+1}\{(k+1)(k-2)+8\}=k-1+\frac{7-k}{k+1}$, 而 $\frac{7-k}{k+1}<1$, 所以 $l \leqslant k-1$. 去掉点 $A$,于是还剩下 $k$ 个点, 这 $k$ 个点中至少有 $\frac{1}{2}(k+1)(k-2)+4- (k-1)=\frac{1}{2} k(k-3)+4$ 对点之间的距离是确定的.
由归纳假设知, 这 $k$ 个点所组成的点集是稳定的且 $A$ 至少与这 $k$ 个点中 $\frac{1}{2}(k+1)(k-2)+4- \mathrm{C}_k^2=3$ 个点之间的距离是确定的.
不妨设这 3 个点为 $B 、 C 、 D$, 且 $A B=x$, $A C=y, A D=z$ ( $x, y, z$ 是确定的正数). 这时点 $A$ 也唯一确定, 否则存在另一点 $A^{\prime} \neq A$, 使 $A^{\prime} B=x, A^{\prime} C=y, A^{\prime} D=z$, 于是 $B 、 C 、 D$ 都在线段 $A A^{\prime}$ 的垂直平分线上, 这与已知条件无 3 点共线矛盾, 所以 $M_{k+1}$ 也是稳定的, 这就完成了归纳证明.
%%PROBLEM_END%%


