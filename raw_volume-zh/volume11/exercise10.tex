
%%PROBLEM_BEGIN%%
%%<PROBLEM>%%
问题1. 某个委员会有 $n(\geqslant 5)$ 个成员, 并且有 $n+1$ 个三人委员会, 其中没有两个三人委员会有完全相同的成员.
证明: 存在两个三人委员会恰好有一个成员相同.
%%<SOLUTION>%%
用反证法.
假设任何两个三人委员会或者有两个成员相同,或者没有成员相同.
如果委员会 $A$ 和 $B$ 有公共成员, 那么它们恰有两个公共成员 $a$ 和 $b$. 如果委员会 $B$ 又和 $C$ 有两个公共成员, 那么 $a$ 和 $b$ 中至少有一个属于 $C$, 从而 $C$ 与 $A$ 也应有两个公共成员.
于是, 我们可将所有三人委员会进行分类: 使同一类的两个委员会都恰有两个公共成员,不同类的委员会没有公共成员.
下面证明: 同一类中委员会的个数 $k$ 不大于这类中委员的人数 $h$, 显然 $h \geqslant 3$. 当 $h=3$ 时 $k=1$; 当 $h \geqslant 4$ 时 $k \geqslant 2$, 设 $\{x, a, b\},\{y, a, b\}$ 是其中两个委员会, 则其他同类的委员会只能是 $\{x, y, a\},\{x, y, b\}$ 或 $\{a, b, z\}$ 的形式, 这里 $z$ 最多有 $h-4$ 种选择.
所以 $k \leqslant 4+(h-4)=h$. 于是委员会的总数 $n+1 \leqslant$ 总人数 $n$, 矛盾.
这表明, 至少有两个委员会恰有一个公共成员.
%%PROBLEM_END%%



%%PROBLEM_BEGIN%%
%%<PROBLEM>%%
问题2. 设有 9 枚棋子放在 $8 \times 8$ 国际象棋棋盘的左下角, 每小格内放一枚,组成 $3 \times 3$ 的正方形,规定每枚棋子可以跳过他邻格中的另一枚棋子到一个空着的方格.
即可以关于它有棋子的邻格中心作对称运动 (可以横跳、坚跳或沿对角线斜跳). 要求这些棋子跳到棋盘的另一角, 且仍构成 $3 \times 3$ 正方形,如果达到的是:(1)左上角, (2)右上角, 这一要求能否实现?
%%<SOLUTION>%%
均不能实现.
将棋盘上的行自下而上编号、列自左向右编号为 1,2 , $\cdots, 8$. 第 $i$ 行第 $j$ 列的方格记为 $(i, j)$, 考察 9 枚棋子的纵坐标之和 $S$. 如果它们能跳到左(右)上角 $3 \times 3$ 正方形中, 那么和 $S$ 将增加 $3(6+7+8-1-2-3 )=45$, 即 $S$ 的奇偶性发生了变化,但另一方面, 某棋子跳一次, 和 $S$ 增加 2 (坚跳或斜跳) 或 0 (横跳). 因此, 棋子在跳动过程中和 $S$ 的奇偶性不变, 矛盾.
这表明题中两个要求均不能实现.
%%PROBLEM_END%%



%%PROBLEM_BEGIN%%
%%<PROBLEM>%%
问题3. 设 $n$ 为奇数且大于 1 , 又 $k_1, k_2, \cdots, k_n$ 是给定的整数, 对 $1,2, \cdots, n$ 的 $n !$ 个排列中每一个排列 $a=\left(a_1, a_2, \cdots, a_n\right)$, 记 $S(a)==\sum_{i=1}^n k_i a_i$. 证明: 存在两个排列 $b$ 和 $c, b \neq c$, 使 $S(b)-S(c)$ 被 $n !$ 整除.
%%<SOLUTION>%%
用反证法.
假设对任意排列 $b$ 和 $c, b \neq c$ 都有 $S(b)-S(c) \not \equiv O(\bmod n !) \cdots$ (1), 
于是由(1)知当 $a$ 遍历 $1,2, \cdots, n$ 的 $n !$ 个排列时, $S(a)$ 遍历模 $n$ ! 的一个完全剩余系.
记 $1,2, \cdots, n$ 的所有排列集合为 $\sigma$,于是 $\sum_{a \in \sigma} S(a) \equiv \sum_{k=1}^{n !} k=\frac{1}{2}(n !)(n !+1)(\bmod n !)$. 
又 $n>1$ 为奇数, 故 $\sum_{a \in \sigma} S(a) \equiv \frac{1}{2} n ! \not \equiv 0(\bmod n !) \cdots$ (2). 
另一方面 $\sum_{a \in \sigma} S(a)=\sum_{a \in \sigma} \sum_{i=1}^n k_i a_i=\sum_{i=1}^n k_i\left(\sum_{a \in \sigma} a_i\right)= \frac{(n !)(n+1)}{2} \sum_{i=1}^n k_i \equiv 0(\bmod n !)$; 这与(2)矛盾.
故命题结论得证.
%%PROBLEM_END%%



%%PROBLEM_BEGIN%%
%%<PROBLEM>%%
问题4. 某次聚会共 17 人, 其中每个人都恰好认识另外 4 人, 求证: 存在两人, 他们彼此不认识且没有共同认识的人.
%%<SOLUTION>%%
如图(<FilePath:./figures/fig-c10a4.png>), 用 17 个点 $A_1, A_2, \cdots, A_{17}$ 表示 17 个人,若两人互相认识, 则对应点连一线段, 否则不连线段.
若两人互相认识 (即对应点间连有线段) 或两人有共同的熟人 (即对应两点与同一点连有线段, 这时称这两点张有角) 则称这两人对应的两点是关联的.
问题归结为存在不关联的两点.
用反证法.
假设图中所有的点都是两两关联的,即任意两点或连有线段或张有角.
已知图中共连有 $\frac{1}{2} \times 17 \times 4=34$ 条线段,所张的角有 $17 C_4^2=102$ 个, 且 $34+102= 136=\mathrm{C}_{17}^2$ 恰等于两点组数.
因此, 任意两点或连有线段或张有唯一一个角, 二者必居之一.
可见图中既不存在三角形 (否则连有线段的两点又张有角) 也不存在四边形 (否则有两点张有两个角). 考察其中任意一点 $X$, 它与 4 点 $A 、 B 、 C 、 D$ 连有线段.
由上述讨论知 $A 、 B 、 C 、 D$ 两两之间没有连线.
从 $A$ 、 $B 、 C 、 D$ 出发, 除 $X$ 外各自都与 3 个点连有线段 (如图所示). 这样一共有 17 个点,连有 16 条线段.
从 $A 、 B 、 C 、 D$ 所连出的 4 个 3 点组, 同组内任何两点不连线, 每点都只能与其他组内的点连有线段, 每连一条线段, 使得到一个含点 $X$ 的 5 点圈 (如图中 $X A A_2 B_1 B X$ 和 $X B B_3 C_3 C X$ 等). 由于还要连 $34- 16=18$ 条线, 故包含点 $X$ 的 5 点圈有 18 个, 由于 $X$ 是任意的, 因此, 每一点都包含在 18 个 5 点圈内, 故图中的 5 点圈应有 $\frac{17 \times 18}{5}$ 个, 但 $\frac{17 \times 18}{5}$ 不是整数, 矛盾! 因此, 图中的点不可能是两两关联的, 故一定存在不关联的两点, 即存在两人彼此不认识且没有共同认识的人.
%%PROBLEM_END%%



%%PROBLEM_BEGIN%%
%%<PROBLEM>%%
问题5. 设 $n$ 为正整数, $M$ 是具有下列性质的 $n^2+1$ 个正整数构成的集合: $M$ 中任意 $n+1$ 个数中必有 2 个数, 使得其中一个数是另一个数的倍数, 证明: $M$ 中存在 $n+1$ 个数 $a_1, a_2, \cdots, a_{n+1}$, 使得对任意 $i=1,2, \cdots, n$, 都有 $a_i$ 整除 $a_{i+1}$. 
%%<SOLUTION>%%
对任意 $k$ 个正整数 $x_1, x_2, \cdots, x_k$, 若对 $i=1,2, \cdots, k-1$ 均有 $x_i$ 整除 $x_{i+1}$, 则称 $\left(x_1, x_2, \cdots, x_k\right)$ 为一条长为 $k$ 的链, 且称 $x_1$ 为该链的首元.
对 $M$ 中每个元 $x_i\left(1 \leqslant i \leqslant n^2+1\right)$, 考虑取自 $M$ 的以 $x_i$ 为首元的链中最长的链, 并记这个最长的链的链长为 $l_i\left(i=1,2, \cdots, n^2+1\right)$. 下面我们只需证明: $l_1, l_2$, $\cdots, l_{n^2+1}$ 中至少有一个数不小于 $n+1$. 若对任意 $1 \leqslant i \leqslant n^2+1$ 均有 $l_i \leqslant n$, 则由抽庶原理 $l_1, l_2, \cdots, l_n{ }^2+1$ 中至少有 $\left[\frac{\left(n^2+1\right)-1}{n}\right]+1=n+1$ 个相等.
不妨设 $l_1=l_2=\cdots=l_{n+1}=r$, 而由 $M$ 的性质知 $x_1, x_2, \cdots, x_{n+1}$ 中必有一个数是另一个数的倍数.
不妨设 $x_1$ 整除 $x_2$. 于是将 $x_1$ 置于以 $x_2$ 为首元的那条最长的链的前面, 我们得到一条长为 $l_2+1=r+1$ 且以 $x_1$ 为首元的链, 这与以 $x_1$ 为首元的最长链的长为 $l_1=r$ 矛盾.
故 $l_1, l_2, \cdots, l_{n^2+1}$ 中必有一个数不小于 $n+1$. 不妨设 $l_1 \geqslant n+1$, 即存在 $M$ 中 $n+1$ 个数 $a_1, a_2, \cdots, a_{n+1}$ 使 $a_i$ 整除 $a_{i+1}(i=1,2, \cdots, n)$.
%%PROBLEM_END%%



%%PROBLEM_BEGIN%%
%%<PROBLEM>%%
问题6. 某地区网球俱乐部的 20 名成员进行了 14 场单打比赛, 每人至少上场 1 次.
求证: 必有 6 场比赛, 上场的 12 名队员互不相同.
%%<SOLUTION>%%
记参加第 $i$ 场比赛的选手为 $\left(a_i, b_i\right)(i=1,2, \cdots, 14)$, 并记 $M= \left\{\left(a_i, b_i\right) \mid i=1,2, \cdots, 14\right\}$. 我们称 $M$ 的一个子集为好子集,如果该子集所含选手对中出现的所有选手互不相同.
显然好子集是存在的(例如只含一对选手的子集) 且个数有限.
故存在一个含选手对最多的好子集 $M_0$, 设 $M_0$ 中含有 $r$ 对选手, 只要证 $r \geqslant 6$, 反设 $r \leqslant 5$. 于是 $M_0$ 中没有出现过的 $20-2 r$ 名选手之间没有进行比赛 (因为若有一对选手 $a$ 和 $b$ 比了赛, 则 $M_0 \cup\{(a, b)\}$ 仍然是好子集, 这与 $M_0$ 是含选手对最多的,假设矛盾). 这表明这 $20-2 r$ 名选手参加的比赛一定是同 $M_0$ 中的 $2 r$ 名选手进行的.
又已知每名选手至少参加一场比赛, 故这 20-2r 名选手至少参加了 $20-2 r$ 场比赛, 加上 $M_0$ 中的 $r$ 场比赛, 故总的比赛场次至少为 $(20-2 r)+r=20-r \geqslant 15$, 这与一共只有 14 场比赛矛盾, 所以 $r \geqslant 6$. 即至少有 6 场比赛, 参赛的 12 名选手互不相同.
%%PROBLEM_END%%



%%PROBLEM_BEGIN%%
%%<PROBLEM>%%
问题7. 有 20 个队参加全国足球冠军赛, 为了使已比赛过的任何三个队中都有两个队互相比赛过,最少要进行多少场比赛?
%%<SOLUTION>%%
设经过 $S$ 场比赛可使已经比赛过的任何三个队中都有两个队互相比赛过.
我们选出其中一个比赛场次最少的队 $A$, 设 $A$ 队比赛了 $k$ 场, 于是与 $A$ 比赛过的 $k$ 个队 $B_1, B_2, \cdots, B_k$ 都至少比赛了 $k$ 场, 从与 $A$ 没有比赛过的 $19-k$ 个队 $C_1, C_2, \cdots, C_{19-k}$ 中任取两队 $C_i, C_j(1 \leqslant i<j \leqslant 19-k)$, 则 $A, C_i, C_j$ 中必有两队比赛过一场.
但 $A$ 与 $C_i, A$ 与 $C_j$ 没有比赛过, 故只可能 $C_i$ 与 $C_j$ 比赛一场.
因此 $C_1, C_2, \cdots, C_{19-k}$ 中每个队至少与其余 $18-k$ 个队比赛过一场.
从而 20 个队比赛场次的总和至少为 $k(k+1)+(19-k)(18- k)$, 上述计数中每场比赛计算了两次.
所以 $S \geqslant \frac{1}{2}[k(k+1)+(19-k)(18- k)]=(k-9)^2+90 \geqslant 90$. 另一方面, 如果将 20 个队平分为两组, 同组 10 个队中任何两队安排一场比赛, 不同组的任何两队不安排比赛, 则一共比赛了 $2 \mathrm{C}_{10}^2=90$ 场, 任何三个队中必有 $\left[\frac{3-1}{2}\right]+1=2$ 个队属于同一组, 它们互相比赛过一队.
综上可知, 最少要安排 90 场比赛.
%%PROBLEM_END%%


