
%%TEXT_BEGIN%%
一、染色方法.
染色方法就是根据问题的特点, 对研究的对象用几种颜色染色, 并通过对染色对象 (点、线、区域, …) 以及数种染色对象的组合结构 (两边同色的角、 两边异色的角、三边同色的三角形, 三边不全同色的三角形、同色点对, 异色点对, $\cdots)$ 的数量和性质进行分析和比较, 从而使问题的解答能够比较容易地、直观地给出的一种解题方法.
二、赋值方法赋值方法就是根据问题的特点, 对研究的对象分别赋不同的数值, 并通过对所赋的值进行分析、计算和比较,从而得到问题解答的一种解题方法.
%%TEXT_END%%



%%PROBLEM_BEGIN%%
%%<PROBLEM>%%
例1. 用 $1 \times 1,2 \times 2,3 \times 3$ 的瓷砖铺满 $23 \times 23$ 的地(不允许重叠,也不留空隙), 问最少要用几块 $1 \times 1$ 的瓷砖?(假设每块瓷砖不允许分割成小的瓷砖)
%%<SOLUTION>%%
解:将 $23 \times 23$ 的方格地面 (已画成 $1 \times 1$ 的小方格) 中第 $1,4,7, \cdots$, 19,22 列中小方格染成黑色,其余各列中小方格染成白色,则每块 $2 \times 2$ 的瓷砖或盖住了 2 个白色方格和 2 个黑色方格或盖住了 4 个白色方格,而每块 $3 \times$ 3 的瓷砖盖住了 3 个黑色方格和 6 个白色方格, 如果不用 $1 \times 1$ 的瓷砖, 无论用多少块 $2 \times 2$ 和 $3 \times 3$ 的瓷砖,盖住的白色方格数总是一个偶数,但一共有 $15 \times$ 23 个白色方格, 而 $15 \times 23$ 是一个奇数,矛盾, 故不用 $1 \times 1$ 的瓷砖不可能将 $23 \times 23$ 的正方形铺满.
其次,如图(<FilePath:./figures/fig-c9i1.png>) 表明用 $2 \times 2$ 和 $3 \times 3$ 的瓷砖可铺满 $12 \times 11$ 的矩形,如图(<FilePath:./figures/fig-c9i2.png>) 表明只要用 1 块 $1 \times 1$ 的瓷砖可将 $23 \times 23$ 的正方形铺满.
综上可知,最少要用 1 块 $1 \times 1$ 的瓷砖.
%%PROBLEM_END%%



%%PROBLEM_BEGIN%%
%%<PROBLEM>%%
例2. 给定边长为 10 的正三角形, 用平行于其边的直线将它全部剖分为边长为 1 的小正三角形.
现有 $m$ 个如图(<FilePath:./figures/fig-c9i3.png>) 所示的三角形块且有 25- $m$ 个如图(<FilePath:./figures/fig-c9i4.png>) 所示的四边形块, 问
(1) 若 $m=10$, 能否用它们拼出原三角形?
(2) 求能拼出原三角形的所有 $m$.
%%<SOLUTION>%%
解:如图(<FilePath:./figures/fig-c9i5.png>) 把小正三角形块染上黑白相间的两种颜色.
设 $m$ 个三角形块能覆盖如图(<FilePath:./figures/fig-c9i5.png>) 中 3 个白色小正三角形的有 $x$ 个, 则如图(<FilePath:./figures/fig-c9i5.png>) 中白色小正三角形的个数为
$$
3 x+(m-x)+2(25-m)=2 x+50-m .
$$
而如图(<FilePath:./figures/fig-c9i5.png>) 中共有 55 个白色小正三角形,若 $m$ 个三角形块和 25- $m$ 个四边形块可以覆盖如图(<FilePath:./figures/fig-c9i5.png>), 则
$$
2 x+50-m=55 \text {, 即 } 2 x=m+5 .
$$
由此可知 $m$ 为奇数.
(1) 若 $m=10$, 则 $m$ 为偶数,故不能用 10 个三角形块及 15 个四边形块拼出原三角形;
(2)显然 $m \geqslant x$, 于是 $2 m \geqslant 2 x=m+5$, 即 $m \geqslant 5$, 又 $25-m \geqslant 0$, 所以 $m \leqslant 25$. 故 $m \in M=\left\{n \mid 5 \leqslant n \leqslant 25, n \in \mathbf{N}_{+}, n\right.$ 为奇数 $\}$. 另一方面, 对任意 $m=2 k-1 \in M(3 \leqslant k \leqslant 13)$. 如图(<FilePath:./figures/fig-c9i6.png>), 我们先用 5 个正三角形块覆盖原三角形右侧的 5 个边长为 2 的正三角形 (如图(<FilePath:./figures/fig-c9i6.png>) 中画有阴影的正三角形). 再用 $m-5=2(k-3)$ 个如图(<FilePath:./figures/fig-c9i3.png>) 的正三角形块覆盖图中 $k-3$ 个 $2 \times 2$ 的菱形.
最后余下的 $10-(k-3)=13-k$ 个 $2 \times 2$ 的菱形可用 $25-m=2(13-k)$ 个平行四边形覆盖.
综上可知, 所求的一切 $m$ 组成的集合为 $M$.
%%PROBLEM_END%%



%%PROBLEM_BEGIN%%
%%<PROBLEM>%%
例3. 已知某议会共有 30 位议员, 其中每两人或者是朋友, 或者是政敌, 每位议员恰有 6 个政敌.
每 3 个人组成一个 3 人委员会, 如果一个委员会里 3 个人两两都是朋友或者两两都是政敌,则称之为好委员会.
求所有好委员会的个数.
%%<SOLUTION>%%
解:用 30 个点代表 30 个委员 (其中任意 4 点不共面), 若两位议员是朋友, 则对应两点连一红色线段, 否则连一蓝色线段.
显然好委员的个数就是三边同色的三角形 (简称同色三角形) 的个数.
每个非同色三角形内有 2 个异色角 (从一点出发的两条不同色线段组成的角) 图中每点出发有 6 条蓝色线段, 23 条红色线段可组成以该点为顶点的 $23 \times 6=138$ 个异色角, 图中共有 $138 \times 30=4140$ 个异色角.
所以非同色三角形有 $\frac{1}{2} \times 4140=2070$ 个.
故同色三角形 (即好委员会) 的总数为 $\mathrm{C}_{30}^3-2070=1990$ 个.
%%PROBLEM_END%%



%%PROBLEM_BEGIN%%
%%<PROBLEM>%%
例4. 已知 8 个人中既不存在三个人互相认识, 也不存在四个人两两互相不认识, 问这 8 个人中最少有几对人互相认识? 最少有几对人互相不认识? 说明理由.
%%<SOLUTION>%%
解:用 8 个点表示 8 个人, 若两人互相认识, 则对应点连线染红色, 否则对应点连线染蓝色, 得到一个 2 色完全图 $K_8$, 由已知条件知其中既不存在红色三角形又不存在蓝色完全图 $K_4$.
若从某顶点 $A$ 出发至少有 6 条蓝边, 那么由.
Ramsey 定理知这 6 条蓝边另一端为顶点的 2 色 $K_6$ 中或者有红色三角形或者有蓝色三角形, 后者又导致存在蓝色 $K_4$, 都与已知矛盾.
由此可知, 这个 $K_8$ 中从每个顶点出发至少有 2 条红边, 故图中至少有 $\frac{1}{2} \times 2 \times 8=8$ 条红边.
若从某点 $B$ 出发至少有 4 条红边, 则以这 4 条红边的另一端为顶点的 $K_4$ 中或者有一条红边, 从而导致存在红色三角形, 或者它本身就是一个蓝色 $K_4$, 这都与已知矛盾, 故知图中每点出发至多有 3 条红边, 从而图中一共至多有 $\frac{1}{2} \times 3 \times 8=12$ 条红边.
(1)若 $K_8$ 中恰有 8 条红边, 则每点恰引出两条红边, 从而 8 条红边必构成一个或两个圈 (闭折线). 因为每点都只引出 2 条红边, 故当有两个圈时, 这两个圈无公共顶点, 又图中不存在三角形, 故当有两个圈时, 必然各有 4 条边.
当 8 条红边构成两个圈 $A_1 A_2 A_3 A_4$ 和 $A_5 A_6 A_7 A_8$ 时, $A_1 A_3 A_5 A_7$ 为蓝色 $K_4$, 当 8 条红边构成一个圈 $A_1 A_2 A_3 A_4 A_5 A_6 A_7 A_8$ 时, $A_1 A_3 A_5 A_7$ 也构成蓝色 $K_4$, 均与已知矛盾.
可见, $K_8$ 中至少有 9 条红边.
(2) 设 $K_8$ 中恰有 9 条红边时, 由于其中没有红色三角形且每点至多引出 3 条红边, 至少 2 条红边.
这样一来, $K_8$ 只有下列如图(<FilePath:./figures/fig-c9i7-1.png>)和(<FilePath:./figures/fig-c9i7-2.png>)两种可能(图中实线表红边,蓝边没有画出), 其中如图(<FilePath:./figures/fig-c9i7-1.png>)中虚线表示可将边 $A_2 A_7$ 去掉而代之以 $A_2 A_6$. 容易看出, (a) 和 (b) 中都有蓝色 $K_4: A_1 A_3 A_5 A_7$, 矛盾, 故知 $K_8$ 中至少有 10 条红线.
如图(<FilePath:./figures/fig-c9i8-1.png>),(<FilePath:./figures/fig-c9i8-2.png>)所示, 中都既无红色三角形又无蓝色 $K_4$, 综上可知, 2 色 $K_8$ 中最少有 10 条红边, 最多有 12 条红边, 即 8 个人中最少有 10 对人互相认识, 最多有 12 对人互相认识, 从而最少有 $\mathrm{C}_8^2-12=16$ 对人互相不认识.
%%PROBLEM_END%%



%%PROBLEM_BEGIN%%
%%<PROBLEM>%%
例5. 由 $2 \times 2$ 的方格纸去掉一个方格余下的图形称为拐形.
用这种拐形去覆盖 $5 \times 7$ 的方格板, 每个拐形恰覆盖 3 个方格, 可以重叠但不能超出方格板的边界.
问能否使方格板上每个方格被覆盖的层数都相同? 说明理由.
%%<SOLUTION>%%
解:将 $5 \times 7$ 方格板的每一个小方格内填写数 -2 和 1 如图(<FilePath:./figures/fig-c9i9.png>) 所示.
易见每个拐形覆盖的 3 个数之和非负.
因而无论用多少个拐形覆盖多少次, 盖住的所有数字之和 (一个数被覆盖了几层就计算几次)都是非负的.
另一方面, 方格板上数字的总和为 $12 \times(-2)+23 \times 1=-1$, 当被覆盖 $k$ 层时, 盖住的数字之和等于 $-k$, 这表明不存在
\begin{tabular}{|c|c|c|c|c|c|c|}
\hline-2 & 1 & -2 & 1 & -2 & 1 & -2 \\
\hline 1 & 1 & 1 & 1 & 1 & 1 & 1 \\
\hline-2 & 1 & -2 & 1 & -2 & 1 & -2 \\
\hline 1 & 1 & 1 & 1 & 1 & 1 & 1 \\
\hline-2 & 1 & -2 & 1 & -2 & 1 & -2 \\
\hline
\end{tabular}
满足题中要求的覆盖.
%%PROBLEM_END%%



%%PROBLEM_BEGIN%%
%%<PROBLEM>%%
例6. 把正六边形分成 24 个全等的小正三角形,在 19 个交点处任意写上互不相等的实数.
证明: 24 个小正三角形中至少存在 7 个三角形,其顶点上所写的 3 个实数可按顺时针方向排成递减数列.
%%<SOLUTION>%%
解:将 24 个三角形分为两类: 如图(<FilePath:./figures/fig-c9i10.png>) 所示, 第I类三角形的三个顶点上所写的 3 个实数可按顺时针方向排成递减数列, 其余三角形为第 II 类, 并设第 I 类正三角形有 $N$ 个, 从而第 II 类正三角形有 $24-N$ 个, 我们将小正三角形每条边按照从大数对应顶点到小数对应顶点的方向画一个箭头 (表示递减!)并且每条边沿箭头方向在其右侧赋值 1 ,左侧赋值 -1 (右侧表明箭头方向为顺时针的, 左侧表明箭头方向为逆时针的), 于是第 I 类三角形内所赋 3 个数值之和为 $1+1+(-1)=1$, 第 II 类三角形内所赋 3 个数值之和为 $(-1)+(-1)+1=-1$ 于是, 24 个小正 $(a>b>c)$ 三角形内所赋数值的总和为
$$
N+(-1)(24-N)=2 N-24 .
$$
另一方面, 位于正六边形内的小三角形的每边两侧所赋的值之和为 $1+ (-1)=0$, 而位于正六边形的边上的 12 条小正三角形的边中至少有一条边在正六边形内侧赋的值为 1 , 故正六边形内赋的值的总和至少为 $11 \times(-1)+ 1=-10$.
综合上述两方面, 得 $2 N-24 \geqslant-10$, 解得 $N \geqslant 7$. 即符合题目要求的小正三角形至少有 7 个.
应用上述证题方法不难将上述问题作如下推广:将一个凸(或凹)的多边形完全剖分成三角形区域,要求每一个三角形的顶点不在另一个三角形的边的内部,假设一共剖分为 $n$ 个三角形, 其中有一边位于多边形边界位置的三角形有 $m$ 个 $(n-m$ 为正偶数). 如果在每--个三角形的顶点处任意写上一个互不相等的实数,证明其中至少有 $\frac{1}{2}(n-m)$ +1 个三角形,其顶点处所写的 3 个实数可按顺时针方向排成递减数列.
%%PROBLEM_END%%



%%PROBLEM_BEGIN%%
%%<PROBLEM>%%
例7. 将 $m \times n$ 棋盘(由 $m$ 行 $n$ 列方格组成, $m \geqslant 3, n \geqslant 3$ ) 的所有小方格都染上红、蓝二色之一.
如果两个相邻 (有公共边) 的小方格异色, 则称这两个小方格为一个"标准对". 设棋盘中 "标准对" 的个数为 $S$. 试问: $S$ 是奇数还是偶数? 由哪些方格的颜色确定? 什么情况下 $S$ 为奇数? 什么情况下 $S$ 为偶数? 说明理由.
%%<SOLUTION>%%
解:把所有方格分为 3 类.
第一类方格位于棋盘的四个角上,第二类方格位于棋盘的边界 (不包括四个角)上,其余方格为第三类.
将所有红色方格填上数 1 , 所有蓝色方格填上数 -1 . 记第一类方格的填数分别为 $a, b, c, d$. 第二类方格的填数分别为 $x_1, x_2, \cdots, x_{2 m+2 n-8}$. 第三类方格的填数为 $y_1, y_2, \cdots, y_{(n-2)(m-2)}$. 对任何两个相邻的方格, 在它们的公共边上标上这两个方格内标数的乘积.
设所有公共边上的标数的积为 $H$.
对每个第一类方格, 它有两个邻格, 所以它的标数在 $H$ 中出现两次, 对每个第二类方格, 它有 3 个邻格, 所以它的标数在 $H$ 中出现 3 次, 对于第三类方格, 它有 4 个邻格, 它的标数在 $H$ 中出现 4 次,于是
$$
\begin{aligned}
H & =(a b c d)^2\left(x_1 x_2 \cdots x_{2 m+2 n-8}\right)^3\left(y_1 y_2 \cdots y_{(n-2)(m-2)}\right)^4 \\
& =x_1 x_2 \cdots x_{2 m+2 n-8} .
\end{aligned}
$$
当 $x_1 x_2 \cdots x_{2 m+2 n-8}=1$ 时, $H=1$, 此时有偶数个标准对, 当 $x_1 x_2 \cdots x_{2 m+2 n-8}=-1$ 时, $H=-1$, 此时有奇数个标准对, 这表明: $S$ 的奇偶性由第二类格的颜色确定.
当第二类格中有奇数个蓝色格时, $S$ 为奇数, 当第二类格中有偶数个蓝色格时, $S$ 为偶数.
%%PROBLEM_END%%


