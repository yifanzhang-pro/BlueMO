
%%TEXT_BEGIN%%
组合计数问题是高中数学竟赛中最常见的一类问题,解组合计数问题的基本方法有以下几种:
1. 枚举法(第五讲例 1 的后面一部分);
2. 利用基本计数原理及基本公式(第一讲中例 1 9, 第五讲中例 1);
3. 配对法、映射方法和一般对应方法(第六讲例 1 4, 8 15);
4. 算二次方法 (第七讲例 1 2, 4 7);
5. 递推方法 (第八讲例 $1 \sim 8$ );
6. 利用容斥原理(第一讲例 $10 \sim 12$ );
7. 利用不定方程 $x_1+x_2+\cdots+x_k=n$ 的非负整数解组的公式(第一讲例 7 的解法二,例 $8 \sim 9$ ).
对于某些较复杂的组合计数问题, 有时要联合使用几种计数方法才能解决, 而有些计数问题则分别可用几种不同的计数方法给出几种不同的解答.
%%TEXT_END%%



%%PROBLEM_BEGIN%%
%%<PROBLEM>%%
例1. 有 6 个红球, 3 个蓝球, 3 个黄球.
将这些球放在一条直线上.
假设同色球没有区别,试问: 有多少种不同的放法使得同色球不相邻?
%%<SOLUTION>%%
解:因为有 6 个红球和 6 个非红球, 且任意两个红球之间至少有一个非红球,所以一个蓝球与一个黄球相邻最多出现一次.
若蓝球与黄球不相邻, 则红球和非红球交替放着, 这时红球的放法只有 2 种 (即最左端为红球或非红球), 蓝球有 $\mathrm{C}_6^3$ 种放法, 黄球有 $\mathrm{C}_3^3$ 种放法), 不同的放法有 $2 \mathrm{C}_6^3 \mathrm{C}_3^3=40$ 种.
若有一个蓝球和一个黄球相邻, 可将其合并看成一个 "花球", 则非红球由 2 个蓝球、 2 个黄球和一个花球组成, 于是任意两个相邻红球之间将有一个非红球, 这样的放法有 $\frac{5 !}{2 ! 2 ! 1 !}=30$ (种). 又因花球包含 2 种不同的放法, 所以有一个蓝球与一个黄球相邻的放法有 $2 \times 30=60$ 种.
综上,满足题目条件的放法共有 $40+60=100$ 种.
%%PROBLEM_END%%



%%PROBLEM_BEGIN%%
%%<PROBLEM>%%
例2. 从给定的 6 种不同颜色中选用若干种颜色, 将一个正方体的 6 个面染色,每面恰染一色, 具有公共棱的两个面不同色, 则不同的染色方案有种.
(约定经过翻滚或旋转可以变相同的染色方案是相同的染色方案) 
%%<SOLUTION>%%
解:因有公共顶点的三个面互不同色,故至少要用 3 色,下分 4 种情形.
(1)6种颜色都用时,先将染某种固定颜色的面朝上, 从剩下 5 色中取 1 色染下底面有 $\mathrm{C}_5^1$ 种方法,余下 4 色染余下的 4 个侧面 (应是 4 种颜色的圆排列) 有 $(4-1)$ ! 种染法, 所以 6 种颜色都用时, 染色方案有 $\mathrm{C}_5^1 \cdot(4-1)$ ! = 30 种.
(2)只用 5 种颜色时, 从 6 色中取 5 色有 $\mathrm{C}_6^5$ 种方法, 这时必有一组对面同色, 从 5 色中取 1 色染一组对面, 并将它们朝上和朝下, 有 $\mathrm{C}_5^1$ 种方法, 余下 4 色染余下的 4 个侧面 (应是 4 粒不同颜色珠子的项链), 有 $\frac{1}{2} \cdot(4-1)$ ! 种方法, 所以只用 5 种颜色时, 染色方案有 $\mathrm{C}_6^5 \cdot \mathrm{C}_5^1 \cdot \frac{1}{2} \cdot(4-1) !=90$ 种.
(3)只用 4 种颜色时, 从 6 色中取 4 色有 $\mathrm{C}_6^4$ 种方法, 这时必有 2 组对面同色, 另一组对面不同色, 从 4 色中取 2 色染一组对面, 并将它们朝上和朝下 (这时上、下底面无区别)有 $\mathrm{C}_4^2$ 种方法, 其余 2 色染余下 4 个侧面且使 2 组对面同色 (应是 2 粒不同颜色珠子的项链) 只有 1 种染法, 故只用 4 色时, 染色方案有 $\mathrm{C}_6^4 \mathrm{C}_4^2 \cdot 1=90$ 种.
(4)只用 3 色时, 从 6 色中取 3 色有 $\mathrm{C}_6^3$ 种方法, 这时 3 组对面同色, 用 3 种颜色去给它们染色只有 1 种染法, 所以只用 3 色时, 染色方案有 $\mathrm{C}_6^3 \cdot 1=20$ 种.
综上可知,不同的染色方案共有 $30+90+90+20=230$ 种.
%%PROBLEM_END%%



%%PROBLEM_BEGIN%%
%%<PROBLEM>%%
例3. 凸 $n$ 边形的任意 3 条对角线不交于形内一点, 求这些对角线将凸 $n$ 边形分成的区域的个数.
%%<SOLUTION>%%
解法一设凸 $n$ 边形被对角线分成的区域中边数最多的为 $m$ 边形, 其中三角形有 $n_3$ 个, 四边形有 $n_4$ 个, $\cdots, m$ 边形有 $n_m$ 个, 于是凸 $n$ 边形被分成的区域个数为
$$
S=n_3+n_4+\cdots+n_m .
$$
一方面各区域的顶点数的总和为 $3 n_3+4 n_4+\cdots+m m_m$, 另一方面, 对角线在形内有 $\mathrm{C}_n^4$ 个交点,每个交点是 4 个区域的公共顶点 (因无 3 条对角线交于形内一点), 又凸 $n$ 边形有 $n$ 个顶点, 每个顶点是 $n-2$ 个区域的公共顶点, 所以
$$
3 n_3+4 n_4+\cdots+m m_m=4 \mathrm{C}_n^4+n(n-2) . \label{eq1}
$$
其次, 各区域的内角总和为 $n_3 \cdot 180^{\circ}+n_4 \cdot 360^{\circ}+\cdots+n_m \cdot(m-2) \cdot 180^{\circ}$,
另一方面, 原凸 $n$ 边形的 $n$ 个顶点处的内角和为 $(n-2) \cdot 180^{\circ}$, 又 $\mathrm{C}_n^4$ 个对角线交点处的内角和为 $\mathrm{C}_n^4 \cdot 360^{\circ}$, 所以
$$
\begin{aligned}
& n_3 \cdot 180^{\circ}+n_4 \cdot 360^{\circ}+\cdots+n_m \cdot(m-2) \cdot 180^{\circ} \\
= & (n-2) \cdot 180^{\circ}+\mathrm{C}_n^4 \cdot 360^{\circ},
\end{aligned}
$$
即
$$
n_3+2 n_4+\cdots+(m-2) n_m=2 \mathrm{C}_n^4+(n-2), \label{eq2}
$$
$[式\ref{eq1} - \ref{eq2}] \div 2$, 得
$$
\begin{aligned}
S & =n_3+n_4+\cdots+n_m=\mathrm{C}_n^4+\frac{1}{2}(n-1)(n-2) \\
& =\frac{1}{24}(n-1)(n-2)\left(n^2-3 n+12\right) .
\end{aligned}
$$
%%PROBLEM_END%%



%%PROBLEM_BEGIN%%
%%<PROBLEM>%%
例3. 凸 $n$ 边形的任意 3 条对角线不交于形内一点, 求这些对角线将凸 $n$ 边形分成的区域的个数.
%%<SOLUTION>%%
解法二 每去掉一条对角线, 则区域减少的个数为 $a_i+1$, 这是 $a_i$ 是该对角线与还没有去掉的对角线在形内的交点数, 逐步将 $\mathrm{C}_n^2-n$ 条对角线去掉, 最后, 剩下一个区域,故所求区域数为 $S=1+\sum_{i=1}^{C_n^2-n}\left(a_i+1\right)$, 而 $\sum_{i=1}^{C_n^2 \rightarrow n} a_i$ 恰等于对角线在形内的交点数 $\mathrm{C}_n^4$, 所以
$$
S=1+\mathrm{C}_n^4+\mathrm{C}_n^2-n=\frac{1}{24}(n-1)(n-2)\left(n^2-3 n+12\right) .
$$
%%PROBLEM_END%%



%%PROBLEM_BEGIN%%
%%<PROBLEM>%%
例3. 凸 $n$ 边形的任意 3 条对角线不交于形内一点, 求这些对角线将凸 $n$ 边形分成的区域的个数.
%%<SOLUTION>%%
解法三 设凸 $n$ 边形被对角线分成的区域数为 $S$,加上凸 $n$ 边形外的区域, 共有 $F=S+1$ 个区域, 相当一个凸 $F$ 面体, 它的顶点数为 $V=\mathrm{C}_n^4+n$, 又设它的边数为 $E$, 则从凸 $n$ 边形内每一交点处出发有 4 条边, 共 $4 \mathrm{C}_n^4$ 条边, 另外, 从凸 $n$ 边形每一顶点出发有 $n-3$ 条对角线上的线段, 故从 $n$ 个顶点出发共有 $n(n-3)$ 条线段,但以上计算每条线段计算了 2 次, 故对角线被交点分成的线段数为
$$
\frac{1}{2}\left[4 \mathrm{C}_n^4+n(n-3)\right]=2 \mathrm{C}_n^4+\frac{1}{2} n(n-3),
$$
加上凸 $n$ 边形的 $n$ 条边得
$$
\begin{aligned}
E & =2 \mathrm{C}_n^4+\frac{1}{2} n(n-3)+n \\
& =2 \mathrm{C}_n^4+\mathrm{C}_n^2 .
\end{aligned}
$$
代入欧拉(Euler)公式 $V+F-E=2$ 得
$$
\begin{aligned}
S=F-1 & =E-V+1 \\
& =2 \mathrm{C}_n^4+\mathrm{C}_n^2-\mathrm{C}_n^4-n+1
\end{aligned}
$$
$$
=\frac{1}{24}(n-1)(n-2)\left(n^2-3 n+12\right) .
$$
%%PROBLEM_END%%



%%PROBLEM_BEGIN%%
%%<PROBLEM>%%
例3. 凸 $n$ 边形的任意 3 条对角线不交于形内一点, 求这些对角线将凸 $n$ 边形分成的区域的个数.
%%<SOLUTION>%%
解法四设凸 $n$ 边形被对角线分成的区域数为 $a_n$,于是 $a_3=1, a_4=4$. 在凸 $n-1$ 边形 $P_1 P_2 \cdots P_{n-1}$ 基础上增加一个顶点 $P_n$ 得凸 $n$ 边形 $P_1 P_2 \cdots P_n$, 则增加 1 个区域 $\triangle P_1 P_{n-1} P_n$, 再增加从 $P_n$ 出发的 $n-3$ 条对角线, 则增加的区域数应为这 $n-3$ 条对角线上的交点数 $\mathrm{C}_n^4-\mathrm{C}_{n-1}^4$ 加上 $n-3$. 故得
$$
\begin{aligned}
a_n & =a_{n-1}+\left(\mathrm{C}_n^4-\mathrm{C}_{n-1}^4\right)+(n-3)+1 \\
& =a_{n-1}+\left(\mathrm{C}_n^4-\mathrm{C}_{n-1}^4\right)+n-2(n \geqslant 5) .
\end{aligned}
$$
如果约定 $a_2=0, \mathrm{C}_2^4=\mathrm{C}_3^4=0$, 则上式当 $n=3,4$ 时也成立.
所以
$$
\begin{aligned}
a_n & =a_2+\sum_{k=3}^n\left(a_k-a_{k-1}\right) \\
& =0+\sum_{k=3}^n\left[\left(\mathrm{C}_k^4-\mathrm{C}_{k-1}^4\right)+(k-2)\right] \\
& =\mathrm{C}_n^4+\frac{1}{2}(n-1)(n-2) \\
& =\frac{1}{24}(n-1)(n-2)\left(n^2-3 n+12\right) .
\end{aligned}
$$
%%PROBLEM_END%%



%%PROBLEM_BEGIN%%
%%<PROBLEM>%%
例4. 设 $a_1 a_2 a_3 a_4 a_5$ 是 $1,2,3,4,5$ 的排列, 满足: 当 $i=1,2,3,4$ 时, $a_1 a_2 \cdots a_i$ 都不是 $1,2, \cdots, i$ 的排列, 求这种排列的个数.
%%<SOLUTION>%%
解法一显然 $a_1 \neq 1$.
(1) $a_1=5$ 时, $a_2 a_3 a_4 a_5$ 是 $1,2,3,4$ 的任何排列均满足题目要求, 这时排列有 4 ! 个;
(2) $a_1=4$ 时,一共有 4 ! 个排列, 其中形如 $4 \times \times \times 5$ 的 3 ! 个排列不满足要求, 故这时满足要求的排列有 $4 !-3$ !个;
(3) $a_1=3$ 时,一共有 4 ! 个排列, 其中形如 $3 \times \times \times 5$ 和 $3 \times \times 54$ 的排列不满足要求,故这时满足要求的排列个数为 $4 !-3 !-2 !$;
(4) $a_1=2$ 时,一共有 4 ! 个排列, 其中形如 $215 \times \times, 21453,2 \times \times 54,2 \times \times \times 5$ 的排列不满足要求, 故这时满足要求的排列有 $4 !-2 !-1- 2 !-3 !$ 个.
综上可得, 满足要求的排列共有
$$
\begin{aligned}
& 4 !+(4 !-3 !)+(4 !-3 !-2 !)+(4 !-2 !-1-2 !-3 !) \\
= & 24+18+16+13=71 \text { (个). }
\end{aligned}
$$
%%PROBLEM_END%%



%%PROBLEM_BEGIN%%
%%<PROBLEM>%%
例4. 设 $a_1 a_2 a_3 a_4 a_5$ 是 $1,2,3,4,5$ 的排列, 满足: 当 $i=1,2,3,4$ 时, $a_1 a_2 \cdots a_i$ 都不是 $1,2, \cdots, i$ 的排列, 求这种排列的个数.
%%<SOLUTION>%%
解法二设 $1,2,3,4,5$ 的所有排列组成的集合为 $S$, 令
$$
\begin{aligned}
& A_i=\left\{\left(a_1, a_2, a_3, a_4, a_5\right) \mid\left(a_1, a_2, a_3, a_4, a_5\right) \in S,\right. \\
& \text { 且 } \left.\left(a_1, a_2, \cdots, a_i\right) \text { 是 }(1,2, \cdots, i) \text { 的排列 }\right\}(1 \leqslant i \leqslant 4)
\end{aligned}
$$
于是所求排列个数为 $\left|\complement_S A_1 \cap \complement_S A_2 \cap \complement_S A_3 \cap \complement_S A_4\right|$, 易知
$$
\begin{aligned}
& |S|=5 !,\left|A_1\right|=4 !,\left|A_2\right|=2 ! \cdot 3 !,\left|A_3\right|=3 ! \cdot 2 !, \\
& \left|A_4\right|=4 !,\left|A_1 \cap A_2\right|=3 !,\left|A_1 \cap A_3\right|=2 ! 2 !,\left|A_1 \cap A_4\right|=3 !, \\
& \left|A_2 \cap A_3\right|=2 ! \cdot 2 !,\left|A_2 \cap A_4\right|=2 ! \cdot 2 !,\left|A_3 \cap A_4\right|=3 !, \\
& \left|A_1 \cap A_2 \cap A_3\right|=2 !,\left|A_1 \cap A_2 \cap A_4\right|=2 !,\left|A_1 \cap A_3 \cap A_4\right|=2 !, \\
& \left|A_2 \cap A_3 \cap A_4\right|=2 !,\left|A_1 \cap A_2 \cap A_3 \cap A_4\right|==1 .
\end{aligned}
$$
故由容斥原理可得符合条件的排列个数为
$$
\begin{aligned}
& \left|\complement_S A_1 \cap \complement_S A_2 \cap \complement_S A_3 \cap \complement_S A_4\right| \\
= & |S|-\sum_{i=1}^4\left|A_i\right|+\sum_{1 \leqslant i<j \leqslant 4}\left|A_i \cap A_j\right|-\sum_{1 \leqslant i<j<k \leqslant 4}\left|A_i \cap A_j \cap A_k\right|+ \\
& \left|A_1 \cap A_2 \cap A_3 \cap A_4\right| \\
= & 5 !-(4 !+2 ! \cdot 3 !+3 ! \cdot 2 !+4 !)+(3 !+2 ! \cdot 2 !+3 !+2 ! \cdot 2 !+2 ! \cdot 2 ! \\
& +3 !)-(2 !+2 !+2 !+2 !)+1 \\
= & 120-72+30-8+1=71 \text { (个). }
\end{aligned}
$$
%%<REMARK>%%
注:$\left|A_1 \cap A_3\right|=2 ! \cdot 2$ ! 的理由是 $A_1 \cap A_3$ 中只含形如 $1 \times \times 45$ 和 $1 \times \times 54$ 的排列, $\left|A_1 \cap A_4\right|=3$ ! 的理由是 $A_1 \cap A_4$ 中只含形如 $1 \times \times \times 5$ 的排列,其他各式可类似得出.
%%PROBLEM_END%%



%%PROBLEM_BEGIN%%
%%<PROBLEM>%%
例5. 对每个正整数 $n$, 求集合 $\{1,2, \cdots, n\}$ 的排列 $\left(a_1, a_2, \cdots, a_n\right)$ 的数目, 其中 $a_1, a_2, \cdots, a_n$ 满足对于 $k=1,2, \cdots, n$, 均有 $2\left(a_1+a_2+\cdots+a_k\right)$ 可以被 $k$ 整除.
%%<SOLUTION>%%
解:对于每个正整数 $n$, 设 $x_n$ 为集合 $\{1,2, \cdots, n\}$ 满足条件的排列的数目,并称这些排列是 "好的". 对 $n=1,2,3$, 易知每个排列都是好的, 因此, $x_1=1, x_2=2 !=2, x_3=3 !=6$.
对于 $n>3$, 考虑 $\{1,2, \cdots, n\}$ 的一个好的排列 $\left(a_1, a_2, \cdots a_n\right)$, 由好的排列的定义知 $n-1$ 应是
$$
\begin{aligned}
2\left(a_1+a_2+\cdots+a_{n-1}\right) & =2\left(1+2+\cdots+n-a_n\right) \\
& =n(n+1)-2 a_n=(n-1)(n+2)-\left(2 a_n-2\right)
\end{aligned}
$$
的因数, 因此, $2 a_n-2$ 可以被 $n-1$ 整除.
而 $0 \leqslant 2 a_n-2 \leqslant 2(n-1)$, 于是, $2 a_n -2=0$ 或 $n-1$,或 $2(n-1)$. 即 $a_n=1$ 或 $\frac{n+1}{2}$ 或 $n$.
(1) 若 $a_n=\frac{n+1}{2}$, 则 $n-2$ 应是
$$
\begin{aligned}
2\left(a_1+a_2+\cdots+a_{n-2}\right) & =2\left[(1+2+\cdots+n)-a_n-a_{n-1}\right] \\
& =n(n+1)-(n+1)-2 a_{n-1} \\
& =(n-2)(n+2)-\left(2 a_{n-1}-3\right)
\end{aligned}
$$
的因数,因此, $2 a_{n-1}-3$ 可以被 $n-2$ 整除, 而 $-1 \leqslant 2 a_n-3 \leqslant 2 n-3$,于是 $2 a_n-3=0$ 或 $n-2$ 或 $2 n-4$. 由于 $2 a_n-3$ 是奇数,故它不可能等于 0 或 $2 n-4$, 所以 $2 a_n-3=n-2$, 即 $a_{n-1}=\frac{n+1}{2}=a_n$ 矛盾;
(2) 若 $a_n=n$, 则 $\left(a_1, a_2, \cdots, a_{n-1}\right)$ 是 $\{1,2, \cdots, n-1\}$ 的好的排列, 这样的好的排列有 $x_{n-1}$ 个, 每一个这样的好排列的后面加上 $n$, 使得到 $\{1$, $2, \cdots, n\}$ 的一个满足 $a_n=n$ 的好的排列 $\left\{a_1, a_2, \cdots, a_n\right\}$, 这种对应是一一对应, 故满足 $a_n=n$ 的好的排列 $\left\{a_1, a_2, \cdots, a_n\right\}$ 有 $x_{n-1}$ 个;
(3) 若 $a_n=1$, 则 $\left\{a_1-1, a_2-1, \cdots, a_{n-1}-1\right\}$ 是 $\{1,2, \cdots, n-1\}$ 的好的排列, 这是因为对任意 $k \leqslant n-1$, 有
$$
2\left[\left(a_1-1\right)+\left(a_2-1\right)+\cdots+\left(a_k-1\right)\right]=2\left(a_1+\cdots+a_k\right)-2 k
$$
被 $k$ 整除.
因此, $\{1,2, \cdots, n-1\}$ 的 $x_{n-1}$ 个好的排列中任意一个 $\left(b_1, b_2, \cdots\right.$, $\left.b_{n-1}\right)$ 对应于 $\{1,2, \cdots, n\}$ 的一个好排列 $\left\{b_1+1, b_2+1, \cdots, b_{n-1}+1,1\right\}$, 这种对应是一一对应,故满足 $a_n=1$ 的好的排列有 $x_{n-1}$ 个.
综上知 $x_n=2 x_{n-1}(n \geqslant 4)$. 由 $x_3=6$, 得当 $n \geqslant 4$ 时, $x_n=3 \times 2^{n-2}$, 故满足题目条件的排列个数为 $x_n= \begin{cases}n & (n=1,2), \\ 3 \times 2^{n-2} & (n \geqslant 3) .\end{cases}$
%%PROBLEM_END%%



%%PROBLEM_BEGIN%%
%%<PROBLEM>%%
例6. 设 $M$ 为平面上坐标为 $(p \times 1994,7 p \times 1994)$ 的点, 其中 $p$ 为素数, 求满足下列条件的直角三角形的个数:
(1)三角形的顶点都是整点, $M$ 是直角顶点;
(2) 三角形的内心是坐标原点.
%%<SOLUTION>%%
分析:如图(<FilePath:./figures/fig-c13i1.png>), 直角三角形 $M A B$ 满足条件 (1),(2), 其内切圆半径为 $r$, 过 $O$ 分别作 $M A$, $M B$ 的垂线, 垂足分别为 $C, D$, 则易知 $O C= O D=M C=M D=r=\frac{1}{\sqrt{2}} O M$, 故 Rt $\triangle M A B$ 由 $B D=u$ 和 $A C=v$ 唯一确定.
为此, 我们只要根据条件 (1),(2) 建立 $u, v$ 满足的不定方程 (组), 再通过不定方程 (组) 的解数的计算就可得出所求直角三角形的个数.
这样就通过建立对应关系, 将一个复杂的不熟悉的问题转化为一个熟知的较易的问题来求解.
解法一为了计算方便, 将坐标原点平移到 $M$, 建立新的直角坐标系 $x^{\prime} M y^{\prime}$, 于是在新的坐标系中, $M$ 的坐标为 $(0,0)$, 内心 $O$ 的坐标为 $(-p \times 1994,-7 p \times 1994)$. 所以
$$
\begin{aligned}
|O M| & =\sqrt{(p \times 1994)^2+(7 p \times 1994)^2} \\
& =p \times 1994 \times 5 \sqrt{2} .
\end{aligned}
$$
设 $\triangle M A B$ 符合条件,其内切圆半径为 $r$,过 $O$ 分别作 $M A$ 和 $M B$ 的垂线, 垂足分别为 $C$ 和 $D$, 设 $A M$ 的斜率为 $k$, 易知 $O M$ 的斜率为
$$
k^{\prime}=\frac{7 p \times 1994}{p \times 1994}=7,
$$
注意到 $O$ 为 Rt $\triangle M A B$ 的内心, $\angle O M A=45^{\circ}$, 故由两直线夹角公式得
$$
1=\tan 45^{\circ}=\frac{k^{\prime}-k}{1+k k^{\prime}}=\frac{7-k}{1+7 k},
$$
所以 $k=\frac{3}{4}$, 而 $M B \perp M A$, 故 $M B$ 的斜率为 $k_1=-\frac{4}{3}$. 由此, 可设 $A, B$ 的坐标为 $A(-4 t,-3 t), B\left(3 t^{\prime},-4 t^{\prime}\right)$, 因 $(3,4)=1$, 且 $A, B$ 为整点, 故 $t$, $t^{\prime}$ 皆为正整数, 于是 $M A=5 t, M B=5 t^{\prime}$, 并且 $M C=M D=r=M O \cos 45^{\circ}= p \times 1994 \times 5$.
$$
\text { 记 } B D=u, A C=v \text {, 则 } u=M B-M D=5\left(t^{\prime}-p \times 1994\right), v=
$$
$M A-M C=5(t-p \times 1994)$. 记 $\angle O B D=\alpha, \angle O A C=\beta$, 易知 $\alpha+\beta= \frac{1}{2}(\angle M A B+\angle M B A)=45^{\circ}, \tan \alpha=\frac{r}{u}, \tan \beta=\frac{r}{v}$ 且 $\frac{r}{u}=\tan \alpha= \tan \left(45^{\circ}-\beta\right)=\frac{1-\tan \beta}{1+\tan \beta}=\frac{v-r}{v+r}$, 所以 $u=\frac{r(v+r)}{v-r}$ 记 $v-r=m, \frac{2 r^2}{m}=n$, 则 $u=\frac{r(m+2 r)}{m}=r+n$. 可见 $m$ 及 $n$ 均为 5 的正整数倍, 且
$$
m n=2 r^2=2(p \times 1994 \times 5)^2=2^3 \times 5^2 \times 997^2 \times p^2 .
$$
因为对于 $m, n$ 的一组正整数解 $(m, n)$, 可求出唯一一组 $u, v$ 的正整数解 $(u, v)=(n+r, m+r)$, 从而唯一确定一个符合条件的 Rt $\triangle M A B$, 反之亦真.
故符合条件的 Rt $\triangle M A B$ 的个数 $S$ 等于不定方程 (1)的满足条件 $5 \mid m$ 和 $5 \mid n$ 的正整数解组 $(m, n)$ 的组数.
1. 若 $p \neq 2, p \neq 997$, 则 $\frac{m}{5} \cdot \frac{n}{5}=2^3 \times p^2 \times 997^2$ 的一切解为
$$
\left\{\begin{array}{l}
\frac{m}{5}=2^i \times p^j \times 997^k \\
\frac{n}{5}=2^{3-i} \times p^{2-j} \times 997^{2-k}
\end{array} \quad\left(\begin{array}{l}
i=0,1,2,3 \\
j=0,1,2 \\
k=0,1,2
\end{array}\right),\right.
$$
共有 $4 \times 3 \times 3=36$ 组解.
2. 若 $p=997$, 则 $\frac{m}{5} \cdot \frac{n}{5}=2^3 \times 997^4$ 的一切解为
$$
\left\{\begin{array}{ll}
\frac{m}{5} & =2^i \times 997^j \\
\frac{n}{5} & =2^{3-i} \times 997^{4-j}
\end{array} \quad\left(\begin{array}{l}
i=0,1,2,3 \\
j=0,1,2,3,4
\end{array}\right),\right.
$$
共有 $4 \times 5=20$ 组解.
3. 若 $p=2$, 则 $\frac{m}{5} \cdot \frac{n}{5}=2^5 \times 997^2$ 的一切解为
$$
\left\{\begin{array}{ll}
\frac{m}{5}=2^i \times 997^j \\
\frac{n}{5}=2^{5-i} \times 997^{2-j}
\end{array} \quad\left(\begin{array}{l}
i=0,1,2,3,4,5 \\
j=0,1,2
\end{array}\right),\right.
$$
共有 $6 \times 3=18$ 组解.
故符合条件的三角形个数为
$$
S=\left\{\begin{array}{l}
36, p \neq 2 \text { 且 } p \neq 997 \text { 时, } \\
20, p=997 \text { 时, } \\
18, p=2 \text { 时.
}
\end{array}\right.
$$
%%PROBLEM_END%%



%%PROBLEM_BEGIN%%
%%<PROBLEM>%%
例6. 设 $M$ 为平面上坐标为 $(p \times 1994,7 p \times 1994)$ 的点, 其中 $p$ 为素数, 求满足下列条件的直角三角形的个数:
(1)三角形的顶点都是整点, $M$ 是直角顶点;
(2) 三角形的内心是坐标原点.
%%<SOLUTION>%%
解法二记号同解法一, 并且同解法一可得 $M A=5 t, M B=5 t^{\prime}\left(t, t^{\prime}\right.$ 为正整数), 且 $r=p \times 1994 \times 5$, 记 $r_0=\frac{r}{5}=2 \times 997 \times p(2,997, p$ 皆为素数 $)$ 于是 $A B=\sqrt{M A^2+M B^2}=5 \sqrt{t^2+t^{\prime 2}}$. 由平面几何知识, 易知 $A B=M A+ M B-2 r$, 即 $5 \sqrt{t^2+t^{\prime 2}}=5 t+5 t^{\prime}-10 r_0$, 两边除以 5 后, 再平方并经过整理可得
$$
\left(t-2 r_0\right)\left(t^{\prime}-2 r_0\right)=2 r_0{ }^2=2^3 \times 997^2 \times p^2 .
$$
记 $m_0=t-2 r_0, n_0=t^{\prime}-2 r_0$, 则
$$
m_0 n_0=2^3 \times 997^2 \times p^2 .
$$
易知所求 Rt $\triangle M A B$ 的个数 $S$ 等于(2) 的正整数解组 $\left(m_0, n_0\right)$ 的个数, 同解法一可求得
$$
S=\left\{\begin{array}{l}
36, p \neq 2 \text { 且 } p \neq 997, \\
20, p=997, \\
18, p=2 .
\end{array}\right.
$$
%%PROBLEM_END%%


