
%%TEXT_BEGIN%%
在组合数学的一些问题的证明中常常要用到构造方法, 构造方法可分为直接构造和归纳构造两大类, 当直接构造一个符合条件的对象有困难时, 可考虑从下面几个方面人手.
(1)分析要构造的对象应具有的一些结构,再根据这些结构来构造满足所有条件的对象 (组合分析法)
(2)先构造满足一部分条件的一些部件,再由这些部件来组成满足所有条件的对象 (部件组成法)
(3)先去掉一部分条件,构造满足其余条件的对象,再逐步调整使之满足所有条件(逐步调整法)
如果要构造的对象与正整数 $n$ 有关, 直接构造它又比较困难, 那么可考虑用归纳法去构造它.
%%TEXT_END%%



%%PROBLEM_BEGIN%%
%%<PROBLEM>%%
例1. 在平面直角坐标系中是否存在无穷多个圆组成的集合 $M_0$, 满足
(1) $M_0$ 中任意两个圆至多有一个公共点;
(2) $x$ 轴上每一个有理点都在 $M_0$ 中某个圆上.
%%<SOLUTION>%%
解:对 $x$ 轴上任意有理点 $(r, 0)\left(r=\frac{p}{q}, p 、 q\right.$ 为互素整数, $\left.q>0\right)$, 在 $x$ 轴上方作一个半径为 $R_r>0$ 的圆与 $x$ 轴切于点 $(r, 0)$, 并记这个圆为 $C_r\left(R_r\right)$, 所有这些圆组成的集合记为 $M_0$, 显然 $M_0$ 满足条件 (2), 为了使 $M_0$ 满足条件 (1), 我们来分析 $R_r$ 的值应该是多少.
在 $M_0$ 中任取两个圆 $C_{r_i}\left(R_{r_i}\right)(i=1,2)$, 因为两个圆都在 $x$ 轴上方并且都与 $x$ 轴相切, 所以它们相交的充要条件是圆心距小于两圆半径之和, 即
$$
\sqrt{\left(r_2-r_1\right)^2+\left(R_{r_2}-R_{r_1}\right)^2}<R_{r_2}+R_{r_1},
$$
这等价于 $\left(r_1-r_2\right)^2<4 R_{r_1} R_{r_2}$. 令 $r_i=\frac{p_i}{q_i}\left(p_i, q_i\right.$ 为互素整数且 $q_i>0, i= 1,2)$, 得
$$
\left(p_1 q_2-p_2 q_1\right)^2<4 q_1^2 q_2^2 R_{r_1} R_{r_2} .
$$
若取 $R_{r_i}=\frac{1}{k q_{r_i}^2}(k \geqslant 2)$, 则上式化为
$$
\left(p_1 q_2-p_2 q_1\right)^2<\frac{4}{k^2} \leqslant 1
$$
当 $r_1 \neq r_2$ 时, 上式左端为正整数, 矛盾.
可见, 只要取 $R_r=\frac{1}{k q^2}(k \geqslant 2)$, 所作圆集合 $M_0$ 就满足题目的所有条件.
于是我们证明了存在无穷多个圆组成的集合
$$
M_0=\left\{C_r\left(R_r\right) \mid r=\frac{p}{q}, R_r=\frac{1}{k q^2}(k \geqslant 2), p, q \text { 为互素整数, } q>0\right\} .
$$
满足题目条件 (1)、(2), 显然这样的 $M_0$ 有无穷多个.
%%PROBLEM_END%%



%%PROBLEM_BEGIN%%
%%<PROBLEM>%%
例2. 求证能否将正整数集合 $\mathbf{N}_{+}$分划为两个不相交的集合 $A$ 和 $B$, 满足:
(1) $A$ 中任意三个数不成等差数列;
(2) 不能由 $B$ 中元素组成一个非常数的无穷等差数列.
%%<SOLUTION>%%
分析一设 $A=\left\{a_1, a_2, a_3, \cdots\right\}\left(a_1<a_2<a_3<\cdots\right)$ 及 $B=\mathbf{N}_{+} \backslash A$ 满足题目条件.
因为若三个 $a, b, c(a<b<c)$ 成等差数列, 则 $2 b=a+c>c$, 故只要 $a_{i+1} \geqslant 2 a_i(i=1,2,3, \cdots)$, 则 $A$ 中任意三个数不成等差数列, 要构造这样的 $A$ 是不困难的.
为了使 $B=\mathbf{N}_{+} \backslash A$ 满足条件 (2), 只要满足对任意 $a, d \in \mathbf{N}_{+}$, 等差数列 $\{a+n d\}(n=0,1,2, \cdots)$ 中至少有一项属于 $A$.
解法一将首项为 $a$, 公差为 $d$ 的无穷等差数列用 $(a, d)$ 表示.
易将所有正整数无穷等差数列 (非常数列) "排序" 如下: $(1,1),(1,2),(2,1)$, $(1,3),(2,2),(3,1) \cdots$, 排序的规律是:先看 $a+d$ 的大小,小者排前, $a+d$ 相等的, $a$ 较小的排前, 按下列方式构造数列 $a_1, a_2, a_3, \cdots$. 设 $a_1=1$, 如 $a_1$, $a_2, \cdots, a_n$ 已取出,则在第 $n+1$ 个等差数列中取大于 $2 a_n$ 的某一项为 $a_{n+1}$.
令 $A=\left\{a_1, a_2, \cdots, a_n, \cdots\right\}$, 则因 $a_{n+1}>2 a_n(n=1,2, \cdots)$, 故 $A$ 中任何三个数不成等差数列, 再令 $B=\mathbf{N}_{+} \backslash A$, 则因为任何正整数的非常数列的无穷等差数列都有一项属于 $A$, 故 $B$ 中没有非常数列的无穷等差数列.
于是存在满足题目条件的集合 $A$ 和 $B$.
%%PROBLEM_END%%



%%PROBLEM_BEGIN%%
%%<PROBLEM>%%
例2. 求证能否将正整数集合 $\mathbf{N}_{+}$分划为两个不相交的集合 $A$ 和 $B$, 满足:
(1) $A$ 中任意三个数不成等差数列;
(2) 不能由 $B$ 中元素组成一个非常数的无穷等差数列.
%%<SOLUTION>%%
分析二同分析一知, 易构造满足条件 (1) 的 $A$, 为了使 $B=\mathbf{N}_{+} \backslash A$ 满足条件 (2), 只须满足对任意 $a, d \in \mathbf{N}_{+}$, 无穷等差数列 $\{a+r d\}(r=0,1,2$, …) 中至少有一项属于 $A$, 这只要 $A$ 中每一个数可写成 $a_i=b_i+c_i$ 的形式, 使得对任意 $a, d \in \mathbf{N}_{+}, b_i$ 可无穷多次取到 $a$ 的值, 且存在某个等于 $a$ 的 $b_i$, 它所对应的 $c_i$ 是任意 $d$ 的倍数.
这只要 $A$ 中含有无穷多个形如 $a+m$ ! 的数即可.
于是, 存在正整数 $m \geqslant d$, 取 $r_0=\frac{m !}{d}$, 则 $a+m !=a+r_0 d$ 为无穷等差数列 $\{a+r d\}$ 中一项.
解法二令 $A=A_1 \cup A_2 \cup \cdots \cup A_n \cup \cdots$, 其中 $A_1=\{1 !+1\}, A_2= \{2 !+1,3 !+2\}, A_3=\{4 !+1,5 !+2,6 !+3\}, \cdots, A_n=\{m !+1, m !+ 2, \cdots,(m+n-1) !+n\}$ (其中 $\left.m=\frac{1}{2} n(n-1)+1\right), \cdots, B=\mathbf{N}_{+} \backslash A$, 则将 $A$ 中数从小到大排列时, 从第二项起, 每一项大于前面一项的 2 倍, 故 $A$ 中任意三个数不成等差数列, 其次 $B$ 中没有非常数列的无穷等差数列.
事实上,若 $\{a+n d\}, n=0,1,2, \cdots$ 是 $B$ 中一个非常数列的无穷等差数列, 这里 $a, d \in \mathbf{N}_{+}$, 由 $A$ 的构造知, 存在无穷多个 $m \in \mathbf{N}_{+}$使 $m !+a \in A$, 故存在 $m \geqslant d$, 使 $m !+a \in A$. 令 $n_0=\frac{m !}{d}$, 则 $m !+a=a+n_0 d \in A$, 这与对任意 $n \in \mathbf{N}_{+}, a+n d \in B$ 矛盾.
可见存在满足条件 (1), (2) 的集合 $A$ 与 $B$.
%%PROBLEM_END%%



%%PROBLEM_BEGIN%%
%%<PROBLEM>%%
例3. 是否存在平面内一个有限点集, 使得对于其中每个点, 点集中恰有三个距离它最近的点.
%%<SOLUTION>%%
分析:如图(<FilePath:./figures/fig-c12i1.png>), 由两个有公共底边的正三角形组成的图形共有 4 个顶点, 其中有 2 个点恰有三个距离它最近的点, 但另 2 个点却只有两个距离它最近的点, 不满足要求.
其次,如图(<FilePath:./figures/fig-c12i2.png>) 考虑用 $m$ 条线段将 $m$ 个这样的图形连起来看能否组成一个满足条件的图形.
图 (<FilePath:./figures/fig-c12i2.png>) 中为了保证每个点恰有三个距离它最近的点, 只要 $90^{\circ}<\alpha \leqslant 120^{\circ}$, 并且由凸多边形内角和公式有
$$
m \cdot 120^{\circ}+2 m\left(\alpha+60^{\circ}\right)=(3 m-2) \cdot 180^{\circ},
$$
即 $\alpha=150^{\circ}-\frac{180^{\circ}}{m}$, 代入 $90^{\circ}<\alpha \leqslant 120^{\circ}$ 解得 $3<m \leqslant 6$. 所以 $m=4,5$ 或 6 , 当 $m=4$ 时 $\alpha=105^{\circ} ; m=5$ 时 $\alpha=114^{\circ} ; m=6$ 时 $\alpha=120^{\circ}$.
解存在,如图(<FilePath:./figures/fig-c12i3.png>) 所示的三个点集都具有题目中要求的性质(图中已将每点与它距离最近的三个点相连)
%%<REMARK>%%
注:本例中我们用部件组成法得到了满足题目条件的点集, 而且不止一个, 显然若干个这样点集的并集 (只要每两个这样点集之间点的最小距离大于原点集中任意两点之间距离的最小值) 也满足题目的所有条件, 而在解答中常常将组成的设计及探索过程省略, 只写出了构造的结果.
%%PROBLEM_END%%



%%PROBLEM_BEGIN%%
%%<PROBLEM>%%
例4. 平面内任给 2000 个点,证明: 可以用一些圆形纸片盖住这 2000 个点, 且满足:
(1)这些圆形纸片直径之和不超过 2000;
(2)任意两张圆心纸片的距离大于 1 .
%%<SOLUTION>%%
证明:首先证明满足条件 (1) 的纸片存在.
事实上, 取 2000 张直径为 1 的纸片, 使每张纸片的中心恰在给出的一个点上, 于是这 2000 张纸片盖住了 2000 个已知点,且这些纸片直径之和为 2000 .
其次, 若这些纸片中有两张有公共点 (如图(<FilePath:./figures/fig-c12i4.png>) 中圆 $O_1$ 与圆 $O_2$ ), 则进行一次调整.
如图可用一张直径较大的图形纸片 $O_3$ 代替圆 $O_1$ 和圆 $O_2$, 满足 $O_3$ 在直线 $O_1 O_2$ 上且圆 $O_1$ 和圆 $O_2$ 都内切于圆 $O_3$, 显然圆 $O_3$ 的直径不大于圆 $O_1$ 和圆 $O_2$ 的直径之和, 并且圆 $O_3$ 所包含的已知点到圆 $\mathrm{O}_3$ 周界的距离不小于 $\frac{1}{2}$. 如果还有两张纸片有公共点, 可以继续进行这样的调整, 于是经过有限步可用有限张直径之和不大于 2000 且两两无公共点的圆形纸片盖住已知的2000个点, 并且每个已知点到覆盖它的纸片周界的距离不小于 $\frac{1}{2}$. 设这些圆纸片每两张之间的距离的最小值为 $d$, 则 $d>0$.
最后, 若 $d>1$, 则结论成立, 若 $0<d \leqslant 1$, 则再进行如下调整: 将每张纸片用圆心相同, 但半径缩小 $\frac{1}{2}-\frac{d}{3}$ 的纸片代替, 则这些新纸片仍盖住了已知的2000个点, 它们的直径之和小于 2000 且任意两张圆形纸片的距离至少为
$d+2\left(\frac{1}{2}-\frac{d}{3}\right)=1+\frac{d}{3}>1$, 于是题目结论得证.
%%PROBLEM_END%%



%%PROBLEM_BEGIN%%
%%<PROBLEM>%%
例5. 是否存在一个无穷正整数列 $a_1<a_2<a_3<\cdots$, 使得对任意整数 $A$,数列 $\left\{a_n+A\right\}_{n=1}^{\cdots}$ 中仅有有限个素数?
%%<SOLUTION>%%
分析:若 $|A| \geqslant 2$, 则只要 $n$ 充分大时, $a_n$ 含有因数 $|A|$, 则 $a_n+A$ 为合数, 自然想到取 $a_n=n$ ! 故当 $|A| \geqslant 2$ 时, 数列 $\{n !+A\}$ 中至多只有有限个素数.
但 $A= \pm 1$ 时, $\{n !+1\},\{n !-1\}$ 中的素数个数是否有限, 则不好说了, 为了使对 $A= \pm 1$ 情形也能证明 $a_n \pm 1$ 为合数,注意到因式分解的基本公式,我们只要改为取 $a_n$ 等于 $n$ ! 的奇次幂即可.
解存在.
取 $a_n=(n !)^3$, 则 $A=0$ 时 $\left\{a_n\right\}$ 中没有素数; 当 $|A| \geqslant 2$ 时, 只要 $n \geqslant|A|$, 则 $a_n+A=(n !)^3+A$ 均为 $A$ 的倍数, 不可能为素数; 当 $A= \pm 1$ 时, $a_n \pm 1=(n ! \pm 1)\left[(n !)^2 \mp n !+1\right]$. 当 $n \geqslant 3$ 时为合数.
从而对任何整数 $A,\left\{(n !)^3+A\right\}$ 中只有有限个素数.
%%<REMARK>%%
注:从本例可以看出, 用逐步调整法进行构造时, 有时可将调整的过程省略,而直接根据调整的结果进行构造.
%%PROBLEM_END%%



%%PROBLEM_BEGIN%%
%%<PROBLEM>%%
例6. 是否存在集合 $M$, 满足
(1) $M$ 内恰含有 2011 个正整数;
(2) $M$ 内的每一个数以及任意多个数之和都能写成 $m^k\left(m, k \in \mathbf{N}_{+}, k \geqslant\right.$ 2) 的形式.
%%<SOLUTION>%%
分析:显然集合 $\{1,2, \cdots, 2011\}$ 不满足条件(2), 我们考虑是否存在正整数 $d$ 使集合 $M=\{d, 2 d, 3 d, \cdots, 2011 d\}$ 满足条件(2). 因 $M$ 中每个数及任意多个数之和都属于集合 $S=\left\{d, 2 d, \cdots, \frac{1}{2} \times 2011 \times 2012 d\right\}$, 故只要考虑是否存在正整数 $d$, 使 $S$ 中每个数能写成 $m^k\left(m, k \in \mathbf{N}_{+}, k \geqslant 2\right)$ 的形式.
进一步我们将 $\frac{1}{2} \times 2011 \times 2012$ 用任意正整数 $n$ 代替, 然后用归纳构造法作出符合条件的集合 $S$.
解首先证明下列引理.
引理对任意 $n \in \mathbf{N}_{+}$, 存在 $d_n \in \mathbf{N}_{+}$, 使 $S_n=\left\{d_n, 2 d_n, \cdots, n d_n\right\}$ 中每个数可写成 $m^k\left(m, k \in \mathbf{N}_{+}, k \geqslant 2\right)$ 的形式.
证明 $n=1$ 时, 取 $d_1=3^2$ 知结论成立.
假设对 $n \in \mathbf{N}_{+}$结论成立, 即存在 $d_n \in \mathbf{N}_{+}$, 使 $i d_n=m_i^{k_i}\left(m_i, k_i \in \mathbf{N}_{+}\right.$, $\left.k_i \geqslant 2, i=1,2, \cdots, n\right)$, 取 $d_{n+1}=d_n\left[(n+1) d_n\right]^k$, 其中 $k$ 为 $k_1, k_2, \cdots, k_n$ 的公倍数, 并设 $k=k_i p_i(1 \leqslant i \leqslant n)$,于是当 $1 \leqslant i \leqslant n$ 时, 有
$$
i d_{n+1}=i d_n\left[(n+1) d_n\right]^k=m_i^{k_i}\left[(n+1) d_n\right]^{k_i p_i}
$$
$$
=\left\{m_i\left[(n+1) d_n\right]^{p_i}\right\}^{k_i},
$$
及 $(n+1) d_{n+1}=\left[(n+1) d_n\right]^{k+1}$, 即知对 $n+1$ 结论成立.
于是引理得证.
回到原题, 设 $n_0=\frac{1}{2} \times 2011 \times 2012$, 于是由引理知存在 $d_{n_0}$, 使 $S_{n_0}= \left\{d_{n_0}, 2 d_{n_0}, \cdots, n_0 d_{n_0}\right\}$ 中每个数可写成 $m^k\left(m, k \in \mathbf{N}_{+}, k \geqslant 2\right)$ 的形式, 从而 $M=\left\{d_{n_0}, 2 d_{n_0}, \cdots, 2011 d_{n_0}\right\}$ 满足题目条件(1)和 (2).
%%<REMARK>%%
注:上述证明中 $d_{n+1}$ 的取法是这样想到的, 为了使 $i=1,2, \cdots, n$ 时能用归纳假设, $d_{n+1}$ 中必含有因数 $d_n$, 为了使 $(n+1) d_{n+1}$ 具有 $m^k\left(m, k \in \mathbf{N}_{+}\right.$, $k \geqslant 2)$ 的形式, $d_{n+1}$ 应具有 $d_n\left[(n+1) d_n\right]^k$ 的形式,再为了使当 $1 \leqslant i \leqslant n$ 时, $i d_{n+1}=m_i^{k_i}\left[(n+1) d_n\right]^k$ 具有 $m^k\left(m, k \in \mathbf{N}_{+}, k \geqslant 2\right)$ 的形式, 必须 $k$ 为 $k_i(i= 1,2, \cdots, n)$ 的倍数, 从而 $k$ 为 $k_1, k_2, \cdots, k_n$ 的公倍数.
%%PROBLEM_END%%


