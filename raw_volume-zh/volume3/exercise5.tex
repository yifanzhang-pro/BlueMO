
%%PROBLEM_BEGIN%%
%%<PROBLEM>%%
问题1. 设 $a=\sin (-1), b=\cos (-1), c=\tan (-1)$, 则有 ( ).
(A) $a<b<c$
(B) $b<a<c$
(C) $c<a<b$
(D) $a<c<b$
%%<SOLUTION>%%
C. 因 $-\frac{\pi}{2}<-1<0$, 故 $b>0, a<0, c<0$, 又 $\tan (-1)=\frac{\sin (-1)}{\cos (-1)}< \sin (-1)$, 所以 $c<a<b$, 选 C.
%%PROBLEM_END%%



%%PROBLEM_BEGIN%%
%%<PROBLEM>%%
问题2. 设 $a=\sin 15^{\circ}+\cos 15^{\circ}, b=\sin 16^{\circ}+\cos 16^{\circ}$, 则下列各式中正确的是 ( ).
(A) $a<\frac{a^2+b^2}{2}<b$
(B) $a<b<\frac{a^2+b^2}{2}$
(C) $b<a<\frac{a^2+b^2}{2}$
(D) $b<\frac{a^2+b^2}{2}<a$
%%<SOLUTION>%%
B. 因 $a=\sin 15^{\circ}+\cos 15^{\circ}=\sqrt{2} \sin 60^{\circ}=\frac{\sqrt{6}}{2}, b=\sin 16^{\circ}+\cos 16^{\circ}= \sqrt{2} \sin 61^{\circ}$, 所以 $a<b$. 又 $\frac{a^2+b^2}{2}-b=\frac{1}{2}\left(b^2-2 b+a^2\right)=\frac{1}{2}\left(b^2-2 b+\frac{3}{2}\right)= \frac{1}{2}\left[(b-1)^2+\frac{1}{2}\right]>0$, 所以 $\frac{a^2+b^2}{2}>b$. 选 B.
%%PROBLEM_END%%



%%PROBLEM_BEGIN%%
%%<PROBLEM>%%
问题3. 若 $a=(\tan x)^{\cot x}, b=(\cot x)^{\tan x}, c=(\tan x)^{\cos x}$, 当 $0<x<\frac{\pi}{4}$ 时, 下列不等式成立的是 ( ).
(A) $b<a<c$
(B) $a<b<c$
(C) $a<c<b$
(D) $b<c<a$
%%<SOLUTION>%%
C. 当 $0<x<\frac{\pi}{4}$ 时, $0<\tan x<1$, $\cot x=\frac{\cos x}{\sin x}>\cos x$, 所以 $(\tan x)^{\cot x}<(\tan x)^{\cos x}$, 即 $a<c$. 又 $(\cot x)^{\tan x}=(\tan x)^{-\tan x}>(\tan x)^{\cos x}$, 得 $b>c$, 故 $a<c<b$. 选 C.
%%PROBLEM_END%%



%%PROBLEM_BEGIN%%
%%<PROBLEM>%%
问题4. 设$0 < x < \frac{\pi}{4}$,下列关系中正确的是 ( ).
(A) $\sin (\sin x)<\sin x<\sin (\tan x)$
(B) $\sin (\sin x)<\sin (\tan x)<\sin x$
(C) $\sin (\tan x)<\sin x<\sin (\sin x)$
(D) $\sin x<\sin (\tan x)<\sin (\sin x)$
%%<SOLUTION>%%
A. 利用不等式 $\sin x<x<\tan x$ 得, 当 $0<x<\frac{\pi}{4}$ 时, $0<\sin x< x<\tan x<1$, 且 $y=\sin x$ 在 $\left[0, \frac{\pi}{2}\right]$ 上单调递增, 所以 $\sin (\sin x)<\sin x< \sin (\tan x)$, 故选 A.
%%PROBLEM_END%%



%%PROBLEM_BEGIN%%
%%<PROBLEM>%%
问题5. 已知 $\frac{\pi}{2}<\alpha+\beta<\frac{5 \pi}{2},-\frac{\pi}{2}<\alpha-\beta<\frac{3 \pi}{2}$, 那么 $\cos \alpha$ 与 $\sin \beta$ 的大小关系是( ).
(A) $\cos \alpha=\sin \beta$
(B) $\cos \alpha<\sin \beta$
(C) $\cos \alpha>\sin \beta$
(D) 不能确定
%%<SOLUTION>%%
B. $\cos \alpha-\sin \beta=\sin \left(\frac{\pi}{2}-\alpha\right)-\sin \beta=2 \cos \left(\frac{\pi}{4}-\frac{\alpha-\beta}{2}\right) \sin \left(\frac{\pi}{4}-\right. \left.\frac{\alpha+\beta}{2}\right), \cos \left(\frac{\pi}{4}-\frac{\alpha-\beta}{2}\right)>0, \sin \left(\frac{\pi}{4}-\frac{\alpha+\beta}{2}\right)<0$, 故 $\cos \alpha<\sin \beta$.
%%PROBLEM_END%%



%%PROBLEM_BEGIN%%
%%<PROBLEM>%%
问题6. 设 $\alpha \in\left(\frac{\pi}{4}, \frac{\pi}{2}\right)$, 则 $(\cos \alpha)^{\cos \alpha}=a,(\sin \alpha)^{\cos \alpha}=b,(\cos \alpha)^{\sin \alpha}=c$ 的大小顺序是 ( ).
(A) $a<b<c$
(B) $a<c<b$
(C) $b<a<c$
(D) $c<a<b$
%%<SOLUTION>%%
D. 用特殊值法, 令 $\alpha=\frac{\pi}{3}$, 可比较知 $c<a<b$.
%%PROBLEM_END%%



%%PROBLEM_BEGIN%%
%%<PROBLEM>%%
问题7. 已知 $\sin x \geqslant \frac{\sqrt{3}}{2}$, 则 $x$ 的取值范围是 ( ).
(A) $\left[\frac{\pi}{3}, \frac{2 \pi}{3}\right]$
(B) $\left[k \pi-\frac{4 \pi}{3}, k \pi+\frac{4}{3} \pi\right](k \in \mathbf{Z})$
(C) $\left[2 k \pi+\frac{\pi}{3}, 2 k \pi+\frac{2 \pi}{3}\right](k \in \mathbf{Z})$.
(D) $\left[2 k \pi-\frac{4}{3} \pi, 2 k \pi+\frac{\pi}{3}\right](k \in \mathbf{Z})$
%%<SOLUTION>%%
C. 因 $\sin \frac{\pi}{3}=\frac{\sqrt{3}}{2}, \sin \frac{2 \pi}{3}=\frac{\sqrt{3}}{2}$, 所以满足条件的 $x$ 值为 $2 k \pi+\frac{\pi}{3} \leqslant x \leqslant 2 k \pi+\frac{2 \pi}{3}, k \in \mathbf{Z}$.
%%PROBLEM_END%%



%%PROBLEM_BEGIN%%
%%<PROBLEM>%%
问题8. $\triangle A B C$ 中, 若 $\tan \frac{A}{2} 、 \tan \frac{B}{2} 、 \tan \frac{C}{2}$ 成等比数列, 则角 $B$ 的取值范围是
(A) $\left(0, \frac{\pi}{6}\right]$
(B) $\left(0, \frac{\pi}{3}\right]$
(C) $\left[\frac{\pi}{3}, \frac{2 \pi}{3}\right]$
(D) $\left[\frac{2 \pi}{3}, \pi\right)$
%%<SOLUTION>%%
B. 参照例 13 (2) 可知 $\tan \frac{A}{2} \cdot \tan \frac{B}{2}+\tan \frac{B}{2} \cdot \tan \frac{C}{2}+\tan \frac{C}{2} \cdot \tan \frac{A}{2}=$ 1 , 由题设条件得 $\tan ^2 \frac{B}{2}=\tan \frac{A}{2} \cdot \tan \frac{C}{2}$, 所以 $\tan \frac{A}{2}+\tan \frac{C}{2} \geqslant 2 \sqrt{\tan \frac{A}{2} \cdot \tan \frac{C}{2}}=2 \tan \frac{B}{2}$, 故 $\left(\tan \frac{A}{2}+\tan \frac{C}{2}\right) \tan \frac{B}{2}+\tan \frac{C}{2} \tan \frac{A}{2} \geqslant 3 \tan ^2 \frac{B}{2}$, 从而 $\tan ^2 \frac{B}{2} \leqslant \frac{1}{3}, \tan \frac{B}{2} \leqslant \frac{\sqrt{3}}{3}, 0<\frac{B}{2} \leqslant \frac{\pi}{6}$, 所以 $0<B \leqslant \frac{\pi}{3}$.
%%PROBLEM_END%%



%%PROBLEM_BEGIN%%
%%<PROBLEM>%%
问题9. $\cos 2 x>\cos 2 y$ 成立的一个充分非必要条件是( ).
(A) $|\cos x|>|\cos y|$
(B) $|\cos x|>\cos y$
(C) $\cos x>|\cos y|$
(D) $\cos x+\cos y>0$
%%<SOLUTION>%%
C. $\cos 2 x>\cos 2 y \Leftrightarrow 2 \cos ^2 x-1>2 \cos ^2 y-1 \Leftrightarrow \cos ^2 x>\cos ^2 y \Leftrightarrow |\cos x|>|\cos y|$. 故选 C.
%%PROBLEM_END%%



%%PROBLEM_BEGIN%%
%%<PROBLEM>%%
问题10. 记 $a=\log _{\sin 1} \cos 1, b=\log _{\sin 1} \tan 1, c=\log _{\cos 1} \sin 1, d=\log _{\cos 1} \tan 1$, 则四个数的大小关系是( ).
(A) $a<c<b<d$
(B) $c<d<a<b$
(C) $b<d<c<a$
(D) $d<b<a<c$
%%<SOLUTION>%%
C. 因 $\sin 1>\cos 1, \tan 1>1$, 所以 $\log _{\tan 1} \cos 1<\log _{\tan 1} \sin 1<0$, 故 $\log _{\cos 1} \tan 1>\log _{\sin 1} \tan 1$, 即 $d>b$. 因为 $0<\cos 1<1, \sin 1<\tan 1$, 所以 $\log _{\cos 1} \sin 1>\log _{\cos 1} \tan 1$, 即 $c>d$; 同理 $a=\log _{\sin 1} \cos 1>\log _{\sin 1} \sin 1= \log _{\cos 1} \cos 1>\log _{\cos 1} \sin 1$, 即 $a>c$, 综上所述 $a>c>d>b$.
%%PROBLEM_END%%



%%PROBLEM_BEGIN%%
%%<PROBLEM>%%
问题11. 当 $0<x<2 \pi$ 时, 不等式 $\frac{\cos 2 x+\cos x-1}{\cos 2 x}>2$ 的解集是
%%<SOLUTION>%%
$\left(\frac{\pi}{4}, \frac{\pi}{3}\right) \cup\left(\frac{\pi}{2}, \frac{3 \pi}{4}\right) \cup\left(\frac{5 \pi}{4}, \frac{3 \pi}{2}\right) \cup\left(\frac{5 \pi}{3}, \frac{7 \pi}{4}\right)$. 由原不等式得 $\frac{\cos 2 x-\cos x+1}{\cos 2 x}<0 \Leftrightarrow \frac{2 \cos ^2 x-\cos x}{\cos 2 x}<0 \Leftrightarrow\left\{\begin{array}{l}2 \cos ^2 x-\cos x>0 \\ \cos 2 x<0\end{array}\right.$ 或 $\left\{\begin{array}{l}2 \cos ^2 x-\cos x<0 \\ \cos 2 x>0\end{array} \Leftrightarrow x \in\left(\frac{\pi}{4}, \frac{\pi}{3}\right) \cup\left(\frac{\pi}{2}, \frac{3 \pi}{4}\right) \cup\left(\frac{5 \pi}{4}, \frac{3 \pi}{2}\right) \cup\left(\frac{5 \pi}{3}, \frac{7 \pi}{4}\right)\right.$.
%%PROBLEM_END%%



%%PROBLEM_BEGIN%%
%%<PROBLEM>%%
问题12. 已知 $\sin x<\sin \frac{5 \pi}{4}$, 且 $x$ 为第三象限角, 则 $x$ 的取值范围为
%%<SOLUTION>%%
$\left(2 k \pi+\frac{5 \pi}{4}, 2 k \pi+\frac{3 \pi}{2}\right),(k \in \mathbf{Z})$. 利用函数的单调性可得.
%%PROBLEM_END%%



%%PROBLEM_BEGIN%%
%%<PROBLEM>%%
问题13. 若 $0<x<2 \pi,|\sin x|<\frac{\sqrt{2}}{2}$, 则 $x$ 的取值范围为
%%<SOLUTION>%%
$\left(0, \frac{\pi}{4}\right) \cup\left(\frac{3 \pi}{4}, \frac{5 \pi}{4}\right) \cup\left(\frac{7 \pi}{4}, 2 \pi\right)$.
%%PROBLEM_END%%



%%PROBLEM_BEGIN%%
%%<PROBLEM>%%
问题14. 若 $x \in\left(0, \frac{\pi}{2}\right)$, 则 $\cos (\sin x)$ 与 $\sin (\cos x)$ 的大小关系是
%%<SOLUTION>%%
$\sin (\cos x)<\cos (\sin x)$. 参照例 10 即可得结论.
%%PROBLEM_END%%



%%PROBLEM_BEGIN%%
%%<PROBLEM>%%
问题15. 函数 $y=\sqrt{\cos 2 x}+\sqrt{3-2 \sqrt{3} \tan x-3 \tan ^2 x}$ 的定义域为
%%<SOLUTION>%%
$\left[k \pi-\frac{\pi}{4}, k \pi+\frac{\pi}{6}\right], k \in \mathbf{Z}$. 由不等式组 $\left\{\begin{array}{l}\cos 2 x \geqslant 0, \\ 3-2 \sqrt{3} \tan x-3 \tan ^2 x \geqslant 0\end{array}\right.$ 解得.
%%PROBLEM_END%%



%%PROBLEM_BEGIN%%
%%<PROBLEM>%%
问题16. 已知 $0<\theta<{\frac{\pi}{2}}$ ,则 $1+\cot\theta$ 和 cot $\frac{\theta}{2}$ 的大小关系是
%%<SOLUTION>%%
$1+\cot \theta<\cot \frac{\theta}{2}$.
%%PROBLEM_END%%



%%PROBLEM_BEGIN%%
%%<PROBLEM>%%
问题17. 在 $\triangle A B C$ 中, $A>B$, 下列三个不等式 (1) $\sin A>\sin B$; (2) $\cos A<\cos B$; (3) $\tan A>\tan B$, 其中正确的有
%%<SOLUTION>%%
(1)(2). 在 $\triangle A B C$ 中, $A>B$ 则 $a>b$, 所以 $2 R \sin A>2 R \sin B$, 即 (1) 正确; 又若 $A$ 为针角或直角, 则 $\cos A \leqslant 0<\cos B$; 若 $A$ 为锐角, 则 $\cos A< \cos B$, 所以 (2) 正确; 若 $A$ 为钝角, $\tan A<0$, 所以 (3) 错误.
综上所述, 正确的有 (1)(2).
%%PROBLEM_END%%



%%PROBLEM_BEGIN%%
%%<PROBLEM>%%
问题18. 若函数 $f(1+\cos x)=\cos ^2 x$, 则 $f(1+\cos x)$ 与 $f(1-\cos x)$ 的大小关系是
%%<SOLUTION>%%
$f(1+\cos x)=f(1-\cos x)$. 设 $1+\cos x=t$, 则 $\cos x=t-1$, $f(t)=(t-1)^2$, 所以 $f(1-\cos x)=(1-\cos x-1)^2=\cos ^2 x$. 故 $f(1+\cos x)= f(1-\cos x)$.
%%PROBLEM_END%%



%%PROBLEM_BEGIN%%
%%<PROBLEM>%%
问题19. 当 $y=2 \cos x-3 \sin x$ 取得最大值时, $\tan x=$
%%<SOLUTION>%%
$\frac{3}{2} \cdot y=\sqrt{13}\left(\frac{2}{\sqrt{13}} \cos x-\frac{3}{\sqrt{13}}-\sin x\right)=\sqrt{13} \sin (\varphi-x)$, 其中 $\sin \varphi=\frac{2}{\sqrt{13}}, \cos \varphi=-\frac{3}{\sqrt{13}}$, 所以当 $\sin (\varphi-x)=1$ 时, $y$ 有最大值 $\sqrt{13}$, 此时, $\varphi-x=2 k \pi+\frac{\pi}{2}, x=\varphi-\left(2 k \pi+\frac{\pi}{2}\right), \tan x=\tan \left[\varphi-\left(2 k \pi+\frac{\pi}{2}\right)\right]=$
$$
-\cot \varphi=-\frac{\cos \varphi}{\sin \varphi}=\frac{3}{2} .
$$
%%PROBLEM_END%%



%%PROBLEM_BEGIN%%
%%<PROBLEM>%%
问题20. 知 $\triangle A B C$ 是锐角三角形, $m=\log _{\cos A} \frac{1}{\sin B}, n=\log _{\cos A} \frac{1}{\cos C}$, 则 $m$ 与 $n$ 的大小关系为
%%<SOLUTION>%%
$m>n . \triangle A B C$ 为锐角三角形, 得 $0<A<\frac{\pi}{2}, \frac{\pi}{2}<B+C<\pi$, 所以 $B>\frac{\pi}{2}-C$, 又 $B 、 \frac{\pi}{2}-C$ 均为锐角, 故 $\sin B>\cos C>0$, 从而 $\frac{1}{\sin B}<\frac{1}{\cos C}$, 又 $0<\cos A<1$. 所以 $m>n$.
%%PROBLEM_END%%



%%PROBLEM_BEGIN%%
%%<PROBLEM>%%
问题21. $ A 、 B 、 C$ 为 $\triangle A B C$ 三内角, 求证:
$$
\sin \frac{A}{2} \sin \frac{B}{2} \sin \frac{C}{2} \leqslant \frac{1}{8} .
$$
%%<SOLUTION>%%
因为 $b+c \geqslant 2 \sqrt{b c}$ ( $b=c$ 时, 取等号), 所以 $\frac{a}{2 \sqrt{b c}} \geqslant \frac{a}{b+c}= \frac{\sin A}{\sin B+\sin C}=\frac{2 \sin \frac{A}{2} \cos \frac{A}{2}}{2 \sin \frac{B+C}{2} \cos \frac{B-C}{2}}=\frac{\sin \frac{A}{2}}{\cos \frac{B-C}{2}} \geqslant \sin \frac{A}{2}(B=C$ 时取等号). 同理可得 $\frac{b}{2 \sqrt{a c}} \geqslant \sin \frac{B}{2}, \frac{C}{2 \sqrt{a b}} \geqslant \sin \frac{C}{2}$, 所以 $\sin \frac{A}{2} \sin \frac{B}{2} \sin \frac{C}{2} \leqslant \frac{a}{2 \sqrt{b c}} \cdot \frac{b}{2 \sqrt{a c}} \cdot \frac{c}{2 \sqrt{b a}}=\frac{1}{8}$.
%%PROBLEM_END%%



%%PROBLEM_BEGIN%%
%%<PROBLEM>%%
问题22. 已知 $0<\theta<\frac{\pi}{2}, a 、 b>0$, 求证:
$$
\frac{a}{\sin \theta}+\frac{b}{\cos \theta} \geqslant\left(a^{\frac{2}{3}}+b^{\frac{2}{3}}\right)^{\frac{3}{2}}
$$
%%<SOLUTION>%%
$\frac{a}{\sin \theta}+\frac{b}{\cos \theta} \geqslant\left(a^{\frac{2}{3}}+b^{\frac{2}{3}}\right)^{\frac{3}{2}} \Leftrightarrow\left(\frac{a}{\sin \theta}+\frac{b}{\cos \theta}\right)^2 \geqslant\left(a^{\frac{2}{3}}+b^{\frac{2}{3}}\right)^3 \Leftrightarrow \frac{a^2}{\sin ^2 \theta}+\frac{2 a b}{\sin \theta \cos \theta}+\frac{b^2}{\cos ^2 \theta} \geqslant a^2+b^2+3 \sqrt[3]{a^4 b^2}+3 \sqrt[3]{a^2 b^4} \Leftrightarrow a^2 \cot ^2 \theta+ b^2 \tan ^2 \theta+2 a b \frac{\sin ^2 \theta+\cos ^2 \theta}{\sin \theta \cos \theta} \geqslant 3 \sqrt[3]{a^4 b^2}+3 \sqrt[3]{a^2 b^4} \Leftrightarrow a^2 \cot ^2 \theta+b^2 \tan ^2 \theta+ 2 a b \tan \theta+2 a b \cot \theta \geqslant 3 \sqrt[3]{a^4 b^2}+3 \sqrt[3]{\sqrt{a^2 b^4}},(*)$ 因为 $a^2 \cot ^2 \theta+2 a b \tan \theta= a^2 \cot ^2 \theta+a b \tan \theta+a b \tan \theta \geqslant 3 \sqrt[3]{a^4 b^2}$, 同理 $b^2 \tan ^2 \theta+2 a b \cot \theta \geqslant 3 \sqrt[3]{a^2 b^4}$, 所以 $(*)$ 成立, 原不等式得证.
%%PROBLEM_END%%



%%PROBLEM_BEGIN%%
%%<PROBLEM>%%
问题23. $ A 、 B 、 C$ 为 $\triangle A B C$ 三内角, 证明:
$$
\sqrt{\tan \frac{A}{2} \tan \frac{B}{2}+5}+\sqrt{\tan \frac{B}{2} \tan \frac{C}{2}+5}+\sqrt{\tan \frac{C}{2} \tan \frac{A}{2}+5} \leqslant 4 \sqrt{3} .
$$
%%<SOLUTION>%%
由 $\tan \frac{A}{2} \tan \frac{B}{2}+\tan \frac{B}{2} \tan \frac{C}{2}+\tan \frac{C}{2} \tan \frac{A}{2}=1$, 得 $\left(\tan \frac{A}{2} \tan \frac{B}{2}+5\right)+ \left(\tan \frac{B}{2} \tan \frac{C}{2}+5\right)+\left(\tan \frac{C}{2} \tan \frac{A}{2}+5\right)=16$, 所以 $\left(\sqrt{\tan \frac{A}{2} \tan \frac{B}{2}+5}+\right. \left.\sqrt{\tan \frac{B}{2} \tan \frac{C}{2}+5}+\sqrt{\tan \frac{A}{2} \tan \frac{C}{2}+5}\right)^2=16+2 \sqrt{\tan \frac{A}{2} \tan \frac{B}{2}+5} \times \sqrt{\tan \frac{B}{2} \tan \frac{C}{2}+5}+2 \sqrt{\tan \frac{A}{2} \tan \frac{B}{2}+5} \sqrt{\tan \frac{C}{2} \tan \frac{A}{2}+5}+2 \sqrt{\tan \frac{B}{2} \tan \frac{C}{2}+5} \sqrt{\tan \frac{C}{2} \tan \frac{A}{2}+5} \leqslant 16+\tan \frac{A}{2} \tan \frac{B}{2}+5+\tan \frac{B}{2} \tan \frac{C}{2}+5+\tan \frac{A}{2} \tan \frac{B}{2}+ 5+\tan \frac{C}{2} \tan \frac{A}{2}+5+\tan \frac{B}{2} \tan \frac{C}{2}+5+\tan \frac{C}{2} \tan \frac{A}{2}+5=16+30+2=48$,
所以 $\sqrt{\tan \frac{A}{2} \tan \frac{B}{2}+5}+\sqrt{\tan \frac{B}{2} \tan \frac{C}{2}+5}+\sqrt{\tan \frac{C}{2} \tan \frac{A}{2}+5} \leqslant \sqrt{48}= 4 \sqrt{3}$.
%%PROBLEM_END%%



%%PROBLEM_BEGIN%%
%%<PROBLEM>%%
问题24. 在锐角 $\triangle A B C$ 中,求证:
$$
\sec A+\sec B+\sec C \geqslant \csc \frac{A}{2}+\csc \frac{B}{2}+\csc \frac{C}{2} .
$$
%%<SOLUTION>%%
当 $a 、 b \in \mathbf{R}^{+}$时, 有 $(a+b)\left(\frac{1}{a}+\frac{1}{b}\right) \geqslant 4$, 又 $A 、 B 、 C$ 为锐角 $\triangle A B C$ 的内角, 所以 $\sec A+\sec B=\frac{1}{\cos A}+\frac{1}{\cos B} \geqslant \frac{4}{\cos A+\cos B}= \frac{4}{2 \cos \frac{A+B}{2} \cos \frac{A-B}{2}} \geqslant \frac{2}{\cos \frac{A+B}{2}}=\frac{2}{\sin \frac{C}{2}}=2 \csc \frac{C}{2}$, 同理 $\sec B+\sec C \geqslant 2 \csc \frac{A}{2}, \sec C+\sec A \geqslant 2 \csc \frac{B}{2}$, 三式相加, 得证.
%%PROBLEM_END%%



%%PROBLEM_BEGIN%%
%%<PROBLEM>%%
问题25. 设 $\triangle A B C$ 的外接圆半径为 $R$, 面积为 $S$, 角 $A 、 B 、 C$ 所对的边为 $a 、 b 、 c$.
求证:
$$
\frac{36 S}{(a+b+c)^2} \leqslant \tan \frac{A}{2}+\tan \frac{B}{2}+\tan \frac{C}{2} \leqslant \frac{9 R^2}{4 S} .
$$
%%<SOLUTION>%%
(1) 因为 $\left(\tan \frac{A}{2}+\tan \frac{B}{2}+\tan \frac{C}{2}\right)\left(\cot \frac{A}{2}+\cot \frac{B}{2}+\cot \frac{C}{2}\right) \geqslant 9$, 又 $\cot \frac{A}{2}+\cot \frac{B}{2}+\cot \frac{C}{2}=\frac{(a+b+c)^2}{4 S}$, 所以 $\tan \frac{A}{2}+\tan \frac{B}{2}+\tan \frac{C}{2} \geqslant \frac{9}{\frac{(a+b+c)^2}{4 S}}=\frac{36 S}{(a+b+c)^2}$. (2) 设 $\triangle A B C$ 内切圆半径为 $r$, 易知 $\tan \frac{A}{2}= \frac{2 r}{b+c-a}$, 故 $\tan \frac{A}{2}+\tan \frac{B}{2}+\tan \frac{C}{2}=\frac{2 r}{b+c-a}+\frac{2 r}{c+a-b}+\frac{2 r}{a+b-c}= \frac{4 S}{(a+b+c)(b+c-a)}+\frac{4 S}{(a+b+c)(c+a-b)}+\frac{4 S}{(a+b+c)(a+b-c)} =\frac{S}{p(p-a)}+\frac{S}{p(p-b)}+\frac{S}{p(p-c)}\left(p=\frac{a+b+c}{2}\right)$, 又 $S= \sqrt{p(p-a)(p-b)(p-c)}$, 所以 $4 S\left(\tan \frac{A}{2}+\tan \frac{B}{2}+\tan \frac{C}{2}\right)=4(p-b) (p-c)+4(p-c)(p-a)+4(p-a)(p-b) \leqslant(p-b+p-c)^2+(p-c+ p-a)^2+(p-a+p-b)^2=a^2+b^2+c^2=4 R^2\left(\sin ^2 A+\sin ^2 B+\sin ^2 C\right)$, 因为 $\sin ^2 A+\sin ^2 B+\sin ^2 C \leqslant \frac{9}{4}$, 所以 $4 S\left(\tan \frac{A}{2}+\tan \frac{B}{2}+\tan \frac{C}{2}\right) \leqslant 9 R^2$, 即 $\tan \frac{A}{2}+\tan \frac{B}{2}+\tan \frac{C}{2} \leqslant \frac{9 R^2}{4 S}$.
%%PROBLEM_END%%



%%PROBLEM_BEGIN%%
%%<PROBLEM>%%
问题26. 求证: $1 \leqslant \sqrt{|\sin \alpha|}+\sqrt{|\cos \alpha|} \leqslant 2^{\frac{3}{4}}$.
%%<SOLUTION>%%
原不等式等价于 $1 \leqslant|\sin \alpha|+|\cos \alpha|+2 \sqrt{|\sin \alpha \cos \alpha|} \leqslant 2 \sqrt{2}$, 因为 $(|\sin \alpha|+|\cos \alpha|)^2=\sin ^2 \alpha+\cos ^2 \alpha+2|\sin \alpha \cos \alpha|=1+|\sin 2 \alpha| \leqslant 2$, 所以 $1 \leqslant|\sin \alpha|+|\cos \alpha| \leqslant \sqrt{2}$, 又 $2 \sqrt{|\sin \alpha \cos \alpha|}=\sqrt{2|\sin 2 \alpha|} \leqslant \sqrt{2}$, 故原不等式显然成立.
%%PROBLEM_END%%



%%PROBLEM_BEGIN%%
%%<PROBLEM>%%
问题27. 求证: $\cos x \geqslant 1-\frac{x^2}{2}$.
%%<SOLUTION>%%
先证 $|\sin x| \leqslant|x|, \cos x=1-2 \sin ^2 \frac{x}{2}=1-2\left|\sin \frac{x}{2}\right|^2 \geqslant 1-$
$$
2\left|\frac{x}{2}\right|^2=1-\frac{x^2}{2}
$$
%%PROBLEM_END%%



%%PROBLEM_BEGIN%%
%%<PROBLEM>%%
问题28. 试证明: 对于任何实数 $x,|\sin x|$ 与 $|\sin (x+1)|$ 中至少有一个大于 $\frac{1}{3}$.
%%<SOLUTION>%%
用反证法.
设 $|\sin x| \leqslant \frac{1}{3},|\sin (x+1)| \leqslant \frac{1}{3}$, 如图(<FilePath:./figures/fig-c5p28.png>), 作两条平行直线 $y= \pm \frac{1}{3}$, 则角 $x, x+1$ 的终边与单位圆的交点都落在 $\overparen{A B}$ 或 $\overparen{B^{\prime} A^{\prime}}$ 上, 由于 $\sin \angle A O B=\sin \left(2 \arcsin \frac{1}{3}\right)=\frac{4 \sqrt{2}}{9}<\frac{\sqrt{2}}{2}= \sin \frac{\pi}{4}<\sin 1$, 所以 $\angle A O B<1(\mathrm{rad})$, 导致矛盾.
%%PROBLEM_END%%



%%PROBLEM_BEGIN%%
%%<PROBLEM>%%
问题29. 求实数 $a$ 的取值范围, 使得对任意实数 $x$ 和任意 $\theta \in\left[0, \frac{\pi}{2}\right]$, 恒有 $(x+$
$$
3+2 \sin \theta \cdot \cos \theta)^2+(x+a \sin \theta+a \cos \theta)^2 \geqslant \frac{1}{8} \text {. }
$$
%%<SOLUTION>%%
设 $u=\sin \theta+\cos \theta=\sqrt{2} \sin \left(\theta+\frac{\pi}{4}\right) \in[1, \sqrt{2}]$, 则 $2 \sin \theta \cos \theta= u^2-1$, 原不等式可化为 $\left(x+u^2+2\right)^2+(x+a u)^2 \geqslant \frac{1}{8}$, 即 $2 x^2+2\left(u^2+a u+\right.$ 2) $x+\left(u^2+2\right)^2+a^2 u^2-\frac{1}{8} \geqslant 0$, 所以 $\Delta=4\left(u^2+a u+2\right)^2-8\left(u^2+2\right)^2- 8 a^2 u^2+1 \leqslant 0$. 从而 $\left(u^2-a u+2\right)^2 \geqslant \frac{1}{4}$, 即 $u^2-a u+\frac{3}{2} \geqslant 0$ 或 $u^2-a u+ \frac{5}{2} \leqslant 0$, 故 $a \geqslant u+\frac{5}{2 u}$ 或 $a \leqslant u+\frac{3}{2 u}$. 当 $u \in[1, \sqrt{2}]$ 时, $f(u)=u+\frac{5}{2 u}$ 在 $[1, \sqrt{2}]$ 上为减函数, 故 $a \geqslant 1+\frac{5}{2}=\frac{7}{2}$, 而 $g(u)=u+\frac{3}{2 u} \geqslant \sqrt{6}$, 当且仅当 $u=\sqrt{\frac{3}{2}}$ 时等号成立, 故 $a \leqslant \sqrt{6}$, 综上所述, $a$ 的取值范围为 $a \geqslant \frac{7}{2}$ 或 $a \leqslant \sqrt{6}$.
%%PROBLEM_END%%



%%PROBLEM_BEGIN%%
%%<PROBLEM>%%
问题30. 设 $x 、 y 、 z$ 为实数, $0<x<y<z<\frac{\pi}{2}$, 证明:
$$
\frac{\pi}{2}+2 \sin x \cos y+2 \sin y \cos z>\sin 2 x+\sin 2 y+\sin 2 z .
$$
%%<SOLUTION>%%
原不等式等价于 $\frac{\pi}{2}>2 \sin x(\cos x- \cos y)+2 \sin y(\cos y-\cos z)+\sin 2 z$, 即 $\frac{\pi}{4}> \sin x(\cos x-\cos y)+\sin y(\cos y-\cos z)+ \sin z \cos z$, 如图(<FilePath:./figures/fig-c5p30.png>)在单位圆中, 设半径 $O A 、 O B 、 O C$ 与 $O x$ 的夹角分别为 $x 、 y 、 z$, 则 $S_1=\sin x(\cos x- \cos y), S_2=\sin y(\cos y-\cos z), S_3=\sin z \cdot \cos z$, 显然 $S_1+S_2+S_3<\frac{\pi}{4}$, 故得证.
%%PROBLEM_END%%


