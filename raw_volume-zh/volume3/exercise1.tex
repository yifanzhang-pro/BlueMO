
%%PROBLEM_BEGIN%%
%%<PROBLEM>%%
问题1 函数 $y=\sin \left(x+\frac{3}{2} \pi\right)$ 的图象 ( ).
(A) 关于 $x$ 轴对称
(B) 关于 $y$ 轴对称
(C) 关于原点对称
(D) 关于直线 $x=-\frac{3}{2} \pi$ 对称
%%<SOLUTION>%%
B. 因 $y=\sin \left(x+\frac{3}{2} \pi\right)=-\cos x$, 故选 B.
%%PROBLEM_END%%



%%PROBLEM_BEGIN%%
%%<PROBLEM>%%
问题2 営与正弦曲线 $y=\sin x$ 关于直线 $x=\frac{3 \pi}{4}$ 对称的曲线是 ( ).
(A) $y=\sin x$
(B) $y=\cos x$
(C) $y=-\sin x$
(D) $y=-\cos x$
%%<SOLUTION>%%
D. 若点 $(x, y)$ 在函数 $y=\sin x$ 的图象上, 则它关于直线 $x=\frac{3}{4} \pi$ 对称的点 $\left(\frac{3}{2} \pi-x, y\right)$ 在函数 $y=-\cos x$ 的图象上.
%%PROBLEM_END%%



%%PROBLEM_BEGIN%%
%%<PROBLEM>%%
问题3 函数 $y=-3 \cos \left(-2 x+\frac{\pi}{3}\right)$ 的图象可由 $y=-3 \cos (-2 x)$ 的图象( ).
(A) 向左平移 $\frac{\pi}{3}$ 得到
(B) 向右平移 $\frac{\pi}{3}$ 得到
(C) 向左平移 $\frac{\pi}{6}$ 得到
(D) 向右平移 $\frac{\pi}{6}$ 得到
%%<SOLUTION>%%
D. 原函数可变形为 $y=-3 \cos \left[-2\left(x-\frac{\pi}{6}\right)\right]$.
%%PROBLEM_END%%



%%PROBLEM_BEGIN%%
%%<PROBLEM>%%
问题4 已知 $f(x)=\sin \left(x+\frac{\pi}{2}\right), g(x)=\cos \left(x-\frac{\pi}{2}\right)$, 则 $f(x)$ 的图象 ( ).
(A) 与 $g(x)$ 的图象相同
(B) 与 $g(x)$ 的图象关于 $x$ 轴对称
(C) 是由 $g(x)$ 的图象向左平移 $\frac{\pi}{2}$ 个单位得到
(D) 是由 $g(x)$ 的图象向右平移 $\frac{\pi}{2}$ 个单位得到
%%<SOLUTION>%%
C. 原函数变形为 $f(x)=\cos x, g(x)=\sin x$.
%%PROBLEM_END%%



%%PROBLEM_BEGIN%%
%%<PROBLEM>%%
问题5 函数 $y=2^{-\cos x}$ 的单调递增区间是 ( ).
(A) $[2 k \pi+\pi, 2 k \pi+2 \pi], k \in \mathbf{Z}$
(B) $[k \pi+\pi, k \pi+2 \pi], k \in \mathbf{Z}$
(C) $\left[2 k \pi, 2 k \pi+\frac{\pi}{2}\right], k \in \mathbf{Z}$
(D) $[2 k \pi, 2 k \pi+\pi], k \in \mathbf{Z}$
%%<SOLUTION>%%
D. 因函数 $y=2^u$ 在 $\mathbf{R}$ 上单调递增, 而 $u=-\cos x$ 在 $[2 k \pi, 2 k \pi+\pi] (k \in \mathbf{Z})$ 上也单调递增.
%%PROBLEM_END%%



%%PROBLEM_BEGIN%%
%%<PROBLEM>%%
问题6 函数 $y=\sqrt{\cos (\sin x)}$ 的定义域是 ( ).
(A) $2 k \pi-\frac{\pi}{2} \leqslant x \leqslant 2 k \pi+\frac{\pi}{2}(k \in \mathbf{Z})$
(B) $2 k \pi \leqslant x \leqslant 2 k \pi+\frac{\pi}{2}(k \in \mathbf{Z})$
(C) $2 k \pi-\frac{\pi}{2} \leqslant x \leqslant 2 k \pi(k \in$
(D) $x \in \mathbf{R}$
%%<SOLUTION>%%
D. 由原函数得 $\cos (\sin x) \geqslant 0$, 因为 $|\sin x| \leqslant 1$, 即 $-1 \leqslant \sin x \leqslant 1$, 所以 $\cos (\sin x) \geqslant 0$ 恒成立.
%%PROBLEM_END%%



%%PROBLEM_BEGIN%%
%%<PROBLEM>%%
问题7 存在 $x \in[0,2 \pi)$, 使 $\sin x-\sqrt{3} \cos x=\frac{4 m-6}{4-m}$ 成立, 则 $m$ 的取值范围是
(A) $\left[-1, \frac{7}{3}\right]$
(B) $\left(-\infty, \frac{7}{3}\right]$
(C) $[-1,+\infty)$
(D) $(-\infty,-1) \cup\left(\frac{7}{3},+\infty\right)$
%%<SOLUTION>%%
A. 等式左边 $=2\left(\sin x \cos \frac{\pi}{3}-\cos x \sin \frac{\pi}{3}\right)=2 \sin \left(x-\frac{\pi}{3}\right)$. 因原等式成立, 所以 $\left|\frac{4 m-6}{4-m}\right|=\left|2 \sin \left(x-\frac{\pi}{3}\right)\right| \leqslant 2$, 即 $4 m^2-12 m+9 \leqslant m^2- 8 m+16$, 整理得 $3 m^2-4 m-7 \leqslant 0$, 解得 $-1 \leqslant m \leqslant \frac{7}{3}$.
%%PROBLEM_END%%



%%PROBLEM_BEGIN%%
%%<PROBLEM>%%
问题8. $f(x)$ 是以 5 为周期的奇函数, $f(-1)=1$, 且 $\tan \alpha=3$, 则 $f(20 \sin \alpha \cos \alpha)$ 的值是 ( ).
(A) 1
(B) -1
(C) 3
(D) 8
%%<SOLUTION>%%
B. 当 $\tan \alpha=3$ 时, $20 \sin \alpha \cos \alpha=10 \sin 2 \alpha=10 \times \frac{2 \tan \alpha}{1+\tan ^2 \alpha}=10 \times \frac{2 \times 3}{1+3^2}=6$, 所以 $f(20 \sin \alpha \cos \alpha)=f(6)$, 由条件得 $f(6)=f(5+1)= f(1)=-f(-1)=-1$.
%%PROBLEM_END%%



%%PROBLEM_BEGIN%%
%%<PROBLEM>%%
问题9 函数 $f(x)=\frac{1}{\sec ^2 x}+\sin x$ 在 $|x| \leqslant \frac{\pi}{4}$ 上的最小值是 ( ).
(A) $\frac{\sqrt{2}-1}{2}$
(B) $-\frac{1+\sqrt{2}}{2}$
(C) -1
(D) $\frac{1-\sqrt{2}}{2}$
%%<SOLUTION>%%
D. 因 $f(x)=\frac{1}{\sec ^2 x}+\sin x=\cos ^2 x+\sin x=-\sin ^2 x+\sin x+1= -\left(\sin x-\frac{1}{2}\right)^2+\frac{5}{4}$, 当 $|x| \leqslant \frac{\pi}{4}$ 时, $f(x)_{\text {max }}=f\left(\frac{\pi}{6}\right)=\frac{5}{4}, f(x)_{min}=f\left(-\frac{\pi}{4}\right)=\frac{1-\sqrt{2}}{2}$
%%PROBLEM_END%%



%%PROBLEM_BEGIN%%
%%<PROBLEM>%%
问题10 下列四个命题中, 正确的是 ( ).
(A) 正切函数在整个定义域内是增函数
(B) 周期函数一定有最小正周期
(C) 函数 $y=3 \tan \sqrt{x^2}$ 的图象关于 $y$ 轴对称
(D) 若 $x$ 是第一象限的角, 则 $\sin x$ 是增函数, $\cos x$ 是减函数
%%<SOLUTION>%%
C. 正切函数在 $\left(k \pi-\frac{\pi}{2}, k \pi+\frac{\pi}{2}\right)(k \in \mathbf{Z})$ 上是增函数, 但不是在整个定义域上; 常数函数也是周期函数, 但无最小正周期; 若 $x$ 为第一象限角, 则在 $\left(2 k \pi, 2 k \pi+\frac{\pi}{2}\right)(k \in \mathbf{Z})$ 上 $\sin x$ 是增函数,而 $\cos x$ 是减函数,所以 $\mathrm{A}$ 、 B、D均错误.
又对 $f(x)=3 \tan \sqrt{x^2}$ 来说, $f(-x)=3 \tan \sqrt{(-x)^2}=f(x)$, 这是偶函数,所以其图象关于 $y$ 轴对称, 故选 C.
%%PROBLEM_END%%



%%PROBLEM_BEGIN%%
%%<PROBLEM>%%
问题11 将正弦曲线 $y=\sin (-x)$ 的图象向右平移 $\frac{\pi}{3}$ 个单位, 所得到的函数图象的解析式是 ; 将余弦曲线 $y=\cos (-2 x)$ 的图象向左平移 $\frac{\pi}{6}$ 个单位, 所得到的函数图象的解析式是
%%<SOLUTION>%%
$y=\sin \left(\frac{\pi}{3}-x\right), y=\cos \left(2 x+\frac{\pi}{3}\right)$.
%%PROBLEM_END%%



%%PROBLEM_BEGIN%%
%%<PROBLEM>%%
问题12 函数 $y=\sqrt{\cos x-2 \cos ^2 x}$ 的定义域是 , 值域是
%%<SOLUTION>%%
$\left[2 k \pi+\frac{\pi}{3}, 2 k \pi+\frac{\pi}{2}\right] \cup\left[2 k \pi-\frac{\pi}{2}, 2 k \pi-\frac{\pi}{3}\right](k \in \mathbf{Z}),\left[0, \frac{\sqrt{2}}{4}\right]$.
由条件得 $\cos x-2 \cos ^2 x \geqslant 0$, 即 $0 \leqslant \cos x \leqslant \frac{1}{2}$, 所以定义域为 $\left[2 k \pi+\frac{\pi}{3}, 2 k \pi+\frac{\pi}{2}\right] \cup\left[2 k \pi-\frac{\pi}{2}, 2 k \pi-\frac{\pi}{3}\right](k \in \mathbf{Z})$, 又 $y=\sqrt{-2\left(\cos x-\frac{1}{4}\right)^2+\frac{1}{8}}$, 故值域为 $\left[0, \frac{\sqrt{2}}{4}\right]$.
%%PROBLEM_END%%



%%PROBLEM_BEGIN%%
%%<PROBLEM>%%
问题13 函数 $y=\sin x\left(1+\tan \frac{x}{2}\right)$ 的最小正周期是
%%<SOLUTION>%%
$2 \pi$. 由原函数变换得 $y=\sin x+\sin x \cdot \frac{1-\cos x}{\sin x}=\sin x+1- \cos x=\sqrt{2} \sin \left(x-\frac{\pi}{4}\right)+1$, 故 $T=\frac{2 \pi}{\omega}=2 \pi$.
%%PROBLEM_END%%



%%PROBLEM_BEGIN%%
%%<PROBLEM>%%
问题14 已知 $f(x)=a \sin ^3 x+b \sqrt[3]{x} \cdot \cos ^3 x+4(a, b \in \mathbf{R})$, 且 $f\left(\sin 10^{\circ}\right)=5$, 则 $f\left(\cos 100^{\circ}\right)=$
%%<SOLUTION>%%
3. 由题意得 $f(x)+f(-x)=8$, 而 $\cos 100^{\circ}=-\sin 10^{\circ}$, 所以 $f\left(\sin 10^{\circ}\right)+f\left(-\sin 10^{\circ}\right)=8$, 于是 $f\left(\cos 100^{\circ}\right)=3$.
%%PROBLEM_END%%



%%PROBLEM_BEGIN%%
%%<PROBLEM>%%
问题15 函数 $f(x)=\sqrt{\sin 2 x+\sqrt{3} \cos 2 x-1}$ 的定义域是
%%<SOLUTION>%%
$\left[k \pi-\frac{\pi}{12}, k \pi+\frac{\pi}{4}\right](k \in \mathbf{Z})$. 由原函数得 $\sin 2 x+\sqrt{3} \cos 2 x-1 \geqslant 0$, 即 $2 \sin \left(2 x+\frac{\pi}{3}\right) \geqslant 1,2 k \pi+\frac{\pi}{6} \leqslant 2 x+\frac{\pi}{3} \leqslant 2 k \pi+\frac{5 \pi}{6}$, 解得 $k \pi-\frac{\pi}{12} \leqslant x \leqslant k \pi+\frac{\pi}{4}$, 所以定义域为 $\left[k \pi-\frac{\pi}{12}, k \pi+\frac{\pi}{4}\right](k \in \mathbf{Z})$.
%%PROBLEM_END%%



%%PROBLEM_BEGIN%%
%%<PROBLEM>%%
问题16 函数 $y=\sin \left(2 x+\frac{\pi}{3}\right)$ 图象的对称轴是
%%<SOLUTION>%%
$x=\frac{k \pi}{2}+\frac{\pi}{12}(k \in \mathbf{Z})$. 由 $2 x+\frac{\pi}{3}=k \pi+\frac{\pi}{2}$, 得对称轴 $x=\frac{k \pi}{2}+\frac{\pi}{12}$.
%%PROBLEM_END%%



%%PROBLEM_BEGIN%%
%%<PROBLEM>%%
问题17. 函数 $y=\tan \left(\frac{1}{2} x+\frac{\pi}{6}\right)$ 图象的对称中心是
%%<SOLUTION>%%
$\left(2 k \pi-\frac{\pi}{3}, 0\right)(k \in \mathbf{Z})$. 由 $\frac{1}{2} x+\frac{\pi}{6}=k \pi$, 得对称中心 $\left(2 k \pi-\frac{\pi}{3}, 0\right)$.
%%PROBLEM_END%%



%%PROBLEM_BEGIN%%
%%<PROBLEM>%%
问题18 设函数 $f(x)=\sin (\omega x+\varphi)\left(\omega>0,|\varphi|<\frac{\pi}{2}\right)$, 给出四个判断:
(1) 它的图象关于直线 $x=\frac{\pi}{12}$ 对称; (2) 它的图象关于点 $\left(\frac{\pi}{3}, 0\right)$ 对称;
(3) 它的最小正周期是 $\pi$; (4) 在区间 $\left[-\frac{\pi}{6}, 0\right]$ 上是增函数.
以其中两个论断作为条件, 另两个论断作为结论, 你认为正确的两个命题是
%%<SOLUTION>%%
(1)(3) $\Rightarrow$ (2)(4) 及 (2)(3) $\Rightarrow$ (1)(4). 若 (3) 成立, 则 $\omega=2$, 得 $f(x)=\sin (2 x+\varphi)$, 又 (1) 成立, 则 $x=\frac{\pi}{12}$ 时 $f(x)$ 有最大值或最小值, 即 $\sin \left(\frac{\pi}{6}+\varphi\right)= \pm 1$, 从而
$\varphi=\frac{\pi}{3}$, 所以 $f(x)=\sin \left(2 x+\frac{\pi}{3}\right)$, 它关于 $\left(\frac{\pi}{3}, 0\right)$ 成中心对称, 且在 $\left[-\frac{\pi}{6}, 0\right]$ 上单调递增.
若 (3) 成立, 则 $\omega=2$, 得 $f(x)=\sin (2 x+\varphi)$, 又 (2) 成立, 则当 $x=\frac{\pi}{3}$ 时, $f(x)=0$, 即 $\sin \left(\frac{2 \pi}{3}+\varphi\right)=0$. 从而 $\varphi=\frac{\pi}{3}$, 所以 $f(x)=\sin \left(2 x+\frac{\pi}{3}\right)$, 得 (1)(4) 成立.
%%PROBLEM_END%%



%%PROBLEM_BEGIN%%
%%<PROBLEM>%%
问题19 有四个函数: (1) $y=\sin ^2 x$; (2) $y=\sin |x|$; (3) $y=\tan \frac{x}{2}-\cot \frac{x}{2}$; (4) $y= |\sin x|$. 其中周期为 $\pi$, 且在 $\left(0, \frac{\pi}{2}\right)$ 上是增函数的为
%%<SOLUTION>%%
(1)(3). 因 $y=\sin ^2 x=\frac{1}{2}(1-\cos 2 x)$, 周期为 $\pi ; y=\tan \frac{x}{2}-\cot \frac{x}{2}= \frac{1-\cos x}{\sin x}-\frac{1+\cos x}{\sin x}=-\frac{2 \cos x}{\sin x}=-2 \cot x$. 其定义域为 $\{x \mid x \in \mathbf{R}$, 且 $x \neq k \pi, k \in \mathbf{Z}\}$, 周期为 $\pi$; 可以验证 (1)(3)(4) 符合题意.
%%PROBLEM_END%%



%%PROBLEM_BEGIN%%
%%<PROBLEM>%%
问题20. 方程 $\sin x=\frac{x}{100}$ 的实根个数有个.
%%<SOLUTION>%%
63. 由 $y=\sin x$ 和 $y=\frac{x}{100}$ 都是奇函数知, 它的正负根个数相同,在 $x>0$ 时, 由 $\frac{100}{\pi}=31.8$, 知在 $(0,100]$ 上两函数的图象有 31 个交点, 同理在 $[-100,0)$ 上两函数的图象有 31 个交点, 又 $x=0$ 是方程的根,所以共有 63 个实数根.
%%PROBLEM_END%%



%%PROBLEM_BEGIN%%
%%<PROBLEM>%%
问题21 作出函数 $y=\sqrt{3} \sin 2 x-\cos 2 x-1$ 在一个周期上的图象,并指出它与 $y=\sin x$ 的图象间的关系.
%%<SOLUTION>%%
由 $y=2 \sin \left(2 x-\frac{\pi}{6}\right)-1$ 知 $T=\pi$, 列表由五点描图法作出其图象, 该图象可看作将函数 $y=\sin x$ 的图象横坐标压缩为原来的一半得 $y=\sin 2 x$, 再向右平移 $\frac{\pi}{12}$ 个单位长度, 得 $y=\sin \left(2 x-\frac{\pi}{6}\right)$, 再将纵坐标扩大为原来的 2 倍, 得 $y=2 \sin \left(2 x-\frac{\pi}{6}\right)$, 最后, 将整个图象向下平移 1 个单位, 即得函数 $y= 2 \sin \left(2 x-\frac{\pi}{6}\right)-1$ 的图象.
(图略)
%%PROBLEM_END%%



%%PROBLEM_BEGIN%%
%%<PROBLEM>%%
问题22 已知 $f(x)=2 \sin \left(x+\frac{\theta}{2}\right) \cos \left(x+\frac{\theta}{2}\right)+2 \sqrt{3} \cos ^2\left(x+\frac{\theta}{2}\right)-\sqrt{3}$.
(1) 化简 $f(x)$ 的解析式;
(2) 若 $0 \leqslant \theta \leqslant \pi$, 求 $\theta$ 值,使函数 $f(x)$ 为偶函数;
(3) 在 (2) 的条件下,求满足 $f(x)=1$ 在 $[-\pi, \pi]$ 上的 $x$ 集合.
%%<SOLUTION>%%
(1) $f(x)=\sin (2 x+\theta)+\sqrt{3} \cos (2 x+\theta)=2 \cos \left(2 x+\theta-\frac{\pi}{6}\right)$.
(2) 要使 $f(x)$ 为偶函数, 只要满足 $\theta-\frac{\pi}{6}=k \pi$, 又 $0 \leqslant \theta \leqslant \pi$, 所以 $\theta=\frac{\pi}{6}$.
(3) 由 $f(x)=2 \cos 2 x=1$, 得 $2 x=2 k \pi \pm \frac{\pi}{3}$, 因为 $x \in[-\pi, \pi]$, 所以解集为 $\left\{x \mid x= \pm \frac{5 \pi}{6}, \pm \frac{\pi}{6}\right\}$.
%%PROBLEM_END%%



%%PROBLEM_BEGIN%%
%%<PROBLEM>%%
问题23 已知当 $x \in[0,1]$ 时,不等式 $x^2 \cos \theta-x(1-x)+(1-x)^2 \sin \theta>0$ 恒成立, 试求 $\theta$ 的取值范围.
%%<SOLUTION>%%
若对一切 $x \in[0,1]$, 恒有 $f(x)=x^2 \cos \theta-x(1-x)+(1-x)^2 \sin \theta>0$, 则 $\cos \theta=f(1)>0, \sin \theta=f(0)>0 \cdots$ (1). 取 $x_0=\frac{\sqrt{\sin \theta}}{\sqrt{\cos \theta}+\sqrt{\sin \theta}} \in(0,1)$, 则 $\sqrt{\cos \theta} x_0-\sqrt{\sin \theta}\left(1-x_0\right)=0$. 由于 $f(x)=[\sqrt{\cos \theta} x-\sqrt{\sin \theta}(1- x)]^2+2\left(-\frac{1}{2}+\sqrt{\cos \theta \sin \theta}\right) x(1-x)$. 所以, $0<f\left(x_0\right)=2\left(-\frac{1}{2}+\right. \sqrt{\cos \theta \sin \theta}) x_0\left(1-x_0\right)$. 故 $-\frac{1}{2}+\sqrt{\cos \theta \sin \theta}>0 \cdots$ (2). 反之, 当(1), (2)成立时, $f(0)=\sin \theta>0, f(1)=\cos \theta>0$, 且 $x \in(0,1)$ 时, $f(x) \geqslant 2\left(-\frac{1}{2}+\right. \sqrt{\cos \theta \sin \theta}) x(1-x)>0$. 先在 $[0,2 \pi]$ 中解(1)与(2), 由 $\cos \theta>0, \sin \theta>0$, 可得 $0<\theta<\frac{\pi}{2}$. 又因为 $-\frac{1}{2}+\sqrt{\cos \theta \sin \theta}>0, \sqrt{\cos \theta \sin \theta}>\frac{1}{2}, \sin 2 \theta> \frac{1}{2}$, 注意到 $0<2 \theta<\pi$, 故有 $\frac{\pi}{6}<2 \theta<\frac{5 \pi}{6}$. 所以, $\frac{\pi}{12}<\theta<\frac{5 \pi}{12}$. 因此, 原题中 $\theta$ 的取值范围是 $2 k \pi+\frac{\pi}{12}<\theta<2 k \pi+\frac{5 \pi}{12}, k \in \mathbf{Z}$.
%%PROBLEM_END%%



%%PROBLEM_BEGIN%%
%%<PROBLEM>%%
问题24. 已知函数 $f(x)=\sin (\omega x+\varphi)(\omega>0,0 \leqslant \varphi \leqslant \pi)$ 为 $\mathbf{R}$ 上的偶函数, 其图象关于点 $M\left(\frac{3 \pi}{4}, 0\right)$ 对称, 且在区间 $\left[0, \frac{\pi}{2}\right]$ 上是单调函数, 求 $\varphi$ 和 $\omega$ 的值.
%%<SOLUTION>%%
由 $f(x)$ 是偶函数, 得 $f(-x)=f(x)$, 即 $\sin (-\omega x+\varphi)=\sin (\omega x+ \varphi$ ), 所以 $-\cos \varphi \sin \omega x=\cos \varphi \sin \omega x$ 对任意 $x \in \mathbf{R}$ 都成立, 且 $\omega>0$, 所以 $\cos \varphi=0$. 依题设 $0 \leqslant \varphi \leqslant \pi$, 所以解得 $\varphi=\frac{\pi}{2}$. 由 $f(x)$ 的图象关于点 $M$ 对称, 得 $f\left(\frac{3 \pi}{4}-x\right)=f\left(\frac{3 \pi}{4}+x\right)$, 取 $x=0$, 得 $f\left(\frac{3 \pi}{4}\right)=\sin \left(\frac{3 \omega \pi}{4}+\frac{\pi}{2}\right)= \cos \frac{3 \omega \pi}{4}$, 所以 $f\left(\frac{3 \pi}{4}\right)=\sin \left(\frac{3 \omega \pi}{4}+\frac{\pi}{2}\right)=\cos \frac{3 \omega \pi}{4}$. 所以 $\cos \frac{3 \omega \pi}{4}=0$, 又 $\omega=0$, 得 $\frac{3 \omega \pi}{4}=\frac{\pi}{2}+k \pi, k=1,2,3, \cdots$. 所以 $\omega=\frac{2}{3}(2 k+1), k=0,1,2, \cdots$. 当 $k=0$ 时, $\omega=\frac{2}{3}, f(x)=\sin \left(\frac{2}{3} x+\frac{\pi}{2}\right)$ 在 $\left[0, \frac{\pi}{2}\right]$ 上是减函数; 当 $k=1$ 时, $\omega=2, f(x)=\sin \left(2 x+\frac{\pi}{2}\right)$ 在 $\left[0, \frac{\pi}{2}\right]$ 上是减函数; 当 $k \geqslant 2$ 时, $\omega \geqslant \frac{10}{3}$, $f(x)=\sin \left(\omega x+\frac{\pi}{2}\right)$ 在 $\left[0, \frac{\pi}{2}\right]$ 上不是单调函数; 所以,综上得 $\omega=\frac{2}{3}$ 或 $\omega=2$.
%%PROBLEM_END%%



%%PROBLEM_BEGIN%%
%%<PROBLEM>%%
问题25 正实数 $\alpha 、 \beta 、 a 、 b$ 满足条件 $\alpha<\beta, \alpha+\beta<\pi, a+b<\pi$ 并且 $\frac{\sin a}{\sin b} \leqslant \frac{\sin \alpha}{\sin \beta}$, 求证: $a<b$.
%%<SOLUTION>%%
在 $x>0, y>0, x+y<\pi$ 的条件下, $\sin x<\sin y$ 的充要条件是 $x<y$. 证明: 若 $0<x<y<\frac{\pi}{2}$, 则 $\sin x<\sin y$; 若 $\frac{\pi}{2}<y<\pi-\dot{x}$, 则 $\sin y>\sin (\pi-x)=\sin x$. 这样即得到当 $x<y$ 时均有 $\sin x<\sin y$. 反过来, 若 $\sin x<\sin y$, 则 $2 \cos \frac{x+y}{2} \sin \frac{x-y}{2}<0$, 而 $0<\frac{x+y}{2}<\frac{\pi}{2}$, 所以 $\sin \frac{y-x}{2}>0$, 而 $0<\frac{y-x}{2}<\pi$ 得到 $y>x$. 利用上述结论由 $0<\alpha<\beta, \alpha+\beta<\pi$ 可知 $\sin \alpha<\sin \beta$, 所以 $\frac{\sin a}{\sin b} \leqslant \frac{\sin \alpha}{\sin \beta}<1$, 即 $\sin a<\sin b$, 再用一次上述结论可得 $a<b$.
%%PROBLEM_END%%



%%PROBLEM_BEGIN%%
%%<PROBLEM>%%
问题26. (1) 求函数 $f(\theta)=\cos \frac{\theta}{2} \sin \theta, \theta \in\left(0, \frac{\pi}{2}\right)$ 的最大值;
(2) 求函数 $g(\theta)=\sin \frac{\theta}{2} \cos \theta, \theta \in\left(0, \frac{\pi}{2}\right)$ 的最大值.
%%<SOLUTION>%%
(1) $f(\theta)=2 \sin \frac{\theta}{2} \cdot \cos ^2 \frac{\theta}{2}=2 \sqrt{\sin ^2 \frac{\theta}{2} \cdot \cos ^4 \frac{\theta}{2}}=\sqrt{2}\sqrt{2 \sin ^2 \frac{\theta}{2} \cdot \cos ^2 \frac{\theta}{2} \cdot \cos ^2 \frac{\theta}{2}} \leqslant \sqrt{2} \cdot \sqrt{\left[\frac{2 \sin ^2 \frac{\theta}{2}+\cos ^2 \frac{\theta}{2}+\cos ^2 \frac{\theta}{2}}{3}\right]^3}=\sqrt{2} \cdot \left(\frac{2}{3}\right)^{\frac{3}{2}}=\frac{4 \sqrt{3}}{9}$, 当且仅当 $\theta=2 \arctan \frac{\sqrt{2}}{2}$ 时取等号.
所以, $f(\theta)$ 的最大值为 $\frac{4 \sqrt{3}}{9}$.
(2) 由于 $\theta \in\left(0, \frac{\pi}{2}\right)$, 故 $g(\theta)>0, g^2(\theta)=\sin ^2 \frac{\theta}{2} \cdot \cos ^2 \theta=\frac{1-\cos \theta}{2}$. $\cos ^2 \theta=\frac{2(1-\cos \theta) \cdot \cos \theta \cdot \cos \theta}{4} \leqslant \frac{1}{4} \cdot\left[\frac{2(1-\cos \theta)+\cos \theta+\cos \theta}{3}\right]^3= \frac{2}{27}$, 当且仅当 $\theta=\arccos \frac{2}{3}$ 时取等号.
所以, $g(\theta)$ 的最大值为 $\frac{\sqrt{6}}{9}$. 当然也可以如下这样, $g(\theta)=\sin \frac{\theta}{2} \cdot\left(1-2 \sin ^2 \frac{\theta}{2}\right)=\frac{1}{2} \sqrt{4 \sin ^2 \frac{\theta}{2} \cdot\left(1-2 \sin ^2 \frac{\theta}{2}\right) \cdot\left(1-2 \sin ^2 \frac{\theta}{2}\right)} \leqslant \frac{1}{2} \cdot \sqrt{\left[\frac{4 \sin ^2 \frac{\theta}{2}+\left(1-2 \sin ^2 \frac{\theta}{2}\right)+\left(1-2 \sin ^2 \frac{\theta}{2}\right)}{3}\right]^3}=\frac{\sqrt{6}}{9}$. 当且仅当 $\theta=2 \arcsin \frac{\sqrt{6}}{6}$ 时取等号.
故 $g(\theta)$ 的最大值为 $\frac{\sqrt{6}}{9}$.
%%PROBLEM_END%%



%%PROBLEM_BEGIN%%
%%<PROBLEM>%%
问题27 教室的墙壁上挂着一块黑板, 它的上下边缘分别在学生的水平视线上方 $a$ 米和 $b$ 米, 问学生距墙壁多远时看黑板的视角最大?
%%<SOLUTION>%%
设学生 $P$ 距墙壁 $x$ 米,黑板上、下边缘与学生 $P$ 的水平视线 $P H$ 夹角分别为 $\angle A P H=\alpha, \angle B P H=\beta$, 其中 $\alpha>\beta$, 有 $\tan \alpha=\frac{a}{x}, \tan \beta=\frac{b}{x}$, 则 $\tan (\alpha-\beta)=\frac{\tan \alpha-\tan \beta}{1+\tan \alpha \tan \beta}=\frac{\frac{a}{x}-\frac{b}{x}}{1+\frac{a b}{x^2}} \leqslant \frac{a-b}{2 \sqrt{x \cdot \frac{a b}{x}}}=\frac{a-b}{2 \sqrt{a b}}$, 当且仅当 $x= \frac{a b}{x}$ 即 $x=\sqrt{a b}$ 时, $\tan (\alpha-\beta)$ 取最大值 $\frac{a-b}{2 \sqrt{a b}}$, 由于 $\alpha-\beta$ 为锐角, 故此时 $\alpha-\beta$ 也最大, 即学生距墙壁 $\sqrt{a b}$ 米时看黑板的视角最大, 最大视角为 $\arctan \frac{a-b}{2 \sqrt{a b}}$.
%%PROBLEM_END%%



%%PROBLEM_BEGIN%%
%%<PROBLEM>%%
问题28 已知函数 $f(x)=\tan x, x \in\left(0, \frac{\pi}{2}\right)$, 若 $x_1, x_2 \in\left(0, \frac{\pi}{2}\right)$, 且 $x_1 \neq x_2$, 求证: $\frac{1}{2}\left[f\left(x_1\right)+f\left(x_2\right)\right]>f\left(\frac{x_1+x_2}{2}\right)$.
%%<SOLUTION>%%
要证明 $\frac{1}{2}\left[f\left(x_1\right)+f\left(x_2\right)\right]>f\left(\frac{x_1+x_2}{2}\right) \Leftrightarrow \frac{1}{2}\left[\tan x_1+\tan x_2\right]> \tan \frac{x_1+x_2}{2} \Leftrightarrow \tan \frac{x_1+x_2}{2}-\tan x_1<\tan x_2-\tan \frac{x_1+x_2}{2} \Leftrightarrow \tan \frac{x_2-x_1}{2}\left(1+\tan \frac{x_1+x_2}{2} \cdot \tan x_1\right)<\tan \frac{x_2-x_1}{2} \cdot\left(1+\tan \frac{x_1+x_2}{2} \cdot \tan x_2\right)$, 不妨设 $x_1<x_2$, 则上式等价于 $\tan x_1<\tan x_2$. 显然成立.
%%PROBLEM_END%%



%%PROBLEM_BEGIN%%
%%<PROBLEM>%%
问题29 试求正整数 $k$, 使 $f(x)=\sin k x \cdot \sin ^k x+\cos k x \cdot \cos ^k x-\cos ^k 2 x$ 的值不依赖于 $x$.
%%<SOLUTION>%%
若 $f(x)$ 的值不依赖于 $x$, 即 $f(x)$ 为常数.
由 $f(0)=0$, 得 $f(x)=0$, 取 $x=\frac{\pi}{k}$, 得 $f\left(\frac{\pi}{k}\right)=-\cos ^k \frac{\pi}{k}-\cos ^k \frac{2 \pi}{k}=0$, 若 $k$ 为正偶数, 则 $\cos ^k \frac{\pi}{k} \geqslant 0$, $\cos ^k \frac{2 \pi}{k} \geqslant 0$, 从而 $\cos \frac{\pi}{k}=\cos \frac{2 \pi}{k}=0$, 此时无解; 若 $k$ 为正奇数, 则由 $\cos ^k \frac{\pi}{k}= -\cos ^k \frac{2 \pi}{k}$, 得 $\cos \frac{\pi}{k}=-\cos \frac{2 \pi}{k}$, 所以 $\frac{2 \pi}{k}=2 n \pi \pm\left(\pi-\frac{\pi}{k}\right)$,即 $\frac{3}{k}=2 n \neq 1$ 或 $\frac{1}{k}=2 n-1$,故 $k=\frac{3}{2 n+1}$ 或 $\frac{1}{2 n-1}$, 从而 $k=1$ 或 3 , 但 $k=1$ 时, $f(x)$ 不为常数, $k=3$ 时, $f(x)$ 为常数, 所以 $k=3$.
%%PROBLEM_END%%


