
%%PROBLEM_BEGIN%%
%%<PROBLEM>%%
问题1 下列命题恒成立的是 ( ).
(A) $\arctan x=\operatorname{arccot} \frac{1}{x}(x \neq 0)$
(B) $\arcsin x=\arccos \sqrt{1-x^2} \quad(x \in[-1,1])$
(C) $\arctan x+\operatorname{arccot} x=\pi$
(D) $\operatorname{arccot} x+\operatorname{arccot}(-x)=\pi$
%%<SOLUTION>%%
D. 当 $x<0$ 时, $\arctan x \in\left(-\frac{\pi}{2}, 0\right), \operatorname{arccot} \frac{1}{x} \in\left(\frac{\pi}{2}, \pi\right)$. 故 A 错误, 同理 B 错误, 而 C 显然错误.
%%PROBLEM_END%%



%%PROBLEM_BEGIN%%
%%<PROBLEM>%%
问题2. 若 $(a+1)(b+1)=2$, 则 $\arctan a+\arctan b=(\quad)$.
(A) $\frac{\pi}{4}$
(B) $\frac{5 \pi}{4}$
(C) $\frac{\pi}{4}$ 或 $\frac{5 \pi}{4}$
(D) $\frac{\pi}{4}$ 或 $-\frac{3 \pi}{4}$
%%<SOLUTION>%%
D. 因 $\tan (\arctan a+\arctan b)=\frac{a+b}{1-a b}=1$, 又 $\arctan a+\arctan b \in (-\pi, \pi)$, 所以选 D.
%%PROBLEM_END%%



%%PROBLEM_BEGIN%%
%%<PROBLEM>%%
问题3 若 $\arcsin x+\arctan \frac{1}{7}=\frac{\pi}{4}$, 则 $x$ 等于 ( ).
(A) $\frac{6}{7}$
(B) $\frac{3}{5}$
(C) $\frac{5}{6}$
(D) $\frac{3}{7}$
%%<SOLUTION>%%
B. 由已知等式得 $x=\sin \left(\frac{\pi}{4}-\arctan \frac{1}{7}\right)=\frac{\sqrt{2}}{2} \times \frac{7}{5 \sqrt{2}}-\frac{\sqrt{2}}{2} \times \frac{1}{5 \sqrt{2}}= \frac{3}{5}$, 故选 B.
%%PROBLEM_END%%



%%PROBLEM_BEGIN%%
%%<PROBLEM>%%
问题4. 记 $a=\arccos \left(-\frac{1}{3}\right), b=\arccos \frac{1}{4}, c=\arctan 1$, 则它们的关系是( ).
(A) $a<b<c$
(B) $a<c<b$
(C) $b<c<a$
(D) $c<b<a$
%%<SOLUTION>%%
D. 显然 $a$ 最大, 而 $c=\arctan 1=\frac{\pi}{4}, b=\arccos \frac{1}{4}=\arctan \frac{4}{\sqrt{15}}> \frac{\pi}{4}$, 故选 D.
%%PROBLEM_END%%



%%PROBLEM_BEGIN%%
%%<PROBLEM>%%
问题5. 已知 $x \in\left(\frac{\pi}{4}, \frac{5 \pi}{4}\right)$, 则 $\arcsin \frac{\sin x+\cos x}{\sqrt{2}}$ 的值为 ( ).
(A) $x+\frac{\pi}{4}$
(B) $\frac{\pi}{2}-x$
(C) $x-\frac{3 \pi}{4}$
(D) $\frac{3 \pi}{4}-x$
%%<SOLUTION>%%
D. $\arcsin \frac{\sin x+\cos x}{\sqrt{2}}=\arcsin \left[\sin \left(\frac{\pi}{4}+x\right)\right]$, 因 $x \in\left(\frac{\pi}{4}, \frac{5 \pi}{4}\right)$, 所以 $\frac{\pi}{4}+x \in\left(\frac{\pi}{2}, \frac{3 \pi}{2}\right)$,故原式 $=\pi-\left(\frac{\pi}{4}+x\right)=\frac{3 \pi}{4}-x$.
%%PROBLEM_END%%



%%PROBLEM_BEGIN%%
%%<PROBLEM>%%
问题6. 设 $M=\{(x, y)|| x y \mid=1, x>0\}, N=\{(x, y) \mid \arctan x+ \operatorname{arccot} y=\pi\}$, 那么 ( ).
(A) $M \cup N=\{(x, y)|| x y \mid=1\}$
(B) $M \cup N=M$
(C) $M \cup N=N \cup\{(0,0)\}$
(D) $M \cup N=\{(x, y)|| x y \mid=1$, 且 $x 、 y$ 不同时为负数 $\}$
%%<SOLUTION>%%
B. 在集合 $M$ 中,由 $|x y|=1$ 得 $x y=1$ 或 $x y=-1$ 但 $x>0$,故表示反比例函数在 $\mathrm{I} 、 \mathrm{IN}$ 象限的两支, 在集合 $N$ 中, 由 $\arctan x+\operatorname{arccot} y= \pi \Rightarrow \arctan x=\pi-\operatorname{arccot} y$, 所以 $x=\tan (\arctan x)=\tan (\pi-\operatorname{arccot} y)= -\frac{1}{\cot (\operatorname{arccot} y)}=-\frac{1}{y}$, 即 $x y=-1$, 但当 $x<0$ 时, $-\frac{\pi}{2}<\arctan x<0$, 此时 $y>0,0<\operatorname{arccot} y<\frac{\pi}{2}$, 故 $-\frac{\pi}{2}<\arctan x+\operatorname{arccot} y<\frac{\pi}{2}$, 这与
$\arctan x+\operatorname{arccot} y=\pi$, 矛盾.
所以 $N=\{(x, y) \mid x y=-1, x>0\} \varsubsetneqq\{(x$, $y) \mid x y=1$ 或 $x y=-1, x>0\}=M$, 故 $M \cup N=M$.
%%PROBLEM_END%%



%%PROBLEM_BEGIN%%
%%<PROBLEM>%%
问题7 已知方程 $\arccos \frac{4}{5}-\arccos \left(-\frac{4}{5}\right)=\arcsin x$, 则 ( ).
(A) $x=\frac{24}{25}$
(B) $x=-\frac{24}{25}$
(C) $x=0$
(D) 无解
%%<SOLUTION>%%
D. 因为 $\arccos \left(-\frac{4}{5}\right)=\pi-\arccos \frac{4}{5}$, 原方程化为 $2 \arccos \frac{4}{5}- \arcsin x=\pi$, 因为 $\arccos \frac{4}{5}<\arccos \frac{\sqrt{2}}{2}=\frac{\pi}{4}$, 又 $-\arcsin x \leqslant \frac{\pi}{2}$, 所以 $2 \arccos \frac{4}{5}-\arcsin x<\pi$, 故无解.
%%PROBLEM_END%%



%%PROBLEM_BEGIN%%
%%<PROBLEM>%%
问题8. 方程 $\frac{\cos 2 x}{1+\sin 2 x}=0$ 的解集是 ( ).
(A) $\left\{x \mid x=2 k \pi \pm \frac{\pi}{4}, k \in \mathbf{Z}\right\}$
(B) $\left\{x \mid x=k \pi \pm \frac{\pi}{4}, k \in \mathbf{Z}\right\}$
(C) $\left\{x \mid x=k \pi+\frac{\pi}{4}, k \in \mathbf{Z}\right\}$
(D) $\left\{x \mid x=\frac{1}{2} k \pi+\frac{\pi}{4}, k \in \mathbf{Z}\right\}$
%%<SOLUTION>%%
C. 由原方程得 $2 x=k \pi+\frac{\pi}{2}$ 且 $2 x \neq 2 k \pi+\frac{3 \pi}{2}$, 即 $x=\frac{1}{2} k \pi+\frac{\pi}{4}$ 且 $x \neq k \pi+\frac{3 \pi}{4}$, 故选 C.
%%PROBLEM_END%%



%%PROBLEM_BEGIN%%
%%<PROBLEM>%%
问题9. 方程 $\lg x=\cos 2 x$ 的解有( ).
(A) 2 个
(B) 5 个
(C) 7 个
(D) 无数个
%%<SOLUTION>%%
B. 分别作出 $y=\lg x$ 与 $y=\cos 2 x$ 的图象而观察.
%%PROBLEM_END%%




%%PROBLEM_BEGIN%%
%%<PROBLEM>%%
问题11 函数 $y=\arccos \left(-2 x^2+x\right)$ 的值域是
%%<SOLUTION>%%
$\left[\arccos \frac{1}{8}, \pi\right]$. 因 $-2 x^2+x=-2\left(x-\frac{1}{4}\right)^2+\frac{1}{8} \leqslant \frac{1}{8}$, 所以其值域为 $\left[\arccos \frac{1}{8}, \pi\right]$.
%%PROBLEM_END%%



%%PROBLEM_BEGIN%%
%%<PROBLEM>%%
问题12 函数 $y=\sin x, x \in\left[\pi, \frac{3}{2} \pi\right]$ 的反函数是
%%<SOLUTION>%%
$f^{-1}(x)=\pi-\arcsin x(-1 \leqslant x \leqslant 0)$.
%%PROBLEM_END%%



%%PROBLEM_BEGIN%%
%%<PROBLEM>%%
问题13 不等式 $\arccos x>\arcsin x$ 的解集是
%%<SOLUTION>%%
$\left[-1, \frac{\sqrt{2}}{2}\right)$. 显然 $-1 \leqslant x \leqslant 0$ 时, 原不等式恒成立; 当 $x>0$ 时, 由 $\arccos x>\arcsin x$, 得 $\sqrt{1-x^2}>x$, 即 $x^2<\frac{1}{2}$, 所以 $0<x<\frac{\sqrt{2}}{2}$, 综上所述, $x \in\left[-1, \frac{\sqrt{2}}{2}\right)$.
%%PROBLEM_END%%



%%PROBLEM_BEGIN%%
%%<PROBLEM>%%
问题14. 函数 $f(x)=\arcsin (2 x+1)(-1 \leqslant x \leqslant 0)$, 则 $f^{-1}\left(\frac{\pi}{6}\right)=$
%%<SOLUTION>%%
$-\frac{1}{4}$. 由 $\frac{\pi}{6}=\arcsin (2 x+1)$, 得 $\frac{1}{2}=2 x+1, x=-\frac{1}{4}$.
%%PROBLEM_END%%



%%PROBLEM_BEGIN%%
%%<PROBLEM>%%
问题15 已知函数 $f(x)=4 \pi \arcsin x-[\arccos (-x)]^2$ 的最大值为 $M$, 最小值为 $m$, 则 $M-m=$
%%<SOLUTION>%%
$3 \pi^2$. 因为 $\arccos (-x)=\pi-\arccos x, \arcsin x=\frac{\pi}{2}-\arccos x$, 所以 $f(x)=4 \pi\left(\frac{\pi}{2}-\arccos x\right)-(\pi-\arccos x)^2=-(\arccos x)^2-2 \pi \arccos x+ \pi^2=-(\arccos x+\pi)^2+2 \pi^2$, 又因为 $0 \leqslant \arccos x \leqslant \pi$, 所以 $M=\pi^2$,
$m=-2 \pi^2 \Rightarrow M-m=3 \pi^2$.
%%PROBLEM_END%%



%%PROBLEM_BEGIN%%
%%<PROBLEM>%%
问题16. 方程 $\sin x=\cos \frac{2}{5} \pi$ 的解集为
%%<SOLUTION>%%
$\left\{x \mid x=k \pi+(-1)^k \cdot \frac{1}{10} \pi, k \in \mathbf{Z}\right\}$.
%%PROBLEM_END%%



%%PROBLEM_BEGIN%%
%%<PROBLEM>%%
问题17 设 $[\tan x]$ 表示不超过实数 $\tan x$ 的最大整数, 则方程 $[\tan x]=2 \cos ^2 x$ 的解为
%%<SOLUTION>%%
$x=k \pi+\frac{\pi}{4}, k \in \mathbf{Z}$. 因为 $2 \cos ^2 x \in[0,2]$, 又 $[\tan x]$ 表示整数,故 $[\tan x]=0$ 时, $\cos x=0$ (舍). [ $\tan x]=2$ 时, $\cos ^2 x=1$ (舍). [ $\left.\tan x\right]=1$ 时, $\cos x= \pm \frac{\sqrt{2}}{2}, x= \pm \frac{\pi}{4}+k \pi, k \in \mathbf{Z}$. 验证知 $x=k \pi+\frac{\pi}{4}, k \in \mathbf{Z}$.
%%PROBLEM_END%%



%%PROBLEM_BEGIN%%
%%<PROBLEM>%%
问题18. 方程 $2 \cos ^2 x-\sin x \cos x=\frac{1}{2}$ 的解集为
%%<SOLUTION>%%
$\left\{x \mid x=\frac{k \pi}{2}+\frac{(-1)^k}{2} \arcsin \frac{1}{\sqrt{5}}+\frac{1}{2} \arcsin \frac{2}{\sqrt{5}}, k \in \mathbf{Z}\right\}$.
%%PROBLEM_END%%



%%PROBLEM_BEGIN%%
%%<PROBLEM>%%
问题19 若方程 $\sin x=a$ 在 $\left[\frac{2 \pi}{3}, \frac{5 \pi}{3}\right]$ 中恰有两个不同的解, 则 $a$ 的取值范围是
%%<SOLUTION>%%
$a \in\left(-1,-\frac{1}{2}\right]$.
%%PROBLEM_END%%



%%PROBLEM_BEGIN%%
%%<PROBLEM>%%
问题20 若 $-6<\log _{\frac{1}{\sqrt{2}}} x<-2$, 则方程 $\cos \pi x=1$ 的解集为
%%<SOLUTION>%%
$\{4,6\}$.
%%PROBLEM_END%%



%%PROBLEM_BEGIN%%
%%<PROBLEM>%%
问题21 计算:(1) $\tan \left(2 \arcsin \frac{5}{13}\right)+\tan \left(\frac{1}{2} \arctan \frac{3}{4}\right)$;
(2) $\sin \left(\arcsin \frac{3}{5}+\arccos \frac{4}{5}\right)$.
%%<SOLUTION>%%
(1) 原式 $=\frac{2 \times \frac{5}{12}}{1-\left(\frac{5}{12}\right)^2}+\frac{1-\frac{4}{5}}{\frac{3}{5}}=\frac{120}{119}+\frac{1}{3}=\frac{479}{357}$. $\quad$ (2) 原式 $= 2 \times \frac{3}{5} \times \frac{4}{5}=\frac{24}{25}$
%%PROBLEM_END%%



%%PROBLEM_BEGIN%%
%%<PROBLEM>%%
问题22. 计算: (1) $\arcsin \left(\sin \frac{3 \pi}{4}\right)+\arccos \left(\cos \frac{3 \pi}{4}\right)$;
(2) $\arctan \frac{1}{2}+\arctan \frac{1}{5}+\arctan \frac{1}{8}$.
%%<SOLUTION>%%
(1) 原式 $=\pi-\frac{3 \pi}{4}+\frac{3 \pi}{4}=\pi$. (2) 原式 $=\frac{\pi}{4}$.
%%PROBLEM_END%%



%%PROBLEM_BEGIN%%
%%<PROBLEM>%%
问题23 (1) 求证: $2 \arctan \frac{1}{3}+\arctan \frac{1}{7}=\frac{\pi}{4}$;
(2) 如图(<FilePath:./figures/fig-c4p23.png>),三个相同的正方形相接, 求证: $\alpha+\beta=45^{\circ}$.
%%<SOLUTION>%%
(1) 参照例 8 可证之.
(2) 由条件得 $\tan \alpha=\frac{1}{3}, \tan \beta=\frac{1}{2}$, 于是 $\tan (\alpha+\beta)=\frac{\frac{1}{2}+\frac{1}{3}}{1-\frac{1}{2} \times \frac{1}{3}}=\frac{5}{5}=1$, 所以 $\alpha+\beta=45^{\circ}$.
%%PROBLEM_END%%



%%PROBLEM_BEGIN%%
%%<PROBLEM>%%
问题24 求下列函数的定义域:
(1) $y=\sqrt{\frac{2 \pi}{3}-\arccos \left(\frac{1}{2} x-1\right)}$;
(2) $y=\arctan \frac{1}{x^2-1}$.
%%<SOLUTION>%%
(1) 由不等式 $\frac{2 \pi}{3} \geqslant \arccos \left(\frac{1}{2} x-1\right)$, 得 $1 \geqslant \frac{1}{2} x-1 \geqslant-\frac{1}{2}$, 即 $1 \leqslant x \leqslant 4$, 定义域为 $[1,4]$. (2) 由 $x^2 \neq 1$ 得 $x \neq \pm 1$, 定义域为 $\{x \mid x \neq \pm 1\}$.
%%PROBLEM_END%%



%%PROBLEM_BEGIN%%
%%<PROBLEM>%%
问题25 求函数 $f(x)=\arcsin \left(x-x^2\right)$ 的定义域和值域与单调递增区间.
%%<SOLUTION>%%
定义域为 $\left[\frac{1-\sqrt{5}}{2}, \frac{1+\sqrt{5}}{2}\right]$, 值域为 $\left[-\frac{\pi}{2}, \arcsin \frac{1}{4}\right]$, 单调增区间为 $\left[\frac{1-\sqrt{5}}{2}, \frac{1}{2}\right]$.
%%PROBLEM_END%%



%%PROBLEM_BEGIN%%
%%<PROBLEM>%%
问题26 若 $x_1 、 x_2$ 是方程 $x^2-6 x+7=0$ 的两根, 求 $\arctan x_1+\arctan x_2$ 的值.
%%<SOLUTION>%%
参照例 11 , 得原式 $=\frac{3 \pi}{4}$.
%%PROBLEM_END%%



%%PROBLEM_BEGIN%%
%%<PROBLEM>%%
问题27 已知函数 $y=\cos (2 \arccos x)+4 \sin \left(\arcsin \frac{x}{2}\right)$, 求它的最大值与最小值.
%%<SOLUTION>%%
$y=2 x^2-1+2 x=2\left(x+\frac{1}{2}\right)^2-\frac{3}{2}$, 因 $x \in[-1,1]$, 所以 $y_{\text {max }}=3, y_{\min }=-\frac{3}{2}$.
%%PROBLEM_END%%



%%PROBLEM_BEGIN%%
%%<PROBLEM>%%
问题28 解方程:
(1) $\cos ^2\left(x+\frac{5}{8} \pi\right)-\sin 4 x=2$;
(2) $\sqrt{1+\sin x}-\sqrt{1-\sin x}=2 \cos x$;
(3) $\tan \left(\frac{x}{2}+\frac{\pi}{4}\right)+\sin x+1=0$;
(4) $\frac{\sin 2 x}{\cos x}=\frac{\cos 2 x}{\sin x}$.
%%<SOLUTION>%%
 (1)因 $\cos ^2\left(x+\frac{5}{8} \pi\right)-\sin 4 x \leqslant 2$, 所以只有当 $\cos ^2\left(x+\frac{5}{8} \pi\right)=1$ 且 $\sin 4 x=-1$ 时原方程成立.
但这是不可能的,所以原方程无解.
(2) 原方程可化为 $\left|\sin \frac{x}{2}+\cos \frac{x}{2}\right|-\left|\sin \frac{x}{2}-\cos \frac{x}{2}\right|=2 \cos x$, 由此可解得 $x=k \pi+ \frac{\pi}{3}(k \in \mathbf{Z})$. (3) 原方程化为 $\tan \left(\frac{x}{2}+\frac{\pi}{4}\right)-\cos \left(x+\frac{\pi}{2}\right)+1=0$, 从而 $\tan \left(\frac{x}{2}+\frac{\pi}{4}\right)-1+2 \sin ^2\left(\frac{x}{2}+\frac{\pi}{4}\right)+1=0$. 所以 $\tan \left(\frac{x}{2}+\frac{\pi}{4}\right)+2 \sin ^2\left(\frac{x}{2}+\right. \left.\frac{\pi}{4}\right)=0$, 即 $\sin \left(\frac{x}{2}+\frac{\pi}{4}\right) \cdot \frac{1+\sin \left(x+\frac{\pi}{2}\right)}{\cos \left(\frac{x}{2}+\frac{\pi}{4}\right)}=0$, 故 $\sin \left(\frac{x}{2}+\frac{\pi}{4}\right)=0$ 或 $\cos x=-1$. 从而解集为 $\left\{x \mid x=\frac{3 \pi}{2}+2 k \pi, k \in \mathbf{Z}\right\} \bigcup\{x \mid x=(2 k+1) \pi$, $k \in \mathbf{Z}\}$. (4) 原方程化为 $2 \sin ^2 x=1-2 \sin ^2 x$ 且 $\sin 2 x \neq 0$, 所以解集为 $\left\{x \mid x=\frac{\pi}{6}+\frac{k \pi}{3}, k \in \mathbf{Z}\right.$, 且 $\left.k \neq \frac{3 p-1}{2}, p \in \mathbf{Z}\right\}$.
%%PROBLEM_END%%



%%PROBLEM_BEGIN%%
%%<PROBLEM>%%
问题29 求下列方程的实数解.
$$
\arccos \left|\frac{x^2-1}{x^2+1}\right|+\arcsin \left|\frac{2 x}{x^2+1}\right|+\arccos \left|\frac{x^2-1}{2 x}\right|=\pi .
$$
%%<SOLUTION>%%
令 $x=\tan \alpha, \alpha \in\left(-\frac{\pi}{2}, \frac{\pi}{2}\right)$, 则原方程变为 $\arccos |\cos 2 \alpha|+ \arcsin |\sin 2 \alpha|+\operatorname{arccot}|\cot 2 \alpha|=\pi$. 当 $|\tan \alpha| \geqslant 1$ 时, $-\frac{\pi}{2}<\alpha \leqslant-\frac{\pi}{4}$ 或 $\frac{\pi}{4} \leqslant \alpha<\frac{\pi}{2}$. 若 $-\frac{\pi}{2}<\alpha \leqslant-\frac{\pi}{4}$, 则原方程化为 $\pi+2 \alpha+\pi+2 \alpha+\pi+2 \alpha= \pi \Rightarrow \alpha=-\frac{\pi}{3}, x=-\sqrt{3}$. 类似地, $\frac{\pi}{4} \leqslant \alpha<\frac{\pi}{2}$ 时, $\alpha=\frac{\pi}{3}, x=\sqrt{3}$. 当 $|\tan \alpha|<$ 1 时, $-\frac{\pi}{4}<\alpha<0$ 或 $0<\alpha<\frac{\pi}{4}$. 同样可得 $\alpha=-\frac{\pi}{6}, x=-\frac{\sqrt{3}}{3}$ 或 $\alpha=\frac{\pi}{6}, x= \frac{\sqrt{3}}{3}$. 综上, 原方程的解为 $x= \pm \sqrt{3}$ 或 $x= \pm \frac{\sqrt{3}}{3}$.
%%PROBLEM_END%%



%%PROBLEM_BEGIN%%
%%<PROBLEM>%%
问题30. 解方程: $\arccos x+\arctan \frac{1}{7}=\frac{\pi}{4}$.
%%<SOLUTION>%%
因 $\cos \left(\frac{\pi}{4}-\arctan \frac{1}{7}\right)=\frac{\sqrt{2}}{2} \times \frac{7}{5 \sqrt{2}}+\frac{\sqrt{2}}{2} \times \frac{1}{5 \sqrt{2}}=\frac{4}{5}$, 所以原方程的解为 $x=\frac{4}{5}$.
%%PROBLEM_END%%


