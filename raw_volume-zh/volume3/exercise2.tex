
%%PROBLEM_BEGIN%%
%%<PROBLEM>%%
问题1 已知 $\frac{1+\sin \theta+\cos \theta}{1+\sin \theta-\cos \theta}=\frac{1}{2}$, 则 $\cos \theta$ 的值等于 ( ).
(A) $\frac{3}{5}$
(B) $-\frac{3}{5}$
(C) $-\frac{\sqrt{5}}{5}$
(D) $\frac{4}{5}$
%%<SOLUTION>%%
B. 因 $\frac{1+\sin \theta+\cos \theta}{1+\sin \theta-\cos \theta}=\frac{1}{2}$, 由合分比性质得 $\frac{2 \cos \theta}{2+2 \sin \theta}=\frac{-1}{3}$, 故 $\cos \theta=\frac{1-\tan ^2 \frac{\theta}{2}}{1+\tan ^2 \frac{\theta}{2}}=\frac{1-2^2}{1+2^2}=-\frac{3}{5}$.
%%PROBLEM_END%%



%%PROBLEM_BEGIN%%
%%<PROBLEM>%%
问题2. 化简: $2 \sqrt{1+\sin 8}+\sqrt{2+2 \cos 8}$ 的结果( ).
(A) $2 \sin 4$
(B) $2 \sin 4+4 \cos 4$
(C) $-2 \sin 4-4 \cos 4$
(D) $2 \sin 4-4 \cos 4$
%%<SOLUTION>%%
C. $2 \sqrt{1+\sin 8}+\sqrt{2+2 \cos 8}=2 \sqrt{\sin ^2 4+\cos ^2 4+2 \sin 4 \cos 4}+ \sqrt{2+2\left(2 \cos ^2 4-1\right)}=2|\sin 4+\cos 4|+|2 \cos 4|=-2 \sin 4-2 \cos 4- 2 \cos 4=-2 \sin 4-4 \cos 4$.
%%PROBLEM_END%%



%%PROBLEM_BEGIN%%
%%<PROBLEM>%%
问题3 计算 $\sin 10^{\circ} \sin 30^{\circ} \sin 50^{\circ} \sin 70^{\circ}$ 的值是 ( ).
(A) $\frac{1}{2}$
(B) $\frac{1}{4}$
(C) $\frac{1}{8}$
(D) $\frac{1}{16}$
%%<SOLUTION>%%
D. 参照例 4 可得原式 $=\frac{1}{16}$.
%%PROBLEM_END%%



%%PROBLEM_BEGIN%%
%%<PROBLEM>%%
问题4 化简 $\frac{2 \cos ^2 \alpha-1}{2 \operatorname{sen}\left(\frac{\pi}{4}-\alpha\right) \sin ^2\left(\frac{\pi}{4}+\alpha\right)}$ 等于( ).
(A) 1
(B) -1
(C) $\cos \alpha$
(D) $-\sin \alpha$
%%<SOLUTION>%%
$\mathrm{A}$ 原式 $=\frac{\cos 2 \alpha}{2 \cdot \frac{1-\tan \alpha}{1+\tan \alpha} \cdot \frac{1-\cos \left(\frac{\pi}{2}+2 \alpha\right)}{2}}=\frac{\cos ^2 \alpha-\sin ^2 \alpha}{\frac{\cos \alpha-\sin \alpha}{\cos \alpha+\sin \alpha} \cdot(1+\sin 2 \alpha)}= \frac{(\cos \alpha+\sin \alpha)^2}{1+\sin 2 \alpha}=1$.
%%PROBLEM_END%%



%%PROBLEM_BEGIN%%
%%<PROBLEM>%%
问题5 设 $a=\sin 14^{\circ}+\cos 14^{\circ}, b=\sin 16^{\circ}+\cos 16^{\circ}, c=\frac{\sqrt{6}}{2}$, 则 $a 、 b 、 c$ 的大小关系是 ( ).
(A) $a<b<c$
(B) $a<c<b$
(C) $b<c<a$
(D) $b<a<c$
%%<SOLUTION>%%
B. 因 $a=\sqrt{2} \sin 59^{\circ}, b=\sqrt{2} \sin 61^{\circ}, c=\sqrt{2} \sin 60^{\circ}$, 而 $59^{\circ}<60^{\circ}<61^{\circ}$, 所以 $a<c<b$.
%%PROBLEM_END%%



%%PROBLEM_BEGIN%%
%%<PROBLEM>%%
问题6 化简 $\tan 10^{\circ} \cdot \tan 20^{\circ}+\tan 20^{\circ} \cdot \tan 60^{\circ}+\tan 60^{\circ} \cdot \tan 10^{\circ}$ 的值 () .
(A) 1
(B) 2
(C) $\tan 10^{\circ}$
(D) $\sqrt{3} \tan 20^{\circ}$
%%<SOLUTION>%%
A. 由 $\tan 30^{\circ}=\frac{\tan 20^{\circ}+\tan 10^{\circ}}{1-\tan 20^{\circ} \cdot \tan 10^{\circ}}$, 得 $1-\tan 20^{\circ} \cdot \tan 10^{\circ}= \sqrt{3}\left(\tan 20^{\circ}+\tan 10^{\circ}\right)$, 即 $\tan 10^{\circ} \cdot \tan 20^{\circ}+\sqrt{3} \tan 20^{\circ}+\sqrt{3} \tan 10^{\circ}=1$.
%%PROBLEM_END%%



%%PROBLEM_BEGIN%%
%%<PROBLEM>%%
问题7 已知 $\sin \alpha \cos \beta=\frac{1}{2}$, 则 $\cos \alpha \sin \beta$ 的取值范围 ( ).
(A) $\left[-\frac{3}{2}, \frac{1}{2}\right]$
(B) $[-1,1]$
(C) $\left[-\frac{3}{4}, \frac{3}{4}\right]$
(D) $\left[-\frac{1}{2}, \frac{1}{2}\right]$
%%<SOLUTION>%%
D. 设 $\cos \alpha \sin \beta=t$, 则由 $\sin \alpha \cos \beta=\frac{1}{2}$, 得 $\sin (\alpha+\beta)=t+\frac{1}{2}$, $\sin (\alpha-\beta)=\frac{1}{2}-t$, 于是 $\left|t+\frac{1}{2}\right| \leqslant 1$ 且 $\left|\frac{1}{2}-t\right| \leqslant 1$, 解之得 $-\frac{1}{2} \leqslant t \leqslant \frac{1}{2}$.
%%PROBLEM_END%%



%%PROBLEM_BEGIN%%
%%<PROBLEM>%%
问题8 已知 $\alpha 、 \beta$ 是锐角, $\sin \alpha=x, \cos \beta=y, \cos (\alpha+\beta)=-\frac{3}{5}$, 则 $y$ 与 $x$ 之间的函数关系为 ( ).
(A) $y=-\frac{3}{5} \sqrt{1-x^2}+\frac{4}{5} x\left(\frac{3}{5}<x<1\right)$
(B) $y=-\frac{3}{5} \sqrt{1-x^2}+\frac{4}{5} x(0<x<1)$
(C) $y=-\frac{3}{5} \sqrt{1-x^2}-\frac{4}{5} x\left(0<x<\frac{3}{5}\right)$
(D) $y=-\frac{3}{5} \sqrt{1-x^2}-\frac{4}{5} x(0<x<1)$
%%<SOLUTION>%%
A. 因 $\alpha 、 \beta$ 为锐角, 所以 $\sin (\alpha+\beta)=\frac{4}{5}, \cos \alpha=\sqrt{1-x^2}$, 于是 $y= \cos \beta=\cos (\alpha+\beta-\alpha)=\cos (\alpha+\beta) \cos \alpha+\sin (\alpha+\beta) \sin \beta=-\frac{3}{5} \sqrt{1-x^2}+ \frac{4}{5} x$. 因 $y>0$, 所以 $-\frac{3}{5} \sqrt{1-x^2}+\frac{4}{5} x>0$. 即 $4 x>3 \sqrt{1-x^2}$, 平方得 $25 x^2>9$, 所以 $x>\frac{3}{5}$. 又 $x<1$, 故 $\frac{3}{5}<x<1$.
%%PROBLEM_END%%



%%PROBLEM_BEGIN%%
%%<PROBLEM>%%
问题9 设 $\alpha 、 \beta$ 为钝角, 且 $\sin \alpha=\frac{\sqrt{5}}{5}, \cos \beta=-\frac{3 \sqrt{10}}{10}$, 则 $\alpha+\beta$ 的值为 ( ).
(A) $\frac{3}{4} \pi$
(B) $\frac{5}{4} \pi$
(C) $\frac{7}{4} \pi$
(D) $\frac{5 \pi}{4}$ 或 $\frac{7 \pi}{4}$
%%<SOLUTION>%%
C. 参照例 10 可知 $\alpha+\beta=\frac{7 \pi}{4}$.
%%PROBLEM_END%%



%%PROBLEM_BEGIN%%
%%<PROBLEM>%%
问题10. 若 $\alpha$ 适合条件 $\sin \frac{\alpha}{2}=\frac{1}{2}(\sqrt{1+\sin \alpha}+\sqrt{1-\sin \alpha})$, 则 $\frac{\alpha}{2}$ 的取值范围是 ( ).
(A) $\left[2 k \pi, 2 k \pi+\frac{\pi}{2}\right](k \in \mathbf{Z})$
(B) $\left[2 k \pi+\frac{\pi}{2},(2 k+1) \pi\right](k \in$
(C) $\left[2 k \pi+\frac{\pi}{4}, 2 k \pi+\frac{3 \pi}{4}\right](k \in$
(D) $\left[2 k \pi+\frac{3 \pi}{4}, 2 k \pi+\frac{5 \pi}{4}\right](k \in$
%%<SOLUTION>%%
C. 因 $\frac{1}{2}(\sqrt{1+\sin \alpha}+\sqrt{1-\sin \alpha})=\frac{1}{2}\left(\left|\sin \frac{\alpha}{2}+\cos \frac{\alpha}{2}\right|+\right. \left.\left|\sin \frac{\alpha}{2}-\cos \frac{\alpha}{2}\right|\right)$, 所以要使原等式成立, 则 $\sin \frac{\alpha}{2}+\cos \frac{\alpha}{2} \geqslant 0$, 且 $\sin \frac{\alpha}{2}- \cos \frac{\alpha}{2} \geqslant 0$. 故选 C.
%%PROBLEM_END%%



%%PROBLEM_BEGIN%%
%%<PROBLEM>%%
问题11 已知 $\cos \left(\frac{\pi}{4}+x\right)=m$, 则 $\sin 2 x=$
%%<SOLUTION>%%
$1-2 m^2$. 由 $\cos \left(\frac{\pi}{4}+x\right)=m$, 得 $\cos 2\left(\frac{\pi}{4}+x\right)=2 m^2-1$, 即 $\sin 2 x= -\cos \left(\frac{\pi}{2}+2 x\right)=1-2 m^2$.
%%PROBLEM_END%%



%%PROBLEM_BEGIN%%
%%<PROBLEM>%%
问题12 若 $\frac{\pi}{4}<\alpha<\frac{3 \pi}{4}, 0<\beta<\frac{\pi}{4}, \cos \left(\frac{\pi}{4}-\alpha\right)=\frac{3}{5}, \sin \left(\frac{3 \pi}{4}+\beta\right)=\frac{5}{13}$, 则 $\sin (\alpha+\beta)=$
%%<SOLUTION>%%
$\frac{56}{65}$. 由条件得 $\sin \left(\frac{\pi}{4}-\alpha\right)=-\frac{4}{5}, \cos \left(\frac{3 \pi}{4}+\beta\right)=-\frac{12}{13}$, 所以 $\cos \left[\left(\frac{3 \pi}{4}+\beta\right)-\left(\frac{\pi}{4}-\alpha\right)\right]=\frac{3}{5} \times\left(-\frac{12}{13}\right)+\left(-\frac{4}{5}\right) \times \frac{5}{13}=-\frac{56}{65}$. 即 $\cos \left(\frac{\pi}{2}+\beta+\alpha\right)=-\frac{56}{65}$, 所以 $\sin (\alpha+\beta)=\frac{56}{65}$.
%%PROBLEM_END%%



%%PROBLEM_BEGIN%%
%%<PROBLEM>%%
问题13 若 $f(\tan x)=\sin 2 x$, 则 $f(-1)$ 的值为
%%<SOLUTION>%%
-1 . 由 $f(\tan x)=\sin 2 x=\frac{2 \tan x}{1+\tan ^2 x}$, 得 $f(x)=\frac{2 x}{1+x^2}$, 所以 $f(-1)=\frac{2 \times(-1)}{1+(-1)^2}=-1$.
%%PROBLEM_END%%



%%PROBLEM_BEGIN%%
%%<PROBLEM>%%
问题14 已知 $x+\frac{1}{x}=2 \cos \frac{\pi}{24}$, 则 $x^8+\frac{1}{x^8}$ 的值为
%%<SOLUTION>%%
1. 由 $x+\frac{1}{x}=2 \cos \frac{\pi}{24}$, 得 $x^2+\frac{1}{x^2}=4 \cos ^2 \frac{\pi}{24}-2=2 \cos \frac{\pi}{12}$, 所以 $x^4+ \frac{1}{x^4}=4 \cos ^2 \frac{\pi}{12}-2=2 \cos \frac{\pi}{6}, x^8+\frac{1}{x^8}=4 \cos ^2 \frac{\pi}{6}-2=2 \cos \frac{\pi}{3}=1$.
%%PROBLEM_END%%



%%PROBLEM_BEGIN%%
%%<PROBLEM>%%
问题15 设 $\frac{3 \pi}{2}<\alpha<\frac{7 \pi}{4}$ 且 $2 \cot ^2 \alpha+7 \cot \alpha+3=0$, 则 $\cos 2 \alpha=$
%%<SOLUTION>%%
$-\frac{3}{5}$. 由条件等式得 $(2 \cot \alpha+1)(\cot \alpha+3)=0$, 所以 $\cot \alpha=-\frac{1}{2}$ 或
$\cot \alpha=-3$. 而 $\frac{3 \pi}{2}<\alpha<\frac{7 \pi}{4}$, 故 $\cot \alpha>-1$, 从而 $\cot \alpha=-\frac{1}{2}$, 所以 $\tan \alpha=-2$, $\cos 2 \alpha=\frac{1-\tan ^2 \alpha}{1+\tan ^2 \alpha}=-\frac{3}{5}$.
%%PROBLEM_END%%



%%PROBLEM_BEGIN%%
%%<PROBLEM>%%
问题16 已知 $\sin \frac{\theta}{2}=\sqrt{\frac{x-1}{2 x}}$, 并且 $0<\theta<\frac{\pi}{2}$, 则 $\tan \theta=?$, $ \sin 2 \theta=?$ , $\cos 2 \theta=?$
%%<SOLUTION>%%
$\sqrt{x^2-1}, \frac{2 \sqrt{x^2-1}}{x^2}, \frac{2-x^2}{x^2}$.
%%PROBLEM_END%%



%%PROBLEM_BEGIN%%
%%<PROBLEM>%%
问题17. 计算 $\sin 6^{\circ} \sin 42^{\circ} \sin 66^{\circ} \sin 78^{\circ}=$
%%<SOLUTION>%%
$\frac{1}{16}$. 参照例 4 可得原式 $=\frac{1}{16}$.
%%PROBLEM_END%%



%%PROBLEM_BEGIN%%
%%<PROBLEM>%%
问题18 计算 $\tan 5^{\circ}+\cot 5^{\circ}-\frac{2}{\cos 80^{\circ}}=$
%%<SOLUTION>%%
0 . 原式 $=\frac{\sin 5^{\circ}}{\cos 5^{\circ}}+\frac{\cos 5^{\circ}}{\sin 5^{\circ}}-\frac{2}{\cos 80^{\circ}}=\frac{\sin ^2 5^{\circ}+\cos ^2 5^{\circ}}{\sin 5^{\circ} \cos 5^{\circ}}-\frac{2}{\cos 80^{\circ}}= \frac{2}{\sin 10^{\circ}}-\frac{2}{\sin 10^{\circ}}=0$.
%%PROBLEM_END%%



%%PROBLEM_BEGIN%%
%%<PROBLEM>%%
问题19 已知 $\theta \in\left(\frac{3 \pi}{2}, 2 \pi\right)$, 则 $\sqrt{1+\sin \theta}-\sqrt{1-\sin \theta}=$
%%<SOLUTION>%%
$-2 \sin \frac{\theta}{2}$. 原式 $=\left|\sin \frac{\theta}{2}+\cos \frac{\theta}{2}\right|-\left|\sin \frac{\theta}{2}-\cos \frac{\theta}{2}\right|=-\cos \frac{\theta}{2}- \sin \frac{\theta}{2}-\sin \frac{\theta}{2}+\cos \frac{\theta}{2}=-2 \sin \frac{\theta}{2}$.
%%PROBLEM_END%%



%%PROBLEM_BEGIN%%
%%<PROBLEM>%%
问题20 已知 $\sin \alpha+\sin \beta=\frac{1}{2}, \cos \alpha+\cos \beta=\frac{2}{3}$, 则 $\cos (\alpha-\beta)=$
%%<SOLUTION>%%
$-\frac{47}{72}$. 两式平方和得 $2+2 \cos (\alpha-\beta)=\frac{25}{36}$, 所以原式 $=-\frac{47}{72}$.
%%PROBLEM_END%%



%%PROBLEM_BEGIN%%
%%<PROBLEM>%%
问题21. 求下列各式的值:
(1) $\cos ^2 24^{\circ}+\sin ^2 6^{\circ}+\cos ^2 18^{\circ}$;
(2) $4 \cos ^2 36^{\circ}-\sin 84^{\circ}\left(\sqrt{3}-\tan 6^{\circ}\right)$.
%%<SOLUTION>%%
(1) $\cos ^2 24^{\circ}+\sin ^2 6^{\circ}+\cos ^2 18^{\circ}=\frac{3}{2}+\frac{1}{2}\left(\cos 48^{\circ}-\cos 12^{\circ}+\right. \left.\cos 36^{\circ}\right)=\frac{3}{2}+\frac{1}{2}\left(-2 \sin 30^{\circ} \sin 18^{\circ}+\cos 36^{\circ}\right)=\frac{3}{2}+\frac{1}{2}\left(\sin 54^{\circ}-\sin 18^{\circ}\right)= \frac{3}{2}+\cos 36^{\circ} \sin 18^{\circ}=\frac{3}{2}+\frac{\cos 36^{\circ} \sin 18^{\circ} \cos 18^{\circ}}{\cos 18^{\circ}}=\frac{3}{2}+\frac{\frac{1}{4} \sin 72^{\circ}}{\cos 18^{\circ}}=\frac{7}{4}$.
 (2)参照例 3 的解法, 可得 $4 \cos ^2 36^{\circ}-\sin 84^{\circ}\left(\sqrt{3}-\tan 6^{\circ}\right)=1$.
%%PROBLEM_END%%



%%PROBLEM_BEGIN%%
%%<PROBLEM>%%
问题22. 设 $\cos \left(\alpha-\frac{\beta}{2}\right)=-\frac{1}{9}, \sin \left(\frac{\alpha}{2}-\beta\right)=\frac{2}{3}$, 且 $\frac{\pi}{2}<\alpha<\pi, 0<\beta<\frac{\pi}{2}$, 求 $\sin \frac{\alpha+\beta}{2}$ 和 $\cos (\alpha+\beta)$ 的值.
%%<SOLUTION>%%
由 $\frac{\pi}{2}<\alpha<\pi, 0<\beta<\frac{\pi}{2}$, 得 $\frac{\pi}{4}<\alpha-\frac{\beta}{2}<\pi,-\frac{\pi}{4}<\frac{\alpha}{2}-\beta<\frac{\pi}{2}$,
故由 $\cos \left(\alpha-\frac{\beta}{2}\right)=-\frac{1}{9}$, 得 $\sin \left(\alpha-\frac{\beta}{2}\right)=\frac{4 \sqrt{5}}{9}$, 由 $\sin \left(\frac{\alpha}{2}-\beta\right)=\frac{2}{3}$ 得 $\cos \left(\frac{\alpha}{2}-\beta\right)=\frac{\sqrt{5}}{3}$, 所以 $\sin \frac{\alpha+\beta}{2}=\sin \left[\left(\alpha-\frac{\beta}{2}\right)-\left(\frac{\alpha}{2}-\beta\right)\right]= \sin \left(\alpha-\frac{\beta}{2}\right) \cos \left(\frac{\alpha}{2}-\beta\right)-\cos \left(\alpha-\frac{\beta}{2}\right) \sin \left(\frac{\alpha}{2}-\beta\right)=\frac{4 \sqrt{5}}{9} \times \frac{\sqrt{5}}{3}-\left(-\frac{1}{9}\right) \times \frac{2}{3}=\frac{22}{27}$. 从而 $\cos (\alpha+\beta)=1-2 \sin ^2 \frac{\alpha+\beta}{2}=1-2 \times\left(\frac{22}{27}\right)^2=-\frac{239}{729}$.
%%PROBLEM_END%%



%%PROBLEM_BEGIN%%
%%<PROBLEM>%%
问题23 证明下列恒等式:
(1) $\tan \frac{3 x}{2}-\tan \frac{x}{2}=\frac{2 \sin x}{\cos x+\cos 2 x}$;
(2) $\frac{\sin ^2 A-\sin ^2 B}{\sin A \cos A-\sin B \cos B}=\tan (A+B)$.
%%<SOLUTION>%%
(1) 左边 $=\frac{\sin \frac{3 x}{2}}{\cos \frac{3 x}{2}}-\frac{\sin \frac{x}{2}}{\cos \frac{x}{2}}=\frac{\sin \frac{3 x}{2} \cos \frac{x}{2}-\sin \frac{x}{2} \cos \frac{3 x}{2}}{\cos \frac{3}{2} x \cos \frac{x}{2}}=\frac{\sin \left(\frac{3}{2} x-\frac{x}{2}\right)}{\frac{1}{2}(\cos x+\cos 2 x)}=\frac{2 \sin x}{\cos x+\cos 2 x}=$ 右边.
(2) 左边 $=\frac{\frac{1}{2}(1-\cos 2 A)-\frac{1}{2}(1-\cos 2 B)}{\frac{1}{2} \sin 2 A-\frac{1}{2} \sin 2 B}=\frac{\cos 2 B-\cos 2 A}{\sin 2 A-\sin 2 B} =\frac{-2 \sin (A+B) \sin (B-A)}{2 \cos (A+B) \sin (A-B)}=\tan (A+B)=$ 右边.
%%PROBLEM_END%%



%%PROBLEM_BEGIN%%
%%<PROBLEM>%%
问题24 已知 $a 、 b$ 均为正整数, 且 $a>b, \sin \theta=\frac{2 a b}{a^2+b^2}$, 其中 $\theta \in\left(0, \frac{\pi}{2}\right), A_n= \left(a^2+b^2\right)^n \sin n \theta$. 求证: 对于一切正整数 $n, A_n$ 均为整数.
%%<SOLUTION>%%
设 $B_n=\left(a^2+b^2\right)^n \cos n \theta$. 由 $\sin \theta=\frac{2 a b}{a^2+b^2}$, 得 $\cos \theta=\frac{a^2-b^2}{a^2+b^2}(a> b>0)$, 所以 $\left(a^2+b^2\right) \sin \theta=2 a b,\left(a^2+b^2\right) \cos \theta=a^2-b^2$ 均为整数, 即 $A_1$ 、 $B_1$ 均为整数.
设 $A_k 、 B_k$ 均为整数, 则 $A_{k+1}=\left(a^2+b^2\right)^{k+1} \sin (k+1) \theta=\left(a^2+\right. \left.b^2\right)^{k+1} \sin k \theta \cos \theta+\left(a^2+b^2\right)^{k+1} \cos k \theta \sin \theta=A_k \cdot\left(a^2-b^2\right)+B_k \cdot 2 a b, B_{k+1}= \left(a^2+b^2\right)^{k+1} \cos (k+1) \theta=\left(a^2+b^2\right)^{k+1} \cos k \theta \cos \theta-\left(a^2+b^2\right)^{k+1} \sin k \theta \cos \theta= B_k \cdot\left(a^2-b^2\right)-A_k \cdot 2 a b$. 由 $A_k 、 B_k 、 a 、 b$ 均为整数, 得 $A_{k+1} 、 B_{k+1}$ 均为整数.
由归纳原理可知, 对于一切正整数 $n, A_n$ 为整数.
%%PROBLEM_END%%



%%PROBLEM_BEGIN%%
%%<PROBLEM>%%
问题25 求证: $\sum_{k=0}^n\left(\frac{1}{3}\right)^k \sin ^3\left(3^k \alpha\right)=\frac{3}{4} \sin \alpha-\frac{1}{4 \cdot 3^n} \sin 3^{n+1} \alpha$.
%%<SOLUTION>%%
令 $a_n=\sum_{k=0}^n\left(\frac{1}{3}\right)^k \sin ^3\left(3^k \alpha\right)+\frac{1}{4 \cdot 3^n} \sin 3^{n+1} \alpha-\frac{3}{4} \sin \alpha$, 容易验证 $a_{n+1}-a_n=0$, 所以成立.
%%PROBLEM_END%%



%%PROBLEM_BEGIN%%
%%<PROBLEM>%%
问题26 (1) 求值: $\cos ^4 \frac{\pi}{16}+\cos ^4 \frac{3 \pi}{16}+\cos ^4 \frac{5 \pi}{16}+\cdots+\cos ^4 \frac{15 \pi}{16}$;
(2) 求 $\cos ^5 \frac{\pi}{9}+\cos ^5 \frac{5 \pi}{9}+\cos ^5 \frac{7 \pi}{9}$ 的值.
%%<SOLUTION>%%
(1) 设 $x=\cos ^4 \frac{\pi}{16}+\cos ^4 \frac{3 \pi}{16}+\cos ^4 \frac{5 \pi}{16}+\cos ^4 \frac{7 \pi}{16}, y=\sin ^4 \frac{\pi}{16}+\sin ^4 \frac{3 \pi}{16}+ \sin ^4 \frac{5 \pi}{16}+\sin ^4 \frac{7 \pi}{16}$, 则 $x-y=\left(\cos ^2 \frac{\pi}{16}-\sin ^2 \frac{\pi}{16}\right)+\left(\cos ^2 \frac{3 \pi}{16}-\sin ^2 \frac{3 \pi}{16}\right)+\left(\cos ^2 \frac{5 \pi}{16}-\right. \left.\sin ^2 \frac{5 \pi}{16}\right)+\left(\cos ^2 \frac{7 \pi}{16}-\sin ^2 \frac{7 \pi}{16}\right)=\cos \frac{\pi}{8}+\cos \frac{3 \pi}{8}+\cos \frac{5 \pi}{8}+\cos \frac{7 \pi}{8}=0 ; x+ y=4-2\left(\cos ^2 \frac{\pi}{16} \sin ^2 \frac{\pi}{16}+\cos ^2 \frac{3 \pi}{16} \sin ^2 \frac{3 \pi}{16}+\cos ^2 \frac{5 \pi}{16} \sin ^2 \frac{5 \pi}{16}+\cos ^2 \frac{7 \pi}{16} \sin ^2 \frac{7 \pi}{16}\right)=4- \frac{1}{2}\left(\sin ^2 \frac{\pi}{8}+\sin ^2 \frac{3 \pi}{8}+\sin ^2 \frac{5 \pi}{8}+\sin ^2 \frac{7 \pi}{8}\right)=4-\left(\sin ^2 \frac{\pi}{8}+\sin ^2 \frac{3 \pi}{8}\right)=4- \left(\frac{1-\cos \frac{\pi}{4}}{2}+\frac{1-\cos \frac{3 \pi}{4}}{2}\right)=3$. 故 $x=y=\frac{3}{2}$. (2) 设 $x_1=\cos \frac{\pi}{9}, x_2= \cos \frac{5 \pi}{9}, x_3=\cos \frac{7 \pi}{9}$, 则 $x_1+x_2+x_3=\cos \frac{\pi}{9}+\cos \frac{5 \pi}{9}+\cos \frac{7 \pi}{9}=2 \cos \frac{4 \pi}{9} \cos \frac{\pi}{3}+ \cos \frac{5 \pi}{9}=\cos \frac{4 \pi}{9}+\cos \frac{5 \pi}{9}=0 ; x_1 x_2+x_2 x_3+x_3 x_1=\cos \frac{\pi}{9} \cos \frac{5 \pi}{9}+\cos \frac{5 \pi}{9}$
$$
\begin{aligned}
& \cos \frac{7 \pi}{9}+\cos \frac{7 \pi}{9} \cos \frac{\pi}{9}=\frac{1}{2}\left(\cos \frac{2 \pi}{3}+\cos \frac{4 \pi}{9}+\cos \frac{4 \pi}{3}+\cos \frac{2 \pi}{9}+\cos \frac{8 \pi}{9}+\right. \\
& \left.\cos \frac{2 \pi}{3}\right)=\frac{1}{2}\left[-\frac{3}{2}+\cos \frac{4 \pi}{9}+\left(\cos \frac{2 \pi}{9}+\cos \frac{8 \pi}{9}\right)\right]=-\frac{3}{4}+\frac{1}{2}\left(\cos \frac{4 \pi}{9}+\right. \\
& \left.2 \cos \frac{5 \pi}{9} \cos \frac{\pi}{3}\right)=-\frac{3}{4}+\frac{1}{2}\left(\cos \frac{4 \pi}{9}+\cos \frac{5 \pi}{9}\right)=-\frac{3}{4} ; x_1 \cdot x_2 \cdot x_3=\cos \frac{\pi}{9} \cos \frac{5 \pi}{9} \\
& \cos \frac{7 \pi}{9}=\cos \frac{\pi}{9} \cos \frac{4 \pi}{9} \cos \frac{2 \pi}{9}=\frac{8 \sin \frac{\pi}{9} \cos \frac{\pi}{9} \cos \frac{2 \pi}{9} \cos \frac{4 \pi}{9}}{8 \sin \frac{\pi}{9}}=\frac{\sin \frac{8 \pi}{9}}{8 \sin \frac{\pi}{9}}=\frac{1}{8}, \text { 故 } \\
& x_1 、 x_2 、 x_3 \text { 是方程 } x^3-\frac{3}{4} x-\frac{1}{8}=0 \text { 的根.
所以 } x^5=\frac{3}{4} x^3+\frac{1}{8} x^2= \\
& \frac{3}{4}\left(\frac{3}{4} x+\frac{1}{8}\right)+\frac{1}{8} x^2=\frac{1}{8} x^2+\frac{9}{16} x+\frac{3}{32}, \text { 从而 } \cos ^5 \frac{\pi}{9}+\cos ^5 \frac{5 \pi}{9}+\cos ^5 \frac{7 \pi}{9}= \\
& \frac{1}{8}\left(x_1^2+x_2^2+x_3^2\right)+\frac{9}{16}\left(x_1+x_2+x_3\right)+\frac{3}{32} \times 3=\frac{1}{8}\left[\left(x_1+x_2+x_3\right)^2-\right. \\
& \left.2\left(x_1 x_2+x_2 x_3+x_3 x_1\right)\right]+\frac{9}{16}\left(x_1+x_2+x_3\right)+\frac{9}{32}=\frac{1}{8} \times(-2) \times\left(-\frac{3}{4}\right)+ \\
& \frac{9}{32}=\frac{15}{32} .
\end{aligned}
$$
$x_1 、 x_2 、 x_3$ 是方程 $x^3-\frac{3}{4} x-\frac{1}{8}=0$ 的根.
所以 $x^5=\frac{3}{4} x^3+\frac{1}{8} x^2= \frac{3}{4}\left(\frac{3}{4} x+\frac{1}{8}\right)+\frac{1}{8} x^2=\frac{1}{8} x^2+\frac{9}{16} x+\frac{3}{32}$, 从而 $\cos ^5 \frac{\pi}{9}+\cos ^5 \frac{5 \pi}{9}+\cos ^5 \frac{7 \pi}{9}= \frac{1}{8}\left(x_1^2+x_2^2+x_3^2\right)+\frac{9}{16}\left(x_1+x_2+x_3\right)+\frac{3}{32} \times 3=\frac{1}{8}\left[\left(x_1+x_2+x_3\right)^2-\right. \left.2\left(x_1 x_2+x_2 x_3+x_3 x_1\right)\right]+\frac{9}{16}\left(x_1+x_2+x_3\right)+\frac{9}{32}=\frac{1}{8} \times(-2) \times\left(-\frac{3}{4}\right)+ \frac{9}{32}=\frac{15}{32}$.
%%PROBLEM_END%%



%%PROBLEM_BEGIN%%
%%<PROBLEM>%%
问题27 已知 $\frac{\sin ^2 \gamma}{\sin ^2 \alpha}=1-\frac{\tan (\alpha-\beta)}{\tan \alpha}$, 求证: $\tan ^2 \gamma=\tan \alpha \cdot \tan \beta$.
%%<SOLUTION>%%
$\sin ^2 \gamma=\sin ^2 \alpha \cdot\left[1-\frac{\tan (\alpha-\beta)}{\tan \alpha}\right]=\sin ^2 \alpha \frac{\sin \alpha \cos (\alpha-\beta)-\cos \alpha \sin (\alpha-\beta)}{\sin \alpha \cos (\alpha-\beta)}= \frac{\sin \alpha \cdot \sin \beta}{\cos (\alpha-\beta)}$, 所以 $\tan ^2 \gamma=\frac{\sin ^2 \gamma}{1-\sin ^2 \gamma}=\frac{\frac{\sin \alpha \sin \beta}{\cos (\alpha-\beta)}}{1-\frac{\sin \alpha \sin \beta}{\cos (\alpha-\beta)}}=\frac{\sin \alpha \sin \beta}{\cos (\alpha-\beta)-\sin \alpha \sin \beta}= \frac{\sin \alpha \sin \beta}{\cos \alpha \cos \beta}=\tan \alpha \cdot \tan \beta$.
%%PROBLEM_END%%



%%PROBLEM_BEGIN%%
%%<PROBLEM>%%
问题28. 已知 $\cos \alpha=\tan \beta, \cos \beta=\tan \gamma, \cos \gamma=\tan \alpha$, 则
$$
\sin ^2 \alpha=\sin ^2 \beta=\sin ^2 \gamma=\cos ^4 \alpha=\cos ^4 \beta=\cos ^4 \gamma=4 \sin ^2 18^{\circ} .
$$
%%<SOLUTION>%%
令 $x=\cos \alpha, y=\cos \beta, z=\cos \gamma$, 则 $x^2 y^2=\tan ^2 \beta \cdot \cos ^2 \beta=\sin ^2 \beta= 1-y^2 \cdots$ (1). $y^2 z^2=\tan ^2 \gamma \cdot \cos ^2 \gamma=\sin ^2 \gamma=1-z^2 \cdots$ (2). $z^2 x^2=\tan ^2 \alpha$. $\cos ^2 \alpha=\sin ^2 \alpha=1-x^2 \cdots$ (3). 解之得, $x^2=y^2=z^2=\frac{\sqrt{5}-1}{2}=2 \cdot \sin 18^{\circ}$. 所以 $\cos ^4 \alpha=\cos ^4 \beta=\cos ^4 \gamma=4 \sin ^2 18^{\circ}$, 又 $\tan \alpha=\cos \gamma$, 所以 $\tan ^2 \alpha=\cos ^2 \gamma= \cos ^2 \alpha$, 则 $\sin ^2 \alpha=\cos ^4 \alpha$, 同理, 可得 $\sin ^2 \alpha=\sin ^2 \beta=\sin ^2 \gamma=\cos ^4 \alpha=\cos ^4 \beta= \cos ^4 \gamma=4 \sin ^2 18^{\circ}$.
%%PROBLEM_END%%


