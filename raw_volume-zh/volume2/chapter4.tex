
%%TEXT_BEGIN%%
函数的最大值与最小值我们经常会遇到各种各样的求最大值和最小值问题,这类问题在许多情况下可以归结为求函数的最大值和最小值.
定义设函数 $f(x)$ 的定义域为 $D$. 若存在 $x_0 \in D$, 使得对任意 $x \in D$, 都有
$$
f(x) \leqslant f\left(x_0\right),
$$
则称 $f\left(x_0\right)$ 为函数 $f(x)$ 在 $D$ 上的最大值, 简记为 $f_{\text {max }}$. 若存在 $y_0 \in D$, 使得对任意 $x \in D$, 都有
$$
f(x) \geqslant f\left(y_0\right),
$$
则称 $f\left(y_0\right)$ 为函数 $f(x)$ 在 $D$ 上的最小值, 简记为 $f_{\min }$.
求函数的最大值和最小值问题涉及的知识面较广, 解法也灵活多变, 需要我们有很好的综合能力.
常用的方法有:
(1)配方法;(2)判别式法;(3)不等式法;(4)换元法;(5)构造法;(6)利用函数性质.
下面我们分别来介绍这些方法的应用.
4. 1 配方法利用平方数恒大于或等于 0 , 将所给的函数配成若干个平方及一些常数的代数和的形式, 然后再求其最值就容易了.
%%TEXT_END%%



%%TEXT_BEGIN%%
4.2 判别式法利用实系数一元二次方程有实根, 则它的判别式 $\Delta \geqslant 0$, 从而可以确定系数中参数的范围, 进而求得最值.
特别地, 对于分式函数 $y=\frac{a_1 x^2+b_1 x+c_1}{a_2 x^2+b_2 x+c_2}$ 的最大值与最小值问题, 常用的方法是去分母后, 化为关于 $x$ 的二次方程, 然后用判别式 $\Delta \geqslant 0$, 得出 $y$ 的取值范围, 进而确定出 $y$ 的最大值和最小值.
%%TEXT_END%%



%%TEXT_BEGIN%%
4.3 不等式法不等式与函数的最值问题有着密切的联系, 利用不等式取等号, 就可得到一个最值问题的解, 而许多不等式又可解释为最值问题的解.
%%TEXT_END%%



%%TEXT_BEGIN%%
4.4 换元法通过换元, 把复杂的目标函数变形为较简单的函数形式, 或将不易求得最值的函数形式化为易求得最值的形式, 从而使问题得到解决.
%%TEXT_END%%



%%TEXT_BEGIN%%
4.5 构造法根据欲求最值的函数的特征, 构造反映函数关系的几何图形, 然后借助于图形可较容易地求得最大值和最小值.
%%TEXT_END%%



%%TEXT_BEGIN%%
4.6 利用函数性质若函数 $f(x)$ 在 $[a, b]$ 上是增函数, 则 $f(x)$ 在 $[a, b]$ 上的最大值为 $f(b)$, 最小值为 $f(a)$; 若函数 $f(x)$ 在 $[a, b]$ 上是减函数, 则 $f(x)$ 在 $[a, b]$ 上的最大值为 $f(a)$, 最小值为 $f(b)$.
若函数 $f(x)$ 满足: 当 $x \leqslant x_0$ 时, $f(x)$ 是增函数, 当 $x \geqslant x_0$ 时, $f(x)$ 是减函数, 则 $f\left(x_0\right)$ 是 $f(x)$ 的最大值;
若函数 $f(x)$ 满足: 当 $x \leqslant x_0$ 时, $f(x)$ 是减函数, 当 $x \geqslant x_0$ 时, $f(x)$ 是增函数, 则 $f\left(x_0\right)$ 是 $f(x)$ 的最小值.
%%TEXT_END%%



%%PROBLEM_BEGIN%%
%%<PROBLEM>%%
例1 设 $x, y \in \mathbf{R}$, 求 $u=x^2+x y+y^2-x-2 y+3$ 的最小值.
%%<SOLUTION>%%
分析:把 $u$ 写成若干个非负数的和再加上一个常数, 从而 $u$ 就大于等于这个常数,再说明当 $x, y$ 取某值时, $u$ 可以取到这个常数即可.
解 $u=x^2+(y-1) x+y^2-2 y+3$
$$
=\left[x^2+(y-1) x+\frac{(y-1)^2}{4}\right]+y^2-2 y+3-\frac{(y-1)^2}{4}
$$
$$
\begin{aligned}
& =\left(x+\frac{y-1}{2}\right)^2+\frac{3}{4}\left(y^2-2 y+1\right)+2 \\
& =\left(x+\frac{y-1}{2}\right)^2+\frac{3}{4}(y-1)^2+2 \geqslant 2,
\end{aligned}
$$
又当 $x=0, y=1$ 时等号成立, 所以 $u$ 的最小值为 2 .
说明在估计 $u$ 的下界时, 进行了配方, 需要注意的是这些非负项是可以同时达到 0 的.
%%PROBLEM_END%%



%%PROBLEM_BEGIN%%
%%<PROBLEM>%%
例2 设 $x \in \mathbf{R}^{+}$, 求函数 $f(x)=x^2-x+\frac{1}{x}$ 的最小值.
%%<SOLUTION>%%
分析:先估计 $f(x)$ 的下界, 再说明这个下界是可以达到的.
解
$$
\begin{aligned}
f(x) & =\left(x^2-2 x+1\right)+\left(x+\frac{1}{x}-2\right)+1 \\
& =(x-1)^2+\left(\sqrt{x}-\frac{1}{\sqrt{x}}\right)^2+1 \geqslant 1,
\end{aligned}
$$
又当 $x=1$ 时, $f(x)=1$, 所以 $f(x)$ 的最小值为 1 .
说明在求最大(小)值时, 估计了上 (下) 界后, 一定要举例说明这个界是可以取到的, 才能说这就是最大 (小) 值, 否则就不一定对了.
例如, 本题我们也可以这样估计:
$$
\begin{aligned}
f(x) & =\left(x^2-2 x+1\right)+\left(x+\frac{1}{x}+2\right)-3 \\
& =(x-1)^2+\left(\sqrt{x}+\frac{1}{\sqrt{x}}\right)^2-3 \geqslant-3,
\end{aligned}
$$
但无论 $x$ 取何值时, $f(x) \neq-3$, 即一 3 不能作为 $f(x)$ 的最小值.
%%PROBLEM_END%%



%%PROBLEM_BEGIN%%
%%<PROBLEM>%%
例3 试求函数 $f(x)=(x+1)(x+2)(x+3)(x+4)+5$ 在闭区间 $[-3,3]$ 上的最大值与最小值.
%%<SOLUTION>%%
解:令 $t=x^2+5 x$, 则
$$
\begin{aligned}
f(x) & =\left(x^2+5 x+4\right)\left(x^2+5 x+6\right)+5 \\
& =(t+4)(t+6)+5 \\
& =t^2+10 t+29 .
\end{aligned}
$$
当 $x \in[-3,3]$ 时, $t$ 的取值范围是 $\left[-\frac{25}{4}, 24\right]$, 如图(<FilePath:./figures/fig-c4e3-1.png>)所示.
于是原题转化为在 $\left[-\frac{25}{4}, 24\right]$ 内求二次函数 $f(t)=t^2+10 t+29$ 的最大值和最小值.
因 $f(t)=(t+5)^2+4$, 又 $-5 \in\left[-\frac{25}{4}, 24\right]$, 故当 $t=-5$ 时, $f_{\text {min }}=4$. 而由 $-5=x^2+5 x$ 解得 $x_{1,2}=\frac{-5 \pm \sqrt{5}}{2}$, 但 $\frac{-5-\sqrt{5}}{2} \notin[-3,3]$, 故当 $x= \frac{-5+\sqrt{5}}{2}$ 时, $f_{\text {min }}=4$. 如图(<FilePath:./figures/fig-c4e3-2.png>)所示.
当 $t=-\frac{25}{4}$ 时, $f(t)=5 \frac{9}{16}$; 当 $t=24$ 时, $f(t)=845$. 而 $845>5 \frac{9}{16}$, 所以当 $t=24$, 即 $x=3$ 时, $f_{\text {max }}=845$.
说明一个复杂的函数式, 如能写成二次函数型的复合函数, 即 $f(x)= a g^2(x)+b g(x)+c$ ( $a 、 b 、 c$ 为常数), 此时用配方法求函数的最值问题往往是行之有效的.
%%PROBLEM_END%%



%%PROBLEM_BEGIN%%
%%<PROBLEM>%%
例4 求函数 $y=\frac{x^2-2 x-3}{2 x^2+2 x+1}$ 的最大值和最小值.
%%<SOLUTION>%%
分析:把它写成关于 $x$ 的二次方程形式,利用 $\Delta \geqslant 0$ 来得到 $y$ 的范围.
解去分母并整理, 得
$$
(2 y-1) x^2+2(y+1) x+(y+3)=0 .
$$
当 $y=\frac{1}{2}$ 时, $x=-\frac{7}{6}$.
当 $y \neq \frac{1}{2}$ 时, 这是一个关于 $x$ 的二次方程, 由 $x \in \mathbf{R}$, 所以
$$
\Delta=[2(y+1)]^2-4(2 y-1)(y+3) \geqslant 0 .
$$
解方程, 得 $-4 \leqslant y \leqslant 1$.
当 $y=-4$ 时, $x=-\frac{1}{3}$; 当 $y=1$ 时, $x=-2$.
由此即知, 当 $x=-\frac{1}{3}$ 时, $y$ 取最小值 -4 ; 当 $x=-2$ 时, $y$ 取最大值 1 .
说明在用判别式法求最值时, 应特别注意这个最值能否取到, 即是否有与最值相应的 $x$ 值.
%%PROBLEM_END%%



%%PROBLEM_BEGIN%%
%%<PROBLEM>%%
例5 函数 $y=\frac{a x^2+3 x+b}{x^2+1}$ 的最大值为 $5 \frac{1}{2}$, 最小值为 $\frac{1}{2}$. 求 $a, b$ 的值.
%%<SOLUTION>%%
解:将原式化为
$$
\begin{aligned}
& (a-y) x^2+3 x+(b-y)=0 . \\
& \Delta=9-4(a-y)(b-y) \geqslant 0, \\
& 4 y^2-4(a+b) y+4 a b-9 \leqslant 0 .
\end{aligned}
$$
即显然 $y$ 的值在此不等式所对应的二次方程的两根之间, 根据求根公式,有
$$
\begin{gathered}
\left\{\begin{array}{l}
a+b+\sqrt{(a-b)^2+9}=11, \\
a+b-\sqrt{(a-b)^2+9}=1
\end{array}\right. \\
\Rightarrow\left\{\begin{array} { l } 
{ a + b = 6 , } \\
{ a - b = \pm 4 }
\end{array} \Rightarrow \left\{\begin{array} { l } 
{ a = 5 , } \\
{ b = 1 }
\end{array} \text { 或 } \left\{\begin{array}{l}
a=1, \\
b=5 .
\end{array}\right.\right.\right.
\end{gathered}
$$
以上两组都是满足题设的解.
说明本题为判别式法的一种经典应用, 希望读者能熟练掌握.
从严格意义上来说,此处的判别式法在应用时还必须考虑当 $a-y=0$ 的特殊情形, 可以推出, 此时 $b=a$, 进一步推下去 (用函数 $f(x)=x+\frac{1}{x}$ 的值域的结论) 可以发现, 这种情形无法满足题设.
虽然如此, 但这一点也必须考虑到.
%%PROBLEM_END%%



%%PROBLEM_BEGIN%%
%%<PROBLEM>%%
例6 已知函数 $f(x)=\log _2(x+1)$, 并且当点 $(x, y)$ 在 $y=f(x)$ 的图象上运动时, 点 $\left(\frac{x}{3}, \frac{y}{2}\right)$ 在 $y=g(x)$ 的图象上运动.
求函数 $p(x)=g(x)-f(x)$ 的最大值.
%%<SOLUTION>%%
解:因点 $(x, y)$ 在 $y=f(x)$ 的图象上, 故 $y=\log _2(x+1)$. 又点 $\left(\frac{x}{3}, \frac{y}{2}\right)$ 在 $y=g(x)$ 的图象上,故 $\frac{y}{2}=g\left(\frac{x}{3}\right)$. 从而
$$
g\left(\frac{x}{3}\right)=\frac{1}{2} \log _2(x+1)
$$
即 $g(x)=\frac{1}{2} \log _2(3 x+1)$, 那么
$$
\begin{aligned}
p(x) & =g(x)-f(x)=\frac{1}{2} \log _2(3 x+1)-\log _2(x+1) \\
& =\frac{1}{2} \log _2 u,
\end{aligned}
$$
其中,
$$
u=\frac{3 x+1}{(x+1)^2}=-\frac{2}{(x+1)^2}+\frac{3}{x+1}=-2\left(\frac{1}{x+1}-\frac{3}{4}\right)^2+\frac{9}{8} .
$$
因此当 $\frac{1}{x+1}=\frac{3}{4}$, 即 $x=\frac{1}{3}$ 时, $u_{\text {max }}=\frac{9}{8}$. 从而 $p(x)$ 的最大值为 $\frac{1}{2} \log _2 \frac{9}{8}$.
%%PROBLEM_END%%



%%PROBLEM_BEGIN%%
%%<PROBLEM>%%
例7 已知 $x, y$ 是实数, 且满足 $x^2+x y+y^2=3$, 求 $u=x^2-x y+y^2$ 的最大值与最小值.
%%<SOLUTION>%%
分析:这是一个有关最值的经典问题, 有多种解法, 这里介绍两种最常见的解法.
由于已知条件都是二次的代数式, 故我们试图将其与判别式或基本不等式建立联系.
解法一将题设中的两式相减再除以 2 , 得 $x y=\frac{3-u}{2}$, 再用题设的第一式相加此式及用题设的第二式相减此式分别得
$$
(x+y)^2=\frac{9-u}{2}
$$
和
$$
(x-y)^2=\frac{3 u-3}{2},
$$
可得
$$
1 \leqslant u \leqslant 9
$$
而容易验证当 $x=\sqrt{3}, y=-\sqrt{3}$ 时, $u=9$;
当 $x=y=1$ 时, $u=1$.
所以 $u_{\max }=9, u_{\min }=1$.
解法二因为
$$
\begin{aligned}
u & =x^2-x y+y^2 \\
& =x^2+x y+y^2-2 x y \\
& =3-2 x y,
\end{aligned}
$$
由于
$$
x^2+y^2 \geqslant 2 x y \text { 及 } x^2+y^2 \geqslant-2 x y,
$$
将此两式分别代入 $x^2+x y+y^2=3$, 得 $x y \leqslant 1$ 和 $x y \geqslant-3$.
所以
$$
1 \leqslant u=3-2 x y \leqslant 9 .
$$
再从推导的过程中可以看到:
当 $x=\sqrt{3}, y=-\sqrt{3}$ 时, $u=9$ ;
当 $x=y=1$ 时, $u=1$.
所以
$$
u_{\max }=9, u_{\min }=1 \text {. }
$$
%%PROBLEM_END%%



%%PROBLEM_BEGIN%%
%%<PROBLEM>%%
例8 求函数 $y=\frac{2-\cos x}{4+3 \cos x}$ 的最大值和最小值.
%%<SOLUTION>%%
分析:这类问题我们常可以利用 $|\cos x| \leqslant 1,|\sin x| \leqslant 1, \mid a \sin x+ b \cos x\left|=\sqrt{a^2+b^2}\right| \sin (x+\varphi) \mid \leqslant \sqrt{a^2+b^2}$ 来处理.
解由 $y=\frac{2-\cos x}{4+3 \cos x}$, 得
$$
\begin{gathered}
(3 y+1) \cos x=2-4 y, \\
|3 y+1| \cdot|\cos x|=|2-4 y| .
\end{gathered}
$$
所以
$$
|3 y+1| \geqslant|2-4 y| \text {, }
$$
即
$$
9 y^2+6 y+1 \geqslant 4-16 y+16 y^2 \text {. }
$$
解不等式, 得 $\frac{1}{7} \leqslant y \leqslant 3$.
当 $\cos x=1$. 即 $x=2 k \pi(k \in \mathbf{Z})$ 时, $y=\frac{1}{7}$;
当 $\cos x=-1$, 即 $x=(2 k+1) \pi(k \in \mathbf{Z})$ 时, $y=3$.
所以
$$
y_{\min }=\frac{1}{7}, y_{\max }=3 \text {. }
$$
%%PROBLEM_END%%



%%PROBLEM_BEGIN%%
%%<PROBLEM>%%
例9 设函数 $f:(0,1) \rightarrow \mathbf{R}$ 定义为
$$
f(x)= \begin{cases}x, & \text { 当 } x \text { 是无理数时; } \\ \frac{p+1}{q}, & \text { 当 } x=\frac{p}{q},(p, q)=1,0<p<q \text { 时.
}\end{cases}
$$
求 $f(x)$ 在区间 $\left(\frac{7}{8}, \frac{8}{9}\right)$ 上的最大值.
%%<SOLUTION>%%
分析:因为当 $x$ 是无理数, 且 $x \in\left(\frac{7}{8}, \frac{8}{9}\right)$ 时, $f(x)=x<\frac{8}{9}$. 而 $f\left(\frac{15}{17}\right)=\frac{16}{17}>\frac{8}{9}$, 只需证 $x$ 是有理数时, $f(x) \leqslant \frac{16}{17}$ 即可.
解因为 $\frac{7}{8}<\frac{7+8}{8+9}<\frac{8}{9}$, 即 $\frac{7}{8}<\frac{15}{17}<\frac{8}{9}$.
由定义知 $f\left(\frac{15}{17}\right)=\frac{16}{17}$. 下面证明: $f(x) \leqslant \frac{16}{17}, x \in\left(\frac{7}{8}, \frac{8}{9}\right)$.
(1) 若 $x \in\left(\frac{7}{8}, \frac{8}{9}\right)$, 且 $x$ 是无理数,则 $f(x)=x<\frac{8}{9}<\frac{16}{17}$.
(2) 若 $x \in\left(\frac{7}{8}, \frac{8}{9}\right)$, 且 $x$ 是有理数, 设 $x=\frac{p}{q}$, 其中 $(p, q)=1,0< p<q$, 由于 $\frac{7}{8}<\frac{p}{q}<\frac{8}{9}$, 所以 $\left\{\begin{array}{l}7 q<8 p, \\ 9 p<8 q .\end{array}\right.$ 于是 $\left\{\begin{array}{l}7 q+1 \leqslant 8 p, \\ 9 p+1 \leqslant 8 q .\end{array}\right.$ 故 $7 q+1 \leqslant 8 p \leqslant 8 \cdot \frac{8 q-1}{9}$, 即 $63 q+9 \leqslant 64 q-8$. 所以 $q \geqslant 17$. 因此
$$
\begin{aligned}
f(x) & =f\left(\frac{p}{q}\right)=\frac{p+1}{q} \leqslant \frac{\frac{8 q-1}{9}+1}{q}=\frac{8 q+8}{9 q} \\
& =\frac{8}{9}+\frac{8}{9 q} \leqslant \frac{8}{9}+\frac{8}{9 \times 17}=\frac{16}{17} .
\end{aligned}
$$
综上所述, $f(x)$ 在区间 $\left(\frac{7}{8}, \frac{8}{9}\right)$ 上的最大值为 $f\left(\frac{15}{17}\right)=\frac{16}{17}$.
说明对于两个整数 $x 、 y$, 若 $x>y$, 则 $x \geqslant y+1$. 这一结论非常有用.
本题就是由 $\left\{\begin{array}{l}7 q<8 p, \\ 9 p<8 q\end{array}\right.$ 推出 $\left\{\begin{array}{l}7 q+1 \leqslant 8 p, \\ 9 p+1 \leqslant 8 q,\end{array}\right.$ 从而得出 $q \geqslant 17$ 的.
%%PROBLEM_END%%



%%PROBLEM_BEGIN%%
%%<PROBLEM>%%
例10 已知 $\alpha \in\left[0, \frac{\pi}{2}\right]$, 求 $y=\sqrt{5-4 \sin \alpha}+\sin \alpha$ 的最小值和最大值.
%%<SOLUTION>%%
分析:通过变量代换,把 $y$ 表示成二次函数的形式.
解设 $x=\sqrt{5-4 \sin \alpha}$, 因为 $0 \leqslant \sin \alpha \leqslant 1$, 所以 $1 \leqslant x \leqslant \sqrt{5}$, 且
$$
\begin{aligned}
& \text { i } \sin \alpha=\frac{5-x^2}{4} \text {, 于是 } \\
& y=x+\frac{5-x^2}{4}=-\frac{1}{4}(x-2)^2+\frac{9}{4}(1 \leqslant x \leqslant \sqrt{5}) .
\end{aligned}
$$
故当 $x=2$ 时, $y$ 的最大值为 $\frac{9}{4}$; 当 $x=1$ 时, $y$ 的最小值为 2 .
说明通过换元, 常常可以把较复杂的形式转化为较简单的形式, 从而使问题得以解决.
%%PROBLEM_END%%



%%PROBLEM_BEGIN%%
%%<PROBLEM>%%
例 11 已知 $x^2+4 y^2=4 x$, 求下列各式的最大值与最小值.
(1) $u=x^2+y^2$; (2) $v=x+y$.
%%<SOLUTION>%%
解:条件式可化为
$$
\frac{(x-2)^2}{4}+y^2=1 .
$$
设 $\left\{\begin{array}{l}x=2+2 \cos \theta, \\ y=\sin \theta\end{array}\right.$ ( $\theta$ 为参数), 则
(1) $u=(2+2 \cos \theta)^2+\sin ^2 \theta=3\left(\cos \theta+\frac{4}{3}\right)^2-\frac{1}{3}$,
故
$$
u_{\min }=0, u_{\max }=16 \text {. }
$$
(2) $v=2+2 \cos \theta+\sin \theta=\sqrt{5} \sin (\theta+\varphi)+2$,
所以
$$
v_{\min }=2-\sqrt{5}, v_{\max }=2+\sqrt{5} \text {. }
$$
说明若 $u^2+v^2=1$, 则令 $u=\sin \alpha, v=\cos \alpha, \alpha \in[0,2 \pi)$. 若 $u, v \in \mathbf{R}^{+}$, 且 $u+v=1$, 则令 $u=\sin ^2 \alpha, v=\cos ^2 \alpha, \alpha \in\left(0, \frac{\pi}{2}\right)$ 等等.
这些代换能帮助我们简化问题, 从而解决问题.
%%PROBLEM_END%%



%%PROBLEM_BEGIN%%
%%<PROBLEM>%%
例12 在约束条件 $x \geqslant 0, y \geqslant 0$ 及 $3 \leqslant x+y \leqslant 5$ 下, 求函数 $u=x^2- x y+y^2$ 的最大值和最小值.
%%<SOLUTION>%%
解:令 $x=a \sin ^2 \theta, y=a \cos ^2 \theta, \theta \in[0,2 \pi)$, 则 $3 \leqslant a \leqslant 5$. 于是
$$
\begin{aligned}
& \begin{aligned}
u & =a^2 \sin ^4 \theta-a^2 \sin ^2 \theta \cos ^2 \theta+a^2 \cos ^4 \theta \\
& =a^2\left[\left(\sin ^2 \theta+\cos ^2 \theta\right)^2-3 \sin ^2 \theta \cos ^2 \theta\right] \\
& =a^2\left(1-\frac{3}{4} \sin ^2 2 \theta\right) .
\end{aligned} \\
& \leqslant 2 \text {. 所以 } u_{\text {min }}=\frac{9}{4}, u_{\text {max }}=25 .
\end{aligned}
$$
从而 $\frac{9}{4} \leqslant u \leqslant 25$. 所以 $u_{\text {min }}=\frac{9}{4}, u_{\text {max }}=25$.
注意, $\frac{9}{4}$ 和 25 都是能取到的.
%%PROBLEM_END%%



%%PROBLEM_BEGIN%%
%%<PROBLEM>%%
例13 设 $f: \mathbf{R} \rightarrow \mathbf{R}$, 满足 $f(\cot x)=\cos 2 x+\sin 2 x$ 对所有 $0<x<\pi$ 成立, 又 $g(x)=f(x) f(1-x), x \in[-1,1]$, 求 $g(x)$ 的最大值和最小值.
%%<SOLUTION>%%
解:因为 $f(\cot x)=\frac{1-\tan ^2 x+2 \tan x}{1+\tan ^2 x}$, 令 $t=\cot x$, 则 $f(t)= \frac{t^2+2 t-1}{t^2+1}$.
故 $\quad g(x)=\frac{x^2+2 x-1}{x^2+1} \cdot \frac{(1-x)^2+2(1-x)-1}{(1-x)^2+1}$
$$
=\frac{\left(x^2-x\right)^2-8\left(x^2-x\right)-2}{\left(x^2-x\right)^2+2\left(x^2-x\right)+2} \text {. }
$$
令 $x^2-x=u, x \in[--1,1]$, 则 $u \in\left[-\frac{1}{4}, 2\right]$. 从而
$$
g(u)=\frac{u^2-8 u-2}{u^2+2 u+2}=1-\frac{10 u+4}{u^2+2 u+2}=1-2 \cdot \frac{5 u+2}{u^2+2 u+2} .
$$
令 $5 u+2=s, s \in\left[\frac{3}{4}, 12\right]$, 则 $u=\frac{s-2}{5}$, 于是
$$
\begin{gathered}
u^2+2 u+2=\frac{1}{25}\left(s^2+6 s+34\right), \\
\frac{5 u+2}{u^2+2 u+2}=\frac{25 s}{s^2+6 s+34}=\frac{25}{s+\frac{34}{s}+6} .
\end{gathered}
$$
由耐克函数的单调性知 $\frac{553}{12} \geqslant s+\frac{34}{s} \geqslant 2 \sqrt{34}$, 等号分别在 $s=\sqrt{34}, s=\frac{3}{4}$ 时取到.
所以
$$
\begin{gathered}
g_{\min }=1-2 \times \frac{25}{2 \sqrt{34}+6}=4-\sqrt{34}, \\
g_{\max }=1-2 \times \frac{25}{\frac{553}{12}+6}=\frac{1}{25} .
\end{gathered}
$$
说明本题来自于越南数学奥林匹克试题, 解决本题最关键也是最难的一步就是 $\circledast$ 式,需要一定的代数功夫.
%%PROBLEM_END%%



%%PROBLEM_BEGIN%%
%%<PROBLEM>%%
例14 求函数 $f(x)=\sqrt{x^4-3 x^2-6 x+13}-\sqrt{x^4-x^2+1}$ 的最大值, 及此时 $x$ 的值.
%%<SOLUTION>%%
分析:将原式整理成
$$
f(x)=\sqrt{(x-3)^2+\left(x^2-2\right)^2}-\sqrt{x^2+\left(x^2-1\right)^2}
$$
后, 可以发现 $\sqrt{(x-3)^2+\left(x^2-2\right)^2}$ 表示点 $P\left(x, x^2\right)$ 到点 $A(3,2)$ 的距离, $\sqrt{x^2+\left(x^2-1\right)}$ 表示点 $P\left(x, x^2\right)$ 到点 $B(0,1)$ 的距离, 再适当地用几何意义来解题.
解因为 $f(x)=\sqrt{(x-3)^2+\left(x^2-2\right)^2}-\sqrt{x^2+\left(x^2-1\right)^2}$, 它表示点 $P\left(x, x^2\right)$ 到点 $A(3,2)$ 与点 $B(0,1)$ 的距离之差, 如图(<FilePath:./figures/fig-c4e14.png>) 所示.
而 $P\left(x, x^2\right)$ 在抛物线 $y=x^2$ 上,易知
$$
|P A|-|P B| \leqslant|A B|
$$
取等号时, $P$ 在 $A B$ 的延长线上.
$A B$ 的方程为
$$
y=\frac{1}{3} x+1
$$
解方程组 $\left\{\begin{array}{l}y=x^2, \\ y=\frac{1}{3} x+1,\end{array}\right.$, 得
$$
x_1=\frac{1-\sqrt{37}}{6}, x_2=\frac{1+\sqrt{37}}{6} \text {. }
$$
因为点 $P$ 在 $A B$ 的延长线上, 舍去 $x_2$, 所以 $x=\frac{1-\sqrt{37}}{6}$.
最小距离为 $|A B|=\sqrt{10}$, 这正是需求的函数的最大值.
说明对于根号里的代数式, 我们常常通过配方, 把它看成平面上两个点之间的距离.
%%PROBLEM_END%%



%%PROBLEM_BEGIN%%
%%<PROBLEM>%%
例15 设 $x \in\left(0, \frac{\pi}{2}\right), y \in(0,+\infty)$, 求
$$
(\sqrt{2} \cos x-y)^2+\left(\sqrt{2} \sin x-\frac{9}{y}\right)^2
$$
的最小值.
%%<SOLUTION>%%
分析:联想到两点间的距离公式, $(\sqrt{2} \cos x-y)^2+\left(\sqrt{2} \sin x-\frac{9}{y}\right)^2$ 可视为平面直角坐标系中点 $A(\sqrt{2} \cos x, \sqrt{2} \sin x)$ 与 $B\left(y, \frac{9}{y}\right)$ 的距离的平方, 而这两点又可看作参数 $x, y$ 所确定的曲线上的点, 问题即可转化为求两曲线上点之间距离的最小值问题.
解如图(<FilePath:./figures/fig-c4e15.png>)所示, 建立直角坐标系 $u O v$.
设 $C_1:\left\{\begin{array}{l}u=\sqrt{2} \cos x, \\ v=\sqrt{2} \sin x,\end{array} x \in\left(0, \frac{\pi}{2}\right)\right.$
及 $C_2:\left\{\begin{array}{l}u=y, \\ v=\frac{9}{y}, y \in(0,+\infty) .\end{array}\right.$
则 $(\sqrt{2} \cos x-y)^2+\left(\sqrt{2} \sin x-\frac{9}{y}\right)^2$ 表示曲线 $C_1$ 上点 $A(\sqrt{2} \cos x, \sqrt{2} \sin x)$ 与曲线 $C_2$ 上点 $B\left(y, \frac{9}{y}\right)$ 之间的距离的平方.
作出曲线 $C_1: u^2+v^2=2(u>0, v>0)$ 及 $C_2: u v=9(u>0, v>0)$ 的图象, 显然, 当 $|O B|$ 取得最小值而 $O 、 A 、 B$ 三点共线时, $|A B|^2$ 最小.
因为
$$
|O B|=\sqrt{y^2+\left(\frac{9}{y}\right)^2} \geqslant 3 \sqrt{2} .
$$
当 $y=3$ 时,等号成立, 所以
$$
\begin{gathered}
|A B|=|O B|-|O A|=3 \sqrt{2}-\sqrt{2}=2 \sqrt{2} . \\
|A B|^2=8 .
\end{gathered}
$$
所以, $(\sqrt{2} \cos x-y)^2+\left(\sqrt{2} \sin x-\frac{9}{y}\right)^2$ 的最小值为 8 .
%%PROBLEM_END%%



%%PROBLEM_BEGIN%%
%%<PROBLEM>%%
例16 若实数 $x, y$ 满足关系
$$
\left\{\begin{array}{l}
y \geqslant x^2, \\
2 x^2+2 x y+y^2 \leqslant 5,
\end{array}\right.
$$
求函数 $w=2 x+y$ 的最大值和最小值.
%%<SOLUTION>%%
解:令 $u=x, v=x+y$, 则原问题就转化为在约束条件
$$
\left\{\begin{array}{l}
u^2+u \leqslant v, \\
u^2+v^2 \leqslant 5
\end{array}\right.
$$
下,求函数 $w=u+v$ 的最大值和最小值.
把约束条件在 $u v$ 平面上表示出来, 它就是如图 (<FilePath:./figures/fig-c4e16.png>) 所示的阴影部分(包括边界).
令 $u+v=m$, 即 $v=-u+m$, 它是倾角为 $\frac{3}{4} \pi$ 的直线系, $m$ 为它的纵截距.
我们把问题又转化为在直线系: $v=-u+m$ 中确定这样的直线, 它通过图 4-5 所示的闭区域中的至少一个点,且使得纵截距取得最大值和最小值.
从图 (<FilePath:./figures/fig-c4e16.png>) 中可以看出, 直线系中与抛物线 $v= u^2+u$ 相切于点 $T$ 的直线 $l_1$ 所对应的 $m_1$ 最小, 过抛物线与圆的右边的一个交点 $P$ 的直线 $l_2$ 所对应的 $m_2$ 最大,下面求 $m_1$ 和 $m_2$. 由
$$
\left\{\begin{array}{l}
u+v=m_1, \\
v=u^2+u
\end{array}\right.
$$
中消去 $v$, 得 $u^2+2 u-m_1=0 . \Delta=4+4 m_1=0$, 故 $m_1=-1$.
解方程组
$$
\left\{\begin{array}{l}
v=u^2+u, \\
u^2+v^2=5,
\end{array}\right.
$$
得点 $P$ 的坐标为 $(1,2)$. 因 $l_2$ 经过点 $P$, 故 $m_2=1+2=3$.
综上所述, 我们有 $w_{\min }=-1, w_{\max }=3$.
说明在约束条件下, 求函数的最大值和最小值是数学中常见的问题, 利用函数图象的性质和曲线系来求解这类问题, 往往能化繁为简, 出奇制胜.
%%PROBLEM_END%%



%%PROBLEM_BEGIN%%
%%<PROBLEM>%%
例17 求函数 $y=\sqrt{2 x^2-3 x+1}+\sqrt{x^2}-2 x$ 的最小值.
%%<SOLUTION>%%
解:先求定义域,再研究函数的单调区间.
易知定义域为 $(-\infty, 0] \cup[2,+\infty)$.
因为 $2 x^2-3 x+1$ 在 $(-\infty, 0]$ 上递减, 在 $[2,+\infty)$ 上递增, 所以 $\sqrt{2 x^2-3 x+1}$ 在 $(-\infty, 0]$ 上递减, 在 $[2,+\infty)$ 上递增.
同理, $\sqrt{x^2-} \overline{2 x}$ 在 $(-\infty, 0]$ 上递减, 在 $[2,+\infty)$ 上递增.
所以 $y==\sqrt{2 x^2-3 x+1}+\sqrt{x^2-2 x}$ 在 $(-\infty, 0]$ 上递减, 在 $[2,+\infty)$ 上递增.
所以 $y_{\min }=\min \{f(0), f(2)\}=\min \{\sqrt{1}, \sqrt{8-6+1}\}=1$.
所以 $y_{\text {min }}=1$ (在 $x=0$ 时取到).
说明本题的函数可看成两个函数的和, 而这两个函数在定义域内的单调性是一致的,利用“单调性一致的两个函数的和仍具有相同单调性”这一性质求出各个单调区间上的最小值, 再比较得出结论.
%%PROBLEM_END%%



%%PROBLEM_BEGIN%%
%%<PROBLEM>%%
例18 设 $a>0, r(x)=\frac{a x^2+1}{x}=a x+\frac{1}{x}$. 试讨论函数 $r(x)$ 在 $(0$, $+\infty)$ 中的单调性, 最小值与最大值.
%%<SOLUTION>%%
解:设 $0<x_1<x_2<+\infty$, 则
$$
r\left(x_2\right)-r\left(x_1\right)=a x_2+\frac{1}{x_2}-a x_1-\frac{1}{x_1}=\left(x_2-x_1\right)\left(a-\frac{1}{x_1 x_2}\right) .
$$
当 $0<x_1<x_2 \leqslant \frac{1}{\sqrt{a}}$ 时, 得
$$
\begin{aligned}
r\left(x_2\right)-r\left(x_1\right) & =\left(x_2-x_1\right)\left(a-\frac{1}{x_1 x_2}\right)<\left(x_2-x_1\right)\left(a-\frac{1}{x_2^2}\right) \\
& \leqslant\left(x_2-x_1\right)(a-a)=0,
\end{aligned}
$$
所以在区间 $\left(0, \frac{1}{\sqrt{a}}\right]$ 上, $r(x)$ 是单调递减的.
当 $\frac{1}{\sqrt{a}} \leqslant x_1<x_2<+\infty$ 时, 得
$$
r\left(x_2\right)-r\left(x_1\right)=\left(x_2-x_1\right)\left(a-\frac{1}{x_1 x_2}\right)>\left(x_2-x_1\right)\left(a-\frac{1}{x_1^2}\right) \geqslant 0,
$$
所以在区间 $\left[\frac{1}{\sqrt{a}},+\infty\right)$ 上, $r(x)$ 是单调递增的.
因此, $r(x)$ 没有最大值, 当且仅当 $x=\frac{1}{\sqrt{a}}$ 时, $r(x)$ 取最小值, 最小值为 $r\left(\frac{1}{\sqrt{a}}\right)=2 \sqrt{a}$.
%%PROBLEM_END%%



%%PROBLEM_BEGIN%%
%%<PROBLEM>%%
例19 已知 $\alpha, \beta$ 是方程 $4 x^2-4 t x-1=0(t \in \mathbf{R})$ 的两个不等实根, 函数 $f(x)=\frac{2 x-t}{x^2+1}$ 的定义域为 $[\alpha, \beta]$.
(1) 求 $g(t)=\max f(x)-\min f(x)$;
(2) 证明: 对于 $u_i \in\left(0, \frac{\pi}{2}\right)(i=1,2,3)$, 若 $\sin u_1+\sin u_2+\sin u_3=$ 1 , 则
$$
\frac{1}{g\left(\tan u_1\right)}+\frac{1}{g\left(\tan u_2\right)}+\frac{1}{g\left(\tan u_3\right)}<\frac{3}{4} \sqrt{6}
$$
%%<SOLUTION>%%
解:(1) 由于 $\alpha<\beta$, 所以
$$
\alpha=\frac{t-\sqrt{t^2+1}}{2}, \beta=\frac{t+\sqrt{t^2+1}}{2} .
$$
令 $u=2 x-t$, 则 $x=\frac{u+t}{2}$,
$$
f(x)=\frac{u}{\left(\frac{u+t}{2}\right)^2+1}=\frac{4 u}{u^2+2 t u+t^2+4} .
$$
若 $u=0$, 即 $x=\frac{t}{2}$, 此时 $f(x)==0$.
若 $u \neq 0$, 则
$$
f(x)=-\frac{4}{u+\frac{t^2+4}{u}+2 t}, u \in\left[-\sqrt{t^2+1}, 0\right) \cup\left(0, \sqrt{t^2+1}\right] .
$$
设 $h(u)=u+\frac{t^2+4}{u}, u \in\left[-\sqrt{t^2+1}, \sqrt{t^2+1}\right]$, 易知 $h(u)$ 在$\left[-\sqrt{t^2+1}, 0\right)$ 上是递减的, 且 $h(u)<0, h(u)$ 在 $\left(0, \sqrt{t^2+1}\right]$ 上也是递减的, 且 $h(u)>0$, 所以, $f(x)$ 在 $\left[-\sqrt{t^2+1}, 0\right)$ 上是递增的, 在 $\left(0, \sqrt{t^2+1}\right]$ 上也是递增的, 且当 $u \in\left[-\sqrt{t^2+1}, 0\right)$ 时, $f(u)<0$, 当 $u \in\left(0, \sqrt{t^2+1}\right]$ 时, $f(u)>0$. 于是 $f(x)$ 在 $[\alpha, \beta]$ 上是递增的.
故
$$
\begin{aligned}
g(t) & =f(\beta)-f(\alpha) \\
& =\frac{(\beta-\alpha)[t(\alpha+\beta)-2 \alpha \beta+2]}{\alpha^2 \beta^2+\alpha^2+\beta^2+1} \\
& =\frac{8 \sqrt{t^2+1}\left(2 t^2+5\right)}{16 t^2+25} .
\end{aligned}
$$
(2) 因为由 (1) 知
$$
\begin{aligned}
g\left(\tan u_i\right) & =\frac{8 \cdot \sec u_i\left(2 \sec ^2 u_i+3\right)}{16 \sec ^2 u_i+9} \\
& =\frac{24 \cos ^2 u_i+16}{9 \cos ^3 u_i+16 \cos u_i}, i=1,2,3 .
\end{aligned}
$$
所以
$$
\begin{aligned}
\frac{1}{g\left(\tan u_i\right)} & =\frac{9 \cos ^3 u_i+16 \cos u_i}{24 \cos ^2 u_i+16} \\
& =\frac{3}{8} \cos u_i+\frac{5 \cos u_i}{12 \cos ^2 u_i+8} \\
& \leqslant \frac{3}{8} \cos u_i+\frac{5 \cos u_i}{2 \sqrt{12 \times 8} \cos u_i} \\
& =\frac{3}{8} \cos u_i+\frac{5 \sqrt{6}}{48}
\end{aligned}
$$
而由柯西不等式, 得
$$
\begin{gathered}
\left(\cos u_1+\cos u_2+\cos u_3\right)^2 \\
\leqslant 3\left(\cos ^2 u_1+\cos ^2 u_2+\cos ^2 u_3\right)
\end{gathered}
$$
$$
\begin{aligned}
& =3\left[3-\left(\sin ^2 u_1+\sin ^2 u_2+\sin ^2 u_3\right)\right] \\
& =9-3\left(\sin ^2 u_1+\sin ^2 u_2+\sin ^2 u_3\right) \\
& \leqslant 9-3 \cdot \frac{1}{3}\left(\sin u_1+\sin u_2+\sin u_3\right)^2=8 .
\end{aligned}
$$
所以
$$
\cos u_1+\cos u_2+\cos u_3 \leqslant 2 \sqrt{2} .
$$
于是
$$
\begin{aligned}
& \frac{1}{g\left(\tan u_1\right)}+\frac{1}{g\left(\tan u_2\right)}+\frac{1}{g\left(\tan u_3\right)} \\
\leqslant & \frac{3}{8}\left(\cos u_1+\cos u_2+\cos u_3\right)+\frac{5 \sqrt{6}}{16} \\
\leqslant & \frac{3 \sqrt{2}}{4}+\frac{5 \sqrt{6}}{16}<\frac{7}{16} \sqrt{6}+\frac{5}{16} \sqrt{6} \\
= & \frac{3}{4} \sqrt{6}
\end{aligned}
$$
%%PROBLEM_END%%


