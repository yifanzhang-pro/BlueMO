
%%PROBLEM_BEGIN%%
%%<PROBLEM>%%
问题1. 某次考试有 5 道选择题, 每题都有 4 个不同的答案供选择, 每人每题恰选一个答案.
在 2000 份答卷中发现存在一个数 $n$, 使得任何 $n$ 份答卷中都存在 4 份, 其中每两份答卷选择的答案都至多有 3 题相同.
求 $n$ 的最小可能值.
%%<SOLUTION>%%
$n$ 的最小可能值是 25 . 将每道题的 4 种答案分别记为 $1 、 2 、 3 、 4$, 每份试卷上的答案记为 $(g, h, i, j, k)$, 其中 $g, h, i, j, k \in\{1,2,3,4\}$. 对于所有 2000 份答案 $(g, h, i, j, k)$, 将后 4 个分量完全相同的看作一类, 则共有 $4^4=256$ 类.
因为 $2000=256 \times 7+208$, 于是必有 8 份试卷属于同一个类 $A$.取出这 8 份试卷,剩下 1992 份试卷中仍有 8 份试卷属于同一个类 $B$. 再取出这 8 份试卷, 剩下 1984 份试卷中仍有 8 份试卷属于同一个类 $C$. 取出的 24 份试卷共属于 3 个类 $A 、 B 、 C$, 于是, 当 $n \leqslant 24$ 时, 从这 24 份试卷中任取 $n$ 份,
则 $n$ 份中的任何 4 份试卷必有 2 份属于同一个类, 不满足题目要求, 于是 $n \geqslant 25$.
下面构造这样的 2000 份答卷, 它们共有 250 种不同答案, 同一种答案的试卷各有 8 份.
这 250 种答案是满足 $g+h+i+j+k \equiv 0(\bmod 4)$ 的所有 $4^4=256$ 种答案 $(g, h, i, j, k)$ 中的任意 250 种.
显然, 对于任何 2 份不同的答案, 它们至多有 3 个分量相同, 否则, 若有 4 个分量相同, 则由同余式可知, 第 5 个分量也相同, 矛盾.
在这样的 2000 份答卷中任取 25 份, 由于相同的答卷至多有 8 份, 从而至少有 4 份试卷是两两不同的, 它们至多有 3 个分量相同,故 $n=25$ 合乎题目要求.
%%PROBLEM_END%%



%%PROBLEM_BEGIN%%
%%<PROBLEM>%%
问题2. 设 $a_1, a_2, \cdots, a_k$ 是以不超过 $n$ 的正整数为项的有限数列, 其中任何一个项的两个相邻项都不同, 且不存在任何四个指标 $p<q<r<s$, 使得 $a_p= a_1 \neq a_q=a_s$. 求 $k$ 的最大值.
%%<SOLUTION>%%
数列 $n, n, n-1, n-1, \cdots, 2,2,1,1,2,2, \cdots, n-1, n-1, n, n$ 合乎要求, 因为 $a_i=a_j$, 必有 $i 、 j$ 相邻或关于数列中间两项 1,1 对称.
所以 $k$ 的最大值不小于 $4 n-2$. 其次, 在任意合乎条件的数列中, 任何连续三项都不能是同一数.
而且若某一项的所有邻项 (注意首项、末项只有一个邻项) 都与该项不同,则在该项旁添加一个与该项相同的项时, 新数列仍满足条件.
于是, 只须考虑每连续两项相同的数列.
进一步, 只须研究任何连续两项都不同的数列的项数 (在每两个相同的项中去掉一个项).下面用归纳法证明: 如果数列 $a_1, a_2, \cdots, a_k$ 中的任何连续两项都不同, 其中每一个项都是不大于 $n$ 的自然数, 且不存在任何四个指标 $p<q<r<s$, 使得 $a_p=a_1 \neq a_q=a_s$. 则 $k \leqslant 2 n-1$. 当 $n=2$ 时, 结论显然成立.
设结论对小于 $n$ 的自然数成立.
考察 $n$ 的情形.
设 $a_1, a_2, \cdots, a_k$ 是一个最长的合乎条件的数列, $a_i \in\{1,2, \cdots$, $n\}(i=1,2, \cdots, k)$. 记 $a_k=t, 1 \leqslant t \leqslant n$. 若 $a_1, a_2, \cdots, a_{k-1}$ 中任何一个项的值都不是 $t$, 则可在 $a_1$ 之前添加一项 $t$, 得到更长的数列, 矛盾.
于是 $a_1$, $a_2, \cdots, a_{k-1}$ 中至少有一个项是 $t$. 设 $a_v$ 是 $a_1, a_2, \cdots, a_{k-1}$ 中下标最大的一个为 $t$ 的项 $(1 \leqslant v \leqslant k-2)$, 则 $a_{v+1}, a_{v+2}, \cdots, a_{k-1}$ 都不是 $t$. 若 $v=1$, 则 $a_2$, $a_3, \cdots, a_{k-1}$ 中没有等于 $t$ 的项.
考察数列 $a_2, a_3, \cdots, a_{k-1}$, 每一项都在 $\{1$, $2, \cdots, n\} \backslash\{t\}$ 中, 且满足题设条件, 由归纳假设知, $k-2 \leqslant 2(n-1)-1$, 所以 $k \leqslant 2 n-1$, 结论成立.
若 $v>1$, 则令 $A=\left\{a_1, a_2, \cdots, a_v\right\}, B=\left\{a_{v+1}\right.$, $\left.a_{v+2}, \cdots, a_{k-1}\right\}$, 则 $A 、 B$ 中分别没有数相同的连续两项, 且 $B$ 的项都在 $\{1$, $2, \cdots, n\} \backslash\{t\}$ 中, $A$ 的项都在 $\{1,2, \cdots, n\} \backslash\{B\}$ 中, 从而都可利用归纳假设.
设 $A$ 中互异的项的个数为 $p, B$ 中互异的项的个数为 $q$, 则 $p+q \leqslant n$. 这样分别对 $A 、 B$ 使用归纳假设, 有 $v \leqslant 2 p-1, k-v-1 \leqslant 2 q-1$. 所以 $k \leqslant 2 q+ v \leqslant 2 q+2 p-1=2(p+q)-1 \leqslant 2 n-1$. 故 $k$ 的最大值为 $4 n-2$.
%%PROBLEM_END%%



%%PROBLEM_BEGIN%%
%%<PROBLEM>%%
问题3. 设有 $2^n$ 个由数字 $0 、 1$ 组成的有限数列, 其中任何一个数列都不是另一个数列的前段.
求所有数列的长度和 $S$ 的最小值.
%%<SOLUTION>%%
一个数列可以看成是一个排列.
对两个不同的排列, 其中一个不是另一个的前段的一个充分条件是: 它们的长度相等.
假设它们的长度都是 $r$, 那么互异的排列有 $2^r$ 个.
但题给的数列有 $2^n$ 个, 所以 $r=n$. 即长为 $n$ 的互异的排列有 $2^n$ 个,它们中任何一个不是另一个的前段, 此时 $S=n \times 2^n$. 下面证明 $S \geqslant n \times 2^n$. 我们只须把任意一个合乎条件的排列集中的每一个排列都操作到长度不小于 $n$. 我们称长度为 $n$ 的排列为标准排列, 长度小 (大) 于 $n$ 的排列为短 (长) 排列.
如果合乎条件的排列集中存在短排列, 则必存在长排列.
否则, 每个短排列至少可以扩充为 2 个互异的标准排列, 使得标准排列个数超过 $2^n$, 矛盾.
任取一个短排列 $A$, 必存在长排列 $B$. 去掉 $A, B$, 加人两个排列 : $A \cup \{0\}, A \cup\{1\}$, 得到的排列集仍合乎条件.
因为 $|A|<n,|B|>n$, 于是操作后长度和 $S$ 的增量为 $|A|+1+|A|+1-|A|-|B|=2+|A|-|B| \leqslant 2+(n-1)-(n+1)=0$, 即 $S$ 不增.
如此下去, 直至排列中不存在短排列, 必有 $S \geqslant n \times 2^n$.
%%PROBLEM_END%%



%%PROBLEM_BEGIN%%
%%<PROBLEM>%%
问题4. 在 $m \times n(m>1, n>1)$ 棋盘上放有 $r$ 只棋, 每个格最多一只棋.
若 $r$ 只棋具有如下的性质 $p$ : 每行每列至少有一只棋.
但去掉其中任何一只棋, 则它们便不再具有上述的性质 $p$. 求 $r$ 的最大值 $r(m, n)$.
%%<SOLUTION>%%
$r(m, n)=m+n-2$. 一方面, 由构图可知, $r=m+n-2$ 是可能的(第一行与第一列各格除第一格外都放棋).下面证明: $r<m+n-1$. 即证明棋盘上放有至少 $m+n-1$ 只棋时, 棋盘中有可去棋.
对 $m+n$ 归纳.
当 $m=2, n=$ 2 , 棋盘上放有至少 3 只棋, 显然有可去棋.
设结论对 $m+n<k$ 的正整数 $(m$, $n$ )成立.
考察 $m+n=k$ 的情形.
此时,棋盘中至少放有 $k-1$ 只棋.
若 $n=2$ 或 $m=2$, 则结论显然成立.
若 $m>2$ 且 $n>2$, 则由于棋盘上至少放有 $m+ n-1>m$ 只棋, 必有一行至少含有两只棋.
不妨设第一行的棋最多, 共有 $t$ 只棋, 设为 $a_{11}, a_{12}, \cdots, a_{1 t}(t>1)$. 则这些棋所在的列中不能再有其他的棋, 比如, 若 $a_{11}$ 所在的列中还有一只棋, 则 $a_{11}$ 可去.
于是, 去掉第一行和前 $t$ 列, 剩下 $(m-1) \times(n-t)$ 棋盘, 此棋盘中至少有 $m+n-1-t=(m-1)+(n-t)$ 只棋.
注意到 $(m-1)+(n-t)=k-t-1<k$, 若 $n-t \geqslant 2$, 对 $(m-1) \times (n-t)$ 棋盘利用归纳假设, 知 $(m-1) \times(n-t)$ 棋盘中存在可去棋.
若 $n-t \leqslant$ 1 , 则棋均分布在第 1 行或第 $n$ 列上.
亦容易得出有可去棋.
命题获证.
%%PROBLEM_END%%



%%PROBLEM_BEGIN%%
%%<PROBLEM>%%
问题5. 设 $F=\left\{A_1, A_2, \cdots, A_k\right\}$ 是 $X=\{1,2, \cdots, n\}$ 的子集族, 且满足:
(1) $\left|A_i\right|=3$, (2) $\left|A_i \cap A_j\right| \leqslant 1$. 记 $|F|$ 的最大值为 $f(n)$, 求证: $\frac{n^2-4 n}{6} \leqslant f(n) \leqslant \frac{n^2-n}{6}$. 
%%<SOLUTION>%%
因为 $\left|A_j \cap B_j\right| \leqslant 1$, 我们来计算二元子集的个数 $S$. 一方面, $\left|A_i\right|=3$, 从而每个 $A_i$ 中有 $C_3^2=3$ 个二元集, 于是, $F$ 中的 $k$ 个集合可产生 $3 k$ 个二元集.
由于 $\left|A_i \cap B_j\right| \leqslant 1$, 这 $3 k$ 个二元集互异.
所以, $S \geqslant 3 k$. 另一方面, 设 $|X|=n$, 则 $X$ 中的二元集的总数为 $S=\mathrm{C}_n^2$. 所以, $\mathrm{C}_n^2=S \geqslant 3 k$, 所以, $k \leqslant \frac{n^2-n}{6}$. 不等式右边获证.
下面, 我们只须构造出一个合乎条件的子集族 $F$, 使 $|F| \geqslant \frac{n^2-4 n}{6}$. 为了构造 $F$, 只须构造出若干个 3 -数组 $\{a, b, c\}$, 使 $\left|A_i \cap A_j\right| \leqslant 1$. 考察反面: $\left|A_i \cap A_j\right| \geqslant 2$, 此时 $A_i 、 A_j$ 中第三个元素要互异,否则 $A_i=A_j$. 现在要设法破坏这一性质, 即当 $A_i 、 A_j$ 中有两个元素对应相等时, 第三个元素必然相等.
这表明 $a 、 b 、 c$ 中只有两个自由量, 第三个量 $c$ 由前两个量 $a 、 b$ 唯一确定.
由此想到 $a 、 b 、 c$ 满足一个方程, 最简单的方程是线性方程: $a+b+c=0$. 但 $a 、 b 、 c$ 都是正整数, 此等式不成立, 可将此方程改为同余式: $a+b+c \equiv 0(\bmod n)$. 考察所有满足 $a+b+c \equiv 0(\bmod n)$ 的三数组 $\{a, b, c\}$, 每一个这样的三数组组成一个集合, 则这些集合合乎条件.
现在证明这些集合的个数不少于 $\frac{n^2-4 n}{6}$. 注意到满足 $a+b+c \equiv 0(\bmod n)$ 的三数组 $\{a, b, c\}$ 由其中的两个数比如 $a 、 b$ 唯一确定, 这是因为 $c= \left\{\begin{array}{ll}n-(a+b) & (a+b<n) \\ 2 n-(a+b) & (a+b \geqslant n)\end{array}\right.$ (1). 因此, 我们只须考虑 $a 、 b$ 分别有多少种取值.
首先, $a$ 有 $n$ 种取值方法.
当 $a$ 取定后, $b$ 的取值应满足如下两个条件: (1) $b \neq a$; (2) $b$ 的值应使 $c \neq a$ 和 $b$. 其中 $c$ 由(1)确定.
显然, (2) 成立的一个充分条件是 $n-(a+b) \neq a$ 和 $b$, 且 $2 n-(a+b) \neq a$ 和 $b$. 即 $b \neq n-2 a$ 和 $\frac{n-a}{2}$, 且 $b \neq 2 n-2 a$ 和 $\frac{2 n-a}{2}$. 但 $n-2 a$ 和 $2 n-2 a$ 不同属于 $X$, 否则 $1 \leqslant n-2 a \leqslant n$, $1 \leqslant 2 n-2 a \leqslant n$, 有 $2 a+1 \leqslant n \leqslant 2 a$, 矛盾.
于是, 满足 (2) 的 $b$ 最多有 3 个值不能取, 连同 (1), $b$ 最多有 4 个值不能取, 从而至少有 $n-4$ 种取法, 于是, 这样的三数组至少有 $n(n-4)$ 个.
但每个三数组 $\{a, b, c\}$ 有 6 个不同的顺序, 被计数 6 次, 故合乎条件的三数组至少有 $\frac{n(n-4)}{6}$ 个.
综上所述, 命题获证.
%%PROBLEM_END%%



%%PROBLEM_BEGIN%%
%%<PROBLEM>%%
问题6. 在 $m \times n(m>1, n>1)$ 棋盘 $C$ 中, 每格填一个数, 使对任何正整数 $p 、 q$ 及任何 $p \times q$ 矩形, 相对顶点两格所填的数的和相等.
若对适当的 $r$ 个格填数后, 余下各格所填的数被唯一确定, 求 $r$ 的最小值.
%%<SOLUTION>%%
填好数表的第一行和第一列后, 数表被唯一确定, 此时, 数表只填了 $m+n-1$ 个数.
即 $r=m+n-1$ 时, 存在相应的填法.
下面证明, 对所有合乎条件的填法, 有 $r \geqslant m+n-1$. 用反证法.
即 $r \leqslant m+n-2$ 时, 不论怎样填表, 数表都不唯一确定.
对 $m+n$ 归纳.
当 $m+n=4$ 时, $m=n=2,2 \times 2$ 数表中填 2 个数, 数表不唯一确定, 结论成立.
设结论对 $m+n=k$ 成立.
考察 $m+n= k+1>4$ 时的情形.
数表填人 $r \leqslant m+n-2=k-1$ 个数, 我们要证明此数表不唯一确定.
为了利用假设, 应去掉数表的一行或一列.
而且去掉的行或列应具有这样的性质: (1) 去掉这行后, 数表至少还有两行 (否则不能利用归纳假设). 为此, 不妨设 $m \leqslant n$, 则 $n>2$. 于是可去掉一个列.
(2) 去掉这列后, $(m-1) \times n$ 数表中的数不多于 $(m-1)+n-1=k-2$ 个数, 即去掉的列中至少有一个数.
(3) 加上去掉的这个列, 数表仍不唯一确定, 这只须此列中的数不多于一个 (一个充分条件). 实际上, 若有一个数 $a$, 在子表中不被唯一确定, 而被后加上的一列中的两个数 $b 、 c$ 及与 $a$ 同列的一个数 $d$ 唯一确定, 但这列中只有一个数,则不妨设 $c$ 是新填人的, 则 $c$ 只能由另一列中的两个数 $e$ 、 $f$ 确定.
于是, $a$ 可由子表中的 $e 、 f 、 d$ 唯一确定.
矛盾.
现在要找到合乎上述两个条件的一个列.
即此列中恰有一个数.
注意到 $m \leqslant n$, 所以, 棋盘中的数的个数 $r \leqslant k-1=m+n-2 \leqslant 2 n-2<2 n$. 所以, 至少有一列不多于一个数.
若此列中没有数, 则此列不唯一确定, 结论成立.
若此列中恰有一个数 $a$, 则去掉此列, 其余的数必可唯一确定, 否则假设 $b$ 不唯一确定, 则 $b$ 要由 $a$ 确定, 从而 $a, b$ 是同一矩形的两个顶点,且此矩形中除 $b$ 以外的 3 个数都已知, 这 3 个数中除 $a$ 外必有一个与 $a$ 同列, 矛盾.
于是, $(m-1) \times n$ 数表对应的最小值 $\leqslant r-1 \leqslant k-2=m+n-3<m+n-2$, 与归纳假设矛盾.
%%PROBLEM_END%%


