
%%PROBLEM_BEGIN%%
%%<PROBLEM>%%
问题1. 1650 个学生排成 22 行、75 列.
已知其中任意两列处于同一行的两个人中, 性别相同的学生都不超过 11 对.
证明: 男生的个数不超过 928.
%%<SOLUTION>%%
设第 $i$ 行的男生数为 $a_i$, 则女生数为 $75-a_i$. 依题意, 可知, $\sum_{i=1}^{22}\left(\mathrm{C}_{a_i}^2+\right. \left.\mathrm{C}_{75-a_i}^2\right) \leqslant 11 \times \mathrm{C}_{75}^2$. 这是因为任意给定的两列处于同一行的两个人中, 性别相同的学生不超过 11 对, 故所有性别相同的两人对的个数不大于 $11 \times \mathrm{C}_{75}^2$. 于是, 我们有 $\sum_{i=1}^{22}\left(a_i^2-75 a_i\right) \leqslant-30525$, 即 $\sum_{i=1}^{22}\left(2 a_i-75\right)^2 \leqslant 1650$. 利用柯西不等式, 可知 $\left[\sum_{i=1}^{22}\left(2 a_i-75\right)\right]^2 \leqslant 22 \sum_{i=1}^{22}\left(2 a_i-75\right)^2 \leqslant 36300$, 因此, $\sum_{i=1}^{22}\left(2 a_i-75\right)<$ 191 , 从而 $\sum_{i=1}^{22} a_i<\frac{191+1650}{2}<921$. 所以,男生的个数不超过 928 .
%%PROBLEM_END%%



%%PROBLEM_BEGIN%%
%%<PROBLEM>%%
问题2. 一次会议有 $12 k$ 个人参加, 每人恰与其中 $3 k+6$ 个人打过招呼.
对任意两人, 与他们打过招呼的人数都相等, 问: 此次会议有多少个人参加?
%%<SOLUTION>%%
用点表示人, 对打过招呼的两个人, 对应的两个点连线, 得到一个简单图.
计算此图中角的个数 $S$. 一方面, 对 $G$ 计算, 每个点引出 $3 k+6$ 条边, 有 $\mathrm{C}_{3 k+6}^2$ 个角.
$12 k$ 个点, 共有 $12 k \mathrm{C}_{3 k+6}^2$ 个角.
又每个角都有顶点, 且不同的顶点对应的角不同, 于是 $S=12 k \mathrm{C}_{3 k+6}^2$. 另一方面, 从 " 2 人组"出发, 因为对任意两人,与他们打过招呼的人数都相等.
设这样的人数为 $t$, 则每个 2 人组都得到 $t$ 个角.
又每个角都对应一个 2 人组, 且不同的 2 人组对应的角不同, 于是 $S= t \mathrm{C}_{12 k}^2$. 所以 $12 k \mathrm{C}_{3 k+6}^2=t \mathrm{C}_{12 k}^2$, 解得 $t=\frac{(3 k+6)(3 k+5)}{12 k-1}$. 所以 $16 t= \frac{(12 k-1+25)(12 k-1+21)}{12 k-1}=(12 k-1)+25+21+\frac{25 \times 21}{12 k-1}$. 显然,(3, $12 k-1)=1$, 所以 $12 k-1 \mid 25 \times 7$. 注意到 $12 k-1$ 模 4 余 3 , 所以 $12 k-1=7$, $5 \times 7,5^2 \times 7$, 其中只有 $12 k-1=5 \times 7$ 有整数解 $k=3 、 t=6$, 所以会议的人数只可能是 36 . 下面证明: 36 人的会议是可能的.
即存在 36 阶图 $G, G$ 中每个点的度是 15 , 并且对每一对点, 同时与它们相连的点都有 6 个.
先作 6 个完全图 $K_6$. 每个完全图的顶点都用 $1 、 2 、 3 、 4 、 5 、 6$ 编号.
现在将这 6 个完全图的有关点用边联接, 构成图 $G . G$ 中的顶点记为 $(i, j)$, 它表示第 $i$ 个完全图中的第 $j$ 个顶点.
注意到每个点已与所在完全图中的 5 个点相连, 现将每个点再与纵坐标相同的点相连, 则每个点又连了 5 条边.
再将坐标差相同的点相连, 即 $i-j=i^{\prime}-j^{\prime}$, 则 $(i, j)$ 与 $\left(i^{\prime}, j^{\prime}\right)$ 相连.
这样, 每个点又引出了 5 条边.
所以每个点都连了 15 条边且每个点向它不在的完全图都有 2 点相连, 其中一个点与它的纵坐标相同, 另一个点与它的坐标差相同.
对任意两个点, 若某个点与它们都相连, 则称这两个点对了一个角.
下面证明 $G$ 中的任何两个点都对了 6 个角.
实际上, 考察任意的两个点 $(i, j) 、\left(i^{\prime}, j^{\prime}\right)$, 若 $i=i^{\prime}$, 即这两个点在同一个完全图中, 于是它们在该完全图中对了 4 个角.
此外, 恰有两个点: $\left(i^{\prime}-\right. \left.j^{\prime}+j, j\right)$ 和 $\left(i-j+j^{\prime}, j^{\prime}\right)$ 这两个点与它们相连, 所以它们共对了 6 个角.
若 $i \neq i^{\prime}$, 且 $i-j=i^{\prime}-j^{\prime}$, 则 $\left(i, j^{\prime}\right)$ 和 $\left(i^{\prime}, j\right)$ 与它们都相连.
此外, 对 $i, i^{\prime}$ 以外的 4 个 $i^{\prime \prime}$, 点 $\left(i^{\prime \prime}, i^{\prime \prime}-i+j\right)$ 与它们都相连, 其他点都不同时与它们相连.
所以它们对了 6 个角.
若 $i \neq i^{\prime}$, 且 $i-j \neq i^{\prime}-j^{\prime}$, 则 $\left(i, j^{\prime}\right) 、\left(i^{\prime}, j\right) 、\left(i, i-i^{\prime}+\right. \left.j^{\prime}\right)$ 和 $\left(i^{\prime}, i^{\prime}-i+j\right)$ 与它们都相连.
此外, 还有两点 $\left(i^{\prime}-j^{\prime}+j, j\right) 、(i-j+ \left.j^{\prime}, j^{\prime}\right)$ 与它们都相连, 其他点都不同时与它们相连.
所以它们对了 6 个角.
综上述,参加会议的人数为 36 .
%%PROBLEM_END%%



%%PROBLEM_BEGIN%%
%%<PROBLEM>%%
问题3. 有 $n$ 名选手参加比赛, 历时 $k$ 天, 其中任何一天 $n$ 名选手的得分都恰好是 . $1,2,3, \cdots, n$ 的一个排列.
如果在第 $k$ 天末, 每个选手的总分都是 26 . 求 $(n, k)$ 的所有可能取值.
%%<SOLUTION>%%
由条件"每个选手的总分都是 26 ", 想到计算第 $k$ 天后 $n$ 名选手得分之和 $S$. 一方面, 每天的得分为 $1+2+3+\cdots+n$, 所以, $S=k(1+2+\cdots+n)$. 另一方面, 每个选手得 26 分, 从而 $S=26 n$. 所以, $k(n+1)=52$. 所以 $(n, k)= (51,1) 、(25,2) 、(12,4) 、(3,13)$. 其中 $(n, k)=(51,1)$ 时, 各选手的总分互异, 矛盾, 故舍去.
由下面的构造可知, 其他 3 种情况都是可能的.
当 $(n$, $k)=(25,2)$ 时, 第 $i$ 号选手的名次集合为 $A_i=\{i, 26-i\}(i=1,2, \cdots$, $25)$. 当 $(n, k)=(12,4)$ 时, 第 $i$ 号选手的名次集合为 $A_i=\{i, 13-i, i, 13-i\}(i=1,2, \cdots, 12)$. 当 $(n, k)=(3,13)$ 时, 各选手的名次集合为 $A_1= \{2,3,1\} \cup\{1,3,1,3, \cdots, 1,3\} 、 A_2=\{3,1,2\} \cup\{2,2,2,2, \cdots, 2$, 2\}、 $A_3=\{1,2,3\} \cup\{3,1,3,1, \cdots, 3,1\}$.
%%PROBLEM_END%%



%%PROBLEM_BEGIN%%
%%<PROBLEM>%%
问题4. 某班有 30 个学生, 年龄互不相同, 每个学生在同班中有相同个数的朋友.
对某个学生 $A$, 若 $A$ 的年龄比 $A$ 的一半以上 (不包括一半) 朋友大, 则称 $A$ 为大龄的.
问大龄的学生最多有多少个?
%%<SOLUTION>%%
设有 $t$ 个大龄学生, 每个学生都有 $k$ 个朋友.
为叙述问题方便, 用 30 个点表示 30 个学生.
对任何两个点 $A 、 B$, 如果 $A 、 B$ 是朋友, 且 $A$ 的年龄大于 $B$ 的年龄, 则连一条指向 $B$ 的有向边, 得到一个竟赛图.
称大龄学生对应的点为 "大点", 设所有的大点为 $A_1, A_2, \cdots, A_t$. 依题意, 有 $d^{+}\left(A_i\right)>d\left(A_i\right)$, $d\left(A_i\right)=d^{+}\left(A_i\right)+d^{-}\left(A_i\right)=k$, 于是 $d^{+}\left(A_i\right) \geqslant \frac{k+1}{2}$. 
不妨设 $A_1$ 的年龄 $\leqslant A_2$ 的年龄 $\leqslant \cdots \leqslant A_t$ 的年龄, 则 $d^{+}\left(A_1\right) \leqslant 30-t\left(A_1\right.$ 最多向 $A_1, A_2, \cdots, A_t$ 外的 $30-t$ 个点引出边 $), d^{+}\left(A_t\right)=k$. 计算所有 "大点" 的出度的和 $S$. 
一方面, $S=d^{+}\left(A_1\right)+d^{+}\left(A_2\right)+\cdots+d^{+}\left(A_{t-1}\right)+k \geqslant \frac{k+1}{2}(t-1)+k$. 另一方面, $S \leqslant\|G\|=15 k$, 所以 $15 k=\|G\| \geqslant S \geqslant \frac{k+1}{2}(t-1)+k$, 所以 $t \leqslant \frac{28 k}{k+1}+1 . \label{eq1}$. 
此外, $\frac{k+1}{2} \leqslant d^{+}\left(A_1\right) \leqslant 30-t$, 所以 $k \leqslant 59-2 t . \label{eq2}$. 由式\ref{eq1},\ref{eq2}消去 $k$ (利用(1)右边关于 $k$ 的函数的单调性), 有 $t \leqslant \frac{28(59-2 t)}{60-2 t}+1$, 即 $t^2- 59 t+856 \geqslant 0 . \label{eq3}$ . 
但 $t \leqslant 30$, 使 式\ref{eq3}成立的最大整数是 $t=25$. 即大龄学生不多于 25 . 最后, $t=25$ 是可能的.
实际上, 当 $t=25$ 时, 以上不等式成立等号, 代入 式\ref{eq2} , 解得 $k=9$.
将 $1,2, \cdots, 30$ 排成 6 行 (如图(<FilePath:./figures/fig-c11a4.png>)),
规定 $i 、 j$ 是一对朋友, 当且仅当 $i 、 j$ 满足下列 3 个条件之一:
(1) $i 、 j$ 在相邻的行中但不同列.
(2) $i 、 j$ 同列,但其中一个在最后一行.
(3) $i 、 j$ 都在第一行中.
此时, 每人有 9 个朋友.
比如 1 的 9 个朋友为 $2 、 3 、 4 、 5 、 7 、 8 、 9 、 10$ 、 26 . 故 $t$ 的最大值为 25 .
%%PROBLEM_END%%



%%PROBLEM_BEGIN%%
%%<PROBLEM>%%
问题5. 若干学生参加考试, 共 4 个选择题, 每题有 3 个选择支.
已知: 任何 3 个考生都有一个题, 他们的答案各不相同.
求考生人数的最大值.
%%<SOLUTION>%%
设有 $n$ 个考生.
易知, 每个考生的答卷都是一个长为 4 的序列, 此序列由 $A 、 B 、 C$ 三个字母组成.
从而本题等价于对 $n \times 4$ 方格棋盘的格 3 -染色,使 "任何三行都有某列的 3 个格两两异色". 设所有列中的异色格对的总数为 $S$. 一方面,对任何一列, 设该列中有 $a$ 个格为 $A$ 色, $b$ 个格为 $B$ 色, $c$ 个格为 $C$ 色, 则该列中的异色对有 $a b+b c+c a$ 个.
注意到 $3(a b+b c+c a) \leqslant a^2+b^2+c^2+2(a b+b c+c a)=(a+b+c)^2=n^2$, 所以, 4 列中至多有 $\frac{4 n^2}{3}$ 个异色对.
即 $S \leqslant \frac{4 n^2}{3}$. 另一方面, 任何 3 行都有一列包含 3 色, 此 3 色构成 3 个异色对, 于是有 $3 \mathrm{C}_n^3$ 个异色对.
但同一个异色对可能出现在 $n^{--}$个不同的 $3-$ 行组中, 这样, $S \geqslant 3 \cdot \frac{\mathrm{C}_n^3}{n-2}$. 所以, $3 \cdot \frac{\mathrm{C}_n^3}{n-2} \leqslant S \leqslant \frac{4 n^2}{3}$. 但此不等式恒成立, 不能求出 $n$ 的范围.
读者可以考虑, 这一估计是否可以改进? 从另一角度考虑"任何三行都有某列的 3 个格两两异色"的反面: 存在三行, 这三行的所有列的 3 个格都只有两色.
我们用反证法找每列只有两色的那些列.
通过尝试, 发现 $n<10$. 否则, 取其中任意 10 行得到 $10 \times 4$ 方格表 $M$. 考察 $M$ 的第一列格, 必有一种颜色至多出现 3 次.
于是, 至少有 7 行, 这 7 行的首列只有两色.
考察这 7 行的第二列, 必有一种颜色至多出现 2 次.
从而在上述 7 行中至少有 5 行, 这 5 行的首列、第二列都只有两色.
再考察这 5 行的第三列, 必有一颜色至多出现 1 次.
从而在上述 5 行中至少有 4 行, 这 4 行的首列、第二列、第三列都只有两色.
再考察这 4 行的最后一列, 必有一颜色至多出现 1 次, 从而在上述 4 行中至少有 3 行, 这 3 行的每一列都只有两色.
设这 3 行为 $A_1 、 A_2 、 A_3$, 则 $A_1$ 、 $A_2 、 A_3$ 的任何一列都有两个格同色,矛盾.
当 $n=9$ 时, 9 个人对每道题选择的答案代号分别为 $(1,2,1,2) 、(2,3,2,2) 、(3,1,3,2) 、(1,1,2,1)$ 、 $(2,2,3,1) 、(3,3,1,1) 、(1,3,3,3) 、(2,1,1,3) 、(3,2,2,3)$, 所以 $n=9$ 是可能的, 故 $n$ 的最大值为 9 .
%%PROBLEM_END%%



%%PROBLEM_BEGIN%%
%%<PROBLEM>%%
问题6. 给定 25 人, 其中每 5 人可以组成一个委员会, 且每两个委员会至多有一个公共成员.
求证: 委员会的个数不多于 30 . 
%%<SOLUTION>%%
将每两个委员至多有一个公共成员理解为 $\left|A_i \cap A_j\right| \leqslant 1$, 则应记 $A_i$ 为第 $i$ 个委员会的人的集合.
再令 $X$ 为给定的 25 个人的集合, $F=\left\{A_1\right.$, $\left.A_2, \cdots, A_k\right\}$. 我们来计算所有二人组的总数 $S$. 一方面, $\left|A_i\right|=5$, 从而每个 $A_i$ 中有 $\mathrm{C}_5^2=10$ 个二人组, 于是, $F$ 中的 $k$ 个集合可产生 $10 k$ 个二人组.
由于 $\left|A_i \cap B_j\right| \leqslant 1$, 这 $10 k$ 个二人组互异.
所以, $S \geqslant 10 k$. 另一方面, 设 $|X|=25$, 则 $X$ 中的二人组的总数为 $S=\mathrm{C}_{25}^2$. 所以, $\mathrm{C}_{25}^2=S \geqslant 10 k$, 所以, $k \leqslant 30$. 命题获证.
%%PROBLEM_END%%


