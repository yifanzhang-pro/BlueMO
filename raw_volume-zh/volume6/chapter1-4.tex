
%%TEXT_BEGIN%%
数列的通项与求和.
按一定次序排列的一列数称为数列, 其中的每一个数都叫这个数列的项, 依次叫做该数列的第一项 (或称为首项), 第二项, $\cdots$,第 $n$ 项, $\cdots$.
数列的一般形式可写为
$$
a_1, a_2, \cdots, a_n, \cdots .
$$
简记为 $\left\{a_n\right\}$. 如果 $\left\{a_n\right\}$ 的第 $n$ 项 $a_n$ 可以用 $n$ 的代数式表示, 那么这个公式就称为此数列的通项公式.
由数列的上述定义, 可知数列在本质上是定义在正整数集上的一个函数.
相关的问题中, 求数列的通项公式及前 $n$ 项之和的公式是最基本、最常见的.
用 $S_n$ 表示数列 $\left\{a_n\right\}$ 的前 $n$ 项之和, 那么它与通项之间有如下的关系
$$
a_1=S_1 ; a_n=S_n-S_{n-1}, n=2,3, \cdots .
$$
为表述与讨论方便,我们引人下面的一些概念:
如果一个数列的项数是有限的, 那么称它为有穷数列, 否则称它为无穷数列.
数列 $\left\{a_n\right\}$ 如果满足: 对任意 $n \in \mathbf{N}^*$, 都有 $a_n<a_{n+1}$ (或者 $a_n>a_{n+1}$ ), 那么称它为递增 (或者递减)数列; 如果只是 $a_n \leqslant a_{n+1}$ (或者 $a_n \geqslant a_{n+1}$ ), 那么称它为不减(或者不增)数列.
如果存在常数 $M$,使得对任意 $n \in \mathbf{N}^*$, 都有 $\left|a_n\right| \leqslant M$, 那么实数数列 $\left\{a_n\right\}$ 称为有界数列.
%%TEXT_END%%



%%PROBLEM_BEGIN%%
%%<PROBLEM>%%
例1. 数列 $\left\{a_n\right\}$ 满足: 对任意非负整数 $m 、 n(m \geqslant n)$, 都有 $a_{m+n}+a_{m \rightarrow n}= \frac{1}{2}\left(a_{2 m}+a_{2 n}\right)$,且 $a_1=1$. 求该数列的通项公式.
%%<SOLUTION>%%
解:利用题给的条件可知, 对任意 $m \in \mathbf{N}$, 都有
$$
\frac{1}{2}\left(a_{2 m}+a_{2 m}\right)=a_{2 m}+a_0=2\left(a_m+a_m\right)
$$
可知 $a_0=0$,且 $a_{2 m}=4 a_m$.
结合 $a_1=1$, 可发现当 $n \in\{0,1,2\}$ 时,都有 $a_n=n^2$, 这是否就是数列的通项公式呢?
假设 $a_{m-1}=(m-1)^2, a_m=m^2$, 那么由条件知
$$
a_{m+1}+a_{m-1}=\frac{1}{2}\left(a_{2 m}+a_2\right)=\frac{1}{2}\left(4 a_m+4 a_1\right),
$$
故 $a_{m+1}=2 a_m-a_{m-1}+2 a_1=2 m^2-(m-1)^2+2=m^2+2 m+1=(m+1)^2$. 这样,由数学归纳法原理, 可知对 $n \in \mathbf{N}$, 都有 $a_n=n^2$.
综上所述,所求数列的通项公式为 $a_n=n^2$.
说明利用已知条件求数列通项是与数列相关的问题中经常出现的, 本质上, 此题还是一个特殊的函数方程问题, 这与数列是定义在正整数集的函数有关.
%%PROBLEM_END%%



%%PROBLEM_BEGIN%%
%%<PROBLEM>%%
例2. 设 $n$ 是给定的正整数,数列 $a_0, a_1, a_2, \cdots, a_n$ 满足 $a_0=\frac{1}{2}, a_k= a_{k-1}+\frac{a_{k-1}^2}{n}, k=1,2, \cdots, n$. 证明: $1-\frac{1}{n}<a_n<1$.
%%<SOLUTION>%%
证明:由条件,可知对任意 $1 \leqslant k \leqslant n$, 都有 $a_{k-1}<a_k$, 故对任意 $0 \leqslant k \leqslant n$, 都有 $a_k>0$.
现在对条件式作变形
$$
\frac{1}{a_k}=\frac{n}{n a_{k-1}+a_{k-1}^2}=\frac{1}{a_{k-1}}-\frac{1}{a_{k-1}+n},
$$
将上式移项得
$$
\frac{1}{a_{k-1}+n}=\frac{1}{a_{k-1}}-\frac{1}{a_k} . \label{eq1}
$$
对式\ref{eq1}将下标 $k$ 从 1 到 $n$ 求和, 得
$$
\begin{aligned}
\sum_{k=1}^n \frac{1}{a_{k-1}+n} & =\left(\frac{1}{a_0}-\frac{1}{a_1}\right)+\left(\frac{1}{a_1}-\frac{1}{a_2}\right)+\cdots+\left(\frac{1}{a_{n-1}}-\frac{1}{a_n}\right) \\
& =\frac{1}{a_0}-\frac{1}{a_n}=2-\frac{1}{a_n} .
\end{aligned}
$$
结合 $a_{k-1}>0$, 可知
$$
2-\frac{1}{a_n}=\sum_{k=1}^n \frac{1}{a_{k-1}+n}<\sum_{k=1}^n \frac{1}{n}=1,
$$
于是 $a_n<1$.
再由 $a_k>a_{k-1}$ 知 $0<a_0<a_1<\cdots<a_n<1$, 故
$$
2-\frac{1}{a_n}=\sum_{k=1}^n \frac{1}{a_{k-1}+n}>\sum_{k=1}^n \frac{1}{1+n}=\frac{n}{n+1},
$$
得 $a_n>\frac{n+1}{n+2}=1-\frac{1}{n+2}>1-\frac{1}{n}$.
所以,命题成立.
说明这里对条件式"取倒数"是基于 "裂项"的思想, 在数列求和中经常会先"裂项",在求和时达到前后相消的效果.
%%PROBLEM_END%%



%%PROBLEM_BEGIN%%
%%<PROBLEM>%%
例3. 对 $n \in \mathbf{N}^*$, 设 $a_n=\frac{n}{(n-1)^{\frac{4}{3}}+n^{\frac{4}{3}}+(n+1)^{\frac{4}{3}}}$. 证明: $a_1+ a_2+\cdots+a_{999}<50$.
%%<SOLUTION>%%
证明:基本的想法是从局部往整体去处理, 为此, 对 $a_n$ 作恰当放大, 达到裂项相消的目的.
注意到 $x^3-y^3=(x-y)\left(x^2+x y+y^2\right)$, 令 $x=(n+1)^{\frac{2}{3}}, y=(n-1)^{\frac{2}{3}}$, 则 $x y=\left(n^2-1\right)^{\frac{2}{3}}<n^{\frac{4}{3}}$. 故
$$
\begin{aligned}
a_n & <\frac{n}{x^2+x y+y^2}=\frac{n(x-y)}{x^3-y^3} \\
& =\frac{n(x-y)}{(n+1)^2-(n-1)^2}=\frac{1}{4}(x-y) \\
& =\frac{1}{4}\left((n+1)^{\frac{2}{3}}-(n-1)^{\frac{2}{3}}\right) .
\end{aligned}
$$
所以
$$
\begin{aligned}
a_1+\cdots+a_{999} & <\frac{1}{4} \sum_{n=1}^{999}\left((n+1)^{\frac{2}{3}}-(n-1)^{\frac{2}{3}}\right) \\
& =\frac{1}{4}\left(\sum_{n=2}^{1000} n^{\frac{2}{3}}-\sum_{n=0}^{998} n^{\frac{2}{3}}\right) \\
& =\frac{1}{4}\left(1000^{\frac{2}{3}}+999^{\frac{2}{3}}-1\right) \\
& <\frac{1}{2} \times 1000^{\frac{2}{3}}=50 .
\end{aligned}
$$
命题获证.
说明 "先放缩再求和" 在处理与数列求和相关的不等式时是一个重要的方法.
%%PROBLEM_END%%



%%PROBLEM_BEGIN%%
%%<PROBLEM>%%
例4. 设 $k \in \mathbf{N}^*$,且 $k \equiv 3(\bmod 4)$. 定义
$$
\mathrm{S}_n=\mathrm{C}_n^0-\mathrm{C}_n^2 k+\mathrm{C}_n^4 k^2-\mathrm{C}_n^6 k^3+\cdots .
$$
%%<SOLUTION>%%
证明: 对任意 $n \in \mathbf{N}^*$, 都有 $2^{n-1} \mid S_n$.
证明利用复数来处理.
由 $S_n$ 的定义结合二项式定理, 可知
$$
S_n=\operatorname{Re}(1+\sqrt{k} \mathrm{i})^n,
$$
这里 $\mathrm{i}$ 为虚数单位, $\operatorname{Re}(z)$ 表示复数 $z$ 的实部.
于是
$$
S_n=\frac{1}{2}\left((1+\sqrt{k} i)^n+(1-\sqrt{k} i)^n\right),
$$
进而, 我们记 $x=1+\sqrt{k} \mathrm{i}, y=1-\sqrt{k} \mathrm{i}$, 则
$$
\begin{aligned}
S_{n+2} & =\frac{1}{2}\left(x^{n+2}+y^{n+2}\right)=\frac{1}{2}\left(\left(x^{n+1}+y^{n+1}\right)(x+y)-x y\left(x^n+y^n\right)\right) \\
& =(x+y) S_{n+1}-x y S_n \\
& =2 S_{n+1}-(1+k) S_n .
\end{aligned}
$$
并且 $S_1=1, S_2=1-k$.
下证: 对任意 $n \in \mathbf{N}^*$, 都有 $2^{n-1} \mid S_n$.
利用 $k \equiv 3(\bmod 4)$, 可知当 $n=1,2$ 时,命题成立; 现设命题对 $n, n+1$ 都成立, 即 $2^{n-1} \mid S_n$ 且 $2^n \mid S_{n+1}$, 则对 $n+2$ 的情形, 由 $1+k \equiv 1+3 \equiv 0 (\bmod 4)$, 及
$$
S_{n+2}=2 S_{n+1}-(1+k) S_n,
$$
可知 $2^{n+1} \mid S_{n+2}$ (因为 $2^{n+1}\left|2 S_{n+1}, 2^{n+1}\right|(1+k) S_n$ ). 所以, 对任意 $n \in \mathbf{N}^*$, 都有 $2^{n-1} \mid S_n$.
说明从条件出发,作出恰当转换, 建立数列的递推关系,再结合数学归纳法处理.
这样的解题思路有"思路清晰、一气呵成"之感, 把握起来也较方便.
%%PROBLEM_END%%



%%PROBLEM_BEGIN%%
%%<PROBLEM>%%
例5. 一个由正整数组成的递增数列 $\left\{a_n\right\}$ 的前面若干项为
$$
1 ; 2,4 ; 5,7,9 ; 10,12,14,16 ; 17, \cdots .
$$
其结构是: 1 个奇数, 2 个偶数, 3 个奇数, 4 个偶数, $\cdots$.
证明: 对任意 $n \in \mathbf{N}^*$, 都有 $a_n=2 n-\left[\frac{1+\sqrt{8 n-7}}{2}\right]$.
%%<SOLUTION>%%
证明:如果存在 $k \in \mathbf{N}^*$, 使得 $n=1+2+\cdots+k$, 那么称正整数 $n$ 是一个三角形数.
现在定义数列 $\left\{b_n\right\}: b_1=1$,
$$
b_{n+1}-b_n= \begin{cases}1, & \text { 若 } n \text { 是一个三角形数, } \\ 2, & \text { 若 } n \text { 不是一个三角形数.
}\end{cases} \label{eq1}
$$
则由数列 $\left\{a_n\right\}$ 的结构结合数学归纳法可知 $a_n=b_n$.
进一步, 由于满足式\ref{eq1}的数列 $\left\{b_n\right\}$ 是存在而且唯一的, 因此, 为证命题成立, 我们只需证明 (注意: 当 $n=1$ 时, $2 n-\left[\frac{1+\sqrt{8 n-7}}{2}\right]=1$ ):
$$
c_n= \begin{cases}1, & \text { 若 } n \text { 是一个三角形数, } \\ 2, & \text { 若 } n \text { 不是一个三角形数.
}\end{cases} \label{eq2}
$$
这里 $c_n=2(n+1)-\left[\frac{1+\sqrt{8(n+1)-7}}{2}\right]-\left(2 n-\left[\frac{1+\sqrt{8 n-7}}{2}\right]\right)$.
为此, 先证明:
当且仅当 $n$ 是一个三角形数时, $\frac{1+\sqrt{8(n+1)-7}}{2} \in \mathbf{N}^*, \label{eq3}$.
事实上, 若存在 $k \in \mathbf{N}^*$, 使得 $n=1+2+\cdots+k=\frac{k(k+1)}{2}$, 则
$$
\frac{1+\sqrt{8(n+1)-7}}{2}=\frac{1+\sqrt{4 k(k+1)+1}}{2}=\frac{1+2 k+1}{2}=k+1 \in \mathbf{N}^* \text {; 另 }
$$
一方面, 若 $n$ 不是三角形数, 则存在 $k \in \mathbf{N}^*$, 使得 $\frac{k(k+1)}{2}<n< \frac{(k+1)(k+2)}{2}$ (即 $n$ 介于两个相邻三角形数之间). 同上计算, 可知
$$
k+1<\frac{1+\sqrt{8(n+1)-7}}{2}<k+2 .
$$
所以,式\ref{eq3}成立.
回证式\ref{eq2}成立, 由于 $c_n=2+\left[\frac{1+\sqrt{8 n-7}}{2}\right]-\left[\frac{1+\sqrt{8(n+1)-7}}{2}\right]$, 而当 $n \in \mathbf{N}^*$ 时,有
$$
\begin{aligned}
0 & <\frac{1+\sqrt{8(n+1)-7}}{2}-\frac{1+\sqrt{8 n-7}}{2}=\frac{1}{2}(\sqrt{8 n+1}-\sqrt{8 n-7}) \\
& =\frac{1}{2} \cdot-\frac{8 n+1-(8 n-7)}{\sqrt{8 n+1}+\sqrt{8 n-7}}=\frac{4}{\sqrt{8 n+1}+\sqrt{8 n-7}} \leqslant \frac{4}{\sqrt{8+1}+\sqrt{8-7}} \\
& =1 .
\end{aligned}
$$
故当且仅当 $\frac{1+\sqrt{8(n+1)-7}}{2} \in \mathbf{N}^*$ 时, $c_n=2-1=1$; 而对其他的 $n$, 有 $c_n=2-0=2$.
利用结论式\ref{eq3}可知,\ref{eq2}成立.
综上可知 $a_n=2 n-\left[\frac{1+\sqrt{8 n-7}}{2}\right]$.
说明这是一个分群数列求通项的问题, 式\ref{eq1}反映的正是数列相邻两项的关系, 这里的方法是先有结果只需证明时采用的特殊手法, 如果要求自己去发现该数列的通项公式题目就困难了.
%%PROBLEM_END%%



%%PROBLEM_BEGIN%%
%%<PROBLEM>%%
例6. 我们称一个有穷数列 $a_0, a_1, \cdots, a_n$ 为 $k$ 平衡的, 如果
$$
a_0+a_k+a_{2 k}+\cdots=a_1+a_{k+1}+a_{2 k+1}+\cdots=\cdots=a_{k-1}+a_{2 k-1}+a_{3 k-1}+\cdots
$$
已知数列 $a_0, a_1, \cdots, a_{49}$ 对 $k=3,5,7,11,13,17$ 而言, 都是 $k$ 平衡的.
证明: $a_0=a_1=\cdots=a_{49}=0$.
%%<SOLUTION>%%
证明:考察多项式
$$
f(x)=a_0+a_1 x+\cdots+a_{49} x^{49} . \label{eq1}
$$
我们的思路是去证 $f(x)$ 有 50 个不同的复根, 从而导出 $f(x)$ 是一一个零多项式, 得到 $a_0=\cdots=a_{49}=0$.
对 $k \in\{3,5,7,11,13,17\}$, 设 $\varepsilon(\neq 1)$ 是一个 $k$ 次单位根, 则当 $m \equiv n(\bmod k)$ 时, 有 $\varepsilon^m=\varepsilon^n$, 于是
$$
\begin{aligned}
f(\varepsilon)= & \left(a_0+a_k+a_{2 k}+\cdots\right)+\left(a_1+a_{k+1}+\cdots\right) \varepsilon+\cdots \\
& +\left(a_{k-1}+a_{2 k-1}+\cdots\right) \varepsilon^{k-1} \\
= & \left(a_0+a_k+a_{2 k}+\cdots\right)\left(1+\varepsilon+\varepsilon^2+\cdots+\varepsilon^{k-1}\right) \\
= & 0
\end{aligned}
$$
这里已用到条件式及 $\varepsilon$ 是多项式 $1+x+\cdots+x^{k-1}$ 的根.
因此, 对 $k \in\{3,5,7,11,13,17\}$ 及 $\varepsilon=\mathrm{e}^{\mathrm{i} \frac{2 m_x}{k}}(1 \leqslant m \leqslant k-1)$, 可知 $\varepsilon$ 都是式\ref{eq1}的复根, 而 $k$ 取不同的素数,故所得复根互不相同.
这表明, $f(x)$ 有 $(3-1)+(5-1)+\cdots+(17-1)=50$ 个不同复根, 它只能是零多项式.
命题获证.
说明 式\ref{eq1}所列出的多项式称为数列 $\left\{a_m\right\}$ 的母函数, 它在求数列通项中经常被用到,在第 7 讲中还会提到.
%%PROBLEM_END%%


