
%%TEXT_BEGIN%%
不定方程, 是指未知数的个数多于方程的个数, 而未知数的取值范围受某些限制 (如整数、正整数、有理数等) 的方程.
不定方程是数论的一个重要课题,数学竞赛中也常涉及这方面的问题.
初等范围内, 处理不定方程主要有三种方法: 分解方法, 同余方法, 以及 (不等式)估计方法.
分解方法则是最为基本的方法.
分解方法的主要功效, 大致地说, 是通过 "分解" 将原方程分解为若干个易于处理的方程.
这里说的"分解"包含两个方面的手法: 其一, 是代数 (整式) 的分解; 其二, 是应用整数的某些性质 (唯一分解定理, 互素的性质等) 导出适用的分解.
分解方法当然没有固定的程序可循.
有时, 分解相当困难, 或分解方式较多而难以选择; 有时, 进一步的论证则很不容易.
本节的一些例子就已表现了这些.
分解方法常和别的方法结合使用, 请参考本单元及后面的一些例子.
%%TEXT_END%%



%%PROBLEM_BEGIN%%
%%<PROBLEM>%%
例1. 一个正整数, 加上 100 , 为一完全平方数, 若加上 168 , 则为另一个完全平方数, 求此数.
%%<SOLUTION>%%
解:设所求的数为 $x$, 由题意, 有正整数 $y 、 z$, 使得
$$
\left\{\begin{array}{l}
x+100=y^2, \\
x+168=z^2 .
\end{array}\right.
$$
从上面两个方程中消去 $x$, 得出
$$
z^2-y^2=68 .
$$
将这个二元二次方程的左边分解因式, 而将右边作标准分解, 得
$$
(z-y)(z+y)=2^2 \times 17 . \label{eq1}
$$
由于 $z-y$ 及 $z+y$ 都是正整数, 且 $z-y<z+y$, 故由式\ref{eq1}及唯一分解定理推出, 必有
$$
\left\{\begin{array} { l } 
{ z - y = 1 , } \\
{ z + y = 2 ^ { 2 } \times 1 7 ; }
\end{array} \left\{\begin{array} { l } 
{ z - y = 2 , } \\
{ z + y = 2 \times 1 7 ; }
\end{array} \left\{\begin{array}{l}
z-y=2^2 \\
z+y=17
\end{array}\right.\right.\right. 
$$
逐一解这些二元一次方程组, 可得出 $y=16, z=18$, 故 $x=156$.
%%PROBLEM_END%%



%%PROBLEM_BEGIN%%
%%<PROBLEM>%%
例2. 求不定方程:
$$
x^4+y^4+z^4=2 x^2 y^2+2 y^2 z^2+2 z^2 x^2+24
$$
的全部整数解.
%%<SOLUTION>%%
解:关键的一步 (也是本题的主要困难) 是看出方程可分解为
$$
(x+y+z)(x+y-z)(y+z-x)(z+x-y)=-2^3 \times 3 . \label{eq1}
$$
因上式左边四个因数都是整数, 由唯一分解定理,可类似于例 1 那样, 将 式\ref{eq1} 分解为若干个 (四元一次) 方程组来求解.
这虽然也能够解决问题, 但却较为麻烦.
我们(基于式\ref{eq1})采用下面的处理:因素数 2 整除式\ref{eq1}的右边,故式\ref{eq1}的左边四个因数中至少有一个被 2 整除.
另一方面, 这四个数中任意两个的和显然是偶数,故它们的奇偶性相同, 从而现在都是偶数,即 式\ref{eq1}的左边被 $2^4$ 整除, 但式\ref{eq1}的右边不是 $2^4$ 的倍数, 因此方程无整数解.
%%PROBLEM_END%%



%%PROBLEM_BEGIN%%
%%<PROBLEM>%%
例3. 证明: 两个连续正整数之积不能是完全平方,也不能是完全立方.
%%<SOLUTION>%%
证明:反证法,我们假设有正整数 $x, y$,使得
$$
x(x+1)=y^2 .
$$
将方程两边乘以 4 ,变形为 $(2 x+1)^2=4 y^2+1$, 这可分解为
$$
(2 x+1+2 y)(2 x+1-2 y)==1 .
$$
因左边两个因数都是正整数,故有
$$
\left\{\begin{array}{l}
2 x+1+2 y=1 \\
2 x+1-2 y=1
\end{array}\right.
$$
解得 $x=y=0$, 矛盾.
这就证明了问题中的第一个断言.
然而, 对于方程
$$
x(x+1)=y^3,
$$
上面的分解方法不易奏效.
我们采用另一种 (基于数的性质的) 分解: 设所说的方程有正整数解 $x 、 y$,则由于 $x$ 和 $x+1$ 互素, 而它们的积是一个完全立方,故 $x$ 和 $x+1$ 都是正整数的立方, 即
$$
x=u^3, x+1=v^3, y=u v,
$$
$u, v$ 都是正整数,由此产生 $v^3-u^3=1$, 故
$$
(v-u)\left(v^2+u v+u^2\right)=1,
$$
这显然不可能.
不难看到, 用类似的论证, 可证明连续两个正整数之积不会是整数的 $k$ 次幂 (这里 $k \geqslant 2$ ).
判明一个乘积中的各个因数互素往往非常重要,下面的例 4, 例 5 均是如此.
%%PROBLEM_END%%



%%PROBLEM_BEGIN%%
%%<PROBLEM>%%
例4. 证明: 方程
$$
y+y^2=x+x^2+x^3
$$
没有 $x \neq 0$ 的整数解.
%%<SOLUTION>%%
证明:设方程有 $x \neq 0$ 的整数解, 将它分解为
$$
(y-x)(y+x+1)=x^3 . \label{eq1}
$$
我们先证明 $(y-x, y+x+1)=1$. 若这不正确, 则有一个素数 $p$ 为 $y- x$ 与 $y+x+1$ 的一个公约数.
由式\ref{eq1}知 $p \mid x^3$, 故素数 $p$ 整除 $x$, 结合 $p \mid(y-x)$ 知 $p \mid y$, 但 $p \mid(x+y+1)$, 从而 $p \mid 1$, 这不可能, 故式\ref{eq1}的左边两因数互素.
因式\ref{eq1}的右边是一个完全立方, 从而有整数 $a 、 b$, 使得
$$
y-x=a^3, y+x+1=b^3, x=a b .
$$
消去 $x, y$ 得到
$$
b^3-a^3=2 a b+1 . \label{eq2}
$$
现在证明方程式\ref{eq2}无整数解, 由此便导出了矛盾.
我们将式\ref{eq2}分解为
$$
(b-a)\left(b^2+a b+a^2\right)=2 a b+1 . \label{eq3}
$$
注意 $x=a b$ 而 $x \neq 0$, 故 $a b \neq 0$. 若 $a b>0$, 则由式\ref{eq3}易知 $b-a>0$, 因 $a 、 b$ 为整数,故 $b-a \geqslant 1$, 于是(3)的左边 $\geqslant b^2+a b+a^2>3 a b>$ 右边; 若 $a b<0$, 则 $|b-a| \geqslant 2$, 故式\ref{eq3}的左边的绝对值 $\geqslant 2\left(a^2+b^2-|a b|\right)>2|a b|$, 而式\ref{eq3}的右边的绝对值 $<2|a b|$, 因此式\ref{eq3}不能成立, 这就证明了问题中的方程没有 $x \neq 0$ 的整数解.
方程式\ref{eq3}无解的论证, 采用了不等式估计 (左边的绝对值总大于右边的绝对值), 这就是所谓的估计法.
(数论中的) 估计法往往需着眼于整数, 利用整数的各种性质产生适用的不等式.
例如, 上述论证应用了整数的最基本的性质: 若整数 $x>0$, 则 $x \geqslant 1$.
%%PROBLEM_END%%



%%PROBLEM_BEGIN%%
%%<PROBLEM>%%
例5. 设 $k$ 是给定的正整数, $k \geqslant 2$, 证明: 连续三个正整数的积不能是整数的 $k$ 次幂.
%%<SOLUTION>%%
证明:假设有正整数 $x \geqslant 2$ 及 $y$, 使得
$$
(x-1) x(x+1)=y^k . \label{eq1}
$$
请注意上面左端的三个因数 $x-1 、 x 、 x+1$ 并非总两两互素, 因此不能由式\ref{eq1}推出它们都是 $k$ 次方幕.
克服这个困难的一种方法是将式\ref{eq1}变形为
$$
\left(x^2-1\right) x=y^k . \label{eq2}
$$
因 $x$ 和 $x^2-1$ 互素,故由式\ref{eq2}推出,有正整数 $a 、 b$,使得
$$
x=a^k, x^2-1=b^k, a b=y,
$$
由此我们有
$$
\begin{aligned}
1 & =a^{2 k}-b^k=\left(a^2\right)^k-b^k \\
& =\left(a^2-b\right)\left(a^{2 k-2}+a^{2 k-4} b+\cdots+a^2 b^{k-2}+b^{k-1}\right),
\end{aligned}
$$
由于 $x \geqslant 2$, 故 $a \geqslant 2$, 又 $k \geqslant 2$, 故上式后一个因数必大于 1 , 导出矛盾.
%%PROBLEM_END%%



%%PROBLEM_BEGIN%%
%%<PROBLEM>%%
例6. 求 $\left(x^2-y^2\right)^2=1+16 y$ 的全部整数解.
%%<SOLUTION>%%
解:因方程左边 $\geqslant 0$, 故右边 $\geqslant 0$, 从而 $y \geqslant 0$. 又显然 $x^2-y^2 \neq 0$, 而 $x_1 y$ 为整数, 故 $|x| \geqslant y+1$, 或 $|x| \leqslant y-1$.
当 $|x| \geqslant y+1$ 时,方程左边 $\geqslant\left((y+1)^2-y^2\right)^2=(2 y+1)^2$.
当 $|x| \leqslant y-1$ 时, 此时 $y-1 \geqslant 0$, 且 $y^2-x^2 \geqslant y^2-(y-1)^2=2 y- 1>0$, 故方程左边 $\geqslant(2 y-1)^2$.
因此由原方程产生
$$
(2 y-1)^2 \leqslant 1+16 y,
$$
故有 $0 \leqslant y \leqslant 5$. 逐一检验可求出全部整数解为 $\left(x_1 y\right)=( \pm 1,0),( \pm 4,3)$, $( \pm 4,5)$.
%%PROBLEM_END%%



%%PROBLEM_BEGIN%%
%%<PROBLEM>%%
例7. 设正整数 $x 、 y 、 z$ 满足 $2 x^x=y^y+z^z$, 则 $x=y=z$.
%%<SOLUTION>%%
证明:首先, 将 $(x+1)^{x+1}$ 展开即知
$$
(x+1)^{x+1}>x^{x+1}+(x+1) x^x>2 x^x, \label{eq1}
$$
由此可知 $y 、 z$ 必须均 $\leqslant x$ : 因若 $y 、 z$ 中有大于 $x$ 的, 无妨设 $y>x$, 因 $y 、 x$ 为整数,故 $y \geqslant x+1$, 从而
$$
y^y+z^z>y^y \geqslant(x+1)^y \geqslant(x+1)^{x+1}>2 x^x \text { (应用 式\ref{eq1}), }
$$
产生矛盾.
因此 $y \leqslant x, z \leqslant x$, 故
$$
y^y+z^z \leqslant x^x+x^x=2 x^x,
$$
结合原方程知, 必须有 $y=x$, 且 $z=x$, 故 $x=y=z$. 证毕.
%%PROBLEM_END%%


