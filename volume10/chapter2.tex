
%%TEXT_BEGIN%%
最大公约数与最小公倍数.
最大公约数是数论中的一个重要概念.
设 $a 、 b$ 不全为零, 同时整除 $a 、 b$ 的整数 (如 \pm 1 ) 称为它们的公约数.
因 $a 、 b$ 不全为零, 故由第 1 单元中性质 (3) 推知, $a 、 b$ 的公约数只有有限多个, 我们将其中最大的一个称为 $a 、 b$ 的最大公约数, 用符号 $(a, b)$ 表示.
显然, 最大公约数是一个正整数.
当 $(a, b)=1$ 时 (即 $a, b$ 的公约数只有 \pm 1 ), 我们称 $a$ 与 $b$ 互素 (互质). 读者在后面将看到,这种情形特别重要.
对于多于两个的 (不全为零的) 整数 $a, b, \cdots, c$, 可类似地定义它们的最大公约数 $(a, b, \cdots, c)$. 若 $(a, b, \cdots, c)=1$, 则称 $a, b, \cdots, c$ 互素.
请注意, 此时并不能推出 $a, b, \cdots, c$ 两两互素 (即其中任意两个都互素); 但反过来, 若 $a, b, \cdots, c$ 两两互素,则显然有 $(a, b, \cdots, c)=1$.
由定义不难得出最大公约数的一些简单性质:
任意改变 $a 、 b$ 的符号不改变 $(a, b)$ 的值, 即有 $( \pm a, \pm b)=(a, b)$;
$(a, b)$ 关于 $a 、 b$ 对称,即有 $(a, b)=(b, a)$;
$(a, b)$ 作为 $b$ 的函数, 以 $a$ 为周期, 即对任意整数 $x$, 有 $(a, b+a x)=(a, b)$.
下面 (1) 中的结论, 是建立最大公约数的性质的基础, 通常称为裴蜀等式.
(1) 设 $a 、 b$ 是不全为 0 的整数,则存在整数 $x 、 y$,使得
$$
a x+b y=(a, b) \text {. }
$$
顺便提及,若 $x=x_0, y=y_0$ 是满足上式的一对整数,则等式
$$
a\left(x_0+b u\right)+b\left(y_0-a u\right)=(a, b)(u \text { 为任意整数 })
$$
表明,满足上式的 $x 、 y$ 有无穷多组; 并且,在 $a b>0$ 时,可选择 $x$ 为正 (负) 数, 此时 $y$ 则相应地为负 (正) 数.
由 (1)易于推出下面的
(2) 两个整数 $a 、 b$ 互素的充分必要条件是存在整数 $x 、 y$,使得
$$
a x+b y=1 \text {. }
$$
事实上, 条件的必要性是 (1) 的特例.
反过来, 若有 $x 、 y$ 使等式成立, 设 $(a, b)=d$, 则 $d \mid a$ 且 $d \mid b$, 故 $d \mid a x$ 及 $d \mid b y$,于是 $d \mid(a x+b y)$, 即 $d \mid 1$, 从而 $d=1$.
由(1)及 (2)不难导出下面的几个基本结论:
(3) 若 $m|a, m| b$, 则 $m \mid(a, b)$, 即 $a 、 b$ 的任一个公约数都是它们的最大公约数的约数.
(4) 若 $m>0$, 则 $(m a, m b)=m(a, b)$.
(5) 若 $(a, b)=d$, 则 $\left(\frac{a}{d}, \frac{b}{d}\right)=1$. 因此, 由两个不互素的整数,可自然地产生一对互素的整数.
(6) 若 $(a, m)=1,(b, m)=1$, 则 $(a b, m)=1$. 这表明,与一个固定整数互素的整数之集关于乘法封闭.
由此可推出: 若 $(a, b)=1$, 则对任意 $k>0$ 有 $\left(a^k, b\right)=1$, 进而对任意 $l>0$ 有 $\left(a^k, b^l\right)=1$.
(7) 设 $b \mid a c$. 若 $(b, c)=1$, 则 $b \mid a$.
(8) 设正整数 $a 、 b$ 之积是一个整数的 $k$ 次幂 $(k \geqslant 2)$. 若 $(a, b)=1$, 则 $a 、 b$ 都是整数的 $k$ 次幂.
一般地, 设正整数 $a, b, \cdots, c$ 之积是一个整数的 $k$ 次幕.
若 $a, b, \cdots, c$ 两两互素, 则 $a, b, \cdots, c$ 都是整数的 $k$ 次幂.
(6)、(7)、(8)表现了互素的重要性, 它们的应用也最为广泛.
现在, 我们简单地谈谈最小公倍数.
设 $a 、 b$ 是两个非零整数,一个同时为 $a 、 b$ 倍数的数称为它们的一个公倍数.
$a 、 b$ 的公倍数显然有无穷多个, 这其中最小的正数称为 $a 、 b$ 的最小公倍数,记作 $[a, b]$. 对于多个非零整数 $a, b, \cdots, c$, 可类似地定义它们的最小公倍数 $[a, b, \cdots, c]$.
下面是最小公倍数的主要性质.
(9) $a$ 与 $b$ 的任一公倍数都是 $[a, b]$ 的倍数.
对于多于两个整数的情形, 类似的结论也成立.
(10) 两个整数 $a 、 b$ 的最大公约数与最小公倍数满足
$$
(a, b)[a, b]=|a b| .
$$
但请注意, 对于多于两个整数的情形, 类似的结论不成立 (请读者举出例子). 然而我们有下面的
(11) 若 $a, b, \cdots, c$ 两两互素,则有
$$
[a, b, \cdots, c]=|a b \cdots c| .
$$
由此及(9) 可知, 若 $a|d, b| d, \cdots, c \mid d$, 且 $a, b, \cdots, c$ 两两互素, 则有 $a b{ }^{\prime} \cdot c \mid d$.
互素,在数论中相当重要, 往往是许多问题的关键或基础.
数学竞赛中, 有一些问题要求证明两个整数互素 (或求它们的最大公约数), 下面几个例子表现了处理这些问题的一个基本方法.
%%TEXT_END%%



%%PROBLEM_BEGIN%%
%%<PROBLEM>%%
例1. 对任意整数 $n$, 证明分数 $\frac{21 n+4}{14 n+3}$ 是既约分数.
%%<SOLUTION>%%
证明:问题即要证明 $21 n+4$ 与 $14 n+3$ 互素.
易知这两数适合裴蜀等式
$$
3(14 n+3)-2(21 n+4)=1,
$$
因此所说的结论成立.
一般来说,互素整数 $a 、 b$ 适合的裴蜀等式不易导出, 因此我们常采用下述的变通手法: 制造一个与裴蜀等式类似的辅助等式
$$
a x+b y=r,
$$
其中 $r$ 是一个适当的整数.
若设 $(a, b)=d$, 则由上式知 $d \mid r$. 所谓适当的 $r$ 是指: 由 $d \mid r$ 能够通过进一步的论证导出 $d=1$, 或者 $r$ 的约数较少, 可以由排除法证得结论.
此外, 上述辅助等式等价于 $a \mid(b y-r)$ 或 $b \mid(a x-r)$, 有时, 这些由整除更容易导出来.
%%PROBLEM_END%%



%%PROBLEM_BEGIN%%
%%<PROBLEM>%%
例2. 设 $n$ 是正整数,证明 $(n !+1,(n+1) !+1)=1$.
%%<SOLUTION>%%
证明:我们有等式
$$
(n !+1)(n+1)-((n+1) !+1)=n . \label{eq1}
$$
设 $d=(n !+1,(n+1) !+1)$, 则由 式\ref{eq1} 知 $d \mid n$.
进一步,因 $d \mid n$ 故 $d \mid n !$, 结合 $d \mid(n !+1)$ 可知 $d \mid 1$, 故 $d=1$.
%%PROBLEM_END%%



%%PROBLEM_BEGIN%%
%%<PROBLEM>%%
例3. 记 $F_k=2^{2^k}+1, k \geqslant 0$. 证明: 若 $m \neq n$, 则 $\left(F_m, F_n\right)=1$.
%%<SOLUTION>%%
证明:不妨设 $m>n$. 论证的关键是利用 $F_n \mid\left(F_m-2\right)$ (见第 1 单元例 $2)$, 即有一个整数 $x$, 使得
$$
F_m+x F_n=2 .
$$
设 $d=\left(F_m, F_n\right)$, 则由上式推出 $d \mid 2$, 所以 $d=1$ 或 2 . 但 $F_n$ 显然是奇数,故必须 $d=1$.
%%<REMARK>%%
注:$F_k(k \geqslant 0)$ 称为费马 (Fermat) 数.
例 3 表明,费马数两两互素, 这是费马数的一个有趣的基本性质.
%%PROBLEM_END%%



%%PROBLEM_BEGIN%%
%%<PROBLEM>%%
例4. 设 $a>1, m, n>0$, 证明:
$$
\left(a^m-1, a^n-1\right)=a^{(m, n)}-1 .
$$
%%<SOLUTION>%%
证明:设 $D=\left(a^m-1, a^n-1\right)$. 我们通过证明 $\left(a^{(m, n)}-1\right) \mid D$ 及 $D \mid \left(a^{(m, n)}-1\right)$ 来导出 $D==a^{(m, n)}-1$, 这是数论中证明两数相等的常用手法.
因为 $(m, n)|m,(m, n)| n$, 由第 1 单元中分解公式 (5) 即知 $\left(a^{(m, n)}-1\right) \mid \left(a^m-1\right)$, 以及 $\left(a^{(m, n)}-1\right) \mid\left(a^n-1\right)$. 故由本单元的性质 (3) 可知, $a^{(m, n)}-1$ 整除 $\left(a^m-1, a^n-1\right)$, 即 $\left(a^{(m, n)}-1\right) \mid D$.
为了证明 $D \mid\left(a^{(m, n)}-1\right)$, 我们设 $d=(m, n)$. 因 $m, n>0$, 故可选择 $u$, $v>0$, 使得 (参见本单元 (1) 中的注释)
$$
m u-n v=d . \label{eq1}
$$
因为 $D \mid\left(a^m-1\right)$, 故更有 $D \mid\left(a^{m u}-1\right)$. 同样, $D \mid\left(a^{n u}-1\right)$. 故 $D \mid\left(a^{m u}-\right. \left.a^{n o}\right)$, 从而由 式\ref{eq1}, 得
$$
D \mid a^{n v}\left(a^d-1\right) . \label{eq2}
$$
此外, 因 $a>1$, 及 $D \mid\left(a^m-1\right)$, 故 $(D, a)=1$, 进而 $\left(D, a^{n v}\right)=1$. 于是, 从 式\ref{eq2} 及性质 (7) 导出 $D \mid\left(a^d-1\right)$, 即 $D \mid\left(a^{(m, n)}-1\right)$.
综合已证得的两方面的结果, 可知 $D=a^{(m, n)}-1$.
%%PROBLEM_END%%



%%PROBLEM_BEGIN%%
%%<PROBLEM>%%
例5. 设 $m, n>0, m n \mid\left(m^2+n^2\right)$, 则 $m=n$.
%%<SOLUTION>%%
证明:设 $(m, n)=d$, 则 $m=m_1 d, n=n_1 d$, 其中 $\left(m_1, n_1\right)=1$.
于是, 已知条件化为 $m_1 n_1 \mid\left(m_1^2+n_1^2\right)$, 故更有 $m_1 \mid\left(m_1^2+n_1^2\right)$, 从而 $m_1 \mid n_1^2$.
但 $\left(m_1, n_1\right)=1$, 故 $\left(m_1, n_1^2\right)=1$. 结合 $m_1 \mid n_1^2$, 可知必须 $m_1=1$. 同理 $n_1=$ 1 , 因此 $m=n$.
%%<REMARK>%%
注1 对两个给定的不全为零的整数, 我们常借助它们的最大公约数, 并应用性质(5), 产生两个互素的整数, 以利用互素的性质作进一步论证 (参见性质 (6)、(7)). 就本题而言, 由于 $m n$ 为二次式, $m^2+n^2$ 为二次齐次式, 上述手续的功效,实质上是将问题化归成 $m 、 n$ 互素这种特殊情形.
注2 在某些问题中,已知的条件 (或已证得的结论) $c \mid a$ 并不适用,我们可试着选取 $c$ 的一个适当的约数 $b$, 并从 $c \mid a$ 过渡到 (较弱的结论) $b \mid a$, 以期望后者提供适宜于进一步论证的信息.
本题中,我们便是由 $m_1 n_1 \mid\left(m_1^2+n_1^2\right)$ 产生了 $m_1 \mid n_1^2$, 进而导出 $m_1=1$.
%%PROBLEM_END%%



%%PROBLEM_BEGIN%%
%%<PROBLEM>%%
例6. 设正整数 $a 、 b 、 c$ 的最大公约数为 1 , 并且
$$
\frac{a b}{a-b}=c \text {. }
$$
证明: $a-b$ 是一个完全平方数.
%%<SOLUTION>%%
证明:设 $(a, b)=d$, 则 $a=d a_1, b=d b_1$, 其中 $\left(a_1, b_1\right)=1$. 由于 $(a$, $b, c)=1$, 故有 $(d, c)=1$.
现在, 问题中的等式可化为
$$
d a_1 b_1=c a_1-c b_1, \label{eq1}
$$
由此可见 $a_1$ 整除 $c b_1$. 因 $\left(a_1, b_1\right)=1$, 故 $a_1 \mid c$. 同样得 $b_1 \mid c$. 再由 $\left(a_1, b_1\right)=$ 1 便推出 $a_1 b_1 \mid c$.
设 $c=a_1 b_1 k$, 其中 $k$ 是-一个正整数.
一方面,显然 $k$ 整除 $c$; 另一方面,结合 \ref{eq1} 式得 $d=k\left(a_1-b_1\right)$, 故 $k \mid d$. 从而 $k \mid(c, d)$ . 但 $(c, d)=1$, 故 $k=1$.
因此 $d=a_1-b_1$. 故 $a-b=d\left(a_1-b_1\right)=d^2$. 这就证明了 $a-b$ 是一个完全平方数.
%%PROBLEM_END%%



%%PROBLEM_BEGIN%%
%%<PROBLEM>%%
例7. 设 $k$ 为正奇数,证明: $1+2+\cdots+n$ 整除 $1^k+2^k+\cdots+n^k$.
%%<SOLUTION>%%
证明:因为 $1+2+\cdots+n=\frac{n(n+1)}{2}$, 故问题等价于证明: $n(n+1)$ 整除 $2\left(1^k+2^k+\cdots+n^k\right)$. 因 $n$ 与 $n+1$ 互素, 所以这又等价于证明
$$
n \mid 2\left(1^k+2^k+\cdots+n^k\right)
$$
及
$$
(n+1) \mid 2\left(1^k+2^k+\cdots+n^k\right) .
$$
事实上,由于 $k$ 为奇数,故由第 1 单元中分解公式 (6), 可知
$$
\begin{aligned}
& 2\left(1^k+2^k+\cdots+n^k\right) \\
= & {\left[1^k+(n-1)^k\right]+\left[2^k+(n-2)^k\right]+\cdots+\left[(n-1)^k+1^k\right]+2 n^k }
\end{aligned}
$$
是 $n$ 的倍数.
同理,
$$
2\left(1^k+2^k+\cdots+n^k\right)=\left[1^k+n^k\right]+\left[2^k+(n-1)^k\right]+\cdots+\left[n^k+1^k\right]
$$
是 $n+1$ 的倍数.
%%<REMARK>%%
注:整除问题中, 有时直接证明 $b \mid a$ 不易人手.
若 $b$ 可分解为 $b=b_1 b_2$, 其中 $\left(b_1, b_2\right)=1$, 则我们可将原命题 $b \mid a$ 分解为等价的两个命题 $b_1 \mid a$ 及 $b_2 \mid a$, 后者可能更容易导出来.
例 7 应用了这一基本手法, 例 6 中证明 $a_1 b_1 \mid c$ 也是这样做的.
更一般地, 为了证明 $b \mid a$, 可将 $b$ 分解为若干个两两互素的整数 $b_1$, $b_2, \cdots, b_n$ 之积, 而证明等价的 $b_i \mid a(i=1,2, \cdots, n)$ (参见性质 (11), 并可比较第 1 单元例 3 的注 1 中说的想法).
%%PROBLEM_END%%


