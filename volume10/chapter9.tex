
%%TEXT_BEGIN%%
同余, 是处理不定方程的一个有力工具, 我们常应用同余证明不定方程无整数解, 或导出解应满足的某种必要条件.
这样的论证, 往往灵活多变, 在数学竞赛中尤为多见, 本节将膨取一些例子来表现同余在这方面的应用.
%%TEXT_END%%



%%PROBLEM_BEGIN%%
%%<PROBLEM>%%
例1. 若 $n \equiv 4(\bmod 9)$, 证明不定方程
$$
x^3+y^3+z^3=n . \label{eq1}
$$
没有整数解 $(x, y, z)$.
%%<SOLUTION>%%
证明:若方程式\ref{eq1}有整数解, 则式\ref{eq1}模 9 也有整数解.
熟知, 一完全立方模 9 同余于 $0,1,-1$ 之一,因而
$$
x^3+y^3+z^3 \equiv 0,1,2,3,6,7,8(\bmod 9) .
$$
但 $n \equiv 4(\bmod 9)$, 所以 式\ref{eq1}模 9 无解, 这与前面所说的相违, 故方程 式\ref{eq1}无整数解.
用同余处理不定方程, 核心在于选择适当的模.
%%PROBLEM_END%%



%%PROBLEM_BEGIN%%
%%<PROBLEM>%%
例2. 确定方程
$$
x_1^4+x_2^4+\cdots+x_{14}^4=1599
$$
的全部非负整数解 $\left(x_1, \cdots, x_{14}\right)$ (不计解的排列次序).
%%<SOLUTION>%%
解:模 16 就能够证明方程无整数解, 因为整数的四次幂模 16 同余于 0 或 1 , 故 $x_1^4+x_2^4+\cdots+x_{14}^4$ 模 16 的所有可能值是 $0,1,2, \cdots, 14$, 唯独不能取 15 . 但 $1599 \equiv 15(\bmod 16)$, 因此方程无解,证毕.
之所以选择 16 , 是因为方程左边有 14 项, 剩余类的个数 $\geqslant 15$ 才比较有希望导出矛盾 (这里我们采用同余来证明方程无整数解). 而 $15=3 \times 5$, 根据中国剩余定理,模 15 相当于模 3 与模 5 的作用,不能解决问题.
%%PROBLEM_END%%



%%PROBLEM_BEGIN%%
%%<PROBLEM>%%
例3. 证明: 下列数不能表示为若干个连续整数的立方和.
(1) $385^{97}$;
(2) $366^{17}$.
%%<SOLUTION>%%
证明:利用
$$
1^3+2^3+\cdots+k^3=\left(\frac{k(k+1)}{2}\right)^2
$$
易知, 连续若干个整数的立方和可表示为形式
$$
\left(\frac{m(m+1)}{2}\right)^2-\left(\frac{n(n+1)}{2}\right)^2, \label{eq1}
$$
$m, n$ 为整数.
我们要证明,对于 (1)、(2) 中的整数, 不存在 $m, n$, 使之可表示为式\ref{eq1}的形式.
用分解方法虽也能解决问题, 但相当麻烦; 用同余论证, 则相当直接.
首先, 按整数 $x$ 模 9 分类并逐一检验, 不难得知, $\left(\frac{x(x+1)}{2}\right)^2$ 模 9 同余于 0 及 -1 , 因此,形如式\ref{eq1}的数模 9 只能是 $0,1,-1$. 另一方面, 由欧拉定理知
$$
385^{97} \equiv 385 \times\left(385^{16}\right)^6 \equiv 385 \equiv 7(\bmod 9),
$$
这就证明了 $385^{97}$ 不能表示为式\ref{eq1}的形式.
然而, 因 $366^{17} \equiv 0(\bmod 9)$, 故对于数 $366^{17}$, 模 9 不能解决问题.
我们这次模 7. 易于验证, 对整数 $x$, 数 $\left(\frac{x(x+1)}{2}\right)^2$ 模 7 同余于 0,1 , -1 . 故形如(1)的数模 7 只能是 $0, \pm 1, \pm 2$. 但
$$
366^{17} \equiv 2^{17} \equiv 2 \times 2^4 \equiv 4(\bmod 7) .
$$
因此我们的断言成立.
有整数解的方程, 仅用同余通常不易解决问题, 而需将同余与其他方法 (估计、分解等)结合使用.
我们举几个这样的例子.
%%PROBLEM_END%%



%%PROBLEM_BEGIN%%
%%<PROBLEM>%%
例4. 求所有这样的 2 的幕, 将其(十进制表示中的)首位删去后, 剩下的数仍是一个 2 的幂.
%%<SOLUTION>%%
解:问题即要求出方程
$$
2^n=2^k+a \times 10^m . \label{eq1}
$$
的全部正整数解 $(n, k, m, a)$, 其中 $a=1,2, \cdots, 9$. 将式\ref{eq1}变形为
$$
2^k\left(2^{n-k}-1\right)=a \times 10^m . \label{eq2}
$$
首先证明 $m=1$. 因为若 $m>1$, 则 \ref{eq2} 式右边被 $5^2$ 整除, 从而 $5^2 \mid\left(2^{n-k}-1\right)$. 又易知, 2 模 $5^2$ 的阶是 20 (这只需注意所说的阶整除 $\varphi(25)=20$, 及 $2^{10} \equiv-1(\bmod 25))$, 因此,20 整除 $n-k$, 从而 $2^{20}-1$ 整除 式\ref{eq2} 的左边, 但 $2^{20}-1= \left(2^5\right)^4-1$ 有因子 $2^5-1=31$, 而 31 不整除 式\ref{eq1} 的右边,矛盾!因此 $m=1$.
现在只需在为二位数的 2 的幕中, 检验符合要求的解, 易知这只有 32 和 64 .
%%PROBLEM_END%%



%%PROBLEM_BEGIN%%
%%<PROBLEM>%%
例5. 求所有正整数 $x>1, y>1, z>1$, 使得
$$
1 !+2 !+\cdots+x !=y^z . \label{eq1}
$$
%%<SOLUTION>%%
解:关键一步是证明当 $x \geqslant 8$ 时必有 $z=2$. 因为式\ref{eq1}的左边被 3 整除, 故 $3 \mid y^z$, 从而 $3 \mid y$,于是式\ref{eq1}的右边被 $3^z$ 整除.
另一方面,
$$
1 !+2 !+\cdots+8 !=46233
$$
被 $3^2$ 整除, 但不被 $3^3$ 整除; 而对 $n \geqslant 9$ 有 $3^3 \mid n !$. 所以, 当 $x \geqslant 8$ 时, 式\ref{eq1} 的左边被 $3^2$ 整除而不能被 $3^3$ 整除, 从而 (1) 的右边也如此,即必须 $z=2$.
现在进一步证明,当 $x \geqslant 8$ 时方程 式\ref{eq1} 无解.
模 5 : 当 $x \geqslant 8$ 时, 式\ref{eq1} 的左边 $\equiv 1 !+2 !+3 !+4 ! \equiv 3(\bmod 5)$; 又已证明了此时有 $z=2$, 故 式\ref{eq2} 的右边 $z^2 \equiv 0$, 土 $1(\bmod 5)$, 从而上述断言成立.
最后, 当 $x<8$ 时, 不难通过检验求得 式\ref{eq1} 的解是 $x=y=3, z=2$.
通过比较某个素数在一个等式两边出现的幂次, 以导出结果, 同余的这种变形手法被称为比较素数幂法, 下面的例 6 也应用了这一方法.
%%PROBLEM_END%%



%%PROBLEM_BEGIN%%
%%<PROBLEM>%%
例6. 证明,不定方程
$$
(x+2)^{2 m}=x^n+2 . \label{eq1}
$$
没有正整数解.
%%<SOLUTION>%%
证明:为了后面的论证, 我们先从方程式\ref{eq1}导出一些简单的结论.
显然 $n>1$. 此外, $x$ 必是奇数, 否则将 式\ref{eq1} 模 4 则产生矛盾.
进一步, $n$ 也是奇数, 因为若 $2 \mid n$, 则 $x^n$ 为一个奇数的平方, 从而 式\ref{eq1} 的右边 $\equiv 1+2= 3(\bmod 4)$, 但其左边 $\equiv 1(\bmod 4)$, 这不可能.
故 $2 \nmid n$.
设 $x+1=2^\alpha x_1$, 其中 $x_1$ 为奇数, $\alpha>0$ (因 $x$ 为奇数). 将方程式\ref{eq1}改写为
$$
(x+2)^{2 m}-1=x^n+1 . \label{eq2}
$$
式\ref{eq2}的左边有因子 $(x+2)^2-1=\left(2^\alpha x_1+1\right)^2-1=2^{\alpha+1}\left(2^{\alpha-1} x_1^2+x_1\right)$, 故 $2^{\alpha+1}$ 整除 式\ref{eq2} 的左边.
但另一方面, 由于 $n-1>0$ 为偶数, 用二项式定理易得
$$
x^n+1=x\left(2^\alpha x_1-1\right)^{n-1}+1 \equiv x \cdot 1+1=2^\alpha x_1\left(\bmod 2^{\alpha+1}\right) .
$$
因 $2 \nmid x_1$, 故式\ref{eq2}的右边 $x^n+1 \not \equiv 0\left(\bmod 2^{\alpha+1}\right)$, 矛盾!
%%PROBLEM_END%%



%%PROBLEM_BEGIN%%
%%<PROBLEM>%%
例7. 证明: 不定方程
$$
8^x+15^y=17^z . \label{eq1}
$$
的全部正整数解是 $x=y=z=2$.
%%<SOLUTION>%%
证明:我们先用同余证明, $y$ 和 $z$ 都是偶数.
方程式\ref{eq1}模 4 , 得到
$$
(-1)^y \equiv 1(\bmod 4),
$$
从而 $y$ 是偶数.
将方程(1)模 16 ,得到
$$
8^x+(-1)^y \equiv 1(\bmod 16),
$$
即 $8^x \equiv 0(\bmod 16)$, 故 $x \geqslant 2$.
注意 $17^2 \equiv 1,15^2 \equiv 1(\bmod 32)$, 故若 $z$ 是奇数, 则由 $2 \mid y$ 及 $x \geqslant 2$, 可从式\ref{eq1}得出
$$
1 \equiv 17(\bmod 32),
$$
这不可能.
所以 $z$ 必为一个偶数.
设 $y=2 y_1, z=2 z_1$, 则方程(1)可分解为
$$
\left(17^{z_1}-15^{y_1}\right)\left(17^{z_1}+15^{y_1}\right)=8^x . \label{eq2}
$$
易知式\ref{eq2}中左边两个因数的最大公约数为 2 , 而式\ref{eq2}的右边是 2 的幂, 故必须有
$$
\left\{\begin{array}{l}
17^{x_1}-15^{y_1}=2 . \label{eq3}\\
17^{x_1}+15^{y_1}=2^{3 x-1} . \label{eq4}
\end{array}\right.
$$
将式\ref{eq3}模 32 可知, $z_1$ 与 $y_1$ 必须都是奇数 (否则, 式\ref{eq3}的左边 $\equiv 0,-14,16(\bmod 32)$). 将式\ref{eq3}、\ref{eq4}相加, 得
$$
17^{z_1}=1+2^{3 x-2} . \label{eq5}
$$
若 $x \geqslant 3$, 则 式\ref{eq5} 的右边 $\equiv 1(\bmod 32)$; 而因 $z_1$ 为奇数, 故左边 $\equiv 17(\bmod 32)$, 这不可能, 故必有 $x=2$. 由此及 式\ref{eq5} 得 $z_1=1$, 即 $z=2$, 进而易知 $y_1=1$, 即 $y=2$. 因此 $x=y=z=2$.
这一解法, 是同余结合分解方法的典型的例子.
用同余导出 $y 、 z$ 均是偶数,正是为后面的分解方程作准备.
%%PROBLEM_END%%



%%PROBLEM_BEGIN%%
%%<PROBLEM>%%
例8. 证明: 不定方程
$$
(x+1)^y-x^z=1, x, y, z>1 . \label{eq1}
$$
仅有一组正整数解 $x=2, y=2$ 及 $z=3$.
%%<SOLUTION>%%
分析:这里介绍两种解法.
第一种解法基于同余结合分解手法, 相当简单, 第二种解法采用比较素数幂方法, 虽然较为麻烦, 却具有一些代表性.
证明一首先, 将方程式\ref{eq1}模 $x+1$, 得
$$
-(-1)^z \equiv 1(\bmod x+1),
$$
故 $z$ 是奇数.
将式\ref{eq1}分解为
$$
(x+1)^{y-1}=x^{z-1}-x^{z-2}+\cdots-x+1,
$$
由此易知 $x$ 是偶数.
因为若 $x$ 为奇数, 则上式右边为奇数 $(z)$ 个奇数之和, 故是奇数, 而左边是偶数, 产生矛盾.
同样, 将式\ref{eq1}变形为
$$
(x+1)^{y-1}+(x+1)^{y-2}+\cdots+(x+1)+1=x^{z-1},
$$
可见 $y$ 也是偶数.
现在设 $x=2 x_1, y=2 y_1$, 则式\ref{eq1}可分解为
$$
\left((x+1)^{y_1}-1\right)\left((x+1)^{y_1}+1\right)=x^z . \label{eq2}
$$
因 $x$ 是偶数, 故 $(x+1)^{y_1}-1$ 与 $(x+1)^{y_1}+1$ 的最大公约数是 2 , 又显然有 $x \mid(x+1)^{y_1}-1$. 由这些及式\ref{eq2}推出, 必须
$$
(x+1)^{y_1}-1=2 x_1^z,(x+1)^{y_1}+1=2^{z-1} .
$$
因此 $2^{z-1}>2 x_1^z$, 故 $x_1=1$, 即 $x=2$, 从而易得 $y=2$ 及 $z=3$.
%%PROBLEM_END%%



%%PROBLEM_BEGIN%%
%%<PROBLEM>%%
例8. 证明: 不定方程
$$
(x+1)^y-x^z=1, x, y, z>1 . \label{eq1}
$$
仅有一组正整数解 $x=2, y=2$ 及 $z=3$.
%%<SOLUTION>%%
证明二这一证明分两步进行.
首先证明 $x$ 没有奇素数因子.
采用反证法, 设有一个奇素数 $p$, 使 $p \mid x$, 设 $x=p^a x_1$, 其中 $a \geqslant 1, p \nmid x_1$. 由二项式定理,可将式\ref{eq1}变形为
$$
x y+\sum_{i=2}^y \mathrm{C}_y^i x^i=x^z . \label{eq3}
$$
由此可见 $x^2 \mid x y$, 即 $x \mid y$, 从而 $p \mid y$. 设 $y=p^b y_1, p \nmid y_1$, 则 $b \geqslant a$. 我们将通过比较\ref{eq3}式两边所含 $p$ 的幂次来导出矛盾.
对 $2 \leqslant i \leqslant y$, 设 $p^c \| i$, 则在
$$
\mathrm{C}_y^i x^i=\frac{y}{i} \mathrm{C}_{y-1}^{i-1} x^i=\frac{p^b y_1}{i} \mathrm{C}_{y-1}^{i-1}\left(p^a x_1\right)^i
$$
中, $p$ 的幂次至少是 $d=b+a i-c$. 若 $c=0$, 则 $d>a+b$; 若 $c>0$, 则由 $p \geqslant$ 3 得 $p^c>c+1$, 又 $p^c \mid i$, 故 $p^c \leqslant i$. 因此, $i>c+1$, 从而
$$
d>b+a+c(a-1) \geqslant a+b .
$$
故我们总有 $d \geqslant a+b+1$, 于是 $p^{a+b+1} \mid \mathrm{C}_y^i x^i(2 \leqslant i \leqslant y)$, 进而有
$$
p^{a+b+1} \mid \sum_{i=2}^y \mathrm{C}_y^i x^i .
$$
又 $p^{a+b} \| x y$, 因此\ref{eq3}式左边含 $p$ 的幂次为 $a+b$.
另一方面, 由于 $p^a \| x$, 故 $p^{a z} \| x^z$, 即 \ref{eq3} 式右边含 $p$ 的幂次为 $a z$. 但由原方程式\ref{eq1}可见 $z>y$, 又 $p^b \mid y$, 故 $y \geqslant p^b$, 从而
$$
a z>a y \geqslant a p^b \geqslant a(b+1) \geqslant a+b .
$$
因此\ref{eq3}式左、右两边含 $p$ 的幕次不等, 这不可能.
所以 $x$ 不含奇素数因子, 即 $x$ 为 2 的幕.
设 $x=2^k(k \geqslant 1)$. 由前面证明过的 $x \mid y$, 可知 $y$ 是偶数,设 $y=2 y_1$. 方程式\ref{eq1}可分解为
$$
\left(\left(2^k+1\right)^{y_1}-1\right)\left(\left(2^k+1\right)^{y_1}+1\right)=2^{k z},
$$
因上式左边两个因数的最大公约数为 2 , 而右边是 2 的幂, 故必须
$$
\left(2^k+1\right)^{y_1}-1=2,\left(2^k+1\right)^{y_1}+1=2^{k z-1} .
$$
因此 $k=y_1=1$, 即 $x=y=2$, 故 $z=3$.
%%PROBLEM_END%%


