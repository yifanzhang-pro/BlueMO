
%%PROBLEM_BEGIN%%
%%<PROBLEM>%%
问题1. 设 $\triangle A B C$ 的内切圆 $\Gamma$ 与 $B C$ 切于点 $D, D^{\prime}$ 是圆 $\Gamma$ 上的点, 且 $D D^{\prime}$ 为圆 $\Gamma$ 的直径, 过 $D^{\prime}$ 作圆 $\Gamma$ 的切线与 $A D$ 交于点 $X$, 过 $X$ 作圆 $\Gamma$ 的不同于 $X D^{\prime}$ 的切线,切点为 $N$. 证明: $\triangle B C N$ 的外接圆与圆 $\Gamma$ 切于点 $N$.
%%<SOLUTION>%%
证明: 显然, $A B \neq A C$. 下设 $A B> A C$. 如图(<FilePath:./figures/fig-c9a1.png>), 设圆 $\Gamma$ 与 $A C 、 A B$ 分别切于点 $E 、 F$, 且设 $F E$ 与 $B C$ 交于点 $K$. 则 $K$ 是点 $A$ 关于圆 $\Gamma$ 的极线 $F E$ 上的点.
由配极原则知, $A$ 也是点 $K$ 关于圆 $\Gamma$ 的极线上的点.
因为点 $D$ 在点 $K$ 关于圆 $\Gamma$ 的极线上,所以, $K$ 关于圆 $\Gamma$ 的极线为 $A D$. 同理, 设 $D^{\prime} N$ 与 $B C$ 交于点 $K^{\prime}$, 则 $K^{\prime}$ 关于圆 $\Gamma$ 的极线为 $D X$. 由于 $A D$
与 $D X$ 为同一条直线, 因此, $K^{\prime}=K$. 因为 $B, C ; D, K$ 是调和点列, 且 $\angle D^{\prime} N D=90^{\circ}$, 所以, $N D$ 是 $\angle B N C$ 的角平分线.
设 $N B 、 N C$ 分别与圆 $\Gamma$ 交于点 $P 、 Q$. 则 $D$ 为弧 $\widehat{P Q}$ 的中点.
于是, $P Q / / B C$. 由 $\angle X N P=\angle P Q N=\angle B C N$, 知 $X N$ 与 $\triangle B C N$ 的外接圆切于点 $N$. 从而, $\triangle B C N$ 的外接圆与圆 $\Gamma$ 切于点 $N$.
%%PROBLEM_END%%



%%PROBLEM_BEGIN%%
%%<PROBLEM>%%
问题2. 圆 $O_1$ 与圆 $O_2$ 交于 $A 、 B$ 两点.
过点 $O_1$ 的直线 $D C$ 交圆 $O_1$ 于 $D$ 且切圆 $O_2$ 于 $C, C A$ 切圆 $O_1$ 于 $A$, 圆 $O_1$ 的弦 $A E$ 与直线 $D C$ 垂直.
过 $A$ 作 $A F$ 垂直于 $D E, F$ 为垂足.
求证: $B D$ 平分线段 $A F$.
%%<SOLUTION>%%
证明: 如图(<FilePath:./figures/fig-c9a2.png>), 延长 $C B$ 交 $\odot O_1$ 于 $G$, 对于 $\odot O_1$, 由 $C A$ 是切线, $C O_1 \perp A E$ 知, 直线 $A E$ 是 $C$ 的极线,故 $E C$ 是过 $E$ 点的 $\odot O_1$ 的切线.
由配极定理 3 知四边形 $A B E G$ 为调和四边形.
由配极定理 $2, D G 、 D B 、 D A 、 D E$ 成调和线束.
设 $A E$ 与 $C D$ 交点为 $H$, 连 $A B 、 B E 、 G D$, 又 $\angle B C D=\angle B A C=\angle B E A$, 所以 $H 、 E$ 、 $C 、 B$ 四点共圆.
于是 $\angle E B C=\angle E H C=90^{\circ}$, 故 $\angle G D E=\angle E B C=90^{\circ}$, 即 $D G / / A F$, 结合
$D G 、 D B 、 D A 、 D E$ 成调和线束知 $D B$ 平分 $A F$.
%%PROBLEM_END%%



%%PROBLEM_BEGIN%%
%%<PROBLEM>%%
问题3. 凸四边形 $A B C D$ 外切于 $\odot O, A B 、 B C 、 C D 、 D A$ 上的切点分别是 $E 、 F 、 G 、 H$, 直线 $H E$ 与 $F G$ 相交于点 $P$. 求证: $O P \perp A C$.
%%<SOLUTION>%%
证明: 如图(<FilePath:./figures/fig-c9a3.png>), 以 $\odot O$ 为基圆, 易知 $H E$ 是 $A$ 的极线, $F G$ 是 $C$ 的极线.
而 $H E$ 与 $F G$ 交于 $P$, 因此 $P$ 点既在 $A$ 点极线上,也在 $C$ 点极线上由配极性质 1 知,故 $A 、 C$ 都在 $P$ 点极线上, 从而 $A C$ 即为 $P$ 点的极线.
因此 $O P \perp A C$. 原命题得证.
%%PROBLEM_END%%



%%PROBLEM_BEGIN%%
%%<PROBLEM>%%
问题4. $G$ 是 $\triangle A B C$ 的重心, $M 、 N$ 分别是 $A C 、 A B$ 的中点.
设 $\triangle A N C$ 和 $\triangle A M B$ 的外接圆相交于 $A$ 和 $P, \triangle A M N$ 的外接圆交 $A P$ 于 $T$. 求 $A T: A P$.
%%<SOLUTION>%%
解: 如图(<FilePath:./figures/fig-c9a4-1.png>), 以 $A$ 为反演中心, 单位长度为反演幂,
并记 $K$ 点的反象为 $K^{\prime}$. 则由 $A 、 M 、 P 、 B$ 四点共圆及定理 5 知 $M^{\prime} 、 P^{\prime} 、 B^{\prime}$ 三点共线.
同理 $N^{\prime} 、 P^{\prime} 、 C^{\prime}$ 也共线.
又 $\frac{A N^{\prime}}{A B^{\prime}}=\frac{\frac{1}{A N}}{\frac{1}{A B}}=\frac{A B}{A N}=2$. 所以 $B^{\prime}$ 为 $A N^{\prime}$ 的中点, 同理 $C^{\prime}$ 为 $A M^{\prime}$ 中点, 于是 $P^{\prime}$ 为 $\triangle A N^{\prime} B^{\prime}$ 的重心.
反演后的图形成为图(<FilePath:./figures/fig-c9a4-2.png>), , 于是 $\frac{A T}{A P}=\frac{\frac{1}{A T^{\prime}}}{\frac{1}{A P^{\prime}}}=\frac{A P^{\prime}}{A T^{\prime}}=\frac{2}{3}$.
%%PROBLEM_END%%



%%PROBLEM_BEGIN%%
%%<PROBLEM>%%
问题5. 设 $P$ 为 $\triangle A B C$ 内一点, 令 $\alpha=\angle B P C-\angle A, \beta=\angle C P A-\angle B, \gamma= \angle A P B-\angle C$. 求证: $\frac{P A \cdot \sin A}{\sin \alpha}=\frac{P B \cdot \sin B}{\sin \beta}=\frac{P C \cdot \sin C}{\sin \gamma}$.
%%<SOLUTION>%%
证明: 如图(<FilePath:./figures/fig-c9a5.png>), 以 $P$ 为反演中心, 单位长度为反演幂, 设 $A 、 B 、 C$ 的反点分别为 $A^{\prime} 、 B^{\prime} 、 C^{\prime}$, 因点 $P$ 在 $\triangle A B C$ 内, 所以, 点 $P$ 也在 $\triangle A^{\prime} B^{\prime} C^{\prime}$ 内, 由定理 $1, \angle B^{\prime} A^{\prime} P=\angle P B A, \angle P A^{\prime} C^{\prime}=\angle A C P$, 所以 $\angle B^{\prime} A^{\prime} C^{\prime}=\angle P B A+\angle A C P=\angle B P C-$. $\angle A=\alpha$; 同理 $\angle C^{\prime} B^{\prime} A^{\prime}=\beta, \angle A^{\prime} C^{\prime} B^{\prime}=\gamma$. 又由定理 2, 有 $B^{\prime} C^{\prime}=\frac{B C}{P B \cdot P C}, C^{\prime} A^{\prime}=\frac{C A}{P C \cdot P A}$, $B^{\prime} A^{\prime}=\frac{A B}{P A \cdot P B}$, 对 $\triangle A^{\prime} B^{\prime} C^{\prime}$ 用正弦定理并将上面三式代入即得
$$
\frac{P A \cdot B C}{\sin \alpha}=\frac{P B \cdot C A}{\sin \beta}=\frac{P C \cdot A B}{\sin \gamma} \text {, 即等价于所证.
}
$$
%%PROBLEM_END%%



%%PROBLEM_BEGIN%%
%%<PROBLEM>%%
问题6. 如图(<FilePath:./figures/fig-c9p6.png>), 在弓形中, 内接一对相切的圆, 对每一对相切的圆, 通过它们的切点引公切线.
证明: 所有的切线通过一个点.
%%<SOLUTION>%%
证明: 法一: 如图(<FilePath:./figures/fig-c9a6-1.png>), 设 $P$ 是两圆 $\odot O_1 、 \odot O_2$ 的切点.
作以 $P$ 为反演中心的反演变换, 于是, 在点 $P$ 处相切的两圆反形为一对平行直线 $l_1 / / l_2$, 而和它们相切的弦和弧, 变为 $A^{\prime} L^{\prime} B^{\prime}$ 和 $A^{\prime} K^{\prime} B^{\prime}$, 且 $\widehat{A^{\prime} L^{\prime} B^{\prime}}=\widehat{A^{\prime} K^{\prime} B^{\prime}}$, 公切线 $K L$ 变为 $K^{\prime} L^{\prime}$, 且与 $l_1$ 、 $l_2$ 平行.
所以,直线 $A^{\prime} B^{\prime}$ 垂直平分 $K^{\prime} L^{\prime}$. 换言之, 过点 $A 、 P 、 B$ 的弧平分弓形角 $A 、 B$ 且垂直直线 $K L$. 然而, 恰存在一个过点 $A 、 B$ 的圆, 平分 $\angle A 、 \angle B$ (它的中心 $O$ 是从点 $A 、 B$ 分别向 $\angle A 、 \angle B$ 的平分线引的垂线的交点), 直线 $K L$ 垂直这个圆, 因此, 通过它的中心.
于是, 条件中所有直线都通过点 $O$.
%%PROBLEM_END%%



%%PROBLEM_BEGIN%%
%%<PROBLEM>%%
问题6. 如图(<FilePath:./figures/fig-c9p6.png>), 在弓形中, 内接一对相切的圆, 对每一对相切的圆, 通过它们的切点引公切线.
证明: 所有的切线通过一个点.
%%<SOLUTION>%%
法二: 如图(<FilePath:./figures/fig-c9a6-2.png>), 连结两个切点 $T 、 S$ 及 $V 、 U$. 设它们相交于 $M$. 则由 "圆的初步" 习题 19 的引理知 $M$ 为优弧 $A B$ 的中点 ( $M$ 为定点). 且由 $\angle B A M=\angle B T M= \angle A T M$ 有 $\triangle A S M \backsim \triangle T A M$, 从而 $M A^2=M S \cdot M T$. 同理 $M B^2=M U \cdot M V$. 故 $M S \cdot M T=M U \cdot M V, M$ 在 $\odot O_1$ 与 $\odot O_2$ 的根轴上.
而 $K L$ 是两圆的公切线, 也是两圆的根轴.
故 $M$ 在 $K L$ 上, 即所有切线都过定点 $M$.
%%PROBLEM_END%%



%%PROBLEM_BEGIN%%
%%<PROBLEM>%%
问题7. 如图(<FilePath:./figures/fig-c9p7.png>), 在线段 $A B$ 上取点 $C$, 以线段 $A C 、 B C 、 A B$ 为直径分别作圆, $\odot O$ 与这三个圆都相切.
证明: $\odot O$ 的直径等于它的圆心到直线 $A B$ 的距离.
%%<SOLUTION>%%
证明: 如图(<FilePath:./figures/fig-c9a7.png>), 以点 $C$ 为反演中心作反演变换.
以 $A C$ 、 $B C 、 A B$ 为直径的圆分别反演成直线 $A^{\prime} D 、 B^{\prime} E$ 、以 $A^{\prime} B^{\prime}$ 为直径的圆, 且直线 $A^{\prime} D 、 B^{\prime} E$ 与 $A^{\prime} B^{\prime}$ 垂直, $\odot O$ 反演成 $\odot O^{\prime}$, 且与直线 $A^{\prime} D 、 B^{\prime} E$ 及以 $A^{\prime} B^{\prime}$ 为直径的圆都相切.
由于 $\odot O 、 \odot O^{\prime}$ 关于点 $C$ 位似, 所以, $\odot O$ 的直径与圆心到 $A B$ 的距离的比等于 $\odot O^{\prime}$ 的直径与圆心到 $A^{\prime} B^{\prime}$ 的距离的比.
易知后者的比值为 1 .
%%PROBLEM_END%%



%%PROBLEM_BEGIN%%
%%<PROBLEM>%%
问题8. 已知圆内接四边形 $A B C D$, 直线 $A D$ 和 $B C$ 交于点 $E$, 且点 $C$ 在点 $B 、 E$ 之间, 对角线 $A C 、 B D$ 交于 $F$, 设点 $M$ 为边 $C D$ 的中点, 点 $N$ 是 $\triangle A B M$ 的外接圆上的不同于 $M$ 的点, 且满足 $\frac{A N}{B N}=\frac{A M}{B M}$. 证明: $E 、 F 、 N$ 三点共线.
%%<SOLUTION>%%
证明: 如图(<FilePath:./figures/fig-c9a8.png>), 延长 $C D 、 B A$ 交于点 $P$, 则直线 $E F$ 即为点 $P$ 关于 $\odot O$ 的极线.
(定理 1) 欲证 $N$ 在直线 $E F$ 上, 只需证 $N$ 对 $\odot O$ 的极线过 $P$ 点.
设 $\triangle A M B$ 外接圆为 $\odot O^{\prime}$, 因为 $A 、 M 、 B 、 N$ 均在 $\odot O^{\prime}$ 上, 且 $\frac{A M}{M B}= \frac{A N}{N B} \cdot A M \cdot N B=A N \cdot M B$, 故 $A M B N$ 为调和四边形.
所以点 $M 、 N$
处的两条切线交于直线 $A B$ 上,设为 $R$, 取 $A B$ 中点 $S$.
可知, 点 $N$ 对 $\odot O$ 的幂等于点 $N$ 对 $\triangle O S M$ 外接圆的幂.
过 $P$ 作 $P W \perp O N$ 于 $W$, 则 $O 、 M 、 P 、 W 、 S$ 共圆.
则 $N$ 为 $\odot O$ 的幂等于 $N$ 对 $\triangle O W P$ 外接圆的幂.
从而 $P$ 点在 $N$ 关于 $\odot O$ 的极线上,故结论成立,得证.
%%PROBLEM_END%%



%%PROBLEM_BEGIN%%
%%<PROBLEM>%%
问题9. 已知三角形 $A B C$ 及其内切圆 $O, E 、 F 、 G$ 分别为 $B C 、 B A 、 A C$ 边上的切点, $H$ 为边 $B C$ 上高 $A D$ 的中心, $E H$ 交圆 $O$ 于 $I$. 求证: $I E$ 平分 $\angle B I C$.
%%<SOLUTION>%%
证明: 如图(<FilePath:./figures/fig-c9a9.png>), 设直线 $G F$ 与直线 $B C$ 交于点 $K$, 延长 $K I$ 交圆 $O$ 于 $M$,并设 $A E$ 与 $I M$ 交于点 $L$, 连结 $M E$. 由上章例 2 的证明过程知 $C 、 B 、 E 、 K$ 成调和点列, 又注意到 $A$ 的极线 $F G$ 过 $K, E$ 的极线 $B C$ 过 $K$, 所以 $K$ 的极线过 $A 、 E$, 即 $K$ 的极线为直线 $A E$,于是, $K 、 L$ 共轭, 故 $K 、 L 、 I 、 M$ 成调和点列, 于是从 $E$ 点出发的线束 $E K 、 E L 、 E I 、 E M$ 或 $E D 、 E A 、 E H 、 E M$ 成调和线束, 结合 $D H=H A$, 于是 $D 、 A 、 H 、 \infty$ 成调和点列, 仍由 $E$ 点出发知, $E D 、 E A 、 E H 、 E \infty$ (即过 $E$ 平行于直线 $D A$ 的直线) 成调和线束, 那么 $E \infty$ 与 $E M$ 重合, 即 $E M / / D A$, 或 $E M \perp B C$, 注意到 $B C$ 是 $E$ 点处切线即 $O E \perp B C$, 故 $M 、 O 、 E$ 三点共线, 于是 $M E$ 是圆 $O$ 的直径, $K I \perp I E$, 结合 $K 、 E 、 B 、 C$ 成调和点列, 于是 $I E$ 平分 $\angle B I C$.
%%PROBLEM_END%%



%%PROBLEM_BEGIN%%
%%<PROBLEM>%%
问题10. 在四个圆中,每个圆都和其他的两个圆外切.
证明:四个切点位于同一个圆上.
%%<SOLUTION>%%
证明: 设这四个圆为圆 $A$ 、圆 $B$ 、圆 $C$ 、 圆 $D$,
取 $A 、 B$ 的公切点 $E$, 以 $E$ 为反演中心, 单位长度为反演半径作反演变换得到以下命题:
如图(<FilePath:./figures/fig-c9a10.png>), 已知直线 $l_1, l_2$ 平行, 有圆 $O_1$ 和圆 $O_2$ 相切于点 $X$, 且 $O_1$ 与 $l_1$ 切于 $Y, O_2$ 与 $l_2$ 切于 $Z$, 则 $X 、 Y 、 Z$ 三点共线,该结论由三角形的相似得证.
%%PROBLEM_END%%



%%PROBLEM_BEGIN%%
%%<PROBLEM>%%
问题11. 已知 $\triangle A B C$ 的中线 $A M$ 交其内切圆 $\Gamma$ 于点 $K 、 L$, 分别过 $K 、 L$ 且平行于 $B C$ 的直线交圆 $\Gamma$ 于点 $X 、 Y, A X 、 A Y$ 分别交 $B C$ 于点 $P 、 Q$. 证明: $B P=C Q$.
%%<SOLUTION>%%
证明: 如图(<FilePath:./figures/fig-c9a11.png>), 设内切圆圆心为 $I, \odot I$ 在 $B C$ 上切点为 $H$. 过 $A$ 作平行于 $B C$ 的直线 $l$, 设内切圆在边 $A B 、 A C$ 上的切点分别为 $D 、 E$, 设直线 $D E$ 与 $l$ 交于点 $F$, 又设 $D E$ 交 $A M$ 于 $G$, 由于 $M$ 平分线段 $A B$, 故 $A C 、 A B 、 A M 、 l$ 成调和线束, 或 $A E 、 A D 、 A G 、 A F$ 成调和线束, 于是 $E 、 D 、 G 、 F$ 成调和点列, 结合配极性质 3 知 $F$ 的极线过 $G$, 又 $A$ 的极线 $D E$ 过 $F$, 于是 $F$ 的极线也过 $A$, 那么 $F$ 的极线是直线 $A G$, 又 $G$ 的极线过 $A$, 所以 $G$ 的极线为 $A F$, 又 $A F / / B C$, 所以 $G$ 的极线 $/ / H$ 的切线 $B C$, 于是 $G H$ 的极点为无穷远点 (无穷远点的极线过圆心), 因此 $G H$ 过内心 $I$. 也就是说 $G I \perp B C$, 以及 $B C / / L Y / / X K, X 、 K ; L 、 Y$ 分别关于直线 $I G$ 对称, 故 $X 、 G 、 Y$ 共线, $M P=X K \cdot \frac{A K}{A M}=L Y \cdot \frac{K G}{G M} \cdot \frac{A K}{A M}=L Y \cdot \frac{A L}{A M}$ (这步用到 $A$ 、 $G 、 K 、 L$ 成调和点列 $)=M Q$. 于是 $B P=C Q$.
%%PROBLEM_END%%



%%PROBLEM_BEGIN%%
%%<PROBLEM>%%
问题12. 四边形 $A B C D$ 有内切圆 $\odot I, E 、 F 、 G 、 H$ 分别是 $\odot I$ 在四边 $A B 、 B C$ 、 $C D 、 D A$ 上的切点.
求证: $A C 、 B D 、 E G 、 F H$ 四线共点.
%%<SOLUTION>%%
证明: 如图(<FilePath:./figures/fig-c9a12.png>), 设 $A B$ 和 $C D$ 交于点 $O, A D$ 和 $B C$ 交于点 $P$. 则对于 $\odot I, O$ 的极线为 $E G, P$ 的极线为 $F H$, 于是 $E G$ 和 $F H$ 交于点 $X, X$ 为 $O P$ 的极点.
令 $E H$ 交 $F G$ 于点 $N, E F$ 交 $G H$ 于点 $M$, 则 $M$ 为 $D$ 的极线与 $B$ 的极线的交点, 因此 $M$ 的极线过 $B 、 D$, 即 $M$ 的极线为直线 $B D$. 同理, $N$ 的极线为直线 $A C$, 设 $A C$ 交 $B D$ 于 $Y$, 又由配极定理 1 知, $M$ 的极线过 $N, N$ 的极线过 $M$. 我们发现 $Y$ 的极线过 $M 、 N$. 故 $Y$ 的极线为直线 $M N$. 对六边形 $E E H G G F$ (退化六边形) 使用帕斯卡定理 $O 、 M 、 N$ 共线, 同理 $M 、 N 、 P$ 共线.
于是 $X$ 的极线 $=Y$ 的极线, 所以 $X=Y$. 即 $A C$ 、 $B D 、 E G 、 F H$ 四线共点.
%%<REMARK>%%
注:: 此题虽然可以由三角方法解决 (见本章例 13 引理), 但这里我们用配极方法给出另一种解答, 供读者欣赏.
%%PROBLEM_END%%


