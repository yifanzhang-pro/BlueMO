
%%TEXT_BEGIN%%
圆的初步.
1. 圆的内容非常丰富, 许多平面儿何竞赛问题都和它有关, 其中四点共圆是圆的一个极其重要的问题.
2. 圆和有关的角:
(1) 同弧所对的圆周角相等;
(2) 弦切角等于弦所对的圆周角;
(3) 顶点在某圆内部的角, 叫做这圆的圆内角.
圆的圆内角, 等于它本身及其对顶角包含的弧所对的圆周角之和; 顶点在某圆外部而两边与圆均有公共点的角, 叫做这圆的圆外角.
圆的圆外角, 等于它包含的两弧所对的圆周角之差.
3. 多值有向角: 我们知道, 射线绕着它的端点依逆时针的方向旋转为正角, 顺时针的方向为负角.
假定有两直线 $l, l^{\prime}$, 它们或相交或平行或重合.
任意选定一点 $O$, 通过 $O$ 作两直线 $h, h^{\prime}$ 使分别平行 (或重合) 于 $l, l^{\prime}$, 然后将 $h$ 绕 $O$ 点依任何方向旋转, 而每当 $h$ 重合于 $h^{\prime}$ 一次, $h$ 便旋过一个角度, 这个角度或小于等于直角, 或大于等于直角, 甚或大于若干周角.
这些角度视旋转方向为正向或负向而规定它们的值是正的或负的.
现在我们把这样得到的角度都当作 $l$ 与 $l^{\prime}$ 所做成角的角度, 并用记号 " $4 l l, l^{\prime}$ "来表示.
凡两直线做成的角若是按这个方法来测定的, 那么称为多值有向角.
应该指出, 这样的角只注意于旋转的方向, $l$ 与 $l^{\prime}$ 本身的正负向是无需给定的.
又书写记号 " Х $l, l^{\prime}$ "时, 必须注意 $l$ 与 $l^{\prime}$ 的先后次序, 不得错乱.
假定两个多值有向角的通值能够一对一地对应相等, 那么我们就说这两个多值有向角相等.
在这个定义中, 不难晓得多值有向角的相等具有反身性, 对称性,传递性.
显然,若两个多值有向角相等, 则它们的最小非负值必相等.
有了多值有向角这个概念, 就可以导出一个有关三点共线的命题: 三点 $A 、 B 、 C$ 共线的充要条件是: $\measuredangle A B C=0$ 或 $\backslash P A B=\measuredangle \backslash P A C$.
4. 四点共圆的条件.
四点 $A 、 B 、 C 、 D$ (不论次序) 共圆的必要且充分条件为 $\Varangle . A C B= \measuredangle A D B \neq 0$.
此外,切割线定理, 相交弦定理的逆定理都可作为四点共圆的依据.
5. 与圆有关的两个著名定理:
(1) 托勒密定理: 在凸四边形 $A B C D$ 中, $A B \times C D+A D \times B C \geqslant A C \times B D$. 当且仅当四边形 $A B C D$ 是圆内接四边形时,等号成立.
(2) 西姆松定理: 过三角形外接圆上异于三角形顶点的任意一点作三边的垂线,则三垂足点共线 (此线常称为西姆松线).
西姆松定理的逆定理也是成立的: 若一点在三角形三边所在直线上的射影共线, 则该点在此三角形的外接圆上.
6. 密克 (Miquel) 定理: 设在 $\triangle A B C$ 三边 $B C$ 、 $C A 、 A B$ 所在直线上任取一点 $X 、 Y 、 Z$ (如图(<FilePath:./figures/fig-c4i1.png>)), 则三圆 $\odot A Y Z 、 \odot B Z X 、 \odot C X Y$ 共点.
事实上,令 $\odot B Z X$ 与 $\odot C X Y$ 的第二交点为 $O$, 连结 $O X, O Y 、 O Z$, 则 $\measuredangle A Y O=\not C X O=\measuredangle B Z O$. 因知$O$ 点在 $\odot A Y Z$.上.
%%TEXT_END%%



%%PROBLEM_BEGIN%%
%%<PROBLEM>%%
例1. 如图(<FilePath:./figures/fig-c4i2.png>), 设 $B$ 是圆 $S_1$ 上的点, 过 $B$ 作圆 $S_1$ 的切线, $A$ 为该切线上异于 $B$ 的点, 又 $C$ 不是圆 $S_1$ 上的点, 且线段 $A C$ 交圆 $S_1$ 于两个不同的点.
圆 $S_2$ 与 $A C$ 相切于点 $C$, 与圆 $S_1$ 相切于点 $D$, 且 $D$ 与 $B$ 在直线 $A C$ 的两侧.
证明: $\triangle B C D$ 的外心在 $\triangle A B C$ 的外接圆上.
%%<SOLUTION>%%
证明:作两圆公切线 $T T^{\prime}$ 满足 $T$ 与 $A$ 在 $B D$ 同侧, 取 $B D$ 中点 $E, C D$ 中点 $F$, 连结$B K 、 E K 、 D K 、 F K 、 C K$.
因 $\angle T D B=\angle A B D, \angle T^{\prime} D C=\angle D C A$, 则
$$
\begin{aligned}
\angle B D C & =180^{\circ}-\angle T D B+\angle T^{\prime} D C \\
& =180^{\circ}-\angle A B D+\angle D C A \\
& =180^{\circ}-(\angle A B C-\angle D B C)+(\angle D C B-\angle A C B) \\
& =180^{\circ}-\angle A B C-\angle A C B+\angle D B C+\angle D C B \\
& =\angle B A C+180^{\circ}-\angle B D C .
\end{aligned}
$$
于是
$$
2 \angle B D C=180^{\circ}+\angle B A C .
$$
故
$$
\begin{aligned}
\angle B K C & =\angle B K D+\angle D K C \\
& =2 \cdot(\angle E K D+\angle D K F) \\
& =2 \cdot \angle E K F \\
& =2 \cdot\left(180^{\circ}-\angle B D C\right) \\
& =180^{\circ}-\angle B A C .
\end{aligned}
$$
因此 $K$ 在 $\triangle A B C$ 的外接圆上.
%%PROBLEM_END%%



%%PROBLEM_BEGIN%%
%%<PROBLEM>%%
例2. 如图(<FilePath:./figures/fig-c4i3.png>), $\triangle A B C$ 的内切圆 $I$ 切 $B C 、 C A$ 于 $D 、 E, K 、 L$ 为 $A B 、 A C$ 的中点, 则 $D E$ 与 $K L$ 的交点 $T$ 在 $B I$ 的延长线上.
%%<SOLUTION>%%
证明:设直线 $B I$ 分别交 $E D 、 K L$ 于 $T^{\prime \prime} 、 T^{\prime}$, 连结 $T^{\prime} B 、 T^{\prime \prime} B 、 A T^{\prime} 、 A I 、 I D 、 E T^{\prime}$.
易知 $\angle K B T^{\prime}=\angle K T^{\prime} B$.
所以 $B K=K T^{\prime}=A K$, 从而 $\angle A T^{\prime} B=90^{\circ}$.
又 $\angle A E I=90^{\circ}$, 故 $A 、 E 、 T^{\prime} 、 I$ 四点共圆.
又 $\angle C E T^{\prime}=\frac{\pi-\angle A C B}{2}=\angle A B C+\angle B A C=\angle A I T^{\prime \prime}$,
故 $A 、 E 、 T^{\prime \prime} 、 I$ 四点共圆.
又 $T^{\prime} 、 T^{\prime \prime}$ 都在直线 $B I$ 上, 所以 $\triangle A E I$ 的外接圆与直线 $B I$ 的交点为 $I$ 、 $T^{\prime} 、 T^{\prime \prime}$, 故 $T^{\prime}$ 与 $T^{\prime \prime}$ 重合, 即 $T$ 在直线 $B I$ 上.
%%PROBLEM_END%%



%%PROBLEM_BEGIN%%
%%<PROBLEM>%%
例3. 如图(<FilePath:./figures/fig-c4i4.png>), 设 $A 、 B$ 是定点, $C$ 是动点, 且 $\angle A C B=\alpha$ 是定角, 其中, $0^{\circ}<\alpha<180^{\circ} . \triangle A B C$ 的内切圆 $\odot I$ 在边 $B C 、 C A 、 A B$ 上的切点分别为 $F 、 E 、 D$, $E F$ 分别与直线 $A I 、 B I$ 交于点 $M 、 N$. 证明: 线段 $M N$ 的长是定长,且 $\triangle D M N$ 的外接圆过一个定点.
%%<SOLUTION>%%
证明:取线段 $A B$ 的中点 $O$.
因 $\angle C E F=90^{\circ}-\frac{1}{2} \angle C=180^{\circ}-\angle A I B= \angle A I N$.
所以 $I 、 N 、 E 、 A$ 四点共圆.
又 $I 、 E 、 A 、 D$ 四点共圆, 则 $I 、 N 、 E 、 A 、 D$ 五点共圆.
同理, $B 、 F 、 M 、 I 、 D$ 五点共圆.
从而, $\angle A N B=\angle A E I=90^{\circ}, \angle A M B=\angle I F B=90^{\circ}$.
因此, 点 $M 、 N$ 均在以 $A B$ 为直径的圆上.
则
$$
\begin{aligned}
\angle M O N & =2 \angle M A N=2 \angle M B N \\
& =\angle N A M+\angle N B M \\
& =\angle N D I+\angle M D I=\angle M D N .
\end{aligned}
$$
故 $M 、 N 、 O 、 D$ 四点共圆.
因此, $\triangle M N D$ 的外接圆过定点 $O$.
另一方面, $M N=A B \sin \angle N A M=A B \sin \angle I E F=A B \sin \frac{C}{2}=A B \sin \frac{\alpha}{2} \text { 为定值.
}$
%%PROBLEM_END%%



%%PROBLEM_BEGIN%%
%%<PROBLEM>%%
例4. 如图(<FilePath:./figures/fig-c4i5.png>), $A B$ 是圆 $O$ 的直径.
$C$ 与 $D$ 是互异的圆 $O$ 上的两点, 且在 $A B$ 的一侧.
过 $C 、 D$ 作圆的切线交于点 $E$. 线段 $A D$ 与 $B C$ 交于点 $F$, 直线 $E F$ 交 $A B$ 于 $M$. 求证: $E, C, M, D$ 共圆.
%%<SOLUTION>%%
证明:连结 $E O 、 C O 、 D O 、 C A$. 由 $\angle C O E= \angle C A F$, 知 Rt $\triangle O O E \backsim$ Rt $\triangle C A F$.
所以, $\frac{C E}{C F}=\frac{C O}{C A}$. 又 $\angle E C F=90^{\circ}-\angle B C O=\angle O C A$,
则 $\triangle E C F \backsim \triangle O C A$.
故 $\angle C A O=\angle C F E=\angle B F M$. 于是, $\angle F M B=\angle A C B=90^{\circ}$.
因此, $O 、 M 、 D 、 E 、 C$ 五点共圆.
%%PROBLEM_END%%



%%PROBLEM_BEGIN%%
%%<PROBLEM>%%
例5. 如图(<FilePath:./figures/fig-c4i6.png>), 给定锐角三角形 $A B C, O$ 为外心, 直线 $A O$ 交边 $B C$ 于 $D$, 动点 $E 、 F$ 在 $A B 、 A C$ 上, 使得 $A 、 E 、 D 、 F$ 四点共圆.
求证: 线段 $E F$ 在 $B C$ 上的投影为恒定长度.
%%<SOLUTION>%%
证明:取 $A B$ 上一点 $E^{\prime}, A C$ 上一点 $F^{\prime}$ 满足 $A 、 E^{\prime} 、 D 、 F^{\prime}$ 四点共圆.
下面证明: $E F$ 在 $B C$ 上的投影长度等于 $E^{\prime} F^{\prime}$ 在 $B C$ 上的投影长度, 由正弦定理,
$$
\begin{aligned}
\frac{D E}{D F} & =\frac{\sin \angle E F D}{\sin \angle D E F} \\
& =\frac{\sin \angle E A D}{\sin \angle D A F} \\
& =\frac{\sin \left(\frac{\pi}{2}-C\right)}{\sin \left(\frac{\pi}{2}-B\right)} \\
& =\frac{\cos C}{\cos B},
\end{aligned}
$$
连结 $D E 、 D E^{\prime} 、 D F 、 D F^{\prime}$, 则 $\angle D E^{\prime} B=\angle D F^{\prime} A, \angle D E B=\angle D F A$, 所以 $\triangle D E E^{\prime} \backsim \triangle D F F^{\prime}$, 于是 $\frac{E E^{\prime}}{F F^{\prime}}=\frac{D E}{D F}=\frac{\cos C}{\cos B}$, 即 $E E^{\prime} \cdot \cos B=F F^{\prime} \cdot \cos C$, 故 $E F$ 在 $B C$ 上的投影长度 $=A E$ 在 $B C$ 上的投影长度 $+A F$ 在 $B C$ 上的投影长度
$$
\begin{aligned}
= & A E \cdot \cos B+A F \cdot \cos C=A E \cdot \cos B+A F \\
& \cdot \cos C+E E^{\prime} \cdot \cos B-F F^{\prime} \cdot \cos C \\
= & A E^{\prime} \cdot \cos B+A F^{\prime} \cdot \cos C \\
= & E^{\prime} F^{\prime} \text { 在 } B C \text { 上的投影长度, 为与 } E 、 F \text { 具体位 }
\end{aligned}
$$
$=E^{\prime} F^{\prime}$ 在 $B C$ 上的投影长度, 为与 $E 、 F$ 具体位置无关的常数.
%%PROBLEM_END%%



%%PROBLEM_BEGIN%%
%%<PROBLEM>%%
例6. 如图(<FilePath:./figures/fig-c4i7.png>) 在 $\triangle A B C$ 中, $\angle B A C= 90^{\circ}$, 点 $E$ 在 $\triangle A B C$ 的外接圆 $\Gamma$ 的弧 $B C$ (不含点 $A)$ 内, $A E>E C$, 连结 $E C$ 并延长至点 $F$, 使得 $\angle E A C=\angle C A F$, 连结 $B F$ 交圆 $\Gamma$ 于点 $D$, 连结 $E D$, 记 $\triangle D E F$ 的外心为 $O$, 求证 : $A 、 C 、 O$ 三点共线.
%%<SOLUTION>%%
证明:作 $\triangle A E F$ 的外接圆交 $A C$ 延长线于点 $O^{\prime}$, 连结 $O^{\prime} E 、 O^{\prime} F$. 注意到 $A C$ 平分 $\angle E A F$, 所以 $A O^{\prime}$ 在圆 $A E F$ 内平分 $\angle A$, 则 $O^{\prime}$ 为 $\overparen{E F}$ 的中点, $O^{\prime} E=O^{\prime} F$.
由于 $\angle E O^{\prime} F=180^{\circ}-\angle E A F=180^{\circ}-2 \angle E A O^{\prime}, \label{eq1}$,
$$
\angle E D B=\angle E A B=90^{\circ}-\angle E A O, \label{eq2}
$$
由 式\ref{eq1}、\ref{eq2} 知
$$
\angle E D B=\frac{1}{2} \angle E O^{\prime} F \text {. }
$$
从而 $D$ 在以 $O^{\prime}$ 为圆心, $O^{\prime} E$ 为半径的圆上.
所以 $O^{\prime} D=O^{\prime} E=O^{\prime} F, O^{\prime}$ 与 $O$ 重合, 于是 $A 、 C 、 O$ 共线.
%%PROBLEM_END%%



%%PROBLEM_BEGIN%%
%%<PROBLEM>%%
例7. 如图(<FilePath:./figures/fig-c4i8.png>), $M 、 N$ 分别为锐角 $\triangle A B C (\angle A<\angle B)$ 的外接圆 $\Gamma$ 上弧 $B C$ 、弧 $A C$ 的中点.
过点 $C$ 作 $P C / / M N$ 交圆 $\Gamma$ 于点 $P, I$ 为 $\triangle A B C$ 的内心, 连结 $P I$ 并延长交圆 $\Gamma$ 于 $T$.
(1) 求证: $M P \cdot M T=N P \cdot N T$;
(2) 在弧 $A B$ (不含点 $C)$ 上任取一点 $Q(\neq A, T$, $B)$, 记 $\triangle A Q C 、 \triangle Q C B$ 的内心分别为 $I_1 、 I_2$. 求证 : $Q$ 、 $I_1 、 I_2 、 T$ 四点共圆.
%%<SOLUTION>%%
证明:(1) $P C / / N M \Rightarrow$ 等腰梯形 $P C M N$, 连结 $I M 、 I C 、 C M 、 A I 、 C N$, $I$ 为内心, 故 $A I$ 延长线过 $\overparen{B C}$ 中点 $M$, 于是
$$
\begin{aligned}
\angle C I M & =\angle C A I+\angle I C A \\
& =\angle B A I+\angle I C A \\
& =\angle B A M+\angle I C A \\
& =\angle B C M+\angle I C B \\
& =\angle I C M,
\end{aligned}
$$
故 $I M=C M$, 又 $P C M N$ 为等腰梯形,有 $C M=P N$, 于是 $I M=N P$, 同理可证 $P M=I N$.
由此可得四边形 $M I N P$ 为平行四边形, 即 $P I$ 平分 $M N$.
所以 $T I$ 平分线段 $M N, S_{\triangle P N T}=S_{\triangle P M T} \Rightarrow \frac{1}{2} \cdot P M \cdot T M \cdot \sin \angle P M T= \frac{1}{2} \cdot P N \cdot T N \cdot \sin \angle P N T$, 又 $\angle P M T$ 与 $\angle P N T$ 互补, $\sin \angle P M T= \sin \angle P N T$, 于是 $P M \cdot T M=P N \cdot T N$.
(2)易知 $Q 、 I_1 、 N$ 共线; $Q 、 I_2 、 M$ 共线, 连结 $N Q 、 M Q 、 I_1 T 、 I_2 T$, 首先证明 $\triangle I_1 N T \backsim \triangle I_2 M T$, 这是由于 $N I_1=N C . M I_2=M C$, 又 $\frac{N I_1}{N T}=\frac{N C}{N T}=\frac{M P}{N T}= \frac{N P}{M T}=\frac{M C}{M T}=\frac{M I_2}{M T}$,
$\angle I_1 N T=\angle Q N T=\angle Q M T=\angle I_2 M T$, 由此可得 $\triangle I_1 N T \backsim \triangle I_2 M T$, 从而有 $\angle Q I_1 T=180^{\circ}-\angle N I_1 T=180^{\circ}-\angle M I_2 T=\angle Q I_2 T$.
于是, $Q 、 I_1 、 I_2 、 T$ 四点共圆.
%%<REMARK>%%
注:1. (2)这道题多次在数学竟赛中出现, 是一道较难的问题,但是给出命题 (1)以后,两部分都不算太难.
2. 如图(<FilePath:./figures/fig-c4i9.png>), $I$ 为 $\triangle A B C$ 内心, $A I$ 与 $\triangle A B C$ 外接圆交于 $D$, 则 $D B=D I=D C$, 本题用到了这个内心的重要性质, 有的人称之为"鸡爪定理".
%%PROBLEM_END%%



%%PROBLEM_BEGIN%%
%%<PROBLEM>%%
例8. 设 $L$ 在 $\triangle A B C$ 的边 $B A$ 上,延长 $C A$ 至 $K$ 使 $\angle C K B=\frac{1}{2} \angle C L B$, 延长 $C B$ 至 $M$ 使 $\angle C M A=\frac{1}{2} \angle C L A$. 设 $\triangle C M K$ 的外心为 $O$, 则 $O L \perp A B$.
%%<SOLUTION>%%
证明:如图(<FilePath:./figures/fig-c4i10.png>),作 $\triangle C M K$ 外接圆 $W$, 设 $M A \cap W=\{S\}, K B \cap W=\{R\}$, 则 $\angle C O S=2$ • $\angle C M S=\angle C L A, \angle C O R=2 \cdot \angle C K R=\angle C L B$. 从而 $\angle C O S+\angle C O R=\angle C L A+\angle C L B=180^{\circ}$. 所以 $S, O 、 R$ 三点共线.
对 $C 、 L 、 M 、 S 、 R 、 K$ 使用帕斯卡定理知 $C P \cap S R=P, C P$ 表示过 $C$ 的圆 $W$ 的切线, $C M \cap K R=B, C K \cap S M=A$, 则 $P 、 B 、 A$ 三点共线.
从而过 $C$ 的切线、SR、 $A B$ 交于点 $P$. 由 $\angle O O R= \angle C L B$ 知 $C 、 O 、 L 、 P$ 共圆.
从而 $\angle O L B=\angle O L A=90^{\circ}$, 即 $O L \perp A B$. 注此题用到了帕斯卡定理:
如图(<FilePath:./figures/fig-c4i11.png>), 对圆内接六边形 $A B C D E F$, 设 $A B \cap D E=\{X\}, B C \cap E F=\{Y\}, C D \cap F A=\{Z\}$, 则 $X$ 、 $Y 、 Z$ 三点共线.
这是一个很有用的结论.
其证明见习题 2 第 16 题.
%%PROBLEM_END%%



%%PROBLEM_BEGIN%%
%%<PROBLEM>%%
例9. 如图(<FilePath:./figures/fig-c4i12.png>), 已知 $\triangle A B C$ 内切圆 $\odot I$ 分别与边 $A B 、 B C$ 切于点 $F 、 D$, 直线 $A D 、 C F$ 分别与 $\odot I$ 交于另一点 $H 、 K$. 求证 : $\frac{F D \cdot H K}{F H \cdot D K}=3$.
%%<SOLUTION>%%
证明:设 $A F=x, B F=y, C D=z$. 由斯特瓦尔特定理得 $A D^2=\frac{B D}{B C} \cdot A C^2+\frac{C D}{B} \frac{D}{C} \cdot A B^2- B D \cdot D C=\frac{y(x+z)^2+z(x+y)^2}{y+z}-y z=x^2+ \frac{4 x y z}{y+z}$. 由切割线定理得: $A H=\frac{A F^2}{A D}=\frac{x^2}{A D}$. 故 $H D=A D-A H=\frac{A D^2-x^2}{A D}=\frac{4 x y z}{A D(y+z)}$.
同理
$$
K F=\frac{4 x y z}{C F(x+y)} .
$$
因为 $\triangle C D K \backsim \triangle C F D$, 所以 $D K=\frac{D F \cdot C D}{C F}=\frac{D F}{C F} \cdot z$.
又因为 $\triangle A F H \backsim \triangle A D F$, 所以 $F H=\frac{D F \cdot A F}{A D}=\frac{D F}{A D} \cdot x$.
由余弦定理得:
$$
\begin{gathered}
D F^2=B D^2+B F^2-2 B D \cdot B F \cos B \\
=2 y^2\left[1-\frac{(y+z)^2+(x+y)^2-(x+z)^2}{2(x+y)}\right] \\
=\frac{4 x y^2 z}{(x+y)(y+z)} . \\
\text { 故 } \frac{K F \cdot H \cdot D}{F H \cdot D K}=\frac{\frac{4 x y z}{C F(x+y)} \cdot \frac{4 x y z}{\frac{A D(y+z)}{D F} \cdot x \cdot \frac{D F}{C F} \cdot z}}{\frac{D D}{C D}} \\
=\frac{16 x y^2 z}{D F^2(x+y)(y+z)}=4
\end{gathered} \label{eq1}
$$
对圆内接四边形 $D K H F$ 应用托勒密定理得
$$
K F \cdot H D=D F \cdot H K+F H \cdot D K . 
$$
再结合式\ref{eq1}得
$$
\frac{F D \cdot H K}{F H \cdot D K}=3 .
$$
%%PROBLEM_END%%



%%PROBLEM_BEGIN%%
%%<PROBLEM>%%
例10. 已知 $\odot O_1$ 与 $\odot O_2$ 外切于点 $T$, 一直线与 $\odot O_2$ 相切于点 $X$, 与 $\odot O_1$ 交于点 $A 、 B$, 且点 $B$ 在线段 $A X$ 的内部, 直线 $X T$ 与 $\odot O_1$ 交于另一点 $S, C$ 是不包含点 $A 、 B$ 的 $\overparen{T S}$ 上的一点, 过点 $C$ 作 $\odot O_2$ 的切线, 切点为 $Y$, 且线段 $C Y$ 与线段 $S T$ 不相交, 直线 $S C$ 与 $X Y$ 交于点 $I$. 证明: (1) $C 、 T 、 I 、 Y$ 四点共圆; (2) $I$ 是 $\triangle A B C$ 的 $\angle A$ 内的旁切圆的圆心.
%%<SOLUTION>%%
证明:(1) 如图(<FilePath:./figures/fig-c4i13.png>), $T$ 为 $\odot O_1$ 与 $\odot O_2$ 的公切点, 则 $\overparen{S T}$ 对应的度数等于 $\overparen{X T}$ 对应的度数, 即 $\angle S A T=\angle X Y T$ <1>.
而 $\angle T C I=\angle S A T$, 所以 $\angle T C I=\angle X Y T$, 即 $\angle I Y T=\angle I C T$.
故 $C 、 I 、 T 、 Y$ 四点共圆 <2>. 
(2) 因为 $\angle S A T=\angle S B T, \angle A X S=\angle X Y T$, 结合 <1> 式得
$$
\angle S A T=\angle A X S, \angle S B T=\angle A X S .
$$
又 $\angle A S T=\angle X S A, \angle B S T=\angle B S X$, 所以 $\triangle S A T \backsim \triangle S X A, \triangle S B T \backsim \triangle S X B$.
则 $S A^2=S T \cdot S X, S B^2=S T \cdot S X$.
又 $\angle S I T=\angle C Y T$ (由 <2> 得), $\angle S I T=\angle S X I$, 而 $\angle I S T=\angle X S I$, 所以 $\triangle S I T \backsim \triangle S X I$, 从而 $S I^2=S T \cdot S X$.
则 $S A=S I=S B$, 即 $S$ 为 $\triangle A B I$ 的外心.
所以 $\angle X B I=\frac{1}{2} \angle A S I=\frac{1}{2} \angle A S C=\frac{1}{2} \angle C B X$.
故 $B I$ 平分 $\angle C B X$, 而 $\angle B C I=\angle B A S$, 又 $S$ 为 $\triangle A B I$ 的外心, 则 $\angle B A S=-\frac{1}{2} \angle A S B+90^{\circ}$, 则 $2 \angle B A S+\angle A S B-90^{\circ}$.
又 $\angle B A S=\angle B C I, \angle A S B=\angle A C B$, 即 $2 \angle B C I+\angle A C B=90^{\circ}$.
所以 $C I$ 是 $\angle A C B$ 的外角平分线, 又 $B I$ 平分 $\angle C B X$.
故 $I$ 为 $\triangle A B C$ 的 $\angle A$ 内的旁切圆圆心.
证毕.
%%PROBLEM_END%%


