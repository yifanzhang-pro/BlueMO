
%%TEXT_BEGIN%%
圆内接四边形中的不等式.
圆内接四边形不仅有着丰富的几何等量关系, 也有很多有趣的极值性质.
因为圆内接四边形的边可用对应的圆心角的三角函数表示, 这就使得三角方法在处理圆内接四边形的几何不等式时能派上用场.
下面就是这样的一个例子.
%%TEXT_END%%



%%TEXT_BEGIN%%
圆内接四边形有一个著名的极值性质: 四条边给定的四边形中, 内接于圆的四边形面积最大.
一个给定边长的圆内接四边形的面积有很好的解析公式, 这就是下面的定理.
定理设一个圆的内接凸四边形的边长依次为 $a 、 b 、 c 、 d$, 又设 $s$ 为该四边形周长的一半, 则四边形的面积 $F$ 为
$$
F=\sqrt{(s-a)(s-b)(s-c)(s-d)} .
$$
这个定理是三角形熟知结果的推广.
如果取 $d=0$, 我们便得到通常的三角形面积的海伦公式.
证明设定理中的四边形为 $A B C D$, 如图(<FilePath:./figures/fig-c3i4.png>)  $A B=a, B C=b, C D=c, D A=d$.
如果 $A B C D$ 是长方形, 证明立即得出.
如果 $A B C D$ 不是长方形, 设 $B C$ 与 $A D$ 相交于圆外的点 $E$. 记 $C E=x, D E=y$, 则由三角形面积公式有
$$
S_{\triangle C D E}=\frac{1}{4} \sqrt{(x+y+c)(x+y-c)(x-y+c)(-x+y+c)} . \label{eq1}
$$
但 $\triangle A B E \backsim \triangle C D E$, 所以
$$
\frac{S_{\triangle A B E}}{S_{\triangle C D E}}=\frac{a^2}{c^2}
$$
由此推得
$$
\frac{F}{S_{\triangle C D E}}=\frac{c^2-a^2}{c^2}, \label{eq2}
$$
又由比例式
$$
\frac{x}{c}=\frac{y-d}{a}, \frac{y}{c}=\frac{x-\underline{b}}{a},
$$
相加解出 $x+y$, 得
$$
x+y+c=\frac{c}{c-a}(-a+b+c+d),
$$
$x+y-c$ 等等的类似表达式, 都可以立即得到.
将它们代入式\ref{eq1}并化简, 得
$$
S_{\triangle C D E}=\frac{c^2}{c^2-a^2} \sqrt{(s-a)(s-b)(s-c)(s-d)},
$$
代入式\ref{eq2}, 便知结论成立.
推广可以证明任一边长为 $a 、 b 、 c 、 d$,一对对角的和为 $2 u$ 的凸四边形, 面积 $F$ 可由下式给出
$$
F^2=(s-a)(s-b)(s-c)(s-d)-a b c d \cos ^2 u .
$$
由于证明包含长而乏味的三角化简, 我们不在这里给出.
由这个公式立即看出 : 四条边给定的四边形中, 内接于圆的面积最大.
%%TEXT_END%%



%%PROBLEM_BEGIN%%
%%<PROBLEM>%%
例1. 已知四边形 $A B C D$ 是圆的内接四边形,证明:
$$
|A B-C D|+|A D-B C| \geqslant 2|A C-B D| .
$$
%%<SOLUTION>%%
证明:如图(<FilePath:./figures/fig-c3i1.png>), 设四边形 $A B C D$ 外接圆的圆心为 $O$, 该外接圆的半径为 $1, \angle A O B=2 \alpha, \angle B O C=2 \beta$, $\angle C O D=2 \gamma, \angle D O A=2 \delta$, 则
$$
\alpha+\beta+\gamma+\delta=\pi \text {. }
$$
不妨设 $\alpha \geqslant \gamma, \beta \geqslant \delta$, 则
$$
\begin{aligned}
& |A B-C D|=2|\sin \alpha-\sin \gamma| \\
= & 4\left|\sin \frac{\alpha-\gamma}{2} \cos \frac{\alpha+\gamma}{2}\right|=4\left|\sin \frac{\alpha-\gamma}{2} \sin \frac{\beta+\delta}{2}\right| .
\end{aligned}
$$
同理
$$
\begin{aligned}
& |A D-B C|=4\left|\sin \frac{\beta-\delta}{2} \sin \frac{\alpha+\gamma}{2}\right| \\
& |A C-B D|=4\left|\sin \frac{\beta-\delta}{2} \sin \frac{\alpha-\gamma}{2}\right|
\end{aligned}
$$
因此
$$
\begin{aligned}
|A B-C D|-|A C-B D| & =4\left|\sin \frac{\alpha-\gamma}{2}\right|\left(\left|\sin \frac{\beta+\delta}{2}\right|-\left|\sin \frac{\beta-\delta}{2}\right|\right) \\
& =4\left|\sin \frac{\alpha-\gamma}{2}\right|\left(\sin \frac{\beta+\delta}{2}-\sin \frac{\beta-\delta}{2}\right)
\end{aligned}
$$
$$
\begin{aligned}
& =4\left|\sin \frac{\alpha-\gamma}{2}\right| \cdot\left(2 \cos \frac{\beta}{2} \cdot \sin \frac{\delta}{2}\right) \\
& \geqslant 0 .
\end{aligned}
$$
故
$$
|A B-C D| \geqslant|A C-B D| \text {. }
$$
同理
$$
|A D-B C| \geqslant|A C-B D| \text {. }
$$
将这两个不等式相加即得求证结果.
圆内接四边形中又有一种更特殊的四边形叫双圆四边形.
所谓双圆四边形是既有外接圆又有内切圆的四边形.
%%PROBLEM_END%%



%%PROBLEM_BEGIN%%
%%<PROBLEM>%%
例2. 凸四边形 $A B C D$ 既有内切圆又有外接圆,已知它的外接圆半径为 $R$, 面积为 $S$, 四边形的边长分别为 $a 、 b 、 c 、 d$, 证明:
$$
a b c+a b d+a c d+b c d \leqslant 2 \sqrt{S}\left(S+2 R^2\right) . \label{eq1}
$$
%%<SOLUTION>%%
证明:1 如图(<FilePath:./figures/fig-c3i2.png>), 设四边形 $A B C D$ 的外接圆和内切圆的圆心分别为 $O$ 和 $I$. 内切圆与边 $A B=a 、 B C=b 、 C D=c 、 D A=d$ 的切点分别为 $K 、 L 、 M 、 N$. 设 $\angle A I N=\angle 1, \angle B I K=\angle 2, \angle C I L=\angle 3$, $\angle D I M=\angle 4$, 并记 $A K=A N=a^{\prime}, B L=B K=b^{\prime}$, $C L=C M=c^{\prime}, D M=D N=d^{\prime}$.
不妨设内切圆 $I$ 的半径 $r$ 为 1 . 由 $A B C D$ 有内切圆知
$$
a+c=b+d \text {. }
$$
这时若记式\ref{eq1}的左边为 $H$, 则
$$
H=(a+c) b d+(b+d) a c=\frac{1}{2}(a+b+c+d)(a c+b d) . \label{eq2}
$$
又 $\quad a=a^{\prime}+b^{\prime}, b=b^{\prime}+c^{\prime}, c=c^{\prime}+d^{\prime}, d=d^{\prime}+a^{\prime}$,
将这些表达式代入式\ref{eq2}的右边便得
$$
H=\left(a^{\prime}+b^{\prime}+c^{\prime}+d^{\prime}\right)\left[\left(a^{\prime}+b^{\prime}\right)\left(c^{\prime}+d^{\prime}\right)+\left(b^{\prime}+c^{\prime}\right)\left(d^{\prime}+a^{\prime}\right)\right] . \label{eq3}
$$
又由 $\angle A+\angle C=180^{\circ}$, 可得
$$
\angle 1+\angle 3=90^{\circ} .
$$
因此 $\triangle A I N \backsim \triangle I C L$, 由此得
$$
a^{\prime} c^{\prime}=A N \cdot C L=N I \cdot I L=1 . \label{eq4}
$$
同理
$$
b^{\prime} d^{\prime}=1 . \label{eq5}
$$
又注意到
$$
S=\frac{a+b+c+d}{2} \cdot r=a^{\prime}+b^{\prime}+c^{\prime}+d^{\prime}, \label{eq6}
$$
因此由式\ref{eq4}、\ref{eq5}、式\ref{eq6}可得
$$
H=S\left[4+\left(a^{\prime}+c^{\prime}\right)\left(b^{\prime}+d^{\prime}\right)\right] . \label{eq7}
$$
另一方面, 由正弦定理并注意到 $\angle B+2 \angle 2=180^{\circ}$, 有
$$
\begin{aligned}
R & =\frac{A C}{2 \sin \angle B}=\frac{A C}{2 \sin 2 \angle 2}=\frac{A C}{4}\left(\tan \angle 2+\frac{1}{\tan \angle 2}\right) \\
& =\frac{1}{4} A C(\tan \angle 2+\tan \angle 4) \\
& =\frac{1}{4} A C \cdot\left(b^{\prime}+d^{\prime}\right) .
\end{aligned}
$$
同理
$$
R=\frac{1}{4} B D \cdot\left(a^{\prime}+c^{\prime}\right) .
$$
因此
$$
R^2=\frac{1}{16} A C \cdot B D\left(a^{\prime}+c^{\prime}\right)\left(b^{\prime}+d^{\prime}\right),
$$
但是
$$
S=\frac{1}{2} A C \cdot B D \cdot \sin \alpha \leqslant \frac{1}{2} A C \cdot B D,
$$
这里 $\alpha$ 为对角线 $A C$ 与 $B D$ 夹角.
因此
$$
R^2 \geqslant \frac{1}{8} S\left(a^{\prime}+c^{\prime}\right)\left(b^{\prime}+d^{\prime}\right) .
$$
由上可知
$$
\text { 式\ref{eq1} } \begin{aligned}
\text { 的右边 } & \geqslant 2 \sqrt{S}\left(S+\frac{S}{4}\left(a^{\prime}+c^{\prime}\right)\left(b^{\prime}+d^{\prime}\right)\right) \\
& =\frac{S^{\frac{3}{2}}}{2}\left[4+\left(a^{\prime}+c^{\prime}\right)\left(b^{\prime}+d^{\prime}\right)\right] . \label{eq8}
\end{aligned}
$$
因此由式\ref{eq7}、\ref{eq8}知, 要证式\ref{eq1}, 只需证明 $\frac{1}{2} S^{\frac{1}{2}} \geqslant 1$, 这等价于
$$
\sqrt{a^{\prime}+b^{\prime}+c^{\prime}+d^{\prime}} \geqslant 2 . \label{eq9}
$$
而由 $a^{\prime} c^{\prime}=1, b^{\prime} d^{\prime}=1$ 可知
$$
a^{\prime}+b^{\prime}+c^{\prime}+d^{\prime} \geqslant 2 \sqrt{a^{\prime} c^{\prime}}+2 \sqrt{b^{\prime} d^{\prime}}=4,
$$
式\ref{eq9}得证.
%%PROBLEM_END%%



%%PROBLEM_BEGIN%%
%%<PROBLEM>%%
例2. 凸四边形 $A B C D$ 既有内切圆又有外接圆,已知它的外接圆半径为 $R$, 面积为 $S$, 四边形的边长分别为 $a 、 b 、 c 、 d$, 证明:
$$
a b c+a b d+a c d+b c d \leqslant 2 \sqrt{S}\left(S+2 R^2\right) . \label{eq1}
$$
%%<SOLUTION>%%
证明 2 先证引理.
引理设 $\triangle A B C$ 中, $\angle A \geqslant 90^{\circ}$, 则 $\frac{b+c}{a} \leqslant \sqrt{2}$.
证明
$$
\begin{aligned}
\frac{b+c}{a} & =\frac{\sin B+\sin C}{\sin A}=2 \frac{\sin \frac{B+C}{2} \cos \frac{B-C}{2}}{\sin A} \\
& \leqslant \frac{2 \cos \frac{A}{2}}{2 \sin \frac{A}{2} \cos \frac{A}{2}}=\frac{1}{\sin \frac{A}{2}} \leqslant \sqrt{2} .
\end{aligned}
$$
下面证明原题中的不等式.
如图(<FilePath:./figures/fig-c3i3.png>), 设四边形 $A B C D$ 的四边 $A B 、 B C 、 C D$ 、 $D A$ 的长分别为 $a 、 b 、 c 、 d$, 再设 $A B C D$ 的内切圆的半径为 1 . 注意到
$$
a+c=b+d=\frac{1}{2}(a+b+c+d) \cdot 1=S,
$$
可得
$$
\begin{aligned}
H & =a b c+a b d+a c d+b c d \\
& =a c(b+d)+b d(a+c)=(a c+b d) S . \label{eq1}
\end{aligned}
$$
现设 $A C$ 的中垂线为 $l, D$ 关于 $l$ 的对称点为 $E$, 则
$$
\triangle A C D \cong \triangle C A E .
$$
因此 $A E=c, C E=d$, 且 $\angle E=\angle D=\pi-\angle B$, 由此可知 $A 、 E 、 C 、 B$ 四点共圆, 故
$$
S=\frac{1}{2}(a c+b d) \sin \alpha, \label{eq2}
$$
其中 $\alpha=\angle E A B$.
由式\ref{eq1}、\ref{eq2}知原不等式等价于
$$
\frac{2 S}{\sin \alpha} \cdot S \leqslant 2 \sqrt{S}\left(S+2 R^2\right) . \label{eq3}
$$
注意到 $R=\frac{B E}{2 \sin \alpha}$, 因此式\ref{eq3}进一步等价于
$$
S^{\frac{3}{2}} \leqslant S \sin \alpha+\frac{B E^2}{2 \sin \alpha}, \label{eq4}
$$
但由平均值不等式
\ref{eq4} 的右端 $\geqslant 2 \sqrt{\frac{S \cdot B E^2}{2}}$.
因此要证式\ref{eq1}, 只需证明
$$
\sqrt{2} B E \geqslant S . \label{eq5}
$$
事实上, 由 $\angle E A B+\angle E C B=180^{\circ}$, 不妨设 $\angle E A B \geqslant 90^{\circ}$. 对 $\triangle A B E$ 应用引理可知 $\frac{a+c}{B E} \leqslant \sqrt{2}$, 故
$$
\sqrt{2} B E \geqslant a+c=S,
$$
\ref{eq1}式得证.
上面的证法综合运用三角、几何的技巧,构造了一个新的共圆四边形,实现了问题的转化.
%%PROBLEM_END%%



%%PROBLEM_BEGIN%%
%%<PROBLEM>%%
例3. (Popa 不等式) 如果一个凸四边形的四边满足 $a \leqslant b \leqslant c \leqslant d$, 面积为 $F$, 求证:
$$
F \leqslant \frac{3 \sqrt{3}}{4} c^2 . \label{eq1}
$$
%%<SOLUTION>%%
证明:由于边长给定的四边形中, 圆内接四边形的面积最大, 因此我们仅需对圆内接四边形证明式\ref{eq1}便可.
这时
$$
F^2=(s-a)(s-b)(s-c)(s-d),
$$
其中 $s=\frac{1}{2}(a+b+c+d)$. 但是 $s-d=(a+b+c)-s$, 因此由算术几何平均值不等式可得
$$
\begin{aligned}
F^2 & =3^3\left(\frac{1}{3} s-\frac{1}{3} a\right)\left(\frac{1}{3} s-\frac{1}{3} b\right)\left(\frac{1}{3} s-\frac{1}{3} c\right)(a+b+c-s) \\
& \leqslant 3^3\left[\frac{\left(\frac{1}{3} s-\frac{1}{3} a\right)+\left(\frac{1}{3} s-\frac{1}{3} b\right)+\left(\frac{1}{3} s-\frac{1}{3} c\right)+(a+b+c-s)}{4}\right]^4 \\
& =3^3\left(\frac{a+b+c}{3 \cdot 2}\right)^4 \leqslant 3^3\left(\frac{c}{2}\right)^4 .
\end{aligned}
$$
最后一步用了 $a \leqslant b \leqslant c$.
两边开方, 由此便得
$$
F \leqslant \frac{3 \sqrt{3}}{4} c^2
$$
得证.
%%PROBLEM_END%%



%%PROBLEM_BEGIN%%
%%<PROBLEM>%%
例4. (高灵不等式)设凸四边形 $A B C D$ 和 $A^{\prime} B^{\prime} C^{\prime} D^{\prime}$ 的四边分别为 $a$ 、 $b 、 c 、 d$ 和 $a^{\prime} 、 b^{\prime} 、 c^{\prime} 、 d^{\prime}$, 它们的面积分别为 $F 、 F^{\prime}$. 令
$$
K=4(a d+b c)\left(a^{\prime} d^{\prime}+b^{\prime} c^{\prime}\right)-\left(a^2-b^2-c^2+d^2\right)\left(a^{\prime 2}-b^{\prime 2}-c^{\prime 2}+d^{\prime 2}\right) \text {. }
$$
求证:
$$
K \geqslant 16 F F^{\prime} \text {. }
$$
%%<SOLUTION>%%
证明:由于给定边长的四边形以圆的内接四边形具有最大面积, 因此仅需考虑 $A B C D$ 和 $A^{\prime} B^{\prime} C^{\prime} D^{\prime}$ 均为圆内接四边形的情况.
如图(<FilePath:./figures/fig-c3i5.png>), 因为 $\angle B+\angle D=180^{\circ}$, 所以
$$
2 F=(a d+b c) \sin B , \label{eq1}
$$
类似的有
$$
2 F^{\prime}=\left(a^{\prime} d^{\prime}+b^{\prime} c^{\prime}\right) \sin B^{\prime} . \label{eq2}
$$
另一方面, 由余弦定理
$$
\begin{aligned}
A C^2 & =b^2+c^2+2 b c \cos B \\
& =a^2+d^2-2 a d \cos B,
\end{aligned}
$$
因此 $\quad a^2-b^2-c^2+d^2=2(a d+b c) \cos B, \label{eq3}$,
类似的有 $\quad a^{\prime 2}-b^{\prime 2}-c^{\prime 2}+d^{\prime 2}=2\left(a^{\prime} d^{\prime}+b^{\prime} c^{\prime}\right) \cos B^{\prime}, \label{eq4}$.
由式\ref{eq1}、\ref{eq4}可得
$$
K-16 F F^{\prime}=4(a d+b c)\left(a^{\prime} d^{\prime}+b^{\prime} c^{\prime}\right)\left(1-\cos \left(B-B^{\prime}\right)\right) \geqslant 0,
$$
故原不等式成立.
%%<REMARK>%%
注:由上面的证法, 实际上可把高灵不等式写的更一般一些:
$$
0 \leqslant K-16 F F^{\prime} \leqslant 8(a d+b c)\left(a^{\prime} d^{\prime}+b^{\prime} c^{\prime}\right),
$$
左边的不等式即为高灵不等式.
高灵不等式可看作是著名的 Neuberg-Pedoe 不等式在四边形中的推广.
在这一节的最后, 我们研究一个难度较大的关于双圆四边形的极值问题, 这里要介绍的解法由向振同学 (原长沙市一中学生, 曾获 2003 年第 44 届 $\mathrm{IMO}$ 金牌)给出.
%%PROBLEM_END%%



%%PROBLEM_BEGIN%%
%%<PROBLEM>%%
例5. 给定外接圆半径 $R$ 和面积 $S$ 不变的双圆四边形 $A B C D$ (这里 $S \leqslant 2 R^2$ ), 求 $p l m$ 的最大值, 其中 $p$ 是四边形 $A B C D$ 的半周长, $l 、 m$ 分别为它的两条对角线长.
%%<SOLUTION>%%
解:如图(<FilePath:./figures/fig-c3i6.png>), 我们可以用三个参数 $r 、 \alpha 、 \beta(\gamma \in(0,+\infty), \alpha 、 \beta \in \left.\left(0, \frac{\pi}{2}\right)\right)$ 来确定一个双圆四边形 $A B C D$, 这里的 $r$ 是四边形 $A B C D$ 的内切圆半径, $\alpha=\angle A I K, \beta=\angle B I K$, 其中 $I$ 是四边形 $A B C D$ 的内切圆的圆心, $K$ 是圆 $I$ 与边 $A B$ 的切点.
下面证明:
$$
\begin{gathered}
S=r^2\left(\frac{2}{\sin 2 \alpha}+\frac{2}{\sin 2 \beta}\right), \label{eq1}\\
R^2=r^2\left(\frac{1}{\sin 2 \alpha \sin 2 \beta}+\frac{1}{\sin ^2 2 \alpha \sin ^2 2 \beta}\right) . \label{eq2}
\end{gathered}
$$
先证式\ref{eq1}. 因为半周长
$$
p=r(\tan \alpha+\cot \alpha+\tan \beta+\cot \beta)=r\left(\frac{2}{\sin 2 \alpha}+\frac{2}{\sin 2 \beta}\right),
$$
所以
$$
S=r p=r^2\left(\frac{2}{\sin 2 \alpha}+\frac{2}{\sin 2 \beta}\right)
$$
这就是式\ref{eq1}.
再证式\ref{eq2}. 在 $\triangle A B D$ 中易知
$$
A B=r(\tan \alpha+\tan \beta), A D=r(\tan \alpha+\cot \beta), \angle D A B=\pi-2 \alpha,
$$
故由余弦定理并通过三角化简可得
$$
\begin{aligned}
B D^2= & r^2\left[(\tan \alpha+\tan \beta)^2+(\tan \alpha+\cot \beta)^2+\right. \\
& 2 \cos 2 \alpha(\tan \alpha+\tan \beta)(\tan \alpha+\cot \beta)] \\
= & r^2\left(\tan \alpha \cdot \frac{2}{\sin 2 \beta} \cdot 4 \cos ^2 \alpha+\frac{4}{\sin ^2 2 \beta}\right),
\end{aligned}
$$
故
$$
R^2=\frac{B D^2}{4 \sin ^2 2 \alpha}=r^2\left(\frac{1}{\sin 2 \alpha \sin 2 \beta}+\frac{1}{\sin ^2 2 \alpha \sin ^2 2 \beta}\right),
$$
这就是式\ref{eq1}.
现令 $a=\sin 2 \alpha, b=\sin 2 \beta$, 则 $a, b \in(0,1]$, 且(1)、(2)可写为
$$
\begin{aligned}
S & =2 r^2 \frac{a+b}{a b}, \label{eq3}\\
R^2 & =r^2 \frac{1+a b}{a^2 b^2}, \label{eq4}
\end{aligned}
$$
式\ref{eq3}除以\ref{eq4}可得
$$
\frac{a b(a+b)}{1+a b}=\frac{S}{2 R^2} . \label{eq5}
$$
式\ref{eq5}是 $a 、 b$ 所满足的约束条件, 我们在此条件下求 $p l m$ 的最大值.
易知
$$
\begin{gathered}
p=r \cdot\left(\frac{2}{a}+\frac{2}{b}\right), \\
l m=4 R^2 a b,
\end{gathered}
$$
所以 $(p l m)^2=64 R^4 r^2(a+b)^2=16 R^2 S^2(1+a b)$,
因此
$$
p l m=4 R S \sqrt{1+a b} . \label{eq6}
$$
由式\ref{eq5}得
$$
\frac{S}{2 R^2}=\frac{a b(a+b)}{1+a b} \geqslant \frac{a b \cdot 2 \sqrt{a b}}{1+a b} . \label{eq7}
$$
令 $\sqrt{a b}=x$, 则 $x \in(0,1]$, 于是式\ref{eq7}可写为
$$
4 R^2 \cdot x^3-S \cdot x^2-S \leqslant 0 . \label{eq8}
$$
令函数 $f(x)=4 R^2 \cdot x^3-S x^2-S$, 注意到
$$
f(0)=-S<0, f(1)=4 R^2-2 S \geqslant 0,
$$
且
$$
f^{\prime}(x)= \begin{cases}\geqslant 0, & x \geqslant \frac{S}{6 R^2}, \\ <0, & 0<x<\frac{S}{6 R^2},\end{cases}
$$
故 $f(x)$ 在 $(0,1]$ 上先递减再递增.
由此知 $f(x)$ 在 $(0,1)$ 上有唯一的实根 $t$, 如图(<FilePath:./figures/fig-c3i7.png>) 所示.
由式\ref{eq8}知 $f(\sqrt{a b}) \leqslant 0$, 因此
$$
\sqrt{a b} \leqslant t,
$$
于是 $a b \leqslant t^2$, 从而代入式\ref{eq6}知
$$
p l m=4 R S \sqrt{1+a b} \leqslant 4 R S \sqrt{1+t^2},
$$
当 $a=b=t$ 时, 等号成立.
故所求的 $p l m$ 的最大值为 $4 R S \sqrt{1+t^2}$, 其中 $t$ 是方程 $4 R^2 x^3-S x^2-S=0$ 在区间 $(0,1]$ 上的根.
%%PROBLEM_END%%


