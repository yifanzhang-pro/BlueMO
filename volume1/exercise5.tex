
%%PROBLEM_BEGIN%%
%%<PROBLEM>%%
问题1 在集合 $M=\{1,2, \cdots, 10\}$ 的所有子集中,有这样一族不同的子集,它们两两的交集都不是空集,那么这族子集最多有?个.
%%<SOLUTION>%%
$2^9$. 显然不含空集.
按所含元素的多少把这族子集分为 10 类, 设含 $i$ 个元素的子集为 $A_i$, 其个数为 $a_i$. 易知 $a_1+a_2+\cdots+a_{10} \leqslant \mathrm{C}_9^0+\mathrm{C}_9^1+\cdots+\mathrm{C}_9^9=2^9$.
%%PROBLEM_END%%



%%PROBLEM_BEGIN%%
%%<PROBLEM>%%
问题2 设集合 $M=\{1,2, \cdots, 1000\}$, 现对 $M$ 的任一非空子集 $X$, 令 $\alpha_X$ 表示 $X$ 中最大数与最小数之和.
则所有这样的 $\alpha_X$ 的算术平均值为
%%<SOLUTION>%%
1001 . 构造子集 $X^{\prime}=\{1001-x \mid x \in X\}$, 则所有非空子集分成两类 $X^{\prime}=X$ 和 $X^{\prime} \neq X$. 当 $X^{\prime}=X$ 时, 必有 $X^{\prime}=X=M$, 于是 $\alpha_X=1001$. 当 $X^{\prime} \neq X$ 时, 设 $x 、 y$ 分别是 $X$ 中的最大数与最小数, 则 $1001-x 、 1001-y$ 分别是 $X^{\prime}$ 中的最小数与最大数.
于是, $\alpha_X=x+y, \alpha_{X^{\prime}}=2002-x-y$. 从而, $\frac{\alpha_X+\alpha_{X^{\prime}}}{2}=1001$. 因此, 所求的 $\alpha_X$ 的算术平均值为 1001 .
%%PROBLEM_END%%



%%PROBLEM_BEGIN%%
%%<PROBLEM>%%
问题3 对于 $\{1,2, \cdots, n\}$ 和它的每个非空子集, 我们定义“交替和”如下:把子集中的数按从大到小的顺序排列, 然后从最大的数开始交替地加减各数 (例如, $\{1,2,4,6,9\}$ 的交替和是 $9-6+4-2+1=6$, 而 $\{5\}$ 的交替和就是 5). 对于 $n=7$; 求所有这些交替和的总和.
%%<SOLUTION>%%
集合 $\{1,2,3,4,5,6,7\}$ 中每个元素在子集中均出现 $2^6=64$ 次.
可计算 $1,2,3,4,5,6$ 在子集中按从大到小的顺序排列时各有 32 次在奇数位, 32 次在偶数位,因此子集中这些数的交替和的总和为 0 ; 而 7 也出现 64 次, 且均取正值, 所以所有子集的交替和的总和为 $7 \times 64=448$.
%%PROBLEM_END%%



%%PROBLEM_BEGIN%%
%%<PROBLEM>%%
问题4 $X=\{1,2, \cdots, 2 n+1\} . A$ 是 $X$ 的子集, 且具有性质: $A$ 中任意两个数的和不在 $A$ 中, 求 $\max |A|$.
%%<SOLUTION>%%
取 $A=\{1,3,5, \cdots, 2 n+1\}$ 合乎要求.
故 $\max |A| \geqslant n+1$. 另一方面可设合乎要求的 $A$ 中有 $k(k \leqslant n+1)$ 个奇数: $a_1>a_2>\cdots>a_k$. 眇然,偶数 $a_1-a_2<a_1-a_3<\cdots<a_1-a_k$ 都不能在 $A$ 中, 所以 $A$ 中至多有 $n-(k-1) =n+1-k$ 个偶数.
从而, $\max |A| \leqslant k+(n+1-k)=n+1$.
%%PROBLEM_END%%



%%PROBLEM_BEGIN%%
%%<PROBLEM>%%
问题5 在集合 $\{1,2, \cdots, n\}$ 中, 任意取出一个子集, 计算它的元素之和.
问所有子集元素之和的总和是多少?
%%<SOLUTION>%%
$\{1,2, \cdots, n\}$ 中含 $k(1 \leqslant k \leqslant n)$ 的子集有 $2^{n-1}$ 个, 故其总和为 $2^{n-1}(1+2+\cdots+n)=2^{n-2} n(n+1)$.
%%PROBLEM_END%%



%%PROBLEM_BEGIN%%
%%<PROBLEM>%%
问题6 如果一个正整数集合中没有 3 个数是两两互质的, 则称之为“异质”的.
问从 1 到 16 的整数集合中 “异质”的子集合的元素的最大数目是多少?
%%<SOLUTION>%%
“异质”子集至多有 2 个数取自 $\{1,2,3,5,7,11,13\}$, 因而至多有 $16-7+2=11$ 个数字, 如 $\{2,3,4,6,8,9,10,12,14,15,16\}$.
%%PROBLEM_END%%



%%PROBLEM_BEGIN%%
%%<PROBLEM>%%
问题7 设 $S$ 为 $\{1,2, \cdots, 9\}$ 的子集, 且 $S$ 中任意两个不同的数作和, 所得的数两两不同,问: $S$ 中最多有多少个元素?
%%<SOLUTION>%%
容易验证: 当 $S=\{1,2,3,5,8\}$ 时符合题中要求.
如果 $T \subseteq\{1$, $2, \cdots, 9\},\{T\} \geqslant 6, T$ 中任意两个不同的数之和介于 3 与 17 之间, 故至多可以形成 15 个不同的和数.
而 $T$ 中任取两个数, 有至少 $\mathrm{C}_6^2=15$ 种取法.
如果 $T$ 满足条件, 则所得和数中必须 3 与 17 同时出现, 即 $1 、 2 、 8 、 9$ 都在 $T$ 中出现, 但这时 $1+9=2+8, T$ 不合题意.
%%PROBLEM_END%%



%%PROBLEM_BEGIN%%
%%<PROBLEM>%%
问题8 设 $r(r \geqslant 2)$ 是一个固定的正整数, $F$ 是一个无限集合族, 且每个集合中有 $r$ 个元素.
若 $F$ 中任意两个集合的交集非空, 证明: 存在--个具有 $r^{--1}$ 个元素的集合与 $F$ 中的每一个集合的交集均非空.
%%<SOLUTION>%%
可以证明命题: 如果集合 $A$ 的元素个数小于 $r$, 且包含于 $F$ 的无穷多个集合中, 则要么 $A$ 与 $F$ 中所有集合的交集非空, 要么存在一个 $x \notin A$, 使得 $A \cup\{x\}$, 包含于 $F$ 的无穷多个集合中.
当然,这样的集合 $A$ (如空集) 是存在的.
重复运用这个命题 $r$ 次, 即得所要证明的结论.
因为具有 $r$ 个元素的集合不可能包含于 $F$ 的无穷多个集合中.
假设 $F$ 中的某个集合 $R=\left\{x_1, x_2, \cdots, x_r\right\}$ 与集合 $A$ 的交集是空集, 由于 $F$ 中有无穷多个集合包含 $A$, 且每一个集合与 $R$ 的交集非空, 于是, 存在某个 $x_i$ 属于无穷多个集合.
设 $x=x_i$, 则 $A \cup\{x\}$ 包含于 $F$ 的无穷多个集合中.
%%PROBLEM_END%%



%%PROBLEM_BEGIN%%
%%<PROBLEM>%%
问题9 设集合 $S=\{1,2, \cdots, 50\}$. 试求最小正整数 $n$, 使得 $S$ 中的每个 $n$ 元子集中都有 3 个数能作为直角三角形的三边长.
%%<SOLUTION>%%
引理如果正整数 $x, y, z$ 满足方程(1) $x^2+y^2=z^2$, 则 3 个数中至少有 1 个数是 5 的倍数.
这是因为 $(5 k+1)^2=25 k^2+10 k+1 \equiv 1(\bmod 5),(5 k+ 2)^2=25 k^2+20 k+4 \equiv-1(\bmod 5),(5 k+3)^2=25 k^2+30 k+9 \equiv-1(\bmod 5), (5 k+4)^2=25 k^2+40 k+16 \equiv 1(\bmod 5)$, 所以,如果 $x$ 和 $y$ 都不是 5 的倍数, 则 $x^2$ 和 $y^2$ 都模 5 等于 1 或 -1 . 从而 $z^2$ 只能模 5 等于 0 , 因此, $z$ 是 5 的倍数.
考察以 $10,15,25,40,45$ 分别作为直角三角形 1 条边长的所有勾股数组.
因为方程 (1)的正整数解可以表为 (2): $x=k\left(a^2-b^2\right), y=2 k a b, z=k\left(a^2+b^2\right)$, 其中 $k, a, b \in \mathbf{N}$, 且 $(a, b)=1, a>b$, 故知这样的勾股数共有下列 11 组: $(10,8,6),(26,24,10),(15,12,9),(17,15,8),(39,36,15),(25,24,7),(40,32,24),(41,40,9),(45,36,27),(25,20,15),(50,40,30)$. 注意到前 9 组勾股数中每组都有 $8 、 9 、 24 、 36$ 这 4 个数之一, 可知集合 $M=S-\{5,8,9,20,24,30,35,36,50\}$ 中任何 3 个数都不是一组勾股数.
所以,所求的最小正整数 $n \geqslant 42$.
另一方面, 在下列 9 组勾股数 $(3,4,5),(7,24,25),(8,15,17),(9, 40,41),(12,35,37),(14,48,50),(16,30,34),(20,21,29),(27, 36,45)$ 中出现的 27 个数互不相同.
故对 $S$ 的任一个 42 元子集 $M$, 不在 $M$ 中的 8 个数至多属于其中的 8 组, 从而至少有 1 组勾股数全在 $M$ 中.
故所求的最小正整数 $n=42$.
%%PROBLEM_END%%



%%PROBLEM_BEGIN%%
%%<PROBLEM>%%
问题10 设 $P$ 是一个奇质数, 考虑集合 $\{1,2, \cdots, 2 p\}$ 满足以下两个条件的子集 $A$ :
(i) $A$ 恰有 $p$ 个元素;
(ii) $A$ 中所有元素之和可被 $p$ 整除.
试求所有这样的子集 $A$ 的个数.
%%<SOLUTION>%%
记 $U=\{1,2, \cdots, p\}, V=\{p+1, p+2, \cdots, 2 p\}, W=\{1,2, \cdots, 2 p\}$, 除去 $U$ 和 $V$ 而外, $W$ 的所有其他的 $p$ 元子集 $E$ 都使得 $E \cap U \neq \varnothing$, $E \cap V \neq \varnothing$. 若 $W$ 的两个这样的 $p$ 元子集 $S$ 和 $T$ 同时满足: $S \cap V=T \cap V$; 只要编号适当, $S \cap U$ 的元素 $s_1, s_2, \cdots, s_m$ 和 $T \cap U$ 的元素 $t_1, t_2, \cdots, t_m$ 对适当的 $k \in\{0,1, \cdots, p-1\}$ 满足同余式组 $s_i-t_i \equiv k(\bmod p), i=1,2, \cdots, m$. 就约定将这两个子集 $S$ 和 $T$ 归人同一类.
对于同一类中的不同子集 $S$ 和 $T$, 显然有 $k \neq 0$, 因而 $\sum_{i=1}^p s_i-\sum_{i=1}^p t_i \equiv m k \not \neq 0(\bmod p)$. 对于同一类中的不同子集, 它们各自元素的和模 $p$ 的余数不相同.
因而每一类恰含 $p$ 个子集, 其中仅一个适合题目的条件(ii).
综上所述, 在 $W=\{1,2, \cdots, 2 p\}$ 的 $\mathrm{C}_{2 p}^p$ 个 $p$ 元子集当中, 除去 $U$ 和 $V$ 这两个特定子集外, 每 $p$ 个子集分成一类, 每类恰有一个子集满足题目的条件 (ii). 据此很容易算出, $W=\{1,2, \cdots, 2 p\}$ 的适合条件(i) 和(ii) 的子集的总数为 $\frac{1}{p}\left(\mathrm{C}_{2 p}^p-2\right)+2$.
%%PROBLEM_END%%



%%PROBLEM_BEGIN%%
%%<PROBLEM>%%
问题11 设 $n \in \mathbf{N}^*, n \geqslant 2, S$ 是一个 $n$ 元集合.
求最小的正整数 $k$, 使得存在 $S$ 的子集 $A_1, A_2, \cdots, A_k$ 具有如下性质:对 $S$ 中的任意两个不同元 $a 、 b$, 存在 $j \in\{1,2, \cdots, k\}$,使得 $A_j \bigcap\{a, b\}$ 为 $S$ 的一元子集.
%%<SOLUTION>%%
构造如右表格: 
\begin{tabular}{|c|c|c|c|c|c|}
\hline & 1 & 2 & 3 & $\cdots$ & $n$ \\
\hline$A_1$ & & & & & \\
\hline$A_2$ & & & & & \\
\hline$A_3$ & & & & & \\
\hline$\vdots$ & & & & & \\
\hline$A_k$ & & & & & \\
\hline & $P_1$ & $P_2$ & $P_3$ & $\cdots$ & $P_n$ \\
\hline
\end{tabular}
如果 $i \in A_j$, 那么在 $A_j$ 所在行、 $i$ 所在列处的方格中标上 1 , 其余的方格中标上 0 . 考虑表格中的列构成的序列 $P_1, P_2, \cdots, P_n$. 我们证明: $S$ 的子集 $A_1, A_2, \cdots, A_k$ 具有题中性质的充要条件是: $P_1, P_2, \cdots, P_n$ 两两不同.
若 $P_1, P_2, \cdots, P_n$ 两两不同,则对任意 $a, b \in S, a \neq b$, 有 $P_a \neq P_b$. 于是在某一行 (设为第 $j$行)上,第 $a$ 列与第 $b$ 列的方格中一个为 1 , 而另一个为 0 . 这表明 $A_j \cap\{a, b\}$ 为单元集, 故 $A_1, A_2, \cdots, A_k$ 具有题中性质.
由于对任意 $a, b \in S, a \neq b$, 存在 $j \in\{1,2, \cdots, k\}$, 使 $A_j \cap\{a, b\}$ 为单元素集, 故 $P_a$ 与 $P_b$ 在第 $j$ 行处的两个方格中的数一个为 1 , 而另一个为 0 , 故 $P_a \neq P_b$. 所以, $P_1, P_2, \cdots, P_n$ 两两不同.
由于由 $0 、 1$ 构成的 $k$ 元序列有且仅有 $2^k$ 个两两不同, 从而由抽屉原则及前面所证明的结论知 $2^k \geqslant n$. 所以, 所求的最小正整数 $k$ 为不小于 $\log _2 n$ 的最小正整数.
%%PROBLEM_END%%



%%PROBLEM_BEGIN%%
%%<PROBLEM>%%
问题12 设 $S=\{1,2, \cdots, 50\}$. 求最小自然数 $k$, 使 $S$ 的任一 $k$ 元子集中都存在两个不同的数 $a$ 和 $b$, 满足 $(a+b) \mid a b$.
%%<SOLUTION>%%
设有 $a 、 b \in S$ 满足 $(a+b) \mid a b$. 记 $c=(a, b)$, 于是 $a=c a_1, b=c b_1$, 其中 $a_1 、 b_1 \in \mathbf{N}^*$ 且有 $\left(a_1, b_1\right)=1, a_1 \neq b_1$, 不妨设 $a_1>b_1$. 由于 $a+b= c\left(a_1+b_1\right), a b=c^2 a_1 b_1$, 因此 $\left(a_1+b_1\right) \mid c a_1 b_1$. 又由于 $\left(a_1+b_1, a_1\right)=1$, $\left(a_1+b_1, b_1\right)=1$, 因此 $\left(a_1+b_1\right) \mid c$. 而 $a+b \leqslant 99$, 即 $c\left(a_1+b_1\right) \leqslant 99$, 所以 $3 \leqslant a_1+b_1 \leqslant 9$. 由此可知, $S$ 中满足 $(a+b) \mid a b$ 的不同数对 $(a, b)$ 共有 23 对: $a_1+b_1=3$ 时, 有 $(6,3) ,(12,6) ,(18,9) ,(24,12) ,(30,15) ,(36,18)$ , $(42,21),(48,24) ; a_1+b_1=4$ 时, 有 $(12,4),(24,8),(36,12), (48, 16) ; a_1+b_1=5$ 时, 有 $(20,5), (40,10),(15,10),(30,20),(45,30)$; $a_1+b_1=6$ 时, 有 $(30,6) ; a_1+b_1=7$ 时, 有 $(42,7),(35,14) ,(28,21)$; $a_1+b_1=8$ 时,有 $(40,24) ; a_1+b_1=9$ 时,有 $(45,36)$.
令 $M=\{6,12,15,18,20,21,24,35,40,42,45,48\}$. 则上述 23 个数对中的每一个数对都至少包含 $M$ 中的 1 个元素.
令 $T=S-M$. 则 $T$ 中任何两数都不能成为满足要求的数对 $(a, b)$. 因为 $|T|=38$, 所以所求最小自然数 $K \geqslant 39$.
另一方面,下列 12 个满足题中要求的数对互不相交: $(6,3),(12,4)$, $(20,5),(42,7),(24,8),(18,9),(40,10),(35,14),(30,15),(48, 16),(28,21),(45,36)$. 对于 $S$ 中任一 39 元子集 $R$, 它只比 $S$ 少 11 个元素, 而这 11 个元素至多属于上述 12 个数对中的 11 个, 因此, 必有 12 对中的 1 对属于 $R$. 故所求的最小自然数 $K=39$.
%%PROBLEM_END%%



%%PROBLEM_BEGIN%%
%%<PROBLEM>%%
问题13 集合 $Z$ 由 $n$ 个元素组成, $Z$ 中最多有多少个这样的 3 元子集, 使得其中任意两个 3 元子集都恰好有一个公共元.
%%<SOLUTION>%%
用 $k_n$ 表示所求的数.
设从集合 $Z$ 中取出 $k_n$ 个 3 元子集, 其中任意两个都恰好有一个公共元, 分三种可能情况:
(1) 集合 $Z$ 中的每个元素都至多出现在两个 3 元子集中.
设 $\{a, b, c\}$ 是其中一个 3 元子集, 则其他任何一个 3 元子集都与 $\{a, b, c\}$ 相交, 而且所有其他子集中至多有一个含元素 $a$, 至多有一个含元素 $b$, 至多有一个含元素 $c$. 因此, 子集最多有 $1+3 \times 1=4$ 个, 即 $k_n \leqslant 4$.
(2) 集合 $Z$ 中有一个元素出现在三个 3 元子集中,但集合 $Z$ 的每一个元素至多出现在三个 3 元子集中, 则设 $\{a, b, c\}$ 是其中一个 3 元子集, 于是其他任意一个子集都与它相交, 而且所有其他子集中至多有两个集合包含元素 $a$,至多有两个集合包含元素 $b$, 至多有两个集合包含元素 $c$. 因此, 所有子集至多有 $1+3 \times 2=7$ 个, 即 $k_n \leqslant 7$.
(3) 集合 $Z$ 中含有元素 $a$, 它至少属于 4 个 3 元子集, 则这 4 子集也应包含元素 $a$. 否则, 这样的子集与 4 个 3 元子集中每一个恰有一个公共元素, 所以它至少含有 4 个元素.
于是在此情形下, 有 $1+2 k_n \leqslant n$, 即 $k_n \leqslant\left[\frac{n-1}{2}\right]$.
当 $n=1,2,3,4,5$ 时, 显然有 $k_1=k_2=0, k_3=k_4=1, k_5=2$. 当 $n=6$ 时, 集合 $Z$ 的每个元素至多属于 2 个 3 元子集, 否则 3 元子集的并将含有 7 个元素, 因此情形 (1)成立, 即 $k_6 \leqslant 4$. 另一方面, 设 $Z=\{a, b, c, d, e, f\}$, 则 3 元子集取为 $\{a, b, c\},\{c, d, e\},\{e, f, a\},\{b, d, f\}$.于是 $k_6=4$.
设 $a \in\{7,8,9, \cdots, 16\}$, 则当出现情形 (3) 时, 3 元子集的个数至多为 $\left[\frac{16-1}{2}\right]=7$, 当出现情形 (1)、(2) 时, 3 元子集的个数也至多为 7 . 另一方面, 如果集合 $Z$ 的元素中的 7 个为 $a, b, c, d, e, f, g$, 则有 7 个 3 元子集: $\{a, b, c\},\{c, d, e\},\{e, f, a\},\{b, d, f\},\{a, g, d\},\{b, g, e\},\{c, g, f\}$.于是 $n=7,8,9, \cdots, 16$ 时, $k_n=7$.
最后, 当 $n \geqslant 17$ 时, 则不论哪种情形总有 $k_n \leqslant\left[\frac{n-1}{2}\right]$. 而且如下选取3元子集时达到上界: 集合 $Z$ 中取一个元素为所有 3 元子集的公共元,并将所有其他元素配成对 (当 $n$ 为偶数时需去掉一个), 且与公共元素一起组成 3 元子集,故当 $n \geqslant 17$ 时, $k_n=\left[\frac{n-1}{2}\right]$.
%%PROBLEM_END%%



%%PROBLEM_BEGIN%%
%%<PROBLEM>%%
问题14 设 $S=\{1,2, \cdots, 15\}$, 从 $S$ 中取出 $n$ 个子集 $A_1, A_2, \cdots, A_n$, 满足下列条件:
(i) $\left|A_i\right|=7, i=1,2, \cdots, n$;
(ii) $\left|A_i \cap A_j\right| \leqslant 3,1 \leqslant i<j \leqslant n$;
(iii)对 $S$ 的任何 3 元子集 $M$, 都存在某个 $A_k, 1 \leqslant k \leqslant n$, 使得 $M \subset A_k$. 求这样一组子集的个数 $n$ 的最小值.
%%<SOLUTION>%%
若有 $a \in S$ 且至多属于 6 个子集 $A_{i_1}, A_{i_2}, \cdots, A_{i_6}$, 则每个 $A_{i_j}$ 中除 $a$ 之外还有 6 个元素, 共可组成含 $a$ 的三元组的个数为 $\mathrm{C}_6^2=15$. 于是, 6 个子集共可组成不同的含 $a$ 的三元组的个数至多为 90 个.
另一方面, $S$ 中所有不同的含 $a$ 三元组的个数为 $C_{14}^2=7 \times 13=91>90$, 无法使 (iii) 成立.
所以, 为使条件 (i) 一 (iii) 成立, $S$ 中的每个数都至少属于 7 个子集.
这样一来, 必有 $n \geqslant 15$.
用字典排列法可以写出满足题中要求的 15 个 7 元子集: $\{1,2,3,4,5, 6,7\},\{1,2,3,8,9,10,11\},\{1,2,3,12,13,14,15\},\{1,4,5,8,9, 12,13\},\{1,4,5,10,11,14,15\},\{1,6,7,8,9,14,15\},\{1,6,7,10,11,12,13\},\{2,4,6,8,10,12,14\},\{2,4,6,9,11,13,15\},\{2,5,7,8,10,13,15\},\{2,5,7,9,11,12,14\},\{3,4,7,8,11,12,15\},\{3,4,7,9,10,13,14\},\{3,5,6,8,11,13,14\},\{3,5,6,9,10,12,15\}$.
%%PROBLEM_END%%



%%PROBLEM_BEGIN%%
%%<PROBLEM>%%
问题15 设 $S \subseteq\{1,2, \cdots, 2002\}$. 对任意 $a, b \in S$ ( $a 、 b$ 可以相同), 总有 $a b \notin S$, 求 $|S|$ 的最大值.
%%<SOLUTION>%%
首先, $1 \notin S$. 其次, 若 $a \in\{2,3,4,5,6\}$ 且 $a \in S$, 则以下 45 对数对中, 每对的两个数不能同时属于 $S:(1, a),(2,2 a), \cdots,\left(a-1,(a-1) a\right),(a+1,(a+1) a),(a+2,(a+2) a), \cdots,(2 a-1,(2 a-1) a), \cdots(44 a+1, (44 a+1) a),(44 a+2,(44 a+2) a), \cdots,(45 a-1,(45 a-1) a)$. 由于 $(45 a- 1) a \leqslant(45 \times 6-1) \times 6<2002$, 所以以上 90 个小于 2002 的数中至少有一半不属于 $S$. 从而 $|S| \leqslant 2002-45=1957$.
再次, 若 $2,3,4,5,6 \notin S$, 考虑以下 38 个数对: $\left(7,7^2\right),\left(8,8^2\right), \cdots, \left(k, k^2\right), \cdots,\left(44,44^2\right)$, 若有某一对中的两个数 $k_0, k_0^2 \in S$, 则令 $a=b=k_0$, 有 $a, b, a b \in S$, 与题设矛盾! 因此这里至少有 38 个数不属于 $S$, 再减去 1 , $2, \cdots, 6$, 有 $|S| \leqslant 2002-38-6=1958$. 又 $S=\{45,46,47, \cdots, 2002\}$ 满足要求.
对任意 $a, b \in S, a b \geqslant 45^2=2025>2002$, 即 $a b \notin S$, 此时 $|S|=1958$.
%%PROBLEM_END%%



%%PROBLEM_BEGIN%%
%%<PROBLEM>%%
问题16 称子集 $A \subseteq M=\{1,2, \cdots, 11\}$ 是好的,如果它有下述性质: “如果 $2 k \in A$, 则 $2 k-1 \in A$, 且 $2 k+1 \in A$ ” (空集和 $M$ 都是好的). $M$ 有多少个好子集?
%%<SOLUTION>%%
设 $n(A)$ 为属于 $A$ 的偶数的个数.
情形 $0: n(A)=0$. 只须确定 $A$ 中的奇数.
有 $2^6$ 个好子集.
情形 $1: n(A)=1$. 对偶数的选取有 5 种可能性.
有 $5 \times 2^4$ 个好集合 $A$.
情形 $2: n(A)=2$. (I ) 在好子集中的偶数是相邻的.
有 $4 \times 2^3$ 个好子集.
(II) $A$ 中的两个偶数不相邻.
有 $6 \times 2^2$ 个好子集.
共有 56 个好子集.
情形 $3: n(A)=3$. (I ) A 中的偶数是相邻的.
有 $3 \times 2^2$ 个好子集.
II) $A$ 中的任意两个偶数都不相邻.
$A=\{1,2,3,5,6,7,9,10,11\}$ 是惟一的选择.
(III) $A$ 的 3 个偶数中恰好两个是相邻的.
有 $6 \times 2=12$ 个好子集.
共有 25 个好子集.
情形 $4: n(A)=4$. ( I ) $2 \notin A$ 或 $10 \notin A$. 有 4 个好子集.
(II ) $2 \in A$ 且 $10 \in A$. 有 3 个好子集.
共有 7 个好子集.
情形 $5: n(A)=5$, 则 $A=M, 1$ 种可能性.
最后, 好子集的总数是 $2^6+5 \times 2^4+56+25+7+1=233$.
%%PROBLEM_END%%



%%PROBLEM_BEGIN%%
%%<PROBLEM>%%
问题17 设 $n$ 为给定的正整数, $D_n$ 为 $2^n 3^n 5^n$ 的所有正因数所成的集合, $S \subseteq D_n$, 且 $S$ 中任一数都不能整除 $S$ 中另一数.
求 $|S|$ 的最大值.
%%<SOLUTION>%%
显然 $D_n$ 中的每一个数都有形式 $2^\alpha 3^\beta 5^\gamma$, 其中 $0 \leqslant \alpha, \beta, \gamma \leqslant n$, 下面用数组 $(\alpha, \beta, \gamma)$ 表示数 $2^\alpha 3^\beta 5^\gamma$. 考察如下集合: $A_{i, j}=\{(i, j, \alpha), 0 \leqslant \alpha \leqslant n- j\} \cup\{(i, \beta, n-j), j \leqslant \beta \leqslant n\}, i==0,1,2, \cdots, n, j=0,1,2, \cdots,\left[\frac{n}{2}\right]$; $B_{i, j}=\{(k, i, j), k=0,1,2, \cdots, n\}, i=\left[\frac{n}{2}\right]+1, \cdots, n, j=0,1,2, \cdots,n-\left[-\frac{n}{2}\right]-1$. 共有 $\left(\left[\frac{n}{2}\right]+1\right) \cdot(n+1)+\left(n-\left[\frac{n}{2}\right]\right)^2=\frac{3(n+1)^2+1}{4}(n$ 为偶数), 或 $\left(\left[\frac{n}{2}\right]+1\right)(n+1)+\left(n-\left[\frac{n}{2}\right]\right)^2=\frac{3(n+1)^2}{4}(n$ 为奇数), 即 $\left[\frac{3(n+1)^2+1}{4}\right]$ 个集合.
又因为每个集合中的数至多有一个属于 $S$, 可得以上集合互不相交且包含所有的数, 即 $\cup A_{i, j} \cup B_{i, j}=D_n$. 因而可得
$$
|S| \leqslant\left[\frac{3(n+1)^2+1}{4}\right]
$$
另一方面, 考察满足 $\alpha+\beta+\gamma=n+t$ (其中 $t=\left[\frac{n}{2}\right]$) 的数组 $(\alpha, \beta, \gamma)$ 的个数.
由条件 $0 \leqslant \alpha 、 \beta 、 \gamma \leqslant n$, 可得 $(n+t+1)+(n+t)+\cdots+1-3[t+ (t-1)+\cdots+1]=\left[\frac{3(n+1)^2+1}{4}\right]$. (其中 $t=\left[\frac{n}{2}\right]$ 当 $\alpha, \beta, \gamma$ 中有一个大于 $n$ 时, 共有 $3(t+(t-1)+\cdots+1)$ 组解, 故应除去) 取 $S= \left\{2^\alpha 3^\beta 5^\gamma \mid \alpha+\beta+\gamma=n+\left[\frac{n}{2}\right], 0 \leqslant \alpha, \beta, \gamma \leqslant n\right\}$, 可得 $|S|=\left[\frac{3(n+1)^2+1}{4}\right]$, 且 $S$ 中任意两数不具备倍数关系, 否则, 不妨设 $2^{\alpha_1} 3^{\beta_1} 5^{\gamma_1} \mid 2^{\alpha_2} 3^{\beta_2} 5^{\gamma_2}$, 则 $\alpha_1 \leqslant \alpha_2, \beta_1 \leqslant \beta_2, \gamma_1 \leqslant \gamma_2, \alpha_1+\beta_1+\gamma_1=\alpha_2+\beta_2+\gamma_2$, 因此 $\alpha_1=\alpha_2, \beta_1=\beta_2, \gamma_1=\gamma_2$, 产生矛盾! 综上可知, $|S|_{\text {max }}=\left[\frac{3(n+1)^2+1}{4}\right]$.
%%PROBLEM_END%%


