
%%PROBLEM_BEGIN%%
%%<PROBLEM>%%
问题1: 已知三元实数集 $A=\{x, x y, x+y\}, B=\{0,|x|, y\}$, 且 $A=B$, 则$x^{2005}+y^{2005}=$
%%<SOLUTION>%%
解: 0 .
%%PROBLEM_END%%



%%PROBLEM_BEGIN%%
%%<PROBLEM>%%
问题2: 设集合 $S=\left\{(x, y) \mid x-\frac{1}{2} y^2\right.$ 为奇数, $\left.x, y \in \mathbf{R}\right\}, T=\{(x, y) \left.\sin (2 \pi x)-\sin \left(\pi y^2\right)=\cos (2 \pi x)-\cos \left(\pi y^2\right), x, y \in \mathbf{R}\right\}$. 则 $S$ 与 $T$ 的关系是
%%<SOLUTION>%%
解: $S \varsubsetneqq T$. 当 $x=\frac{1}{2} y^2+$ 奇数时, 显然 $\sin (2 \pi x)-\sin \left(\pi y^2\right)=\cos (2 \pi x)-\cos \left(\pi y^2\right)$ 成立, $S \subseteq T$. 但满足 $x=\frac{1}{2} y^2$ 的点 $(x, y) \in T$, 而不属于 $S$, 故 $S \varsubsetneqq T$.
%%PROBLEM_END%%



%%PROBLEM_BEGIN%%
%%<PROBLEM>%%
问题3: 集合 $M=\{u \mid u=12 m+8 n+4 l, m, n, l \in \mathbf{Z}\}$ 与 $N=\{u \mid u=20 p+ 16 q+12 r, p, q, r \in \mathbf{Z}\}$ 的关系为
%%<SOLUTION>%%
解: $M=N$. 因 $12 m+8 n+4 l=4(3 m+2 n+l), 20 p+16 q+12 r=4(5 p+4 q+3 r),(3,2,1)=1,(5,4,3)=1$, 由裴蜀定理可知 $3 m+2 n+ l$ 与 $5 p+4 q+3 r$ 均可表示所有整数.
所以, $M=N=\{k \mid k=4 t, t \in \mathbf{Z}\}$.
%%PROBLEM_END%%



%%PROBLEM_BEGIN%%
%%<PROBLEM>%%
问题4: 设 $A=\{(x, y) \mid 0 \leqslant x \leqslant 2,0 \leqslant y \leqslant 2\}, B=\{(x, y) \mid x \leqslant 10, y \geqslant 2, y \leqslant x-4\}$ 是直角坐标平面 $x O y$ 上的点集.
则 $C=\left\{frac{x_1+x_2}{2}\frac{y_1+y_2}{2}\right\} \mid\left(x_1, y_1\right) \in A,\left(x_2, y_2\right) \in B\right\}$ 所成图形的面积是
%%<SOLUTION>%%
解: 7. 如图(<FilePath:./figures/fig-c1p4.png>),集合 $A$ 为正方形 $O A B C$, 集合 $B$ 为 Rt $\triangle D E F . O D 、 A E 、 B F 、 C F$ 、 $C D$ 的中点依次为 $M(3,1) 、 N(6,1)$ 、 $P(6,4) 、 Q(5,4) 、 R(3,2)$. 所成图形面积 $S_{M N P Q R}=7$.
%%PROBLEM_END%%



%%PROBLEM_BEGIN%%
%%<PROBLEM>%%
问题5: 已知非空数集 $M \subseteq\{1,2,3,4,5\}$, 则满足条件“若 $x \in M$, 则 $6-x \in M$ ” 的集合 $M$ 的个数是
%%<SOLUTION>%%
解: 7. 因为 $1+5=2+4=3+3$, 故 $M$可以是 $\{3\},\{1,5\},\{2,4\},\{1,3,5\},\{2,3,4\},\{1,2,4,5\},\{1,2,3$, $4,5\}$.
%%PROBLEM_END%%



%%PROBLEM_BEGIN%%
%%<PROBLEM>%%
问题6: 设 $a \in \mathbf{R}^{+}, A=\left\{(x, y) \mid(x-1)^2+(y-2)^2 \leqslant \frac{4}{5}\right\}$ 与 $B=\{(x, y) \mid |x-1|+2|y-2| \leqslant a\}$ 是直角坐标平面 $x O y$ 内的点集.
则 $A \subseteq B$ 的充要条件是
%%<SOLUTION>%%
解: $a \geqslant 2$. 集合 $A$ 为以 $(1,2)$ 为圆心 $、 \frac{2}{\sqrt{5}}$ 为半径的圆面.
集合 $B$ 为以 $(1,2)$ 为对角线交点的菱形, 且平行于 $x$ 轴的对角线长为 $2 a$, 平行于 $y$ 轴的对角线长为 $a$. 由 $A \subseteq B$ 知, 当 $a \cdot \frac{a}{2}=\frac{2}{\sqrt{5}} \cdot \sqrt{a^2+\left(\frac{a}{2}\right)^2}$ 时 $a$ 值最小, 所以 $a_{\min }=2$.
%%PROBLEM_END%%



%%PROBLEM_BEGIN%%
%%<PROBLEM>%%
问题7: 集合 $\left\{x \mid-1 \leqslant \log _{\frac{1}{x}} 10<-\frac{1}{2}, x>1\right.$ 且 $\left.x \in \mathbf{N}\right\}$ 的真子集的个数是
%%<SOLUTION>%%
解: $2^{90}-1$.
%%PROBLEM_END%%



%%PROBLEM_BEGIN%%
%%<PROBLEM>%%
问题8: 已知 $A=\left\{x \mid x^2-4 x+3<0, x \in \mathbf{R}\right\}, B=\left\{x \mid 2^{1-x}+a \leqslant 0, x^2-\right. 2(a+7) x+5 \leqslant 0, x \in \mathbf{R}\}$. 若 $A \subseteq B$, 则实数 $a$ 的取值范围是
%%<SOLUTION>%%
解: $-4 \leqslant a \leqslant-1$. 易知 $A=(1,3)$. 记 $f(x)=2^{1-x}+a, g(x)=x^2-2(a+7) x+5 . A \subseteq B$ 表明, 当 $1<x<3$ 时, 函数 $f(x)$ 与 $g(x)$ 的图象都在$x$ 轴的下方.
$A \subseteq B$ 的充要条件是: $f(1) \leqslant 0, g(1) \leqslant 0$ 和 $f(3) \leqslant 0, g(3) \leqslant 0$ 同时成立.
解之即得.
%%PROBLEM_END%%



%%PROBLEM_BEGIN%%
%%<PROBLEM>%%
问题9: 已知 $M=\left\{x \mid x=a^2+1, a \in \mathbf{N}^*\right\}, N=\left\{x \mid x=b^2-4 b+5, b \in \mathbf{N}^*\right\}$, 则 $M$ 与 $N$ 的关系是
%%<SOLUTION>%%
解: $M \varsubsetneqq N$. 由 $a^2+1=(a+2)^2-4 a+5$ 知 $M \subseteq N$. 但 $1 \in N, 1 \notin M$.
%%PROBLEM_END%%



%%PROBLEM_BEGIN%%
%%<PROBLEM>%%
问题10: 非空集合 $S$ 满足:
(1) $S \subseteq\{1,2, \cdots, 2 n+1\}, n \in \mathbf{N}^*$;
(2) 若 $a \in S$, 则有 $2 n+2-a \in S$.
那么, 同时满足 (1)、(2) 的非空集合 $S$ 的个数是
%%<SOLUTION>%%
解: $2^{n+1}-1$. 把自然数 $1,2, \cdots, 2 n+1$ 搭配成 $n+1$ 个数组 $\{1,2 n+1\}, \{2,2 n\}, \cdots,\{n, n+2\},\{n+1\} . S$ 的元素从以上 $n+1$ 组选取, 有 $\mathrm{C}_{n+1}^1+ \mathrm{C}_{n+1}^2+\cdots+\mathrm{C}_{n+1}^{n+1}=2^{n+1}-1$ 种取法.
%%PROBLEM_END%%



%%PROBLEM_BEGIN%%
%%<PROBLEM>%%
问题11: 设由模为 1 的 $n(2<n<6)$ 个复数组成的集合 $S$ 满足下面两条:
(1) $1 \in S$;
(2) 若 $z_1 \in S, z_2 \in S$, 则 $z_1-2 z_2 \cos \theta \in S$, 其中 $\theta=\arg \frac{z_1}{z_2}$.
则集合 $S=$
%%<SOLUTION>%%
解: $\{1,-1, \mathrm{i},-\mathrm{i}\}$. 当 $z_1=z_2=z$ 时, 若 $z \in S$, 则 $z_1-2 z_2 \cos \theta=-z \in S$. 因 $|z|=1$, 所以 $|z|=|-z|=1$. 这说明 $S$ 中含有偶数个元素.
又 $2<n<6$, 所以 $n=4$. 由 $1 \in S$, 得 $-1 \in S$. 设 $z_1=\cos \alpha+i \sin \alpha(\sin \alpha \neq 0$, $0 \leqslant \alpha<2 \pi), z_2=1, \theta=\arg \left(\frac{z_1}{z_2}\right)=\alpha$. 若 $z_1 \in S$, 则 $z_1-2 \cos \theta=-\cos \alpha+i \sin \alpha \in S$. 因为 $\sin \alpha \neq 0$, 故 $\cos \alpha+i \sin \alpha \neq-(\cos \alpha+i \sin \alpha)$, 所以 $\cos \alpha+\mathrm{i} \sin \alpha=-\cos \alpha+\mathrm{i} \sin \alpha$, 即 $\cos \alpha=0, \sin \alpha= \pm 1$. 所以 $i \in S,-i \in S$.
%%PROBLEM_END%%



%%PROBLEM_BEGIN%%
%%<PROBLEM>%%
问题12: 集合 $A=\left\{x_1, x_2, x_3, x_4, x_5\right\}$,计算 $A$ 中的二元子集两元素之和组成集合 $B=\{3,4,5,6,7,8,9,10,11,13\}$. 则 $A=$
%%<SOLUTION>%%
解: $\{1,2,3,5,8\}$. 不妨设 $x_1<x_2<x_3<x_4<x_5$. 则 $x_1+x_2=3$, $x_4+x_5=13$. 又 $4\left(x_1+x_2+x_3+x_4+x_5\right)=3+4+5+6+7+8+9+10+11+13$, 即 $x_1+x_2+x_3+x_4+x_5=19$, 从而得 $x_3=19-3-13=3$. 又 $x_1+ x_3=4$, 从而 $x_1=1$. 又 $x_3+x_5=11$, 从而 $x_5=8$, 所以 $x_2=2, x_4=5$.
%%PROBLEM_END%%



%%PROBLEM_BEGIN%%
%%<PROBLEM>%%
问题13: 设 $E=\{1,2,3, \cdots, 200\}, G=\left\{a_1, a_2, a_3, \cdots, a_{100}\right\} \subseteq E$, 且 $G$ 具有下列两条性质:
(1)对任何 $1 \leqslant i<j \leqslant 100$, 恒有 $a_i+a_j \neq 201$;
(2) $\sum_{i=1}^{100} a_i=10080$.
试证明: $G$ 中的奇数的个数是 4 的倍数, 且 $G$ 中所有数字的平方和为一个定数.
%%<SOLUTION>%%
解: 由已知得 $\sum_{k=1}^{200} k^2=\sum_{i=1}^{100} a_i^2+\sum_{i=1}^{100}\left(201-a_i\right)^2=2 \sum_{i=1}^{100} a_i^2-402 \sum_{i=1}^{100} a_i+201^2 \times 100$. 由 (2) 及上式得 $\sum_{i=1}^{100} a_i^2$ 为常数.
设 $G$ 中有 $x$ 个奇数, 则由上式可得 $4 \equiv 2 x-0+4(\bmod 8)$, 故 $x \equiv 0(\bmod 4)$.
%%PROBLEM_END%%



%%PROBLEM_BEGIN%%
%%<PROBLEM>%%
问题14: 称有限集 $S$ 的所有元素的乘积为 $S$ 的 “积数”, 给定数集 $M= \left\{\frac{1}{2}, \frac{1}{3}, \cdots, \frac{1}{100}\right\}$. 求集 $M$ 的所有含偶数个元素的子集的“积数”之和.
%%<SOLUTION>%%
解: 设集合 $M$ 的所有含偶数个元数的子集的积数之和为 $x$, 所有含奇数个元素的子集的积数之和为 $y$, 则 $x+y=\left(1+\frac{1}{2}\right)\left(1+\frac{1}{3}\right) \cdots\left(1+\frac{1}{100}\right)-1, x-y=\left(1-\frac{1}{2}\right)\left(1-\frac{1}{3}\right) \cdots\left(1-\frac{1}{100}\right)-1$. 所以 $x+y=\frac{99}{2}, x-y= -\frac{99}{100}$. 解得 $x=\frac{4851}{200}$.
%%PROBLEM_END%%



%%PROBLEM_BEGIN%%
%%<PROBLEM>%%
问题15: 考虑实数 $x$ 在 3 进制中的表达式.
$K$ 是区间 $[0,1]$ 内所有这样的数 $x$ 的集合,并且 $x$ 的每位数字是 0 或 2. 如果 $S=\{x+y \mid x, y \in K\}$, 求证: $S=\{z \mid 0 \leqslant z \leqslant 2\}=[0,2]$.
%%<SOLUTION>%%
解: 在 $K$ 内 $x$ 和 $y$ 的每位数字是 0 或 2 , 因此, $\frac{x}{2}$ 和 $\frac{y}{2}$ 的每位数字是 0 或 1 , 从而 $\frac{x}{2}+\frac{y}{2}$ 的每位数字在 3 进制下是 $0 、 1$ 或 2 , 并且由 $x \in[0,1]$, $y \in[0,1]$ 可知 $\frac{x}{2}+\frac{y}{2} \in[0,1]$. 反过来, 对于 $[0,1]$ 上的任何一个数, 它在 3 进制下的每位数字是 $0 、 1$ 或 2 , 显然可以写成两个在 3 进制下每位数字是 0或 1 的数的和.
也就是说, 都可以写成 $\frac{x}{2}+\frac{y}{2}, x, y \in K$ 的形式.
因此, 我们有 $\left\{\frac{x}{2}+\frac{y}{2} \mid x, y \in K\right\}=[0,1]$, 故得 $S=\{x+y \mid x, y \in K\}=[0,2]$.
%%PROBLEM_END%%



%%PROBLEM_BEGIN%%
%%<PROBLEM>%%
问题16: 设 $S=\{1,2,3,4\}, n$ 项的数列: $a_1, a_2, \cdots, a_n$ 有如下性质: 对于 $S$ 的任何一个非空子集 $B(B$ 的元素个数记为 $|B|)$, 在该数列中有相邻的 $|B|$项恰好组成集合 $B$. 求 $n$ 的最小值.
%%<SOLUTION>%%
解: 首先证明 $s$ 中的每个数在数列 $a_1, a_2, \cdots, a_n$ 中至少出现 2 次.
事实上, 若 $s$ 中的某个数在这个数列中只出现一次, 由于含这个数的二元子集共有 3 个, 但在数列中含这个数的相邻两项至多有两种取法, 因此不可能 3 个含这个数的二元子集都在数列相邻两项中出现.
矛盾.
由此可得, $n \geqslant 8$. 另一方面, 数列 $3,1,2,3,4,1,2,4$ 满足题设条件, 且只有 8 项.
所以, $n$ 的最小值为 8 .
%%PROBLEM_END%%



%%PROBLEM_BEGIN%%
%%<PROBLEM>%%
问题17: 设集合 $M=\{1,2,3, \cdots, 1000\}$, 现对 $M$ 的任一非空子集 $X$,令 $\alpha_X$ 表示 $X$ 中最大数与最小数之和.
求所有这样的 $\alpha_X$ 的算术平均值.
%%<SOLUTION>%%
解: 构造子集 $X^{\prime}=\{1001-x \mid x \in X\}$, 则所有非空子集分成两类 $X^{\prime}=X$ 和 $X^{\prime} \neq X$. 当 $X^{\prime}=X$ 时,必有 $X^{\prime}=X=M$, 于是, $\alpha_X=1001$. 当 $X^{\prime} \neq X$ 时, 设 $x 、 y$ 分别是 $X$ 中的最大数与最小数, 则 $1001-x 、 1001-y$ 分别是 $X^{\prime}$ 中的最小数与最大数.
于是, $\alpha_X=x+y, \alpha_{X^{\prime}}=2002-x-y$. 从而, $\frac{\alpha_X+\alpha_{X^{\prime}}}{2}=1001$. 因此,所有的 $\alpha_X$ 的算术平均值为 1001 .
%%PROBLEM_END%%



%%PROBLEM_BEGIN%%
%%<PROBLEM>%%
问题18: 设 $S$ 为满足下列条件的有理数集合:
(1) 若 $a \in S, b \in S$, 则 $a+b \in S, a b \in S$;
(2) 对任一个有理数 $r, 3$ 个关系 $r \in S 、-r \in S 、 r=0$ 中有且仅有一个成立.
证明: $S$ 是由全体正有理数组成的集合.
%%<SOLUTION>%%
解: 对任意的 $r \in \mathbf{Q}, r \neq 0$, 由 (2) 知 $r \in S,-r \in S$ 之一成立.
再由 (1), 若 $r \in S$, 则 $r^2 \in S$; 若 $-r \in S$, 则 $r^2=(-r) \cdot(-r) \in S$. 总之, 对任意的非零 $r \in \mathbf{Q}$ 均有 $r^2 \in S$. 取 $r=1$, 则 $1=1^2 \in S$. 由 (1), $2=1+1 \in S, 3=1+ 2 \in S, \cdots$, 可知全体正整数都属于 $S$. 设 $p, q \in \mathbf{N}$, 由 (1), $p q \in S$. 又由前证知 $\frac{1}{q^2} \in S$, 所以 $\frac{p}{q}=p q \cdot\left(\frac{1}{q^2}\right) \in S$. 因此, $S$ 含有全体正有理数.
再由(2)知, 0 及全体负有理数不属于 $S$, 即 $S$ 是由全体正有理数组成的集合.
%%PROBLEM_END%%



%%PROBLEM_BEGIN%%
%%<PROBLEM>%%
问题19: $S_1 、 S_2 、 S_3$ 为非空整数集合, 对于 $1 、 2 、 3$ 的任意一个排列 $i 、 j 、 k$, 若 $x \in S_i, y \in S_j$, 则 $y-x \in S_k$.
(1) 证明: 3 个集合中至少有两个相等.
(2) 3 个集合中是否可能有两个集合无公共元素?
%%<SOLUTION>%%
解: (1) 由已知, 若 $x \in S_i, y \in S_j$, 则 $y-x \in S_k,(y-x)-y= -x \in S_i$, 所以每个集合中均有非负元素.
当三个集合中的元素都为零时,命题显然成立.
否则, 设 $S_1 、 S_2 、 S_3$ 中的最小正元素为 $a$, 不妨设 $a \in S_1$. 设 $b$ 为 $S_2 、 S_3$ 中最小的非负元素, 不妨设 $b \in S_2$. 则 $b-a \in S_3$. 若 $b>0$, 则 $0 \leqslant b- a<b$, 与 $b$ 的取法矛盾, 所以 $b=0$. 任取 $x \in S_1$, 因 $0 \in S_2$, 故 $x-0=x \in S_3$, 所以 $S_1 \subseteq S_3$. 同理, $S_3 \subseteq S_1$. 所以 $S_1=S_3$.
(2)可能.
例如 $S_1=S_2=\{$ 奇数 $\}, S_3=\{$ 偶数 $\}$, 显然满足条件, 但 $S_1$ 和 $S_2$ 与 $S_3$ 都无公共元素.
%%PROBLEM_END%%



%%PROBLEM_BEGIN%%
%%<PROBLEM>%%
问题20: 若 $x \geqslant 1, x^x=x_0, x_0 \in\left(k^k,(k+1)^{(k+1)}\right) \cap \mathbf{Q}$, 其中 $k \in \mathbf{N}^*$. 求证: $x \in \mathbf{Q}^C$. (其中, $\mathbf{Q}$ 为有理数集, $\mathbf{Q}^C$ 为无理数集)
%%<SOLUTION>%%
解: 因为当 $x_2 \geqslant x_1 \geqslant 1$ 时, 有 $x_2^{x_2} \geqslant x_1^{x_2} \geqslant x_1^{x_1}$. 所以 $y=x^x$ 在 $[1,+\infty)$ 上单调递增.
所以 $x^x=x_0$ 在 $[1,+\infty)$ 上有且仅有一解, 且 $x \in(k, k+1)$. 假设 $x^x=x_0$ 的解为有理数, 可设 $x=\frac{n}{m}, m, n \in \mathbf{N}^*,(m, n)=1$, 且 $m \neq 1$;$p, q \in \mathbf{R}^{+},\left(\frac{q}{p}\right)^m=\frac{n}{m}$. 所以 $x^x=\left(\frac{n}{m}\right)^{\frac{n}{m}}=\left(\frac{q}{p}\right)^n=x_0 \in \mathbf{Q}$. 又因为 $(m$, $n)=1$, 所以存在 $a, b \in \mathbf{Z}$, 使得 $a m+b n=1$, 所以 $\frac{q}{p}=\left(\frac{q}{p}\right)^{a m+b n}= \left[\left(\frac{q}{p}\right)^m\right]^a \cdot\left[\left(\frac{q}{p}\right)^n\right]^b \in \mathbf{Q}$. 所以不妨设 $p, q \in \mathbf{N}^*$, 且 $(p, q)=1$, 则 $\left(p^m\right.$, $\left.q^m\right)=1, \frac{n}{m}=\frac{q^m}{p^m}$. 易得 $m \mid p^m$, 且 $p^m \mid m$, 所以 $m=p^m$. 注意到 $m$ 为大于 1 的整数,矛盾.
所以 $x$ 为无理数.
%%PROBLEM_END%%


