
%%PROBLEM_BEGIN%%
%%<PROBLEM>%%
问题1 已知 $S=\left\{\frac{(3 n) !}{6^n \cdot n} \mid n \in \mathbf{N}^*\right\}$, 设 $s \in S$. 求证 $: s \in \mathbf{N}^*$.
%%<SOLUTION>%%
显然 $2^n \mid(3 n)$ !. 而 $3^n \cdot n !=3 \cdot 6 \cdot 9 \cdot \cdots \cdot 3 n$, 所以 $\left(3^n \cdot n !\right) \mid(3 n)$ !.
%%PROBLEM_END%%



%%PROBLEM_BEGIN%%
%%<PROBLEM>%%
问题2 数集 $M$ 由 2003 个不同的正数组成,对于 $M$ 中任何三个不同的元素 $a 、 b$ 、 $c$, 数 $a^2+b c$ 都是有理数.
证明: 可以找到一个正整数 $n$, 使得对于 $M$ 中任何数 $a$, 数 $a \sqrt{n}$ 都是有理数.
%%<SOLUTION>%%
$a, b, c, d \in M$, 且两两不同.
由 $d^2+a b \in \mathbf{Q}, d^2+b c \in \mathbf{Q}$, 得 $b c-a b \in \mathbf{Q}$. 故 $a^2+a b=a^2+b c+(a b-b c) \in \mathbf{Q}$, 同理 $b^2+a b \in \mathbf{Q}$. 从而, 对 $M$ 中任何两个不同的数 $a 、 b$, 都有 $q=\frac{a}{b}=\frac{a^2+a b}{b^2+a b} \in \mathbf{Q}$. 于是, $a=q b$, 从而 $a^2+a b= b^2\left(q^2+q\right)=l \in \mathbf{Q}, b=\sqrt{\frac{l}{q^2+q}}=\sqrt{\frac{m}{k}}, m, k \in \mathbf{N}$. 令 $n=m k$, 得 $b \sqrt{n}=m \in \mathbf{Q}$. 故对任何 $c \in M$, 有 $c \sqrt{n}=\frac{c}{b} \cdot b \sqrt{n} \in \mathbf{Q}$.
%%PROBLEM_END%%



%%PROBLEM_BEGIN%%
%%<PROBLEM>%%
问题3 已知由 2003 个正数组成的集合,该集合中的任意两个数 $a$ 与 $b(a>b)$ 的和 $a+b$ 与差 $a-b$ 中至少有一个属于该集合.
证明: 若将该集合中的数按递增的顺序排列, 则相邻两个数的差相同.
%%<SOLUTION>%%
将 2003 个正数按递增顺序排列, 并记 $A=\left\{a_1, a_2, \cdots, a_{2003}\right\}$. 因为 $\left(a_{2003}+a_i\right) \notin A$, 所以 $\left(a_{2003}-a_i\right) \in A, i=1,2, \cdots, 2002$. 即 $a_{2003}-a_{2002}< a_{2003}-a_{2001}<\cdots<a_{2003}-a_1<a_{2003} \in A$. 故 $a_{2003}-a_i=a_{2003-i}, i=1,2, \cdots$, 2002. 同理 $a_{2002}-a_i=a_{2002-i}, i=1,2, \cdots, 2001$. 所以, $a_{i+1}=a_{2003}-a_{2002-i}= a_i+a_{2003}-a_{2002}$. 从而, $a_{i+1}-a_i=a_{2003}-a_{2002}, i=1,2, \cdots, 2002$.
%%PROBLEM_END%%



%%PROBLEM_BEGIN%%
%%<PROBLEM>%%
问题4 设 $S=\left\{\frac{m n}{m^2+n^2} \mid m, n \in \mathbf{N}^*\right\}$. 求证: 如果 $x, y \in S$, 且 $x<y$, 那么一定存在 $z \in S$, 使得 $x<z<y$.
%%<SOLUTION>%%
设 $x, y \in S, x=\frac{m n}{m^2+n^2}, y=\frac{a b}{a^2+b^2}, x<y$. 不妨设 $m \leqslant n, a \leqslant b$. 考虑函数 $f(x)=\frac{x}{1+x^2}$, 易证 $f(x)$ 在 $[0,1]$ 上严格递增.
所以对所有 $c, d \in[0,1]$, 有 $f(c)<f(d) \Leftrightarrow c<d$. 因为 $f\left(\frac{m}{n}\right)=\frac{m n}{m^2+n^2}<\frac{a b}{a^2+b^2}=f\left(\frac{a}{b}\right)$, 所以 $\frac{m}{n}<\frac{a}{b}$. 因此, 可选择有理数 $\frac{p}{q}$ (其中 $p, q \in \mathbf{N}^*$ ), 使得 $\frac{m}{n}<\frac{p}{q}<\frac{a}{b}$(例如取 $\left.\frac{p}{q}=-\frac{1}{2} \cdot\left(\frac{m}{n}+\frac{a}{b}\right)\right)$, 那么, 就有 $f\left(\frac{m}{n}\right)<f\left(\frac{p}{q}\right)<f\left(\frac{a}{b}\right)$. 令 $z=f \left(\frac{p}{q}\right)=\frac{p q}{p^2+q^2}$ 即可.
%%PROBLEM_END%%



%%PROBLEM_BEGIN%%
%%<PROBLEM>%%
问题5 求所有的由不同正整数 (至少 2 个)组成的集合,使其中各数之和等于它们的积.
%%<SOLUTION>%%
显然, 1 在集合中起着保持积不动而增大和的作用, 而且它是具有这种性质的惟一正整数.
先设集合中的 $n$ 个数中不含 1 且 $1<a_1<a_2<\cdots< a_n$, 于是 $a_j>j, j=1,2, \cdots, n$. 因 $n \geqslant 2$, 故有 $a_n a_{n-1}-a_n-a_{n-1}= (a_n -  1) \left(a_{n-1}-1\right)-1 \geqslant 1, a_n a_{n-1} \geqslant a_n+a_{n-1}+1$. 其中等号成立当且仅当 $a_n=3$, $a_{n-1}=2$. 从而当 $n \geqslant 3$ 时, $a_n a_{n-1} a_{n-2} \geqslant 2 a_n a_{n-1}>a_n+a_{n-1}+a_{n-2}+1$. 依此类推, 便得 $a_n a_{n-1} \cdots a_1>a_n+a_{n-1}+\cdots+a_1+1$. 这表明, 凡不含 1 的集合都不满足要求,而且当 $n \geqslant 3$ 时, 即使将 1 添人集合中也不能满足要求.
当 $n=2$ 时, 只有 $a_0=1, a_1=2, a_2=3$ 才满足题中要求.
又当 $n=1$ 时, $a_1 \cdot 1<a_1+1$, 所以任何 $\left\{1, a_1\right\}$ 都不满足题中要求.
总结起来, 满足题中要求的集合只有一个, 即为 $\{1,2,3\}$.
%%PROBLEM_END%%



%%PROBLEM_BEGIN%%
%%<PROBLEM>%%
问题6 设集合 $M=\left\{x_1, x_2, \cdots, x_{30}\right\}$ 由 30 个互不相同的正数组成, $A_n(1 \leqslant n \leqslant 30)$ 是 $M$ 中所有的 $n$ 个不同元素之和的和数.
证明: 若 $A_{15}>A_{10}$, 则 $A_1>1$.
%%<SOLUTION>%%
只需证明: 如果 $A_1 \leqslant 1$, 那么对一切 $1 \leqslant n \leqslant 29$, 都有 $A_{n+1}<A_n$. 由于 $A_1 \leqslant 1$, 所以, $A_n \geqslant A_1 A_n$. 将 $A_1$ 与 $A_n$ 乘开, 并且整理以后, 可知 $A_1 A_n= A_{n+1}+S_n$, 易证 $S_n>0$, 由此即得所证.
%%PROBLEM_END%%



%%PROBLEM_BEGIN%%
%%<PROBLEM>%%
问题7 $S$ 为 $m$ 个无序正整数对 $(a, b)(1 \leqslant a<b \leqslant n)$ 所成的集合.
证明: 至少有 $4 m \cdot \frac{m-\frac{n^2}{4}}{3 n}$ 个无序三元数组 $(a, b, c)$, 使得 $(a, b),(b, c),(c, a)$ 都属于 $S$.
%%<SOLUTION>%%
考虑 $n$ 个点 $1,2, \cdots, n$. 如果 $(i, j) \in S$, 则在 $i$ 与 $j$ 之间连一条线.
我们来求这个图中的三角形的个数.
设 $(i, j) \in S$, 并且自 $i$ 引出的线有 $d_i$ 条, 则以 $(i, j)$ 为边的三角形至少有 $d_i+d_j-n$ 个.
由于每个三角形有三条边, 所以 $S$ 中至少有 $\frac{1}{3} \sum_{(i, j) \in S}\left(d_i+d_j-n\right)$ (1)个三角形.
$\sum_{(i, j) \in S} 1=m, \sum_{(i, j) \in S} n=n m$. (2)
对于每个固定的 $i$, 恰有 $d_i$ 个 $j$ 使 $(i, j) \in S$, 所以在(1)中的 $d_i$ 出现了 $d_i$ 次.
注意 $(i, j)$ 既可作为自 $i$ 引出的边, 又可作为自 $j$ 引出的边, 被计算了 2 次.
因此 $\sum_{(i, j) \in S}\left(d_i+d_j\right)=\sum_{i=1}^n d_i^2$. 由柯西不等式, $\sum_{i=1}^n d_i^2 \geqslant \frac{1}{n}\left(\sum_{i=1}^n d_i\right)^2= \frac{1}{n}(2 m)^2=\frac{4 m^2}{n}$. 由(1)、(2)及上式得 $T \geqslant \frac{1}{3}\left(\frac{4 m^2}{n}-m m\right)=4 m \cdot \frac{m-\frac{n^2}{4}}{3 n}$.
%%PROBLEM_END%%



%%PROBLEM_BEGIN%%
%%<PROBLEM>%%
问题8 设 $L$ 是坐标平面中的一个子集.
定义如下:
$$
L=\{(41 x+2 y, 59 x+15 y) \mid x, y \in \mathbf{Z}\} .
$$
试证: 每个以原点为中心, 面积等于 1990 的平行四边形至少包含集 $L$ 中的两个点.
%%<SOLUTION>%%
设 $F$ 是以 $(0,0),(41,59),(43,74),(2,15)$ 为顶点的平行四边形, 它的 4 个顶点都属于 $L$, 且 $F$ 中其他点都不属于 $L$. 将 $F$ 在坐标平面上向各方向平移,便形成以 $F$ 为基本区域的网络, 网络的结点都是 $L$ 中的点.
$F$ 的面积
$$
S_F=\left|\begin{array}{ccc}
0 & 0 & 1 \\
41 & 59 & 1 \\
2 & 15 & 1
\end{array}\right|=497
$$
设 $P$ 是以原点为中心, 面积等于 1990 的平行四边形.
作以原点为中心, 相似比为 $\frac{1}{2}$ 的位似变换.
记平行四边形 $P$ 的位似象为 $P^{\prime}$, 则 $P^{\prime}$ 的面积为 $1990 \times \frac{1}{4}=497 \frac{1}{2}>497$. 这样一来, 当将平行四边形 $P^{\prime}$ 被网络所分成的诸块都平移到基本区域 $F$ 中时, 必有两点重叠.
设这两点是 $D_1^{\prime}\left(x_1, y_1\right)$ 和 $D_2^{\prime}\left(x_2, y_2\right)$. 由于平移是沿网格线移动的, 所以点 $M\left(x_1-x_2, y_1-y_2\right) \in L$.
另一方面, 因为 $D_1^{\prime}, D_2^{\prime} \in P^{\prime}$, 所以 $D_1\left(2 x_1, 2 y_1\right), D_2\left(2 x_2, 2 y_2\right) \in P$. 又因 $P$ 是以原点为中心的平行四边形, 所以 $D_3\left(-2 x_2,-2 y_2\right) \in P$. 从而线段 $D_1 D_3$ 的中点 $M \in P$, 且 $M \neq(0,0)$. 这就证明了平行四边形 $P$ 中至少含有 $L$ 中的两个点.
%%PROBLEM_END%%



%%PROBLEM_BEGIN%%
%%<PROBLEM>%%
问题9 证明: 在集合 $\left\{1,2,3, \cdots, \frac{3^n+1}{2}\right\}\left(n \in \mathbf{N}^*\right)$ 中可取出 $2^n$ 个数, 其中无三个数成等差数列.
%%<SOLUTION>%%
用数学归纳法.
当 $n=k+1$ 时, 可分别从 $A_1=\left\{1,2, \cdots, \frac{3^k+1}{2}\right\}$ 及 $A_2=\left\{3^k+1,3^k+2, \cdots, 3^k+\frac{3^k+1}{2}\right\}$ 中各取 $2^k$ 个数满足条件.
%%PROBLEM_END%%



%%PROBLEM_BEGIN%%
%%<PROBLEM>%%
问题10 设 $a_j 、 b_j 、 c_j$ 为整数, 这里 $1 \leqslant j \leqslant N$, 且对任意的 $j$, 数 $a_j 、 b_j 、 c_j$ 中至少有一个为奇数.
证明: 存在一组数 $r 、 s 、 t$, 使得集合 $ \{r a_j+s b_j+t c_j \mid 1 \leqslant j \leqslant N\}$ 中, 至少有 $\frac{4 N}{7}$ 个数为奇数.
%%<SOLUTION>%%
考虑不全为零的 7 个数组 $(x, y, z)$, 其中 $x, y, z \in\{0,1\}$. 容易证明: 若 $a_j 、 b_j 、 c_j$ 不全为偶数, 则集合 $A_j=\{x a_j+y b_j+z c_j \mid x, y, z \in\{0,1\}\}$ 中恰有 4 个为偶数, 也恰有 4 个为奇数, 这里 $1 \leqslant j \leqslant N$. 当然, 在 $x=y= z=0$ 时, $x a_j+y b_j+z c_j$ 为偶数.
由此可知 $ \{x a_j+y b_j+z c_j \mid x, y, z \in\{0,1\}, x 、 y, z$ 不全为零, $1 \leqslant j \leqslant N\}$ 中, 恰有 $4 N$ 个数为奇数.
于是, 由抽庶原则, 可知存在一组数 $(x, y, z), x, y, z \in\{0,1\}, x 、 y 、 z$ 不全为零, 使得 $\left\{x a_j+y b_j+z c_j \mid 1 \leqslant j \leqslant N\right\}$ 中至少有 $\frac{4 N}{7}$ 个数为奇数.
%%PROBLEM_END%%



%%PROBLEM_BEGIN%%
%%<PROBLEM>%%
问题11 平面上不含零向量的集合 $A$, 若其至少有三个元素, 且对任意 $u \in A$, 存在 $v, w \in A$, 使 $v \neq w, u=v+w$, 则称 $A$ 具有性质 $S$. 证明:
(1) 对任意 $n \geqslant 6$, 存在具有性质 $S$ 的向量集;
(2) 具有性质 $S$ 的有限向量集合都至少有 6 个元素.
%%<SOLUTION>%%
(1) 对 $n(n \geqslant 6)$ 进行归纳.
当 $n=6$ 时, 考虑 $\triangle A B C$ 及 $A B 、 B C 、 C A$ 、 $B A 、 C B 、 A C$. 对于具有性质 $S$ 的 $n$ 元集合 $A$, 设其非零向量为 $\boldsymbol{v}_1, \boldsymbol{v}_2, \cdots, v_n$. 设 $v_i 、 v_j$ 是 $A$ 的两个不同向量, $v_i$ 与 $v_j$ 的夹角是 $A$ 中各向量之间的最小角.
则 $\left(\boldsymbol{v}_i+\boldsymbol{v}_j\right) \notin A$, 否则与最小性矛盾.
因此, $A \cup\left\{\boldsymbol{v}_i+\boldsymbol{v}_j\right\}$ 有 $(n+1)$ 个元素, 且满足性质 $S$.
(2) 考虑一个均由 $O$ 为始点的具有性质 $S$ 的向量集合 $A= \{\boldsymbol{O} \boldsymbol{X}_1,\boldsymbol{O} \boldsymbol{X}_2, \cdots, \boldsymbol{O X _ { n }} \}$, 若 $\boldsymbol{u}$ 与 $\boldsymbol{v}$ 不平行, 且使得 $\boldsymbol{u}$ 或 $\boldsymbol{v}$ 平行于 $A$ 中的一个向量或 $\boldsymbol{X}_i \boldsymbol{X}_j (i \neq j)$ 中的一个向量.
记 $\boldsymbol{O} \boldsymbol{X}_i=a_i \boldsymbol{u}+b_i \boldsymbol{v}$, 对向量 $\boldsymbol{O} \boldsymbol{X}_i$ 分解, $i=1,2, \cdots, n$. 实数集合 $M=\left\{a_1, a_2, \cdots, a_n\right\}$ 具有类似于 $S$ 的性质.
设 $M$ 中的最大数为 $a$. 显然, $a>0$, 存在 $b 、 c>0$, 使得 $a=b+c, b \neq c$. 否则, $a$ 不是 $M$ 中的最大元素.
同理, 对于 $M$ 中的最小元素 $a^{\prime}$, 存在 $b^{\prime}, c^{\prime} \in M$, 且 $b^{\prime} 、 c^{\prime}<0, b^{\prime} \neq c^{\prime}$, 使得 $a^{\prime}=b^{\prime}+c^{\prime}$. 由此得出 $M$ 中的 6 个不同元素.
%%PROBLEM_END%%



%%PROBLEM_BEGIN%%
%%<PROBLEM>%%
问题12 平面上的点集 $H$ 称为是好的, 如果 $H$ 中任意 3 个点都存在一条对称轴, 使得这 3 个点关于这条对称轴对称.
证明:
(1) 一个好的集合不一定是轴对称的;
(2) 如果一个好的集合中恰有 2003 个点,则这 2003 个点在一条直线上.
%%<SOLUTION>%%
(1) 如图(<FilePath:./figures/fig-c6p12-1.png>), $\triangle A B C 、 \triangle A D C 、 \triangle B C D$ 均为等腰三角形, $A 、 B 、 D$ 也共线.
所以, 任意三个点皆有一条对称轴.
故它是一个好的集合.
但是 $A 、 B$ 、 $C 、 D$ 不是轴对称的.
(2) 反证法.
假设结论不成立.
于是, 不可能有集合中的 6 个点共线.
否则, 在这条直线外必有 1 个属于集合的点 $K$, 过点 $K$ 作此直线的垂线, 则此直线上必有至少 3 个点在这条垂线的同侧, 记为 $A 、 B 、 C$ (如图(<FilePath:./figures/fig-c6p12-2.png>)). 因为 $\angle K C B>\frac{\pi}{2}$, 所以, $B K>B C$. 由于 $K 、 C 、 B$ 有对称轴, 则 $B C=C K$. 同理, $A C=C K$,矛盾.
故不可能有集合中的 6 个点共线.
不妨设 $A 、 B$ 为这个集合中距离最短的两个点(如图(<FilePath:./figures/fig-c6p12-3.png>)). 则其余 2001 个点有以下 4 种情况:(i) 在线段 $A B$ 中垂线上; (ii) 在 $A B$ 所在直线上; (iii) 在以 $A$ 为圆心、 $A B$ 长为半径的圆上; (iv) 在以 $B$ 为圆心、 $A B$ 长为半径的圆上.
由前面的证明可知, (i)、(ii) 两种情况点的总数不超过 10 个.
又因为 $A B$ 的距离最小, 所以 (iii)、(iv)两种情况点的总数不超过 10 个.
故 $10+10+2<2003$. 矛盾.
%%PROBLEM_END%%



%%PROBLEM_BEGIN%%
%%<PROBLEM>%%
问题13 一个正整数的集合 $C$ 称为 “好集”, 是指对任何整数 $k$, 都存在着 $a, b \in C$, $a \neq b$, 使得数 $a+k$ 与 $b+k$ 不是互质的数.
证明: 如果一个好集 $C$ 的元素之和为 2003 , 则存在一个 $c \in C$, 使得集合 $C \backslash\{c\}$ 仍是一个“好集”.
%%<SOLUTION>%%
设 $p_1, p_2, \cdots, p_n$ 是 $C$ 中两个数的差的所有可能的质因子.
假定对每个 $p_i$, 都存在一个剩余 $\alpha_i$, 使得 $C$ 中至多有一个数关于模 $p_i$ 与 $\alpha_i$ 同余.
利用中国剩余定理 (即孙子定理) 可得, 存在一个整数 $k$, 满足 $k \equiv p_i-\alpha_i(\bmod p_i ), i=1,2, \cdots, n$. 利用题中的条件可得, 存在某个 $j$ 和某个 $a, b \in C$, 使得 $p_j$ 整除 $a+k$ 与 $b+k$. 于是, $a$ 和 $b$ 关于模 $p_j$ 与 $\alpha_j$ 同余.
这与 $\alpha_j$ 的假定矛盾.
由此可以断定关于模 $p$ 的每个剩余, 在 $C$ 的数的剩余中至少出现两次.
假定每个剩余都恰好出现两次, 则 $C$ 中元素的和等于 $p r+2(0+1+\cdots+ p-1)=p(r+p-1), r \geqslant 1$, 这与 2003 是质数矛盾.
因此, 一定存在某个剩余, 它至少出现三次.
将具有这种性质的 $C$ 中的元素删除一个, 就得到了一一个新的“好集”.
%%PROBLEM_END%%



%%PROBLEM_BEGIN%%
%%<PROBLEM>%%
问题14 试证: 存在一个具有如下性质的正整数的集合 $A$, 使对任何由无限多个素数组成的集合 $S$, 都存在自然数 $k \geqslant 2, m \in A$ 及 $n \notin A, m$ 和 $n$ 均为 $S$ 中 $k$ 个不同元素的乘积.
%%<SOLUTION>%%
设 $q_1, q_2, \cdots, q_j, \cdots$ 是全体素数从小到大排成的数列, 即有 $q_1=2, q_2=3, q_3=5, q_4=7, q_5=11, q_6=13, \cdots$. 令 $A_1=\left\{2 q_i \mid i=2,3, \cdots\right\}$, $A_2=\left\{3 q_i q_j \mid i<j, i=3,4, \cdots, j=4,5, \cdots\right\}$. 一般地, 对正整数 $h$, 令 $A_h=\left\{q_h q_{i_1} q_{i_2} \cdots q_{i_h} \mid h<i_1<i_2<\cdots<i_h\right\}$. 最后再令 $A==\bigcup_{h=1}^{\infty} A_h$.
设 $S$ 是由无限多个素数组成的集合 $S=\left\{p_1, p_2, \cdots, p_t, \cdots\right\}$, 其中 $p_1<p_2<\cdots<p_t<\cdots$. 于是有 $p_1 \geqslant 2, p_2 \geqslant 3, p_3 \geqslant 5, \cdots$. 设 $p_i=q_{j_i}, i=1,2, \cdots$. 于是 $m=p_1 p_2 \cdots p_{j_1+1}=q_{j_1} q_{j_2} \cdots q_{j_{j_1+1}} \in A, n=p_2 p_3 \cdots p_{j_1+2}= q_{j_2} q_{j_3} \cdots q_{j_{j_2+2}} \notin A$. 可见, 只要取 $k=j_1+1$ 就可以了.
%%PROBLEM_END%%



%%PROBLEM_BEGIN%%
%%<PROBLEM>%%
问题15 求最小的正整数 $n$, 使得 $S=\{1,2, \cdots, 150\}$ 的每个 $n$ 元子集都含有 4 个两两互质的数 (已知 $S$ 中共有 35 个素数).
%%<SOLUTION>%%
考虑 $S$ 中 2 或 3 或 5 的倍数的个数,有 $\left[\frac{150}{2}\right]+\left[\frac{150}{3}\right]+\left[\frac{150}{5}\right]- \left[\frac{150}{2 \times 3}\right]-\left[\frac{150}{2 \times 5}\right]-\left[\frac{150}{3 \times 5}\right]+\left[\frac{150}{2 \times 3 \times 5}\right]=110$. 当 $n=110$ 时, 可以全取 2 或 3 或 5 的倍数, 所以在这个子集里无论如何也找不到 4 个两两互素的数.
因此, $n \geqslant 111$.
$n=111$ 合乎要求.
为此构造如下 6 个数组: $A_1=\{1\} \bigcup\{S$ 中 35 个素数 $\}, A_2=\{2 \times 67,3 \times 43,5 \times 17,7 \times 19,11 \times 13\}, A_3=\{2 \times 53,3 \times 47,5 \times 23,7 \times 17\}, A_4=\left\{2^2, 3^2, 5^2, 7^2, 11^2\right\}, A_5= \{2 \times 19,3^3, 5 \times 13,7 \times 11\}, A_6=\left\{2^3, 3 \times 23,5 \times 11,7 \times 13\right\}$. 令 $A=A_1 \cup A_2 \cup A_3 \cup \cdots \cup A_6$. 上述每一个 $A_i(i=1,2, \cdots, 6)$ 中至少有 4 个元素并且两两互素, 且 $A_i \cap A_j=\varnothing(1 \leqslant i<j \leqslant 6),|A|=58$. 这样, 若从 $S$ 中取出 111 个数, 则 $A$ 中至少被取出 19 个数, 由抽屈原则必有某 $A_i(i=1,2, \cdots, 6)$ 被选出 4 个, 而这四个数是两两互素的.
%%PROBLEM_END%%



%%PROBLEM_BEGIN%%
%%<PROBLEM>%%
问题16 对于任意正整数 $n$, 记 $n$ 的所有正约数组成的集合为 $S_n$. 证明: $S_n$ 中至多有一半元素的个位数为 3 .
%%<SOLUTION>%%
考虑如下四种情况:
(1) $n$ 能被 5 整除, 设 $d_1, d_2, \cdots, d_m$ 为 $S_n$ 中所有个位数为 3 的元素,则 $S_n$ 中还包括 $5 d_1, 5 d_2, \cdots, 5 d_m$ 这 $m$ 个个位数为 5 的元素, 所以 $S_n$ 中至多有一半元素的个位数为 3 .
(2) $n$ 不能被 5 整除,但 $2 \mid n$. 仿 (1) 可证.
(3) $n$ 不能被 5 整除, 且 $n$ 的质因子的个位数均为 1 或 9 , 则 $S_n$ 中所有的元素的个位数均为 1 或 9. 结论成立.
(4) $n$ 不能被 5 整除, 且 $2 \times n$, 但 $n$ 有个位数为 3 或 7 的质因子 $p$, 令 $n= p^r q$, 其中 $q$ 和 $r$ 都是正整数, $p$ 和 $q$ 互质.
设 $S_q=\left\{a_1, a_2, \cdots, a_k\right\}$ 为 $q$ 的所有正约数组成的集合, 将 $S_n$ 中的元素写成如下方阵:
$$
\begin{aligned}
& a_1, a_1 p, a_1 p^2, \cdots, a_1 p^r, \\
& a_2, a_2 p, a_2 p^2, \cdots, a_2 p^r, \\
& \cdots \ldots \ldots \ldots \ldots \ldots \ldots \ldots \ldots \ldots . . \\
& a_k, a_k p, a_k p^2, \cdots, a_k p^r .
\end{aligned}
$$
对于 $d_i=a_r p^l$,选择 $a_r p^{l-1}$ 或 $a_r p^{l+1}$ 之一与之配对(所选之数必须在 $S_n$ 中). 设 $e_i$ 为所选之数, 我们称 $\left(d_i, e_i\right)$ 为一对朋友.
如果 $d_i$ 的个位数为 3 , 则由 $p$ 的个位数是 3 或 7 ,知 $e_i$ 的个位数不是 3 . 假设 $d_i$ 和 $d_j$ 的个位数都是 3 , 且有相同的朋友 $e=a_s p^t$, 则 $\left\{d_i, d_j\right\}=\left\{a_s p^{t-1}, a_s p^{t+1}\right\}$, 因为 $p$ 的个位数为 3 或 7 , 所以 $p^2$ 的个位数是 9 , 而 $n$ 不能被 5 整除, 故 $a_s$ 的个位数不为 0 , 所以 $a_s p^{t-1}, a_s p^{t-1} \cdot p^2=a_s p^{t+1}$ 的个位数不同, 这与 $d_i$ 和 $d_j$ 的个位数都是 3 矛盾, 所以每个个位数为 3 的 $d_i$ 均有不同的朋友.
综上所述, $S_n$ 中每个个位数为 3 的元素, 均与一个 $S_n$ 中个位数不为 3 的元素为朋友, 而且两个个位数为 3 的不同元素的朋友也是不同的, 所以 $S_n$ 中至多有一半元素的个位数为 3 .
%%PROBLEM_END%%



%%PROBLEM_BEGIN%%
%%<PROBLEM>%%
问题17 设 $X=\{1,2, \cdots, 2001\}$. 求最小正整数 $m$, 使其适合要求: 对 $X$ 的任何一个 $m$ 元子集 $W$,都存在 $u, v \in W$ ( $u$ 和 $v$ 可以相同),使得 $u+v$ 是 2 的方幂.
%%<SOLUTION>%%
构造一个使题中要求不被满足且又含元素最多的例子, 这个子集不能含 2 的任一方幕且每对数 $\left\{2^r+a, 2^r-a\right\}$ 中只能有 1 个含在集中.
令 $Y=\{2001,2000, \cdots, 1025\} \cup\{46,45, \cdots, 33\} \cup\{17\} \cup\{14,13, \cdots, 9\} \cup \{1\}$, 则有 $|Y|=998$. 对任何 $u 、 v \in Y$, 不妨设 $u \geqslant v$ 且有 $2^r<u \leqslant 2^r+ a<2^{r+1}$, 其中当 $r$ 分别取值 10、5、4、3 时, 相应的 $a$ 值依次为 $977 、 14 、 1 、 6$ . 对 (1) 若 $2^r<v \leqslant u$, (2) 若 $1 \leqslant v<2^r$, 可证明 $u+v$ 不能是 2 的方幂.
故知所求的最小正整数 $m \geqslant 999$.
将 $X$ 划分成下列 999 个互不相交的子集: $A_i=\{1024-i, 1024+i\}, i= 1,2, \cdots, 977, B_j=\{32-j, 32+j\}, j=1,2, \cdots, 14, C=\{15,17\}, D_k=\{8-k, 8+k\}, k=1,2, \cdots, 6, E=\{1,8,16,32,1024\}$. 对于 $S$ 的任何一个 999 元子集 $W$, 若 $W \cap E \neq \varnothing$, 则从其中任取一个元素的 2 倍都是 2 的方幕; 若 $W \cap E=\varnothing$, 则 $W$ 中的 999 个元素分属于前面的 998 个 2 元子集.
由抽庶原理知 $W$ 中必有不同的 $u$ 和 $v$, 属于其中同一子集.
显然, $u+v$ 为 2 的方幂.
故所求的最小正整数 $m=999$.
%%PROBLEM_END%%



%%PROBLEM_BEGIN%%
%%<PROBLEM>%%
问题18 平面内一个整点的有限集 $S$ 称为一个双邻集, 如果对 $S$ 内每个点 $(p, q)$, 恰有点 $(p+1, q) 、(p, q+1) 、(p-1, q) 、(p, q-1)$ 中的两点在 $S$ 内.
问对怎样的正整数 $n$, 存在一个双邻集恰包含 $n$ 个整点?
%%<SOLUTION>%%
先证明结论: 一个双邻集 $S$ 恰包含 $n$ 个整点,则 $n$ 必为偶数.
用线段连结 $S$ 中的相邻整点.
因为 $S$ 中每一整点恰连出两条线段, 因此, 双邻集 $S$ 内全部整点可用连结相邻整点的线段组成有限个不自交的闭折线图形.
在每一闭折线图形中, 每两个相邻整点的横纵坐标之和只相差 1 , 横纵坐标之和依次为偶数、奇数、偶数、奇数……交替出现.
由于是闭折线, 则任一个闭折线图形整点个数必为偶数个.
再证: $n=4$ 和 $n$ (偶数) $\geqslant 8$.
当 $n=2$ 时, 2 个整点显然无法构成一个双邻集.
当 $n=6$ 时, 由于 3 个及 3 个以下的整点无法组成闭折线图形, 则 6 个整点要成为一个双邻集, 6 个点组成的闭折线图形只能是图(<FilePath:./figures/fig-c6p18.png>) 所示的两图形之一,两图形显然都不是双邻集.
4 个整点 $(p, q) 、(p+1, q) 、(p+1, q-1) 、(p, q-1)$ 恰组成一个双邻集 (边长为 1 的正方形), 则 $n=4$.
注意到 10 个整点 $(p, q) 、(p+1, q) 、(p+2, q) 、(p+3, q) 、(p, q- 1) 、(p+3, q-1) 、(p, q-2) 、(p+1, q-2) 、(p+2, q-2) 、(p+3, q-2)$ 也组成一个双邻集 (长为 3 宽为 2 的矩形边上的 10 个整点).
因此, 当 $n=4 k(k \in \mathbf{N})$ 时, 取 $k$ 个 4 整点组成的双邻集,每两个双邻集的距离 (一个相邻集中任一点到另一双邻集中任一点距离的最小值) 大于 1 , 将这 $k$ 个 4 整点双邻集合并为一个集合, 这个集合当然是恰含 $4 k$ 个整点的双邻集.
当 $n=4 k+2(k \geqslant 2)$ 时, 由于 $n=4(k-2)+10$, 取 $k-2$ 个 4 整点组成的双邻集, 取一个 10 整点组成的双邻集, 每两个双邻集的距离大于 1 . 将这 $k-2$ 个 4 整点双邻集与一个 10 整点双邻集合并为一个集合,这个集合当然是恰含 $4 k+2(k \geqslant 2)$ 个整点的双邻集.
%%PROBLEM_END%%


