
%%TEXT_BEGIN%%
奇偶.
整数集可按照其元素能否被 2 整除划分为奇数集与偶数集.
关于奇偶性有如下基本性质:
性质 1 : 奇数不等于偶数.
性质 2 : 两个整数的和与差具有相同的奇偶性.
性质 3: $m \pm n$ 为偶数的充要条件是 $m, n$ 具有相同的奇偶性; $m \pm n$ 为奇数的充要条件是 $m, n$ 具有不同的奇偶性.
性质 4 : 奇数个奇数的和是奇数, 偶数个奇数的和是偶数.
性质 5:一个整数为奇数的充要条件是它的约数都是奇数.
性质 6:任意一个正整数都可以表示为 $n=2^p \cdot q$ 的形式, 这里 $p \in \mathbf{N}, q$为奇数.
作为整数的属性而言, 奇偶性是极基本的, 但具体解题时, 奇偶分析涉及的面很广, 并且包含了许多重要的想法和处理问题的技巧.
%%TEXT_END%%



%%PROBLEM_BEGIN%%
%%<PROBLEM>%%
例1. 设 $a_1, a_2, \cdots, a_n \in \mathbf{Z}$, 其中 $n$ 为正奇数, $\left(b_1, b_2, \cdots, b_n\right)$ 是 $\left(a_1\right.$, $\left.a_2, \cdots, a_n\right)$ 的任-一排列.
证明: $\prod_{i=1}^n\left(a_i+\left|b_i\right|\right)$ 与 $\prod_{i=1}^n\left(a_i-\left|b_i\right|\right)$ 均为偶数.
%%<SOLUTION>%%
证明:由于对整数 $x$, 总有 $|x| \equiv x \equiv-x(\bmod 2)$, 故由已知得
$$
\sum_{i=1}^n\left(a_i+\left|b_i\right|\right) \equiv \sum_{i=1}^n\left(a_i-b_i\right) \equiv \sum_{i=1}^n a_i-\sum_{i=1}^n b_i \equiv \equiv(\bmod 2),
$$
又上式左边是奇数项 ( $n$ 项) 求和, 因此必有某个 $a_i+\left|b_i\right|(1 \leqslant i \leqslant n)$ 为偶数,从而 $\prod_{i=1}^n\left(a_i+\left|b_i\right|\right)$ 为偶数.
同理可证得 $\prod_{i=1}^n\left(a_i-\left|b_i\right|\right)$ 为偶数.
证明: $\prod_{i=1}^n\left(a_i+\left|b_i\right|\right)$ 与 $\prod_{i=1}^n\left(a_i-\left|b_i\right|\right)$ 均为偶数.
%%<REMARK>%%
注:本题的各种特殊形式在竞赛题中并不少见.
纵观整个解题过程似乎平淡无奇, 却已用到了前述多条奇偶性质.
这些性质虽然简单明显, 却常常是我们平时解题的出发点.
尤其突出的一点是, 当把奇偶性看成模 2 的一种分类时, 整数取绝对值相当于作恒等运算, 加减实质上是同一种运算, 乘法则等同于某种意义下的逻辑运算, 这些事实给奇偶分析提供了极大的便利.
%%PROBLEM_END%%



%%PROBLEM_BEGIN%%
%%<PROBLEM>%%
例2. 设 $a_1, a_2, \cdots, a_n(n \geqslant 4)$ 中的每一个值或等于 1 , 或等于 -1 , 且
$$
a_1 a_2 a_3 a_4+a_2 a_3 a_4 a_5+\cdots+a_{n-1} a_n a_1 a_2+a_n a_1 a_2 a_3=0 .
$$
求证: $4 \mid n$. 
%%<SOLUTION>%%
证明:不妨记 $a_{n+i}=a_i(i=1,2,3)$, 已知条件可以写为
$$
\sum_{k=1}^n a_k a_{k+1} a_{k+2} a_{k+3}=0 . \label{eq1}
$$
由已知得, 对 $k=1,2, \cdots, n$, 有 $a_k a_{k+1} a_{k+2} a_{k+3} \in\{1,-1\}$, 因此在 式\ref{eq1} 左边的和式中取 1 的项与取 -1 的项一样多, 不妨记有 $m$ 项为 $1, m$ 项为 -1 , 故 $n$ 为偶数,且 $n=2 m$.
进一步考虑到 $\prod_{k=1}^n a_k a_{k+1} a_{k+2} a_{k+3}=\left(a_1 a_2 \cdots a_n\right)^4=1$ 恒成立, 而该式左边的值为
$$
1^m \cdot(-1)^m=(-1)^m,
$$
因此 $m$ 为偶数,故 $n=2 m$ 为 4 的倍数.
%%<REMARK>%%
注:本题的证明分两步,各自体现了奇偶分析的想法:第一步是考虑到 $n$个值为奇数 $( \pm 1)$ 的项相加等于偶数 $(0)$, 先推出 $n$ 为偶数; 第二步是用到积为正数的若干个数字中负数必有偶数个, 从而得出 $m=\frac{n}{2}$ 仍是偶数.
%%PROBLEM_END%%



%%PROBLEM_BEGIN%%
%%<PROBLEM>%%
例3. 设 $a, b$ 是正整数,且满足关系式
$$
(11111+a)(11111-b)=123456789 .
$$
求证: $a-b$ 是 4 的倍数.
%%<SOLUTION>%%
证明:由已知条件可得 $11111+a$ 与 $11111-b$ 均为奇数,所以 $a, b$ 均为偶数, 又由已知条件得
$$
11111(a-b)=a b+2468,
$$
因为 $a b$ 是 4 的倍数, $2468=4 \times 617$ 也是 4 的倍数, 所以 $11111 \times(a-b)$ 是 4 的倍数,故 $a--b$ 是 4 的倍数.
%%<REMARK>%%
注:本题通过奇偶性分析避免了过多的情况枚举.
另外,如从模 4 同余的角度考虑, 可进一步减少运算量.
%%PROBLEM_END%%



%%PROBLEM_BEGIN%%
%%<PROBLEM>%%
例4. 从集合 $\{0,1,2, \cdots, 13,14\}$ 中选出 10 个不同的数填人图 (<FilePath:./figures/fig-c7i1.png>) 中圆圈内,使每两个用线相连的圆圈中的数所成差的绝对值各不相同, 能否做到这一点? 证明你的结论.
%%<SOLUTION>%%
解:结论是否定的.
若不然,那么所说的差的绝对值共有 14 个, 它们互不相同, 并且均不大于 14 , 不小于 1 , 因此它们只能是 $1,2,3, \cdots, 14$, 从而它们的和.
$$
S=1+2+\cdots+14=7 \times 15=105
$$
是一个奇数.
另一方面, 每个圆圈与偶数个 (2 个或 4 个) 圆圈相连, 设填人的数为 $a$, 那么 $a$ 在 $S$ 中出现偶数次 (2 次或 4 次). 偶数个 $a$ 用加、减号相连, 运算结果必为偶数.
因此, $S$ 是 10 个偶数的和, 从而 $S$ 是偶数.
从上面可知, $S$ 既是奇数又要是偶数, 矛盾!
%%PROBLEM_END%%



%%PROBLEM_BEGIN%%
%%<PROBLEM>%%
例5. 已知直角坐标平面内有 $n$ 个整点, 满足任意三点不共线, 且所构成的三角形面积不是整数.
求 $n$ 的最大可能值.
%%<SOLUTION>%%
解:将平面上的整点分成如下四个集合:
$$
\begin{gathered}
M_1=\{(x, y) \mid x \equiv y \equiv 0(\bmod 2)\}, \\
M_2=\{(x, y) \mid x \equiv 1(\bmod 2), y \equiv 0(\bmod 2)\}, \\
M_3=\{(x, y) \mid x \equiv 0(\bmod 2), y \equiv 1(\bmod 2)\}, \\
M_4=\{(x, y) \mid x \equiv y \equiv 1(\bmod 2)\} .
\end{gathered}
$$
假如 $n \geqslant 5$, 由抽屉原理可知, 必有两点 $A\left(x_1, y_1\right), B\left(x_2, y_2\right)$ 属于上述同一集合, 则 $A B$ 中点 $M\left(\frac{x_1+x_2}{2}, \frac{y_1+y_2}{2}\right)$ 为整点.
任取 $n$ 个点中的另一点 $C$, 注意整点三角形面积的 2 倍必为整数, 则 $S_{\triangle A B C}=2 S_{\triangle A M C} \in \mathbf{Z}$, 矛盾.
因此 $n \leqslant 4$.
另一方面, 取 $(0,0),(1,0),(0,1),(1,1)$ 这四个点, 易验证满足题意.
综上, $n$ 的最大可能值为 4 .
%%<REMARK>%%
注:本例通过坐标的奇偶性来制造抽屉,这是研究整点问题的一种常用的技巧.
读者可以进一步思考如下两个稍难的问题:
(1)已知直角坐标平面内有 $n$ 个整点, 满足任意三点不共线, 且所构成的三角形面积为奇数.
求 $n$ 的最大可能值.
(2)已知直角坐标平面内有 $n$ 个整点, 满足任意三点不共线, 且所构成的三角形面积不是偶数.
求 $n$ 的最大可能值.
%%PROBLEM_END%%



%%PROBLEM_BEGIN%%
%%<PROBLEM>%%
例6. 设 $n \in \mathbf{N}^*$, 且使得 $37.5^n+26.5^n$ 为正整数, 求 $n$ 的值.
%%<SOLUTION>%%
解:易知 $37.5^n+26.5^n=\frac{1}{2^n}\left(75^n+53^n\right)$.
当 $n$ 为正偶数时,
$$
75^n+53^n \equiv(-1)^n+1^n \equiv 2(\bmod 4),
$$
即 $75^n+53^n=4 m+2$, 这里 $m \in \mathbf{N}^*$. 故 $37.5^n+26.5^n=\frac{1}{2^{n-1}}(2 m+1)$ 不是正整数.
当 $n$ 为正奇数时,
$$
\begin{aligned}
75^n+53^n & =(75+53)\left(75^{n-1}-75^{n-2} \cdot 53+\cdots-75 \cdot 53^{n-2}+53^{n-1}\right) \\
& =2^7 \cdot\left(75^{n-1}-75^{n-2} \cdot 53+\cdots-75 \cdot 53^{n-2}+53^{n-1}\right) .
\end{aligned}
$$
上式括号内有 $n$ 项,每一项都是奇数, 因而和为奇数.
由此可见, 只有当 $n=1,3,5,7$ 时, $37.5^n+26.5^n$ 为正整数.
%%<REMARK>%%
注:我们常常会遇到需对奇偶作分类讨论的情况, 而分类讨论的本质是 "加条件解题", 是求解数学问题的一种基本思想.
本题采用了对 $n$ 的奇偶性分类讨论的解题策略, 这与题目本身的特点有极大关系.
就 $a^n+b^n$ ( $a, b$ 为整数,例如本题中 $a=75, b=53$ ) 的结构而言, 当 $n$ 为奇数时便于因式分解, 当 $n$为偶数时则便于从模 4 的剩余类考虑问题.
在两种情形的讨论中, 人为创设的前提条件 " $n$ 为奇数" 或是 " $n$ 为偶数" 均对进一步分析问题起到了重要的作用.
%%PROBLEM_END%%



%%PROBLEM_BEGIN%%
%%<PROBLEM>%%
例7. 设 $a, b, c, d$ 为奇数, $0<a<b<c<d$, 且 $a d=b c$. 证明: 如果 $a+d=2^k, b+c=2^m$ ( $k, m$ 为整数), 则 $a=1$. 
%%<SOLUTION>%%
证明:首先因为 $a d=b c$, 且 $d-a>c-b>0$, 所以
$$
(a+d)^2=(d-a)^2+4 a d>(c-b)^2+4 b c=(b+c)^2,
$$
故 $k>m$. 又
$$
a\left(2^k-a\right)=a d=b c=b\left(2^m-b\right),
$$
从而 $2^m b-2^k a=b^2-a^2$, 即
$$
2^m\left(b-2^{k-m} a\right)=(b+a)(b-a), \label{eq1}
$$
注意 $a, b$ 为奇数,故 $b+a$ 与 $b-a$ 均为偶数,且它们的差为 $2 a \equiv 2(\bmod 4)$,故其中必有一个只能被 2 整除而不能被 4 整除, 再注意 $b-2^{k-m} a$ 为奇数, 故只有两类情形:
$$
\left\{\begin{array} { l } 
{ b + a = 2 ^ { m - 1 } u , } \\
{ b - a = 2 v }
\end{array} \text { 或 } \left\{\begin{array}{l}
b+a=2 v, \\
b-a=2^{m-1} u,
\end{array}, u, v \in \mathbf{N}^*\right.\right. \label{eq2}
$$
在其中任一情形下均有 $u v=b-2^{k-m} a<b-a \leqslant 2 v$, 故 $u=1$, 又由于
$$
b-a<b<\frac{b+c}{2}=2^{m-1},
$$
故第二类情形不会发生,故只可能
$$
\left\{\begin{array}{l}
b+a=2^{m-1} \\
b-a=2\left(b-2^{k-m} a\right)
\end{array}\right.
$$
消去 $b$ 得 $a=2^{2 m-2-k}$, 又 $a$ 为奇数,故 $a=1$.
%%<REMARK>%%
注本题中, 奇偶分析思想的重要性体现在解题的一个关键环节之中:在得到形如 式\ref{eq1} 的一个分解之后, 根据 $a, b$ 为奇数推得关于 $b+a, b-a$, $b-2^{k-m} a$ 的一系列奇偶性质,通过奇偶分析最终化整为零,归结为对 式\ref{eq2} 的两种情形的考察.
当然 式\ref{eq1} 之前的铺垫与 式\ref{eq2} 之后的收尾还涉及到放缩的思想, 整个问题具有较强的技巧性.
最后附带指出, 本题中的整数 $k, m$ 应满足 $k=2 m-2, m \geqslant 3$, 且 $a, b$, $c, d$ 的值分别为 $a=1, b=2^{m-1}-1, c=2^{m-1}+1, d=2^{2 m-2}-1$.
%%PROBLEM_END%%



%%PROBLEM_BEGIN%%
%%<PROBLEM>%%
例8. 设有一个正 $2 n+1$ 边形 $(n>1)$. 两人按如下法则做游戏: 轮流在该正多边形内画对角线; 每人每次画一条新的 (以前没有画过的)对角线,而它恰好与已画出的偶数条对角线相交 (交点在正多边形内); 凡无法按照要求画出对角线者即为负方.
问: 谁有取胜策略?
%%<SOLUTION>%%
解:将先开始的人称为甲, 后开始的人称为乙.
我们断言: 如果 $n$ 为奇数,则乙必胜; 如果 $n$ 为偶数,则甲必胜.
对正 $2 n+1$ 边形的任何一条对角线来说, 它两侧的顶点个数和为奇数, 必有一侧有偶数个顶点.
因此每条对角线与偶数条其他对角线相交.
假设到某个时刻游戏无法继续, 那么此时每条未画出的对角线都与奇数条已画的对角线相交, 也与奇数条未画的对角线相交.
这样的情况只能出现在未画的对角线条数为偶数的时刻 (事实上, 假设此时未画的对角线 $d_i$ 共奇数条, 由于每个 $d_i$ 上共有奇数个它们相互之间的交点, 因而从所有 $d_i$ 上数得的交点总数为奇数, 但每个交点恰被计数两次, 数出的交点数理应为偶数, 矛盾). 由此可知, 甲能取胜当且仅当该正 $2 n+1$ 边形的对角线总数为奇数.
在正 $2 n+1$ 边形中, 对角线共有 $\frac{(2 n+1)(2 n-2)}{2}=(n-1)(2 n+1)$ 条,
所以当 $n$ 为奇数时, 对角线有偶数条, 乙必胜; 当 $n$ 为偶数时, 则甲必胜.
而且任何一方取胜不需要制定特别的策略.
%%<REMARK>%%
注:本题不妨先对 $n=2,3$ 等较小情况予以探索, 发现 $n=2$ 时甲必胜, $n=3$ 时乙必胜, 同时也发现当 $n=3$ 时情况已经变得相当复杂, 很难真正为乙设计一种合适的取胜策略, 但另一方面又能发现, 乙获胜似乎是自然而然的, 无需特别的策略.
于是我们再回到题目条件,充分利用"正 $2 n+1$ 边形" 及"恰好与已画出的偶数条对角线相交" 这些涉及奇偶性的信息来作分析, 并结合了"算两次" 的技巧,最终获知游戏必停止于 "偶数条对角线未画" 的时刻.
从而, 一旦确定正 $2 n+1$ 边形对角线条数的奇偶性, 就能确定获胜方.
%%PROBLEM_END%%


