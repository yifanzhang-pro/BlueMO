
%%TEXT_BEGIN%%
极端原理就是一种从特殊对象看问题的方法, 它以对象数量上的极端情况, 比如最大值、最小值、最长、最短等, 为出发点, 寻找解题的突破口和答案.
极端原理作为一种解题的思想, 在几何、数论、组合、图论等方面都有着广泛的应用.
利用这个简单而又通俗的原理可以解决不少与存在性有关的数学问题和其他问题.
在具体解题中,需要我们作具体分析.
在应用极端原理时, 我们应积极利用如下事实 (1)、(2), 并注意事实 (3):
(1) 有限个数中一定有最大数和最小数;
(2) 无限个正整数中有最小数;
(3) 无限个实数不一定有最大数或最小数.
%%TEXT_END%%



%%PROBLEM_BEGIN%%
%%<PROBLEM>%%
例1. 已知 $2 n+1\left(n \in \mathbf{N}^*\right)$ 个星球两两之间距离不相等.
每个星球上的人都观察离该星球最近的一个星球.
求证: 必存在一个星球未被观察.
%%<SOLUTION>%%
证明:我们用数学归纳法证明结论.
当 $n=1$ 时,显然距离最近的一对星球相互观察, 另一个星球未被观察, 命题成立.
对 $n(n \geqslant 2)$ 的情况, 因为 $2 n+1$ 个星球两两之间距离是有限个数且两两不等, 根据事实 (1), 必有唯一的一对星球距离最近, 它们必相互观察.
现在去掉这两个星球, 根据归纳假设, 必有一个星球没被观察到, 因此对一般的 $n$ 命题也成立.
%%<REMARK>%%
注:在用数学归纳法证明中, 需将 $n$ 的情况转化为小于 $n$ 的情况.
不难发现只要两个星球距离很近, 以至于它们不去观察别的星球, 那么就能将这两个星球去掉.
根据事实 (1), 有限个距离中必有最小值, 而距离的最小性足以保证上述目的可以达到.
一般地, 在对元素个数 $n$ 施行数学归纳法证明时, 可以考虑具有某种极端性质的元素, 试图将其去掉, 从而将 $n$ 的情况转化为小于 $n$ 的情况.
%%PROBLEM_END%%



%%PROBLEM_BEGIN%%
%%<PROBLEM>%%
例2. 平面上有 $n(n \geqslant 5)$ 个点.
将它们用红、蓝两色染色.
设任何 3 个同色点不共线.
求证: 存在一个三角形使得
(1)它的三个顶点染有相同颜色;
(2)这三角形至少有一条边上不包含另一种颜色的点.
%%<SOLUTION>%%
证明:于点数 $n \geqslant 5$, 且所有的点只染两种颜色, 所以, 至少有三点同色.
因此,存在三个顶点同色的三角形.
我们在这些顶点同色的三角形中取一个面积最小的三角形.
如果这个三角形的每一条边上都有一个另一种颜色的点, 那么我们就找到了另一个三个顶点同色的三角形, 而且, 这个三角形具有更小的面积, 这是不可能的.
因此题设的三角形一定存在.
%%<REMARK>%%
注:本题证明中使用了极端原理.
我们考虑的是面积, 是一种很好的想法,值得我们仔细体会.
%%PROBLEM_END%%



%%PROBLEM_BEGIN%%
%%<PROBLEM>%%
例3. 证明不定方程 $x^3+2 y^3=4 z^3$ 没有正整数解 $(x, y, z)$.
%%<SOLUTION>%%
证明:假设方程有正整数解 $\left(x_1, y_1, z_1\right)$. 由于
$$
x_1^3+2 y_1^3=4 z_1^3,
$$
所以 $x_1^3$ 是偶数,故 $x_1$ 是偶数.
设 $x_1=2 x_2$, 则
$$
\begin{gathered}
8 x_2^3+2 y_1^3=4 z_1^3, \\
4 x_2^3+y_1^3=2 z_1^3,
\end{gathered}
$$
即故 $y_1$ 是偶数.
设 $y_1=2 y_2$, 则
$$
\begin{aligned}
& 4 x_2^3+8 y_1^3=2 z_1^3, \\
& 2 x_2^3+4 y_1^3=z_1^3,
\end{aligned}
$$
故 $z_1$ 是偶数.
设 $z_1=2 z_2$, 则
$$
2 x_2^3+4 y_1^3=8 z_2^3 .
$$
所以 $\left(x_2, y_2, z_2\right)$ 也是方程的一组正整数解, 且 $x_2<x_1$.
进而可得 $\left(x_3, y_3, z_3\right),\left(x_4, y_4, z_4\right), \cdots$ 也是方程的正整数解, 且
$$
x_1=2 x_2=2^2 x_3=\cdots,
$$
根据事实 (2), 无限个正整数中有最小数, 而正整数不可能无限递减, 矛盾.
所以原方程没有正整数解.
%%<REMARK>%%
注:本题所采用的是无穷递降法.
这个方法的特点是: 从所设条件出发, 构造出某个无穷递降的过程, 得到一系列具有某种性质的对象 $a_1, a_2, a_3$, … 当需要用反证法证明这种对象不存在时, 只需说明这个过程不能无穷延续, 此时可能会涉及事实 (2) 等形式的极端原理.
在证明的陈述上, 若将极端原理与反证法相结合, 可写得更清晰简洁一些: 假设方程有正整数解, 不妨设 $\left(x_1, y_1, z_1\right)$ 是使 $x_1$ 最小的一组正整数解, 然后照原证法推得 $\left(x_2, y_2, z_2\right)$ 也是方程的一组正整数解, 但 $x_2<x_1$, 与 $x_1$ 的最小性矛盾,所以原方程没有正整数解.
%%PROBLEM_END%%



%%PROBLEM_BEGIN%%
%%<PROBLEM>%%
例4. 任给一个 $m$ 行 $n$ 列的实数矩阵
$$
\left(\begin{array}{cccc}
a_{11} & a_{12} & \cdots & a_{1 n} \\
a_{21} & a_{22} & \cdots & a_{2 n} \\
\cdots & \cdots & \cdots & \cdots \\
a_{m 1} & a_{m 2} & \cdots & a_{m n}
\end{array}\right)
$$
一次操作指的是同时改变某一行或某一列的所有数的符号, 其余数均不变.
求证: 可经过有限次操作, 使得每一行、每一列的所有数之和均为非负数.
%%<SOLUTION>%%
证明:首先, 无论经过多少次操作, 矩阵中每个元素的绝对值不变, 所以矩阵中所有 $m n$ 个数之和 $S$ 至多 $2^{m n}$ 种可能的取值, 故在操作所能达到的一切 $S$ 的值中, 必有最大者.
取一个使 $S$ 达到最大的矩阵, 我们证明此时每一行、每一列的所有数之和均为非负数.
不然的话,不妨设第 $i$ 行各数之和小于 0 , 则对该行再进行一次操作, 新矩阵中第 $i$ 行各数之和大于 0 , 而其余数字保持不变, 于是新矩阵中各数之和必大于 $S$, 与 $S$ 的最大性矛盾!
故命题成立.
%%<REMARK>%%
注:由于所给矩阵中的元素没有具体数值, 要从正面人手直接证明经过多少步操作一定能达到要求并不可行,需另辟蹊径.
在上述证明中, 我们考虑 "使所有 $m n$ 个数之和达到最大的矩阵"这种极端情形.
方面, 根据事实 (1), 这样的矩阵确实存在; 另一方面, 因为 "每一行、每一列所有数之和均为非负数"与 "矩阵中所有数之和尽可能大"本身是和谐一致的, 因此我们试图对和最大的矩阵验证条件成立是一个自然合理的设想.
大体上说,在找某种极端的对象时,应注意与解题目的和谐一致.
%%PROBLEM_END%%



%%PROBLEM_BEGIN%%
%%<PROBLEM>%%
例5. 证明: 任意一个四面体中总有一个顶点, 使得从这个顶点引出的三条棱可构成一个三角形的三边.
%%<SOLUTION>%%
证明: 设四面体为 $A B C D$, 其中最长的棱为 $A B$. 我们证明自顶点 $A$ 或顶点 $B$ 所引出的三条棱可构成一个三角形的三边.
事实上,在 $\triangle A B C$ 与 $\triangle A B D$ 中分别有
$$
A C+C B>A B, A D+D B>A B,
$$
两式相加得
$$
A C+C B+A D+D B>2 A B,
$$
重新组合得
$$
(A C+A D)+(B C+B D)>2 A B,
$$
所以 $A C+A D>A B$ 与 $B C+B D>A B$ 中必有一项成立, 不妨设 $A C+ A D>A B$, 由于已设 $A B$ 是最长的棱, 故从顶点 $A$ 引出的三条棱可构成三角形的三边.
%%<REMARK>%%
注:一般而言, 若要考虑从某个顶点 (比如点 $A$ ) 所引出的三条棱可否构成三角形的三边, 需考察三个条件, 即 $A B, A C, A D$ 是否满足任意两数之和大于第三者.
然而, 一旦已知 $A B$ 是最长的棱, 就只需要集中精力考察 $A C+ A D>A B$ 这一个条件了.
本题中, 极端性假设所带来的效果正是使目标更为集中,最终证明了满足题意的顶点必能在 $A$ 与 $B$ 中选出.
%%PROBLEM_END%%



%%PROBLEM_BEGIN%%
%%<PROBLEM>%%
例6. 设集合 $A_1, A_2, \cdots, A_n$ 与集合 $B_1, B_2, \cdots, B_n$ 是集合 $M$ 的两个分划, 且满足对任意两个交集为空集的集合 $A_i, B_j(1 \leqslant i \leqslant n, 1 \leqslant j \leqslant n)$, 都有 $\left|A_i \cup B_j\right| \geqslant n$ (其中 $|X|$ 表示集合 $X$ 的元素个数). 求证: $|M| \geqslant \frac{n^2}{2}$.
%%<SOLUTION>%%
证明:不妨设 $A_i, B_j(1 \leqslant i \leqslant n, 1 \leqslant j \leqslant n)$ 中元素个数最少的为 $A_1$, $\left|A_1\right|=p$.
若 $p \geqslant \frac{n}{2}$, 则
$$
|M|=\left|A_1\right|+\left|A_2\right|+\cdots+\left|A_n\right| \geqslant n p \geqslant \frac{n^2}{2} .
$$
若 $p<\frac{n}{2}$, 则将 $B_1, B_2, \cdots, B_n$ 分为两类: 第一类是与 $A_1$ 的交非空的, 不妨设为 $B_1, B_2, \cdots, B_q$, 它们中每个集合的元素个数至少为 $p$; 第二类是与 $A_1$ 的交集为空集的, 为 $B_{q+1}, B_{q+2}, \cdots, B_n$, 它们中每个集合与 $A_1$ 的元素个数之和不小于 $n$, 故每个集合所含元素至少为 $n-p$ 个.
因此
$$
\begin{aligned}
|M| & =\left|B_1\right|+\left|B_2\right|+\cdots+\left|B_n\right| \geqslant p q+(n-p)(n-q) \\
& =\frac{n^2}{2}+\frac{1}{2}(n-2 p)(n-2 q) .
\end{aligned}
$$
由于 $\left|A_1\right|=p$, 且 $B_1, B_2, \cdots, B_q$ 两两交集为空集, 故 $q \leqslant p<\frac{n}{2}$, 因此
$$
(n-2 p)(n-2 q)>0,
$$
故 $|M|>\frac{n^2}{2}$.
综上可知 $|M| \geqslant \frac{n^2}{2}$.
%%<REMARK>%%
注一本题中所考虑的极端情形是 "所有 $A_i, B_j$ 中元素个数最少的集合". 由于本题所要估计的是 $|M|$ 的下界, 而 " $A_1$ 元素最少" 实则导致如下几点 "益处":
(1) 与 $A_1$ 不相交的集合 $B_j$ 较多;
(2) 每个与 $A_1$ 不相交的集合 $B_j$ 满足 $\left|A_i \cup B_j\right| \geqslant n$, 势必含有较多的元素;
(3) 与 $A_1$ 相交的那些集合 $B_j$ 不会很多;
(4) 每个与 $A_1$ 相交的集合 $B_j$ 所含元素个数不至于太小 (不能小于 $\left|A_1\right|$ ). 因此,始终围绕元素最少的集合 $A_1$ 进行讨论便在情理之中了.
解题中应充分挖掘极端元素的价值, 而在寻找极端元素时也应注意角度的选取.
注二:某题如下:
设
$$
\left(\begin{array}{cccc}
a_{11} & a_{12} & \cdots & a_{1 n} \\
a_{21} & a_{22} & \cdots & a_{2 n} \\
\cdots & \cdots & \cdots & \cdots \\
a_{n 1} & a_{n 2} & \cdots & a_{n n}
\end{array}\right)
$$
是一个由非负整数组成的方阵.
已知如果 $a_{i j}=0$, 那么
$$
a_{i 1}+a_{i 2}+\cdots+a_{i n}+a_{1 j}+a_{2 j}+\cdots+a_{n j} \geqslant n,
$$
证明 : 方阵中的所有元素之和不小于 $\frac{n^2}{2}$.
这是本例的另一种表达形式.
%%PROBLEM_END%%



%%PROBLEM_BEGIN%%
%%<PROBLEM>%%
例7. 设有 $n(\geqslant 7)$ 个圆, 其中任意 3 个圆都不两两相交 (包括相切), 求证:一定可以找到一个圆, 它至多只能与 5 个圆相交.
%%<SOLUTION>%%
证明:如图(<FilePath:./figures/fig-c6i1.png>),设 $n$ 个圆的圆心分别为 $O_1, O_2, \cdots, O_n$, 取 $n$ 个圆中半径最小的圆, 设为 $\odot O_1$.
若 $\odot O_1$ 与 6 个 (或多于 6 个圆) 圆 $O_2, O_3, \cdots, O_7$ 相交, 连接 $O_1 O_2$,
$O_1 O_3, \cdots, O_1 O_7$, 则 $\angle O_2 O_1 O_3, \angle O_3 O_1 O_4$, $\angle O_4 O_1 O_5, \angle O_5 O_1 O_6, \angle O_6 O_1 O_7, \angle O_7 O_1 O_2$ 中必有一个角不超过 $\frac{360^{\circ}}{6}=60^{\circ}$, 不妨设 $\angle O_2 O_1 O_3 \leqslant 60^{\circ}$.
连接 $\mathrm{O}_2 \mathrm{O}_3$, 令 $\odot O_1, \odot O_2, \odot O_3$ 的半径是 $r_1$, $r_2, r_3$, 则 $r_1 \leqslant r_2, r_1 \leqslant r_3$.
由 $\odot O_1$ 与 $\odot O_2$ 相交, $\odot O_1$ 与 $\odot O_3$ 相交, 得 $r_1+ r_2 \geqslant O_1 O_2, r_1+r_3 \geqslant O_1 O_3$, 则
$$
\begin{aligned}
& O_1 O_2 \leqslant r_1+r_2 \leqslant r_2+r_3, \\
& O_1 O_3 \leqslant r_1+r_3 \leqslant r_2+r_3 .
\end{aligned}
$$
而 $\angle O_2 O_1 O_3 \leqslant 60^{\circ}$, 那么其余两个角中必有一个角 $\geqslant 60^{\circ}$, 令 $\angle O_3 \geqslant 60^{\circ}$, 就有 $O_1 O_2 \geqslant O_2 O_3$, 所以有
$$
\mathrm{O}_2 \mathrm{O}_3 \leqslant \dot{r}_1+r_2 \leqslant r_2+r_3,
$$
从而知 $\odot \mathrm{O}_2$ 与 $\odot \mathrm{O}_3$ 必相交, 由此推得 $\odot O_1, \odot \mathrm{O}_2, \odot \mathrm{O}_3$ 两两相交, 这与题中的条件是矛盾的.
所以一定存在一个圆至多只能与 5 个圆相交.
%%PROBLEM_END%%



%%PROBLEM_BEGIN%%
%%<PROBLEM>%%
例8. 求所有的整数 $k$, 使得存在正整数 $a$ 和 $b$, 满足 $\frac{b+1}{a}+\frac{a+1}{b}=k$. 
%%<SOLUTION>%%
解:对于固定的 $k$, 在满足 $\frac{b+1}{a}+\frac{a+1}{b}=k$ 的数对 $(a, b)$ 中, 取一组 ( $a$, $b$ ) 使得 $b$ 最小,则
$$
x^2+(1-k b) x+b^2+b=0
$$
的一根为 $x=a$.
设另一根为 $x=a^{\prime}$, 则由 $a+a^{\prime}=k b-1$ 知 $a^{\prime} \in \mathbf{Z}$, 且 $a \cdot a^{\prime}=b(b+1)$, 因此 $a^{\prime}>0$.
又 $\frac{b+1}{a^{\prime}}+\frac{a^{\prime}+1}{b}=k$, 由 $b$ 的假定知 $a \geqslant b, a^{\prime} \geqslant b$, 因此 $a, a^{\prime}$ 必为 $b$, $b+1$ 的一个排列.
这样就有 $k=\frac{a+a^{\prime}+1}{b}=2+\frac{2}{b}$.
所以 $b=1,2$, 从而 $k=3,4$.
取 $a=b=1$ 知 $k=4$ 可取到,取 $a=b=2$ 知 $k=3$ 可取到.
所以 $k=3,4$.
%%PROBLEM_END%%



%%PROBLEM_BEGIN%%
%%<PROBLEM>%%
例9. 给定平面上不全在一条直线上的 $n$ 个点, 则必有一条直线恰好通过这 $n$ 个点中的两个点.
%%<SOLUTION>%%
证明:因为平面的 $n$ 个点不全在同一条直线上, 于是这 $n$ 个点中任意两点所确定的直线外, 都有这 $n$ 点中的其他点.
对于每条直线, 求出不在这条直线上的点到这条直线的距离.
这些距离的个数可能很多, 但总是有限多个.
因此, 根据事实 (1), 其中必有一个最小的, 不妨设为 $d_0$.
不妨设点 $P$ 到点 $A$ 和点 $B$ 所确定的直线的距离是最小距离 $d_0$. 我们证明直线 $A B$ 是通过 $n$ 个点中的恰好两个点的直线.
下面我们用反证法.
设 $P H=d_0$. 如果在直线 $A B$ 上还有 $n$ 点中的点 $C$, 则 $A 、 B 、 C$ 至少有两个点落在垂足 $H$ 的同侧, 不妨设这两个点是 $B$ 和 $C$, 它们在直线 $A B$ 上点 $H$ 的同侧, 如图(<FilePath:./figures/fig-c6i2.png>). 作直线 $P B$, 则直线 $P B$ 也是通过 $n$ 点组中两个点的一条直线.
我们过点 $C$ 作直线 $P B$ 的垂线 $C D$, 垂足为 $D$, 则有
$$
d_0 \geqslant H E \geqslant C D \text {, }
$$
这与 $P H$ 的最小性矛盾, 所以直线 $A B$ 上不能有 $n$ 点组中的其他点.
直线 $A B$ 就是我们要找的只通过两点的直线.
%%<REMARK>%%
注:这个问题是英国数学家西尔维斯特提出的,故被称为西尔维斯特问题.
该问题看似简单, 但西尔维斯特生前却没能解决它, 尔后不少数学家也曾试图给以证明,也没能成功.
这种状况持续了 50 年之久.
本题证明的关键在于考察极端.
通常在解题过程中, 一些极端性质是在操作调整的探索过程中发现并予以利用的, 例如西尔维斯特问题中, 一旦发现距离可以产生更小的距离, 问题就解决了.
另有一个与西尔维斯特问题对偶的命题: 在平面上给定 $n$ 条两两不平行的直线, 若对于他们中任何两条直线的交点, 都有这 $n$ 条直线中的另一条过这个点,则这 $n$ 条直线共点.
证明过程相仿, 留给读者练习.
%%PROBLEM_END%%


