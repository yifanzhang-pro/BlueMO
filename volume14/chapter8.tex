
%%TEXT_BEGIN%%
面积是平面几何中的一个重要概念.
在处理一些几何问题时, 以考虑面积作为计算或论证出发点的方法,称为面积方法.
面积公式不仅可用于计算面积或证明面积关系,还可用来证明与面积不明显相关的几何命题 (平面几何中几乎所有的计算与证明都能用面积来解), 有时候会收到事半功倍的效果.
三角形的面积公式是最基本的面积公式, 且具有多种形态.
借助这些公式,我们非但可以推导出其他图形的许多面积公式, 也可以得到一些与面积有关的性质定理,例如等积变形定理、共角定理、共边定理等, 由此使线段比和面积比相互转化.
%%TEXT_END%%



%%PROBLEM_BEGIN%%
%%<PROBLEM>%%
例1. 在等腰直角 $\triangle A B C$ 中, $\angle B A C=90^{\circ}$, 点 $D$ 是边 $A C$ 的中点, 过点 $A$ 作 $B D$ 的垂线与边 $B C$ 交于点 $F$. 求证: $B F=2 F C$.
%%<SOLUTION>%%
解:如图(<FilePath:./figures/fig-c8i2.png>), $\frac{B F}{F C}=\frac{S_{\triangle A B F}}{S_{\triangle A F C}}=\frac{S_{\triangle A B F}}{S_{\triangle A B D}} \cdot \frac{S_{\triangle A B D}}{S_{\triangle A F C}}$, 显然 $\angle B A F=\angle A D B, \angle F A C=\angle A B D$, 所以
$$
\begin{aligned}
& \frac{S_{\triangle A B F}}{S_{\triangle A B D}}=\frac{A B \cdot A F}{A D \cdot B D}, \\
& \frac{S_{\triangle A B D}}{S_{\triangle A F C}}=\frac{A B \cdot B D}{A F \cdot A C},
\end{aligned}
$$
因此 $\frac{B F}{F C}=\frac{A B}{A D} \cdot \frac{A B}{A C}=2$, 即 $B F=2 F C$.
%%PROBLEM_END%%



%%PROBLEM_BEGIN%%
%%<PROBLEM>%%
例2. 如图(<FilePath:./figures/fig-c8i1.png>), 在凸四边形 $A B C D$ 的边 $A B$ 和 $B C$ 上取点 $E 、 F$, 使得线段 $D E$ 、 $D F$ 分对角线 $A C$ 为三等分.
已知 $S_{\triangle A D E}=S_{\triangle C D F}=\frac{1}{4} S_{A B C D}$, 证明: $A B C D$ 是平行四边形.
%%<SOLUTION>%%
证明:设 $D E 、 D F$ 分别与 $A C$ 交于 $P 、 Q$. 连接 $B D$, 交 $A C$ 与 $M$.
由 $A P=Q C$ 得 $S_{\triangle A D P}=S_{\triangle C D Q}$, 又 $S_{\triangle A D E}=S_{\triangle C D F}$, 所以 $S_{\triangle A E P}=S_{\triangle C F Q}$. 故 $E 、 F$ 到 $A C$ 的距离相等, 因此 $E F / / A C$.
设 $\frac{A B}{A E}=\frac{C B}{C F}=k$, 则
$$
\frac{S_{\triangle A D B}}{S_{\triangle A D E}}=\frac{A B}{A E}=k, \frac{S_{\triangle C D B}}{S_{\triangle C D F}}=\frac{C B}{C F}=k,
$$
所以
$$
\begin{aligned}
S_{A B C D} & =S_{\triangle A D B}+S_{\triangle C D B}=k\left(S_{\triangle A D E}+S_{\triangle C D F}\right) \\
& =k \cdot \frac{1}{2} S_{A B C D},
\end{aligned}
$$
即 $k=2$. 因而 $\frac{A Q}{A P}=\frac{A B}{A E}=2$, 所以 $B Q / / E P$. 同理有 $B P / / F Q$. 因此 $B P D Q$ 为平行四边形, 故 $B M=M D, P M=M Q$, 又 $A P=Q C$, 所以 $A M=M C$, 即 $A C 、 B D$ 互相平分, 故而 $A B C D$ 是平行四边形.
%%<REMARK>%%
注:本题中一些重要的平行关系都是通过面积关系导出的: 先是通过面积的运算得到 $E F / / A C$, 再是通过图形关系列出 $S_{A B C D}$ 满足的面积等式, 为证明 $B Q / / E P$ 与 $B P / / F Q$ 起到桥梁作用.
面积法的特点是把各已知量和未知量用面积公式联系起来, 使几何元素之间的关系变成数量关系, 通过运算达到求证的结果, 很多场合下这可以降低分析问题或添置补助线的难度, 使证明简洁明快.
%%PROBLEM_END%%



%%PROBLEM_BEGIN%%
%%<PROBLEM>%%
例3. 设 $\triangle A B C$ 中, 顶点 $A, B, C$ 的对边分别为 $a, b, c$, 内心 $I$ 到顶点 $A, B, C$ 的距离分别为 $m, n, l$. 求证:
$$
a l^2+b m^2+c n^2=a b c .
$$
%%<SOLUTION>%%
证明:如图(<FilePath:./figures/fig-c8i3.png>), 设 $\triangle A B C$ 内切圆与三边 $B C, C A, A B$ 分别相切于 $D, E, F$.
由于 $\angle A F I=\angle A E I=90^{\circ}$, 故四边形 $A E I F$ 为圆内接四边形, 且 $A I$ 为该圆的直径.
又显然有 $A I \perp E F$, 故由四边形面积公式可得
$$
\begin{aligned}
S_{A E I F} & =\frac{1}{2} A I \cdot E F=\frac{1}{2} A I \cdot A I \sin A \\
& =\frac{1}{2} l^2 \cdot \frac{a}{2 R}=\frac{a l^2}{4 R},
\end{aligned}
$$
其中第 2、第 3 个等号分别是对 $\triangle A E F$ 与 $\triangle A B C$ 用了正弦定理, $R$ 为 $\triangle A B C$ 的外接圆半径.
同理可得 $S_{B F I D}=\frac{b m^2}{4 R}, S_{C D I E}=\frac{c n^2}{4 R}$.
所以 $S_{\triangle A B C}=S_{A E I F}+S_{B F I D}+S_{C D I E}=\frac{a l^2+b m^2+c n^2}{4 R}$.
但另一方面, $S_{\triangle A B C}=\frac{a b c}{4 R}$, 从而 $a l^2+b m^2+c n^2=a b c$.
%%<REMARK>%%
注:用两种不同的方法计算同一个面积, 得到的结果应当相等, 这是面积法的一种基本思想 (参看第 16 节"算两次"). 本题中正是将 $\triangle A B C$ 分割为 3 个四边形, 建立了面积等式.
考虑到这 3 个四边形都有外接圆, 且对角线相互垂直, 因此在用已知量表示它们面积时没有实质的困难, 而引人 $\triangle A B C$ 的外接圆半径 $R$ 又可以消去角的正弦, 起到过渡作用.
三角形与四边形的面积公式揭示了边角等基本元素之间的内在关系,而三角形的面积又常常能和内心、外心等 (及有关量) 相联系, 这是用面积证题时值得注意的一点.
例如就本题图形出发, 读者不妨证一下另一个有趣的结论: $\frac{S_{\triangle A B C}}{S_{\triangle D E F}}=\frac{2 R}{r}$, 其中 $R, r$ 分别为 $\triangle A B C$ 外接圆半径与内切圆半径.
%%PROBLEM_END%%



%%PROBLEM_BEGIN%%
%%<PROBLEM>%%
例4. 已知圆内接六边形 $A B C D E F$ 中, $A B \cdot C D \cdot E F=B C \cdot D E \cdot F A$, 证明 $A D, B E, C F$ 三线共点.
%%<SOLUTION>%%
证明: 如图(<FilePath:./figures/fig-c8i4.png>), 连接 $A C, C E, E A$. 记 $A C$ 交 $B E$ 于点 $P, C E$ 交 $A D$ 于点 $Q, E A$ 交 $C E$ 于点 $R$.
在圆内接四边形 $A B C E$ 中, $\angle B A E$ 与 $\angle B C E$ 互补, 故 $\sin \angle B A E=\sin \angle B C E$. 从而由共边定理和三角形面积公式可得:
$$
\frac{A P}{P C}=\frac{S_{\triangle B A E}}{S_{\triangle B C E}}=\frac{A B \cdot A E}{B C \cdot C E} .
$$
同理可得
$$
\begin{aligned}
& \frac{C Q}{Q E}=\frac{S_{\triangle C A D}}{S_{\triangle E A D}}=\frac{A C \cdot C D}{A E \cdot D E}, \\
& \frac{E R}{R A}=\frac{S_{\triangle E C F}}{S_{\triangle A C F}}=\frac{C E \cdot E F}{A C \cdot A F} .
\end{aligned}
$$
以上三式相乘可得
$$
\begin{aligned}
\frac{A P}{P C} \cdot \frac{C Q}{Q E} \cdot \frac{E R}{R A} & =\frac{A B \cdot A E}{B C} \cdot \frac{A C \cdot C E}{A E \cdot D E} \cdot \frac{C E \cdot E F}{A C \cdot A F} \\
& =\frac{A B \cdot C D \cdot E F}{B C \cdot D E \cdot F A}=1,
\end{aligned}
$$
又 $A D, B E, C F$ 不平行,故由 Ceva 定理的逆定理可得 $A D, B E, C F$ 三线共点.
%%<REMARK>%%
注:本题通过两种不同的方式表示 $\frac{S_{\triangle B A E}}{S_{\triangle B C E}}$ 等三个面积比例式, 建立了线段的比例关系,最后用 Ceva 定理的逆定理证明三线共点.
Ceva 定理有几种等价形式,使用时可选择适合的角度,例如本题用角元形式的 Ceva 定理书写会更为简洁, 但其本质是一样的.
%%PROBLEM_END%%



%%PROBLEM_BEGIN%%
%%<PROBLEM>%%
例5. 如图(<FilePath:./figures/fig-c8i5.png>), 延长凸四边形 $A B C D$ 的边 $A B 、 D C$ 交于点 $E$, 延长边 $A D 、 B C$ 交于点 $F$. 求证: $A C 、 B D 、 E F$ 的中点 $M 、 N 、 L$ 这三点共线 (这条线称为完全四边形 $A B C D E F$ 的"牛顿线").
%%<SOLUTION>%%
证明:连接 $M B 、 M D 、 M E 、 M F 、 N E 、 N F 、 M N$.
由 $M 、 N$ 分别是 $A C 、 B D$ 的中点可得:
$$
\begin{aligned}
S_{\triangle M D E} & =S_{\triangle M N E}+S_{\triangle M D N}+S_{\triangle E D N} \\
& =S_{\triangle M N E}+S_{\triangle M B N}+S_{\triangle E B N}=2 S_{\triangle M N E}+S_{\triangle M B E},
\end{aligned}
$$
故
$$
\begin{aligned}
S_{\triangle M N E} & =\frac{S_{\triangle M D E}-S_{\triangle M B E}}{2} \\
& =\frac{S_{\triangle A D E}-S_{\triangle C B E}}{4}=-\frac{1}{4} S_{A B C D} .
\end{aligned}
$$
同理可得
$$
S_{\triangle M N F}=\frac{S_{\triangle M B F}-S_{\triangle M D F}}{2}=\frac{S_{\triangle A B F}-S_{\triangle C D F}}{4}=\frac{1}{4} S_{A B C D} .
$$
从而 $S_{\triangle M N E}=S_{\triangle M N F}$.
由于 $E 、 F$ 在直线 $M N$ 异侧 (在已知的图形关系下, 有直线 $M N$ 与线段 $E F$ 相交), 故直线 $M N$ 平分线段 $E F$, 即 $M 、 N 、 L$ 三点共线.
%%<REMARK>%%
注:本题中,我们先充分运用 " $M 、 N$ 为 $A C 、 B D$ 中点"的条件进行面积转换.
最后, 我们利用 $S_{\triangle M N E}= S_{\triangle M N F}$ 证得另一个中点 $L$ 在直线 $M N$ 上, 事实上, 这是基于以下定理:
定理设点 $P$ 为 $\triangle A B C$ 所在平面上一点, 直线 $C P$ 与 $A B$ (或其延长线) 相交于点 $D$, 如图(<FilePath:./figures/fig-c8i6.png>), 则 $\frac{S_{\triangle A P C}}{S_{\triangle B P C}}=\frac{A D}{B D}$.
因此,若 $P$ 在 $\triangle A B C$ 内, $D$ 在线段 $A B$ 上,且 $\frac{S_{\triangle A P C}}{S_{\triangle B P C}}=\frac{A D}{B D}$, 则 $C 、 P 、 D$ 三点共线.
根据这个结论不难完成本题最后的证明步骤.
可见, 面积方法是证明三点共线的方法之一.
在组合几何方面, 也常常要从面积人手考虑一些问题.
下例是一个覆盖问题,证明中用到了面积重叠原理:
将 $n$ 个面积为 $S_i(1 \leqslant i \leqslant n)$ 的区域放人一个面积为 $S_0$ 的区域 $C$, 若这 $n$ 个区域的面积总和大于 $k S_0$, 则 $C$ 中必有一点被其中至少 $k+1$ 个区域覆盖; 若面积总和小于 $k S_0$, 则 $C$ 中必有一点被其中至多 $k-1$ 个区域覆盖.
%%PROBLEM_END%%



%%PROBLEM_BEGIN%%
%%<PROBLEM>%%
例6. 在半径为 16 的圆中有 650 个红点, 证明: 可作一个内半径为 2 、外半径为 3 的圆环 $C$, 使 $C$ 内 (不含边界) 至少含有 10 个红点.
%%<SOLUTION>%%
证明:以 650 个红点中的每一点为中心, 作内半径为 $2+\varepsilon$, 外半径为 $3- \varepsilon$ 的圆环 $C_i(1 \leqslant i \leqslant 650)$, 每个 $C_i$ 的面积为
$$
S_i=\pi\left((3-\varepsilon)^2-(2+\varepsilon)^2\right)=5 \pi(1-2 \varepsilon),
$$
其中 $0<\varepsilon<\frac{1}{2}$.
显然这些圆环都在一个半径为 19 的圆 $C_0$ 内, 其中 $C_0$ 的面积 $S_0=361 \pi$.
取定 $\varepsilon=0.0001$, 则
$$
\sum_{i=1}^{650} S_i=3250 \pi \times 0.9998>3249 \pi=9 S_0,
$$
故由面积重叠原理可知, $C_0$ 中必有一点被不少于 10 个圆环 $C_i$ 覆盖.
以这一点为中心, 作一个内半径为 2 、外半径为 3 的圆环 $C$, 其内部必含有 10 个 $C_i$ 的中心, 故圆环 $C$ 满足题意.
%%<REMARK>%%
注:面积重叠原理可谓几何上的"抽屉原理". 在考虑一些覆盖、嵌人、重叠问题时, 常通过面积来证明一些存在性的结论(有时需辅以膨胀、收缩的技巧).
%%PROBLEM_END%%



%%PROBLEM_BEGIN%%
%%<PROBLEM>%%
例7. 已知圆 $O$ 在平面直角坐标系中, 半径为 $R$, 圆周上整点个数为 $n(n \geqslant 3)$. 证明: $n<2 \pi \cdot \sqrt[3]{R^2}$.
%%<SOLUTION>%%
证明:设圆周上的所有整点按逆时针排列为 $A_1, A_2, \cdots, A_n(n \geqslant 3)$.
约定 $A_{n+1}=A_1, A_{n+2}=A_2$. 对 $i=1,2, \cdots, n$, 由于 $\triangle A_i A_{i+1} A_{i+2}$ 为整点三角形,其面积 $S_{\triangle A_i A_{i+1} A_{i+2}} \geqslant \frac{1}{2}$.
另一方面, 设 $\angle A_i O A_{i+1}=\theta_i, i=1,2, \cdots, n$, 则
$$
\begin{aligned}
& \left|A_i A_{i+1}\right|=2 R \sin \frac{\theta_i}{2},\left|A_{i+1} A_{i+2}\right|=2 R \sin \frac{\theta_{i+1}}{2}, \\
& \angle A_i A_{i+1} A_{i+2}=\frac{\pi-\theta_i}{2}+\frac{\pi-\theta_{i+1}}{2}=\pi-\frac{\theta_i+\theta_{i+1}}{2},
\end{aligned}
$$
故
$$
\begin{aligned}
S_{\triangle A_i A_{i+1} A_{i+2}} & =\frac{1}{2} \cdot\left|A_i A_{i+1}\right| \cdot\left|A_{i+1} A_{i+2}\right| \cdot \sin \angle A_i A_{i+1} A_{i+2} \\
& =\frac{1}{2} \cdot 2 R \sin \frac{\theta_i}{2} \cdot 2 R \sin \frac{\theta_{i+1}}{2} \cdot \sin \frac{\theta_i+\theta_{i+1}}{2} \\
& <2 R^2 \cdot \frac{\theta_i}{2} \cdot \frac{\theta_{i+1}}{2} \cdot \frac{\theta_i+\theta_{i+1}}{2} .
\end{aligned}
$$
所以
$$
\frac{1}{2}<\frac{R^2}{4} \cdot \theta_i \theta_{i+1}\left(\theta_i+\theta_{i+1}\right) \leqslant \frac{R^2}{16}\left(\theta_i+\theta_{i+1}\right)^3,
$$
从而
$$
\theta_i+\theta_{i+1}>\frac{2}{\sqrt[3]{R^2}} \label{eq1}
$$
注意到 $\sum_{i=1}^n \theta_i=2 \pi$, 故 式\ref{eq1} 中令 $i=1,2, \cdots, n$, 并将 $n$ 个不等式相加得
$$
4 \pi>n \cdot \frac{2}{\sqrt[3]{R^2}}
$$
即 $n<2 \pi \cdot \sqrt[3]{R^2}$.
%%<REMARK>%%
注:由于圆 $O$ 的周长 $2 \pi R$ 是 $R$ 的一阶量, 本例的结论 $n<2 \pi \cdot \sqrt[3]{R^2}$, 实际上说明了随着 $R$ 的增大, 圆周上的整点分布大致会越来越稀疏, 也因如此, 若能找到一个量来刻画这种特征, 将有助于解决问题.
上述解法中所找的量是"整点三角形的面积", 它确实可以用来刻画这种稀疏性: 在半径很大的圆周上,假设依次有三个整点 $A, B, C$ 且它们十分临近,则 $\triangle A B C$ 的面积必然小于 $\frac{1}{2}$, 与 $\triangle A B C$ 为整点三角形矛盾.
利用这一点, 我们可以估计每相邻两个圆心角之和 $\theta_i+\theta_{i+1}$ 的下界, 最终得到整点个数 $n$ 的上界.
一般的整点问题中有不少与数论有关,但本例中圆 $O$ 的圆心位置和半径并未给出有效的信息,这里,整点三角形面积的"离散性"起了关键作用.
请读者思考如何把结论推广到椭圆的情形.
%%PROBLEM_END%%


