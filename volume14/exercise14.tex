
%%PROBLEM_BEGIN%%
%%<PROBLEM>%%
问题1. 一张 $8 \times 8$ 的方格表切去对角两个方格, 这个缺角的方格表能否用 31 张 $2 \times 1$ 的骨牌覆盖?
%%<SOLUTION>%%
答案是不能把这个残缺的方格表盖住.
事实上, 把 $8 \times 8$ 的方格表按国际象棋棋盘黑白染色, 每张骨牌盖住一个黑格和一个白格.
若有能够盖住棋盘的方法, 他们将盖住 31 个黑格和 31 个白格.
但缺角方格表上 30 个格子是一种颜色的, 32 个格子是另一种颜色的.
%%PROBLEM_END%%



%%PROBLEM_BEGIN%%
%%<PROBLEM>%%
问题2. 如图(<FilePath:./figures/fig-c14p2.png>),把正方体形的房子分割成 27 个相等的小房间,每相邻两个房间 (即有公共面的两个小正方体) 都有门相通, 在中心的那个小正方体中有一只甲虫, 甲虫能从每个小房间走到与它相邻的任何一个小房间去.
如果要求甲虫只能走到每个小房间一次,那么甲虫能走遍所有的小房间吗?
%%<SOLUTION>%%
甲虫不能走遍所有的小房间.
我们如图(<FilePath:./figures/fig-c14a2.png>), 将正方体分割成 27 个小正方体 (每个小正方体表示一间房间), 涂上黑白相间的两种颜色, 使得中心的小正方体染成白色,再使两个相邻的小正方体染上不同的颜色.
显然, 在 27 个小正方体中, 14 个是黑的, 13 个是白的.
甲虫从中间的白色小正方体出发, 每走一步, 方格就改变一种颜色.
故它走 27 步, 应该经过 14 个白色的小正方体、13 个黑色的小正方体.
因此在 27 步中至少有一个小正方体, 甲虫进去过两次.
由此可见, 如果要求甲虫到每一个小房间只去一次, 那么甲虫不能走遍所有的小房间.
%%PROBLEM_END%%



%%PROBLEM_BEGIN%%
%%<PROBLEM>%%
问题3. 把正三角形划分为 $n^2$ 个同样大小的小正三角形, 把这些小正三角形的一部分标上号码 $1,2, \cdots, m$, 使号码相邻的三角形有相邻边.
求证: $m \leqslant n^2-n+1$.
%%<SOLUTION>%%
将 $n^2$ 小正三角形如图(<FilePath:./figures/fig-c14a3.png>), 黑、白染色,则黑三角形共有
$$
1+2+3+\cdots+n=\frac{1}{2} n(n+1) \text { 个, }
$$
白三角形共有
$$
1+2+3+\cdots+(n-1)=\frac{1}{2} n(n-1) \text { 个.
}
$$
由于要求"号码相邻的三角形有相邻边", 且有相邻号码的两个三角形染有不同的颜色, 因此标上号码的黑三角形至多比标上号码的白三角形多 1 个, 所以 $m \leqslant 2 \times \frac{1}{2} n(n-1)+1:=n^2-n+1$.
%%PROBLEM_END%%



%%PROBLEM_BEGIN%%
%%<PROBLEM>%%
问题4. 用不相交的对角线把凸 $n$ 边形划分成三角形,并且在多边形的每个顶点处汇集奇数个三角形.
证明: $3 \mid n$.
%%<SOLUTION>%%
用黑白两种颜色给这些三角形染色, 使得任何有公共边的两个三角形的颜色不同.
由于已知 $n$ 边形的每个顶点都是奇数个三角形的顶点,所以这 $n$ 边形的每条边都属于同一颜色的三角形 (不妨染为黑色, 如图(<FilePath:./figures/fig-c14a4.png>)所示), 可知黑色三角形比白色三角形多 $n$ 个角.
假设 $n$ 边形被分成 $x$ 个黑色三角形及 $y$ 个白色三角形, 则它们的数目之差为 $n=3 x-3 y=3(x-y)$, 为 3 的整数倍, 命题得证.
%%PROBLEM_END%%



%%PROBLEM_BEGIN%%
%%<PROBLEM>%%
问题5. 有 9 个人,其中任意 3 个人中总有 2 个互相认识.
证明: 必存在 4 人, 他们相互之间都认识.
%%<SOLUTION>%%
将人对应成点, 两人间的关系对应成两点连线的颜色, 两人不相识对应为红色, 两人相识对应为蓝色, 于是原题变成如下染色问题:
二染色 $K_9$, 设不存在红色三角形,证明必存在一个各边为蓝色的 $K_4$.
考虑一点 $A$ 引出的 8 条线段,分以下 3 种情况:
(1) 有 4 条 (或 4 条以上) 红色线段, 如图(<FilePath:./figures/fig-c14a5.png>). 现考虑 $B_1, B_2, B_3, B_4$ 间所连线段的颜色.
由题意, 不存在红色三角形, 因此 $B_1, B_2, B_3, B_4$ 组成蓝色的 $K_4$, 原命题成立.
(2). 有 6 条(或 6 条以上)蓝色线段.
现考虑这 6 条蓝色线段除 $A$ 点外的 6 个端点.
熟知它们组成的 $K_6$ 中必存在单色三角形, 由于不存在红色三角形, 因此必是蓝色三角形,此蓝色三角形加上 $A$ 点组成蓝色的 $K_4$.
(3)恰有 3 条红色线段、 5 条蓝色线段,此时另外 8 个点中必存在一点,它引出的红色线段不是 3 条, 于是可用 (1) 或 (2) 去证明.
否则, 每个点引出的红色线段都恰好 3 条,则红色线段总数为 $\frac{3 \times 9}{2}=13.5$, 这是不可能的.
%%PROBLEM_END%%



%%PROBLEM_BEGIN%%
%%<PROBLEM>%%
问题6. 平面上有 6 个点, 每三点的两两连线都组成一个不等边的三角形.
求证: 一定可以找到两对三角形, 使每对三角形的公共边既是其中一个三角形的最长边, 又是另一个三角形的最短边.
%%<SOLUTION>%%
记 6 个点为 $A_1, A_2, \cdots, A_6$. 将 $A_i A_j(1 \leqslant i<j \leqslant 6)$ 中可作为某个三角形最长边的边都染为红色, 其余边染为蓝色,构成 2 色完全图 $K_6$.
由例 1 后面的讨论知, $K_6$ 中至少有两个同色三角形 $T_1, T_2$, 但由染色方法可知 $T_1, T_2$ 的最长边均为红色, 因此 $T_1, T_2$ 为红色三角形.
对 $i=1,2$, 由于 $T_i$ 的最短边是红的, 必为另一个三角形 $T_i^{\prime}$ 的最长边, 所以 $\left(T_i, T_i^{\prime}\right)$ 是一对满足条件的三角形.
命题得证.
%%PROBLEM_END%%



%%PROBLEM_BEGIN%%
%%<PROBLEM>%%
问题7. $9 \times 9 \times 9$ 正方体的每个侧面都由单位方格组成, 用 $2 \times 1$ 的矩形沿方格线不重叠、无缝隙地贴满正方体的表面,其中会有一些 $2 \times 1$ 的矩形"跨越" 两个侧面.
求证: "跨越"两个侧面的 $2 \times 1$ 矩形的个数一定是奇数.
%%<SOLUTION>%%
将 $9 \times 9 \times 9$ 正方体每个侧面上的方格都相间染成黑色与白色, 且使得各个角上的方格均为黑色.
这种染色方案导致如下结论:
(1)每个面上都有 41 个黑格和 40 个白格,黑格总数比白格多 6 个;
(2)每个 $2 \times 1$ 矩形中的两个方格同色当且仅当这个矩形"跨越"两个侧面.
所以, 在 "跨越"两个侧面的 $2 \times 1$ 矩形中, 黑色矩形比白色矩形多 3 个, 因此它们的个数之和为奇数.
%%PROBLEM_END%%



%%PROBLEM_BEGIN%%
%%<PROBLEM>%%
问题8. 在 $8 \times 8$ 的棋盘的每个方格中任写一个正整数,然后施行以下操作: 任取一个 $3 \times 3$ 或 $4 \times 4$ 的正方形 "子棋盘", 将其中每个数都加 1 . 能否经过有限次操作,使棋盘中每个数字都是 10 的倍数?
%%<SOLUTION>%%
如图(<FilePath:./figures/fig-c14a8-1.png>), 的方法将棋盘中的 24 个方格涂上颜色.
假如一开始染色方格上的所有数之和 $S$ 为奇数 (这很容易做到), 那么由于每个 $3 \times 3$ 或 $4 \times 4$ "子棋盘" 恰好覆盖这 24 个方格中的偶数个, 因此每次操作必使 $S$ 增加一个偶数, 从而 $S$ 总为奇数,不可能使所有方格内都出现 10 的倍数.
%%<REMARK>%%
注:本题的染色方法有很多种, 例如按如图(<FilePath:./figures/fig-c14a8-2.png>), 的染色方法可以证明更强的命题: 即使允许对任意一个 $2 \times 2$ 的 "子棋盘"进行每个数都加 1 的操作,仍不能保证棋盘中的数都变成偶数.
%%PROBLEM_END%%



%%PROBLEM_BEGIN%%
%%<PROBLEM>%%
问题9. 如图(<FilePath:./figures/fig-c14p9.png>), $8 \times 9$ 的方格表上已经放置了 6 块 $1 \times 2$ 的多米诺骨牌.
问方格表内最多还能放置多少块两两不重叠的多米诺骨牌?
%%<SOLUTION>%%
如图(<FilePath:./figures/fig-c14a9.png>), 设方格表中已被多米诺骨牌占据的 12 个小方格的集合为 $C$; 右上角、 左下角的小方格分别为 $B 、 D ; B \cup C \cup D$ 左上方的小方格集合为 $A ; B \cup C \cup D$ 右下方的小方格集合为 $E$.
将方格表黑白相间染色, 使右上角 $B$ 为白格, 则推算知: $D$ 为黑格; $A$ 中含有 16 个白格、13 个黑格; $E$ 中含有 16 个黑格、13 个白格.
再放人一块多米诺骨牌必须全属于 $A \cup B \cup D$ 或全属于 $E \cup B \cup D$, 每一集合至多包含 14 块多米诺骨牌 (因为每块骨牌占据一个黑格和一个白格), 因此最多还能两两不重叠地放置不超过 28 块多米诺骨牌.
易知可以如图所示放 28 块骨牌, 因此本题所求结果为 28 .
%%PROBLEM_END%%


