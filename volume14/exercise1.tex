
%%PROBLEM_BEGIN%%
%%<PROBLEM>%%
问题1. 求 $x \sqrt{1-y^2}+y \sqrt{1-x^2}$ 的最大值.
%%<SOLUTION>%%
解:法一:令 $x=\sin \alpha, y=\sin \beta$, 其中 $\alpha, \beta \in\left[-\frac{\pi}{2}, \frac{\pi}{2}\right]$, 则
$$
x \sqrt{1-\overline{y^2}}+y \sqrt{1-x^2}=\sin \alpha \cos \beta+\sin \beta \cos \alpha=\sin (\alpha+\beta) \leqslant 1,
$$
等号可在 $\alpha+\beta=\frac{\pi}{2}$ 时取到.
%%PROBLEM_END%%



%%PROBLEM_BEGIN%%
%%<PROBLEM>%%
问题1. 求 $x \sqrt{1-y^2}+y \sqrt{1-x^2}$ 的最大值.
%%<SOLUTION>%%
解:法二:由基本不等式可以证明如下两个"部分不等式"
$$
x \sqrt{1-y^2} \leqslant \frac{1}{2}\left(x^2+1-y^2\right), y \sqrt{1-x^2} \leqslant \frac{1}{2}\left(y^2+1-x^2\right),
$$
相加即可.
%%PROBLEM_END%%



%%PROBLEM_BEGIN%%
%%<PROBLEM>%%
问题2. 设 $b>a>e$, 证明: $a^b>b^a$.
%%<SOLUTION>%%
取对数后只须证 $b \ln a>a \ln b$, 只须证 $f(x)=\frac{\ln x}{x}(x>\mathrm{e})$ 单调减.
此时显然 $f^{\prime}(x)=\frac{1-\ln x}{x^2}<0$, 故结论成立.
%%<REMARK>%%
注:本题用"作商法"也可证明,而"作差法" 将使问题大大复杂化.
%%PROBLEM_END%%



%%PROBLEM_BEGIN%%
%%<PROBLEM>%%
问题3. 5 这个数, 可以写成 3 个正整数之和, 如果计入不同的顺序, 则有 6 种方式, 即
$5=1+1+3=1+3+1=3+1+1=1+2+2=2+1+2=2+2+1$.
设 $m 、 n$ 都是正整数, 且 $m \leqslant n$, 问 $n$ 可以用多少种方式写为 $m$ 个正整数之和 (计入顺序)?
%%<SOLUTION>%%
把 $n$ 写成 $n$ 个 1 的和:
$$
n=1+1+\cdots+1 \text {. }
$$
我们要求的方式数就转化为从上式中的 $n-1$ 个加号中选 $m-1$ 个的选法数, 即 $\mathrm{C}_{n^{-1}}^{m-1}$.
%%<REMARK>%%
注:计数问题常可用一个集合中的元素与另一其元素更易计数的集合 中的元素"对应"的办法来"化简".
%%PROBLEM_END%%



%%PROBLEM_BEGIN%%
%%<PROBLEM>%%
问题4. 设实数 $x 、 y 、 z$ 大于或等于 1 , 求证:
$$
\left(x^2-2 x+2\right)\left(y^2-2 y+2\right)\left(z^2-2 z+2\right) \leqslant(x y z)^2-2 x y z+2 .
$$
%%<SOLUTION>%%
注意到 $x \geqslant 1, y \geqslant 1$, 所以
$$
\begin{aligned}
& \left(x^2-2 x+2\right)\left(y^2-2 y+2\right)-\left((x y)^2-2 x y+2\right) \\
= & (-2 y+2) x^2+\left(6 y-2 y^2-4\right) x+\left(2 y^2-4 y+2\right)
\end{aligned}
$$
$$
\begin{aligned}
& =-2(y-1)\left(x^2+(y-2) x+1-y\right) \\
& =-2(y-1)(x-1)(x+y-1) \leqslant 0,
\end{aligned}
$$
所以
$$
\left(x^2-2 x+2\right)\left(y^2-2 y+2\right) \leqslant(x y)^2-2 x y+2 .
$$
同理, 因为 $x y \geqslant 1, z \geqslant 1$, 所以
$$
\left((x y)^2-2 x y+2\right)\left(z^2-2 z+2\right) \leqslant(x y z)^2-2 x y z+2 .
$$
从而命题得证.
%%PROBLEM_END%%



%%PROBLEM_BEGIN%%
%%<PROBLEM>%%
问题5. 已知 $n$ 个实数 $x_1, x_2, \cdots, x_n$ 的算术平均值为 $a$, 证明:
$$
\sum_{k=1}^n\left(x_k-a\right)^2 \leqslant \frac{1}{2}\left(\sum_{k=1}^n\left|x_k-a\right|\right)^2 .
$$
%%<SOLUTION>%%
当 $a=0$ 时,有
$$
0=\left(x_1+x_2+\cdots+x_n\right)^2=\sum_{k=1}^n x_k^2+2 \sum_{1 \leqslant i<j \leqslant n} x_i x_j,
$$
所以于是
$$
\begin{aligned}
& \sum_{k=1}^n \dot{x}_k^2=-2 \sum_{1 \leqslant i<j \leqslant n} x_i x_j, \\
& \sum_{k=1}^n x_k^2 \leqslant 2 \sum_{1 \leqslant i<j \leqslant n}\left|x_i x_j\right|,
\end{aligned}
$$
从而
$$
2 \sum_{k=1}^n x_k^2 \leqslant \sum_{k=1}^n x_k^2+2 \sum_{1 \leqslant i<j \leqslant n}\left|x_i x_j\right|=\left(\sum_{k=1}^n\left|x_k\right|\right)^2,
$$
即 $a=0$ 时, 不等式成立.
当 $a \neq 0$ 时, 令 $y_k=x_k-a, k=1,2, \cdots, n$, 则 $y_1, y_2, \cdots, y_n$ 的算术平均值为 0 , 利用上面已经证明的结果, 可得
$$
\sum_{k=1}^n y_k^2 \leqslant \frac{1}{2}\left(\sum_{k=1}^n\left|y_k\right|\right)^2,
$$
故
$$
\sum_{k=1}^n\left(x_k-a\right)^2 \leqslant \frac{1}{2}\left(\sum_{k=1}^n\left|x_k-a\right|\right)^2 .
$$
%%<REMARK>%%
注:本题中 $a=0$ 这种情形是容易证明的.
当 $a \neq 0$ 时, 我们通过一个代换, 把它化归为 $a=0$ 时的情形,进而求得解答,这是一种常用的手法.
%%PROBLEM_END%%



%%PROBLEM_BEGIN%%
%%<PROBLEM>%%
问题6. 设 $P$ 是三角形 $A B C$ 内部的一个点, $D 、 E 、 F$ 分别是由 $P$ 向线段 $B C$ 、 $C A 、 A B$ 作垂线所得的垂足,求使
$$
\frac{B C}{P D}+\frac{C A}{P E}+\frac{A B}{P F}
$$
达到最小时点 $P$ 的位置.
%%<SOLUTION>%%
设 $B C 、 C A 、 A B$ 的长度分别为 $a 、 b 、 c ; P D 、 P E 、 P F$ 的长度分别为 $p 、 q 、 r$. 我们要在 $\triangle A B C$ 的内部找一点 $P$, 使 $\frac{a}{p}+\frac{b}{q}+\frac{c}{r}$ 达到最小.
由于
$$
\begin{aligned}
S_{\triangle A B C} & =S_{\triangle P B C}+S_{\triangle P C A}+S_{\triangle P A B} \\
& =\frac{1}{2} a p+\frac{1}{2} b q+\frac{1}{2} c r,
\end{aligned}
$$
所以, $a p+b q+c r=2 S_{\triangle A B C}$ 是一个与 $P$ 点的位置无关的常数, 因此, 我们可以用 $(a p+b q+c r)\left(\frac{a}{p}+\frac{b}{q}+\frac{c}{r}\right)$ 取最小值来代替使 $\frac{a}{p}+\frac{b}{q}+\frac{c}{r}$ 取最小值(问题已被转化).
由柯西不等式
$$
(a p+b q+c r)\left(\frac{a}{p}+\frac{b}{q}+\frac{c}{r}\right) \geqslant(a+b+c)^2,
$$
并且当 $a p \times \frac{p}{a}=b q \times \frac{q}{b}=c r \times \frac{r}{c}$, 即 $p=q=r$ 时上述不等式取等号, 从而当 $p=q=r$ 时, $(a p+b q+c r)\left(\frac{a}{p}+\frac{b}{q}+\frac{c}{r}\right)$ 取最小值, 也就是说, 当 $P$ 为 $\triangle A B C$ 的内心时, $\frac{a}{p}+\frac{b}{q}+\frac{c}{r}$ 取到最小值.
%%PROBLEM_END%%



%%PROBLEM_BEGIN%%
%%<PROBLEM>%%
问题7. 已知 $t$ 为一元二次方程 $x^2-3 x+1=0$ 的根.
(1) 对任一给定的有理数 $a$, 求有理数 $b, c$, 使得 $(t+a)(b t+c)=1$ 成立;
(2) 将 $\frac{1}{t^2+2}$ 表示成 $d t+e$ 的形式, 其中 $d, e$ 为有理数.
%%<SOLUTION>%%
(1) 解方程得 $t=\frac{3 \pm \sqrt{5}}{2}$ 是无理数, 由 $(t+a)(b t+c)=1$ 得
$$
b t^2+(a b+c) t+a c-1=0 .
$$
因为 $t^2-3 t+1=0$, 所以 $t^2=3 t-1$, 于是上式可化为
$$
(3 b+a b+c) t-b+a c-1=0,
$$
由于 $t$ 是无理数,所以
$$
\left\{\begin{array}{l}
3 b+a b+c=0 \\
-b+a c-1=0
\end{array}\right.
$$
因为 $a, b$ 是有理数,所以 $a^2+3 a+1 \neq 0$, 由上面方程组解得
$$
b=-\frac{1}{a^2+3 a+1}, c=-\frac{a+3}{a^2+3 a+1} \text {. }
$$
(2)因为 $t^2+2=(3 t-1)+2=3 t+1=3\left(t+\frac{1}{3}\right)$, 由(1)知, 对 $a=\frac{1}{3}$, 有
$$
b=-\frac{1}{a^2+3 a+1}=-\frac{9}{19}, c=\frac{a+3}{a^2+3 a+1}=\frac{30}{19},
$$
使得
$$
\left(t+\frac{1}{3}\right)\left(-\frac{9}{19} t+\frac{30}{19}\right)=1,
$$
所以
$$
\frac{1}{t^2+2}=\frac{1}{3}\left(-\frac{9}{19} t+\frac{30}{19}\right)=-\frac{3}{19} t+\frac{10}{19} \text {. }
$$
%%<REMARK>%%
注:本题的第 (2) 小题, 我们就利用了 (1) 的结论, 从而把陌生问题化归为熟悉问题.
%%PROBLEM_END%%



%%PROBLEM_BEGIN%%
%%<PROBLEM>%%
问题8. 设 $x_1, x_2, \cdots, x_n$ 是整数, 并且满足:
(1) $-1 \leqslant x_i \leqslant 2, i=1,2, \cdots, n$;
(2) $x_1+x_2+\cdots+x_n=19$;
(3) $x_1^2+x_2^2+\cdots+x_n^2=99$.
求 $x_1^3+x_2^3+\cdots+x_n^3$ 的最大值和最小值.
%%<SOLUTION>%%
设 $x_1, x_2, \cdots, x_n$ 中有 $r$ 个- $1, s$ 个 $1, t$ 个 2 , 由题设得
$$
\left\{\begin{array}{l}
-r+s+2 t=19 \\
r+s+4 t=99
\end{array}\right.
$$
可得
$$
\left\{\begin{array}{l}
r=40-t \\
s=59-3 t
\end{array}\right.
$$
所以
$$
\left\{\begin{array}{l}
r=40-t \geqslant 0, \\
s=59-3 t \geqslant 0, \\
t \geqslant 0
\end{array}\right.
$$
故
$$
0 \leqslant t \leqslant 19 \text {. }
$$
$$
x_1^3+x_2^3+\cdots+x_n^3=--r+s+8 t=(19-2 t)+8 t=6 t+19,
$$
所以
$$
19 \leqslant x_1^3+x_2^3+\cdots+x_n^3 \leqslant 133 .
$$
又当 $r=40, s=59, t=0$ 时, $x_1^3+x_2^3+\cdots+x_n^3=19$; 当 $r=21, s= 2, t=19$ 时, $x_1^3+x_2^3+\cdots+x_n^3=133$, 所以 $x_1^3+x_2^3+\cdots+x_n^3$ 的最小值为 19 , 最大值为 133 .
%%PROBLEM_END%%



%%PROBLEM_BEGIN%%
%%<PROBLEM>%%
问题9. 设 $x, y$ 为非负整数, 使得 $x+2 y$ 是 5 的倍数, $x+y$ 是 3 的倍数, 且 $2 x+ y \geqslant 99$, 求 $7 x+5 y$ 的最小值.
%%<SOLUTION>%%
设 $x+2 y=5 a, x+y=3 b$, 则 $x=6 b-5 a, y=5 a-3 b$. 于是 $2 x+ y=9 b-5 a, 7 x+5 y=27 b-10 a$. 故
$$
\left\{\begin{array}{l}
6 b-5 a \geqslant 0, \\
5 a-3 b \geqslant 0, \\
9 b-5 a \geqslant 99,
\end{array}\right.
$$
所以 $9 b \geqslant 5 a+99 \geqslant 3 b+99$, 可得 $b \geqslant 17$, 于是 $5 a \geqslant 3 b \geqslant 51$, 可得 $a \geqslant 11$, 进而 $9 b \geqslant 5 a+99 \geqslant 3 \cdot 11+99$, 可得 $b \geqslant 18$.
若 $b=18$, 则 $5 a \leqslant 9 b-99=63, a \leqslant 12$, 从而
$$
7 x+5 y=27 b-10 a \geqslant 27 \times 18-10 \times 12=366 .
$$
若 $b>18$, 则 $7 x+5 y=27 b-10 a=9 b+2(9 b-5 a) \geqslant 9 \times 19+2 \times 99=369$.
综上所述, $7 x+5 y$ 的最小值为 366 .
%%PROBLEM_END%%



%%PROBLEM_BEGIN%%
%%<PROBLEM>%%
问题10. 设 $x 、 y 、 z$ 是正实数, 且满足 $x y z+x+z=y$, 求
$$
p=\frac{2}{x^2+1}-\frac{2}{y^2+1}+\frac{3}{z^2+1}
$$
的最大值.
%%<SOLUTION>%%
由已知条件得 $x+z=(1-x z) y$, 显然, $1-x z \neq 0$, 所以 $y= \frac{x+z}{1-x z}$. 由此联想到正切和公式, 于是令
$$
\alpha=\arctan x, \beta=\arctan y, \gamma=\arctan z, \alpha, \beta, \gamma \in\left(0, \frac{\pi}{2}\right),
$$
则
$$
\tan \beta=\frac{\tan \alpha+\tan \gamma}{1-\tan \alpha \tan \gamma}=\tan (\alpha+\gamma) .
$$
由于 $\beta, \alpha+\beta \in(0, \pi)$, 所以 $\beta=\alpha+\gamma$. 于是
$$
\begin{aligned}
p & =\frac{2}{\tan ^2 \alpha+1}-\frac{2}{\tan ^2 \beta+1}+\frac{3}{\tan ^2 \gamma+1} \\
& =2 \cos ^2 \alpha-2 \cos ^2(\alpha+\gamma)+3 \cos ^2 \gamma \\
& =(\cos 2 \alpha+1)-[\cos (2 \alpha+2 \gamma)+1]+3 \cos ^2 \gamma \\
& =2 \sin \gamma \cdot \sin (2 \alpha+\gamma)+3 \cos ^2 \gamma \\
& \leqslant 2 \sin \gamma+3\left(1-\sin ^2 \gamma\right) \\
& =-3\left(\sin \gamma-\frac{1}{3}\right)^2+\frac{10}{3} \leqslant \frac{10}{3}
\end{aligned}
$$
等号在 $2 \alpha+\gamma==\frac{\pi}{2}, \sin \gamma=\frac{1}{3}$, 即 $a=\frac{\sqrt{2}}{2}, b=\sqrt{2}, c=\frac{\sqrt{2}}{4}$ 时成立, 故欲求的最大值为 $\frac{10}{3}$.
%%PROBLEM_END%%


