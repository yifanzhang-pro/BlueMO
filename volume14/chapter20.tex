
%%TEXT_BEGIN%%
图论是以图为研究对象, 研究顶点和边组成的图形的数学理论和方法, 起源于著名的哥尼斯堡七桥问题.
图论中的图是指由若干个不同的顶点及连接其中某些顶点的边所构成的图形, 通常用 $G$ 表示, 或者更确切地记作 $G(V, E)$, 其中 $V$ 是所有顶点的集合, $E$ 是所有边的集合.
图 $G$ 中, 顶点的位置以及边的曲直长短都是无关紧要的, 我们所关心的是图 $G$ 中顶点和边的组成状况.
图论有一套庞大的概念系统, 下面列举的是其中最基本的概念以及一些本节中会涉及到的概念:
如果图 $G$ 的两个顶点 $v_1, v_2$ 之间有边相连, 则称 $v_1, v_2$ 相邻, 否则称 $v_1$, $v_2$ 不相邻.
如果一条边的两端是同一顶点, 这样的边称为环.
如果两个顶点之间有 $k(k \geqslant 2)$ 条边相连, 那么这些边称为重边.
若一个图既没有环也没有重边, 这样的图称为简单图.
每两个顶点均相邻的简单图称为完全图.
有 $n$ 个顶点的图称为 $\boldsymbol{n}$ 阶图.
其中 $n$ 阶完全图记为 $K_n$.
顶点数和边数都有限的图称为有限图, 否则称为无限图.
图 $G$ 中, 与顶点 $v$ 相邻的边数 (环作两条边计算) 称为顶点 $v$ 的度 (或者次数), 记做 $d(v)$. 若顶点的度是奇数, 则称为奇顶点, 否则称为偶顶点.
若图中的边不考虑起点和终点, 则称为无向图, 否则称为有向图 (有向图有出度和入度的概念). 如无特别说明,一般的图都指无向的简单图.
在图 $G$ 中,一个由不同的边组成的序列: $e_1, e_2, \cdots, e_m$ (其中 $e_i=\left(v_{i-1}\right.$, $\left.\left.v_i\right), i=1,2, \cdots, m\right)$ 称为从 $v_0$ 到 $v_m$ 的链, 其中 $v_0$ 和 $v_m$ 称为这条链的端点, 数 $m$ 称为这条链的长.
如果一条链的两个端点重合, 则称这条链为圈.
如果对图的任何两个顶点 $u, v$, 都存在一条链以 $u, v$ 为端点, 这样的图称为连通图.
关于图 $G$ 的顶点和边数之间的关系, 有如下定理.
定理图 $G$ 中边数的两倍等于顶点度数之和.
设 $G$ 中 $n$ 个顶点为 $v_1, v_2, \cdots, v_n$, 边数为 $e$, 则
$$
d\left(v_1\right)+d\left(v_2\right)+\cdots+d\left(v_n\right)=2 e .
$$
证明所有顶点的度的和 $d\left(v_1\right)+d\left(v_2\right)+\cdots+d\left(v_n\right)$ 表示以顶点 $v_1, v_2$, $\cdots, v_n$ 中某个顶点为端点的边的总数.
由于一条边有两个顶点, 所以图 $G$ 中每条边在和 $d\left(v_1\right)+d\left(v_2\right)+\cdots+d\left(v_n\right)$ 中被计数了两次.
即证.
这个定理通常称为握手引理: 如果许多人在一起握手, 那么握手次数为偶数次, 从而握过奇数次手的人有偶数个.
即得推论推论 图 $G$ 中奇顶点的个数一定是偶数个.
一笔画, 就是纸上给定一个图 $G$, 能否从图 $G$ 的一个顶点出发, 笔不离开纸, 而且连续地沿着每条边恰好一次, 然后回到原来顶点, 从而画出整个图 $G$. 如果图是欧拉图,则可以一笔画出整个图 $G$, 否则不能.
欧拉给出过一个图是否是欧拉图的判别方法.
一笔画定理一个连通图为欧拉图的充要条件是每个顶点的度都是偶数.
由此可以推出,一个图可以一笔画的充要条件是没有奇顶点或者两个奇顶点.
如果有两个奇顶点, 那么这两个奇顶点是一笔画的起始点和结束点.
本节所选的大多数例题和习题本身并非图论问题,但我们采用图论方法求解, 旨在反映图论应用的广泛性与灵活性.
%%TEXT_END%%



%%PROBLEM_BEGIN%%
%%<PROBLEM>%%
例1. $n$ 名选手进行网球对抗赛, 每名选手至多赛一场,每场比赛两名选手参加,已赛完 $n+1$ 场.
证明: 至少有一名选手赛过三次.
%%<SOLUTION>%%
证明: $n$ 名选手用 $n$ 个点 $v_1, v_2, \cdots, v_n$ 表示, 当且仅当 $v_i, v_j$ 所代表的两名选手比赛过时, 令 $v_i, v_j$ 相邻, 于是得到一个含 $n$ 个顶点的简单图.
由于总共赛过 $n+1$ 场, 所以图 $G$ 的边数是 $n+1$. 由定理知
$$
d\left(v_1\right)+d\left(v_2\right)+\cdots+d\left(v_n\right)=2(n+1),
$$
如果图 $G$ 中所有顶点的度都不超过 2 , 则由上式得到
$$
2(n+1)=d\left(v_1\right)+d\left(v_2\right)+\cdots+d\left(v_n\right) \leqslant 2 n,
$$
这不可能.
因此图 $G$ 中至少有一个顶点 $x$, 它的度至少是 3. 于是, 顶点 $x$ 所表示的选手至少赛过三次.
%%PROBLEM_END%%



%%PROBLEM_BEGIN%%
%%<PROBLEM>%%
例2. 设 $S=\left\{x_1, x_2, \cdots, x_n\right\}$ 是平面上的点集,其中 $n \geqslant 3$. 若任意两点之间的距离不小于 1 , 证明: 距离恰好等于 1 的点对不超过 $3 n$ 对.
%%<SOLUTION>%%
证明:这 $n$ 个点为顶点, 两顶点相邻当且仅当它们之间距离为 1 , 得图 G.
对每个 $i \in\{1,2, \cdots, n\}, G$ 中和顶点 $x_i$ 相邻的点在以 $x_i$ 为圆心、 1 为半径的圆周上.
由于 $S$ 中任意两点间距离不小于 1 , 故圆周上至多含有 $S$ 中的 6 个点, 所以 $x_i$ 的度数 $d_i \leqslant 6$.
设 $G$ 的边数为 $e$, 则有
$$
2 e=d_1+d_2+\cdots+d_n \leqslant 6 n,
$$
即 $G$ 的边数 $e$ 不超过 $3 n$. 所以距离等于 1 的点对不超过 $3 n$ 对.
%%<REMARK>%%
注:这里我们利用图论方法证明组合几何命题.
此外, 如果从凸包的角度考虑, 不难将题目中的上界 $3 n$ 加强为 $3 n-6$.
%%PROBLEM_END%%



%%PROBLEM_BEGIN%%
%%<PROBLEM>%%
例3. 空间中的 $n$ 条直线满足任意三条中必有两条异面, 且不存在 3 条两两异面的直线, 求 $n$ 的最大可能值.
%%<SOLUTION>%%
解:先证明 $n \leqslant 5$.
假设存在 6 条直线满足题意, 将这 6 条直线 $l_i$ 对应到 6 个顶点 $v_i(1 \leqslant i \leqslant 6)$ ,当 $l_i$ 与 $l_j$ 共面时,在 $v_i, v_j$ 间连一条红边; 当 $l_i$ 与 $l_j$ 异面时,在 $v_i, v_j$ 间连一条蓝边,构成一个 2 色完全图 $K_6$. 由 Ramsey 定理可知, 2 色完全图 $K_6$ 中必存在同色三角形,这说明有 3 条线两两共面或两两异面, 但均与已知条件矛盾! 故假设不成立.
因此 $n \leqslant 5$.
当 $n=5$ 时,如图(<FilePath:./figures/fig-c20i1.png>) 作正方体 $A B C D- E F G H$, 记 $E H$ 中点为 $M$, 易验证直线 $A B, B F, F H$, $H M, M A$ 满足题意.
综上, $n$ 的最大可能值为 5 .
%%<REMARK>%%
注:本题中,若只从空间形态来考虑问题将不胜其烦, 而一旦转化成图论语言, 回到了拉姆赛(Ramsey)问题这一图论中的重要背景(参考第 14 节例 1), 解题目标就变得十分明确了.
%%PROBLEM_END%%



%%PROBLEM_BEGIN%%
%%<PROBLEM>%%
例4. 已知 $\triangle A B C$ 内有 $m$ 个点, 连同 $A, B, C$ 三点一共 $m+3$ 个点.
以这些点为顶点将 $\triangle A B C$ 分成若干个互不重叠的小三角形.
将 $A, B, C$ 三点分别染成红色、黄色、蓝色.
而三角形内的 $m$ 个点, 每个点任意染成红色、黄色、 蓝色三色之一.
问: 三个顶点颜色都不同的小三角形的个数是奇数还是偶数?
%%<SOLUTION>%%
证明:这样的方法构造一个图 $G$ : 在 $\triangle A B C$ 外及每个小三角形内各取一点代表它们所在的区域,这些点构成图 $G$ 的顶点集合; 当两个顶点 $u, v$ 所在区域有一条公共边, 且公共边端点是红、黄两种颜色时, 在 $u, v$ 间连一条边.
一个具有颜色红、黄、蓝顶点的小三角形对应于图 $G$ 中度为 1 的顶点, 其余小三角形均对应 $G$ 中度为 0 或 2 的顶点, 而 $\triangle A B C$ 外部顶点 $u$ 的度是 1 . 由于图 $G$ 中奇顶点的个数必为偶数, 所以除了 $u$ 之外奇顶点的个数为奇数, 因此三个顶点颜色都不同的小三角形的个数是奇数.
%%<REMARK>%%
注:很多问题转化为图论问题后, 能以图论中比较完整的理论体系作依托来解题 (前面的例 3 就是这方面的典型). 本题从奇偶分析人手, 运用了有限图中 "奇顶点的个数为偶数" 这个熟知的结论, 简单明了地解决了问题.
%%PROBLEM_END%%



%%PROBLEM_BEGIN%%
%%<PROBLEM>%%
例5. 在某学校里有 $b$ 名教师和 $c$ 名学生.
假设满足下列条件:
(1) 每名教师恰好教 $k$ 名学生;
(2) 对任意两名学生,恰好有 $h$ 名教师教他们两人.
证明: $\frac{b}{h}=\frac{c(c-1)}{k(k-1)}$. 
%%<SOLUTION>%%
证明:构造一个图 $G$ : $G$ 含有两组顶点 $A, B$, 其中每个老师对应 $A$ 中的一个顶点, 每个 "学生对"对应 $B$ 中的一个顶点, 若某位老师正好教某一学生对, 则在 $A, B$ 相应顶点之间连一条边.
由 (1) 得, $A$ 中每个顶点与 $B$ 中 $\mathrm{C}_k^2$ 个顶点相邻, 故 $G$ 共有 $b \cdot \mathrm{C}_k^2$ 条边.
由 (2) 得, $B$ 中每个顶点与 $A$ 中 $h$ 个顶点相邻, 故 $G$ 共有 $\mathrm{C}_c^2 \cdot h$ 条边.
从而 $b \cdot \mathrm{C}_k^2=\mathrm{C}_c^2 \cdot h$, 即
$$
\frac{b}{h}=\frac{c}{k}(c-1)
$$
%%<REMARK>%%
注:本题运用模型化思想, 构造图的模型来阐述问题.
上述图 $G$ 的两组顶点中, 每组内不存在相邻顶点, 这样的图称为 2 部图或偶图.
从每组顶点出发, 对 2 部图的边数从两方面计数,即得到我们所需要的恒等式.
%%PROBLEM_END%%



%%PROBLEM_BEGIN%%
%%<PROBLEM>%%
例6. 34 对选手参加双人舞比赛, 赛前, 某些选手互相握手.
同一对的两人不握手.
后来, 某男选手问其他 67 名参赛选手他们与人握手的次数, 得到的答案都不相同.
问: 该男选手的搭档女选手和多少人握过手?
%%<SOLUTION>%%
解: 68 个顶点顶点表示 68 个参赛选手.
对于顶点 $u, v$, 当且仅当 $u, v$ 所表示的两名选手握过手时, 令它们相邻, 于是得到一个 68 个顶点的简单图 $G$.
由于同一对的两名选手不握手, 所以对任意顶点 $u, d(u) \leqslant 66$.
设某男选手为 $x$. 图 $G$ 中除顶点 $x$ 外尚有 67 个点, 由题意, 它们的度各不相同, 因此必有一个点 $v$ 满足 $d(v)=0$, 即 $G$ 中没有一个人和 $v$ 握过手.
所以 $d(w)=66$ 的那名选手 $w$ 只能与 $v$ 来自同一对.
从图 $G$ 中去掉 $v$ 和 $w$, 得到含 66 个顶点的图 $G_1$. 则 $x$ 是 $G_1$ 中的顶点, 并且除 $x$ 之外, 其他顶点的度也都不相同.
因此和前述证明相同, $G_1$ 含有度分别为 0 和 64 的顶点 $p$ 和 $q$, 它们在原来图 $G$ 中的度分别为 1 和 65. 如此继续,
可证得对 $0 \leqslant j \leqslant 33$, 图 $G$ 中含有顶点 $x_j, y_j$, 它们的度分别为 $j$ 和 $66-j$, 而且所代表的选手来自同一对, 特别地, 最后有 $x_{33}=x$, 所以 $d\left(x_{33}\right)=33$. 因此该男选手的搭档女选手握手次数为 33 .
%%<REMARK>%%
注:将有限图 $G$ 的顶点编号, 按度的非降次序 $\left(d_1 \leqslant d_2 \leqslant \cdots \leqslant d_n\right)$ 排列, 所得到的 $\left(d_1, d_2, \cdots, d_n\right)$ 称为 $G$ 的度序列.
度序列是图的基本特征之一.
本题中, 我们从度序列考虑问题, 这是解题的一种重要方法.
%%PROBLEM_END%%



%%PROBLEM_BEGIN%%
%%<PROBLEM>%%
例7. 求满足如下条件的最小正整数 $n$, 在圆 $O$ 的圆周上任取 $n$ 个点 $A_1$, $A_2, \cdots, A_n$, 则在 $\mathrm{C}_n^2$ 个角 $\angle A_i O A_j(1 \leqslant i<j \leqslant n)$ 中, 至少有 2010 个不超过 $120^{\circ}$.
%%<SOLUTION>%%
解:先, 当 $n=90$ 时, 如图(<FilePath:./figures/fig-c20i2.png>), 设 $A B$ 是圆 $O$ 的直径, 在点 $A$ 和 $B$ 的附近分别取 45 个点, 此时, 只有 $2 \mathrm{C}_{45}^2=45 \times 44=1980$ 个角不超过 $120^{\circ}$, 所以, $n=90$ 不满足题意.
当 $n=91$ 时,下面证明至少有 2010 个角不超过 $120^{\circ}$.
把圆周上的 91 个点 $A_1, A_2, \cdots, A_{91}$ 看作一个图的 91 个顶点 $v_1$, $v_2, \cdots, v_{91}$, 若 $\angle A_i O A_j>120^{\circ}$, 则在它们对应的顶点 $v_i, v_j$ 之间连一条边, 这样就得到一个图 $G$.
设图 $G$ 中有 $e$ 条边, 易知, 图中没有三角形.
若 $e=0$, 则有 $\mathrm{C}_{91}^2=4095>2010$. 个角不超过 $120^{\circ}$, 命题得证.
若 $e \geqslant 1$, 不妨设顶点 $v_1, v_2$ 之间有边相连, 因为图中没有三角形, 所以, 对于顶点 $v_i(i=3,4, \cdots, 91)$, 它至多与 $v_1, v_2$ 中的一个有边相连, 所以
$$
d\left(v_1\right)+d\left(v_2\right) \leqslant 89+2=91,
$$
其中 $d(v)$ 表示顶点 $v$ 的度, 即顶点 $v$ 处引出的边数.
因为 $d\left(v_1\right)+d\left(v_2\right)+\cdots+d\left(v_{91}\right)=2 e$, 而对于图 $G$ 中的每一条边的两个顶点 $v_i, v_j$, 都有
$$
d\left(v_i\right)+d\left(v_j\right) \leqslant 91,
$$
于是, 上式对每一条边求和可得
$$
\left(d\left(v_1\right)\right)^2+\left(d\left(v_2\right)\right)^2+\cdots+\left(d\left(v_{91}\right)\right)^2 \leqslant 91 e,
$$
由柯西不等式得
$$
\begin{aligned}
& 91\left[\left(d\left(v_1\right)\right)^2+\left(d\left(v_2\right)\right)^2+\cdots+\left(d\left(v_{91}\right)\right)^2\right] \\
\geqslant & {\left[d\left(v_1\right)+d\left(v_2\right)+\cdots+d\left(v_{91}\right)\right]^2=4 e^2, }
\end{aligned}
$$
所以 $\frac{4 e^2}{91} \leqslant\left(d\left(v_1\right)\right)^2+\left(d\left(v_2\right)\right)^2+\cdots+\left(d\left(v_{91}\right)\right)^2 \leqslant 91 e$, 故
$$
e \leqslant \frac{91^2}{4}<2071
$$
所以, 91 个顶点中, 至少有 $\mathrm{C}_{91}^2-2071=2024>2010$ 个点对, 它们之间没有边相连, 从而, 它们对应的顶点所对应的角不超过 $120^{\circ}$.
综上所述, $n$ 但最小值为 91 .
%%PROBLEM_END%%



%%PROBLEM_BEGIN%%
%%<PROBLEM>%%
例8. 设 $F$ 是一个由整数组成的有限集,满足:
(1) 对任意 $x \in F$, 存在 $y, z \in F$ (可以相同), 使得 $x=y+z$;
(2) 存在 $n \in \mathbf{N}^*$, 使得对任何正整数 $k(1 \leqslant k \leqslant n)$ 及任意 $x_1, x_2, \cdots$, $x_k \in F$ (可以相同), 都有 $\sum_{i=1}^k x_i \neq 0$.
求证: $F$ 至少含有 $2 n+2$ 个元素.
%%<SOLUTION>%%
证明:然 $0 \notin F$, 且 $F$ 中所有元素不全同号(否则,绝对值最小的完素 $x$ 无法表示成 $y+z(y, z \in F)$ 的形式).
设 $x_1, x_2, \cdots, x_m$ 是 $F$ 中所有正的元素.
我们证明 $m \geqslant n+1$.
作含 $m$ 个顶点 $v_1, v_2, \cdots, v_m$ 的有向图 $G$ 如下: 对每个 $i \in\{1,2, \cdots$, $m\}$, 由于存在 $y, z \in F$, 使 $x_i=y+z>0$, 不妨设 $y>0$, 则 $y$ 必等于某个 $x_j(j \neq i)$, 如此就在 $G$ 中作一条有向边 $v_i \rightarrow v_j$.
由 $G$ 的作法可知每个顶点 $v_i$ 的出度为 1 , 因此 $G$ 中必存在圈, 不妨设为
$$
v_{i_1} \rightarrow v_{i_2} \rightarrow \cdots \rightarrow v_{i_k} \rightarrow v_{i_1}(k \leqslant m) .
$$
这表明存在 $z_1, z_2, \cdots, z_k \in F$, 使得
$$
\left\{\begin{array}{l}
x_{i_1}=x_{i_2}+z_1, \\
x_{i_2}=x_{i_3}+z_2, \\
\cdots \\
x_{i_k}=x_{i_1}+z_k,
\end{array}\right.
$$
求和得
$$
z_1+z_2+\cdots+z_k=0 .
$$
为不与条件 (2) 矛盾, 必有 $k>n$, 从而 $m \geqslant k \geqslant n+1$.
同理知 $F$ 中负的元素也至少有 $n+1$ 个, 由此可知 $F$ 至少含有 $2 n+2$ 个元素.
%%<REMARK>%%
注:本题巧妙构造有向图来解题.
从本题的论证中可以发现, 即使对 $F$ 为有限实数集合的情况, 结论仍是成立的.
读者可进一步考虑 $F$ 能否恰含有 $2 n+2$ 个元素.
%%PROBLEM_END%%


