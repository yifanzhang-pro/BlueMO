
%%TEXT_BEGIN%%
赋值法, 是对本身与数量无关的问题巧妙地赋于某些适当的数值, 将其数学化, 然后利用整除性、奇偶性或正负号等的讨论, 使问题得以解决的方法.
许多组合问题和非传统的数论问题常用赋值法求解.
注意到染色法中是用颜色对事物分类, 赋值法则能以"数"替代"色", 因此一般而言, 能用染色法表述和解决的问题都可以用赋值法来处理(不过染色法在表述的形象性等方面有其优势). 赋值法使问题数值化, 进一步可使数值参与运算和推证, 因而又有其独到的施展空间.
常见的赋值方式有: 对点赋值、对线段赋值、对区域赋值以及对其他对象赋值等.
%%TEXT_END%%



%%PROBLEM_BEGIN%%
%%<PROBLEM>%%
例1. 已知 $n$ 个点 $A_1, A_2, \cdots, A_n$ 顺次排在一条直线上, 每个点染上红色或蓝色之一.
如果线段 $A_i A_{i+1}(1 \leqslant i \leqslant n-1)$ 的两端颜色不同, 就称它为标准线段.
已知 $A_1$ 与 $A_n$ 的颜色不同, 证明: 在 $A_i A_{i+1}(i=1,2, \cdots, n-1)$ 中,标准线段的条数为奇数.
%%<SOLUTION>%%
证明: $A_1, A_2, \cdots, A_n$ 中的每一个点 $A_i$ 赋值 $a_i$ : 若 $A_i$ 为红色, 则 $a_i=1$; 若 $A_i$ 为蓝色, 则 $a_i=-1$.
设这 $n-1$ 条线段中有 $m$ 条是标准线段,那么
$$
\left(a_1 a_2\right)\left(a_2 a_3\right) \cdots\left(a_{n-1} a_n\right)=(-1)^m .
$$
另一方面, 有
$$
\left(a_1 a_2\right)\left(a_2 a_3\right) \cdots\left(a_{n-1} a_n\right)=a_1 a_2^2 a_3^2 \cdots a_{n-1}^2 a_n=a_1 a_n=-1 .
$$
所以, $(-1)^m=-1$, 故 $m$ 是奇数, 即标准线段的条数为奇数.
%%<REMARK>%%
注:本题解法颇多.
上述解法中, 我们通过赋值使问题数值化, 并通过数的运算性质简洁而直接地解答了问题.
另一种典型的赋值方法可表述为: 对各点 $A_i$ 中的红点赋值 $a_i=0$, 蓝点赋值 $a_i=1$, 并考察所有线段 $A_i A_{i+1}$ 对应的数值 $a_i+a_{i+1}$ 之和的奇偶性.
这两种赋值方法在相差一个运算级别的意义下是一样的.
%%PROBLEM_END%%



%%PROBLEM_BEGIN%%
%%<PROBLEM>%%
例2. 男女生共 $n$ 人围坐一圆桌, 规定相邻座为同性时两人中间插一枝红花, 异性时两人中间插一枝蓝花.
结果发现所插红花与蓝花数目一样.
证明: $n$ 一定是 4 的倍数.
%%<SOLUTION>%%
证明:由于红花和蓝花数目一样, 故 $n$ 必为偶数, 设 $n=2 m$.
取定一人开始, 将 $n$ 个人依次记为 $1,2, \cdots, n$. 给第 $i(1 \leqslant i \leqslant n)$ 个人赋值 $x_i$, 其中对男生令 $x_i=1$, 对女生令 $x_i=-1$.
根据条件可知, 当 $x_i x_{i+1}=1$ 时,在 $i, i+1$ 两人之间插红花; 当 $x_i x_{i+1}=-1$ 时, 在 $i, i+1$ 两人之间插蓝花 (约定第 $n+1$ 个人就是第 1 个人, $x_{n+1}=x_1$ ).
转化为证明这样的结论: 若在 $x_i x_{i+1}(1 \leqslant i \leqslant n)$ 这 $n=2 m$ 个值中, 有 $m$ 个 $1, m$ 个 -1 , 则 $m$ 为偶数.
事实上, 此时有 $(-1)^m=\prod_{i=1}^n x_i x_{i+1}=\left(x_1 x_2 \cdots x_n\right)^2=1$, 故 $m$ 为偶数.
从而 $n=2 m$ 一定是 4 的倍数.
证毕.
%%<REMARK>%%
注:本题的做法与上一题有所相似.
虽说本题完全可以不用赋值法来分析, 但上述讨论实际上建立了一个组合命题与一个数论命题之间的等价性 (该数论命题为: 若 $x_i \in\{1,-1\}, 1 \leqslant i \leqslant n$, 且 $\sum_{i=1}^n x_i x_{i+1}=0$, 则 $4 \mid n$ ), 从中可以看出赋值法的转化功能.
%%PROBLEM_END%%



%%PROBLEM_BEGIN%%
%%<PROBLEM>%%
例3. 把 $1,2,3, \cdots, 2004$ 这 2004 个正整数随意放置在一个圆周上,统计所有相邻三个数的奇偶性得知: 三个数全是奇数的有 600 组, 恰好两个奇数的有 500 组, 问: 恰好一个奇数的有几组? 全部不是奇数的有几组?
%%<SOLUTION>%%
解:设恰好 1 个奇数的有 $x$ 组, 则全部不是奇数的有
$$
2004-600-500-x=904-x .
$$
将圆周上的数从某个数开始, 依次计为 $x_1, x_2, \cdots, x_{2004}$, 令
$$
y_i= \begin{cases}-1, & \text { 当 } x_i \text { 为奇数时, } \\ 1, & \text { 当 } x_i \text { 为偶数, }\end{cases}
$$
则 $y_1+y_2+\cdots+y_{2004}=0$, 再令
$$
A_i=y_i+y_{i+1}+y_{i+2}= \begin{cases}-3, & \text { 当 } x_i, x_{i+1}, x_{i+2} \text { 全为奇数时, } \\ -1, & \text { 当 } x_i, x_{i+1}, x_{i+2} \text { 恰好 } 2 \text { 个奇数时, } \\ 1, & \text { 当 } x_i, x_{i+1}, x_{i+2} \text { 恰好一个奇数时, } \\ 3, & \text { 当 } x_i, x_{i+1}, x_{i+2} \text { 全为偶数时, }\end{cases}
$$
其中约定 $x_{2004+i}=x_i(i=1,2)$. 于是
$$
\begin{aligned}
0 & =3\left(y_1+y_2+\cdots+y_{2004}\right)=A_1+A_2+\cdots+A_{2004} \\
& =-3 \times 600-500+x+3(904-x),
\end{aligned}
$$
解得 $x=206$.
恰好一个奇数的有 206 组, 全部不是奇数的有 $904-206=698$ 组.
%%PROBLEM_END%%



%%PROBLEM_BEGIN%%
%%<PROBLEM>%%
例4. 将 $5 \times 7$ 棋盘用某种规格的若干张纸片覆盖, 纸片不许超出棋盘, 但可彼此交叠.
考虑下述各种情形, 能否找到一种覆盖方式, 使得棋盘的每个小方格被覆盖的层数相同?(注 : 纸片可翻折使用,但纸片的边必须平行于棋盘的边.)
(1) 纸片为 $1 \times 3$ 的规格;
(2)纸片为 "特里米诺", 即 $2 \times 2$ 方格纸去掉一个方格后所余下的图形;
(3)纸片为 "刀把五", 即 $2 \times 3$ 方格纸去掉一个角上方格后所余下的图形.
%%<SOLUTION>%%
解:1) 按下表,即如图(<FilePath:./figures/fig-c15i2.png>) 对棋盘的每个方格赋值, 易见每张 $1 \times 3$ 纸片所覆盖的三数之和为 0 ,因而无论用多少张纸片,盖住的所有数之和恒等于 0 (一个数被盖了几层就计算几次). 但棋盘上的数字总和为 1 , 假如被恰好覆盖 $k$ 层, 则盖住的数字之和应为 $k$, 因此不存在满足条件的覆盖方式.
\begin{tabular}{|c|c|c|c|c|c|c|}
\hline 2 & -1 & -1 & 2 & -1 & -1 & 2 \\
\hline-1 & -1 & 2 & -1 & -1 & 2 & -1 \\
\hline-1 & 2 & -1 & -1 & 2 & -1 & -1 \\
\hline 2 & -1 & -1 & 2 & -1 & -1 & 2 \\
\hline-1 & -1 & 2 & -1 & -1 & 2 & -1 \\
\hline
\end{tabular}
(2)按下表,即如图(<FilePath:./figures/fig-c15i3.png>) 对棋盘的每个方格赋值,易见每张"特里米诺"纸片所覆盖的三数之和不大于 0 , 因而无论用多少张纸片, 盖住的所有数之和是个非正数.
但棋盘上的数字总和为 1 , 覆盖 $k$ 层时盖住的数字之和等于正数 $k$,因此不存在满足条件的覆盖方式.
\begin{tabular}{|c|c|c|c|c|c|c|}
\hline 2 & -1 & 2 & -1 & 2 & -1 & 2 \\
\hline-1 & -1 & -1 & -1 & -1 & -1 & -1 \\
\hline 2 & -1 & 2 & -1 & 2 & -1 & 2 \\
\hline-1 & -1 & -1 & -1 & -1 & -1 & -1 \\
\hline 2 & -1 & 2 & -1 & 2 & -1 & 2 \\
\hline
\end{tabular}
(3)按下表,即如图(<FilePath:./figures/fig-c15i4.png>) 对棋盘的每个方格赋值,易见每张"刀把五"纸片所覆盖的五数之和不小于 0 , 因而无论用多少张纸片, 盖住的所有数之和是个非负数.
但棋盘上的数字总和为 -5 , 覆盖 $k$ 层时盖住的数字之和小于 0 , 因此不存在满足条件的覆盖方式.
\begin{tabular}{|c|c|c|c|c|c|c|}
\hline-1 & -1 & -1 & -1 & -1 & -1 & -1 \\
\hline-1 & 4 & -1 & 4 & -1 & 4 & -1 \\
\hline-1 & -1 & -1 & -1 & -1 & -1 & -1 \\
\hline-1 & 4 & -1 & 4 & -1 & 4 & -1 \\
\hline-1 & -1 & -1 & -1 & -1 & -1 & -1 \\
\hline
\end{tabular}
%%<REMARK>%%
注:本题的 3 个小题都运用赋值法来求解,但由于每种规格的纸片特征不同, 所以具体做法也不同.
本题中, 棋盘的特性也是重要的.
特别地, 在解 (2)、(3)两小题时, 都是赋以两种不同的值对方格分类,从局部看,这两种分类方式一样 (其中的 " 2 " 和 " 4 " 都是隔行、隔列出现),但整体上来看, $5 \times 7$ 棋盘边界的特性就影响到赋值后解决问题的可行性.
读者不妨自行试探, 仔细体会上述每种赋值方式对具体问题的"针对性" 效果.
另外, 如果在纸片规格和棋盘规格等方面进行推广, 仍有大量问题可以研究.
%%PROBLEM_END%%



%%PROBLEM_BEGIN%%
%%<PROBLEM>%%
例5. 如图(<FilePath:./figures/fig-c15i5.png>) 是一个向右和向下无限的表格.
一开始在左上角 $A$ 格内放一枚棋子, 此后每一步下棋规则如下: 若某格 $P$ 放有棋子, 且它的右边相邻格 $Q$ 和下边相邻格 $R$ 都没有棋子, 则可将 $P$ 中的棋子去掉, 在 $Q 、 R$ 两格中各放一枚棋子.
证明: 无论经过多少步,左上角 $3 \times 3$ 表格中总存在棋子.
%%<SOLUTION>%%
证明:们对第 $i$ 行第 $j$ 列的格子赋值 $\left(\frac{1}{2}\right)^{i+j}, i, j \in \mathbf{N}^*$. 由于
$$
\left(\frac{1}{2}\right)^{i+(j+1)}+\left(\frac{1}{2}\right)^{(i+1)+j}=\left(\frac{1}{2}\right)^{i+j},
$$
故每步下棋不改变所有棋子所在格的赋值之和, 记这个和为 $S$, 其中初始情况下的 $S=\frac{1}{4}$.
假设若干步后,左上角 $3 \times 3$ 表格中不存在棋子, 那么此时
$$
\begin{aligned}
S & \leqslant \sum_{i=1}^{\infty} \sum_{j=1}^{\infty}\left(\frac{1}{2}\right)^{i+j}-\sum_{i=1}^3 \sum_{j=1}^3\left(\frac{1}{2}\right)^{i+j} \\
& =\sum_{i=1}^{\infty}\left(\frac{1}{2}\right)^i \cdot \sum_{j=1}^{\infty}\left(\frac{1}{2}\right)^j-\sum_{i=1}^3\left(\frac{1}{2}\right)^i \cdot \sum_{j=1}^3\left(\frac{1}{2}\right)^j \\
& =1-\left(\frac{7}{8}\right)^2=\frac{15}{64}<\frac{1}{4},
\end{aligned}
$$
矛盾.
故无论经过多少步,左上角 $3 \times 3$ 表格中总存在棋子.
%%<REMARK>%%
注:本题中通过等比赋值的方式给每个格子一个具体的数值, 这种赋值方法以及 " $\frac{1}{2}$ " 这个比值的选用是根据下棋规则而度身定制的, 从而在下棋过程中, 所有棋子所在格的数值之和是一个不变量.
在赋值法解题时常会遇到这样的情况: 需要先明确赋值的目标, 再具体问题具体分析 (试比较本节习题 4).
%%PROBLEM_END%%



%%PROBLEM_BEGIN%%
%%<PROBLEM>%%
例6. $k$ 个开关顺次排成一行, 分别指向上、下、左、右四个方向.
若其中出现三个连续的开关,它们的方向各不相同,则将它们同时调整为第四个方向.
证明: 这个操作不能无限次进行下去.
%%<SOLUTION>%%
证明:将这 $k$ 个开关依次按 $1,2, \cdots, k$ 进行编号.
我们按下述方法对第 $n(1 \leqslant n \leqslant k)$ 个开关赋值 $f(n)$ :
$n=1$ 时, $f(n)=1$;
$n \geqslant 2$ 时, 若第 $n-1$ 个开关和第 $n$ 个开关方向相同, 则 $f(n)=n$; 若不然, 则 $f(n)=1$.
令 $h=f(1) f(2) \cdots f(k)$, 显然 $h>0$. 可以证明, 随着操作次数的增加, $h$ 是递增的.
事实上, 对任意的 $n(1 \leqslant n \leqslant k-2)$, 假定本次操作只对第 $n, n+1, n+2$ 个开关进行调整, 那么在这次操作之前, $f(n) \leqslant n, f(n+1)=f(n+2)=1$, $f(n+3) \leqslant n+3$, 所以
$$
f(n) f(n+1) f(n+2) f(n+3) \leqslant n(n+3) .
$$
进行操作后, 得到 $f^{\prime}(n) \geqslant 1, f^{\prime}(n+1)=n+1, f^{\prime}(n+2)=n+2$, $f^{\prime}(n+3) \geqslant 1$.
所以 $f^{\prime}(n) f^{\prime}(n+1) f^{\prime}(n+2) f^{\prime}(n+3) \geqslant(n+1)(n+2)>f(n) f(n+$ 1) $f(n+2) f(n+3)$.
而操作前后第 $i(1 \leqslant i \leqslant n-1$ 或 $n+4 \leqslant i \leqslant k)$ 个开关的值不变, 即 $f^{\prime}(i)=f(i)$.
于是 $h^{\prime}=f^{\prime}(1) f^{\prime}(2) \cdots f^{\prime}(k)>f(1) f(2) \cdots f(k)=h$.
注意到 $h$ 有界 $(h \leqslant 1 \times 2 \times \cdots \times k)$, 所以 $h$ 只能增长有限次.
所以这个操作只能进行有限次.
%%<REMARK>%%
注:本题中,如果仅简单地对 "方向相同的相邻开关对"进行计数, 则无法说明其恒增性.
我们进行适当赋值, 构造出 "高度函数" $h$, 它是一个恒增量, 这就解决了困难.
我们还可以定义其他的高度函数 $h$. 例如: 对第 $n$ 个开关赋值 $f(n)$ : 若第 $n$ 个开关和第 $n+1$ 个开关方向相反, 则 $f(n)=\sqrt{n}$; 若不然, 则 $f(n)=0$. 定义高度函数 $h=\sum_{n=1}^k f(n)$. 可以证明随着操作次数的增加, $h$ 是递减的.
而 $h>0$, 所以 $h$ 只能减少有限次, 即得证.
读者可以尝试用"凹函数" 来构造更多的高度函数 $h$. 关键是要定义一个正值函数 $f(n)$, 且对于所有的 $1 \leqslant n \leqslant k$, 满足 $f(n) \cdot f(n+3)<f(n+1) \cdot f(n+2)$, 或者 $f(n)+f(n+3)<f(n+1)+f(n+2)$.
%%PROBLEM_END%%



%%PROBLEM_BEGIN%%
%%<PROBLEM>%%
例7. $\mathrm{MO}$ 牌足球由若干多边形皮块用三种不同颜色的丝线缝制而成,
有以下特点:
(1)任一多边形皮块的一条边恰与另一多边形皮块同样长的一条边用一种颜色的丝线缝合;
(2) 足球上每一结点恰好是三个多边形的顶点, 每一结点的三条缝线颜色互不相同.
求证: 可以在 $\mathrm{MO}$ 牌足球的每一结点上放置一个不等于 1 的复数,使得每一多边形皮块的所有顶点上放置的复数的乘积都等于 1. 
%%<SOLUTION>%%
证明:设这三种颜色为红、黄、蓝.
对每条边进行赋值: 红色为 1 , 黄色为 $\omega$, 蓝色为 $\omega^2$, 其中 $\omega=-\frac{1}{2}+\frac{\sqrt{3}}{2} \mathrm{i}$.
对每个结点, 若站在结点上往结点处看, 三条边的值按逆时针方向依次为 $1, \omega, \omega^2$, 则在该结点放 $\omega$, 否则放 $\omega^2$, 其中 $\omega=-\frac{1}{2}+\frac{\sqrt{3}}{2} \mathrm{i}$. 我们来验证这样的放法满足题目条件.
对每个多边形皮块, 设其顶点依逆时针方向依次为结点 $A_1, A_2, \cdots, A_k$, 根据边与结点的赋值规则, 并注意到
$$
\omega=\frac{\omega}{1}=\frac{\omega^2}{\omega}=\frac{1}{\omega^2}, \omega^2=\frac{\omega^2}{1}=\frac{1}{\omega}=\frac{\omega}{\omega^2}
$$
可知, $A_i$ 上的数 $\omega_i$ 总等于 $A_{i-1} A_i$ 的值 $z_i$ 除以 $A_i A_{i+1}$ 的值 $z_{i+1}$ (其中 $i=1$, $2, \cdots, k$, 约定 $\left.A_0=A_k, A_{k+1}=A_1, z_{k+1}=z_1\right)$. 从而
$$
\omega_1 \omega_2 \cdots \omega_k=\frac{z_1}{z_2} \cdot \frac{z_2}{z_3} \cdots \cdot \frac{z_k}{z_1}=1,
$$
即这样的放法满足题意.
证毕.
%%<REMARK>%%
注:本题中的提问是针对结点的, 具体处理时则采取了先对边赋值的做法 (对三种颜色的边分别赋值 $1, \omega, \omega^2$ ), 之后再借助边的值来决定如何对结点赋值.
当然, 本题一个很大的难点在于想到可以只用 $\omega$ 和 $\omega^2$ 完成赋值.
如果是这样, 本题就相应变成一个较简单的问题,并且也可以等价地表述为: 可在 $\mathrm{MO}$ 牌足球的每一结点上放置数字 1 或 2 , 使得每一多边形皮块的所有顶点上放置的数之和是 3 的倍数.
%%PROBLEM_END%%



%%PROBLEM_BEGIN%%
%%<PROBLEM>%%
例8. 设 $n, k$ 为给定正整数,一开始若干个球被分为 $n$ 堆,后来它们又被重新分为 $n+k$ 堆 (每堆球个数至少为 1 ). 证明: 存在至少 $k+1$ 个球, 它们原来所在堆中的球数大于后来所在堆中的球数.
%%<SOLUTION>%%
解:设所有球构成有限集合 $M$, 并对每个球赋值: 若某个球在球数为 $a$
的球堆中, 则对它赋值 $\frac{1}{a}$. 在这种赋值方式下, 每堆球的值之和为 1 , 因此所有球的值之和等于此时球堆的堆数.
对每个球 $A \in M$, 设它先后所在的球堆中球的个数分别是 $x_A$ 和 $y_A$, 记 $d(A)=\frac{1}{y_A}-\frac{1}{x_A}$ 为对球 $A$ 的后一次赋值与前一次赋值之差.
由于
$$
\sum_{A \in M} d(A)=\sum_{A \in M} \frac{1}{y_A}-\sum_{A \in M} \frac{1}{x_A}=(n+k)-n=k,
$$
且
$$
d(A)=\frac{1}{y_A}-\frac{1}{x_A}<\frac{1}{y_A} \leqslant 1,
$$
故至少有 $k+1$ 个 $d(A)=\frac{1}{y_A}-\frac{1}{x_A}>0$, 即 $x_A>y_A$, 从而命题成立.
%%<REMARK>%%
注:注意到每个球都是球堆中的"一份子",本题中的赋值具有某种权重的意义: 若球堆中的球数为 $a$, 则其所占份额为 $\frac{1}{a}$. 于是, 先后两次分堆时球堆堆数之差 $k$, 表现为所有球的份额值总和的先后两次之差, 而 "原来所在球堆中的球数大于后来所在球堆中的球数" 可理解为 "后来球堆中所占的份额大于在原来球堆中所占的份额". 在这样的理解下, 就化为一个代数问题了.
%%PROBLEM_END%%


