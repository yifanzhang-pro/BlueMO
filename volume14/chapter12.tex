
%%TEXT_BEGIN%%
对应与配对.
"对应"是一个基本而又重要的数学概念.
我们也常利用对应来解题.
这种方法的大致想法是这样的:对某个系统中的一个问题, 找到一种对应法则, 通过该法则的作用把这个问题转化成另一个系统中的相应问题,而该问题在新的系统中是可以解决的, 并存在逆对应, 再把新系统的解答逆反回去, 从而求得原来那个问题的解答.
徐利治教授称这种方法为 "关系映射反演原则", 简称 RMI 原则.
"配对" 是指将一些对象 (例如数、元素、子集) 按照某种适当的对应关系两两相配考虑问题,从而简化计算或使命题得以证明.
下面通过一些例题来介绍对应与配对方法的应用.
%%TEXT_END%%



%%PROBLEM_BEGIN%%
%%<PROBLEM>%%
例1. 游戏" 24 点"的规则是: 随机抽出 4 个不大于 10 的正整数(数字可以有重复), 要用它们组成一个四则运算式, 使运算结果等于 24 (数字必须都用,有些情况下是无解的). 某同学为了备战学校的 "24 点"比赛, 提前列出所有可能出现的 4 个数的情况进行研究.
如果不计 4 个数的顺序, 那么他共需列出多少种情况?
%%<SOLUTION>%%
解:一分以下几类讨论:
(1)如果 4 个数全相同,则有 10 种情况.
(2)如果 4 个数中出现两种数字,那么当有 3 个相同时,有 $\mathrm{P}_{10}^2=90$ 种情况; 当有两个相同, 另两个也相同时,有 $\mathrm{C}_{10}^2=45$ 种情况.
共 135 种情况.
(3) 如果 4 个数中出现三种数字, 则必有一种出现两次, 另两种各出现一次.
出现两次的数有 10 种选法, 另两个数字有 $\mathrm{C}_9^2=36$ 种选法, 根据乘法原理可得共 360 种情况.
(4)如果 4 个数两两不同, 则有 $\mathrm{C}_{10}^4=210$ 种情况.
综上,该同学共需列出 $10+135+360+210=715$ 种情况.
%%PROBLEM_END%%



%%PROBLEM_BEGIN%%
%%<PROBLEM>%%
例1. 游戏" 24 点"的规则是: 随机抽出 4 个不大于 10 的正整数(数字可以有重复), 要用它们组成一个四则运算式, 使运算结果等于 24 (数字必须都用,有些情况下是无解的). 某同学为了备战学校的 "24 点"比赛, 提前列出所有可能出现的 4 个数的情况进行研究.
如果不计 4 个数的顺序, 那么他共需列出多少种情况?
%%<SOLUTION>%%
解:二转化为求 4 元有序整数组 $(a, b, c, d)(1 \leqslant a \leqslant b \leqslant c \leqslant d \leqslant$ 10) 的数目.
作映射 $f:(a, b, c, d) \mapsto(a, b+1, c+2, d+3)$, 那么这个映射 $f$ 是从集合
$$
X=\{(a, b, c, d) \mid 1 \leqslant a \leqslant b \leqslant c \leqslant d \leqslant 10\}
$$
到集合
$$
Y=\left\{\left(a^{\prime}, b^{\prime}, c^{\prime}, d^{\prime}\right) \mid 1 \leqslant a^{\prime}<b^{\prime}<c^{\prime}<d^{\prime} \leqslant 13\right\}
$$
的一一对应 (双射), 所以 $X$ 与 $Y$ 元嗉个数相等.
显然 $Y$ 的元素个数就是 $\{1,2, \cdots, 13\}$ 的四元子集个数 $\mathrm{C}_{13}^4=715$, 所以欲求的有序数组 $(a, b, c, d)$ 的数目, 即所需列出的情况数为 715 .
%%<REMARK>%%
注:解法一充分运用分类讨论与分步讨论的思想, 其思路是很清楚的, 但所需考虑的情况较多.
解法二则是利用一一对应关系, 将问题转化为另一种较便于计算的模式.
在计数问题中常常要用到 "对应"的方法.
一般地, 若要计算有限集合 $X$ 的元素个数 $|X|$, 可以考虑与 $X$ 有关系的某个有限集合 $Y$ (其中 $|Y|$ 较易求得) :
如果可以在 $X$ 与 $Y$ 的元素间找到某种一一对应, 那么 $|X|=|Y|$;
如果可以证明 $X$ 中的每个元素与 $Y$ 中的 $k$ 个元素构成双向对应关系, 那么有 $|X|=k|Y|$;
如果可以证明 $Y$ 中的每个元素与 $X$ 中的 $k$ 个元素构成双向对应关系, 那么有 $|X|=\frac{1}{k}|Y|$.
此外也可以通过考察单向的对应关系, 对 $|X|$ 的上下界进行估计.
%%PROBLEM_END%%



%%PROBLEM_BEGIN%%
%%<PROBLEM>%%
例2. 甲乙两队各出 7 名队员按事先排好的顺序出场参加围棋擂台赛, 双方先由 1 号队员比赛, 负者被淘汰, 胜者再与负方的 2 号队员比赛, $\cdots$, 直到有一方队员被淘汰为止, 另一方获得胜利, 形成一种比赛过程.
求所有可能的比赛过程种数.
%%<SOLUTION>%%
解:一在每种比赛过程中, 不妨让负者按照告负的先后顺序走人一个通道, 再让最后一局的胜者连同尚未出战的选手按事先排好的顺序跟进通道.
易见, 每种比赛过程一一对应一种通道里的排队方式, 使得每队 7 名队员的相对位置与事先排好的出场顺序一致.
这样的排队方式由甲队队员所占的 7 个位置唯一决定, 故有 $\mathrm{C}_{14}^7=3432$ 种排队方式, 因此比赛过程种数为 3432 .
%%PROBLEM_END%%



%%PROBLEM_BEGIN%%
%%<PROBLEM>%%
例2. 甲乙两队各出 7 名队员按事先排好的顺序出场参加围棋擂台赛, 双方先由 1 号队员比赛, 负者被淘汰, 胜者再与负方的 2 号队员比赛, $\cdots$, 直到有一方队员被淘汰为止, 另一方获得胜利, 形成一种比赛过程.
求所有可能的比赛过程种数.
%%<SOLUTION>%%
解:二设甲队 $i$ 号队员获胜的场数为 $x_i(1 \leqslant i \leqslant 7)$, 易见, 每种使甲队获胜的比赛过程与不定方程 $x_1+x_2+\cdots+x_7=7$ 的非负整数解 $\left(x_1, x_2, \cdots\right.$, $x_7$ ) 构成一一对应.
因为方程 $x_1+x_2+\cdots+x_7=7$ 的非负整数解的组数为 $\mathrm{C}_{7+6}^6=\mathrm{C}_{13}^6$, 故甲队获胜的比赛过程数为 $\mathrm{C}_{13}^6=1716$. 同理可得, 乙队获胜的比赛过程数为 1716 . 因此所有可能的比赛过程共 3432 种.
%%<REMARK>%%
注:上述两种解法均为对应方法, 其中解法一将问题最为直接地对应到一个组合选取问题.
解法二则是通过先建立半数情形与不定方程解的对应关系, 最终求得结果.
对不定方程 $x_1+x_2+\cdots+x_n=m$ 而言, 其非负整数解组数为 $\mathrm{C}_{m+n-1}^{n-1}$, 正整数解组数为 $\mathrm{C}_{m-1}^{n-1}(m \geqslant n)$. 这是计数中的两个极为常用的结论 (它们本身都可以用对应原理来证明).
%%PROBLEM_END%%



%%PROBLEM_BEGIN%%
%%<PROBLEM>%%
例3. 对于数集 $M$, 称 $M$ 中最大数与最小数之和为 $M$ 的 "特征", 记作 $m(M)$. 求集合 $X=\{1,2, \cdots, n\}$ 的所有非空子集的特征的平均数.
%%<SOLUTION>%%
解: $Y$ 是 $X$ 的子集全体组成的集.
对于 $A \in Y$, 令
$$
A^{\prime}=\{n+1-a \mid a \in A\},
$$
那么 $A^{\prime} \in Y$. 所以
$$
f: A \mapsto A^{\prime}
$$
是集 $Y$ 到 $Y$ 自身的一一对应(双射).
特征的平均数
$$
g=\frac{1}{|Y|} \sum_{A \in Y} m(A)=\frac{1}{2|Y|} \sum_{A \in Y}\left(m(A)+m\left(A^{\prime}\right)\right) .
$$
由于 $A$ 中最大数是 $A^{\prime}$ 中的最小数, 且它们的和为 $n+1, A$ 中最小数是 $A^{\prime}$ 中的最大数, 它们的和也为 $n+1$. 于是
$$
g=\frac{1}{2|Y|} \sum_{A \in Y} 2(n+1)=(n+1) \cdot \frac{1}{|Y|} \sum_{A \in Y} 1=n+1 .
$$
%%<REMARK>%%
注:善于使用配对技巧,常常能使一些表面看来很复杂甚至棘手的问题迎刃而解.
本题中须求 "平均数", 我们将 $A$ 与 $A^{\prime}=\{n+1-a \mid a \in A\}$ 对应 (也可以看成一种配对, 只不过有时会与自身配对), 从而化简了计算.
这与等差数列求和的精神实质是一样的.
又如在考虑某些集合或排列问题时, 可根据问题的特征采取"互补集合对"、"倒序排列"等配对方式,同样十分有效.
%%PROBLEM_END%%



%%PROBLEM_BEGIN%%
%%<PROBLEM>%%
例4. 给定绝对值都不大于 10 的整数 $a, b, c$, 三次多项式 $x^3+a x^2+ b x+c$ 满足条件 $|f(2+\sqrt{3})|<0.0001$, 问: $2+\sqrt{3}$ 是否一定是这个多项式的根?
%%<SOLUTION>%%
解: $2+\sqrt{3}$ 代入得
$$
\begin{aligned}
f(2+\sqrt{3}) & =(2+\sqrt{3})^3+a(2+\sqrt{3})^2+b(2+\sqrt{3})+c \\
& =(26+7 a+2 b+c)+(15+4 a+b) \sqrt{3} .
\end{aligned}
$$
记 $26+7 a+2 b+c=m, 15+4 a+b=n$, 则
$$
|m+n \sqrt{3}|=|f(2+\sqrt{3})|<0.0001,
$$
又 $m, n \in \mathbf{Z}$, 且 $|m| \leqslant 126,|n| \leqslant 65$, 所以
$$
|m-n \sqrt{3}| \leqslant|m|+|n| \sqrt{3}<|m|+2|n| \leqslant 256 .
$$
故
$$
\left|m^2-3 n^2\right|=|m+n \sqrt{3}| \cdot|m-n \sqrt{3}|<0.0001 \times 256<1,
$$
又 $m^2-3 n^2$ 为整数, 所以 $m^2-3 n^2=0$, 这样只能 $m=n=0$, 从而 $f(2+ \sqrt{3})=0$, 即 $2+\sqrt{3}$ 是这个多项式的根.
%%<REMARK>%%
注:一般而言, 对于数的配对, 可以采取相反数、倒数、有理化根式、共轭复数等配对方法.
在本题中, 将 $m+n \sqrt{3}$ 与其对偶式 $m-n \sqrt{3}$ 配对, 利用整数的离散性进行估计, 注意 $\left|m^2-3 n^2\right|$ 若小于 1 , 则必等于 0 .
%%PROBLEM_END%%



%%PROBLEM_BEGIN%%
%%<PROBLEM>%%
例5. 对 $n \in \mathbf{N}^*$, 记不大于 $n$ 且与 $n$ 互素的所有正整数乘积为 $\pi(n)$, 求证:
$$
\pi(n) \equiv \pm 1(\bmod n) .
$$
%%<SOLUTION>%%
证明:当 $n=1,2$ 时,命题显然成立.
以下设 $n \geqslant 3$.
设不大于 $n$ 且与 $n$ 互素的正整数构成集合 $X_n$.
引理: 对任意 $t \in X_n$, 必有唯一的 $t^{\prime} \in X_n$, 使 $t^{\prime} \cdot t \equiv 1(\bmod n)$.
证明: 由于 $t, n$ 互素, 则存在整数 $p, q$ 使 $p t+q n=1$, 且此时 $p, n$ 互素.
取 $t^{\prime} \in X_n$ 使 $t^{\prime} \equiv p(\bmod n)$, 可见 $t^{\prime} \cdot t \equiv 1(\bmod n)$. 又假如 $t_1^{\prime} \cdot t \equiv t_2^{\prime} \cdot t \equiv 1(\bmod n)$, 则 $t_1^{\prime} \equiv t_1^{\prime} \cdot t \cdot t_2^{\prime} \equiv t_2^{\prime}(\bmod n)$. 因此这样的 $t^{\prime} \in X_n$ 是唯一的.
(1) 当 $t \neq t^{\prime}$ 时, 将 $t$ 与 $t^{\prime}$ 归为一组.
设这样的组共有 $i$ 组, 其中每组两数之积 $t^{\prime} \cdot t \equiv 1(\bmod n)$.
(2) 当 $t=t^{\prime}$ 时, 有 $t^2 \equiv 1(\bmod n)$, 则 $(n--t)^2 \equiv n^2-2 t n+t^2 \equiv 1(\bmod n$ ), 故有 $n-t=(n-t)^{\prime}$. 将 $t$ 与 $n-t$ 归为一组 (其中 $t \neq n-t$, 这是因为当 $n$ 为奇数时显然 $t \neq n-t$, 而当 $n$ 为偶数时 $t$ 不能取 $\frac{n}{2}$ ). 设这样的组共有 $j$ 组, 其中每组两数之积 $t(n-t) \equiv-t^2 \equiv-1(\bmod n)$.
由 (1)、(2)可知, $X_n$ 中所有元素经过配对后求积得
$$
\pi(n) \equiv 1^i \cdot(-1)^j \equiv \pm 1(\bmod n) .
$$
综上, 对一切正整数 $n(n=1,2$ 或 $n \geqslant 3), \pi(n) \equiv \pm 1(\bmod n)$.
%%<REMARK>%%
注:本例采用的实质上是 "数论倒数" 及 "模 $n$ 的相反数" 的配对方法.
下面再对本题结果作深人一些的讨论.
对 $t \in X_n$, 定义 $t$ 关于 $n$ 的阶是 $r_n(t)=r$, 其中 $r$ 是使 $t^r \equiv 1(\bmod n)$ 成立的最小正整数.
注意到 $r_n(t)=1$ 当且仅当 $t=1, r_n(t)=2$ 当且仅当 $t=t^{\prime} \neq$ 1 , 本题的证明揭示了这样的事实: 对 $n \geqslant 3$, 当 $X_n$ 中 2 阶元个数模 4 余 1 时, $\pi(n) \equiv-1(\bmod n)$; 当 $X_n$ 中 2 阶元个数模 4 余 3 时, $\pi(n) \equiv 1(\bmod n)$. 下面是两个推论:
推论 1 (Wilson 定理): 若 $p$ 为素数,则 $(p-1) ! \equiv-1(\bmod p)$.
事实上, 对 $p=2$ 可直接验证; 对任何奇素数 $p, p-1$ 是 $X_p$ 中唯一的 2 阶元,因此 $\pi(p)=(p-1) ! \equiv-1(\bmod p)$.
推论 2 若 $n$ 为偶数,则 $\pi(2 n) \equiv 1(\bmod n)$.
事实上, 对 $i \in \mathbf{Z}, i$ 与 $2 n$ 互素当且仅当 $i$ 与 $n$ 互素, 且 $n+i \equiv i(\bmod n)$, 故有
$$
\pi(2 n) \equiv \pi^2(n) \equiv 1(\bmod n) .
$$
一个比推论 2 稍强些的结果是: 若 $4 \mid n, n \geqslant 8$, 则 $\pi(n) \equiv 1(\bmod n)$. 这是由于对每个不超过 2 阶的元 $t \in X_n, 1 \leqslant t<\frac{n}{4}$, 四个数 $t, \frac{n}{2}-t, \frac{n}{2}+t$, $n-t$ 两两不同, 可归为一大组, 且它们的乘积在模 $n$ 的意义下等于 1 . 这也是配对思想的应用.
下面两个例题进一步体现了解题中建立对应的技巧.
%%PROBLEM_END%%



%%PROBLEM_BEGIN%%
%%<PROBLEM>%%
例6. 一条直路上依次有 $2 n+1$ 棵树 $T_1, T_2, \cdots, T_{2 n+1}(n$ 为给定正整数). 一个醉汉从中间位置的树 $T_{n+1}$ 出发, 并按以下规律在这些树之间随机游走 $n$ 分钟: 当他某一分钟末在树 $T_i(2 \leqslant i \leqslant 2 n)$ 位置时, 下一分钟末他分别有 $\frac{1}{4}, \frac{1}{2}, \frac{1}{4}$ 的概率到达 $T_{i-1}, T_i, T_{i+1}$ 位置.
求醉汉 $n$ 分钟末处在每棵树 $T_i(1 \leqslant i \leqslant 2 n+1)$ 位置的概率 $p_i$.
%%<SOLUTION>%%
解:不妨认为 $2 n+1$ 棵树 $T_1, T_2, \cdots, T_{2 n+1}$ 从左到右排列, 每两棵树间距为 1 单位.
将"以 $\frac{1}{2}$ 的概率向左或向右走 0.5 个单位"定义为一次"随机游走". 根据题目所述的概率分布特征, 醉汉每分钟的运动状况恰可分解为两次 "随机游走". 故原问题等价于求醉汉从 $T_{n+1}$ 出发, 经 $2 n$ 步"随机游走"后处在 $T_i$ 位置的概率 $p_i$.
对某个 $i(1 \leqslant i \leqslant 2 n+1)$, 设从 $T_{n+1}$ 出发经 $2 n$ 步"随机游走"到达 $T_i$ 的全过程中, "向右走 0.5 个单位"和"向左走 0.5 个单位"分别有 $k$ 次和 $2 n-k$ 次, 则
$$
n+1+\frac{k-(2 n-k)}{2}=i,
$$
解得 $k=i-1$, 即在 $2 n$ 步中 $i-1$ 次向右"游走", $2 n-(i-1)$ 次向左"游走", 而这样的情形共 $\mathrm{C}_{2 n}^{i-1}$ 种,故所求概率 $p_i=\frac{\mathrm{C}_{2 n}^{i-1}}{2^{2 n}}(1 \leqslant i \leqslant 2 n+1)$.
%%<REMARK>%%
注:本例通过 "运动分解" 的观点, 把一个涉及概率分布的较复杂问题对应到一个较简单的组合计数问题.
若先研究 $n$ 较小的情况, 猜想出结论, 并用递推方法证明,则不如上述方法简洁.
%%PROBLEM_END%%



%%PROBLEM_BEGIN%%
%%<PROBLEM>%%
例7. 若数列 $\left\{a_n\right\}$ 满足: 对任意 $n \in \mathbf{N}^*$, 有 $\sum_{d \mid n} a_d=2^n$, 证明: $n \mid a_n$.
%%<SOLUTION>%%
证明:由条件易知 $a_n$ 的值唯一确定.
定义一个 $0-1$ 序列是 "长为 $k$ 的循环序列", 若它的项数为 $k$, 且可由其前 $d$ 项重复 $\frac{k}{d}$ 次写出来,其中 $d$ 为某个小于 $k$ 且整除 $k$ 的正整数, 同时, 将 $\frac{k}{d}$ 的最大可能值称为该 0-1 序列的循环次数.
其余的 $0-1$ 序列称为 "不循环序列", 并将循环次数定义为 1 .
考虑由长为 $n$ 的 $0-1$ 序列组成的集合 $S$ 的元素个数.
一方面, $S$ 的元素个数显然为 $2^n$.
另一方面, 对 $S$ 中所有序列按循环次数分类计数.
显然每个长为 $n$ 的 $0-$ 1 序列有确定的循环次数 $c$, 且 $c$ 必须是 $n$ 的正约数, 对该正整数 $c$, 记 $d=\frac{n}{c}$. 对每个长为 $n$ 且循环次数是 $c$ 的 $0-1$ 序列, 其前 $d$ 项恰好对应一个长为 $d$ 的 "不循环序列"; 反之亦然.
因此, 若将长为 $m$ 的 "不循环序列" 的个数记为 $b_m$, 则 $b_d$ 等于长为 $n$ 且循环次数是 $c$ 的 $0-1$ 序列的个数.
对所有整除 $n$ 的正整数 $d$ 求和即得 $\sum_{d|n} b_d=2^n$, 特别地 $b_1=2=a_1$, 根据 $\left\{a_n\right\}$ 与 $\left\{b_n\right\}$ 同样的递推关系可知,对一切正整数 $n$,均有 $b_n=a_n$.
注意到每个长为 $n$ 的"不循环序列" 在允许轮换的情况下恰好对应 $n$ 个两两不同的"不循环序列", 因此 $n \mid b_n$, 从而 $n \mid a_n$.
%%<REMARK>%%
注:本题中引人长为 $m$ 的 "不循环序列" 的个数 $b_m$, 并证明 $b_n=a_n$ 及 $n \mid b_n$, 十分巧妙.
证明中有两个关键之处涉及对应的思想方法: 一是将长为 $n$ 且循环次数是 $c$ 的 $0-1$ 序列与长为 $d=\frac{n}{c}$ 的 "不循环序列"一一对应, 得以给出 $\left\{b_n\right\}$ 的递推公式,最终证得 $b_n=a_n$; 二是将每个长为 $n$ 的 "不循环序列" 与通过轮换得到的一组"不循环序列" 作" 1 对 $n$ " 的对应, 由此说明了 $n \mid b_n$.
%%PROBLEM_END%%



%%PROBLEM_BEGIN%%
%%<PROBLEM>%%
例8. 设 $a_n$ 为下述正整数 $N$ 的个数: $N$ 的各位数字之和为 $n$, 且每位数字只能取 1,3 或 4. 求证: 对 $n \in \mathbf{N}^*, a_{2 n}$ 是完全平方数.
%%<SOLUTION>%%
证明:记 $A$ 为数码仅有 $1,3,4$ 的数的全体, $A_n=\{N \in A \mid N$ 的各位数码之和为 $n\}$, 则 $\left|A_n\right|=a_n$. 欲证 $a_{2 n}$ 是完全平方数.
再记 $B$ 为数码仅有 1,2 的数的全体, $B_n=\{N \in B \mid N$ 的各位数码之和为 $n\}$, 令 $\left|B_n\right|=b_n$. 下证 $a_{2 n}=b_n^2$.
作映射 $f: B \rightarrow \mathbf{N}^*$, 对 $N \in B, f(N)$ 是由 $N$ 按如下法则得到的一个数: 把 $N$ 的数码从左向右看, 凡见到 2 , 把它与后面的一个数相加, 用和代替, 再继续看下去, 直到不能做为止 (例如, $f(1221212)=14$ 132, $f(21121221)= 31341)$. 易知 $f$ 是单射,于是
$$
f\left(B_{2 n}\right)=A_{2 n} \cup A_{2 n-2}^{\prime},
$$
其中 $A_{2 n-2}^{\prime}=\left\{10 k+2 \mid k \in A_{2 n-2}\right\}$. 所以
$$
b_{2 n}=a_{2 n}+a_{2 n-2} \text {. }
$$
但 $b_{2 n}=b_n^2+b_{n-1}^2$ (这是因为 $B_{2 n}$ 中的数或是由两个 $B_n$ 中的数拼接, 或是由两个 $B_{n-1}$ 中的数中间放 2 拼接而成), 所以
$$
a_{2 n}+a_{2 n-2}=b_n^2+b_{n-1}^2, n \geqslant 2 .
$$
因 $a_2=b_1^2=1$, 由上式便知, 对一切 $n \in N, a_{2 n}=b_n^2$, 即 $a_{2 n}$ 是完全平方数.
%%PROBLEM_END%%


