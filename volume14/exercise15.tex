
%%PROBLEM_BEGIN%%
%%<PROBLEM>%%
问题1. 有 11 只杯子都杯口朝上放着, 然后将它们任意翻偶数只, 算作一次操作 (翻过的也可以再翻). 证明: 无论操作多少次, 不能使 11 只杯子杯口都朝下.
%%<SOLUTION>%%
将杯口朝上的杯子赋值为 1 , 杯口朝下的杯子赋值为 -1 , 然后计算每操作一次后 11 只杯子乘积的正负号:
刚开始时, 11 只杯子都口朝上,所以它们乘积的符号为: $1^{11}=1$, 当翻动 $n$ 个杯子 ( $n$ 为偶数且 $n \leqslant 10)$ 使其杯口朝下时, 它们乘积的符号为:
$$
1^{11-n} \cdot(-1)^n=1 \times 1=1,
$$
显然, 无论 $n$ 是小于 11 的什么偶数, 乘积的符号均为正, 而 11 只杯子都口朝下时,乘积为 $(-1)^{11}=-1$, 故不可能使 11 只杯子杯口都朝下.
%%PROBLEM_END%%



%%PROBLEM_BEGIN%%
%%<PROBLEM>%%
问题2. 已知 $\triangle A B C$ 内有 $m$ 个点, 连同 $A, B, C$ 三点一共 $m+3$ 个点.
以这些点为顶点将 $\triangle A B C$ 分成若干个互不重叠的小三角形.
将 $A, B, C$ 三点分别染成红色、黄色、蓝色.
而三角形内的 $m$ 个点, 每个点任意染成红色、黄色、蓝色三色之一.
问: 三个顶点颜色都不同的小三角形的个数是奇数还是偶数?
%%<SOLUTION>%%
将每个小三角形的各条边按如下规则赋值: 若边的两端点同色, 赋值为 0 ; 若边的两端点异色, 赋值为 1 . 这样每个小三角形各边赋值之和, 有如下三种情况:
(1) 三个顶点都不同色的小三角形,赋值和为 3 ;
(2)只有两个顶点同色的小三角形,赋值和为 2;
(3) 三个顶点都同色的小三角形,赋值和为 0 .
另设所有小三角形的边赋值总和为 $S$, 上述 (1)、(2)、(3) 三类小三角形的个数分别为 $x, y, z$,于是有
$$
S=3 x+2 y+0 z=3 x+2 y . \label{eq1}
$$
在计算所有小三角形边的赋值总和 $S$ 时, 除了 $\triangle A B C$ 三边 $A B, B C$, $C A$ 外, 其余各边都被重复计算了两次, 设这些边的赋值和为 $k$, 则 $S=3+2 k$, 显然 $S$ 是一个奇数, 由 \ref{eq1} 式知 $x$ 是一个奇数, 即三个顶点颜色都不同的三角形的个数是一个奇数.
%%PROBLEM_END%%



%%PROBLEM_BEGIN%%
%%<PROBLEM>%%
问题3. 正方形 $A B C D$ 分割为 $n^2$ 个相等的小正方形 ( $n$ 是正整数), 把相对的顶点 $A, C$ 染成红色, 把 $B, D$ 染成蓝色, 其他交点任意染成红, 蓝两色之一.
证明: 恰有三个顶点同色的小正方形的数目必是偶数.
%%<SOLUTION>%%
对所有小正方形的顶点 $P_i\left(i=1,2, \cdots,(n+1)^2\right)$ 进行赋值, 将 $P_i$ 与数 $a_i$ 相对应: 若 $P_i$ 为红色, 记 $a_i$ 为 1 ; 若 $P_i$ 为蓝色, 记 $a_i$ 为 -1 .
对 $n^2$ 个小正方形进行编号.
对 $j=1,2, \cdots, n^2$, 设第 $j$ 个小正方形的四个顶点上的数字之积为 $A_j$, 于是, 若一个小正方形恰有三个顶点同色, 则 $A_i=$ -1 ; 否则 $A_i=1$. 于是问题就转化为证明在 $A_1, A_2, \cdots, A_n{ }^2$ 中, -1 的个数为偶数.
现在考虑乘积 $A_1 A_2 \cdots A_n{ }^2$. 对于正方形内部的交点, 每一点上相应的数在乘积中出现 4 次; 大正方形边上而不是顶点的交点上相应的数在乘积中出现 2 次; $A, B, C, D$ 这四点上相应的数在乘积中各出现一次,于是
$$
A_1 A_2 \cdots A_n^2=1 \times(-1) \times 1 \times(-1)=1 .
$$
所以,在 $A_1, A_2, \cdots, A_n{ }^2$ 中, -1 的个数为偶数.
即恰有三个顶点同色的小正方形的数目必是偶数.
%%PROBLEM_END%%



%%PROBLEM_BEGIN%%
%%<PROBLEM>%%
问题4. 如图(<FilePath:./figures/fig-c15p4.png>), 是一个向右和向下无限的表格.
一开始在左上角 $A$ 格内放一枚棋子, 此后每一步下棋规则如下: 若某格 $P$ 放有棋子, 且它的右边相邻两格 $Q$ 和 $R$ 都没有棋子, 则可将 $P$ 中的棋子去掉, 在 $Q 、 R$ 两格中各放一枚棋子; 同样若 $P$ 的下边相邻两格 $S$ 和 $T$ 都没有棋子, 则可将 $P$ 中的棋子去掉, 在 $S 、 T$ 两格中各放一枚棋子.
问: 能否在有限步后让所有棋子都不出现在前 4 列中?
%%<SOLUTION>%%
我们对第 $i$ 行第 $j$ 列的格子赋值 $\lambda^{i+j}, i, j \in \mathbf{N}^*$, 其中 $\lambda=\frac{\sqrt{5}-1}{2}$.
由于 $\lambda^{i+(j+1)}+\lambda^{i+(j+2)}=\lambda^{(i+1)+j}+\lambda^{(i+2)+j}=\lambda^{i+j}$, 故每下一步棋不改变所有棋子所在格的赋值之和, 记这个和为 $S$, 其中初始情况下的 $S=\lambda^2$.
假设下棋过程中某一时刻所有棋子都不出现在前 4 列, 那么此时
$$
S \leqslant \sum_{i=1}^{\infty} \sum_{j=5}^{\infty} \lambda^{i+j}=\lambda^6\left(\sum_{i=0}^{\infty} \lambda^i\right)^2=\lambda^6\left(\frac{1}{1-\lambda}\right)^2=\lambda^6\left(\frac{1}{\lambda^2}\right)^2=\lambda^2,
$$
说明从第 5 列开始的所有格子都已被摆满,这是不可能的.
因此,有限步后无法让所有棋子都不出现在前 4 列中.
%%PROBLEM_END%%



%%PROBLEM_BEGIN%%
%%<PROBLEM>%%
问题5. 如图(<FilePath:./figures/fig-c15p5.png>), 平面上由边长为 1 的正三角形构成一个 (无穷的)三角形网格.
三角形的顶点称为格点, 距离为 1 的格点称为相邻格点.
$A 、 B$ 两只青蛙进行跳跃游戏.
"一次跳跃" 是指青蛙从所在的格点跳至相邻的格点.
" $A 、 B$ 的一轮跳跃" 是指它们按下列规则进行的先 $A$ 后 $B$ 的跳跃:
规则(1): $A$ 任意跳一次,则 $B$ 沿与 $A$ 相同的跳跃方向跳跃一次,或沿与之相反的方向跳跃两次.
规则 (2) : 当 $A 、 B$ 两所在的格点相邻时,它们可执行规则 (1)完成一轮跳跃, 也可以由 $A$ 连跳两次, 每次跳跃均保持与 $B$ 相邻, 而 $B$ 则留在原地不动.
若 $A 、 B$ 的起始位置为两个相邻格点, 问: 能否经过有限轮跳跃, 使 $A$ 、 $B$ 恰好位于对方的起始位置上?
%%<SOLUTION>%%
不可能.
不妨设 $B$ 的起始位置在 $A$ 右方相邻格点上.
现为每个格点赋值如下:
先将 $A$ 初始所在格点赋值 1 , 以后赋值规则如下: 任意一个格点赋值为它左边相邻格点处乘以 $\omega$ (其中 $\omega=\frac{-1+\sqrt{3} i}{2}$ ), 又是它斜左下方与之相邻格点处值乘以 $\omega^2$.
开始时 $A=1, B=\omega$, 而由规则知, 任意一轮跳跃不改变 $A 、 B$ 所在格点的值的比值.
若最终 $A 、 B$ 能换位, 则 $\frac{1}{\omega}=\frac{\omega}{1}$, 矛盾!所以不能做到.
%%PROBLEM_END%%



%%PROBLEM_BEGIN%%
%%<PROBLEM>%%
问题6. 将 $3000 \times 3000$ 的正方形以任意方式分割为多米诺骨牌(指 $1 \times 2$ 的矩形) 的并.
证明: 一定可以用三种颜色对这些多米诺骨牌染色, 使得每种颜色的多米诺骨牌的个数都相等, 且对每个多米诺骨牌而言, 与其同色且相邻的多米诺骨牌的个数不大于 2 (两个多米诺骨牌称为相邻的, 是指它们各自所含的小方格至少有一个是彼此相邻的). 
%%<SOLUTION>%%
首先将 $3000 \times 3000$ 的方格表中 $(i, j)$ 位置的方格填上数 $a_{i j} \in\{1,2,3\}$, 其中 $a_{i j} \equiv i+j-1(\bmod 3)$.
我们按下面原则染色: 如果多米诺骨牌不含数字 $k \in \{1,2,3\}$, 则将其染成颜色 $k$. 下面证明这种染色是符合要求的.
考虑任意一个多米诺骨牌 $A$, 不妨设它水平放置且是颜色 3 的, 这样它包含的小方格所含数字为 $1 、 2$. 由 $a_{i j}$ 的
\begin{tabular}{|c|c|c|c|c|}
\hline 1 & 2 & 3 & 1 & $\cdots$ \\
\hline 2 & 3 & 1 & 2 & $\cdots$ \\
\hline 3 & 1 & 2 & 3 & $\cdots$ \\
\hline 1 & 2 & 3 & 1 & $\cdots$ \\
\hline$\cdots$ & $\cdots$ & $\cdots$ & $\cdots$ & $\cdots$ \\
\hline
\end{tabular}
填法知, 与 $A$ 相邻的 6 个小方格中有 4 个含数字 3 , 因此在上述染色方案下, 这 4 个小方格不会包含在颜色 3 的多米诺骨牌中.
这样, 与 $A$ 相邻的颜色 3 的多米诺骨牌至多只有 2 个.
最后证明,所有二种颜色的多米诺骨牌个数一样.
考虑颜色 2 与颜色 3 的多米诺骨牌的个数和, 它正好等于表格中含 1 的小方格的个数 3000 000. 类似可得, 颜色 1 与颜色 3 多米诺骨牌个数之和以及颜色 1 与颜色 2 多米诺骨牌个数之和都为 3000 000. 这样我们得到每种颜色的多米诺骨牌的个
\begin{tabular}{|l|l|l|l|}
\hline 2 & 3 & 1 & 2 \\
\hline 3 & 1 & 2 & 3 \\
\hline 1 & 2 & 3 & 1 \\
\hline
\end{tabular}
数都为 1500000 .
%%PROBLEM_END%%


