
%%PROBLEM_BEGIN%%
%%<PROBLEM>%%
问题1. 设 $X=\{1,2,3, \cdots, 1993\}, A$ 是 $X$ 的子集, 且满足:(1)对 $A$ 中任何两个数 $x \neq y$, 有 93 不整除 $x \pm y$. (2) $S(A)=1993$. 求 $|A|$ 的最大值.
%%<SOLUTION>%%
将 $X$ 划分为 47 个子集: $A_i=\{x \mid x \equiv i$, 或 $x \equiv 93-i(\bmod 93)\}$, $i=0,1,2, \cdots, 46$. 对任何 $x, y \in A_i$, 都有 $x-y \equiv 0(\bmod 93)$, 或 $x+y \equiv 0(\bmod 93)$, 所以 $A$ 至多含有 $A_i$ 中的一个数.
于是 $|A| \leqslant 47$. 令 $A=\{10$, $11, \cdots, 46\} \cup\{93,94, \cdots, 101\} \cup\{84\}$, 则 $A$ 合乎要求.
故 $|A|$ 的最大值为 47.
%%PROBLEM_END%%



%%PROBLEM_BEGIN%%
%%<PROBLEM>%%
问题2. 设 $X=\{1,2,3, \cdots, 10\}, A$ 是 $X$ 的子集, 且对任何 $x<y<z, x, y$, $z \in A$, 都存在一个三角形三边的长分别为 $x 、 y 、 z$. 求 $|A|$ 的最大值.
%%<SOLUTION>%%
首先, 令 $A_1=\{1,2,3,5,8\}, A_2=\{4,6,10\}, A_3=\{7,9\}$; 则 $A_i$ 中的任何 3 个数不构成三角形, 从而 $A$ 最多含有 $A_i$ 中的 2 个数, 所以 $|A| \leqslant 2 \times 3=6$. 其次, 令 $A=\{5,6,7,8,9,10\}$, 则 $A$ 合乎要求.
故 $|A|$ 的最大值为 6 .
%%PROBLEM_END%%



%%PROBLEM_BEGIN%%
%%<PROBLEM>%%
问题3. 设 $X=\{1,2,3, \cdots, 20\}, A$ 是 $X$ 的子集, 且对任何 $x<y<z, x, y$, $z \in A$, 都存.
在一个三角形三边的长分别为 $x 、 y 、 z$. 求 $|A|$ 的最大值.
%%<SOLUTION>%%
首先, 令 $A_1=\{1,2,3,5,8,13\}, A_2=\{4,6,10,16\}, A_3=\{7$, $12,19\}, A_4=\{9,11,20\}, A_5=\{14,15\}, A_6=\{17,18\}$, 则 $A_i$ 中的任何 3 个数不构成三角形, 从而 $A$ 最多含有 $A_i$ 中的 2 个数, 所以 $|A| \leqslant 2 \times 6=12$. 但若 $|A|=12$, 则 $14,15,17,18 \in A$, 于是, $1,2,3,5 \notin A$, 所以 8 , $13 \in A$, 于是, $7,9 \notin A$, 所以 $12,19,11,20 \in A$. 但 $8+12=20$, 矛盾.
所以 $|A| \neq 12$, 所以 $|A| \leqslant 11$. 其次, 令 $A=\{10,11,12, \cdots, 20\}$, 则 $A$ 合乎要求.
故 $|A|$ 的最大值为 11 .
%%PROBLEM_END%%



%%PROBLEM_BEGIN%%
%%<PROBLEM>%%
问题4. 设 $X=\{00,01, \cdots, 98,99\}$ 是 100 个二位数码的集合, $A$ 是 $X$ 的子集, 满足: 对任何一个由 0 到 9 中的数字构成的无穷序列中, 都有两个相邻的数字组成的二位数码属于 $A$, 求 $|A|$ 的最小值.
%%<SOLUTION>%%
令 $A_{i j}=\{\overline{i j}, \overline{j i}\}, i, j \in\{0,1,2, \cdots, 9\}$, 则 $A$ 至少含有 $A_{i j}$ 中的一个元素, 否则, 无穷序列 $i j i j i j i j \cdots$ 无相邻数码属于 $A$, 矛盾.
显然集合 $A_{i j}$ 共有 $10+\mathrm{C}_{10}^2=55$ 个(其中, $A_{00}, A_{11}, \cdots, A_{99}$. 有 10 个), 所以, $|A| \geqslant 55$. 此外, 令 $A=\{\overline{i j} \mid 0 \leqslant i \leqslant j \leqslant 9\}$, (即 $A_{i j}$ 中均取 $i \leqslant j$ 的下标), 则 $|A|=55$. 此时, 对任何无穷序列, 设它的最小数字为 $i$, 排在 $i$ 后面的一个数字为 $j$, 则 $i \leqslant j$, 那么 $\overline{i j} \in A$. 故 $|A|$ 的最小值为 55 .
%%PROBLEM_END%%



%%PROBLEM_BEGIN%%
%%<PROBLEM>%%
问题5. 在 $1,2, \cdots, 20$ 中最多能选出多少个数, 使其中任何一个选出来的数都不是另一个选出来的数的 2 倍.
并问: 这样的取数方法有多少种?
%%<SOLUTION>%%
令 $A_1=\{1,2,4,8,16\}, A_2=\{3,6,12\}, A_3=\{5,10,20\}$, $A_4=\{7,14\}, A_5=\{9,18\}, A_6=\{11\}, A_7=\{13\}, A_8=\{15\}, A_9=\{17\}, A_{10}=\{19\}$. 因为 $\{1,2\} 、\{4,8\} 、\{16\}$ 三个集合的每一个中最多取出一个数, 所以 $A_1$ 中最多可取 3 个数.
如果 $A_1$ 中取出 3 个数, 则必取 16 , 于是不能取 8 , 所以必取 4 , 于是不能取 2 , 所以必取 1 , 因而只有唯一的方法在 $A_1$ 中取出 3 个数.
同理可知, $A_2$ 和 $A_3$ 中分别最多可取 2 个数, 而且只有唯一的方法取出 2 个数.
$A_4 、 A_5$ 中都最多可取 1 个数, 但都有 2 种方法取出 2 个数.
其他集合都最多取出 1 . 个数, 且只有唯一的取法.
于是, 最多可取出 $3+2+ 2+(1+1)+1 \cdot 5=14$ 个数.
而且等号可以成立.
取数共有 $2 \cdot 2=4$ 种方法.
取出来的数构成的集合为 $X \cup Y_i(i=1 、 2 、 3 、 4)$. 其中 $X=\{1,4,16,3$, $12,5,20,11,13,15,17,19\}, Y_1=\{7,18\}, Y_2=\{7,9\}, Y_3=\{14$, $18\}, Y_4=\{14,9\}$. 故取出来的数的个数的最大值为 14 .
另解: 令 $A_1=\{1,2\}, A_2=\{3,6\}, A_3=\{4,8\}, A_4=\{5,10\}$, $A_5=\{7,14\}, A_6=\{9,18\}$. 则每个集合中最多取出一个数, 所以这些集合中最多可取 6 个数.
这些集合外还有 8 个数, 所以最多可取出 $6+8=14$ 个数.
若取出 14 个数, 则必取 $11 、 12 、 13 、 15 、 16 、 17 、 19 、 20$. 注意取了 $12 、 16 、 20$ 后不能取 $6 、 8 、 10$, 所以必取 $3 、 4 、 5$. 又取了 4 后不能取 2 , 所以必取 1 . 剩下 $A_5 、 A_6$ 中各取出 1 个数, 共有 $2 \cdot 2=4$ 种方法.
%%PROBLEM_END%%



%%PROBLEM_BEGIN%%
%%<PROBLEM>%%
问题6. 自然数 $k$ 满足如下性质: 在 $1,2, \cdots, 1988$ 中, 可取出 $k$ 个不同的数, 使其中任何两个数的和不被这两个数的差整除.
求 $k$ 的最大值.
%%<SOLUTION>%%
$k$ 的最大值为 663. 首先证明 $k \leqslant 663$. 我们注意如下的事实: 当 $x- y=1$ 或 2 时,有 $x-y \mid x+y$. 由此可知,在任何连续 3 个自然数中任取两个数 $x 、 y$, 必有 $x-y \mid x+y . k>663$ 时, 令 $A_i=\{3 i-2,3 i-1,3 i\}(i=1$, $2, \cdots, 662), A_{663}=\{1987,1988\}$, 将取出的 $k \geqslant 664$ 个数归人上述 663 个集合, 至少有一个集合含有其中的两个数 $x 、 y$, 此时具然有 $x-y \mid x+y$, 矛盾.
其次, 当 $k=663$ 时, 令 $A=\{1,4,7, \cdots, 1987\}$, 对 $A$ 中的任何两个数 $a_i 、 a_j$, 有 $a_i-a_j=(3 i-2)-(3 j-2)=3(i-j), a_i+a_j=(3 i-2)+(3 j-2)= 3(i+j-1)-1$, 所以 $a_i-a_j$ 不整除 $a_i+a_j$.
%%PROBLEM_END%%



%%PROBLEM_BEGIN%%
%%<PROBLEM>%%
问题7. 在集合 $X=\{1,2, \cdots, 50\}$ 的子集 $S$ 中, 任何两个元素的平方和不是 7 的倍数, 求 $|S|$ 的最大值.
%%<SOLUTION>%%
将 $X$ 划分为 2 个子集: $A_1=\{x \mid x \equiv 0(\bmod 7), x \in X\}, A_2= X \backslash A_1$. 则 $\left|A_1\right|=7,\left|A_2\right|=43$. 显然, $S$ 最多含有 $A_1$ 中的 1 个数, 于是 $|S| \leqslant 43+1=44$. 另一方面, 对任何整数 $x$, 若 $x \equiv 0 、 \pm 1 、 \pm 2 、 \pm 3(\bmod 7)$, 则 $x^2 \equiv 0 、 1 、 4 、 2(\bmod 7)$, 由此可见, 如果 $x^2+y^2 \equiv 0(\bmod 7)$, 则 $x^2 \equiv y^2 \equiv 0(\bmod 7)$, 于是, 令 $S=A_2 \cup\{7\}$, 则 $S$ 合乎要求, 此时 $|S|=44$. 故 $|S|$ 的最大值为 44 .
%%PROBLEM_END%%



%%PROBLEM_BEGIN%%
%%<PROBLEM>%%
问题8. 设 $X=\{1,2, \cdots, 1995\}, A$ 是 $X$ 的子集, 当 $x \in A$ 时, $19 x \notin A$, 求 $|A|$ 的最大值.
%%<SOLUTION>%%
令 $A_k=\{k, 19 k\}, k=6,7, \cdots, 105$, 则 $A$ 最多含有 $A_k$ 中的 1 个数, 于是 $|A| \leqslant 1995-100=1895$. 另一方面, 令 $A=\{1,2,3,4,5\} \cup \{106,107, \cdots, 1995\}$, 则 $A$ 合乎要求, 此时 $|A|=1895$. 故 $|A|$ 的最大值为 1895 .
%%PROBLEM_END%%


