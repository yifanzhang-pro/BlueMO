
%%TEXT_BEGIN%%
缩小包围圈.
先找一个合乎条件的"大范围",然后逐步缩小范围,使其范围仍然合乎条件,再思考范围在变化过程中为什么能继续合乎条件, 找到使范围合乎条件的本质因素, 由此使范围达到最佳估计; 或者分析确定范围的各种因素, 考察其中某些因素的"功能"(对范围的直接影响)是否可以优化或改进, 从而使范围的估计更精确.
我们称这种估计方法为缩小包围圈.
%%TEXT_END%%



%%PROBLEM_BEGIN%%
%%<PROBLEM>%%
例1. 在 $n \times n$ 棋盘 $C$ 中, 每格填一个数, 表的边缘各格所填的数都为 -1 . 对其他的任何空格, 填上与它同行或同列中它两侧最靠近它的已填的两个数的积.
求表中 1 的个数的最大值 $f(n)$ 与最小值 $g(n)$. 
\begin{tabular}{|c|c|c|c|c|c|}
\hline-1 & -1 & -1 & -1 & -1 & -1 \\
\hline-1 & $a$ & & & $b$ & -1 \\
\hline-1 & & & & & -1 \\
\hline-1 & $e$ & & & & -1 \\
\hline-1 & $d$ & $f$ & - & $c$ & -1 \\
\hline-1 & -1 & -1 & -1 & -1 & -1 \\
\hline
\end{tabular}
%%<SOLUTION>%%
分析:解首先, 显然有 $f(3)=g(3)=1$.
当 $n>3$ 时,我们希望 1 尽可能多.
能否都为 1 ? 通过尝试, 发现至少有一个 -1 . 实际上, 考察紧靠外围一周 -1 的格, 其中至少有一个格填 -1 . 否则, 考察表中 $a 、 b 、 c 、 d$ 四格, 设其中最后一个填数的格是 $d$. 而 $a 、 b 、 c$ 三格都已填 1 ,则 $d$ 不论何时填数,都只能填 -1 ,矛盾.
这样, $f(n) \leqslant(n-2)^2-1$.
其次, 由行积, $a$ 处可填 1 , 由列积, $b$ 处可填 1 , 再由行积, $e 、 c$ 处可填 1 , 又由列积, $f$ 处可填 1 , 这样, $d$ 处填 -1 . 对其他各格, 第二行和倒数第二行, 都由行积可填 1 , 其余的格都由列积可填 1 . 此时, 共有 $(n-2)^2-1$ 个 1 . 所以,
$f(n)=(n-2)^2-1$.
下面证明: $n>3$ 时, $g(n)=n-2$.
若 $g(n)=r \leqslant n-3$, 则表中 $r$ 个 1 至多占住 $n-3$ 行和 $n-3$ 列, 于是在中间的 $n-2$ 行和 $n-2$ 列中, 必有一行和一列都为 -1 , 但这是不可能的, 因为它们的交叉处无法填人 -1 , 矛盾.
所以, $g(n) \geqslant n-2$.
另一方面, 将第 2 列 $n-2$ 个格都填 1 (按行填), 再对每一行, 从右至左按行可填 -1 , 此时表中共有 $n-2$ 个 1 . 故 $g(n)=n-2$.
综上所述,
$$
f(n)= \begin{cases}1 & (n=3) ; \\ (n-2)^2-1 & (n>3) .\end{cases}
$$
$g(n)=n-2$.
%%PROBLEM_END%%



%%PROBLEM_BEGIN%%
%%<PROBLEM>%%
例2. 有 $h$ 个 $8 \times 8$ 棋盘, 每个棋盘上的格均可适当填上 $1,2,3, \cdots, 64$, 每个格填一个数, 使任何两个棋盘以任何方式重合时, 相同位置上的数不同.
求 $h$ 的最大值.
%%<SOLUTION>%%
分析:解.
先看特殊情形.
考察 $2 \times 2$ 棋盘, 我们发现两个这样的棋盘上任何两个格都有可能重合.
从而同一个棋盘上的 4 个格只能看作是一个类, 同类中的格只能填互异的正整数.
所以 $h \leqslant\left[\frac{\text { 填数的个数 }}{\text { 同类格的个数 }}\right]=\left[\frac{2 \times 2}{4}\right]=1$. 此时 $h$ 的最大值为 1 .
再考察 $3 \times 3$ 棋盘, 如右表, 棋盘上的格可以分为 $A 、 B 、 C 三$ 类,同一类中的任何两个格都有可能重合.
从而同一类中的格只
$B \quad C \quad B$ 能填互异的正整数.
注意到最大的类中有 4 个格.
所以
$\begin{array}{llll}C & A & C\end{array} h \leqslant\left[\frac{9}{4}\right]=2$.
当 $h=2$ 时,两个棋盘的填数如下:
$\begin{array}{llllll}2 & 6 & 3 & 1 & 2 & 6 \\ 9 & 1 & 7 & 5 & 9 & 3 \\ 5 & 8 & 4 & 8 & 4 & 7\end{array}$
所以 $h$ 的最大值为 2 ,
再考察 $4 \times 4$ 棋盘, 棋盘上的格可以分作 4 类,且同一类中的两个格位于两个不同的棋盘时均有可能重合.
从而同一类中的格的填数不能相同.
此时, 最大的类有 4 个数, 所以 $h \leqslant\left[\frac{16}{4}\right]=4$.
$\begin{array}{llll}B & C & D & B \\ D & A & A & C \\ C & A & A & D \\ B & D & C & B\end{array}$
而 $h=4$ 时, 只须 4 个棋盘的同类格中的数互不相同.
先填 4 个棋盘的 16
个 $A$ 类格.
第一个棋盘的 $A$ 类格填 $1 、 2 、 3 、 4$ ;第二个棋盘的 $A$ 类格填 5、6、 $7 、 8$; 第三个棋盘的 $A$ 类格填 9、10、11、12;第四个棋盘的 $A$ 类格填 $13 、 14$ 、 15、16. 再填 4 个棋盘的 $B$ 类格.
第一个棋盘的 $B$ 类格填 5、6、7、8;第二个棋盘的 $A$ 类格填 9、10、11、12;第三个棋盘的 $A$ 类格填 $13 、 14 、 15 、 16$; 第四个棋盘的 $A$ 类格填 $1 、 2 、 3 、 4$. 如此轮换, 得 $C 、 D$ 两类格的填法.
这种填法等价于将第一个棋盘的 $A$ 类格填 $1 、 2 、 3 、 4, B$ 类格填 $5 、 6 、 7 、 8, C$ 类格填 $9 、 10 、 11 、 12, D$ 类格填 $13 、 14 、 15 、 16$. 而第二个棋盘上的填数正好是第一个棋盘上的对应格的填数加 4(大于 16 者取除以 16 的余数), 第三个棋盘上的填数正好是第二个棋盘上的对应格的填数加 4 (大于 16 者取除以 16 的余数) 等等.
如下表:
$\begin{array}{rrrrrrrrrrrrrrrr}5 & 9 & 14 & 6 & 9 & 13 & 2 & 10 & 13 & 1 & 6 & 14 & 1 & 5 & 10 & 2 \\ 13 & 1 & 2 & 10 & 1 & 5 & 6 & 14 & 5 & 9 & 10 & 2 & 9 & 13 & 14 & 6 \\ 12 & 3 & 4 & 15 & 16 & 7 & 8 & 3 & 4 & 11 & 12 & 7 & 8 & 15 & 16 & 11 \\ 8 & 16 & 11 & 7 & 12 & 4 & 15 & 11 & 16 & 8 & 3 & 15 & 4 & 12 & 7 & 3\end{array}$
由上讨论不难知道, $8 \times 8$ 棋盘的格可以分为 16 个不同的类, 分别用 16 个字母 $A, B, \cdots, P$ 表示(如下表). 不同类的格不论以何种方式叠合棋盘都不会重合,而同类的格则有可能重合,于是同类格中要填互异的正整数.
每个棋盘的 $A$ 类格 4 个格,所以 $h \leqslant\left[\frac{64}{4}\right]=16$. 而 $h=16$ 时, 只须 16 个棋盘的同类格填数互不相同.
先填第一个棋盘的 64 个格.
第一个棋盘的 $A$ 类格填 $1 、 2 、 3 、 4 ; B$ 类格填 $5 、 6 、 7 、 8 ; \cdots \cdots, P$ 类格填 $61 、 62 、 63 、 64$. 而第二个棋盘上的填数正好是第一个棋盘上的对应格 的填数加 4 (大于 64 者取除以 64 的余数), 第三个棋盘上的填数正好是第二个棋盘上的对应格的填数加 4 (大于 64 者取除以 64 的余数) 等等.
64 个数正好填满同类格的 64 个格.
故 $h_{\max }=\left[\frac{8 \times 8}{4}\right]==16$.
另一种构造方式是: 将 64 个格分为 16 组: $A_1, A_2, \cdots, A_{16}$, 每组 4 个格.
对于第一个棋盘,第一组的 4 个格填 $1 、 2 、 3 、 4$, 第二组的 4 个格填 5、6、7、 8 , 第 16 组的 4 个格填 $61 、 62 、 63 、 64$. 对于第 $i$ 个棋盘, 将第 $i-1$ 个棋盘的第 $j$ 组所填的数作为将第 $i$ 个棋盘的第 $j+1$ 组填的数.
一般地, $h$ 个棋盘, 有 $4 h$ 个格, 这 $4 h$ 个格都有可能重合, 从而必须填互异数, 所以 $4 h \leqslant n^2$. 所以 $h \leqslant \frac{n^2}{4}$, 又 $\frac{n^2}{4} \in \mathbf{Z}$, 所以 $h \leqslant\left[\frac{n^2}{4}\right]$. 仿照类似的构造可使 $h=\left[\frac{n^2}{4}\right]$, 故 $h(n)=\left[\frac{n^2}{4}\right]$.
%%PROBLEM_END%%



%%PROBLEM_BEGIN%%
%%<PROBLEM>%%
例3. 彩票上依次排列着 50 个空格, 每个参加者都在彩票上填人 1 至 50 的整数 (每个数在同一张彩票中恰出现一次), 主持人亦填一张作底.
如果某人所填的数列有一个位置与底票上对应位置上填的数相同, 则可中彩.
试问: 一个参加者至少要填多少张彩票, 才能保证自己一定中彩?
%%<SOLUTION>%%
分析:解若适当填 $k$ 张彩票可以中彩, 则称 $k$ 是中彩的.
将所填的 $k$ 张彩票排成 $k \times 50$ 的数表:
\begin{tabular}{|c|c|c|c|c|}
\hline$a_1$, & $a_2$ & $a_3$ & $\cdots$, & $a_{50}$ \\
\hline$b_1$, & $b_2$ & $b_3$ & $\cdots$, & $b_{50}$ \\
\hline$\ldots$ & $\ldots$ & $\ldots$ & $\ldots$ & $\ldots$ \\
\hline$c_1$ & $c_2$ & $c_3$ & $\cdots$, & $c_{50}$ \\
\hline$x_1$, & $x_2$ & $x_3$ & $\cdots$, & $x_{50}$ \\
\hline
\end{tabular}
所谓 $k$ 是中彩的,即不论 $x_1, x_2, x_3, \cdots, x_{50}$ 如何填, 数表中至少有一个列, 此列中至少有一个数与此列中的 $x$ 相等.
我们称这样的数为好数.
这样, $k$ 是中彩的, 等价于存在 $k \times 50$ 数表, 使表中至少有一个好数.
显然, $k=50$ 是好的.
实际上, $50 \times 50$ 数表有 50 行、50 列, 每行是 1,2 , $3, \cdots, 50$ 的一个排列, 表中共有 50 个 1 , 我们可以将 50 个 1 占住 50 列, 即每列一个 1 . 这样, 不论 $x_1, x_2, x_3, \cdots, x_{50}$ 如何填, 其中的 1 必与表中的某个 1 同列.
进一步, $k=49$ 是好的.
可这样类似构造如下表:
$\begin{array}{ccccccc:c} & 1 & 2 & 3 & 4 & \cdots & 49 & 50 \\ & 49 & 1 & 2 & 3 & \cdots & 48 & 50 \\ & 48 & 49 & 1 & 2 & \cdots & 47 & 50 \\ \cdots & \cdots & \cdots & \cdots & \cdots & \cdots & \cdots \\ & 2 & 3 & 4 & 5 & \cdots & 1 & 50\end{array}$
因为底票中的数码 $1,2, \cdots, 49$ 不能都填在最后一个位置上, 即必有一个数码填在前面 49 个位置中, 它必与某一行的相应数码相同.
由此我们发现, 可让每个数 $i$ 都占住每一列, 这样, 在此表的下方的任何一列填人 $1,2, \cdots$, 50 中的任何一个数, 都必与该列中的一个数相等.
由此可见, 只要构造如下的数表:
$\begin{array}{ccccccc}1, & 2, & 3, & 4, & \cdots, & a-1, & a \\ 2, & 3, & 4, & 5, & \cdots, & a, & 1 \\ 3, & 4, & 5, & 6, & \ldots, & 1, & 2 \\ \ldots & \cdots & \ldots & \ldots & \cdots, & \cdots, & \cdots \\ a, & 1, & 2, & 3, & \cdots, & a-2, & a-1\end{array}$
而且底票上的数 $1,2, \cdots, a$ 中至少有一个填人前 $a$ 列, 则前 $k$ 列中必有一个好数.
要使底票上的数 $1,2, \cdots, a$ 中至少有一个填入前 $a$. 列, 只须 $1,2, \cdots$, $a$ 这 $a$ 个数不能都填在后 $50-a$ 列, 即 $50-a<a$, 得 $a>25$. 所以, $a \geqslant 26$. 由此可见, $k=26$ 是中彩的.
下面证明 $k=25$ 不是中彩的.
实际上, 考察 $25 \times 50$ 数表, 我们证明可以适当地填一张底票 $P: x_1, x_2, x_3, \cdots, x_{50}$, 使表中没有一个好数.
先填 $P$ 中的数 1 . 由于表中共有 25 个 1 , 有 50 列, 必有一个列中没有 1 , 在此列中填 1 即可.
按此方法再依次填 $2,3, \cdots, a-1$. 直至 $a$ 不能按上述方法填入.
此时, 必有 $a \geqslant$ 26. 否则, $a<26, P$ 中至多填了 $a-1 \leqslant 24$ 个数, 占了 24 个 $P$ 中的格, 但表中只有 25 个 $a$, 占了 25 个格, 共占了 $24+25$ 个格, $P$ 中至少还有一个格可填 $a$, 矛盾.
此外, 由于表中只有 25 个 $a$, 占了 25 个列, $P$ 中至少还有 25 个格可填 $a$. 但 $a$ 不能填人, 意味着这 25 个格被先填入的 $1,2,3, \cdots, a-1$ 中的 25 个数 (记为 $x_1, x_2, x_3, \cdots, x_{25}$ ) 占住.
考察 $P$ 中任一个空格 $D$, 由于 $D$ 所在的列只有 25 个数, 于是 $D$ 中至多有 25 个数不能填人.
这样, $a, x_1, x_2, x_3, \cdots, x_{25}$ 中至少有一个数可填人 $D$ 中, 但 $a$ 不能填人 $D$, 所以 $x_1, x_2, x_3, \cdots, x_{25}$ 中必有一个数 $x_i$ 可填人 $D$. 于是, 将 $x_i$ 填人 $D$, 并将 $a$ 填入 $x_i$ 原来所在的位置, 则底票上又新填人了一个数.
如此下去, 可将底票的空格填满, 使其中任何一个数都不与它所在列中任何一个数相同.
综上所述, $k$ 的最小值为 26 .
%%PROBLEM_END%%



%%PROBLEM_BEGIN%%
%%<PROBLEM>%%
例4. 某歌舞团有 $n(n>3)$ 名演员, 他们编排了一些节目, 每个节目都由 3 个演员同台表演.
在一次演出中, 他们发现: 能适当安排若干个节目, 使团中每 2 个演员都恰有一次在这次演出中同台表演, 求 $n$ 的最小值.
%%<SOLUTION>%%
解: $n$ 个点表示 $n$ 个演员, 若某 2 个演员有一次同台表演则将对应的点连边, 那么, 本题的条件等价于: 能将 $n$ 阶完全图 $K_n$ 分割为若干个 3 阶完全图 $K_3$, 使每一条边都恰属于一个 $K_3$.
显然, $\mathrm{C}_3^2 \mid \mathrm{C}_n^2$, 即 $6 \mid n(n-1)$, 所以 $3 \mid n$, 或 $3 \mid n-1$.
其次, (研究 $n$ 的另外的性质, 缩小包围圈), 考察含点 $A$ (以为顶点) 的边,
共有 $n-1$ 条, 每条边都恰属于一个 $K_3$, 从而共有 $n-1$ 个含点 $A$ (以为顶点) 的 $K_3$. 但每个含点 $A$ 的 $K_3$ 都有 2 条含点 $A$ 的边, 从而每个 $K_3$ 都被计算 2 次, 于是 $2 \mid n-1$, 所以 $n$ 为奇数.
由上可知, $3 \mid n$ ( $n$ 为奇数), 或 $6 \mid n-1$, 即 $n=6 k+3$, 或 $6 k+1\left(k \in \mathbf{N}_{+}\right)$, 于是 $n \geqslant 7$.
当 $n=7$ 时,将 7 个点用 $0,1,2,3,4,5,6$ 表示, 对 $m=0,1,2,3,4$ , $5,6,7$, 令 $m 、 m+1 、 m+3$ 组成一个 $K_3$ (下标模 7 理解, 也就是将 $0 、 1 、 3$ 构成的 $K_3$ 依次旋转 6 次), 则 7 个 $K_3$ 是合乎条件的分割.
综上所述, $n$ 的最小值为 7 .
%%PROBLEM_END%%



%%PROBLEM_BEGIN%%
%%<PROBLEM>%%
例5. 若自然数 $n$ 满足这样的条件: 存在由 $n$ 个实数组成的数列, 使得任何连续 17 个项之和为正, 而任何连续 10 个项之和为负, 求 $n$ 的最大值.
%%<SOLUTION>%%
解:$n$ 的最大值为 25 .
先证明 $n \leqslant 25$, 用反证法.
假设可以写出这样的 $n$ 个实数: $a_1, a_2, \cdots, a_n$, 且 $n \geqslant 26$, 则有如下性质:
(1) 任意连续 7 个项之和为正.
实际上, 考察连续 7 个项之和 $a_{i+1}+a_{i+2}+\cdots+a_{i+7}(i=0,1,2, \cdots, n- 7)$, 如果 $i \geqslant 10$, 则因为 $\left(a_{i-9}+a_{i-8}+\cdots+a_i\right)+\left(a_{i+1}+a_{i+2}+\cdots+a_{i+7}\right)>0$, 而 $a_{i-9}+a_{i-8}+\cdots+a_i<0$, 所以 $a_{i+1}+a_{i+2}+\cdots+a_{i+7}>0$. 若 $i \leqslant 9$, 则 $i+ 17 \leqslant 9+17=26 \leqslant n$. 因为 $\left(a_{i+1}+a_{i+2}+\cdots+a_{i+7}\right)+\left(a_{i+8}+a_{i+9}+\cdots+a_{i+17}\right)>$ 0 , 而 $a_{i+8}+a_{i+9}+\cdots+a_{i+17}<0$, 所以 $a_{i+1}+a_{i+2}+\cdots+a_{i+7}>0$.
(2) 任意连续 3 个项之和为负.
实际上, 考察连续 3 个项之和 $a_{i+1}+a_{i+2}+a_{i+3}(i=0,1,2, \cdots, n-3)$, 如果 $i \geqslant 7$, 则因为 $\left(a_{i-6}+a_{i-5}+\cdots+a_i\right)+\left(a_{i+1}+a_{i+2}+a_{i+3}\right)<0$, 而 $a_{i-6}+ a_{i-5}+\cdots+a_i>0$, 所以 $a_{i+1}+a_{i+2}+a_{i+3}<0$. 若 $i \leqslant 6$, 则 $i+10 \leqslant 6+10= 16 \leqslant n$. 因为 $\left(a_{i+1}+a_{i+2}+a_{i+3}\right)+\left(a_{i+4}+a_{i+5}+\cdots+a_{i+10}\right)<0$, 而 $a_{i+4}+ a_{i+5}+\cdots+a_{i+10}>0$, 所以 $a_{i+1}+a_{i+2}+a_{i+3}<0$.
(3) 任意连续 4 个项之和为正.
实际上,考察连续 4 个项之和 $a_{i+1}+a_{i+2}+a_{i+3}+a_{i+4}(i=0,1,2, \cdots$, $n-4)$, 如果 $i \geqslant 3$, 则因为 $\left(a_{i-2}+a_{i-1}+a_i\right)+\left(a_{i+1}+a_{i+2}+a_{i+3}+a_{i+4}\right)>0$, 而 $a_{i-2}+a_{i-1}+a_i<0$, 所以 $a_{i+1}+a_{i+2}+a_{i+3}+a_{i+4}>0$. 若 $i \leqslant 2$, 则 $i+7 \leqslant 2+7=9 \leqslant n$. 因为 $\left(a_{i+1}+a_{i+2}+a_{i+3}+a_{i+4}\right)+\left(a_{i+5}+a_{i+6}+a_{i+7}\right)>0$, 而 $a_{i+5}+a_{i+6}+a_{i+7}<0$, 所以 $a_{i+1}+a_{i+2}+a_{i+3}+a_{i+4}>0$.
(4)每个项都为正.
实际上, 考察任意一个项 $a_i(i=1,2, \cdots, n)$, 如果 $i \geqslant 4$, 则因为 $\left(a_{i-3}+\right. \left.a_{i-2}+a_{i-1}\right)+a_i>0$, 而 $a_{i-3}+a_{i-2}+a_{i-1}<0$, 所以 $a_i>0$. 若 $i \leqslant 3$, 则 $i+ 3 \leqslant 3+3=6 \leqslant n$. 因为 $a_i+\left(a_{i+1}+a_{i+2}+a_{i+3}\right)>0$, 而 $a_{i+1}+a_{i+2}+a_{i+3}<$ 0 , 所以 $a_i>0$.
显然 (4)与"任何连续 10 个项之和为负"矛盾, 所以 $n \leqslant 25$.
此外, $n=25$ 时, 我们可以构造出这样的 25 个实数.
设 25 个合乎条件的实数为 $a_1, a_2, \cdots, a_{25}$, 我们先研究它应满足的若干必要条件.
从前面的论证可知, 对 $i=0,1,2, \cdots, 25-7$, 除 $i=9$ 外,任意连续 7 个项之和 $a_{i+1}+a_{i+2}+\cdots+a_{i+7}$ 为正, 于是除 $i=6 、 16$ 外, 任意连续 3 个项之和 $a_{i+1}+a_{i+2}+a_{i+3}$ 为负, 为了使构造简单, 我们可综合考虑对称构造、周期构造和待定参数构造技巧, 设 25 个实数为
$$
a, a, b, a, a, b, c, c, c, b, a, a, d, a, a, b, c, c, c, b, a, a, b, a, a,
$$
其中 $a 、 b 、 c 、 d$ 为待定参数,满足 $a+a+b<0$, 取 $a=1, b=-3$ 实验,则数列变为
$$
1,1,-3,1,1,-3, c, c, c,-3,1,1, d, 1,1,-3, c, c, c,-3,
$$
$1,1,-3,1,1$.
为了使任何连续 10 个项之和为负, 要求 $d+(1+1-3)+3 c+(-3+ 1+1)<0$, 即 $d+3 c<2$;
为了使任何连续 17 个项之和为正, 要求 $(1-3)+3 c+(-3+1+1)+d+ (1+1-3)+3 c+(-3+1)>0$, 即 $d+6 c>6$. 所以 $6-6 c<d<2-3 c$, 由 $6- 6 c<2-3 c$, 得 $c>\frac{4}{3}$, 取 $c=1.5$, 则 $-3<d<-2.5$, 于是, 取 $d=-2.6$, 得到如下构造:
$$
1,1,-3,1,1,-3,1.5,1.5,1.5,-3,1,1,-2.6,1,1,-3,
$$
1. $5,1.5,1.5,-3,1,1,-3,1,1$.
为了得到各项绝对值较小的整数序列, 取 $d=-2.75$, 构造变为:
$$
1,1,-3,1,1,-3,1.5,1.5,1.5,-3,1,1,-2.75,1,1,-3 \text {, }
$$
$1.5,1.5,1.5,-3,1,1,-3,1,1$.
再将每个数乘以 4 , 得到如下构造:
$$
4,4,-12,4,4,-12,6,6,6,-12,4,4,-11,4,4,-12,6,6 \text {, }
$$
$6,-12,4,4,-12,4,4$.
%%<REMARK>%%
注:: 在上述构造中, 如果令 $a=c, b=d$, 则所构造的数列为 $a, a, b, a, a, b, a, a, a, b, a, a, b, a, a, b, a, a, a, b, a, a, b, a, a$, 其中 $a 、 b$ 为待定参数, 满足 $a+a+b<0,2(a+a+b)+3 a+b<0$,
$6(a+a+b)-b>0$, 取 $a=8, b=-19$ 即可.
于是, 得到的另一个合乎条件的数列为:
$$
8,8,-19,8,8,-19,8,8,8,-19,8,8,-19,8,8,-19,8,8 \text {, }
$$
$8,-19,8,8,-19,8,8$.
%%PROBLEM_END%%


