
%%TEXT_BEGIN%%
参数估计有些极值问题,因变动的因素较多,从表面上看情况相当复杂.
但适当引人新的参数, 便可将极值函数用参数表出.
这样, 离散极值问题被转化为一元或二元 (参数) 函数的极值问题, 从而使问题得到简化.
我们称这种求极值的方法为参数估计.
%%TEXT_END%%



%%PROBLEM_BEGIN%%
%%<PROBLEM>%%
例1. 已知 20 名体操运动员表演后, 9 名裁判分别给他们判定 1 到 20 名的名次.
已知, 每一个运动员得到的 9 个名次中, 最大者与最小者至多相差 3 . 现将各人得到的名次的和排列为 $c_1 \leqslant c_2 \leqslant \cdots \leqslant c_{20}$. 求 $c_1$ 的最大值.
%%<SOLUTION>%%
分析:解首先注意解题目标为 $c_1 \leqslant P$, 它等价于存在一个 $i$, 使得 $c_i \leqslant P$. 于是, 对任何一个选手的得分之和进行估计都可得到 $c_1$ 的估计.
要使得分 (名次)尽可能小, 则希望判他第一的多, 所以可估计得第一最多的选手的得分.
有一种情况是不证自明的, 即有选手得了 9 个第一, 此时显然有 $c_1 \leqslant 9$. 进而想到, 当"第一"分布比较分散时, 则应利用整体估计, 考察各个得了第一的选手的得分之和.
设共有 $r$ 个选手被判为第一.
(1) $r=1$, 则 $c_1=9$.
(2) $r=2$, 则因有 9 个第一, 至少有一个选手 $A$ 得到 5 个第一.
由于同一个选手所得到的各个得分相差不超过 3 , 从而 $A$ 的另 4 个得分中的任何一个都不超过 4. 于是, $A$ 的得分之和不多于 $5+4 \times 4=21$, 所以, $c_1 \leqslant 21$.
(3) $r=3$, 考虑各选手的总得分的总和 $S$. 他们共得到 9 个第一, 还有另外 18 个名次.
每个名次的得分至多为 4 . 所以 $S \leqslant 9+9 \times 3+9 \times 4=72$, 所以, $c_1 \leqslant \frac{72}{3}=24$.
(4) $r=4$, 同样估计此 4 个人的得分总和, 有 $S \leqslant 90$, 所以, $c_1 \leqslant\left[\frac{90}{4}\right]=$ 22.
(5) $r \geqslant 5$, 此 $r$ 个人的每个得分均不多于 4 , 从而至少有 $9 r \geqslant 5 \times 9=45$ 个不高于 4 的名次.
但 9 个裁判判定的 1 到 4 的名次至多有 $9 \times 4=36$ 个,
矛盾.
综上所述, $c_1 \leqslant 24$.
最后, $c_1=24$ 是可能的.
实际上, 令 $c_1=c_2=c_3=(1+1+1)+(3+ 3+3)+(4+4+4), c_4=(2+2+2+2+2)+(5+5+5+5), c_5=(2+ 2+2+2)+(5+5+5+5+5)$, 而 $c_i=i+i+\cdots+i(i=6,7,8, \cdots, 20)$. 见下表:
\begin{tabular}{|c|c|c|c|c|c|c|c|c|c|c|c|}
\hline 选手 & 1 & 2 & 3 & 4 & 5 & 6 & 7 & 8 & 9 & $\cdots$ & 20 \\
\hline 1 & 1 & 3 & 4 & 2 & 5 & 6 & 7 & 8 & 9 & $\cdots$ & 20 \\
\hline 2 & 1 & 3 & 4 & 2 & 5 & 6 & 7 & 8 & 9 & $\cdots$ & 20 \\
\hline 3 & 1 & 3 & 4 & 2 & 5 & 6 & 7 & 8 & 9 & $\cdots$ & 20 \\
\hline 4 & 3 & 4 & 1 & 2 & 5 & 6 & 7 & 8 & 9 & $\cdots$ & 20 \\
\hline 5 & 3 & 4 & 1 & 2 & 5 & 6 & 7 & 8 & 9 & $\cdots$ & 20 \\
\hline 6 & 3 & 4 & 1 & 5 & 2 & 6 & 7 & 8 & 9 & $\cdots$ & 20 \\
\hline 7 & 4 & 1 & 3 & 5 & 2 & 6 & 7 & 8 & 9 & $\cdots$ & 20 \\
\hline 8 & 4 & 1 & 3 & 5 & 2 & 6 & 7 & 8 & 9 & $\cdots$ & 20 \\
\hline 9 & 4 & 1 & 3 & 5 & 2 & 6 & 7 & 8 & 9 & $\cdots$ & 20 \\
\hline 名次和 & 24 & 24 & 24 & 30 & 33 & 54 & 63 & 72 & 81 & $\cdots$ & 180 \\
\hline
\end{tabular}
故 $c_1$ 的最大值为 24 .
%%PROBLEM_END%%



%%PROBLEM_BEGIN%%
%%<PROBLEM>%%
例2. 有 $r$ 个人参加象棋比赛, 每 2 人都比一局, 每局胜者得 2 分, 负者得 0 分, 平局每人 1 分.
比赛后, 恰有一人胜的局数最少, 且只有他得分最多, 求 $r$ 的最小值.
%%<SOLUTION>%%
分析:解先考虑条件: "恰有一人胜的局数最少, 且只有他得分最多", 设此人为 $A$. 为了计算 $A$ 的得分, 可引人参数: 设 $A$ 胜 $n$ 局, 平 $m$ 局, 则 $A$ 的得分为 $2 n+m$.
再考虑目标: " $r \geqslant$ ?", 显然成立的不等式是: $r \geqslant(m+n)+1$, 因而可先分别估计 $m 、 n$ 的范围.
为了利用 " $A$ 得分最多", 应计算其他人的得分.
这就要知道其他人胜、负局数,这正好利用条件: " $A$ 胜的局数最少". 对 $A$ 以外的任何人而言, 至少胜 $n+1$ 场, 至少得分 $2 n+2$. 于是, $2 n+2<2 n+m, m>2$, 即 $m \geqslant 3$.
这个估计虽不是最优的, 但由此可找到得分更高的人.
因为 $m \geqslant 3$ 表明: 至少有 3 个人与 $A$ 打成平局.
设其中的一个人为 $B$. 那么, $B$ 至少得 $(2 n+2)+1=2 n+3$ 分.
于是, $2 n+3<2 n+m, m>3$, 即 $m \geqslant 4$.
下证 $n>0$. 实际上, 若 $n=0$, 即 $A$ 未胜一场, 则 $A$ 的得分 $S(A) \leqslant r-1$. 但 $r$ 个人的得分总和为 $2 \mathrm{C}_r^2=r(r-1)$, 每个人平均可得分 $r-1$. 但 $A$ 的得分最多,所以, $S(A)=r-1$. 这样, 每个人都得 $r-1$ 分, 矛盾.
综上, $r \geqslant(m+n)+1 \geqslant 6$.
最后, $r=6$ 是可能的,各人得分如下表所示:
\begin{tabular}{|c|c|c|c|c|c|c|c|}
\hline & $A$ & $B$ & $C$ & $D$ & $E$ & $F$ & 总分 \\
\hline$A$ & & 1 & 1 & 1 & 1 & 2 & 6 \\
\hline$B$ & 1 & & 2 & 0 & 0 & 2 & 5 \\
\hline$C$ & 1 & 0 & & 0 & 2 & 2 & 5 \\
\hline$D$ & 1 & 2 & 2 & & 0 & 0 & 5 \\
\hline$E$ & 1 & 2 & 0 & 2 & & 0 & 5 \\
\hline$F$ & 0 & 0 & 0 & 2 & 2 & & 4 \\
\hline
\end{tabular}
故 $r$ 的最小值是 6 .
%%PROBLEM_END%%



%%PROBLEM_BEGIN%%
%%<PROBLEM>%%
例3. 有 $n(n \geqslant 5)$ 支足球队进行单循环赛.
每两队赛一场,胜队得 3 分, 负队得 0 分, 平局各得 1 分.
结果取得倒数第 3 名的队, 得分比名次在前面的队都少, 比后两名都多; 胜场数比名次在前面的队都多, 却比后两名都少.
求队数 $n$ 的最小值.
%%<SOLUTION>%%
解: $A_1, A_2, \cdots, A_{n-3}$ 是得分高的队, $B$ 是得分倒数第三的队, $C_1 、 C_2$ 是得分低的队.
引人参数: 设 $B$ 胜 $x$ 场平 $y$ 场负 $z$ 场, 则 $n=x+y+z+1$. 由于 $C_1$ 至少胜 $(x+1)$ 场, 所以 $3 x+y \geqslant 3(x+1)+1$, 即 $y \geqslant 4$. 由于 $A_1$ 至多胜 $(x-1)$ 场, 至多平 $(n-x)$ 场, 所以 $3 x+y+1 \leqslant 3(x-1)+(n-x)= 3 x+y+z-2$, 于是 $z \geqslant 3$, 即 $B$ 至少输 3 场, 因此 $A_1, \cdots, A_{n-3}$ 中至少有一队胜了 $B$. 于是 $x \geqslant 2$.
(1) 若所有 $A_i$ 均与 $C_1 、 C_2 、 B$ 中某个打平, 由于 $y=n-x-z-1 \leqslant n-6$, 故此 $(n-3)$ 队中至少有 3 个队和 $C_1 、 C_2$ 之一打平.
于是 $C_1 、 C_2$ 中有一队至少平两场, 故 $3 x+y \geqslant 3(x+1)+2+1$, 即 $y \geqslant 6$. 于是 $n \geqslant 12$. 但其等号不能成立, 否则由 $z=3$ 知 $A_i$ 均没有输过, 又至多胜 $x-1=1$ 场, 故至少平 10 场.
由此得 $B 、 C_1 、 C_2$ 一共至少平 18 场, 但由于 $y=6$, 故 $C_1 、 C_2$ 至多平 ${ }_1 2$ 场,于是 $18 \leqslant 6+2+2$,矛盾.
所以 $n \geqslant 13$.
(2) 若有某个 $A_i$ (设为 $A_1$ ) 和 $B 、 C_1 、 C_2$ 均未打平, 则它至多平 $(n-4)$ 场, 于是 $3 x+y+1 \leqslant 3(x-1)+(n-4)$, 得 $n \geqslant y+8 \geqslant 12$. 此时等号也不能成立, 否则 $y=4$, 由前面推导知 $C_1 、 C_2$ 没有平过, 且 $A_i$ 至少胜 $(x-1)$ 场, 否则它要平 11 场, 但 $C_1$ 未平过, 矛盾.
由此得 $A_i$ 胜 $(x-1)$ 场, 平 8 或 9 场.
设 $A_i$ 中有 $k$ 个平 8 场, $(9-k)$ 个平 9 场, $C_1 、 C_2$ 均胜 $(x+1)$ 场, 负 $(10-x)$ 场, 可得 $9(x-1)+x+2(x+1)=k(4-x)+(9-k)(3-x)+7-x+2(10-x)$, 即 $24 x=k+61$, 此方程在 $0 \leqslant k \leqslant 9$ 时无整数解, 所以 $n \geqslant 13$.
综合 (1)、(2), 对所有情况, 都有 $n \geqslant 13$.
另一方面, 当 $n=13$ 时, 令各队间的比赛结果如下表所示, 可知 $n=13$ 满足要求.
故 $n_{\min }=13$.
\begin{tabular}{|c|c|c|c|c|c|c|c|c|c|c|c|c|c|}
\hline & $A_1$ & $A_2$ & $A_3$ & $A_4$ & $A_5$ & $A_6$ & $A_7$ & $A_8$ & $A_9$ & $A_{10}$ & $B$ & $C_1$ & $C_2$ \\
\hline$A_1$ & & 3 & 1 & 1 & 1 & 1 & 1 & 1 & 1 & 1 & 1 & 0 & 3 \\
\hline$A_2$ & 0 & & 1 & 1 & 1 & 1 & 1 & 1 & 1 & 1 & 1 & 3 & 3 \\
\hline$A_3$ & 1 & 1 & & 3 & 1 & 1 & 1 & 1 & 1 & 1 & 1 & 0 & 3 \\
\hline$A_4$ & 1 & 1 & 0 & & 1 & 1 & 1 & 1 & 1 & 1 & 1 & 3 & 3 \\
\hline$A_5$ & 1 & 1 & 1 & 1 & & 1 & 1 & 1 & 1 & 1 & 3 & 0 & 3 \\
\hline$A_6$ & 1 & 1 & 1 & 1 & 1 & & 1 & 1 & 1 & 1 & 3 & 3 & 0 \\
\hline$A_7$ & 1 & 1 & 1 & 1 & 1 & 1 & & 1 & 1 & 1 & 3 & 3 & 0 \\
\hline$A_8$ & 1 & 1 & 1 & 1 & 1 & 1 & 1 & & 1 & 1 & 3 & 3 & 0 \\
\hline$A_9$ & 1 & 1 & 1 & 1 & 1 & 1 & 1 & 1 & & 1 & 3 & 3 & 0 \\
\hline$A_{10}$ & 1 & 1 & 1 & 1 & 1 & 1 & 1 & 1 & 1 & & 0 & 3 & 3 \\
\hline$B$ & 1 & 1 & 1 & 1 & 0 & 0 & 0 & 0 & 0 & 3 & & 3 & 3 \\
\hline$C_1$ & 3 & 0 & 3 & 0 & 3 & 0 & 0 & 0 & 0 & 0 & 0 & & 3 \\
\hline$C_2$ & 0 & 0 & 0 & 0 & 0 & 3 & 3 & 3 & 3 & 0 & 0 & 0 & \\
\hline
\end{tabular}
%%PROBLEM_END%%



%%PROBLEM_BEGIN%%
%%<PROBLEM>%%
例4. 有 1000 张编号为 $000,001, \cdots, 999$ 的证件和 100 个编号为 00 , $01, \cdots, 99$ 的盒子.
若盒子的号码可以由证件的号码划掉一个数字而得到, 则该证件可以放人该盒子中.
若选择 $k$ 个盒子可以装下所有证件, 求 $k$ 的最小值.
%%<SOLUTION>%%
分析:解找一个充分条件,使选定若干个盒子能装下所有的证件.
考察所有含有数字 $a 、 b 、 c(a 、 b 、 c$ 可能相同)为编号的证件, 要装下这些证件, 则一定有含有其中某两个数 (比如 $a 、 b$ ) 为编号的盒子.
为了保证数字的不同排列顺序为编号的证件都能装下,一个充分条件是, 编号为 $\overline{a b} 、 \overline{b a}$ 的盒子都被选.
也就是说,要使得"任何三个数中都有两个数被选取, 且这两个数的任何顺序都被选取". 由此联想到抽屉原理: 3 个数归人 2 个集合, 必有一个集合中有两个数.
于是, 将 $0,1,2, \cdots, 9$ 划分为两个子集 $A 、 B$, 只要同一集合中任何 2 数组都被选取, 且每个 2 数组的任何顺序(即所有可重复元素的 2 元排列)都被选取 (保证不管什么顺序都可放下), 则这些编号合乎要求.
设 $|A|=k,|B|=10-k$, 则 $A 、 B$ 中的元素可重复的 2 元排列分别有 $k^2 、(10-k)^2$ 个, 所以这样的 2 元排列共有 $k^2+(10-k)^2=2(k-5)^2+ 50 \geqslant 50$ 个.
取 $k=5$, 即 $|A|=|B|=5$, 比如, $A=\{0,1,2,3,4\}, B=\{5,6,7$, $8,9\}$. 则 $A 、 B$ 中的元素可重复的 2 元排列各有 25 个,相应的 50 个编号合乎要求.
下面证明 $k \geqslant 50$.
设选用的 $k$ 个盒子中, 以 9 为首位的编号最少, 设有 $m$ (参数) 个, 记为 $\overline{9 a_i} (i=1,2, \cdots, m)$. 令 $A=\left\{a_1, a_2, \cdots, a_m\right\}$, 任取不属于 $A$ 的两个数 $a 、 b$, 考察编号为 $\overline{9 a b}$ 的证件.
因为 $a$ 不属于 $A$, 所以没有编号为 $\overline{9 a}$ 的盒子.
同样, 没有编号为 $\overline{9 b}$ 的盒子.
于是必选编号为 $\overline{a b}$ 的盒子, 注意到 $a 、 b$ 均有 $10-m$ 个取值, 于是, 这样的盒子应有 $(10-m)^2$ 个, 且这些盒子都不以 $a_1, a_2, \cdots, a_m$ 为首位.
又由 "最少性", 以 $a_1, a_2, \cdots, a_m$ 为首位的盒子至少都有 $m$ 个, 所以这样的盒子至少有 $m^2$ 个.
于是, $k \geqslant m^2+(10-m)^2 \geqslant \frac{1}{2}[m+(10-m)]^2$. $=50$.
综上所述, $k$ 的最小值为 50 .
%%PROBLEM_END%%



%%PROBLEM_BEGIN%%
%%<PROBLEM>%%
例5. 求具有如下性质的最小自然数 $n$ : 把正 $n$ 边形 $S$ 的任何 5 个顶点染红色时, 总有 $S$ 的一条对称轴 $L$, 使每一红点关于 $L$ 的对称点都不是红点.
%%<SOLUTION>%%
解:正 $n$ 边形为 $A_1, A_2, \cdots, A_n$, 它有 $n$ 条对称轴.
记过边 $A_1 A_n$ 的中点的对称轴记为 $L_1$, 过 $A_1$ 的对称轴为 $L_2$, 过边 $A_1 A_2$ 的中点的对称轴记为 $L_3$, 过 $A_2$ 的对称轴为 $L_4, \cdots$, 过边 $A_{\left[\frac{n}{2}\right]} A_{\left[\frac{n+1}{2}\right]}$ 的中点的对称轴记为 $L_n$ (其中 $A_{\left[\frac{n}{2}\right]}$ 与 $A_{\left[\frac{n+1}{2}\right]}$ 可能重合). 易知, $A_i$ 与 $A_j$ 关于 $L_m$ 对称, 等价于 $i+j=m(\bmod n)$ (规定 $A_{n+i}=A_i$ ).
对任意集合 $X=\left\{a_1, a_2, \cdots, a_t\right\}$, 定义 $X^*=\left\{a_i+a_j \mid 1 \leqslant i \leqslant j \leqslant t\right\}$. 对于 $k \in\{3,4,5, \cdots, 13\}$, 将正 $k$ 边形的 5 个顶点 $A_1 、 A_2 、 A_4 、 A_6 、 A_7$ 染红色 (其中下标模 $k$ 理解, 且当互异顶点的个数少于 5 时, 将任意若干个点重复染红, 使之共有 5 个红点). 令 $P=\{1,2,4,6,7\}$, 则 $P^*=\{2,3,4, \cdots$, $14\}$, 所以 $P^*$ 中含有模 $k$ 的完系 (含有 $k$ 个连续自然数, 各数模 $k$ 理解), 于是, 对任意 $m(1 \leqslant m \leqslant k), P^*$ 中必有某个 $x \equiv m(\bmod k)$, 即存在 $i+j \equiv m(\bmod k)$, 其中 $i 、 j \in P$. 于是 $A_i 、 A_j$ 是关于 $L_m$ 对称的两个点.
所以,对任何
$k \in\{3,4,5, \cdots, 13\}$ 都不合乎要求,即 $n \geqslant 14$.
另一方面, 对正 14 边形, 将它的任意 5 个顶点 $A_{i_1}, A_{i_2}, \cdots, A_{i_5}$ 染红色, 令 $P=\left\{i_1, i_2, \cdots, i_5\right\}$,引人参数: 设 $P$ 中有 $r$ 个奇数, $5-r$ 个偶数, 则 $P^*$ 中奇数的个数为: $r(5-r) \leqslant\left[\left(\frac{5}{2}\right)^2\right]=6$ (其中 $P 、 P^*$ 中的数按模 14 理解, 这不改变各数的奇偶性). 于是, $\{1,3,5,7,9,11,13\}$ 中必有一个奇数 $m$, 使 $m \notin P^*$, 即对任何 $i 、 j \in P, i+j \notin P^*, i+j \neq m(\bmod 14)$. 于是任何.
个红点关于 $L_m$ 不对称.
故 $n_{\min }=14$.
%%PROBLEM_END%%



%%PROBLEM_BEGIN%%
%%<PROBLEM>%%
例6. 对于整数 $n \geqslant 4$, 求出最小的整数 $f(n)$, 使得对于任何正整数 $m$, 集合 $\{m, m+1, \cdots, m+n-1\}$ 的任一个 $f(n)$ 元子集中, 均有至少 3 个两两互素的元素.
%%<SOLUTION>%%
解: $n \geqslant 4$ 时, 对集合 $M=\{m, m+1, m+2, \cdots, m+n-1\}$, 若 $m$ 为奇数,则 $m, m+1, m+2$ 两两互质; 若 $m$ 为偶数,则 $m+1, m+2, m+3$ 两两互质, 于是 $M$ 的所有 $n$ 元子集中都至少有 3 个两两互质的数, 所以 $f(n)$ 存在, 且 $f(n) \leqslant n$.
设 $T_n=\{t \mid t \leqslant n+1$, 且 $2 \mid t$ 或 $3 \mid t\}$, 则 $T_n$ 为 $\{2,3, \cdots, n+1\}$ 的子集, 但 $T_n$ 中任何 3 个数都不两两互质,所以 $f(n) \geqslant\left|T_n\right|+1$.
由容斥原理, $\left|T_n\right|=\left[\frac{n+1}{2}\right]+\left[\frac{n+1}{3}\right]-\left[\frac{n+1}{6}\right]+1$, 所以 $f(n) \geqslant \left[\frac{n+1}{2}\right]+\left[\frac{n+1}{3}\right]-\left[\frac{n+1}{6}\right]+1$.
此外, 注意到 $\{m, m+1, m+2, \cdots, m+n\}=\{m, m+1, m+2, \cdots$, $m+n-1\} \cup\{m+n\}$, 所以 $f(n+1) \leqslant f(n)+1$.
所以 $f(4) \geqslant 4, f(5) \geqslant 5, f(6) \geqslant 5, f(7) \geqslant 6, f(8) \geqslant 7, f(9) \geqslant 8$.
下面证明 $f(6)=5$.
设 $x_1, x_2, x_3, x_4, x_5 \in\{m, m+1, m+2, \cdots, m+5\}$, 并设 $x_i(1 \leqslant i \leqslant 5)$ 中有 $k$ 个为奇数.
因为 $\{m, m+1, m+2, \cdots, m+5\}$ 中最多 3 个偶数, 从而 $x_1, x_2, x_3$, $x_4, x_5$ 中最多 3 个偶数, 所以 $k \geqslant 2$.
因为 $\{m, m+1, m+2, \cdots, m+5\}$ 中最多 3 个奇数, 从而 $x_1, x_2, x_3$, $x_4, x_5$ 中最多 3 个奇数, 所以 $k \leqslant 3$.
若 $k=3$, 则 3 个奇数两两互质;
若 $k=2$, 则不妨设 $x_1, x_2$ 为奇数, $x_3, x_4, x_5$ 为偶数, 当 $3 \leqslant i<j \leqslant 5$ 时, $\left|x_i-x_j\right| \leqslant(m+5)-m=5$, 但 $x_i-x_j$ 为偶数, 所以 $\left|x_i-x_j\right|=2$ 或 4 , 于是 $x_i \equiv x_j(\bmod 3), x_i \equiv x_j(\bmod 5)$, 从而 $x_3, x_4, x_5$ 至多 1 个为 3 的倍数,
也至多 1 个为 5 的倍数, 于是至少 1 个既不是 3 的倍数又不是 5 的倍数, 设这个数为 $x_3$.
考察 3 个数 $x_1, x_2, x_3$, 因为 $1 \leqslant i<j \leqslant 3$ 时, $\left|x_i-x_j\right| \leqslant(m+5)- m=5$, 所以 $x_i 、 x_j$ 的公因数不大于 5 , 但 $x_3$ 既不是 3 的倍数又不是 5 的倍数, 所以 $\left(x_1, x_3\right)=\left(x_2, x_3\right)=1$. 又 $x_1-x_2$ 为偶数, 所以 $\left|x_1-x_2\right|=2$ 或 4 , 所以 $\left(x_1, x_2\right)=1$, 即 $x_1, x_2, x_3$ 两两互质, 所以 $f(6)=5$.
由 $f(7) \geqslant 6, f(7) \leqslant f(6)+1=5+1=6$, 有 $f(7)=6$.
类似地, $f(8)=7, f(9)=8$.
这表明, 当 $4 \leqslant n \leqslant 9$ 时,
$$
f(n)=\left[\frac{n+1}{2}\right]+\left[\frac{n+1}{3}\right]-\left[\frac{n+1}{6}\right]+1 . \label{eq1}
$$
下面用数学归纳法证明 \ref{eq1} 式对所有大于 3 的正整数成立.
假设 $n \leqslant k(k \geqslant 9)$ 时 式\ref{eq1} 成立, 当 $n=k+1$ 时, 因为 $\{m, m+1, m+2$, $\cdots, m+k\}=\{m, m+1, m+2, \cdots, m+k-6\} \cup\{m+k-5, m+k-4$, $\cdots, m+k\}$, 所以 $f(k+1) \leqslant f(k-5)+f(6)-1$, 由归纳假设, 有
$$
\begin{aligned}
f(k+1) \leqslant & \left(\left[\frac{k-5+1}{2}\right]+\left[\frac{k-5+1}{3}\right]-\left[\frac{k-5+1}{6}\right]+1\right)+ \\
& \left(\left[\frac{6+1}{2}\right]+\left[\frac{6+1}{3}\right]-\left[\frac{6+1}{6}\right]+1\right)-1 \\
= & \left(\left[\frac{k}{2}\right]-2+\left[\frac{k-1}{3}\right]-1-\left[\frac{k-4}{6}\right]+1\right)+(3+2-1+1)-1 \\
= & \left(\left[\frac{k}{2}\right]+1+\left[\frac{k-1}{3}\right]+1-\left[\frac{k-4}{6}\right]-1\right)+1 \\
= & {\left[\frac{k+2}{2}\right]+\left[\frac{k+2}{3}\right]-\left[\frac{k+2}{6}\right]+1, \text { (1)式成立.
} }
\end{aligned}
$$
故对所有大于 3 的正整数 $n$,有 $f(n)=\left[\frac{n+1}{2}\right]+\left[\frac{n+1}{3}\right]-\left[\frac{n+1}{6}\right]+1$.
%%PROBLEM_END%%



%%PROBLEM_BEGIN%%
%%<PROBLEM>%%
例7. 某国足球联赛有 $n(n \geqslant 6)$ 支球队参加, 每支球队都有两套不同颜色的队服, 一套为主场队服, 一套为客场队服.
两支球队进行比赛时, 若两队的主场队服颜色不同, 则两队均穿主场队服; 若两队的主场队服颜色相同, 则主场队穿其主场队服, 客场队穿其客场队服.
已知任意两场在四支不同球队间的比赛中至少出现 3 种不同颜色的队服.
所有 $n$ 支球队的共 $2 n$ 套队服中, 至少使用了多少种不同的颜色?
%%<SOLUTION>%%
解:些队服至少使用了 $n-1$ 种不同的颜色.
首先构造使用 $n-1$ 种颜色的例子: 设这 $n-1$ 种颜色为 $C_1, C_2, \cdots, C_{n-1}$,
$n$ 支队伍为 $T_1, T_2, \cdots, T_n$. 其中, 队伍 $T_1 、 T_2$ 的主场队服的颜色均为 $C_1$, 客场队服的颜色均为 $C_2$, 队伍 $T_i(3 \leqslant i \leqslant n)$ 的主场队服颜色为 $C_{i-1}$, 客场队服颜色为 $C_{i-2}$.
由题目所述的规则, 当两队比赛时, 主客场两支球队的主场队服颜色均会出现在场上.
对于任意两场在四支不同球队间的比赛, 这四支球队的主场队服至少有三种不同的颜色, 这些颜色都会出现在比赛中, 因此这样的设计满足题目条件.
下面假设可以使用不超过 $n-2$ 种颜色, 使得任意两场在四支不同球队间的比赛中至少出现 3 种不同颜色的队服.
不妨设恰好使用了 $n-2$ 种颜色, 并设这 $n-2$ 种颜色为 $C_1, C_2, \cdots, C_{n-2}$. 令 $x_i$ 为主场队服颜色为 $C_i$ 的队伍的个数, 显然 $\sum_{i=1}^{n-2} x_i=n$, 由抽屉原理至少有一个 $x_i \geqslant 2$, 不妨设 $x_1 \geqslant 2$.
若另外还有一个 $x_j \geqslant 2$, 则不妨设 $a 、 c$ 是两支主场队服颜色为 $C_1$ 的队伍, $b 、 d$ 是两支主场队服颜色为 $C_j$ 的队伍, $a$ 队主场 $b$ 队客场的比赛与 $c$ 队主场 $d$ 队客场的比赛中, 四支队伍均穿其主场队服, 场.
队员的队服都是 $C_1$ 和 $C_j$ 两种颜色,矛盾, 因此其余的 $x_j$ 全部为 0 或 1 , 故 $x_1 \geqslant n-((n-2)-1)=3$.
若 $x_1=3$, 则 $x_2=x_3=\cdots=x_{n-2}=1$, 设主场队服颜色为 $C_1$ 的三支队伍为 $a 、 b 、 c$, 并选取另一球队 $d$, 使得 $d$ 的主场队服颜色与 $b$ 的客场队服颜色相同 (由 $x_2=x_2==\cdots=x_{n-2}=1$ 必然可以选出), 这样在 $a$ 队主场 $b$ 队客场的比赛与 $c$ 队主场 $d$ 队客场的比赛中, 只有 $b$ 队穿客场队服, 场上队员的队服只有两种不同颜色,矛盾.
若 $x_1 \geqslant 4$, 则考虑所有主场队服颜色为 $C_1$ 的队伍的客场队服, 以及其他队伍的主场队服.
这一共 $n$ 套队服仅有 $n-2$ 种不同的颜色供选择, 由抽屉原理必然存在两套队服颜色相同, 因为前面已经证明除 $x_1$ 外其余的 $x_j$ 全部为 0 或 1 , 所以必然出现下面两种情况之一:
情况 1 : 有两支主场队服颜色为 $C_1$ 的队伍, 它们的客场队服颜色也相同.
设这两支队伍为 $b$ 和 $d$, 另取两支主场队服颜色为 $C_1$ 的队伍 $a 、 c$, 则在 $a$ 队主场 $b$ 队客场的比赛与 $c$ 队主场 $d$ 队客场的比赛中, $b 、 d$ 两队穿客场队服, 场上队员的队服只有两种不同颜色,矛盾.
情况 2 : 有一支主场队服颜色为 $C_1$ 的队伍的客场队服, 与另一支队伍的主场队服颜色相同.
设前者为 $b$, 后者为 $d$, 另取两支主场队服颜色为 $C_1$ 的队伍 $a 、 c$, 则在 $a$ 队主场 $b$ 队客场的比赛与 $c$ 队主场 $d$ 队客场的比赛中, 只有 $b$ 队穿客场队服,场上队员的队服只有两种不同颜色,矛盾.
因此, 假设不成立, 即无法使用不超过 $n-2$ 种颜色来达到题目要求, 故这些队服至少使用了 $n-1$ 种不同的颜色.
%%PROBLEM_END%%


