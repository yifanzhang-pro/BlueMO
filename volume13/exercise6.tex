
%%PROBLEM_BEGIN%%
%%<PROBLEM>%%
问题1. 设 $X$ 是 $\mathbf{N}^*$ 的子集, $X$ 的最小元为 1 ,最大元为 100 ,对 $X$ 中任何一个大于 1 的数,都可表成 $X$ 中两个数 (可以相同) 的和, 求 $|X|$ 的最小值.
%%<SOLUTION>%%
$|X|_{\min }=9$. 设 $X=\left\{a_1, a_2, \cdots, a_n\right\}$, 且 $1=a_1<a_2<\cdots<a_n=$ 100. 我们估计 $a_k 、 a_{k-1}$ 间的间距.
任取 $a_k \in X(k>1)$, 都存在 $a_i 、 a_j$, 使 $a_k=a_i+a_j$, 其中必有 $a_i<a_k, a_j<a_k$, 即 $a_i \leqslant a_{k-1}, a_j \leqslant a_{k-1}$, 所以 $a_k= a_i+a_j \leqslant a_{k-1}+a_{k-1}=2 a_{k-1}$. 若对所有 $k$, 有 $a_k=2 a_{k-1}$, 则 $X$ 中的所有数都是 2 的方幂, 与 $100 \in X$ 矛盾.
于是存在 $k$, 使 $a_k<2 a_{k-1}=a_{k-1}+a_{k-1}$, 这样, $a_k \leqslant a_{k-1}+a_{k-2} \leqslant 2 a_{k-2}+a_{k-2}=3 a_{k-2}$. 利用上述找到的 $k$, 我们有 $100=a_n \leqslant 2 a_{n-1} \leqslant 4 a_{n-2} \leqslant \cdots \leqslant 2^{n-k} a_k \leqslant 3 \times 2^{n-k} a_{k-2} \leqslant 3 \times 2^{n-k+1} a_{k-3} \leqslant \cdots \leqslant 3 \times 2^{n-3} a_1= 3 \times 2^{n-3}$, 所以 $n \geqslant 9$. 另一方面, 当 $n=9$ 时, 存在合乎条件的集合 $X=\{100$, $50,25,13,12,6,3,2,1\}$, 所以 $n$ 的最小值为 9 . 另解: 设 $X=\left\{a_1, a_2, \cdots\right.$, $\left.a_n\right\}$, 且 $1=a_1<a_2<\cdots<a_n=100$. 我们估计 $a_k 、 a_{k-1}$ 间的间距.
任取 $a_k \in X(k>1)$, 都存在 $a_i 、 a_j$, 使 $a_k=a_i+a_j$, 其中必有 $a_i<a_k, a_j<a_k$, 即 $a_i \leqslant a_{k-1}, a_j \leqslant a_{k-1}$, 所以 $a_k=a_i+a_j \leqslant a_{k-1}+a_{k-1}=2 a_{k-1}$. 所以 $100=a_n \leqslant 2 a_{n-1} \leqslant 4 a_{n-2} \leqslant \cdots \leqslant 2^{n-1} a_1=2^{n-1}$, 所以 $n \geqslant 8$. 若 $n=8$, 则 $1=a_1<a_2<\cdots< a_8=100$. 易知, $a_2=2=2^{2-1}$, 否则 $a_2$ 不能表成 $X$ 中的两个数的和.
再注意到$a_8=100$ 不是 2 的方幂, 可设 $X$ 中 $a_k=2^{k-1}(k=1,2, \cdots, t-1), a_t \neq 2^{t-1}$. 由于 $a_t$ 可表成 $X$ 中两数之和, 所以存在 $a_i 、 a_j(i, j<t)$, 使 $a_t=a_i+a_j$. 若 $a_i=a_j$, 则 $a_t=2 a_i=2 \times 2^{i-1}=2^i$. 若 $i=t-1$, 则与 $a_t \neq 2^{t-1}$ 矛盾.
所以 $i< t-1$, 所以 $a_t=2 a_i=2^i=a_{i+1}$, 与 $X$ 中的数互异矛盾.
所以 $a_i \neq a_j$. 所以 $a_j<2^{t-2}, a_i<2^{t-3}$, 所以 $a_t=a_i+a_j<2^{t-2}+2^{t-3}$. 所以 $100=a_8=2^{8-t} a_t< 2^{8-t}\left(2^{t-2}+2^{t-3}\right)=2^6+2^5=96$,矛盾.
所以 $n \neq 8$, 于是 $n \geqslant 9$.
%%PROBLEM_END%%



%%PROBLEM_BEGIN%%
%%<PROBLEM>%%
问题2. 的在 $n \times n$ 棋盘 $C$ 中, 两个具有公共顶点的格称为是相连的.
将 $1,2,3, \cdots$, $n^2$ 分别填人各格中, 每格填一个数.
若任何相连的两个格的数至多相差 $g$, 则称 $g$ 为一个 $C$-间隙.
求出最小的 $C$-间隙 $C_g$.
%%<SOLUTION>%%
$C_{\dot{g}}=n+1$. 首先, 第 $i$ 行依次填 $(i-1) n+1,(i-1) n+2, \cdots,(i-1) n+ n$. 此时, 数表的 $C$-间隙为 $n+1$. 对任何一个数表, 设 $g$ 是它的 $C$-间隙.
即对任何两个相连的数 $x 、 y$, 有 $|x-y| \leqslant g$. 我们证明: $g \geqslant n+1$. 将 1 和 $n^2$ 所在的格用一条链连接, 此链 (包括 1 和 $n^2$ 所在的格) 至多有 $n$ 个格.
不妨设共有 $m$ 个格 $(m \leqslant n)$, 这 $m$ 个格中的数依次为 $a_1=1, a_2, a_3, \cdots, a_m=n^2$. 考察各相连两数之差, 有 $\left|a_2-a_1\right|+\left|a_3-a_2\right|+\cdots+\left|a_m-a_{m-1}\right| \geqslant\left(a_2-a_1\right)+ \left(a_3-a_2\right)+\cdots+\left(a_m-a_{m-1}\right)=a_m-a_1=n^2-1$. 于是, 必有一个 $i$, 使 $\mid a_i- a_{i-1} \mid \geqslant \frac{n^2-1}{m-1} \geqslant \frac{n^2-1}{n-1}=n+1$.
%%PROBLEM_END%%



%%PROBLEM_BEGIN%%
%%<PROBLEM>%%
问题3. 设 2005 条线段首尾相连,组成封闭的折线,且折线的任何两段都不在同一条直线上, 那么, 此折线自身相交的交点最多有多少个?
%%<SOLUTION>%%
考虑一般问题: 设 $f(n)$ 是 $n$ ( $n$ 为奇数) 段折线自身相交的交点的个数的最大值.
取定一个点, 由此点出发沿折线前进, 依次经过的段分别叫做第 $1,2, \cdots, n$ 段.
画前两段时, 没有交点.
再画第 3 段,最多与第一段有一个交点.
考虑第 4 段, 它最多与前两段有两个交点.
如此下去, 画第 $i$ 段时, 它最多与前面的 $i-2$ 段有 $i-2$ 个交点 ( $i$ 段与 $i-1$ 段相连, 不相交.
$i$ 段与本身不相交), 其中 $i=1,2, \cdots, n-1$. 最后, 画第 $n$ 段时, 最多产生 $n-3$ 个交点, 从而最多有 $\sum_{i=3}^{n-1}(i-2)+(n-3)=\frac{1}{2}(n-2)(n-3)+\frac{1}{2} \cdot 2(n-3)=\frac{1}{2} n(n-3)$ 个交点.
所以 $f(n) \leqslant \frac{1}{2} n(n-3)$. 另一方面, 当 $n$ 为奇数时, 存在 $n$ 段折线, 使 $f(n)=\frac{1}{2} n(n-3)$. 实际上, 取定两点 $A_1 、 A_2$, 以 $A_1 A_2$ 为直径作圆.
从 $A_1$ 开始, 在半圆上按逆时钟方向排列点 $A_1, A_3, \cdots, A_n$; 从 $A_2$ 开始, 在另一半圆上按逆时钟方向排列点 $A_2, A_4, A_6, \cdots, A_{n-1}$, 则闭折线 $A_1 A_2 \cdots A_n$ 各段之间有 $\frac{1}{2} n(n-3)$ 个交点.
故本题答案是: $f(2005)=\frac{1}{2}(2005 \times 2002)= 2005 \times 1001=2007005$.
%%PROBLEM_END%%



%%PROBLEM_BEGIN%%
%%<PROBLEM>%%
问题4. 设集合 $M=\{1,2, \cdots, 10\}$ 的五元子集 $A_1, A_2, \cdots, A_k$ 满足条件: $M$ 中的任意两个元素最多在两个子集 $A_i$ 与 $A_j(i \neq j)$ 内出现, 求 $k$ 的最大值.
%%<SOLUTION>%%
记 $i(i=1,2, \cdots, 10)$ 在 $A_1, A_2, \cdots, A_k$ 中出现的次数为 $d(i)$. 首先证明 $d(i) \leqslant 4(i=1,2, \cdots, 10)$. 事实上,对 $i \in M, i$ 与 $M$ 中另 9 个元素中的某个元素 $j(j \neq i)$ 组成二元组 $(i, j)$ 在所有满足题设条件的五元子集 $A_1,A_2, \cdots, A_k$ 中最多出现两次, 因此 $i$ 参与组成的 9 个二元组 $(i, j)$ 最多出现 $2 \times 9=18$ 次.
由于 $\left|A_j\right|=5(j=1,2, \cdots, k)$, 而每个含 $i$ 的子集恰有 4 个二元组 $(i, j)$, 因此 $4 \cdot d(i) \leqslant 18$, 故 $d(i) \leqslant 4$. 其次, $k$ 个 5 元子集共有 $5 k$ 个元素, 而每个 $M$ 的元素在 $A_1, \cdots, A_k$ 中出现的次数为 $d(i)$, 从而 $5 k= d(1)+\cdots+d(10) \leqslant 4 \times 10$, 所以 $k \leqslant 8$. 最后, 当 $k=8$ 时, 下述 8 个 5 元数集满足要求: $A_1=\{1,2,3,4,5\} 、 A_2=\{1,6,7,8,9\} 、 A_3=\{1,3,5,6$, $8\} 、 A_4=\{1,2,4,7,9\} 、 A_5=\{2,3,6,7,10\} 、 A_6=\{3,4,7,8,10\}$ 、 $A_7=\{4,5,8,9,10\} 、 A_8=\{2,5,6,9,10\}$, 故 $k_{\max }=8$.
%%PROBLEM_END%%


