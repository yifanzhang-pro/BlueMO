
%%PROBLEM_BEGIN%%
%%<PROBLEM>%%
问题1. 设 $a 、 b 、 c 、 a+b-c 、 b+c-a 、 c+a-b 、 a+b+c$ 是 7 个两两不同的质数,且 $a 、 b 、 c$ 中有两个数的和是 800 . 设 $d$ 是这 7 个质数中最大数与最小数的差,求 $d$ 的最大可能值.
%%<SOLUTION>%%
不妨设 $a<b<c$, 则 7 个数中的最小数与最大数分别为 $a+b-c 、 a+ b+c$, 于是 $d=(a+b+c)-(a+b-c)=2 c$. 又 $a+b-c>0$, 所以 $c<a+ b<a+c<b+c$, 但 $a+b 、 a+c 、 b+c$ 中有一个为 800 , 所以 $c<800$. 又 $799=17 \times 47$ 、 798 都不是质数, 所以 $c \leqslant 797, d=2 c \leqslant 1594$. 令 $c=797$, $a+b=800$, 注意到 $b<c=797$, 满足 $a+b=800$ 的最小质数解 ( $a$, $b)=(13,787)$ (因为 $795,793=13 \times 61,789=3 \times 263$ 都不是质数), 此时, $a+b-c=3 、 a-b+c=23 、-a+b+c=1571 、 a+b+c=1597$ 都是质数, 故 $d$ 的最大可能值是 1594 .
%%PROBLEM_END%%



%%PROBLEM_BEGIN%%
%%<PROBLEM>%%
问题2. 设 $2 n$ 个实数 $a_1, a_2, \cdots, a_{2 n}$, 满足条件 $\sum_{i=1}^{2 n-1}\left(a_{i+1}-a_i\right)^2=1$, 求 $\left(a_{n+1}+\right. \left.a_{n+2}+\cdots+a_{2 n}\right)-\left(a_1+a_2+\cdots+a_n\right)$ 的最大值.
%%<SOLUTION>%%
当 $n=1$ 时, $\left(a_2-a_1\right)^2=1$, 所以 $a_2-a_1= \pm 1$, 易知此时欲求的最大值为 1 . 当 $n \geqslant 2$ 时, 设 $x_1=a_1, x_{i+1}=a_{i+1}-a_i, i=1,2, \cdots, 2 n-1$, 则 $\sum_{i=2}^{2 n} x_i^2=1$, 且 $a_k=x_1+x_2+\cdots+x_k, k=1,2, \cdots, 2 n$. 所以, 由柯西不等式得 $\left(a_{n+1}+a_{n+2}+\cdots+a_{2 n}\right)-\left(a_1+a_2+\cdots+a_n\right)=n\left(x_1+x_2+\cdots+\right. \left.x_n\right)+n x_{n+1}+(n-1) x_{n+2}+\cdots+x_{2 n}-\left[n x_1+(n-1) x_2+\cdots+x_n\right]=x_2+ 2 x_3+\cdots+(n-1) x_n+n x_{n+1}+(n-1) x_{n+2}+\cdots+x_{2 n} \leqslant \sqrt{1^2+2^2+\cdots+(n-1)^2+n^2+(n-1)^2+\cdots+1^2} \sqrt{x_2^2+x_3^2+\cdots+x_{2 n}^2}= \sqrt{n^2+2 \times \frac{(n-1) n(2(n-1)+1)}{6}}=\sqrt{\frac{n\left(2 n^2+1\right)}{3}}$, 当 $a_k=\frac{\sqrt{3} k(k-1)}{2 \sqrt{n\left(2 n^2+1\right)}}$
$$
(k=1,2, \cdots, n+1), a_{n+k}=\frac{\sqrt{3}\left(n^2+2 n k-n-k^2+k\right)}{2 \sqrt{n\left(2 n^2+1\right)}}(k=2,3, \cdots, n)
$$
时, 上述不等式等号成立.
所以, $\left(a_{n+1}+a_{n+2}+\cdots+a_{2 n}\right)-\left(a_1+a_2+\cdots+a_n\right)$ 的最大值为 $\sqrt{\frac{n\left(2 n^2+1\right)}{3}}$.
%%PROBLEM_END%%



%%PROBLEM_BEGIN%%
%%<PROBLEM>%%
问题3. 设 $a_1, a_2, \cdots, a_n$ 是 $1,2, \cdots, n$ 的一个排列, 求 $S_n=\left|a_1-1\right|+\mid a_2- 2|+\cdots+| a_n-n \mid$ 的最大值.
%%<SOLUTION>%%
解:的关键是去掉绝对值符号.
注意到 $\left|a_i-i\right|$ 等于 $a_i-i$ 或 $i-a_i$,
因此, 去掉绝对值符号后, 和式中负号的个数不变.
即不论 $a_1, a_2, \cdots, a_n$ 如何排列, 去掉绝对值符号后, 和式中均有 $n$ 个负号.
这样, 当 $n$ 为偶数, $S_n \leqslant n+n+(n-1)+(n-1)+\cdots+\left(\frac{n}{2}+1\right)+\left(\frac{n}{2}+1\right)-\frac{n}{2}-\frac{n}{2}-\cdots-1- 1=\frac{n^2}{2}$. 当 $n$ 为奇数, 可得类似结果.
总之, $S_n \leqslant\left[\frac{n^2}{2}\right]$. 又当 $\left(a_1, a_2, \cdots\right.$, $\left.a_n\right)=(n, n-1, n-2, \cdots, 2,1)$ 时, $S_n=\left[\frac{n^2}{2}\right]$. 所以 $S_n$ 的最大值为 $\left[\frac{n^2}{2}\right]$.
%%PROBLEM_END%%



%%PROBLEM_BEGIN%%
%%<PROBLEM>%%
问题4. 设 $x_k(k=1,2, \cdots, 1991)$ 满足 $\left|x_1-x_2\right|+\left|x_2-x_3\right|+\cdots+\mid x_{1990}- x_{1991} \mid=1991$. 令 $y_k=\frac{x_1+x_2+\cdots+x_k}{k}(k=1,2, \cdots, 1991)$. 求 $F= \left|y_1-y_2\right|+\left|y_2-y_3\right|+\cdots+\left|y_{1990}-y_{1991}\right|$ 的最大值.
%%<SOLUTION>%%
对每个 $k(1 \leqslant k \leqslant 1990)$, 有 $\left|y_k-y_{k+1}\right|=\mid \frac{x_1+x_2+\cdots+x_k}{k}- \frac{x_1+x_2+\cdots+x_{k+1}}{k+1}|=| \frac{x_1+x_2+\cdots+x_k-k x_{k+1}}{k(k+1)}\left|=\frac{1}{k(k+1)}\right|\left(x_1-\right. \left.x_2\right)+2\left(x_2-x_3\right)+3\left(x_3-x_4\right)+\cdots+k\left(x_k-x_{k+1}\right)\left|\leqslant \frac{1}{k(k+1)}\right|\left(x_1-x_2\right) \mid+ \left|2\left(x_2-x_3\right)\right|+\left|3\left(x_3-x_4\right)\right|+\cdots+\left|k\left(x_k-x_{k+1}\right)\right|=\frac{1}{k(k+1)} \sum_{i=1}^k i\left|x_i-x_{i+1}\right|$, 所以 $\sum_{k=1}^{1990}\left|y_k-y_{k+1}\right| \leqslant \sum_{k=1}^{1990}\left[\frac{1}{k(k+1)} \sum_{i=1}^k i\left|x_i-x_{i+1}\right|\right]= \sum_{k=1}^{1990} \sum_{i \leqslant k}\left[\frac{1}{k(k+1)} \cdot i\left|x_i-x_{i+1}\right|\right]=\sum_{i=1}^{1990} \sum_{k \geqslant i}\left[\frac{1}{k(k+1)} \cdot i\left|x_i-x_{i+1}\right|\right]= \sum_{i=1}^{1990}\left(i\left|x_i-x_{i+1}\right|\right) \sum_{k=i}^{1990} \frac{1}{k(k+1)}=\sum_{i=1}^{1990}\left(i\left|x_i-x_{i+1}\right|\right) \sum_{k=i}^{1990}\left(\frac{1}{k}-\frac{1}{k+1}\right)= \sum_{i=1}^{1990}\left(i\left|x_i-x_{i+1}\right|\right)\left(\frac{1}{i}-\frac{1}{1991}\right)=\sum_{i=1}^{1990}\left(\left|x_i-x_{i+1}\right|\right)\left(1-\frac{i}{1991}\right) \leqslant \sum_{i=1}^{1990}\left(\left|x_i-x_{i+1}\right|\right)\left(1-\frac{1}{1991}\right)=1991\left(1-\frac{1}{1991}\right)=1990$. 其中等号在 $x_1= 1991, x_2=x_3=\cdots=x_{1991}=0$ 时成立.
故 $F_{\text {max }}=1990$.
%%PROBLEM_END%%



%%PROBLEM_BEGIN%%
%%<PROBLEM>%%
问题5. 设 $x_1, x_2, \cdots, x_{1990}$ 是 $1,2, \cdots, 1990$ 的一个排列, 求 $F=|\cdots| \mid x_1- x_2\left|-x_3\right|-\cdots\left|-x_{1990}\right|$ 的最大值.
%%<SOLUTION>%%
$F_2=|1-2|=1 \leqslant 2, F_3=|| 1-2|-3|=2 \leqslant 3, F_4=|| \mid 1- 2|-3|-4 \mid=4 \leqslant 4$. 一般地, 猜想 $F_n \leqslant n$. 下用数学归纳法证明.
首先注意到 $x 、 y>0$ 时, $|x-y| \leqslant \max \{x, y\}$. 所以 $\left|x_1-x_2\right| \leqslant \max \left\{x_1, x_2\right\}$, ||$x_1-x_2\left|-x_3\right| \leqslant \max \left\{\max \left\{x_1, x_2\right\}, x_3\right\}=\max \left\{x_1, x_2, x_3\right\}$. 设 $|\cdots|\left|x_1-x_2\right|-x_3|-\cdots|-x_k \mid \leqslant \max \left\{x_1, x_2, x_3, \cdots, x_k\right\}$, 则 $|\cdots| \mid x_1- x_2\left|-x_3\right|-\cdots\left|-x_{k+1}\right| \leqslant \max \left\{\max \left\{x_1, x_2, x_3, \cdots, x_k\right\}, x_{k+1}\right\}=\max \left\{x_1, x_2, x_3, \cdots, x_k, x_{k+1}\right\} \label{eq1}$, 所以 $F=||\left|x_1-x_2\right|-x_3|-\cdots|-x_{1990} \mid \leqslant \max \left\{x_1, x_2, x_3, \cdots, x_{1990}\right\}=1990$. 由初值可知, 上述不等式 \ref{eq1} 的等号不一定成立.
实际上, $k=1990$ 时不成立等号.
因为去掉绝对值符号和改变项的正负符号, 代数式 $F$ 的值的奇偶性不变, 所以 $F \equiv x_1+x_2+x_3+\cdots+x_{1990}=1+ 2+\cdots+1990 \equiv 1(\bmod 2) .1990 \equiv 0(\bmod 2)$, 所以 $F \leqslant 1989$. 下面构造 $x_1$, $x_2, x_3, \cdots, x_{1990}$, 使 $F=1989$. 我们的策略是使 "和"中尽可能产生 0 . 注意到要使差最大,必须是最大的减最小的,即 $1990-1=1989$. 于是希望其他 1988 个数相互抵消.
考察 4 个连续自然数 $n+1 、 n+2 、 n+3 、 n+4$, 我们有 || $\mid n+ 3-(n+1)|-(n+4)|-(n+2) \mid=0$. 将 $x_1, x_2, x_3, \cdots, x_{1988}$ 按相连 4 个数一组, 分为 497 组, 第 $k$ 组为: $\left(x_{4 k+1}, x_{4 k+2}, x_{4 k+3}, x_{4 k+4}\right)(k=0,1$, $2, \cdots, 496)$. 令 $\left(x_{4 k+1}, x_{4 k+2}, x_{4 k+3}, x_{4 k+4}\right)=(4 k+2,4 k+4,4 k+5,4 k+ 3)$, 则 ||$\left|x_{4 k+1}-x_{4 k+2}\right|-x_{4 k+3}\left|-x_{4 k+4}\right|=|||(4 k+2)-(4 k+4)|-(4 k+ 5)|-(4 k+3)|=0$. 于是, 再令 $x_{1989}=1990, x_{1990}=1$, 则 $F=\mid 1990- 1 \mid=1989$.
%%PROBLEM_END%%



%%PROBLEM_BEGIN%%
%%<PROBLEM>%%
问题6. 设 $0<p \leqslant a_i \leqslant q, b_i$ 是 $a_i$ 的一个排列 $(1 \leqslant i \leqslant n)$, 求 $F=\sum_{i=1}^n \frac{a_i}{b_i}$ 的最值.
%%<SOLUTION>%%
不妨设 $a_1 \leqslant a_2 \leqslant \cdots \leqslant a_n$, 由于函数在闭域中连续, 所以 $F$ 存在最大、 最小值.
由排序不等式, 有 $F=\sum_{i=1}^n \frac{a_i}{b_i} \geqslant \sum_{i=1}^n \frac{a_i}{a_i}=\sum_{i=1}^n 1=n$, 等号在 $a_i= b_i(1 \leqslant i \leqslant n)$ 时成立, 所以 $F_{\min }=n$. 又由排序不等式, 有 $F=\sum_{i=1}^n \frac{a_i}{b_i} \leqslant \sum_{i=1}^n \frac{a_i}{a_{n+1-i}}=F^{\prime}$, 下面求 $F^{\prime}=\sum_{i=1}^n \frac{a_i}{a_{n+1-i}}$ 的最大值.
由对称性, $2 F^{\prime}= \sum_{i=1}^n\left(\frac{a_i}{a_{n+1-i}}+\frac{a_{n+1-i}}{a_i}\right)$. 因为 $p \leqslant a_i \leqslant q$, 所以 $\frac{p}{q} \leqslant \frac{a_i}{a_{n+1-i}} \leqslant \frac{q}{p}$. 而 $f(x)=x+\frac{1}{x}$ 在 $(0,1]$ 上单调递减, 在 $[1, \infty)$ 上单调递增, 所以 $2 F^{\prime}$ 只能在 $\frac{a_i}{a_{n+1-i}} \in \left\{\frac{p}{q}, \frac{q}{p}\right\}$ 时达到最大.
当 $2 \mid n$ 时, $\frac{a_i}{a_{n+1-i}}$ 都可取到 $\frac{p}{q}$ 或 $\frac{q}{p}$, 此时 $F_{\text {max }}=\frac{n}{2}$. $\left(\frac{p}{q}+\frac{q}{p}\right)=n+\left[\frac{n}{2}\right]\left(\sqrt{\frac{p}{q}}-\sqrt{\frac{q}{p}}\right)^2$. 当 2 不整除 $n$ 时, 总有 $\frac{a\left[\frac{n}{2}\right]+1}{a_{n+1-\left(\left[\frac{n}{2}\right]+1\right)}}= \frac{a\left[\frac{n}{2}\right]+1}{a_{n-\left[\frac{n}{2}\right]}}=1$, 其余 $\frac{a_i}{a_{n+1-i}}$ 都可取到 $\frac{p}{q}$ 或 $\frac{q}{p}$. 此时, $F_{\text {max }}=\frac{n-1}{2} \cdot\left(\frac{p}{q}+\frac{q}{p}\right)+ 1=n+\left[\frac{n}{2}\right]\left(\sqrt{\frac{p}{q}}-\sqrt{\frac{q}{p}}\right)^2$. 故 $F$ 的最小值为 $n$, 最大值为 $n+\left[\frac{n}{2}\right]$. $\left(\sqrt{\frac{p}{q}}-\sqrt{\frac{q}{p}}\right)^2$
%%PROBLEM_END%%


