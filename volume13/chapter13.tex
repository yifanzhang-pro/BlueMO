
%%TEXT_BEGIN%%
所谓考察特例, 是指考察问题包含的一些简单的特殊情形, 从中发现解题途径.
它常包括如下 4 种情形:
情形 1 考察"最坏"的特例——种最特殊的情况.
情形 2 由充分条件、必要条件寻找特例.
对此,一个找使性质 $P$ 成立的充分条件的方法是: 假设所求对象不满足要求 $P$, 由此导出若干性质, 然后设法破坏其中一个性质即可.
情形 3 先考虑原问题在特殊情况下如何解决, 然后将一般情况变换到特殊情况处理.
情形 4 由特殊情况发现一般规律, 猜想问题的结论, 最后用数学归纳法加以证明.
%%TEXT_END%%



%%PROBLEM_BEGIN%%
%%<PROBLEM>%%
例1. 某市有 $n$ 所中学, 第 $i$ 所中学派出 $c_i$ 名学生 $\left(1 \leqslant c_i \leqslant 39\right)$ 到体育馆看球赛, 其中 $\sum_{i=1}^n c_i=1990$. 看台上每排有 199 个座位, 要求同一学校的学生坐在同一排.
问: 最少要安排多少个排, 才能使所有学生一定能够坐下? 
%%<SOLUTION>%%
分析:解先考虑最坏的情况是什么.
所谓情况最坏, 是指每个横排空下来的位置最多.
显然, 如果各校的人数有多有少, 是比较好安排的, 因为剩下的空位可以让人数少的学校的学生坐.
于是, 较坏的情况是每所学校派出的人数较多而又比较"整齐". 于是, 可先设想所有学校派出的人数相等.
假定每所学校都派出 $r$ 人, 我们考察 $r$ 为何值时, 会使横排空余的位置最多.
列表估计如下:
\begin{tabular}{|c|c|c|c|c|c|c|c|c|c|c|c|c|}
\hline 各学校所派人数 $r$ & 39 & 38 & 37 & 36 & 35 & 34 & 33 & 32 & 31 & 30 & 29 & $r \leqslant 28$ \\
\hline 每排可安排学校个数 & 5 & 5 & 5 & 5 & 5 & 5 & 6 & 6 & 6 & 6 & 6 & $\cdots$ \\
\hline 每排空余位子数 $t$ & 4 & 9 & 14 & 19 & 24 & 29 & 1 & 7 & 13 & 19 & 25 & $t \leqslant 28$ \\
\hline
\end{tabular}
由上表可知, 当每所学校派出的人数都为 34 时, 每个横排的空位最多, 为 29.
注意到 $1990=34 \times 58+18$. 于是, 先考察有 58 所学校各派 34 人, 另一所学校派 18 人的情形, 看需要多少个横排.
此时, 每个横排最多坐 6 所学校, 而且坐 6 所学校的排只能有一排, 即坐派了 18 人的学校的那一排.
于是, 所需要的排数至少是 $1+\left[\frac{58-6}{5}\right]+1=12$.
最后, 我们证明,对任何情形, 12 排是足够的.
最自然的一个排法如下:
先排第一排,使第一排的空位数 $x_1$ 最小.
再排第二排: 又使第二排的空位数 $x_2$ 最小.
显然 $x_1 \leqslant x_2$. 否则, $x_1>x_2$, 将第 2 排的学生坐到第 1 排, 可使第 1 排的空位数减少, 这与 $x_1$ 的最小性矛盾.
如此下去, 直至排第 11 排,使第 11 排的空位数 $x_{11}$ 最小.
我们证明, 按这样的排法, 必定可将剩下的所有学生都排在第 12 排.
实际上, 设排完 12 排以后还有 $x$ 人没有排下, 则 $x>x_{12}$. 否则, 将此 $x$ 人坐在第 12 排即可.
所以, $x=1990-\sum_{i=1}^{12}\left(199-x_i\right) \geqslant x_{12}$, 由此得 $x_1+x_2+\cdots+ x_{11}>398$, 所以, $398<x_1+x_2+\cdots+x_{11} \leqslant 11 x_{11}$, 所以, $x_{11} \geqslant 37$. 这表明:第 12 排中各校的学生人数都至少是 38 人 (任何一个学校的学生都不能坐在前一排的 $x_{11}$ 个空位上). 又每所学校至多派 39 人, 从而坐在第 12 排的学校所派的学生都是 38 或 39 人.
注意到 $5 \times 39<199<6 \times 38$, 于是第 12 排中恰好坐了 5 个学校的学生.
设其中有 $k$ 个学校来 38 人, $5-k$ 个学校来 39 人 $(0 \leqslant k \leqslant$ 5), 则第 12 排的空位数为
$$
199-k \times 38-(5-k) \times 39=199+k-5 \times 39=4+k \leqslant 9 .
$$
但 $x_{11} \geqslant 37$,将第 12 排的学生坐到第 11 排,可使第 11 排的空位数减少, 与 $x_{11}$ 是最少的矛盾.
%%<REMARK>%%
注:: 我们还有一种更直观的排法: 先安排前 10 排,一个个学校的学生就座, 直至某行再坐不下任何一个学校的学生为止.
再坐第二排, 如此下去, 先坐好前 10 排.
我们来估计还剩下多少学校的学生没有安排座位.
不难知道, 没有就座的学校至多有 9 个.
否则, 至少有 10 所学校的学生没有安排座位.
由于每所学校的学生都不能安排在前 10 排中的任何一排就座, 这意味着剩下的任何一所学校的学生与前 10 排中任何一排坐的学生之和多于 199 个.
这样, 总人数就多于 1990 , 矛盾.
所以, 安排 10 个横排以后, 至多剩下 9 所学校, 而每个横排至少坐 5 个学校的学生 $(5 \times 39<199)$, 用两个横排可以将剩下的学生全部坐下.
%%PROBLEM_END%%



%%PROBLEM_BEGIN%%
%%<PROBLEM>%%
例2. 在 $n \times n$ 棋盘上放有 $r$ 只棋, 每个格最多一只棋.
若 $r$ 只棋具有如下的性质 $p$ : 每行每列至少有一只棋.
但去掉其中任何一只棋, 则它们便不再具有上述的性质 $p$. 求 $r$ 的最大值 $r_n$.
%%<SOLUTION>%%
分析:解先考虑简单情形.
为了叙述问题的方便, 对于具有性质 $p$ 的棋盘, 如果去掉其中一只棋以后, 棋盘仍然具有性质 $p$, 则称那只棋为可去棋.
当 $n=2$ 时, $r_2<3$, 否则, 棋盘上放 3 只棋, 必有一个角上的棋为可去棋.
矛盾.
当 $n=3$ 时, $r_3<5$. 否则, 棋盘上放 5 只棋, 必有一行有两只棋.
不妨设第一行的前两格都有一只棋 $a_{11} 、 a_{12}$. 此时, 第一和第二列都不能再有棋.
比如第一列还有一只棋 $A$, 则棋 $a_{11}$ 可去.
若第二列有一只棋 $B$, 则棋 $a_{12}$ 可去.
这样剩下的 3 只都在第三列,此时,位于第一行第三列的棋 $a_{13}$ 是可去棋.
一般地,对自然数 $n$, 我们猜想有 $r_n<2 n-1$.
实际上, 由上面的证明过程可知, 若某个行有两只棋 $a 、 b$, 则棋 $a 、 b$ 所在的列无其他的棋.
比如, 若 $a$ 所在的列还有一只棋, 则 $a$ 是可去棋, 矛盾.
利用此性质,可适当去掉一行一列,将 $n$ 的问题化归为 $n-1$ 的问题.
下面证明 $r_n<2 n-1$. 即证明如下的结论:
若棋盘中至少放有 $2 n-1$ 只棋, 则棋盘上必有可去棋. (*)
证法 1: 对 $n$ 用归纳法.
设结论 (*) 对小于 $n$ 的自然数成立.
考察 $n \times n$ 棋盘, 为了利用归纳假设,应去掉一行和一列, 且使剩下的棋盘中至少 $2 n-3$ 只棋.
这就要求去掉的行和列中一共只包含有 2 只棋.
因此, 我们要找到恰有一只棋的行和列.
但棋盘中的棋子数不少于 $2 n-1$, 并不意味着棋盘中的棋子数为 $2 n-1$. 因此, 不能由抽庶原理找到恰有一只棋的行和列.
注意到前面所证的结论: 若某个行有两只棋, 则这两只棋所在的列没有棋.
由此便可找到合乎要求的列.
由于棋盘中至少有 $2 n-1$ 只棋, 由抽屉原理, 至少有一行有两只棋.
不妨设第一行的前两格各有一只棋 $a 、 b$ (如图(<FilePath:./figures/fig-c13i1.png>) ). 此时,第一、第二列不能再有棋.
比如第一列还有一只棋则棋 $a$ 可去.
第二列有一只棋, 则棋 $b$ 可去.
这样剩下的 $2 n-3$ 只都在后 $n-2$ 列, 由抽庶原理,必有某个列有两只棋.
不妨设第 3 列的第 $i$ 格和第 $j$ 格各有一只棋 $c 、 d$, 其中 $i 、 j$ 中至少有一个不为 1 . 不妨设 $i \neq 1$, 则 $c$ 所在的行不能再有棋, 否则棋 $c$ 可去.
这样, $a$ 所在的列只有一只棋, $c$ 所在的行只有一只棋.
去掉这行和这列,对剩下的棋盘使用归纳假设,命题 (*) 获证.
最后, 如图(<FilePath:./figures/fig-c13i2.png>) 可知, $r=2 n-2$ 是可能的.
故 $r_n=2 n-2$.
证法 2: 为了找到可去棋, 先考虑棋在什么条件下可去.
如果 $A$ 是可去棋,则 $A$ 所在的行至少 2 只棋, $A$ 所在的列也至少 2 只棋.
于是, 可取定至少有 2 只棋的行, 再从中找有 2 只棋的列.
设第 $i$ 行的棋子数为 $a_i(i=1,2, \cdots, n)$, 不妨设 $a_1 \leqslant a_2 \leqslant \cdots \leqslant a_n$, 则 $a_1+a_2+\cdots+a_n=2 n-1$. 因为每行至少一只棋, 所以 $a_1=1$.
假定 $a_1=a_2=\cdots=a_i=1(1 \leqslant i \leqslant n-1), 2 \leqslant a_{i+1} \leqslant a_{i+2} \leqslant \cdots \leqslant a_n$, 则
$$
\begin{aligned}
a_{i+1}+a_{i+2}+\cdots+a_n & =2 n-1-\left(a_1+a_2+\cdots+a_i\right) \\
& =2 n-1-i \\
& =n+(n-1-i) \geqslant n .
\end{aligned}
$$
前 $i$ 行的 $i$ 只棋最多占住 $i$ 个列(如图(<FilePath:./figures/fig-c13i3.png>) 的阴影方格所示), 不妨设这 $i$ 只棋都在前 $k(k \leqslant i)$ 列中.
如果后 $n-i$ 行中有一只棋 $A$ 在前 $k$ 列中,由于 $A$ 所在的行至少有 2 只棋, 所以 $A$ 可去.
如果后 $n-i$ 行的棋都在后 $n-k$ 列中, 但 $n-k<n$, 而 $a_{i+1}+a_{i+2}+\cdots+ a_n \geqslant n$, 所以必有一列有 2 只棋 $B 、 C$, 但 $B$ 所在的行也有 2 只棋, 所以 $B$ 可去.
%%PROBLEM_END%%



%%PROBLEM_BEGIN%%
%%<PROBLEM>%%
例3. 在 $19 \times 89$ 棋盘中最多可以放多少只棋,使任何 $2 \times 2$ 的矩形内不多于 2 只棋.
%%<SOLUTION>%%
分析:解注意到 19 和 89 都是奇数, 从而可以考虑一般的 $(2 m-1) \times (2 n-1)$ 棋盘(其中 $m 、 n$ 中至少一个大于 1 ). 可以考虑对 $m$ 归纳.
设棋盘中可以放的棋数的最大值为 $r_m$.
(1) 当 $m=1$ 时,每个格都可以放棋,所以, $r_1=2 n-1$.
(2) 当 $m=2$ 时,由图 (<FilePath:./figures/fig-c13i4.png>) 所示的放法可以猜想: $r_2=2(2 n-1)$, 即 $3 \times(2 n-1)$ 棋盘最多可以放 $4 n-2$ 只棋.
由于 $3 \times(2 n-1)$ 棋盘要去掉两行才能化为 $1 \times(2 n-1)$ 棋盘.
那么去掉的两行中至多有多少只棋呢? 为此, 我们要研究一下 $2 \times(2 n-1)$ 棋盘, 而这是我们已经跳过了的情形.
这种情形对于研究 $3 \times(2 n-1)$ 棋盘也许有帮助.
因此,我们回头看看 $2 \times(2 n-1)$ 棋盘.
如图(<FilePath:./figures/fig-c13i5.png>), 在 $2 \times(2 n-1)$ 棋盘中, 有 $r \leqslant 2 n$, 且等号只能以如图(<FilePath:./figures/fig-c13i5.png>) 的方式唯一实现.
实际上,第一列至多有两只棋, 而后 $2 n-2$ 列可以划分为 $n-1$ 个 $2 \times 2$ 棋盘, 每个 $2 \times 2$ 棋盘至多可以放 2 只棋.
所以, $r \leqslant 2+2(n-1)=2 n$. 若 $r=2 n$, 则第一列必有两只棋, 且每个 $2 \times 2$ 棋盘中都恰有 2 只棋.
于是, 第一列有两只棋, 则第二列中无棋, 于是第三列中有两只棋.
如此下去, 所有奇数列中都有两只棋, 而所有偶数列中都没有棋.
即等号以唯一的方式出现.
现在考虑 $3 \times(2 n-1)$ 棋盘.
我们要证明 $r \leqslant 2(2 n-1)$. 很自然地, 应将 $3 \times (2 n-1)$ 棋盘化为一个 $2 \times(2 n-1)$ 棋盘(简称为 $A$ 盘) 和一个 $1 \times(2 n-1)$ 棋盘 (简称为 $B$ 盘) 处理.
由前面的讨论可知, $r_A \leqslant 2 n, r_B \leqslant 2 n-1$. 所以
$$
r=r_A+r_B \leqslant 4 n-1 . \label{eq1}
$$
若 式\ref{eq1} 成立等号, 则 $r_A=2 n, r_B=2 n-1$. 此时, 棋盘中棋的放置如图(<FilePath:./figures/fig-c13i6.png>) 所示,但其中有一个 $2 \times 2$ 正方形放了 3 只棋.
矛盾.
所以, $r \leqslant 4 n-2$.
以上证明存在一个漏洞: 要使棋盘中有 $2 \times 2$ 正方形, 必须 $n>1$, 即棋盘中至少要有两列.
因此,要优化假设: $m \leqslant n$. 此时, 必有 $n \geqslant m>1$.
一般地, 对 $(2 m-1) \times(2 n-1)$ 棋盘, 若 $m \leqslant n$, 我们证明: $r \leqslant m(2 n-1)$. 
对 $m$ 归纳.
当 $m=1$ 时, 结论显然成立, 设结论对于小于 $m$ 的自然数成立, 考察 $(2 m-1) \times(2 n-1)$ 棋盘, 我们将之划分为一个 $A$ 盘: $2 \times(2 n-1)$ 棋盘和一个 $B$ 盘: $(2 m-3) \times(2 n-1)$ 棋盘.
由前面的讨论和归纳假设可知 $r_A \leqslant 2 n, r_B \leqslant(m-1)(2 n-1)$, 所以
$$
r=r_A+r_B \leqslant 2 n+(m-1)(2 n-1)=m(2 n-1)+1 . \label{eq2}
$$
若 \ref{eq2} 式成立等号,则 $r_A=2 n, r_B=(m-1)(2 n-1)$. 由 $r_A=2 n$ 知,整个 $(2 m-1) \times(2 n-1)$ 棋盘的第一行中恰有 $n$ 只棋, 于是, 将后 $2 m-2$ 行划分为
$m-1$ 个 $2 \times(2 n-1)$ 棋盘, 每个 $2 \times(2 n-1)$ 棋盘中不多于 $2 n$ 只棋, 于是, 棋盘中的棋子的个数 $r \leqslant n+2 n(m-1)=2 m n-n \leqslant 2 m n-m=m(2 n-1)$, 与 $r=m \cdot(2 n-1)+1$ 矛盾.
所以, \ref{eq2} 式不成立等号.
即 $r \leqslant m(2 n-1)$. 命题获证.
最后, 将棋盘的奇数行的每个格都放一只棋, 有 $r=m(2 n-1)$, 所以 $r_m=m(2 n-1)$. 特别地, 令 $m=10 、 n=45$, 有 $19 \times 89$ 棋盘中至多可放 890 只棋.
%%PROBLEM_END%%



%%PROBLEM_BEGIN%%
%%<PROBLEM>%%
例4. 一个 $9 \times 9$ 的棋盘的方格被染成黑白两种颜色, 使得与每个白格相邻的格中黑格的数目多于白格的数目, 与每个黑格相邻的格中白格的数目多于黑格的数目 (至少有一条公共边的两格称为相邻). 求所有这样的染色方式中,黑、白格数目之差的最大值.
%%<SOLUTION>%%
分析:解要使染色满足条件, 则每个方格至多有一个邻格的颜色与其相同, 因而不能出现如下一些特殊情形: (1) $3-\mathrm{L}$ 型的 3 个方格同色.
(2) $1 \times 3$ 的矩形的 3 个方格同色.
如果棋盘中任何 2 个相邻方格的颜色不同, 则黑、白格数目之差不大于 1 .
如果棋盘中存在 2 个相邻方格 $A 、 B$ 的颜色相同, 不妨设 $A 、 B$ 在同一行 (如图(<FilePath:./figures/fig-c13i7.png>) 所示). 考察与这行相邻的行, 由于棋盘中没有同色的 3-L 型, 所以此行中与 $A 、 B$ 相邻的两个方格与 $A 、 B$ 异色.
如此下去可知, $A 、 B$ 所在的两列中, 同行的两个方格都同色, 同列的方格颜色黑白相间 (相邻两格异色).下面证明,棋盘的任何一列中都没有两个相邻的格同色.
否则, 设 $P 、 Q$ 是某列中相邻的方格, $P 、 Q$ 同色, 同上可证, $P 、 Q$ 所在的两行中, 同列的两个方格都同色、同行的方格颜色黑白相间.
此时, 考察 $A 、 B$ 两列与 $P 、 Q$ 两行交叉的 4 个格, 由于同行同色且同列同色,所以 4 个格同色,矛盾.
所以整个棋盘的每一列的方格的颜色都是黑白相间.
去掉第一行, 则剩下的棋盘的每一列中黑、白格数目相等, 从而剩下的棋盘中所有方格黑、白格数目相等.
而在第一行中, 由于没有同色的 $1 \times 3$ 矩形, 所以每 3 个格中黑白格数目之差不大于 1 , 于是第一行中黑白格数目之差不大于 3 .
所以,对任何合乎条件的染色,棋盘中黑、白格数目之差不大于 3 . 又如图(<FilePath:./figures/fig-c13i8.png>) 的染色符合要求, 此时黑白格个数之差为 3 .
综上所述,所求的最大值为 3 .
%%PROBLEM_END%%



%%PROBLEM_BEGIN%%
%%<PROBLEM>%%
例5. 设正整数 $n \geqslant 3, a_1, a_2, \cdots, a_n$ 是任意 $n$ 个互异的实数, 其和为正数.
如果它们的一个排列 $b_1, b_2, \cdots, b_n$ 满足:对任意的 $k=1,2, \cdots, n$, 均有 $b_1+b_2+\cdots+b_k>0$, 则称这个排列是好的.
求好的排列个数的最小值.
%%<SOLUTION>%%
分析:解考察最坏情形: 对 $k=1,2, \cdots, n, b_1+b_2+\cdots+b_k>0$ 都很难满足, 这只需 $a_1, a_2, \cdots, a_n$ 中的负数尽可能多.
取 $a_2, \cdots, a_n$ 均为负数, 而 $a_1=-a_2-\cdots-a_n+1$. 此时任何一个好的排列 $\left(b_1, b_2, \cdots, b_n\right)$, 均有 $b_1= a_1$, 而 $b_2, \cdots, b_n$ 可以是 $a_2, \cdots, a_n$ 任意排列, 故此时有 $(n-1)$ ! 个好的排列.
下面证明至少有.
$(n-1)$ ! 个好的排列.
注意到 $(n-1)$ ! 是将 $a_1, a_2, \cdots, a_n$ 排在圆周上的不同圆排列的个数, 我们先证明每一个圆排列对应一个好排列.
为方便, 称好排列的首项为好数.
我们只需证明每个圆排列中必存在一个好数.
方法 1: 对 $n$ 归纳.
当 $n=1$ 时,结论显然成立.
假设对一切 $n<k$ 结论成立, 考虑 $n=k$ 的情况.
若所有数都为正, 则结论显然成立, 是因每个数都是好数.
若至少存在一个非正数, 但 $a_1+a_2+\cdots+a_k>0$, 所以至少有一个正数.
将每一个正数和按逆时钟顺序在它之后的下一个正数之间的所有数编为一组, 每组至少有一个数, 且至少有一组有至少两个数 (由于不是所有数都为正), 故至多有 $k-1$ 组.
对每组数求和, 得到少于 $k$ 个和.
将这些和按它们所在组的顺序写在圆周上, 由于这些和的总和为正, 由归纳假设知, 这些和中存在一个和为好数.
考虑这个和所在的组中的那个正数, 则这个数是整个圆排列中的好数.
由归纳原理,结论成立.
方法 2: 利用极端性原理.
对任何一个圆排列 $\left(b_1, b_2, \cdots, b_n\right)$, 考察所有以 $b_i$ 为首项的部分和: $b_i+b_{i+1}+\cdots+b_{i+t}$, 其中大于 $n$ 的下标取模 $n$ 的余数.
对所有 $i=1,2, \cdots, n$ 和所有 $t=0,1,2, \cdots, n-1$, 必存在一个最小的部分和 $b_i+b_{i+1}+\cdots+b_{i+t}$. 因为至少存在一个非正数, 所以 $b_i+b_{i+1}+\cdots+ b_{i+t} \leqslant 0$. 在所有这样的最小和中又设项数 $t+1$ 最大的一个为 $b_i+b_{i+1}+\cdots+ b_{i+t}$, 我们证明 $b_{i+t+1}$ 是好数.
实际上, 若存在正整数 $k$, 使 $b_{i+t+1}+b_{i+t+2}+\cdots+b_{i+t+k} \leqslant 0$, 则 $\left(b_i+\right. \left.b_{i+1}+\cdots+b_{i+t}\right)+\left(b_{i+t+1}+b_{i+t+2}+\cdots+b_{i+t+k}\right) \leqslant b_i+b_{i+1}+\cdots+b_{i+\iota}$, 这与和 $b_i+b_{i+1}+\cdots+b_{i+t}$ 最小且项数最多矛盾.
由于共有 $(n-1)$ ! 个圆排列, 而每个圆排列至少对应一个好排列, 且不同的圆排列对应的好排列是不同的,故至少有 $(n-1)$ ! 个好的排列.
综上所述, 所求的最小值为 $(n-1) !$.
%%PROBLEM_END%%



%%PROBLEM_BEGIN%%
%%<PROBLEM>%%
例6. 岛上住着 $n$ 个本地人, 他们中每两个人要么是朋友, 要么是敌人.
一天, 首领要求每位居民 (包括首领自己) 按以下原则自己做一条石头项链: 每两个朋友间, 他们的项链上至少有一块石头相同; 每两个敌人间, 他们的项链上没有相同的石头 (一条项链上可以无石头). 求证: 要完成首领的命令, 需要 $\left[\frac{n^2}{4}\right]$ 种不同的石头; 而石头种数少于 $\left[\frac{n^2}{4}\right]$ 时,此命令可能无法实现.
%%<SOLUTION>%%
分析:解当 $n=1$ 时,结论显然成立.
设 $n>1$, 记需要的不同石头种数的最小值为 $S_n$. 当 $n=2$ 时, 设两个人为 $A 、 B$, 如果 $A 、 B$ 是敌人, 则 $S_2=0$. 如果 $A 、 B$ 是朋友, 则 $S_2=1$. 结论成立.
当 $n=3$ 时, 设三个人为 $A 、 B 、 C$, 如果 $A 、 B 、 C$ 两两是敌人, 则 $S_3=0$. 如果 $A 、 B 、 C$ 两两是朋友, 则 $S_3=1$. 如果 $A 、 B 、 C$ 中有一个二人组是朋友, 另两个二人组是敌人, 则两个为朋友的二人组需要 1 种石头, 此时 $S_3=1$. 如果 $A 、 B 、 C$ 中有一个二人组是敌人, 另两个二人组是朋友, 则两个为朋友的二人组需要 2 种不同的石头.
否则, 3 人拥有同一种石头,但其中有两个人是敌人,矛盾, 此时 $S_3=2$.
由前面的一些特例, 我们发现一个有用的规律: 如果 3 个人中有一个二人组是敌人, 另两个二人组是朋友, 则两个为朋友的二人组需要 2 种不同的石头.
再考虑 $n=4$ 的情形, 设四个人为 $A 、 B 、 C 、 D$, 如果为朋友的二人组不多于 4 , 则 $S_4 \leqslant 4$. 如果为朋友的二人组为 5 , 另一个二人组为敌人, 不妨设 $A 、 B$ 为敌人,则 $A C D$ 是朋友三角形,设他们拥有同种的石头 $1 . B C D$ 是朋友三角形, 设他们拥有同种的石头 2. 此时 $A 、 B 、 C 、 D$ 的项链分别为 $\{1\},\{2\},\{1,2\}$, $\{1,2\}$, 合乎条件.
此时 $S_4=2$. 如果 $A 、 B 、 C 、 D$ 两两都是朋友, 则 $S_4 \stackrel{?}{=} 1$.
现在考察何时有 $S_4=4$. 此时, 显然有 4 个为朋友的二人组, 另两个二人组为敌人.
如果为敌人的两个二人组有公共的人, 不妨设 $A 、 B$ 为敌人且 $A 、 C$ 为敌人.
因为 $B C D$ 是朋友三角形, 设他们拥有共同的石头 1 , 再注意到 $A 、 D$ 是朋友, 设他们拥有共同的石头 2. 此时 $A 、 B 、 C 、 D$ 的项链分别为 $\{2\},\{1\}$, $\{1\},\{1,2\}$, 合乎条件, 此时 $S_4=2$. 如果为敌人的两个二人组没有公共的人, 不妨设 $A 、 B$ 为敌人且 $C 、 D$ 为敌人.
此时, $A 、 B 、 C 、 D$ 被分为 2 组, 每组 2 人,
同一组的 2 个人是敌人, 而任何不同组的 2 个人都是朋友.
此时, 每个为朋友的二人组对应一种石头, 我们证明: 4 个为朋友的二人组对应的石头互不相同.
实际上, 如果某 2 个为朋友的二人组对应相同的石头, 而这 2 个为朋友的二人组至少包含 3 个不同的人, 他们拥有公共的石头.
但将 3 人归人前述的两组,必有 2 人在同一组,他们应该是敌人,矛盾.
于是 $S_4=4$.
有上述一些特例不难发现一般情况下的构造方法.
当 $n$ 为奇数时, 设 $n=2 k+1$, 将 $n$ 个人分成两组,一组 $k$ 人, 另一组 $k+1$ 人, 令同一组的任何 2 个人都是敌人, 而任何不同组的任何 2 个人都是朋友.
此时, 共有 $k(k+1)$ 个为朋友的二人组, 每个为朋友的二人组对应一块石头, 我们证明: $k(k+1)$ 个为朋友的二人组对应的石头互不相同.
实际上, 如果某 2 个为朋友的二人组对应相同的石头, 而这 2 个为朋友的二人组至少包含 3 个不同的人, 他们拥有公共的石头.
但将 3 人归人前述的两组, 必有 2 人在同一组, 他们应该是敌人,矛盾.
于是, 此时至少需要 $k(k+1)=\left[\frac{n^2}{4}\right]$ 块石头.
当 $n$ 为偶数时, 设 $n=2 k$, 类似地, 将 $n$ 个人分成两组, 每组 $k$ 人, 则至少需要 $k^2=\left[\frac{n^2}{4}\right]$ 块石头.
下面证明, $S=\left[\frac{n^2}{4}\right]$ 时, 可按要求构造项链.
对 $n$ 归纳.
假定 $n=k$ 时结论成立, 考虑 $n=k+1$ 的情形, 我们来分析增量 $\Delta=S_{k+1}-S_k=\left[\frac{(k+1)^2}{4}\right]-\left[\frac{k^2}{4}\right]$. 为了便于计算 $\Delta$, 应讨论 $k$ 的奇偶情况.
当 $k$ 为奇数时, 设 $k=2 r+1$, 此时, $\Delta=\left[\frac{(k+1)^2}{4}\right]-\left[\frac{k^2}{4}\right]=(r+ 1)^2-\left(r^2+r\right)=r+1=\frac{k+1}{2}$. 当 $k$ 为偶数时, 设 $k=2 r$, 此时, $\Delta= \left[\frac{(k+1)^2}{4}\right]-\left[\frac{k^2}{4}\right]=\left(r^2+r\right)-r^2=r=\frac{k}{2}$. 由此可见, 由 $k$ 到 $k+1$, 增加的石头种数为 $\frac{k+1}{2}$ (当 $k$ 为奇数时) 或 $\frac{k}{2}$ (当 $k$ 为偶数时). 由此想到将原来 $k$ 个人分成 $\frac{k+1}{2}$ (当 $k$ 为奇数时) 或 $\frac{k}{2}$ (当 $k$ 为偶数时) 组,每组不多于 2 人 ( $k$ 为奇数时恰有一组为 1 人, 其余各组都是 2 人, 而 $k$ 为偶数时, 每组都是 2 人). 希望新增加 1 人 $P$ 后, $P$ 与每一组至多需要一块新的石头.
这一要求能否实现? 如果某组中的 2 人与 $P$ 都是敌人, 则无需增加新石头.
如果该组中的 2 人与 $P$ 一是朋友一是敌人, 则将为朋友的 2 人各增加一块相同新石头即可.
但如果该组中的 2 人之间是敌人, 而他们与 $P$ 都是朋友呢, 此时每人需要增加一块不同的新石头, 需要 2 块新石头.
由此可见, 与 $P$ 都是朋友且互为敌人的 2 人不能在同一组, 但这样的分组也未必能实现, 因为 $P$ 的敌人个数也许比朋友个数多.
现在, 换一个角度思考, 如果固定某两个为朋友的人 $A 、 B$ (假定这两人是新增加的), 则原来的每个人与 $A 、 B$ 之间至多增加一块石头, 因此采用 $k$ 到 $k+2$ 的归纳方式即可完成证明.
假定 $k$ 个人时结论成立, 考虑 $k+2$ 个人的情形.
如果 $k+2$ 人中没有朋友, 则结论显然成立 (无需石头). 此外, 设 $A 、 B$ 是朋友, 则由归纳假设, 另 $k$ 个人之间至多需要 $\left[\frac{k^2}{4}\right]$ 种石头.
考察这 $k$ 个人中任意一个人 $P, P$ 与 $A 、 B$ 构成一个 3 人组, 我们证明此 3 人组只需增加一种新石头.
实际上, 如果 $A 、 B$ 与 $P$ 都是敌人, 则无需增加新石头.
如果 $A 、 B$ 与 $P$ 一是朋友一是敌人, 则将为朋友的 2 人各增加一块新石头即可.
如果 $A 、 B$ 与 $P$ 都是朋友, 则每人增加一块新石头即可.
由 $P$ 的任意性可知, $k$ 个人至多增加 $k$ 种新石头.
又 $A 、 B$ 之间至多需要一种新石头, 所以 $k+2$ 人至多需要 $\left[\frac{k^2}{4}\right]+ k+1=\left[\frac{(k+2)^2}{4}\right]$ 种石头.
综上所述, 命题获证.
%%PROBLEM_END%%



%%PROBLEM_BEGIN%%
%%<PROBLEM>%%
例7. 设 $m 、 n$ 为正整数, $m<2001, n<2002$. 有 $2001 \times 2002$ 个不同的实数, 将这些数填人 $2001 \times 2002$ 棋盘的方格, 使得每个方格内恰有一个数.
如果某个方格内的数小于其所在列的至少 $m$ 个数, 也小于其所在行的至少 $n$ 个数, 则将此方格称为 "坏格". 对所有填数方法, 求坏格个数 $S$ 的最小值.
%%<SOLUTION>%%
分析:解考察一种特殊情形: 将 $1,2,3, \cdots, 2001 \times 2002$ 按自然顺序填人 $2001 \times 2002$ 的棋盘的方格 (如下表), 此时坏格个数 $S=(2001-$ m) $(2002-n)$.
\begin{tabular}{|c|c|c|c|}
\hline 1 & 2 & $\cdots$ & 2002 \\
\hline 2003 & 2004 & $\cdots$ & 4004 \\
\hline$\cdots$ & $\cdots$ & $\cdots$ & $\cdots$ \\
\hline $2000 \times 2002+1$ & $2000 \times 2002+2$ & $\cdots$ & $2001 \times 2002$ \\
\hline
\end{tabular}
我们猜想, $S$ 的最小值为 $(2001-m)(2002-n)$. 一般地, 对 $p \times q(m<p$, $n<q)$ 棋盘, $S$ 的最小值为 $(p-m)(q-n)$. 下面用数学归纳法证明.
假定结论对 $p \times q$ 棋盘成立, 考虑 $(p+1) \times q$ 棋盘.
为了利用归纳假设, 应去掉一行, 此行应有至少 $(p+1-m)(q-n)-(p-m)(q-n)=q-n$ 个坏格.
但一行中至多 $q-n$ 个坏格, 因为该行中只有较小的前 $q-n$ 个数才小于该行中至少 $n$ 个数.
为叙述问题方便, 如果一个格所填的数小于它所在的行至少 $n$ 个数, 则称这个格是 "行坏" 的, 否则称为 "行好的". 类似定义 "列坏"、"列好" 的.
这样, 坏格就是行坏列坏的格.
由上面的分析, 我们需要有那样一行, 它不含有行坏列好的格.
由对称性, 也只需要有那样一列, 它不含有列坏行好的格.
引理: $p \times q(m<p, n<q)$ 棋盘中要么存在这样一行, 它不含有行坏列好的格;要么存在这样一列, 它不含有列坏行好的格.
引理的证明: 如果棋盘中没有行坏列好的格 (或没有行好列坏的格), 此时每行 (或每列) 都符合要求.
如果既有行坏列好的格又有行好列坏的格, 取出这些格中填数最小的一个格 $A$, 不妨设它是行好列坏的, 它所填数为 $x$. 若它所在的行有一个行坏列好的格 $B$, 设 $B$ 填的数为 $y$. 一方面, 由 $x$ 的最小性得 $x<y$. 另一方面, 由于 $A$ 是行好的而 $B$ 是行坏的, 有 $x>y$, 矛盾.
引理获证.
下面证明: 对任意 $p \times q(m<p, n<q)$ 棋盘, 坏格的个数不少于 ( $p- m)(q-n)$.
对 $p+q$ 归纳.
当 $p+q=m+n+2$ 时, 由于 $p \geqslant m+1, q \geqslant n+1$, 所以 $p=m+1, q=n+1$. 因为所有数中最小者所在的格必为坏格, 故坏格个数不少于 $1=(p-m)(q-n)$, 结论成立.
假设 $p+q=t$ 时结论成立, 考虑 $p+q= t+1$ 时的情形.
由引理, 不妨设存在一行, 它不含行坏列好的格.
将这一行去掉, 则成为一个 $(p-1)$ 行 $q$ 列的棋盘.
由于 $p+q-1=t$, 故由归纳假设, 此棋盘中坏格不少于 $(p-1-m)(q-n)$ 个.
添上原来去掉的那一行, 原来的坏格仍是坏格, 而此行中的行坏格必是列坏格 (由于此行不含行坏列好的格), 从而也必是坏格.
又此行中行坏格有 $(q-n)$ 个, 故坏格总数不少于 $(p-1- m)(q-n)+q-n=(p-m)(q-n)$. 由归纳原理, 命题获证.
对原题,坏格的个数不少于 $(2001-m)(2002-n)$. 结合前面的构造可知, 坏格个数最小值为 $(2001-m)(2002-n)$.
%%PROBLEM_END%%



%%PROBLEM_BEGIN%%
%%<PROBLEM>%%
例8. 正 2006 边形 $P$ 的一条对角线称为好的, 如果它的两端点将 $P$ 的边界分成的两部分各含 $P$ 的奇数条边.
特别地, 称 $P$ 的边也是好的.
设 $P$ 被不在 $P$ 的内部相交的 2003 条对角线剖分为三角形,试求这种剖分图中有两条边为好的等腰三角形个数的最大值.
%%<SOLUTION>%%
解:两条边为好的等腰三角形为好三角形, 先考察特例.
对于正方形,本质上只有一种剖分, 此时好三角形个数为 2 ;
对于正 6 边形,本质上只有 3 种剖分, 此时好三角形个数最大值为 3 , 而且我们发现达到最大值时,好三角形的腰都是 $P$ 的边.
对一般情况, 不难发现, 正 $2 n$ 边形的剖分中好三角形个数最大值为 $n$.
实际上,对于剖分图中的任一三角形 $A B C, P$ 的边界被 $A 、 B 、 C$ 分为 3 段,将 $A-B$ 段 (按逆时钟方向从 $A$ 到 $B$ ) 所含 $P$ 的边数记作 $m(A B)$, 以此类推.
由于 $m(A B)+m(B C)+m(C A)=2006$, 故等腰三角形若有两条好边, 则恰有两条好边,且两条好边是两腰 (否则 3 条边都是好边,矛盾).
考虑任一好三角形 $A B C$, 其中 $A B=A C$, 若 $A-B$ 段上有别的好三角形, 则将其两腰所截下的 $P$ 的边全部去掉, 则去掉的 $P$ 的边数为偶数,如此下去, 直至 $A-B$ 段上没有好三角形, 由于 $A-B$ 段上共有奇数条边, 至少有一条边 $\alpha$ 没有去掉(如果 $A B$ 本身是 $P$ 的一条边,则 $\alpha=A B$ ), $\alpha$ 不属于比 $A B$ 小的腰段.
同理, $A-C$ 段上也去掉若干个好三角形后有 $P$ 的一边 $\beta$ 不属于比 $A C$ 小的腰段,令 $\triangle A B C$ 对应于 2 元集 $\{\alpha, \beta\}$.
对于同一剖分中的两个不同的好三角形 $\triangle A B C 、 \triangle A_1 B_1 C_1$, 它们对应的 2 元集分别为 $\{\alpha, \beta\} 、\left\{\alpha_1, \beta_1\right\}$. 如果 $\triangle A_1 B_1 C_1$ 不位于 $\triangle A B C$ 的腰段, 则 $\triangle A_1 B_1 C_1$ 位于 $\triangle A B C$ 的 $B-C$ 段, 此时, $\{\alpha, \beta\}$ 中的边在 $\triangle A B C$ 的腰段上, $\left\{\alpha_1, \beta_1\right\}$ 中的边在 $\triangle A B C$ 的 $B-C$ 段上, 所以 $\{\alpha, \beta\}$ 与 $\left\{\alpha_1, \beta_1\right\}$ 没有公共的边; 如果 $\triangle A_1 B_1 C_1$ 位于 $\triangle A B C$ 的腰段上, 设在 $A-B$ 段上, 则 2 元集 $\{\alpha, \beta\}$ 中的边属于去掉 $\triangle A_1 B_1 C_1$ 的腰段上的边, 而 2 元集 $\left\{\alpha_1, \beta_1\right\}$ 中的边是 $\triangle A_1 B_1 C_1$ 的腰段上的边,从而两个 2 元集没有公共的边.
注意到 2006 条边最多有 $\frac{2006}{2}=1003$ 个两两无公共元的 2 元子集, 所以好三角形不多于 1003 个.
最后, 设 $P=A_1 A_2 \cdots A_{2006}$, 用对角线 $A_1 A_{2 k+1}(1 \leqslant k \leqslant 1002)$ 及 $A_{2 k+1} A_{2 k+3} (1 \leqslant k \leqslant 1001)$ 所作的剖分图恰有 1003 个好三角形.
因此,好三角形个数的最大值是 1003 .
%%PROBLEM_END%%



%%PROBLEM_BEGIN%%
%%<PROBLEM>%%
例9. 给定正整数 $a$, 设 $X=\left\{a_1, a_2, a_3, \cdots, a_n\right\}$ 是由正整数构成的集合, 其中 $a_1 \leqslant a_2 \leqslant a_3 \leqslant \cdots \leqslant a_n$, 若对任何整数 $p(1 \leqslant p \leqslant a)$, 都存在 $X$ 的子集 $A$, 使 $S(A)=p$, 其中规定 $S(A)$ 为集合 $A$ 中的元素的和, 求 $n$ 的最小值.
%%<SOLUTION>%%
解:究特例, 对 $a=1,2,3,4$ 进行试验, 得到相应的最小值 $n=1,2$, 2,3 , 并由此发现使 $n$ 达到最小的满足条件的集合 $X=\left\{a_1, a_2, a_3, \cdots, a_n\right\}$ 具有如下性质: 对任何 $i=1,2, \cdots, n$, 都有 $a_i \leqslant 2^{i-1}$.
实际上, 反设存在 $i(1 \leqslant i \leqslant n)$, 使 $a_i \geqslant 2^{i-1}+1$, 并设 $i$ 是这样的 $i$ 中的最小者(极端假设), 即 $a_1 \leqslant 2^0, a_2 \leqslant 2^1, a_3 \leqslant 2^2, \cdots, a_{i-1} \leqslant 2^{i-2}$, 而 $a_i \geqslant 2^{i-1}+1$, 那么, 对 $X$ 的任一不含 $a_i, a_{i+1}, \cdots, a_n$ 中任何元素的子集 $A$, 有 $S(A) \leqslant a_1+a_2+\cdots+a_{i-1} \leqslant 2^0+2^1+2^2+\cdots+2^{i-2}=2^{i-1}-1$, 而对 $X$ 的任一含有 $a_i, a_{i+1}, \cdots, a_n$ 中至少一个元素的子集 $A$, 有 $S(A) \geqslant a_i \geqslant 2^{i-1}+1$, 于是, 不存在 $X$ 的子集 $A$, 使 $S(A)=2^{i-1}$, 所以 $a \leqslant 2^{i-1}-1$.
因为 $X$ 的子集的和跑遍了 $1,2, \cdots, a$, 而 $X$ 的任一含有 $a_i, a_{i+1}, \cdots, a_n$ 中至少一个元素的子集 $A$, 都有 $S(A)>a$, 于是 $X \backslash\left\{a_i, a_{i+1}, \cdots, a_n\right\}$ 的子集的和也跑遍了 $1,2, \cdots, a$, 这与 $n$ 的最小性矛盾.
对给定的 $a$, 设 $2^r \leqslant a<2^{r+1}$, 若 $n \leqslant r$, 则因为 $a_i \leqslant 2^{i-1}(i=1,2, \cdots, n)$, 所以对 $X$ 的任何子集 $A$, 有 $S(A) \leqslant S(X)=a_1+a_2+\cdots+a_n \leqslant 2^0+2^1+ 2^2+\cdots+2^{n-1}=2^n-1 \leqslant 2^r-1<2^r \leqslant a$, 所以不存在 $X$ 的子集 $A$, 使 $S(A)=a$,矛盾,所以 $n \geqslant r+1$.
当 $n=r+1$ 时,令 $a_i=2^{i-1}(i=1,2, \cdots, r), a_{r+1}=a+1-2^r$,下证 $X=\left\{a_1, a_2, a_3, \cdots, a_{r+1}\right\}$ 满足条件.
实际上, 由二进制可知, $\left\{a_1, a_2, a_3, \cdots, a_r\right\}$ 的子集的和跑遍了 $1,2, \cdots$, $2^r-1, X$ 的含有 $a_{r+1}$ 的子集的和跑遍了 $a_{r+1}, a_{r+1}+1, a_{r+1}+2, \cdots, a_{r+1}+ 2^r-1=a$, 又 $2^r \leqslant a \leqslant 2^{r+1}$, 有 $a_{r+1}=a+1-2^r<2^{r+1}+1-2^r=2^r+1$, 所以 $a_{r+1} \leqslant 2^r$,于是 $X$ 的子集的和跑遍了 $1,2, \cdots, a$.
综上所述, $n$ 的最小值为 $r+1$, 其中 $r=\left[\log _2 a\right]$.
%%PROBLEM_END%%


