
%%TEXT_BEGIN%%
估计的一种典型的方法是算两次, 它的基本模式是:
对于集合 $X=\left\{a_1, a_2, \cdots, a_n\right\}$, 设 $F=\left\{A_1, A_2, \cdots, A_k\right\}$ 是 $X$ 的子集族 (常常是有交划分, 即可能有某两个子集的交非空), 其中第 $i$ 个子集 $A_i$ 的元素个数为 $r_i(i=1,2, \cdots, k)$.
用两种方法计算某种量 $\Omega$ (称为"中间量") 的个数 $|\Omega|$. 一方面,对整体 $X$ 计算, 常常是考虑每个元素对整体的贡献, 得 $|\Omega|=f(n)$ (总个数). 另一方面, 从子集族 $F$ 计算, 由第 $i$ 子集 $A_i$ 得 $\Omega$ 的个数为 $\left|\Omega_i\right|=f_i(k, r)$,于是
$$
\sum_{i=1}^n f_i(k, r)=\sum_{i=1}^n\left|\Omega_i\right| \leqslant|\Omega|=f(n) .
$$
这里要求,对不同的子集 $A_i 、 A_j$, 由它们得到的中间量 $\Omega$ 是不同的.
因此,常常要适当选取中间量 $\Omega$, 以保证满足这一要求.
如果对不同的子集 $A_i$ 、 $A_j$, 它们有公共的 $\Omega$, 则要去掉重复计数.
算两次的关键是"算什么". 对此,没有统一的模式,但中间量 $\Omega$ 的选择常有如下一些方法:
角:当估计的量与同色三角形有关时, 可计算同色角、异色角, 这是一种 "减元"策略.
$r$ 子集:若各子集满足条件: $\left|A_i \cap A_j\right| \leqslant r$, 则计算 $r+1$ 元子集(我们称为 "加元" 策略). 此时, 对不同的子集 $A_i 、 A_j$, 它们的 $r+1$ 元子集是互异的.
否则, $A_i 、 A_j$ 有公共的 $r+1$ 元子集, 则 $\left|A_i \cap A_j\right| \geqslant r+1$,与 $\left|A_i \cap A_j\right| \leqslant r$ 矛盾.
对子:将具有特殊关系的 2 个元素配对 (并非任意的 2 元子集), 并称之为对子,计算这样的对子的个数.
此时, 常常需要去掉重复计数.
次数:比如, 某种元素出现的总次数, 参加某种活动的人次数.
得分:某比赛选手所得的总分、各选手得分的总和.
我们先看两个利用 "算两次" 技巧计数的例子, 因为计数常常是某些极值问题中的子问题.
%%TEXT_END%%



%%PROBLEM_BEGIN%%
%%<PROBLEM>%%
例1. 某次议会中有 30 名议员, 每两位议员或为政敌,或为朋友.
而且, 每个议员都恰有 6 个政敌.
对于由 3 个议员组成的委员会, 若这 3 人中任何两个人都是朋友或任何两个人都是政敌,则称之为奇异委员会.
问: 共有多少个奇异委员会?
%%<SOLUTION>%%
解:题具有明显的图论色彩: 或为朋友或为政敌一一连边或不连边 (染红色或染蓝色); 每个议员都有 6 个政敌一一每点引出 6 条红边; 奇异委员会一一同色三角形.
考察 30 个点的完全图.
每点代表一个议员, 若某两个议员为政敌, 则将这两点间的边染红色, 否则染蓝色.
由题意, 每个点恰引出 6 条红边.
我们来计算图中同色三角形 (子集)的个数 $(|F|)$. 采用减元技巧选择中间量: 计算同色角的总数 $S$.
一方面, 对 $X$ 而言, 可从每个点出发计算, 因为每点引出 6 条红边、23 条蓝边, 有 $\mathrm{C}_6^2+\mathrm{C}_{23}^2=268$ 个同色角.
那么, 30 个点共有 $30 \times 268=8040$ 个同色角.
不同顶点引出的同色角显然不同, 于是, $S=8040$.
另一方面, 每个同色角都在某个三角形中, 设所有三角形中有 $x$ 个同色三角形,则有 $\mathrm{C}_{30}^3-x=4060-x$ 个不同色的三角形.
对于每个同色三角形, 它有 3 个同色角; 对于每个三边不同色的三角形, 它有 1 个同色角.
于是 $S=3 x+ 4060-x=2 x+4060$.
由 $8040=S=2 x+4060$, 解得 $x=1990$.
%%<REMARK>%%
注:: 计算异色角更简单, 是因同色三角形中异色角的个数为 0 . 此时, $0 \times x+2(4060-x)=\mathrm{C}_6^1 \times \mathrm{C}_{23}^1 \times 30$.
%%PROBLEM_END%%



%%PROBLEM_BEGIN%%
%%<PROBLEM>%%
例2. 平面上有 18 个点, 其中任意三点不共线, 每两点用线段联接, 将这些线段染红、蓝 2 色, 每条线段只染一种颜色.
已知其中某点 $A$ 引出的红色线段为奇数条,且其余的 17 点引出的红色线段数互不相等.
(1) 求此图中红色三角形的个数;
(2) 求此图中恰有两边为红色的三角形的个数.
%%<SOLUTION>%%
解:掉图中的所有蓝色边, 得到一个简单图 $G$. 由条件知, $d(A)$ 为奇数, 其他非 $A$ 的点的度属于 $\{0,1,2, \cdots, 17\}$. 易知, 度为 0 和 17 的点不同时存在,于是,各个点的度分别为 $0,1,2, \cdots, 16$ 或 $1,2, \cdots, 17$.
注意到 $\sum d(x)=d(A)+0+1+2+\cdots+16$ 为奇数, 矛盾.
所以各顶点的度只能分别是 $1,2, \cdots, 17$. 将各顶点记为 $V_1, V_2, \cdots, V_{17}$, 其中 $d\left(V_i\right)=i (i=1,2, \cdots, 17)$.
因为 $d\left(V_{17}\right)=17$, 所以 $V_{17}$ 与 $V_1$ 连.
又 $d\left(V_1\right)=1$, 所以 $V_1$ 仅与 $V_{17}$ 连, 所以 $V_{16}$ 不与 $V_1$ 连, 但 $d\left(V_{16}\right)=16$, 所以 $V_{16}$ 与 $V_2, V_3, \cdots, V_{17}$ 连.
如此下去, $V_i$ 恰与 $V_{18-i}, V_{19-i}, \cdots, V_{17}$ 连 $(i=1,2, \cdots, 8)$, 而 $V_{18-i}$ 恰与 $V_1$, $V_2, \cdots, V_{i-1}$ 以外的各点连 $(i=1,2, \cdots, 8)$. 再看 $V_9$, 它应与 $A, V_{17}, V_{16}$, $V_{15}, \cdots, V_{10}$ 连.
由 $V_9$ 与 $A$ 连可知, $V_1, V_2, \cdots, V_8$ 不与 $A$ 连, $V_9, V_{10}, \cdots$, $V_{17}$ 与 $A$ 连, 所以 $d(A)=9$.
令 $M=\left\{V_1, V_2, \cdots, V_8\right\}, N=\left\{A, V_9, V_{10}, \cdots, V_{17}\right\}$, 则 $M$ 中的点互不相连, 而 $N$ 中的点两两相连, 且 $M$ 中的点 $V_i$ 恰与 $N$ 中的 $i$ 个点相连.
(1) 问题等价于图 $G$ 中三角形的个数.
首先, $N$ 中有 $\mathrm{C}_{10}^3$ 个三角形.
另外, 对 $M$ 中的任意一个点 $V_i$, 它与 $N$ 中 $i$ 个点连边, 所以可得到 $\mathrm{C}_i^2$ 个三角形, 这样的三角形有 $\sum_{i=1}^8 \mathrm{C}_i^2=\mathrm{C}_9^3$ 个.
所以,红色三角形共有 $\mathrm{C}_{10}^3+\mathrm{C}_9^3=204$ 个.
(2) 对红色角总数 $S$ 计算两次.
一方面, 对 $G$ 计算.
因为点 $V_i$ 引出 $i$ 条边 $(i=2,3, \cdots, 17)$, 所以以 $V_i$ 为顶点的红色角有 $\mathrm{C}_i^2$ 个.
所以 $S=\mathrm{C}_9^2$ ( $A$ 引出的红色角) $+\sum_{i=2}^{17} \mathrm{C}_i^2=\mathrm{C}_9^2+\mathrm{C}_{18}^3$.
另一方面, 每个红色角都在某个三角形中, 设所有三角形中有 $x$ 个是恰有两条红色边的三角形, 每个这样的三角形恰有一个红色角, 得到 $x$ 个红色角.
又由 (1) 知, 有 204 个红色三角形, 每个这样的三角形恰有 3 个红色角, 得到 $204 \times 3=612$ 个红色角.
此外, 其他三角形中无红色角.
所以 $S=x+612$.
由 $\mathrm{C}_9^2+\mathrm{C}_{18}^3=S=x+612$, 解得 $x=240$.
%%PROBLEM_END%%



%%PROBLEM_BEGIN%%
%%<PROBLEM>%%
例3. 设 $S=\{1,2, \cdots, 15\}$. 从 $S$ 中取出 $n$ 个子集 $A_1, A_2, \cdots, A_n$, 满足下列条件:
(1) $\left|A_i\right|=7(i=1,2, \cdots, n)$;
(2) $\left|A_i \cap A_j\right| \leqslant 3(1 \leqslant i<j \leqslant n)$;
(3)对 $S$ 中任何 3 元子集 $M$, 存在某个 $A_k$, 使 $M \subset A_k$.
求这样的子集个数 $n$ 的最小值.
%%<SOLUTION>%%
分析:解由条件(1) $\left|A_i\right|=7(i=1,2, \cdots, n)$, 想到计算各元素在各子集中出现的总次数 $S_1$.
一方面, 从每个子集人手, 有 $S_1=7 n$. 另一方面, 从每个元素人手, 设 $i(i=1,2, \cdots, 15)$ 出现的次数为 $r_i$, 则有 $S_1=\sum_{i=1}^{15} r_i$, 所以 $7 n=S_1=\sum_{i=1}^{15} r_i$. 下面只须求 $\sum_{i=1}^{15} r_i$ 的变化范围, 一个充分条件是求出每个 $r_i$ 的范围.
不失一般性, 先求 $r_1$ 的范围.
由条件 (3), 想到计算含有 1 的所有 3 元子集的个数 $S_2$.
一方面, 从整体 $S$ 人手, 有 $S_2=\mathrm{C}_{14}^2=91$. 另一方面, 考察所有 $r_1$ 个含 1 的子集 $A_i$, 每个这样的子集中有 $\mathrm{C}_6^2=15$ 个含 1 的 3 元子集, 于是共有 $15 r_1$ 个含 1 的 3 元子集.
由条件 (3) 可知, 这样计算的含 1 的 3 元子集没有遗漏, 所以 $15 r_1 \geqslant S_2=91$, 所以 $r_1 \geqslant 7$. 同理, 对所有 $i=1,2, \cdots, 15$, 有 $r_i \geqslant 7$. 于是 $7 n=\sum_{i=1}^{15} r_i \geqslant \sum_{i=1}^{15} 7=15 \times 7$, 所以 $n \geqslant 15$.
当 $n=15$ 时, 令 $A_i=\{1+i-1,2+i-1,4+i-1,5+i-1,6+i-1$, $11+i-1,13+i-1\}(i=1,2, \cdots, 15)$, 若集合中的数大于 15 , 则取其除以 15 的余数代之.
不难验证, 这样的 15 个集合符合题目的所有条件.
综上所述, $n$ 的最小值为 15 .
%%PROBLEM_END%%



%%PROBLEM_BEGIN%%
%%<PROBLEM>%%
例4. 有 8 位歌手参加艺术节, 今要为他们安排 $m$ 次演出, 每次由其中 4 位登台表演, 要求 8 位歌手中任意两位同时演出的次数都一样多.
请设计一种方案, 使得演出的次数 $m$ 最少 .
%%<SOLUTION>%%
分析:解先看条件: "安排 $m$ 次演出, 每次由其中 4 位登台表演", 它等价于: 有 $m$ 个子集,每个集合都是 4 个元素.
再看条件: "任意两位同时演出的次数都一样多", 它等价于: 每个 2 元子集在上述各 4 元子集中出现的次数相等, 不妨设都是 $r$ 次.
由此想到分别从这两方面出发, 计算各演员出场的总次数 $S$.
一方面, $m$ 次演出,每次由其中 4 位登台表演, 于是 $S=4 m$.
另一方面,设任意两名演员同时演出的次数都为 $r$, 则 $\mathrm{C}_8^2$ 个 2 元对共出现 $r \mathrm{C}_8^2$ 次.
每个 2 元对出现 1 次对应于 2 个出场次数,于是共得 $2 r \mathrm{C}_8^2$ 次出场.
但 $A$ 的每次出场都同时出现在 3 个含 $A$ 的对子中, 被计算 3 次.
于是 $S= \frac{2 r \mathrm{C}_8^2}{3}$, 所以 $4 m=S=\frac{2 r \mathrm{C}_8^2}{3}$, 即 $6 m=r \mathrm{C}_8^2, 3 m=14 r$, 所以 $3 \mid r, r \geqslant 3$. 所以 $3 m=14 r \geqslant 14 \times 3=42$, 所以 $m \geqslant 14$.
当 $m=14$ 时,如下安排合乎条件,故 $m$ 的最小值为 14 .
$\begin{array}{llllllll}A & 1234 & 1256 & 1278 & 1357 & 1368 & 1458 & 1467 \\ A^{\prime} & 5678 & 3478 & 3456 & 2468 & 2457 & 2367 & 2358\end{array}$
%%<REMARK>%%
注:: 如果计算对子出场的总次数 $T$, 则可避免重复计数:
一方面, $m$ 次演出, 每次由其中 4 位登台表演, 有 $\mathrm{C}_4^2=6$ 个对子出场.
于是 $T=m \mathrm{C}_4^2=6 m$. 另一方面, 设任意两名演员同时演出的次数都为 $r$, 则 $\mathrm{C}_8^2$ 对共出场 $r \mathrm{C}_8^2$ 次.
所以 $6 m=T=r \mathrm{C}_8^2$. 下略.
%%PROBLEM_END%%



%%PROBLEM_BEGIN%%
%%<PROBLEM>%%
例5. 地面上有 10 只鸟在啄食, 其中任意 5 只鸟中至少有 4 只鸟在同一个圆周上.
有鸟最多的一个圆周上至少有几只鸟?
%%<SOLUTION>%%
分析:解用点代表鸟.
设有点最多的圆周上有 $r$ 个点.
显然有 $r \geqslant 4$. 能否有 $r=4$ ? 不妨先对 $r=4$ 进行探索.
若 $r=4$, 即每个圆周上至多 4 个点, 但我们考虑的圆是至少通过其中 4 个点的圆, 所以每个" 4 点圆"上都恰有 4 个点.
第一步: 计算有多少个 "4 点圆". 注意到条件: 任何 5 点组中都有 4 个点共圆,即每个"5 点组"都对应一个"4 点圆", 于是想到利用映射计算圆的个数.
实际上, 每个" 5 点组"对应一个 " 4 点圆", 这样共有 $\mathrm{C}_{10}^5=252$ 个" 4 点圆". 但每个 "4 点圆"可属于 6 个不同的 " 5 点组", 被计数 6 次, 从而 " 4 点圆"的个数为 $\frac{252}{6}=42$. 这些 "四点圆" 是互异的.
若否, 有两个不同的四点组 $A B C D$ 及 $A^{\prime} B^{\prime} C^{\prime} D^{\prime}$ 在同一个圆周上, 但 $A B C D$ 与 $A^{\prime} B^{\prime} C^{\prime} D^{\prime}$ 中至少有 5 个互异的点,这 5 个点共圆, 与 $r=4$ 矛盾.
第二步: 观察划分, 选择中间量算两次.
上述 42 个不同的 4 点圆可看作 42 个不同(可以相交)的子集, 设为: $M_1, M_2, \cdots, M_{42}$, 它们满足: $\left|M_i\right|=4$, $\left|M_i \cap M_j\right| \leqslant 2$. 采用"加元技巧", 可计算其中三角形的个数 $S$.
一方面, $S=\mathrm{C}_{10}^3=120$. 另一方面, $S \geqslant 42 \mathrm{C}_4^3=168$, 矛盾.
所以 $r>4$.
设 $M$ 是有点最多的圆, 由 $r>4$ 知 $M$ 上至少有 5 个点 $A 、 B 、 C 、 D 、 E$. 下面证明:其他点 (最多一个点除外)都在此圆周上. (*)
实际上, 反设有两个点 $P 、 Q$ 不在圆 $M$ 上, 那么 $P 、 Q 、 A 、 B 、 C$ 这 5 点中有 4 个点共圆.
但 $P 、 Q$ 都不在圆 $A B C$ 上, 只能是 $P 、 Q$ 与 $A 、 B 、 C$ 中的某两个点共圆, 不妨设 $P 、 Q 、 A 、 B$ 共圆 $M_1$.
同样考察 5 点 $P 、 Q 、 C 、 D 、 E$, 必有 $P 、 Q$ 与 $C 、 D 、 E$ 中的某两个点共圆, 不妨设 $P 、 Q 、 C 、 D$ 共圆 $M_2$. 再考察 5 点 $P 、 Q 、 A 、 C 、 E$, 必有 $P 、 Q$ 与 $A 、 C 、 E$ 中的某两个点共圆.
(i) 若 $P 、 Q 、 A 、 C$ 共圆 $M_3$, 则 $M_3$ 与 $M_1$ 重合, 所以 $P Q A B C$ 共圆, $P$ 、 $Q$ 在圆 $M$ 上,矛盾.
(ii) 若 $P 、 Q 、 A 、 E$ 共圆 $M_3$, 则 $M_3$ 与 $M_1$ 重合, 所以 $P Q A B E$ 共圆, $P$ 、 $Q$ 在圆 $M$ 上,矛盾.
(iii)若 $P 、 Q 、 C 、 E$ 共圆 $M_3$, 则 $M_3$ 与 $M_2$ 重合, 所以 $P Q C D E$ 共圆, $P$ 、 $Q$ 在圆 $M$ 上,矛盾.
综上所述, 结论 (*) 成立, 即 $r \geqslant 9$. 最后, $r=9$ 是可能的, 即 9 个点共圆, 另一个点在圆外显然合乎条件.
综上所述, $r$ 的最小值为 9 .
%%<REMARK>%%
注: 本题有相当的难度, 但当年的得分率却出乎意料的高.
其原因是很容易猜出答案, 使思维有明确的方向.
只要否定了 $r=4$, 即可发现 $r=9$, 而构造则是相当容易的.
我们还可进一步考虑: 将题中的 10 个点推广到 $n$ 个点,结论如何?
%%PROBLEM_END%%



%%PROBLEM_BEGIN%%
%%<PROBLEM>%%
例6. 有 16 名学生参加考试, 考题都是选择题, 每题有 4 个选择支, 考完后发现: 任何两人至多有一道题答案相同, 问最多有几道考题?
%%<SOLUTION>%%
分析:解设共有 $n$ 道试题, 我们证明 $n_{\text {max }}=5$.
对每一个题, 16 个学生的答案构成一个长为 16 的只含有 $1 、 2 、 3 、 4$ (答案代号) 的数列, 将 $n$ 个题对应的数列排成一个 $n \times 16$ 的数表.
注意到条件"任何两人至多有一道题答案相同", 可计算每一道题 (每一行中)出现的相同答案构成的对子.
为叙述问题方便, 对每一行, 如果两个答案代号相同, 则将这两个数字用一条线段连接, 称之为同色线段.
设某一行有 $x$ 个 $1 、 y$ 个 $2 、 z$ 个 $3 、 t$ 个 4 , 其中 $x+y+z+t=16$, 则该行共有 $\mathrm{C}_x^2+\mathrm{C}_y^2+ \mathrm{C}_z^2+\mathrm{C}_t^2 \geqslant \mathrm{C}_4^2+\mathrm{C}_4^2+\mathrm{C}_4^2+\mathrm{C}_4^2=24$ (条) 同色线段, 于是 $n$ 行至少有 $24 n$ 条同色线段.
依题意, 任何两条同色线段在第 $n$ 行上的投影互不相同, 于是 $24 n \leqslant \mathrm{C}_{16}^2=120$,所以 $n \leqslant 5$.
若 $n=5$, 则上述所有不等式都成立等号, 于是, 合乎条件的数表应满足以下两点:
(1) 每行 4 个 $1 、 4$ 个 $2 、 4$ 个 $3 、 4$ 个 4 ;
(2) 不同的同色线段在第 $n$ 行上的投影不重合.
不妨设第一行为 1111222233334444 (看成 4 组), 则其他行的每一个组都是 1234 的一个排列 (因为同色线段不重合).一种自然的排列是 1234 , 我们可在表中尽可能多地填人 1234 , 可发现第 2 行相应的组中都可填 1234 , 而第一组 (前 4 列) 除第一行外都可填 1234. 如此下去, 即可得到如下合乎条件的数表,它表明 $n=5$ 是可能的.
\begin{tabular}{|c|c|c|c|c|c|c|c|c|c|c|c|c|c|c|c|c|}
\hline 学生 & 1 & 2 & 3 & 4 & 5 & 6 & 7 & 8 & 9 & 10 & 11 & 12 & 13 & 14 & 15 & 16 \\
\hline 1 & 1 & 1 & 1 & 1 & 2 & 2 & 2 & 2 & 3 & 3 & 3 & 3 & 4 & 4 & 4 & 4 \\
\hline 2 & 1 & 2 & 3 & 4 & 1 & 2 & 3 & 4 & 1 & 2 & 3 & 4 & 1 & 2 & 3 & 4 \\
\hline 3 & 1 & 2 & 3 & 4 & 4 & 3 & 2 & 1 & 3 & 4 & 1 & 2 & 2 & 1 & 4 & 3 \\
\hline 4 & 1 & 2 & 3 & 4 & 2 & 1 & 4 & 3 & 4 & 3 & 2 & 1 & 3 & 4 & 1 & 2 \\
\hline 5 & 1 & 2 & 3 & 4 & 3 & 4 & 1 & 2 & 2 & 1 & 4 & 3 & 4 & 3 & 2 & 1 \\
\hline
\end{tabular}
%%<REMARK>%%
注:: 本题原来的解答是采用整体估计处理的 ,这里采用算两次的技巧,不仅思路自然流畅,更重要的是为后面的构造指明了方向.
%%PROBLEM_END%%



%%PROBLEM_BEGIN%%
%%<PROBLEM>%%
例7. $n$ 个人在某个节日期间互通电话问候,已知其中每个人至多打通了 3 个朋友家的电话; 任何 2 个人之间至多进行 1 次通话; 且任何 3 个人中至少有 2 人,其中一人打通了另一个人家里的电话, 求 $n$ 的最大值.
%%<SOLUTION>%%
解:们需要如下的引理.
引理: $n$ 阶简单图 $G$ 中不存在 $K_3$, 则 $\| G|| \leqslant\left[\frac{n^2}{4}\right]$.
引理的证明: 设 $A$ 是各顶点中度最大的顶点,设与 $A$ 相邻的点的集合为 $M=\left\{A_1\right.$, $\left.A_2, \cdots, A_r\right\}$, 与 $A$ 不相邻的点的集合为 $N=\left\{B_1, B_2, \cdots, B_s\right\}(r+s+1=n)$. 由于 $G$ 中无三角形, 从而 $G$ 在 $M$ 中没有边, 从而 $G$ 的其他边都在 $N$ 中或 $M 、 N$ 之间, 这样的边都是由顶点 $B_1, \cdots, B_s$ 引出的 (如图(<FilePath:./figures/fig-c11i1.png>)).
于是, ||$G|| \leqslant \mathrm{d}(A)+\mathrm{d}\left(B_1\right)+\mathrm{d}\left(B_2\right)+\cdots+\mathrm{d}\left(B_s\right) \leqslant r+r+\cdots+ r=(s+1) r \leqslant\left(\frac{s+r+1}{2}\right)^2=\frac{n^2}{4}$,
又 $\|G\| \in \mathbf{Z}$, 所以 $\|G\| \leqslant\left[\frac{n^2}{4}\right]$.
解答原题: 用 $n$ 个点表示 $n$ 个人, 如果一个人 $A$ 打通了另一个人 $B$ 家里的电话,则连一条从 $A$ 到 $B$ 的有向边, 得到一个简单的有向图 $G$.
一方面, $\bar{G}$ 中无三角形, 由引理有, || $\bar{G}|| \leqslant\left[\frac{n^2}{4}\right]$, 所以 $\| G||=\mathrm{C}_n^2-$ || $\bar{G}|| \geqslant \mathrm{C}_n^2-\left[\frac{n^2}{4}\right]=\left[\frac{(n-1)^2}{4}\right]$; 另一方面, ||$G||=\sum_{i=1}^n \mathrm{~d}^{+}\left(x_i\right) \leqslant \sum_{i=1}^n 3=3 n$, 所以
$$
\left[\frac{(n-1)^2}{4}\right] \leqslant 3 n . \label{eq1}
$$
当 $n$ 为奇数时, 式\ref{eq1} 变为 $\frac{(n-1)^2}{4} \leqslant 3 n$, 解得 $n \leqslant 13$; 当 $n$ 为偶数时, 式\ref{eq1} 变为 $\frac{n^2-2 n}{4} \leqslant 3 n$, 解得 $n \leqslant 14$.
综上所述, $n \leqslant 14$.
最后, $n=14$ 是可能的.
构造两个 $K_7$, 对其中每个七边形 $A_1 A_2 \cdots A_7$, 令 $A_i$ 指向 $A_{i+1}, A_{i+2}, A_{i+3}\left(i=1,2, \cdots, 7, A_{i+7}=A_i\right)$, 则构图合乎条件.
首先, 每个点都恰引出 3 条有向出边, 从而每个人至多打通了 3 个朋友家的电话;
其次, 对任何 3 个点, 由抽庶原理, 必有两个点 $A_i 、 A_j(i<j)$ 在同一个 $K_7$ 中, 若 $j-i \leqslant 3$, 则 $A_i$ 打通了 $A_j$ 家中的电话, 若 $j-i>3$, 则 $A_j$ 打通了 $A_i$ 家中的电话.
%%PROBLEM_END%%



%%PROBLEM_BEGIN%%
%%<PROBLEM>%%
例8. 有 $A 、 B 、 C$ 三人进行乒乓球比赛, 当其中两个人比赛时, 另一个人作裁判, 此场比赛中输者在下一场中当裁判, 另两个人接着比赛.
比了若千场以后,已知 $A$ 共比了 $a$ 场, $B$ 共比了 $b$ 场, 求 $C$ 比的场数的最小值.
%%<SOLUTION>%%
解: $C$ 共比了 $c$ 场, 则比赛的人次数之和为 $a+b+c$, 但每场比赛产生 2 个比赛人次数, 于是一共比赛 $\frac{a+b+c}{2}$ 场.
所以, $C$ 当裁判的场数为 $\frac{a+b+c}{2}-c=\frac{a+b-c}{2}$.
因为若 $C$ 在某场中当裁判, 则他必在下一场中比赛, 从而任何连续两场中 $C$ 都不能连续当裁判, 于是, $\frac{a+b-c}{2} \leqslant c+1$, 解得 $c \geqslant \frac{a+b-2}{3}$.
又 $c$ 为整数, 所以 $c \geqslant\left[\frac{a+b-2}{3}+\frac{2}{3}\right]=\left[\frac{a+b}{3}\right]$.
当 $a+b=3 k$ 时, $c \geqslant\left[\frac{a+b}{3}\right]=k$. 令 $a=2 k, b=k$, 用 $(A, B, C)$ 表示 $A 、 B$ 比赛, $C$ 当裁判的场次, 那么, 所有比赛场次为 $(A, B, C),(A, C, B)$, $(A, B, C),(A, C, B), \cdots,(A, B, C),(A, C, B)$, 共比 $2 k$ 场, 此时 $c=k=\left[\frac{a+b}{3}\right]$;
当 $a+b=3 k+1$ 时, $c \geqslant\left[\frac{a+b}{3}\right]=k$, 但 $a+b+c \equiv a+b-c \equiv 0(\bmod 2)$, 所以 $c \neq k$, 于是 $c \geqslant k+1=\left[\frac{a+b}{3}\right]+1$. 令 $a=2 k, b=k+1$, 用 $(A, B, C)$ 表示 $A 、 B$ 比赛, $C$ 当裁判的场次, 那么, 所有比赛场次为 $(A, B$, $C),(A, C, B),(A, B, C),(A, C, B), \cdots,(A, B, C),(A, C, B), \underline{(B,C, A)}$,
共比 $2 k+1$ 场, 此时 $c=k+1=\left[\frac{a+b}{3}\right]+1$;
$$
\text { 当 } a+b=3 k+2 \text { 时, } c \geqslant\left[\frac{a+b}{3}\right]=k \text {. 令 } a=2 k+1, b=k+1 \text {, 用 }(A, B \text {, }
$$
C) 表示 $A 、 B$ 比赛, $C$ 当裁判的场次, 那么, 所有比赛场次为 $(A, B, C),(A, C$, $B),(A, B, C),(A, C, B), \cdots,(A, B, C),(A, C, B),(A, B, C)$, 共比 $2 k+1$ 场, 此时 $c=k=\left[\frac{a+b}{3}\right]$.
综上所述, $c_{\min }= \begin{cases}{\left[\frac{a+b}{3}\right]} & (a+b \equiv 0,2(\bmod 3) ; \\ {\left[\frac{a+b}{3}\right]+1} & (a+b \equiv 1(\bmod 3) .\end{cases}$
%%PROBLEM_END%%



%%PROBLEM_BEGIN%%
%%<PROBLEM>%%
例9. 设 $|X|=56$, 对 $X$ 的任意 15 个子集, 只要它们中任何 7 个的并不少于 $n$ 个元素, 则这 15 个子集中一定存在其交非空的 3 个集合, 求 $n$ 的最小值.
%%<SOLUTION>%%
解:$n_{\min }=41$.
首先证明 $n=41$ 合乎条件.
用反证法: 假设存在 $X$ 的 15 个子集,它们中任何 7 个的并不少于 41 个元素, 而任何 3 个的交都为空集, 则每个元素至多属于 2 个子集, 不妨设每个元素恰属于 2 个子集 (否则在一些子集中添加一些元素, 上述条件仍然成立), 由抽屉原理, 必有一个子集, 设为 $A$, 至少含有 $\left[\frac{56 \times 2}{15}\right]+1=8$ 个元素, 又设其他 14 个子集为 $A_1$, $A_2, \cdots, A_{14}$.
考察不含 $A$ 的任何 7 个子集, 都对应 $X$ 中的 41 个元素, 所有不含 $A$ 的 7 -子集组一共至少对应 $41 \mathrm{C}_{14}^7$ 个元素.
另一方面, 对于元素 $a$, 若 $a \notin A$, 则 $A_1, A_2, \cdots, A_{14}$ 中有 2 个含有 $a$, 于是 $a$ 被计算 $\left(\mathrm{C}_{14}^7-\mathrm{C}_{12}^7\right)$ 次; 若 $a \in A$, 则 $A_1, A_2, \cdots, A_{14}$ 中有 1 个含有 $a$,于是 $a$ 被计算 $\left(\mathrm{C}_{14}^7-\mathrm{C}_{13}^7\right)$ 次, 于是, $41 \mathrm{C}_{14}^7 \leqslant(56-|A|)\left(\mathrm{C}_{14}^7-\mathrm{C}_{12}^7\right)+ |A| \cdot\left(\mathrm{C}_{14}^7-\mathrm{C}_{13}^7\right)=56\left(\mathrm{C}_{14}^7-\mathrm{C}_{12}^7\right)-|A|\left(\mathrm{C}_{13}^7-\mathrm{C}_{12}^7\right) \leqslant 56\left(\mathrm{C}_{14}^7-\mathrm{C}_{12}^7\right)- 8\left(\mathrm{C}_{13}^7-\mathrm{C}_{12}^7\right)$, 即 $48 \mathrm{C}_{12}^7+8 \mathrm{C}_{13}^7 \leqslant 15 \mathrm{C}_{14}^7$, 化简得, $3 \times 48+4 \times 13 \leqslant 15 \times 13$, 即 $196 \leqslant 195$,矛盾.
其次证明 $n \geqslant 41$, 用反证法.
假定 $n \leqslant 40$, 设 $X=\{1,2, \cdots, 56\}$, 令 $A_i=\{x \in X \mid x \equiv i(\bmod 7)\} (i=1,2, \cdots, 7), B_j=\{x \in X \mid x \equiv j(\bmod 8)\}(j=1,2, \cdots, 8)$, 显然, $\left|A_i\right|=8,\left|A_i \cap A_j\right|=0(1 \leqslant i<j \leqslant 7),\left|B_j\right|=7,\left|B_i \cap B_j\right|=0(1 \leqslant i<j \leqslant 8)$, 此外, 由中国剩余定理, $\left|A_i \cap B_j\right|=1(1 \leqslant i \leqslant 7,1 \leqslant j \leqslant 8)$. 于是, 对其中任何 3 个子集, 必有 2 个同时为 $A_i$, 或同时为 $B_j$, 其交为空集.
对其中任何 7 个子集, 设有 $t(0 \leqslant t \leqslant 7)$ 个为 $A_i, 7-t$ 个为 $B_j$, 则由容斥原理, 这 7 个子集的并的元素个数为 $8 t+7(7-t)-t(7-t)=49-t(6-t) \geqslant 49-9$ (因为 $0 \leqslant t \leqslant 7$ ) $=40$,于是任何 7 个子集的并不少于 40 个元素,但任何 3 个子集的交为空集, 所以 $n \geqslant 41$.
综上所述, $n$ 的最小值为 41 .
%%PROBLEM_END%%


