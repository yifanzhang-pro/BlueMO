
%%PROBLEM_BEGIN%%
%%<PROBLEM>%%
问题1. 在四边形 $A B C D$ 中, $\overrightarrow{A B}=\vec{a}+\vec{b}, \overrightarrow{B C}=-4 \vec{a}-\vec{b}, \overrightarrow{C D}=-5 \vec{a}-2 \vec{b}$, 四边形 $A B C D$ 的形状为 ( ).
(A) 长方形
(B) 平行四边形
(C) 菱形
(D) 梯形
%%<SOLUTION>%%
D.
$$
\overrightarrow{A D}=\overrightarrow{A B}+\overrightarrow{B C}+\overrightarrow{C D}=-8 \vec{a}-2 \vec{b}=2 \overrightarrow{B C}
$$
%%PROBLEM_END%%



%%PROBLEM_BEGIN%%
%%<PROBLEM>%%
问题2. 若 $\overrightarrow{A B}=3 \vec{e}, \overrightarrow{C D}=4 \vec{e}$, 且 $|\overrightarrow{A D}|=|\overrightarrow{C B}|$, 则四边形 $A B C D$ 是( ).
(A) 平行四边形
(B) 梯形
(C) 等腰梯形
(D) 菱形
%%<SOLUTION>%%
C.
$$
A B / / C D, A D=C B \text {. }
$$
%%PROBLEM_END%%



%%PROBLEM_BEGIN%%
%%<PROBLEM>%%
问题3. 平行四边形 $A B C D$ 中, 对角线 $A C$ 和 $B D$ 交于点 $O$. 若 $\overrightarrow{A C}=\vec{a}, \overrightarrow{B D}=\vec{b}$, 那么用 $\vec{a}$ 和 $\vec{b}$ 表示 $\overrightarrow{A B}$ 为 $(\quad)$.
(A) $\frac{1}{2}(\vec{a}+\vec{b})$
(B) $\frac{1}{2}(\vec{a}-\vec{b})$
(C) $\frac{1}{2}(\vec{b}-\vec{a})$
(D) $\vec{b}-\vec{a}$
%%<SOLUTION>%%
B.
$$
\overrightarrow{A B}=\overrightarrow{O B}-\overrightarrow{O A}=\left(-\frac{\vec{b}}{2}\right)-\left(-\frac{\vec{a}}{2}\right)=\frac{1}{2}(\vec{a}-\vec{b}) .
$$
%%PROBLEM_END%%



%%PROBLEM_BEGIN%%
%%<PROBLEM>%%
问题4. 若 $\overrightarrow{O A}=\vec{a}+\vec{b}, \overrightarrow{A B}=3(\vec{a}-\vec{b}), \overrightarrow{C B}=3 \vec{a}+\vec{b}$, 则 $\overrightarrow{O C}=$
%%<SOLUTION>%%
$\vec{a}-3 \vec{b}$.
$$
\begin{aligned}
\overrightarrow{O C} & =\overrightarrow{O A}+\overrightarrow{A B}+\overrightarrow{B C}=\overrightarrow{O A}+\overrightarrow{A B}-\overrightarrow{C B} \\
& =\vec{a}+\vec{b}+3(\vec{a}-\vec{b})-(3 \vec{a}+\vec{b})=\vec{a}-3 \vec{b}
\end{aligned}
$$
%%PROBLEM_END%%



%%PROBLEM_BEGIN%%
%%<PROBLEM>%%
问题5. 已知梯形 $O A B C$ 中, $\overrightarrow{C B} / / \overrightarrow{O A}$, 且 $|\overrightarrow{C B}|=\frac{1}{3}|\overrightarrow{O A}|$. 若 $\overrightarrow{O A}=\vec{a}, \overrightarrow{O C}= \vec{b}$, 则 $\overrightarrow{A B}=$
%%<SOLUTION>%%
$\vec{b}-\frac{2 \vec{a}}{3}$.
$$
\overrightarrow{A B}=\overrightarrow{A O}+\overrightarrow{O C}+\overrightarrow{C B}=-\vec{a}+\vec{b}+\frac{\vec{a}}{3}=\vec{b}-\frac{2 \vec{a}}{3} .
$$
%%PROBLEM_END%%



%%PROBLEM_BEGIN%%
%%<PROBLEM>%%
问题6. 如图(<FilePath:./figures/fig-c4p6.png>), 任意四边形 $A B C D$ 中, $M 、 N$ 分别为 $A D$ 、 $B C$ 的中点, $G$ 为 $M N$ 的中点, $O$ 为平面内的任意一点, 求证:
(1) $\overrightarrow{G A}+\overrightarrow{G B}+\overrightarrow{G C}+\overrightarrow{G D}=\overrightarrow{0}$;
(2) $\overrightarrow{O G}=\frac{1}{4}(\overrightarrow{O A}+\overrightarrow{O B}+\overrightarrow{O C}+\overrightarrow{O D})$.
%%<SOLUTION>%%
(1) 设 $\overrightarrow{G M}=\vec{a}, \overrightarrow{M D}=\vec{m}, \overrightarrow{N C}=\vec{n}$, 则 $\overrightarrow{G N}=-\vec{a}, \overrightarrow{M A}=-\vec{m}, \overrightarrow{N B}= -\vec{n}$
从而 $\overrightarrow{G A}=\vec{a}-\vec{m}, \overrightarrow{G C}=-\vec{a}+\vec{n}, \overrightarrow{G B}=-\vec{a}-\vec{n}, \overrightarrow{G D}=\vec{a}+\vec{m}$.
所以 $\overrightarrow{G A}+\overrightarrow{G B}+\overrightarrow{G C}+\overrightarrow{G D}=\overrightarrow{0}$.
(2) $\overrightarrow{O A}=\overrightarrow{O G}+\overrightarrow{G A}, \overrightarrow{O C}=\overrightarrow{O G}+\overrightarrow{G C}, \overrightarrow{O B}=\overrightarrow{O G}+\overrightarrow{G B}, \overrightarrow{O D}=\overrightarrow{O G}+ \overrightarrow{G D}$
从而 $\overrightarrow{O A}+\overrightarrow{O B}+\overrightarrow{O C}+\overrightarrow{O D}=4 \overrightarrow{O G}+\overrightarrow{G A}+\overrightarrow{G B}+\overrightarrow{G C}+\overrightarrow{G D}=4 \overrightarrow{O G}$, 即 $\overrightarrow{O G}=\frac{1}{4}(\overrightarrow{O A}+\overrightarrow{O B}+\overrightarrow{O C}+\overrightarrow{O D})$, 证毕.
%%PROBLEM_END%%



%%PROBLEM_BEGIN%%
%%<PROBLEM>%%
问题7. 设 $A_1 A_2 A_3 A_4$ 为 $\odot O$ 的内接四边形, $H_1 、 H_2 、 H_3 、 H_4$ 依次为 $\triangle A_2 A_3 A_4$ 、 $\triangle A_3 A_4 A_1 、 \triangle A_4 A_1 A_2 、 \triangle A_1 A_2 A_3$ 的垂心, 求证: $H_1 、 H_2 、 H_3 、 H_4$ 在同一个圆上,并定出该圆的圆心.
%%<SOLUTION>%%
设 $\odot O$ 的半径为 $R$, 设 $\overrightarrow{O A_1}+\overrightarrow{O A_2}+\overrightarrow{O A_3}+\overrightarrow{O A_4}=\overrightarrow{O C}$, 则由例 3 知
$$
\begin{gathered}
\overrightarrow{O H_1}=\overrightarrow{O A_2}+\overrightarrow{O A_3}+\overrightarrow{O A_4}=\overrightarrow{O C}-\overrightarrow{O A_1}, \\
\left|\overrightarrow{H_1 C}\right|=\left|\overrightarrow{O C}-\overrightarrow{O H_1}\right|=\left|\overrightarrow{O A_1}\right|=R .
\end{gathered}
$$
从而
$$
\left|\overrightarrow{H_1 C}\right|=\left|\overrightarrow{O C}-\overrightarrow{O H_1}\right|=\left|\overrightarrow{O A_1}\right|=R
$$
同理 $\left|\overrightarrow{H_2 C}\right|=\left|\overrightarrow{O A_2}\right|=R,\left|\overrightarrow{H_3 C}\right|=\left|\overrightarrow{O A_3}\right|=R,\left|\overrightarrow{H_4 C}\right|=\left|\overrightarrow{O A_4}\right|=R$. 所以 $H_1 、 H_2 、 H_3 、 H_4$ 在同一圆上, 以 $C$ 为圆心, 以 $R$ 为半径, 证毕.
%%PROBLEM_END%%


