
%%TEXT_BEGIN%%
复数的模与幅角.
关于复数的模的概念, 在第一章中已有定义.
在复数的三角形式表示中, 出现了复数的辐角的概念.
有关复数辐角的问题是近年高中数学竞赛的热点问题之一.
面给出复数辐角的定义和一些性质.
设复数 $z=a+b \mathrm{i}(a, b \in \mathbf{R})$ 所对应的向量为 $\overrightarrow{O Z}$, 我们称始边是 $x$ 轴正半轴, 终边是 $\overrightarrow{O Z}$ 的角称为复数 $z$ 的辐角, 记为 $\operatorname{Arg} z$. 在 $[0,2 \pi)$ 内的辐角叫做复数 $z$ 的辐角主值, 记为 $\arg z$. 且有
$$
\operatorname{Arg} z=\arg z+2 k \pi(k \in \mathbf{Z}) .
$$
当 $a \in \mathbf{R}^{+}$时,有
$$
\begin{gathered}
\arg a=0, \arg (-a)=\pi, \\
\arg (a \mathrm{i})=\frac{\pi}{2}, \arg (-a \mathrm{i})=\frac{3 \pi}{2} .
\end{gathered}
$$
0 的辐角是任意的.
非零复数与它的模和辐角主值构成一一对应关系.
两个非零复数相等, 当且仅当它们的模与幅角主值分别相等.
关于复数辐角的运算,有如下结论:
$$
\begin{gathered}
\operatorname{Arg}\left(z_1 z_2\right)=\operatorname{Arg}\left(z_1\right)+\operatorname{Arg}\left(z_2\right), \\
\operatorname{Arg}\left(\frac{z_1}{z_2}\right)=\operatorname{Arg}\left(z_1\right)-\operatorname{Arg}\left(z_2\right)\left(z_2 \neq 0\right), \\
\operatorname{Arg}\left(z^n\right)=n \operatorname{Arg}(z)(n \in \mathbf{Z}) .
\end{gathered}
$$
若复数 $z=a+b \mathrm{i}(a 、 b \in \mathbf{R}, a b \neq 0)$, 则
$$
\arg z= \begin{cases}\arctan \frac{b}{a}, & \text { 点 }(a, b) \text { 在第 I 象限 } \\ \pi+\arctan \frac{b}{a}, & \text { 点 }(a, b) \text { 在第 II 、III 象限 } \\ 2 \pi+\arctan \frac{b}{a}, \text { 点 }(a, b) \text { 在第 IV 象限 }\end{cases}
$$
有了上述准备工作, 我们可以定义复数的三角形式:
设复数 $z=a+b \mathrm{i}(a 、 b \in \mathbf{R})$ 的模等于 $r$, 辐角等于 $\theta$, 则称 $z=r(\cos \theta+ i \sin \theta)$ 为复数 $z=a+b \mathrm{i}(a 、 b \in \mathbf{R})$ 的三角形式.
以下介绍复数在三角形式下的乘法、乘方、除法、开方等运算的法则.
一、复数的乘法与乘方.
若 $z_1=r_1\left(\cos \theta_1+\mathrm{i} \sin \theta_1\right), z_2=r_2\left(\cos \theta_2+\mathrm{i} \sin \theta_2\right)$, 则
$$
\begin{aligned}
z_1 z_2 & =r_1\left(\cos \theta_1+\mathrm{i} \sin \theta_1\right) \cdot r_2\left(\cos \theta_2+\mathrm{i} \sin \theta_2\right) \\
& =r_1 r_2\left[\cos \left(\theta_1+\theta_2\right)+\mathrm{i} \sin \left(\theta_1+\theta_2\right)\right] .
\end{aligned}
$$
两个复数相乘,积的模等于各复数的模的积, 积的辐角等于各复数的辐角的和.
若 $z=r(\cos \theta+i \sin \theta), n \in \mathbf{N}^*$, 则
$$
z^n=[r(\cos \theta+\mathrm{i} \sin \theta)]^n=r^n(\cos n \theta+\mathrm{i} \sin n \theta) .
$$
复数 $n$ 次幂的模等于这个复数模的 $n$ 次幂, 它的辐角等于这个复数辐角的 $n$ 倍, 这个定理叫做棣莫佛 (Abraham de Moivre, 1667-1754 年)定理.
二、复数的除法.
若 $z_1=r_1\left(\cos \theta_1+i \sin \theta_1\right), z_2=r_2\left(\cos \theta_2+i \sin \theta_2\right)$, 则
$$
\begin{aligned}
\frac{z_1}{z_2} & =\frac{r_1\left(\cos \theta_1+\mathrm{i} \sin \theta_1\right)}{r_2\left(\cos \theta_2+\mathrm{i} \sin \theta_2\right)} \\
& =\frac{r_1}{r_2}\left[\cos \left(\theta_1-\theta_2\right)+\mathrm{i} \sin \left(\theta_1-\theta_2\right)\right] .
\end{aligned}
$$
两个复数相除, 商的模等于被除数的模除以除数的模所得的商, 商的辐角等于被除数的辐角减去除数的辐角.
三、复数的开方.
复数 $r(\cos \theta+\mathrm{i} \sin \theta)$ 的 $n\left(n \in \mathbf{N}^*\right)$ 次方根是
$$
\sqrt[n]{r}\left(\cos \frac{\theta+2 k \pi}{n}+\mathrm{i} \sin \frac{\theta+2 k \pi}{n}\right)(k=0,1, \cdots, n-1) .
$$
复数的 $n$ 次方根是 $n$ 个复数, 它们的模都等于这个复数的模的 $n$ 次算术根, 它们的辐角分别等于这个复数的辐角与 $2 \pi$ 的 $0,1, \cdots, n-1$ 倍的和的 $n$ 分之一.
由此可知,方程 $x^n=b(b \in \mathbf{C})$ 的根的几何意义是复平面内的 $n$ 个点, 这些点均匀分布在以原点为圆心 、以 $\sqrt[n]{\mid b T}$ 为半径的圆周上.
四、辐角的三角函数.
设复数 $z=\cos \theta+\mathrm{i} \sin \theta=\mathrm{e}^{\mathrm{i} \theta}$, 则
$$
\begin{gathered}
\cos n \theta=\operatorname{Re}\left(z^n\right)=\frac{z^{2 n}+1}{2 z^n} ; \\
\sin n \theta=\operatorname{Im}\left(z^n\right)=\frac{z^{2 n}-1}{2 z^n \mathrm{i}} ; \\
\tan n \theta=\frac{\operatorname{Im}\left(z^n\right)}{\operatorname{Re}\left(z^n\right)}=\frac{z^{2 n}-1}{\left(z^{2 n}+1\right) \mathrm{i}} .
\end{gathered}
$$
%%TEXT_END%%



%%PROBLEM_BEGIN%%
%%<PROBLEM>%%
例1. 求下列各复数的辐角主值:
(1) $z=\mathrm{i} \cos 100^{\circ}$;
(2) $z=a+b \mathrm{i}\left(a 、 b \in \mathbf{R}, a^2+b^2 \neq 0\right)$;
(3) $z=\sin 4(\cos 4+\operatorname{isin} 4)$;
(4) $z=|\cos \theta|+\mathrm{i}|\sin \theta|, \theta \in\left(\frac{5}{2} \pi, 3 \pi\right)$.
%%<SOLUTION>%%
分析:与解 (1) 因为 $i \cos 100^{\circ}$ 是纯虚数, 且 $\cos 100^{\circ}<0$, 所以
$\arg \left(\right.$ icos $\left.100^{\circ}\right)=\frac{3}{2} \pi$.
(2). 由于 $a 、 b$ 不确定,需讨论, 分六类:
当 $a>0, b>0$ 时,复数对应的点位于第一象限,由辐角主值的定义, $\tan \theta=\frac{b}{a}$, 由反正切函数的定义知, $\arg z=\arctan \frac{b}{a}$.
当 $a<0, b>0$ 时,复数对应的点位于第二象限,
$$
\arg z=\pi-\arctan \frac{b}{|a|}=\pi+\arctan \frac{b}{a} .
$$
当 $a<0, b<0$ 时,复数对应的点位于第三象限, $\arg z=\pi+\arctan \frac{b}{a}$.
当 $a>0, b<0$ 时, 复数对应的点位于第四象限,
$$
\arg z=2 \pi-\arctan \frac{|b|}{a}=2 \pi+\arctan \frac{b}{a} .
$$
当 $a=0$ 时,如果 $b>0, \arg z=\frac{\pi}{2}$; 如果 $b<0, \arg z=\frac{3}{2} \pi$.
当 $b=0$ 时,如果 $a>0, \arg z=0$; 如果 $a<0, \arg z=\pi$.
综上所述
$$
\arg z= \begin{cases}\arctan \frac{b}{a}, & (a>0, b \geqslant 0) \\ \pi+\arctan \frac{b}{a}, & (a<0) \\ 2 \pi+\arctan \frac{b}{a}, & (a>0, b<0) \\ \frac{\pi}{2}, & (a=0, b>0) \\ \frac{3 \pi}{2} . & (a=0, b<0)\end{cases}
$$
(3)不能错误地理解为 4 (因为 $4 \in(0,2 \pi)$ ) 就是该复数的辐角主值, 因为 $\sin 4<0$,已知的复数表达形式不是三角形式, 事实上,
$$
\begin{aligned}
z & =\sin 4(\cos 4+\mathrm{isin} 4) \\
& =-\sin 4[\cos (\pi+4)+\mathrm{i} \sin (\pi+4)] .
\end{aligned}
$$
此时, 不能错误地理解为 $4+\pi$ 是辐角主值 (这是因为 $4+\pi>2 \pi$ ). 该复数的辐角主值是:
$$
(4+\pi)-2 \pi=4-\pi .
$$
(4)首先把绝对值符号去掉, 并"改造"成复数的三角形式:
$$
\begin{aligned}
z & =|\cos \theta|+\mathrm{i}|\sin \theta|==-\cos \theta+\mathrm{i} \sin \theta \\
& =\cos (\pi-\theta)+\mathrm{i} \sin (\pi-\theta), \theta \in\left(\frac{5 \pi}{2}, 3 \pi\right)
\end{aligned}
$$
因为 $(\pi-\theta) \in\left(-2 \pi,-\frac{3}{2} \pi\right)$, 所以 $\arg z=(\pi-\theta)+2 \pi=3 \pi-\theta$.
%%PROBLEM_END%%



%%PROBLEM_BEGIN%%
%%<PROBLEM>%%
例2. 化下列各复数为三角形式:
(1) $z=\sqrt{6}-\sqrt{2} \mathrm{i}$;
(2) $z=r(\cos \theta-i \sin \theta)(r>0)$;
(3) $z=1+\mathrm{i} \tan \theta\left(\theta \neq k \pi+\frac{\pi}{2}, k \in \mathbf{Z}\right)$;
(4) $z=1+\cos \theta-i \sin \theta$.
%%<SOLUTION>%%
分析:与解解决这类问题的关键是准确地掌握复数三角形式的四个外 部特征:
模 $r \geqslant 0$ 一一根据模的定义: 向量 $\overrightarrow{O Z}$ 的长度有 $r \geqslant 0$;
角相同一这是因为两个角表示的是同一个复数的辐角;
余弦在前, 正弦在后一一因为在 $a+b \mathrm{i}$ 中, $a$ 是点 $(a, b)$ 的横坐标, 且 $a= r \cos \theta$, 所以"余弦在前";
"加"相连一由 $a+b \mathrm{i}$, 自然有"加"相连.
(1) 因为 $|z|=2 \sqrt{2}$, 并且
$$
\left\{\begin{array}{l}
\sin \theta=\frac{-\sqrt{2}}{2 \sqrt{2}}=-\frac{1}{2}, \\
\cos \theta=\frac{\sqrt{6}}{2 \sqrt{2}}=\frac{\sqrt{3}}{2} .
\end{array}\right.
$$
解得 $\theta=2 k \pi+\frac{11}{6} \pi(k \in \mathbf{Z})$.
所以 $z=\sqrt{6}-\sqrt{2} \mathrm{i}$ 的三角形式是 $2 \sqrt{2}\left(\cos \frac{11}{6} \pi+\mathrm{i} \sin \frac{11}{6} \pi\right)$.
(2) $z$
$$
\begin{aligned}
z & =r(\cos \theta-\mathrm{i} \sin \theta) \\
& =r[\cos \theta+\mathrm{i} \sin (-\theta)] \\
& =r[\cos (-\theta)+\mathrm{i} \sin (-\theta)] .
\end{aligned}
$$
事实上, $r(\cos \theta+i \sin \theta)$ 与 $r(\cos \theta-i \sin \theta)$ 互为共轭复数; 它们在复平面上对应的向量关于实轴对称, 因此, 它们的模相等, 有一对辐角互为相反数, 即
$$
r(\cos \theta-\mathrm{i} \sin \theta)=r[\cos (-\theta)+\mathrm{i} \sin (-\theta)] .
$$
类似地有
$$
\begin{aligned}
& r(-\cos \theta+\mathrm{i} \sin \theta)=r[\cos (\pi-\theta)+\mathrm{i} \sin (\pi-\theta)], \\
& -r(\cos \theta+\mathrm{i} \sin \theta)=r[\cos (\pi+\theta)+\mathrm{i} \sin (\pi+\theta)], \\
& r(\sin \theta+i \cos \theta)=r\left[\cos \left(\frac{\pi}{2}-\theta\right)+i \sin \left(\frac{\pi}{2}-\theta\right)\right]
\end{aligned}
$$
等等.
(3)利用三角函数公式
$$
\begin{aligned}
z & =1+\mathrm{i} \tan \theta\left(\theta \neq k \pi+\frac{\pi}{2}, k \in \mathbf{Z}\right) \\
& =1+\mathrm{i} \frac{\sin \theta}{\cos \theta} \\
& =\frac{1}{\cos \theta}(\cos \theta+\mathrm{i} \sin \theta) .
\end{aligned}
$$
至此, 不要以为它就是三角形式了, 由于 $\theta$ 的取值范围不确定, $\frac{1}{\cos \theta}$ 可正可负, 需讨论:
当 $\cos \theta>0$ 时,复数的三角形式是 $\sec \theta(\cos \theta+i \sin \theta)$;
当 $\cos \theta<0$ 时,复数的三角形式是 $-\sec \theta[\cos (\pi+\theta)+i \sin (\pi+\theta)]$.
$$
\begin{aligned}
z & =1+\cos \theta-i \sin \theta \\
& =2 \cos ^2 \frac{\theta}{2}-2 i \sin \frac{\theta}{2} \cos \frac{\theta}{2} \\
& =2 \cos \frac{\theta}{2}\left(\cos \frac{\theta}{2}-i \sin \frac{\theta}{2}\right) \\
& =2 \cos \frac{\theta}{2}\left[\cos \left(-\frac{\theta}{2}\right)+i \sin \left(-\frac{\theta}{2}\right)\right]
\end{aligned}
$$
当 $\cos \frac{\theta}{2} \geqslant 0$ 时,复数的三角形式是 $2 \cos \frac{\theta}{2}\left[\cos \left(-\frac{\theta}{2}\right)+i \sin \left(-\frac{\theta}{2}\right)\right]$;
当 $\cos \frac{\theta}{2}<0$ 时,复数的三角形式是 $-2 \cos \frac{\theta}{2}\left[\cos \left(\pi-\frac{\theta}{2}\right)+i \sin \left(\pi-\frac{\theta}{2}\right)\right]$.
%%PROBLEM_END%%



%%PROBLEM_BEGIN%%
%%<PROBLEM>%%
例3. 已知复数 $z_1 、 z_2$ 满足 $\left|z_1\right|=1,\left|z_2\right|=r, \arg \left(z_1-\sqrt{3}\right)=\frac{5 \pi}{6}$,
$\quad \arg \left(\frac{z_2}{z_1}\right)=\frac{\pi}{4}$, 求 $z_1 、 z_2$.
%%<SOLUTION>%%
分析:与解引人复数 $z_1$ 的三角形式, 设 $z_1=(\cos \theta+\mathrm{i} \sin \theta)(0 \leqslant \theta<2 \pi)$, 则
$$
z_1-\sqrt{3}=(\cos \theta-\sqrt{3})+\mathrm{i} \sin \theta .
$$
因为
$$
\arg \left(z_1-\sqrt{3}\right)=\frac{5 \pi}{6},
$$
所以
$$
\frac{\sin \theta}{\cos \theta-\sqrt{3}}=\tan \frac{5 \pi}{6}=-\frac{\sqrt{3}}{3},
$$
即
$$
\begin{gathered}
3 \sin \theta+\sqrt{3} \cos \theta=3, \\
\sin \left(\theta+\frac{\pi}{6}\right)=\frac{\sqrt{3}}{2} .
\end{gathered}
$$
因为
$$
0 \leqslant \theta<2 \pi \text {, }
$$
所以
$$
\frac{\pi}{6} \leqslant \theta+\frac{\pi}{6}<\frac{13 \pi}{6},
$$
于是,有
$\theta+\frac{\pi}{6}=\frac{\pi}{3}$ 或 $\theta+\frac{\pi}{6}=\frac{2 \pi}{3}$,
即
$\theta=\frac{\pi}{6}$ 或 $\theta=\frac{\pi}{2}$.
故
$$
z_1=\frac{\sqrt{3}}{2}+\frac{1}{2} \mathrm{i} \text { 或 } z_1=\mathrm{i} \text {. }
$$
由 $\arg \left(\frac{z_2}{z_1}\right)=\frac{\pi}{4}$, 知
$\arg z_2=\frac{\pi}{4}+\frac{\pi}{6}=\frac{5 \pi}{12}$ 或 $\arg z_2=\frac{\pi}{4}+\frac{\pi}{2}=\frac{3 \pi}{4}$,
故知 $\quad z_2=r\left(\cos \frac{5 \pi}{12}+\mathrm{i} \sin \frac{5 \pi}{12}\right)=\frac{r}{4}[\sqrt{6}-\sqrt{2}+\mathrm{i}(\sqrt{6}+\sqrt{2})]$
或
$$
z_2=r\left(\cos \frac{3 \pi}{4}+\mathrm{i} \sin \frac{3 \pi}{4}\right)=\frac{\sqrt{2}}{2} r(-1+\mathrm{i}) .
$$
综上可知, 所求复数为 $z_1=\frac{\sqrt{3}}{2}+\frac{1}{2} \mathrm{i}, z_2=\frac{r}{4}[\sqrt{6}-\sqrt{2}+\mathrm{i}(\sqrt{6}+\sqrt{2})]$ 或 $z_1=\mathrm{i}, z_2=\frac{\sqrt{2}}{2} r(-1+\mathrm{i})$.
%%PROBLEM_END%%



%%PROBLEM_BEGIN%%
%%<PROBLEM>%%
例4. 已知复数 $z_1 、 z_2$ 满足 $\left|z_1\right|=2,\left|z_2\right|=3,3 z_1-2 z_2=\frac{6}{5}(3+\mathrm{i})$, 试求 $z_1 z_2$ 的值.
%%<SOLUTION>%%
分析:与解由 $\left|z_1\right|==2,\left|z_2\right|=3$ 知, $z_1 \overline{z_1}=4, z_2 \overline{z_2}=9$.
由 $3 z_1-2 z_2=\frac{1}{3} z_2 \overline{z_2} z_1-\frac{1}{2} z_1 \overline{z_1} z_2$
$$
\begin{aligned}
& =\frac{1}{6} z_1 z_2\left(2 \overline{z_2}-3 \overline{z_1}\right) \\
& =-\frac{1}{6} z_1 z_2\left(\overline{3 z_1-2} \overline{z_2}\right),
\end{aligned}
$$
可知
$$
\begin{aligned}
z_1 z_2 & =-6 \cdot \frac{3 z_1-2 z_2}{3 z_1-2 z_2} \\
& =-6 \cdot \frac{3+\mathrm{i}}{3-\mathrm{i}} \\
& =-\frac{24}{5}-\frac{18}{5} \mathrm{i} .
\end{aligned}
$$
%%<REMARK>%%
注:本题亦可设出 $z_1 、 z_2$ 的三角形式求解, 计算较为复杂, 读者不妨一试.
%%PROBLEM_END%%



%%PROBLEM_BEGIN%%
%%<PROBLEM>%%
例5. 设 $\alpha 、 \beta$ 为复数且 $|\alpha|=k$, 证明:
$$
\left|k| \beta |^{\frac{1}{2}}-\alpha \frac{\beta}{|\beta|^{\frac{1}{2}}}\right|^2=2 k[k|\beta|-\operatorname{Re}(\alpha \beta)] .
$$
%%<SOLUTION>%%
分析:与解给出两种证明方法.
证法 1 先证明另一个形式上简单一些的恒等式:
$$
|| w|-w|^2=2|w|(|w|-\operatorname{Re} w) 
$$
这是因为令 $w=a+\mathrm{i} b(a, b \in \mathbf{R})$, 则左边即为 $\left(\sqrt{a^2+b^2}-a\right)^2+b^2$, 右边为 $2\left(a^2+b^2\right)-2 a \sqrt{a^2+b^2}$, 两边显然是相等的.
由此, 令 $w=\alpha-\frac{\beta}{|\beta|^{\frac{1}{2}}}$, 注意到 $\left|\alpha \frac{\beta}{|\beta|^{\frac{1}{2}}}\right|=k|\beta|^{\frac{1}{2}}$ 即知题目中的恒等式是成立的, 证毕.
%%PROBLEM_END%%



%%PROBLEM_BEGIN%%
%%<PROBLEM>%%
例5. 设 $\alpha 、 \beta$ 为复数且 $|\alpha|=k$, 证明:
$$
\left|k| \beta |^{\frac{1}{2}}-\alpha \frac{\beta}{|\beta|^{\frac{1}{2}}}\right|^2=2 k[k|\beta|-\operatorname{Re}(\alpha \beta)] .
$$
%%<SOLUTION>%%
证法 2 利用恒等式: $|\alpha-\beta|^2=|\alpha|^2+|\beta|^2-2 \operatorname{Re} \bar{\alpha} \beta$, 则
$$
\begin{aligned}
\left|k| \beta |^{\frac{1}{2}}-\alpha \frac{\beta}{|\beta|^{\frac{1}{2}}}\right|^2 & =\left(k|\beta|^{\frac{1}{2}}\right)^2+\left(k|\beta|^{\frac{1}{2}}\right)^2-2 \operatorname{Re}(k \alpha \beta) \\
& =2 k[k|\beta|-\operatorname{Re}(\alpha \beta)],
\end{aligned}
$$
证毕.
%%<REMARK>%%
注:事实上,证法 1 中的那个形式上简单一些的恒等式也可以由证法 2 中的那个恒等式推出.
这是一个熟知的结论, 但是有相当多的学生不习惯, 或者说不喜欢用这个式子, 因为觉得这个式子破坏了对称性, 不如 $|\alpha-\beta|^2= |\alpha|^2+|\beta|^2-\bar{\alpha} \beta-\alpha \bar{\beta}$ 来得美观.
但在本题中, 恰恰是这个实形式的恒等式对解题所起的作用要比对称形式的那个式子大 (读者可以自行比较).
%%PROBLEM_END%%



%%PROBLEM_BEGIN%%
%%<PROBLEM>%%
例6. 记 $A=\cos \frac{\pi}{11}+\cos \frac{3 \pi}{11}+\cos \frac{5 \pi}{11}+\cos \frac{7 \pi}{11}+\cos \frac{9 \pi}{11}$,
$$
B=\sin \frac{\pi}{11}+\sin \frac{3 \pi}{11}+\sin \frac{5 \pi}{11}+\sin \frac{7 \pi}{11}+\sin \frac{9 \pi}{11},
$$
求证: $A=\frac{1}{2}, B=\frac{1}{2} \cot \frac{\pi}{22}$.
%%<SOLUTION>%%
分析:与解设 $z=\cos \frac{\pi}{11}+i \sin \frac{\pi}{11}$, 则
$$
A+B \mathrm{i}=z+z^3+z^5+z^7+z^9=\frac{z\left(1-z^{10}\right)}{1-z^2}=\frac{z-z^{11}}{1-z^2}
$$
$$
\begin{aligned}
& =\frac{z-(\cos \pi-i \sin \pi)}{1-z^2}=\frac{z+1}{1-z^2}=\frac{1}{1-z} \\
& =\frac{1-\bar{z}}{(1-z)(1-\bar{z})}=\frac{1-\cos \frac{\pi}{11}+\mathrm{i} \sin \frac{\pi}{11}}{2-(z+\bar{z})} \\
& =\frac{1-\cos \frac{\pi}{11}+\mathrm{i} \sin \frac{\pi}{11}}{2\left(1-\cos \frac{\pi}{11}\right)}=\frac{1}{2}+\frac{1}{2} \cdot \frac{\sin \frac{\pi}{11}}{1-\cos \frac{\pi}{11}} \mathrm{i} \\
& =\frac{1}{2}+\mathrm{i} \cdot \frac{1}{2} \cdot \cot \frac{\pi}{22} .
\end{aligned}
$$
所以 $A=\frac{1}{2}, B=\frac{1}{2} \cot \frac{\pi}{22}$, 证毕.
%%PROBLEM_END%%



%%PROBLEM_BEGIN%%
%%<PROBLEM>%%
例7. 求 $\tan 20^{\circ}+4 \sin 20^{\circ}$ 的值.
%%<SOLUTION>%%
分析:与解引人复数, 设 $z=\cos 20^{\circ}+\mathrm{i} \sin 20^{\circ}$, 则 $\bar{z}=\frac{1}{z}=\cos 20^{\circ}-$ isin $20^{\circ}$. 于是,有
$$
\begin{gathered}
\cos 20^{\circ}=\frac{z+\bar{z}}{2}=\frac{z^2+1}{2 z}, \\
\sin 20^{\circ}=\frac{z-\bar{z}}{2 \mathrm{i}}=\frac{z^2-1}{2 z \mathrm{i}}, \\
z^3=\cos 60^{\circ}+\mathrm{i} \sin 60^{\circ}=\frac{1}{2}+\frac{\sqrt{3}}{2} \mathrm{i} .
\end{gathered}
$$
所以
$$
\begin{aligned}
\tan 20^{\circ}+4 \sin 20^{\circ} & =\frac{z^2-1}{\left(z^2+1\right) \mathrm{i}}+4 \cdot \frac{z^2-1}{2 z \mathrm{i}} \\
& =\frac{2 z^4+z^3-z-2}{\mathrm{i}\left(z^3+z\right)} \\
& =\frac{2\left(\frac{1}{2}+\frac{\sqrt{3}}{2} \mathrm{i}\right) z+\left(\frac{1}{2}+\frac{\sqrt{3}}{2} \mathrm{i}\right)-z-2}{\mathrm{i}\left(\frac{1}{2}+\frac{\sqrt{3}}{2} \mathrm{i}+z\right)} \\
& =\frac{\sqrt{3}\left(z \mathrm{i}+\frac{1}{2} \mathrm{i}-\frac{\sqrt{3}}{2}\right)}{\frac{1}{2} \mathrm{i}+z \mathrm{i}-\frac{\sqrt{3}}{2}} \\
& =\sqrt{3} .
\end{aligned}
$$
故 $\tan 20^{\circ}+4 \sin 20^{\circ}=\sqrt{3}$.
%%<REMARK>%%
注:(1) 以上两题阐明了复数与三角的联系.
(2) 下面给出一组求值题, 有趣的是, 它们的值均为 $\sqrt{3}$.
$$
\begin{aligned}
& \cot 10^{\circ}-4 \cos 10^{\circ} ; \\
& \cot 20^{\circ}-\sec 10^{\circ} ; \\
& \csc 40^{\circ}+\tan 10^{\circ} ; \\
& 4 \sin 40^{\circ}-\tan 40^{\circ} \text {. }
\end{aligned}
$$
%%PROBLEM_END%%



%%PROBLEM_BEGIN%%
%%<PROBLEM>%%
例8. 已知数列 $\left\{a_n\right\} 、\left\{b_n\right\}$, 对大于 1 的整数 $n$,均成立
$$
\begin{aligned}
& a_n=a_{n-1} \cos \theta-b_{n-1} \sin \theta, \\
& b_n=a_{n-1} \sin \theta+b_{n-1} \cos \theta,
\end{aligned}
$$
且 $a_1=1, b_1=\tan \theta$, 其中 $\theta$ 为已知锐角, 试求数列 $\left\{a_n\right\} 、\left\{b_n\right\}$ 的通项公式.
%%<SOLUTION>%%
分析:与解引人复数,构造等比数列.
设 $z_n=a_n+b_n \mathrm{i}\left(a_n, b_n \in \mathbf{R}, n \in \mathbf{N}^*\right)$, 则
$$
\begin{aligned}
\frac{z_n}{z_{n-1}} & =\frac{\left(a_{n-1} \cos \theta-b_{n-1} \sin \theta\right)+\left(a_{n-1} \sin \theta+b_{n-1} \cos \theta\right) \mathrm{i}}{a_{n-1}+b_{n-1} \mathrm{i}} \\
& =\frac{(\cos \theta+\mathrm{i} \sin \theta)\left(a_{n-1}+b_{n-1} \mathrm{i}\right)}{a_{n-1}+b_{n-1} \mathrm{i}} \\
& =\cos \theta+\mathrm{i} \sin \theta,
\end{aligned}
$$
这说明复数列 $\left\{z_n\right\}$ 是以 $z_1=1+\mathrm{i} \tan \theta$ 为首项, 以 $q=\cos \theta+i \sin \theta$ 为公比的等比数列, 于是, 有
$$
\begin{aligned}
z_n & =(1+i \tan \theta)(\cos \theta+i \sin \theta)^{n-1} \\
& =\sec \theta \cdot(\cos \theta+i \sin \theta)(\cos \theta+i \sin \theta)^{n-1} \\
& =\sec \theta \cdot(\cos \theta+i \sin \theta)^n \\
& =(\cos n \theta+i \sin n \theta) \sec \theta,
\end{aligned}
$$
故 $a_n=\sec \theta \cdot \cos n \theta, b_n=\sec \theta \cdot \sin n \theta$.
%%<REMARK>%%
注:这是复数与数列的结合.
%%PROBLEM_END%%


