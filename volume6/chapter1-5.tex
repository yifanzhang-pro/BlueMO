
%%TEXT_BEGIN%%
等差数列与等比数列.
等差数列与等比数列是两类最简单的数列, 它们是其他数列化归的对象.
如果一个数列从第 2 项起,每一项与它的前一项的差(比)等于同一个常数, 那么称它为等差 (比) 数列.
这个常数叫做等差 (比) 数列的公差 (比), 通常用 $d$ 表示公差, $q$ 表示公比.
注意, 由于零不能做分母, 因此, $q$ 不能等于零.
相应于等差 (比)数列的通项与求和有下面的相关公式:
1. 记等差数列 $\left\{a_n\right\}$ 的前 $n$ 项之和为 $S_n$, 则 $a_n=a_1+(n-1) d, S_n= \frac{1}{2}\left(a_1+a_n\right) n=a_1 n+\frac{n(n-1)}{2} d$.
2. 记等比数列 $\left\{a_n\right\}$ 的前 $n$ 项之和为 $S_n$, 则 $a_n=a_1 \cdot q^{n-1}$, 而
$$
S_n= \begin{cases}n a_1, & \text { 若 } q=1, \\ \frac{a_1\left(1-q^n\right)}{1-q}, & \text { 若 } q \neq 1 .\end{cases}
$$
3. 如果等比数列 $\left\{a_n\right\}$ 是一个无穷数列, 其公比 $q$ 满足 $|q|<1$, 那么称它为无穷递缩等比数列, 其所有项之和 $S=\frac{a_1}{1-q}$.
%%TEXT_END%%



%%PROBLEM_BEGIN%%
%%<PROBLEM>%%
例1. 将 $n^2(n \geqslant 4)$ 个正实数排成 $n$ 行 $n$ 列
$$
\begin{array}{cccc}
a_{11} & a_{12} & \cdots & a_{1 n} \\
a_{21} & a_{22} & \cdots & a_{2 n} \\
\cdots & \cdots & \cdots & \cdots \\
a_{n 1} & a_{n 2} & \cdots & a_m
\end{array}
$$
其中每一行的数成等差数列, 每一列的数成等比数列, 并且所有的公比相等.
已知 $a_{24}=1, a_{42}=\frac{1}{8}, a_{43}=\frac{3}{16}$. 求 $a_{11}+a_{22}+\cdots+a_{m n}$ 的值.
%%<SOLUTION>%%
解:设每一列数所成等比数列的公比都为 $q$, 则 $a_{44}=a_{24} \cdot q^2=q^2$.
由于表中的第 4 行成等差数列, 于是 $a_{42}, a_{43}, a_{44}$ 也成等差, 故 $a_{42}+ a_{44}=2 a_{43}$, 得
$$
\frac{1}{8}+q^2=\frac{6}{16},
$$
知 $q^2=\frac{1}{4}$, 结合表中每个数都为正实数得 $q=\frac{1}{2}$.
利用第 4 行中的数成等差数列及 $a_{42}=\frac{1}{8}, a_{43}=\frac{3}{16}$, 知该等差数列的公差为 $\frac{3}{16}-\frac{1}{8}=\frac{1}{16}$, 故首项 $a_{41}=-\frac{1}{8}-\frac{1}{16}=\frac{1}{16}$. 于是, 对任意 $1 \leqslant k \leqslant n$, 都有 $a_{4 k}=\frac{k}{16}$.
现在由第 $k$ 列是以 $\frac{1}{4}$ 为公比的等比数列, 知 $a_{1 k}=a_{4 k} \cdot q^{-3}=\frac{k}{16} \cdot\left(\frac{1}{2}\right)^{-3}= \frac{k}{2}$, 于是, 对任意 $1 \leqslant m \leqslant n$, 都有 $a_{m k}=a_{1 k} \cdot q^{m-1}=\frac{k}{2^m}$. 进而 $a_{m m}=\frac{m}{2^m}$.
记 $S=a_{11}+a_{22}+\cdots+a_{m n}$, 则 $S=\sum_{m=1}^n \frac{m}{2^m}$, 故 $\frac{S}{2}=\sum_{m=1}^n \frac{m}{2^{m+1}}$, 两式相减, 得
$$
\begin{aligned}
\frac{1}{2} S & =\sum_{m=1}^n \frac{m}{2^m}-\sum_{m=1}^n \frac{m}{2^{m+1}} \\
& =\sum_{m=1}^n \frac{m}{2^m}-\sum_{m=2}^{n+1} \frac{m-1}{2^m} \\
& =\frac{1}{2}+\sum_{m=2}^n\left(\frac{m}{2^m}-\frac{m-1}{2^m}\right)-\frac{n}{2^{n+1}} . \\
& =\frac{1}{2}+\sum_{m=2}^n \frac{1}{2^m}-\frac{n}{2^{n+1}} \\
& =\sum_{m=1}^n \frac{1}{2^m}-\frac{n}{2^{n+1}} \\
& =1-\frac{1}{2^n}-\frac{n}{2^{n+1}} \\
& =1-\frac{(n+2)}{2^{n+1}} .
\end{aligned}
$$
所以 $S=2-\frac{n+2}{2^n}$.
%%PROBLEM_END%%



%%PROBLEM_BEGIN%%
%%<PROBLEM>%%
例2. 已知关于 $x$ 的方程
$$
(2 a-1) \sin x+(2-a) \sin 2 x=\sin 3 x
$$
的非负实数解从小到大构成一个无穷等差数列.
求实数 $a$ 的取值范围.
%%<SOLUTION>%%
解:方程变形为
$$
\begin{aligned}
& 2 a \sin x-a \sin 2 x+2 \sin 2 x-\sin x-\sin 3 x=0 \\
\Leftrightarrow & 2 a \sin x(1-\cos x)+2 \sin 2 x-2 \sin 2 x \cos x=0 \\
\Leftrightarrow & (2 a \sin x-2 \sin 2 x)(1-\cos x)=0 .
\end{aligned}
$$
于是 $1-\cos x=0$ 或者 $\sin 2 x=a \sin x$.
前者的所有非负实数解为 $x=2 k_1 \pi, k_1 \in \mathbf{N}$; 对于后者, 方程化为 $\sin x=0$ 或 $\cos x=\frac{a}{2}$, 其中 $\sin x=0$ 的非负实数解为 $x=k_2 \pi, k_2 \in \mathbf{N}$, 而 $\cos x=\frac{a}{2}$ 仅当 $|a| \leqslant 2$ 时有解, 此时非负实数解为 $x=2 k_3 \pi+\arccos \frac{a}{2}$ 或 $x=2 k_4 \pi+\pi+\arccos \frac{a}{2}, k_3 、 k_4 \in \mathbf{N}$.
综上可知, 当 $|a| \geqslant 2$ 时, 方程的非负实数解为 $x=k \pi, k \in \mathbf{N}$, 它们成等差数列; 当 $|a|<2$ 时, 方程的非负实数解为 $x=k \pi, k \in \mathbf{N}$, 或者 $x= 2 k_3 \pi+\arccos \frac{a}{2}$ 或 $x=2 k_4 \pi+\pi+\arccos \frac{a}{2}$, 当且仅当 $\arccos \frac{a}{2}=\frac{\pi}{2}$, 即 $a=$ 0 时, 方程的所有非负实数解从小到大构成等差数列.
所以,满足条件的 $a \in(-\infty,-2] \cup\{0\} \cup[2,+\infty)$.
%%PROBLEM_END%%



%%PROBLEM_BEGIN%%
%%<PROBLEM>%%
例3. 两个由正整数组成的无穷数列满足:一个是以 $d(>0)$ 为公差的等差数列, 一个是以 $q(>1)$ 为公比的等比数列; 这里 $d 、 q$ 互素.
证明: 如果这两个数列有一项相同, 那么存在无穷多项相同.
%%<SOLUTION>%%
证明:可设所给的两个数列分别为 $\{a+n d\}, n=0,1,2, \cdots ;\left\{b q^m\right\}$, $m=0,1,2, \cdots$. 这里 $a 、 b 、 d 、 q$ 都是正整数,且 $q>1$.
如果它们有一项相同, 不妨设两个数列的第一项相同, 否则去掉各数列中的前面的有限项后再讨论, 即 $a=b$. 此时, 为证两个数列有无穷多项相同, 只需证明:存在无穷多个 $m \in \mathbf{N}^*$, 使得
$$
a q^m \equiv a(\bmod d),
$$
这只需 $q^m \equiv 1(\bmod d)$.
注意到 $1, q, q^2, \cdots, q^d$ 除以 $d$ 所得余数只有 $d$ 种不同取值, 由抽庶原则可知, 存在 $0 \leqslant i<j \leqslant d$, 使得 $q^j \equiv q^i(\bmod d)$, 结合 $(d, q)=1$, 得 $q^{j-i} \equiv 1(\bmod d)$, 进而, 对任意 $n \in \mathbf{N}^*$, 令 $m=(j-i) n$, 就有 $q^m=\left(q^{j-i}\right)^n \equiv 1^n= 1(\bmod d)$.
所以,命题成立.
说明熟悉 Euler 定理的同学还可利用当 $(d, q)=1$ 时, $q^{\varphi(d)} \equiv 1(\bmod d)$ 来构造符合要求的 $m$.
%%PROBLEM_END%%



%%PROBLEM_BEGIN%%
%%<PROBLEM>%%
例4. 数列 $\left\{a_n\right\}$ 定义如下
$$
a_1=1000000, a_{n+1}=n\left[\frac{a_n}{n}\right]+n, n=1,2, \cdots .
$$
%%<SOLUTION>%%
证明: 此数列中有一个无穷子数列 (由数列中的项组成的数列称为该数列的子数列)构成一个等差数列.
证明记 $x_n=\frac{a_{n+1}}{n}$, 则对任意 $n \in \mathbf{N}^*$, 都有 $x_n=\left[\frac{a_n}{n}\right]+1 \in \mathbf{N}^*$, 即 $\left\{x_n\right\}$ 是一个由正整数组成的数列.
进一步, 对任意 $n \in \mathbf{N}^*$, 有
$$
\begin{aligned}
x_{n+1} & =\left[\frac{a_{n+1}}{n+1}\right]+1=\left[\frac{n x_n}{n+1}\right]+1 \\
& =x_n+\left[-\frac{x_n}{n+1}\right]+1 \\
& \leqslant x_n+(-1)+1=x_n .
\end{aligned}
$$
这表明 $\left\{x_n\right\}$ 是一个不增数列, 所以, 从某一项起 $\left\{x_n\right\}$ 变为一个常数 (这是由于 $x_n$ 都为正整数). 记这个常数为 $k$, 那么从该项起有 $a_n=k n$, 因而, $\left\{a_n\right\}$ 从该项起构成一个等差数列.
命题获证.
说明本题的结论不依赖于初始值 (只要 $a_1 \geqslant 0$ 即可), 解决过程中用到一个显然的结论: 任意一个由正整数组成的不增无穷数列从某项起将变为常数.
%%PROBLEM_END%%



%%PROBLEM_BEGIN%%
%%<PROBLEM>%%
例5. 对任意给定的正整数 $n \geqslant 3$, 证明: 存在由正整数组成的等差数列 $a_1, a_2, \cdots, a_n$ 和等比数列 $b_1, b_2, \cdots, b_n$, 使得
$$
b_1<a_1<b_2<a_2<\cdots<b_n<a_n . \label{eq1}
$$
%%<SOLUTION>%%
证明:注意到, 指数增长的速度大于线性增长, 因此, 不存在由正整数组成的递增的无穷等差数列 $\left\{a_m\right\}$ 和等比数列 $\left\{b_m\right\}$, 使得对任意 $m \in \mathbf{N}^*$, 都有 $a_m>b_m$, 当然更不能满足 式\ref{eq1}, 本题讨论的是有穷数列, 其构造思路是让 $\left\{b_m\right\}$ 的公比尽量靠近 1 , 但在相邻两项之间又有足够的空间.
考察由下面方式定义的数列 $\left\{a_n\right\}$ 和 $\left\{b_n\right\}$ :
$$
\begin{gathered}
b_1=x^n, b_2=x^{n-1}(1+x), \cdots, b_n=x(1+x)^{n-1} ; \\
a_m=x^{n-1}(1+x)-1+(m-1) x^{n-1}, m=1,2, \cdots, n .
\end{gathered}
$$
其中 $x$ 为待定的正整数.
则 $\left\{a_m\right\}$ 是以 $x^{n-1}$ 为公差的等差数列, $\left\{b_m\right\}$ 是以 $1+\frac{1}{x}$
为公比的等比数列.
故只需证明:存在正整数 $x$,使得式\ref{eq1}成立.
一方面, 对 $1 \leqslant m \leqslant n$, 由于 $a_m=x^n+x^{n-1}-1+(m-1) x^{n-1}$, 故当 $x>1$ 时, 都有 $a_m>x^n$. 因此
$$
a_{m+1}=a_m+x^{n-1}<a_m+\frac{a_m}{x}=a_m\left(1+\frac{1}{x}\right),
$$
结合 $a_1=b_2-1<b_2$ 及 $\left\{b_n\right\}$ 是以 $1+\frac{1}{x}$ 为公比的等比数列, 可知对任意 $1 \leqslant m \leqslant n-1$, 都有 $a_m<b_{m+1}$.
另一方面, 我们证明: 存在 $x \in \mathbf{N}^*(x>1)$, 使得对任意 $1 \leqslant m \leqslant n$, 都有 $b_m<a_m$.
事实上,
$$
\begin{aligned}
b_m & <a_m \Leftrightarrow x^{n-m+1}(1+x)^{m-1}<x^n+m x^{n-1}-1 \\
& \Leftrightarrow x^n+\mathrm{C}_{m-1}^{m-2} x^{n-1}+\mathrm{C}_{m-1}^{m-3} x^{n-2}+\cdots+\mathrm{C}_{m-1}^0 x^{n-m+1} \\
& <x^n+m x^{n-1}-1 \\
& \Leftrightarrow \mathrm{C}_{m-1}^{m-3} x^{n-2}+\cdots+\mathrm{C}_{m-1}^0 x^{n-m+1}<x^{n-1}-1 \\
& \Leftrightarrow x\left(\mathrm{C}_{m-1}^{m-3} x^{n-3}+\cdots+\mathrm{C}_{m-1}^0 x^{n-m}\right)<x^{n-1}-1 .
\end{aligned} \label{eq2}
$$
利用 $n \geqslant m$, 可知 $\mathrm{C}_{n-1}^{m-3} x^{n-3}+\cdots+\mathrm{C}_{n-1}^{m-1} x^{n-m}+\mathrm{C}_{n-1}^1 x+\mathrm{C}_{n-1}^0 \geqslant \mathrm{C}_{m-1}^{m-3} x^{n-3}+\cdots+ \mathrm{C}_{m-1}^{m-1} x^{n-m}$, 因此, 如果
$$
x\left(\mathrm{C}_{n-1}^{n-3} x^{n-3}+\cdots+\mathrm{C}_{n-1}^1 x+\mathrm{C}_{n-1}^0\right)<x^{n-1}-1 . \label{eq3}
$$
成立, 那么式\ref{eq2}成立.
\ref{eq3}式左边是关于 $x$ 的 $n-2$ 次多项式, 而右边是 $x$ 的 $n-1$ 次多项式,所以,当 $x$ 充分大时,\ref{eq3}式成立.
综上可知, 满足条件的数列存在.
%%PROBLEM_END%%



%%PROBLEM_BEGIN%%
%%<PROBLEM>%%
例6. 设 $k(\geqslant 2)$ 为给定的正整数,对任意 $1 \leqslant i \leqslant k$, 数 $a_i 、 d_i$ 都是正整数, 等差数列 $\left\{a_i+n d_i\right\}$ ( $\left.n=0,1,2, \cdots\right)$ 对应的集合为 $A_i=\left\{a_i+n d_i \mid n=\right. 0,1,2, \cdots\}, 1 \leqslant i \leqslant k$. 已知 $A_1, A_2, \cdots, A_k$ 构成 $\mathbf{N}^*$ 的一个 $k$-分划 (即 $A_1$, $\cdots, A_k$ 两两的交为空集, 且 $\left.A_1 \cup \cdots \cup A_k=\mathbf{N}^*\right)$. 证明下述结论:
(1) $\frac{1}{d_1}+\cdots+\frac{1}{d_k}=1$;
(2) $\frac{a_1}{d_1}+\cdots+\frac{a_k}{d_k}=\frac{k+1}{2}$.
%%<SOLUTION>%%
证明:利用母函数方法来处理, 依题中条件可知, 对 $|x|<1$, 有
$$
\sum_{m=1}^{+\infty} x^m=\sum_{i=1}^k\left(\sum_{n=0}^{+\infty} x^{a_i+n d_i}\right)
$$
利用无穷递缩等比数列求和公式, 知
$$
\frac{x}{1-x}=\sum_{i=1}^k \frac{x^{a_i}}{1-x^{d_i}} .
$$
于是,有
$$
x=\sum_{i=1}^k \frac{x^{a_i}}{1+x+\cdots+x^{d_i-1}} . \label{eq1}
$$
上式两边让 $x$ 从左边趋向于 1 , 取极限即有 $\sum_{i=1}^k \frac{1}{d_i}=1$, 从而 式\ref{eq1} 成立.
现在对\ref{eq1}式两边关于 $x$ 求导数,得
$$
1=\sum_{i=1}^k \frac{a_i x^{a_i-1}\left(1+x+\cdots+x^{d_i-1}\right)-x^{a_i}\left(0+1+2 x+\cdots+\left(d_i-1\right) x^{d_i-2}\right)}{\left(1+x+\cdots+x^{d_i-1}\right)^2},
$$
再在上式两边让 $x$ 从左边趋向于 1 , 取极限得
$$
1=\sum_{i=1}^k \frac{a_i d_i-\left(1+2+\cdots+\left(d_i-1\right)\right)}{d_i^2} .
$$
于是
$$
\begin{aligned}
\sum_{i=1}^k \frac{a_i}{d_i} & =1+\sum_{i=1}^k \frac{\left(d_i-1\right) d_i}{2 d_i^2} \\
& =1+\frac{1}{2} \sum_{i=1}^k\left(1-\frac{1}{d_i}\right) \\
& =1+\frac{1}{2}(k-1)=\frac{k+1}{2}
\end{aligned}
$$
这里用到结论 式\ref{eq1}.
所以,命题成立.
%%PROBLEM_END%%


