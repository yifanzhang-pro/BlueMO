
%%TEXT_BEGIN%%
选择恰当的归纳对象.
与正整数有关的命题中有时会出现多个变量, 这时采用数学归纳法处理时,首先应选择好归纳的对象.
%%TEXT_END%%



%%PROBLEM_BEGIN%%
%%<PROBLEM>%%
例1. 设 $m, n \in \mathbf{N}^*$. 证明: 对任意正实数 $x_1, \cdots, x_n ; y_1, \cdots, y_n$. 若 $x_i+ y_i=1, i=1,2, \cdots, n$, 则
$$
\left(1-x_1 \cdots x_n\right)^m+\left(1-y_1^m\right)\left(1-y_2^m\right) \cdots\left(1-y_n^m\right) \geqslant 1 . \label{eq1}
$$
%%<SOLUTION>%%
证明:对 $n$ 归纳.
当 $n=1$ 时, 由条件知
$$
\left(1-x_1\right)^m+\left(1-y_1^m\right)=y_1^m+\left(1-y_1^m\right)=1 .
$$
故式\ref{eq1}对 $n=1$ 成立.
现设式\ref{eq1}对 $n-1(n \geqslant 2)$ 成立, 考虑 $n$ 的情形.
$$
\begin{aligned}
& \left(1-x_1 \cdots x_n\right)^m+\left(1-y_1^m\right) \cdots\left(1-y_n^m\right) \\
= & \left(1-x_1 \cdots x_{n-1}\left(1-y_n\right)\right)^m+\left(1-y_1^m\right) \cdots\left(1-y_n^m\right) \\
\geqslant & \left(1-x_1 \cdots x_{n-1}+x_1 \cdots x_{n-1} y_n\right)^m+\left(1-\left(1-x_1 \cdots x_{n-1}\right)^m\right)\left(1-y_n^m\right) .
\end{aligned}
$$
记 $a=1-x_1 \cdots x_{n-1}, b=y_n$, 由上式知为证 式\ref{eq1} 对 $n$ 成立, 只需证明:
$$
(a+b-a b)^m+\left(1-a^m\right)\left(1-b^m\right) \geqslant 1
$$
对任意 $a, b \in(0,1)$ 都成立.
即证
$$
(a+b-a b)^m \geqslant a^m+b^m-a^m b^m . \label{eq2}
$$
对式\ref{eq2}处理时,我们通过对 $m$ 归纳来进行.
当 $m=1$ 时, 式\ref{eq2} 显然成立; 现设 式\ref{eq2} 对 $m-1(m \geqslant 2)$ 成立, 则
$$
\begin{aligned}
& (a+b-a b)^m-a^m-b^m+a^m b^m \\
\geqslant & \left(a^{m-1}+b^{m-1}-a^{m-1} b^{m-1}\right)(a+b-a b)-a^m-b^m+a^m b^m \\
= & 2 a^m b^m+a b^{m-1}+b a^{m-1}-a^m b^{m-1}-a^{m-1} b^m-a^m b-a b^m \\
= & \left(b^{m-1}-b^m\right)\left(a-a^m\right)+\left(a^{m-1}-a^m\right)\left(b-b^m\right) .
\end{aligned}
$$
注意到 $a, b \in(0,1)$, 故 $b^{m-1} \geqslant b^m, a \geqslant a^m, a^{m-1} \geqslant a^m, b \geqslant b^m$, 所以 $(a+b- a b)^m \geqslant a^m+b^m-a^m b^m$, 即式\ref{eq2}对 $m$ 成立.
综上可知, 命题 式\ref{eq1}成立.
说明这个与两个正整数变量有关的问题, 选择对 $n$ 归纳 (视 $m$ 为常数) 是容易想到的, 因为这时式\ref{eq1}左边的第 2 个加项在作归纳过渡时显得容易处理些.
%%PROBLEM_END%%



%%PROBLEM_BEGIN%%
%%<PROBLEM>%%
例2. 证明: 对任意 $m, n \in \mathbf{N}^*$, 数
$$
S(m, n)=\sum_{i=0}^{2^{n-1}-1}\left(\tan \frac{(2 i+1) \pi}{2^{n+1}}\right)^{2 m}
$$
都为正整数.
%%<SOLUTION>%%
证明:选择 $n$ 作为归纳对象.
当 $n=1$ 时, $S(m, 1)=\left(\tan \frac{\pi}{4}\right)^{2 m}=1$, 故命题对 $n=1$ 及 $m \in \mathbf{N}^*$ 成立.
现设命题对 $n-1$ 及 $m \in \mathbf{N}^*$ 成立, 考虑 $n$ 的情形.
由 $\cot 2 \alpha=\frac{1-\tan ^2 \alpha}{2 \tan \alpha}=\frac{1}{2}(\cot \alpha-\tan \alpha)$ 可知
$$
\tan ^2\left(\frac{\pi}{2}-2 \alpha\right)=\frac{1}{4}(\cot \alpha-\tan \alpha)^2=\frac{1}{4}\left(\tan ^2 \alpha+\cot ^2 \alpha-2\right),
$$
所以,利用首尾配对求和, 知
$$
\begin{aligned}
S(1, n) & =\frac{1}{2} \sum_{i=0}^{2^{n-1}-1}\left(\tan ^2 \frac{(2 i+1) \pi}{2^{n+1}}+\tan ^2 \frac{\left(2\left(2^{n-1}-1-i\right)+1\right) \pi}{2^{n+1}}\right) \\
& =\frac{1}{2} \sum_{i=0}^{2^{n-1}-1}\left(\tan ^2 \frac{(2 i+1) \pi}{2^{n+1}}+\tan ^2\left(\frac{\pi}{2}-\frac{(2 i+1) \pi}{2^{n+1}}\right)\right) \\
& =\frac{1}{2} \sum_{i=0}^{2^{n-1}-1}\left(\tan ^2 \frac{(2 i+1) \pi}{2^{n+1}}+\cot ^2 \frac{(2 i+1) \pi}{2^{n+1}}\right) \\
& =\frac{1}{2} \sum_{i=0}^{2^{n-1}-1}\left(4 \tan ^2 \frac{(2 i+1) \pi}{2^n}+2\right) \\
& =2 \sum_{i=0}^{2^{n-1}-1} \tan ^2 \frac{(2 i+1) \pi}{2^n}+2^{n-1} \\
& =2 \sum_{i=0}^{2^{n-2}-1}\left(\tan ^2 \frac{(2 i+1) \pi}{2^n}+\tan ^2 \frac{\left(2\left(2^{n-1}-1-i\right)+1\right) \pi}{2^n}\right)+2^{n-1} \\
& =2 \sum_{i=0}^{2^{n-2}-1}\left(\tan ^2 \frac{(2 i+1) \pi}{2^n}+\tan ^2\left(\pi-\frac{(2 i+1) \pi}{2^n}\right)\right)+2^{n-1} \\
& =4 \sum_{i=0}^{2^{n-2}-1} \tan ^2 \frac{(2 i+1) \pi}{2^n}+2^{n-1}=4 S(1, n-1)+2^n .
\end{aligned}
$$
从而 $S(1, n) \in \mathbf{N}^*$.
下面设 $S(1, n), S(2, n), \cdots, S(m-1, n)$ 都为正整数, 考虑 $S(m, n)$.
注意到, 对 $k \in \mathbf{N}$, 由二项式定理有
$$
\left(x+x^{-1}\right)^k=\mathrm{C}_k^0\left(x^k+x^{-k}\right)+\mathrm{C}_k^1\left(x^{k-1}+x^{-(k-1)}\right)+\cdots, \label{eq1}
$$
因此
$$
\begin{aligned}
& \left(\frac{1}{4}\left(x+x^{-1}-2\right)\right)^m \\
= & \frac{1}{4^m} \sum_{k=0}^m \mathrm{C}_m^k\left(x+x^{-1}\right)^k \cdot(-2)^{m-k} \\
= & \frac{1}{4^m}\left(\left(x^m+x^{-m}\right)+b_1\left(x^{m-1}+x^{-(m-1)}\right)+\cdots+b_{m-1}\left(x+x^{-1}\right)+b_m\right),
\end{aligned}
$$
这里 $b_1, \cdots, b_m \in \mathbf{Z}$, 并用到式\ref{eq1}的结论.
在上式中令 $x=\tan ^2 \frac{(2 i+1) \pi}{2^{n+1}}$, 并对 $i=0,1,2, \cdots, 2^{n-2}-1$ 求和, 利用 $S(1, n)$ 中类似的计算, 可知
$$
S(m, n-1)=\frac{1}{4^m}\left(S(m, n)+b_1 S(m-1, n)+\cdots+b_{m-1} S(1, n)+b_m\right),
$$
得
$$
S(m, n)=4^m \cdot S(m, n-1)-b_1 S(m-1, n)-\cdots-b_{m-1} S(1, n)-b_m .
$$
从而 $S(m, n) \in \mathbf{Z}$, 而 $S(m, n)$ 中每一项都大于零, 故 $S(m, n) \in \mathbf{N}^*$.
综上可知, 对任意 $m 、 n \in \mathbf{N}^*$, 数 $S(m, n)$ 都为正整数,命题获证.
说明本质上在从 $n-1$ 过渡到 $n$ 的过程中, 我们采用了对 $m$ 再归纳的方法.
在双变量命题中, 这种处理是利用数学归纳法处理时常见的方法.
%%PROBLEM_END%%



%%PROBLEM_BEGIN%%
%%<PROBLEM>%%
例3. 设 $t$ 个非负整数 $a_1, a_2, \cdots, a_t$ 满足
$$
a_i+a_j \leqslant a_{i+j} \leqslant a_i+a_j+1,
$$
这里 $1 \leqslant i, j \leqslant t, i+j \leqslant t$.
求证: 存在 $x \in \mathbf{R}$, 使得对任意 $n \in\{1,2, \cdots, t\}$, 都有 $a_n=[n x]$.
%%<SOLUTION>%%
证明:记 $I_n=\left[\frac{a_n}{n}, \frac{a_n+1}{n}\right), n=1,2, \cdots, t$, 要求证明存在实数
$$
x \in \bigcap_{n=1}^t I_n . \label{eq1}
$$
现设 $L=\max _{1 \leqslant n \leqslant t} \frac{a_n}{n}, U=\min _{1 \leqslant n \leqslant t} \frac{a_n+1}{n}$, 若能证明: $L<U$, 则 式\ref{eq1} 成立 (因为
$\left.\bigcap_{n=1}^t I_n=[L, U)\right)$. 而为证 $L<U$, 只需证明 : 对任意 $m 、 n \in\{1,2, \cdots, t\}$, 都有 $\frac{a_n}{n}<\frac{a_m+1}{m}$, 即
$$
m a_n<n\left(a_m+1\right) . \label{eq2}
$$
下面通过对 $m+n$ 归纳来证明式\ref{eq2}成立.
当 $n+m=2$ 时,有 $m=n=1$, 此时, 式\ref{eq2} 显然成立.
设 式\ref{eq2} 对所有满足 $n+ m \leqslant k$ 的正整数对 $(m, n)$ 成立, 则当 $n+m=k+1$ 时, 如果 $m=n$, 那么 式\ref{eq2} 显然成立; 如果 $m>n$, 那么由归纳假设知 $(m-n) a_n<n\left(a_{m \rightarrow n}+1\right)$, 而由条件中 $a_i+a_j \leqslant a_{i+j}$ 知 $n\left(a_{m \rightarrow n}+a_n\right) \leqslant n a_m$, 从而 $m a_n<n\left(a_m+1\right)$, 即 式\ref{eq2} 成立; 如果 $m<n$, 那么由归纳假设知 $m a_{n-m}<(n-m)\left(a_m+1\right)$, 而由条件中 $a_{i+j} \leqslant a_i+ a_j+1$ 知 $m a_n \leqslant m\left(a_m+a_{n-m}+1\right)$, 所以 $m a_n<n\left(a_m+1\right)$, 即 式\ref{eq2} 成立.
综上可知,对任意 $m 、 n \in\{1,2, \cdots, t\}$, 式\ref{eq2} 都成立.
说明一般地, 对任意 $x \in \mathbf{R}$ 及 $i 、 j \in \mathbf{N}^*$, 都有 $[i x]+[j x] \leqslant[(i+ j) x] \leqslant[i x]+[j x]+1$. 这是关于取整函数的一个性质, 反过来讨论就得到了本题的结论.
注意, 此题的结论只是对任意有限项成立, 并不能对具有题给性质的无穷数列 $a_1, a_2, \cdots$ 找到符合要求的 $x$.
从处理手法上来讲,这里采用对 $m+n$ 来归纳的策略是方便而且适当的.
%%PROBLEM_END%%



%%PROBLEM_BEGIN%%
%%<PROBLEM>%%
例4. 设 $m 、 n$ 为不同的正整数,一个由整数组成的数列满足: 其任意连续 $m$ 项之和为负数, 而任意连续 $n$ 项之和为正数.
问: 这个整数数列最多有多少项?
%%<SOLUTION>%%
解:设 $(m, n)=d, m=m_1 d, n=n_1 d$, 则 $\left(m_1, n_1\right)=1$. 并设数列 $a_1, \cdots, a_t$ 满足条件.
记 $A_i=a_{(i-1) d+1}+\cdots+a_{i d}, i=1,2, \cdots$.
一方面,若 $t \geqslant\left(m_1+n_1-1\right) d$, 我们考虑下述数表
$$
\begin{aligned}
& A_1, A_2, \cdots, A_{m_1} ; \\
& A_2, A_3, \cdots, A_{m_1+1} ; \\
& \cdots \ldots . . . \cdots \cdots \cdots . . . . \\
& A_{n_1}, A_{n_1+1}, \cdots, A_{n_1+m_1-1} .
\end{aligned}
$$
由题设可知, 上表中每行中各数之和为负数, 每列之和为正数, 这样整个表格中所有数之和, 按行计算为负数, 而按列计算又要为正数, 矛盾.
所以, 应有 $t \leqslant\left(m_1+n_1-1\right) d-1$, 即 $t \leqslant m+n-(m, n)-1$.
另一方面, 我们证明: 存在一个长为 $m+n-(m, n)-1$ 的符合要求的整数数列.
为此, 我们需要证明下述命题.
命题设 $d \in \mathbf{N}^*, s 、 t$ 为两个不同的正整数, $(s, t)=1$, 则存在长为 $(s+ t-1) d-1$ 项的有理数数列, 它的任意连续 $s d$ 项之和为负数, 而任意 $t d$ 项之和为正数 (注意: 将此命题中的数列, 每个数都乘以这个数列的所有数的公分母就可以得到符合题意的整数数列).
对 $\max \{s, t\}=r$ 运用数学归纳法.
当 $\max \{s, t\}=2$ 时,不妨设 $s=2, t=1$ (否则, 将所取数列每个数都乘以 -1 即可), 这时, 任取 $2 d-1$ 个正有理数,可得满足命题的数列.
设命题对 $\max \{s, t\}<r(r \geqslant 3)$ 成立, 考虑 $\max \{s, t\}=r$ 的情形.
不妨设 $s>t$, 注意到 $(s-t, t)=1$, 于是, 由归纳假设知, 存在有理数数列 $b_1, b_2, \cdots, b_{(s-1) d-1}$, 使得其中任意连续 $(s-t) d$ 项之和为负数, 连续 $t d$ 项之和为正数.
我们证明: 存在有理数 $a_1, a_2, \cdots, a_{t d}$, 使得下述不等式组成立.
$$
\left\{\begin{array}{l}a_{d+1}+\cdots+a_1+b_1+\cdots+b_{(s-1) d-1}<0, \\ a_{d+2}+\cdots+a_1+b_1+\cdots+b_{(s-1) d-2}<0, \\ \cdots \cdots \cdots \cdots \cdots \cdots \cdots \cdots \cdots \cdots \cdots \cdots \cdots \cdots \cdots \cdots \cdots \cdots \cdots \cdots \cdots \cdots \cdots \cdots \cdots \cdots \cdots \cdots \cdots \cdots \cdots \cdots \cdots \\ a_{t d}+\cdots+a_1+b_1+\cdots+b_{(s-t) d}<0 .\end{array}\right. \label{eq1}
$$
而且
$$
\left\{\begin{array}{l}
a_{t d}+\cdots+a_1>0, \\
a_{t d-1}+\cdots+a_1+b_1>0, \\
\cdots \cdots \cdots \cdots \cdots \cdots \cdots \cdots \cdots \cdots \cdots \cdots \cdots \cdots \cdots \cdots \cdots \cdots \\
a_1+b_1+\cdots+b_{t d-1}>0 .
\end{array}\right. \label{eq2}
$$
这样, 数列 $a_{t d}, a_{t d-1}, \cdots, a_1, b_1, \cdots, b_{(s-1) d-1}$ 是一个满足命题的长为 $(s+t-$ 1) $d-1$ 的数列, 从而命题获证.
事实上,要不等式组 式\ref{eq1}、\ref{eq2} 同时成立, 我们只需分别选取有理数 $a_1$, $a_2, \cdots, a_d$, 使得 $a_1>-\left(b_1+\cdots+b_{t d-1}\right), a_2>-\left(a_1+b_1+\cdots+b_{t d-2}\right), \cdots$, $a_d>-\left(a_{d-1}+\cdots+a_1+b_1+\cdots+b_{(t-1) d}\right)$, 再取 $a_{d+1}$, 使得 $a_{d+1}$ 为有理数, 且
$$
\begin{gathered}
-\left(a_d+\cdots+a_1+b_1+\cdots+b_{(t-1) d-1}\right) \\
<a_{d+1}<-\left(a_d+\cdots+a_1+b_1+\cdots+b_{(s-1) d-1}\right) .
\end{gathered} \label{eq3}
$$
注意式\ref{eq3}的右边减去左边 $=-\left(b_{(t-1) d}+\cdots+b_{(s-1) d-1}\right)>0$ (这里用到归纳假设), 故满足条件的 $a_{d+1}$ 存在.
依此类推, 可知满足不等式组 式\ref{eq1} 与 \ref{eq2} 的有理数 $a_1, \cdots, a_{t d}$ 存在.
回到原题, 满足条件的整数数列最多有 $m+n-(m, n)-1$ 项.
说明此题中 $m=11, n=6$ 的情形曾作为竞赛题出现过, 由于 $m==11$,
$n=6$ 时的例子容易得到, 而对一般的 $m 、 n$ 例子却非常难找到, 因此当此例在 2000 年国家集训队的测验中出现时,做出的同学非常少.
%%PROBLEM_END%%


