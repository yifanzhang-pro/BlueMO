
%%PROBLEM_BEGIN%%
%%<PROBLEM>%%
问题1 在 $\triangle A B C$ 中, $a 、 b 、 c$ 三边成等差数列, 则 ( ).
(A) $0<B<60^{\circ}$
(B) $0<B \leqslant 60^{\circ}$
(C) $30^{\circ}<B \leqslant 60^{\circ}$
(D) $45^{\circ}<B \leqslant 60^{\circ}$
%%<SOLUTION>%%
B. 由条件得 $2 b=a+c$, 从而 $\cos B=\frac{a^2+c^2-b^2}{2 a c}=$
$$
\begin{aligned}
& \frac{a^2+c^2-\left[\frac{1}{2}(a+c)\right]^2}{2 a c}=\frac{3 a^2+3 c^2-2 a c}{8 a c}=\frac{3(a-c)^2+4 a c}{8 a c} \geqslant \frac{1}{2} \text {, 故 } \\
& 0<B \leqslant 60^{\circ} .
\end{aligned}
$$
%%PROBLEM_END%%



%%PROBLEM_BEGIN%%
%%<PROBLEM>%%
问题2. 在 $\triangle A B C$ 中, 下列表达式: (1) $\sin (A+B)+\sin C$; (2) $\cos (A+B)+\cos C$;
(3) $\tan \frac{A+B}{2} \tan \frac{C}{2}$; (4) $\tan \frac{A+B}{2}+\tan \frac{C}{2}$ 中, 表示常数的是 ( ).
(A) (1)(2)
(B) (1)(3)
(C) (2)(3)
(D) (1)(4)
%%<SOLUTION>%%
C. 在 $\triangle A B C$ 中, $A+B+C=\pi$, 故 $\cos (A+B)+\cos C=0, \tan \frac{A+B}{2}$. $\tan \frac{C}{2}=\tan \frac{\pi-C}{2} \cdot \tan \frac{C}{2}=\cot \frac{C}{2} \cdot \tan \frac{C}{2}=1$, 选 C.
%%PROBLEM_END%%



%%PROBLEM_BEGIN%%
%%<PROBLEM>%%
问题3 在 $\triangle A B C$ 中, $A=60^{\circ}, a=\sqrt{6}, b=4$, 则满足条件的三角形有( ).
(A) 一解
(B) 两解
(C) 无解
(D) 不能确定
%%<SOLUTION>%%
C. 由正弦定理, 得 $\frac{a}{\sin A}=\frac{b}{\sin B}$, 于是 $\sin B=\frac{b \sin A}{a}=\frac{4 \sin 60^{\circ}}{\sqrt{6}}= \sqrt{2}>1$, 故无解.
%%PROBLEM_END%%



%%PROBLEM_BEGIN%%
%%<PROBLEM>%%
问题4 在 $\triangle A B C$ 中, 若 $\sin A-2 \sin B \cos C=0$, 则其形状为().
(A) 直角三角形
(B) 等腰三角形
(C) 等腰直角三角形
(D) 等边三角形
%%<SOLUTION>%%
B. 由 $\sin A=2 \sin B \cos C$, 得 $a=2 b \cdot \frac{a^2+b^2-c^2}{2 b a}$, 从而 $b=c, \triangle A B C$ 是等腰三角形.
%%PROBLEM_END%%



%%PROBLEM_BEGIN%%
%%<PROBLEM>%%
问题5 在直角三角形中, $\angle C=90^{\circ}$, 则 $\sin A \cos ^2\left(45^{\circ}-\frac{B}{2}\right)-\sin \frac{A}{2} \cos \frac{A}{2}(\quad)$.
(A) 有最大值 $\frac{1}{4}$ 和最小值 0
(B) 有最大值 $\frac{1}{4}$ 但无最小值
(C) 既无最大值又无最小值
(D) 有最大值 $\frac{1}{2}$ 但无最小值
%%<SOLUTION>%%
B. 原式 $=\sin A \cdot \frac{1+\cos \left(90^{\circ}-B\right)}{2}-\frac{1}{2} \sin A=\frac{1}{2} \sin A \cdot \sin B= \frac{1}{2} \sin A \cos A=\frac{1}{4} \sin 2 A \leqslant \frac{1}{4}$.
%%PROBLEM_END%%



%%PROBLEM_BEGIN%%
%%<PROBLEM>%%
问题6. 已知锐角 $\triangle A B C$ 的边长分别为 $2 、 3 、 x$, 则第三边 $x$ 适合的条件是 ( ).
(A) $1<x<5$
(B) $\sqrt{5}<x<\sqrt{13}$
(C) $\sqrt{13}<x<5$
(D) $1<x<\sqrt{5}$
%%<SOLUTION>%%
B. 由条件得 $2^2+3^2>x^2$, 且 $2^2+x^2>3^2$, 且 $1<x<5$, 解之得 $\sqrt{5}< x<\sqrt{13}$.
%%PROBLEM_END%%



%%PROBLEM_BEGIN%%
%%<PROBLEM>%%
问题7. $R$ 是 $\triangle A B C$ 的外接圆半径, 若 $a b<4 R^2 \cos A \cos B$, 则 $\triangle A B C$ 的外心位于( ).
(A) 三角形的外部
(B) 三角形的边上
(C) 三角形的内部
(D) 无法判断
%%<SOLUTION>%%
A. 由 $a b<4 R^2 \cos A \cos B$ 得 $\sin A \sin B<\cos A \cos B$, 即 $\cos (A+ B)>0$, 所以 $A+B<\frac{\pi}{2}, \triangle A B C$ 为钝角三角形,其外心在三角形外部.
%%PROBLEM_END%%



%%PROBLEM_BEGIN%%
%%<PROBLEM>%%
问题8. 已知 $\triangle A B C$ 的三边 $a, b, c$ 成等比数列.
$a 、 b 、 c$ 所对的角依次为 $\angle A$ 、 $\angle B 、 \angle C$, 则 $\sin B+\cos B$ 的取值范围是 ( ).
(A) $\left(1,1+\frac{\sqrt{3}}{2}\right]$
(B) $\left[\frac{1}{2}, 1+\frac{\sqrt{3}}{2}\right]$
(C) $(1, \sqrt{2}]$
(D) $\left[\frac{1}{2}, \sqrt{2}\right]$
%%<SOLUTION>%%
C. 由 $a c=b^2=a^2+c^2-2 a c \cos B \geqslant 2 a c-2 a c \cos B \Rightarrow \cos B \geqslant \frac{1}{2} \Rightarrow 0<B \leqslant \frac{\pi}{3}$. 故 $\sin B+\cos B=\sqrt{2} \sin \left(B+\frac{\pi}{4}\right) \in(1, \sqrt{2}]$.
%%PROBLEM_END%%



%%PROBLEM_BEGIN%%
%%<PROBLEM>%%
问题9 已知向量 $\overrightarrow{O P}=\left(2 \cos \left(\frac{\pi}{2}+x\right),-1\right), \overrightarrow{O Q}=\left(-\sin \left(\frac{\pi}{2}-x\right), \cos 2 x\right)$, $f(x)=\overrightarrow{O P} \cdot \overrightarrow{O Q}$. 若 $a 、 b 、 c$ 分别是锐角 $\triangle A B C$ 中 $\angle A 、 \angle B 、 \angle C$ 的对边, 且满足 $f(A)=1, b+c=5+3 \sqrt{2}, a=\sqrt{13}$, 则 $\triangle A B C$ 的面积 $S=$
(A) $15 \sqrt{2}$
(B) 15
(C) $\frac{15 \sqrt{2}}{2}$
(D) $\frac{15}{2}$
%%<SOLUTION>%%
D. 由条件知 $f(x)=\overrightarrow{O P} \cdot \overrightarrow{O Q}=-2 \cos \left(\frac{\pi}{2}+x\right) \sin \left(\frac{\pi}{2}-x\right)- \cos 2 x=2 \sin x \cos x-\cos 2 x=\sqrt{2} \sin \left(2 x-\frac{\pi}{4}\right) \Rightarrow \sqrt{2} \sin \left(2 A-\frac{\pi}{4}\right)=1 \Rightarrow \sin \left(2 A-\frac{\pi}{4}\right)=\frac{\sqrt{2}}{2}$, 又 $\angle A$ 为锐角 $\Rightarrow-\frac{\pi}{4}<2 \angle A-\frac{\pi}{4}<\frac{3 \pi}{4} \Rightarrow 2 \angle A-\frac{\pi}{4}= \frac{\pi}{4} \Rightarrow \angle A=\frac{\pi}{4}$. 由 $b+c=5+3 \sqrt{2}, a=\sqrt{13} \Rightarrow 13=b^2+c^2-2 b c \cos A=(b+ c)^2-2 b c-2 b c \cos A \Rightarrow 13=43+30 \sqrt{2}-(2+\sqrt{2}) b c \Rightarrow b c=15 \sqrt{2}$, 所以 $\triangle A B C$ 的面积 $S=\frac{1}{2} b c \sin A=\frac{1}{2} \times 15 \sqrt{2} \times \frac{\sqrt{2}}{2}=\frac{15}{2}$.
%%PROBLEM_END%%



%%PROBLEM_BEGIN%%
%%<PROBLEM>%%
问题10 设 $\theta$ 是三角形中的最小内角, 且 $a \cos ^2 \frac{\theta}{2}+\sin ^2 \frac{\theta}{2}-\cos ^2 \frac{\theta}{2}-a \sin ^2 \frac{\theta}{2}= a+1$, 则 $a$ 适合的条件是 ( ).
(A) $a<-1$
(B) $a<-3$
(C) $a \leqslant-3$
(D) $-3 \leqslant a<-1$
%%<SOLUTION>%%
C. 由原等式可知 $a \cos \theta-\cos \theta=a+1$, 所以 $\cos \theta=\frac{a+1}{a-1}$, 因 $\theta$ 为 $\triangle A B C$ 的最小内角, 故 $0<\theta \leqslant 60^{\circ}$, 所以 $\frac{1}{2} \leqslant \cos \theta<1$, 即 $\frac{1}{2} \leqslant \frac{a+1}{a-1}<1$, 从而 $a \leqslant-3$.
%%PROBLEM_END%%



%%PROBLEM_BEGIN%%
%%<PROBLEM>%%
问题11 在 $\triangle A B C$ 中, 已知 $\sin \frac{B}{2}=\sin \frac{A}{2} \sin \frac{C}{2}$, 则 $\tan \frac{A}{2} \tan \frac{C}{2}=$
%%<SOLUTION>%%
$\frac{1}{2}$. 因 $\sin \frac{B}{2}=\sin \frac{A}{2} \sin \frac{C}{2}$, 故 $\cos \frac{A+C}{2}=\sin \frac{A}{2} \sin \frac{C}{2}$, 即 $\cos \frac{A}{2} \cos \frac{C}{2}=2 \sin \frac{A}{2} \sin \frac{C}{2}$, 得 $\tan \frac{A}{2} \tan \frac{C}{2}=\frac{1}{2}$.
%%PROBLEM_END%%



%%PROBLEM_BEGIN%%
%%<PROBLEM>%%
问题12 一个直角三角形三内角的正弦值成等比数列, 则其最小内角 $=$
%%<SOLUTION>%%
$\arcsin \frac{\sqrt{5}-1}{2}$. 由条件得 $\sin ^2 A=\sin B$, 且 $A+B=\frac{\pi}{2}$. 从而 $1- \cos ^2 A=\cos A$, 解之得 $\cos A=\frac{\sqrt{5}-1}{2}$, 即 $\sin B=\frac{\sqrt{5}-1}{2}$. 而 $B<A<90^{\circ}$, 故最小内角为 $\arcsin \frac{\sqrt{5}-1}{2}$.
%%PROBLEM_END%%



%%PROBLEM_BEGIN%%
%%<PROBLEM>%%
问题13 一个三角形三内角 $A 、 B 、 C$ 成等差数列, 则 $\cos ^2 A+\cos ^2 C$ 的最小值是
%%<SOLUTION>%%
$\frac{1}{2}$. 由条件得 $B=\frac{\pi}{3}, A+C=\frac{2 \pi}{3}$, 于是 $\cos ^2 A+\cos ^2 C=\cos ^2 A+ \cos ^2\left(\frac{2 \pi}{3}-A\right)=\cos ^2 A+\frac{1}{4} \cos ^2 A-\frac{\sqrt{3}}{2} \sin A \cos A+\frac{3}{4} \sin ^2 A=\frac{3}{4}+\frac{1}{2} \cos ^2 A- \frac{\sqrt{3}}{4} \sin 2 A=\frac{3}{4}+\frac{1}{4}(\cos 2 A+1)-\frac{\sqrt{3}}{4} \sin 2 A=1-\frac{1}{2} \sin \left(2 A-\frac{\pi}{6}\right)$, 因 $0< A<\frac{2 \pi}{3}$, 所以当 $A=\frac{\pi}{3}$ 时有最小值 $\frac{1}{2}$.
%%PROBLEM_END%%



%%PROBLEM_BEGIN%%
%%<PROBLEM>%%
问题14. 在 $\triangle A B C$ 中, $3 \sin A+4 \cos B=6,4 \sin B+3 \cos A=1$, 则角 $C=$
%%<SOLUTION>%%
$30^{\circ}$. 两等式平方相加得 $9+24(\sin A \cos B+\cos A \sin B)+16==37$, 所以 $\sin (A+B)=\frac{1}{2}, A+B=30^{\circ}$ 或 $150^{\circ}$, 但当 $A+B=30^{\circ}$ 时, $\sin A<\frac{1}{2}$, $\cos B<1$, 于是 $3 \sin A+4 \cos B<6$, 所以舍去.
从而 $A+B=150^{\circ}$, 得 $C=30^{\circ}$.
%%PROBLEM_END%%



%%PROBLEM_BEGIN%%
%%<PROBLEM>%%
问题15 在 $\triangle A B C$ 中, 已知 $2 \lg \left(a^2+b^2-c^2\right)=\lg 2+2 \lg a+2 \lg b$, 则角 $C=$
%%<SOLUTION>%%
$45^{\circ}$. 由条件得 $\left(a^2+b^2-c^2\right)^2=2 a^2 b^2$, 所以 $a^2+b^2-c^2=\sqrt{2} a b$, $\cos C=\frac{\sqrt{2}}{2}$, 故 $C=45^{\circ}$.
%%PROBLEM_END%%



%%PROBLEM_BEGIN%%
%%<PROBLEM>%%
问题16. 在 $\triangle A B C$ 中, $a=4, b=6, S_{\triangle}=6 \sqrt{2}$, 则角 $C=$
%%<SOLUTION>%%
$45^{\circ}$ 或 $135^{\circ}$. 由 $S_{\Delta}=\frac{1}{2} a b \sin C$, 得 $\frac{1}{2} \times 4 \times 6 \sin C=6 \sqrt{2}, \sin C=\frac{\sqrt{2}}{2}$, 故 $C=45^{\circ}$ 或 $135^{\circ}$.
%%PROBLEM_END%%



%%PROBLEM_BEGIN%%
%%<PROBLEM>%%
问题17 在 $\triangle A B C$ 中, 若 $\tan A=\frac{1}{2}, \tan B=\frac{1}{3}$, 且最长的边的长为 1 , 则最短的边的长等于
%%<SOLUTION>%%
$\frac{\sqrt{5}}{5}$. 易知 $A C$ 最短.
设 $C D$ 为 $A B$ 边上的高, 长为 $x$, 由 $\tan A=\frac{1}{2}$, $\tan B=\frac{1}{3}$, 可得 $A D=2 x, B D=3 x, A B=5 x=1, x=\frac{1}{5}, A C=\sqrt{5} x=\frac{\sqrt{5}}{5}$
%%PROBLEM_END%%



%%PROBLEM_BEGIN%%
%%<PROBLEM>%%
问题18. $\triangle A B C$ 的外接圆半径为 $2 \sqrt{3}$, 则 $\frac{a+b+c}{\sin A+\sin B+\sin C}=$
%%<SOLUTION>%%
$4 \sqrt{3}$. 由正弦定理得 $\frac{a}{\sin A}=\frac{b}{\sin B}=\frac{c}{\sin C}=\frac{a+b+c}{\sin A+\sin B+\sin C}= 4 \sqrt{3}$.
%%PROBLEM_END%%



%%PROBLEM_BEGIN%%
%%<PROBLEM>%%
问题19 三角形的两边分别为 $3 \mathrm{~cm} 、 5 \mathrm{~cm}$, 它们的夹角的余弦为方程 $5 x^2-7 x- 6=0$ 的根, 则这个三角形的面积为
%%<SOLUTION>%%
$6 \mathrm{~cm}^2$. 方程 $5 x^2-7 x-6=0$ 的根为 $x_1=\frac{-3}{5}, x_2=2$, 从而 $\cos \theta= -\frac{3}{5}, \sin \theta=\frac{4}{5}, S_{\triangle}=\frac{1}{2} \times 3 \times 5 \times \frac{4}{5}=6\left(\mathrm{~cm}^2\right)$.
%%PROBLEM_END%%



%%PROBLEM_BEGIN%%
%%<PROBLEM>%%
问题20. 在 $\triangle A B C$ 中, $B=60^{\circ}$, 面积为 $10 \sqrt{3}$, 外接圆半径为 $\frac{7 \sqrt{3}}{3}$, 则各边长为
%%<SOLUTION>%%
5、7、8. 由条件得 $\left\{\begin{array}{l}a^2+c^2-2 a c \cos 60^{\circ}=b^2, \\ \frac{1}{2} a c \sin 60^{\circ}=10 \sqrt{3}, \\ b=2 \times \frac{7 \sqrt{3}}{3} \sin 60^{\circ} .\end{array}\right.$ 解得 $b=7, a=5, c=8$.
%%PROBLEM_END%%



%%PROBLEM_BEGIN%%
%%<PROBLEM>%%
问题21 锐角 $\triangle A B C$ 中, 若 $\cos ^2 A 、 \cos ^2 B 、 \cos ^2 C$ 的和等于 $\sin ^2 A 、 \sin ^2 B 、 \sin ^2 C$ 中的某个值.
证明: $\tan A 、 \tan B 、 \tan C$ 必可按某顺序组成一个等差数列.
%%<SOLUTION>%%
设 $\cos ^2 A+\cos ^2 B+\cos ^2 C=\sin ^2 B \cdots$ (1). 据余弦定理: $\sin ^2 B= \sin ^2 A+\sin ^2 C-2 \sin A \sin C \cos B \cdots(2) \cdot \sin A \sin C=\cos A \cos C-\cos (A+ C)=\cos A \cos C+\cos B \cdots$ (3). 由 (2)、(3) 得, $\cos ^2 A+\cos ^2 B+\cos ^2 C=1- 2 \cos A \cos B \cos C \cdots$ (4). 由 (1)、(4) 得, $2 \cos A \cos B \cos C=1-\sin ^2 B=\cos ^2 B$, $B$ 为锐角.
则 $2 \cos A \cos C=\cos B=-\cos (A+C)=\sin A \sin C-\cos A \cos C$,
所以 $\tan A \tan C=3$, 所以 $\tan B=-\tan (A+C)=\frac{\tan A+\tan C}{\tan A \tan C-1}= \frac{\tan A+\tan C}{2}$, 因此 $\tan A 、 \tan B 、 \tan C$ 成等差数列.
%%PROBLEM_END%%



%%PROBLEM_BEGIN%%
%%<PROBLEM>%%
问题22 在锐角 $\triangle A B C$ 中, 求证: $\cos (B-C) \cos (C-A) \cos (A-B) \geqslant 8 \cos A \cos B \cos C$.
%%<SOLUTION>%%
因为 $\tan A \tan B \tan C=\tan A+\tan B+\tan C$, 故
$$
\begin{aligned}
\frac{\cos (B-C)}{\cos A} & =\frac{\cos (B-C)}{-\cos (B+C)}=\frac{\sin B \sin C+\cos B \cos C}{\sin B \sin C-\cos B \cos C}=\frac{\tan B \tan C+1}{\tan B \tan C-1} \\
& =\frac{\tan A \tan B \tan C+\tan A}{\tan A \tan B \tan C-\tan A}=\frac{2 \tan A+\tan B+\tan C}{\tan B+\tan C} \\
& \geqslant \frac{2 \sqrt{(\tan A+\tan B)(\tan C+\tan A)}}{\tan B+\tan C},
\end{aligned}
$$
同理 $\frac{\cos (C-A)}{\cos B} \geqslant \frac{2 \sqrt{(\tan B+\tan C)(\tan A+\tan B)}}{\tan C+\tan A}$, $\frac{\cos (A-B)}{\cos C} \geqslant \frac{2 \sqrt{(\tan C+\tan A)(\tan B+\tan C)}}{\tan A+\tan B}$, 以上三式相乘, 即得 $\frac{\cos (B-C) \cos (C-A) \cos (A-B)}{\cdot \cos A \cos B \cos C} \geqslant 8$, 当且仅当 $A=B=C=\frac{\pi}{3}$ 时等号成立, 原题得证.
%%PROBLEM_END%%



%%PROBLEM_BEGIN%%
%%<PROBLEM>%%
问题23 $ \triangle \triangle A B C$ 的三条边长为 $a 、 b 、 c$, 证明:
$$
\frac{\left|a^2-b^2\right|}{c}+\frac{\left|b^2-c^2\right|}{a} \geqslant \frac{\left|c^2-a^2\right|}{b} .
$$
%%<SOLUTION>%%
由于 $a=2 R \sin A, b=2 R \sin B, c=2 R \sin C$, 只要证 $\frac{\left|\sin ^2 A-\sin ^2 B\right|}{\sin C}+\frac{\left|\sin ^2 B-\sin ^2 C\right|}{\sin A} \geqslant \frac{\left|\sin ^2 C-\sin ^2 A\right|}{\sin B}$... (1). 因为 $\sin ^2 A- \sin ^2 B=\sin (A+B) \sin (A-B)=\sin C \sin (A-B)$, 故由 (1), 只要证 $\mid \sin (A-$ B) $|+| \sin (B-C)|\geqslant| \sin (C-A) \mid \cdots$ (2). $|\sin (C-A)|=\mid \sin [(A-B)+ (B-C)]|=| \sin (A-B) \cos (B-C)+\cos (A-B) \sin (B-C) \mid \leqslant |\sin (A-B) \cos (B-C)|+|\cos (A-B) \sin (B-C)| \leqslant|\sin (A-B)|+ |\sin (B-C)|$, 当且仅当 $A=B=C=\frac{\pi}{3}$ 时取等号, 此时 $\triangle A B C$ 为正三角形, 即 $a=b=c$.
%%PROBLEM_END%%



%%PROBLEM_BEGIN%%
%%<PROBLEM>%%
问题24 知 $\triangle A B C$ 的外接圆、内切圆半径分别为 $R$ 和 $r$, 求证:
$$
\frac{\cos A}{\sin ^2 A}+\frac{\cos B}{\sin ^2 B}+\frac{\cos C}{\sin ^2 C} \geqslant \frac{R}{r} \text {. }
$$
%%<SOLUTION>%%
设 $a 、 b 、 c$ 为 $\triangle A B C$ 的三边长, 则 $\frac{b^2+c^2}{a}+2 a \geqslant \frac{(b+c)^2}{2 a}+2 a \geqslant 2(b+c)$, 故 $\frac{b^2+c^2}{a} \geqslant 2(b+c-a)$, 同理可得 $\frac{c^2+a^2}{b} \geqslant 2(a+c-b)$, $\frac{a^2+b^2}{c} \geqslant 2(a+b-c)$, 以上三式相加得 $\frac{b^2+c^2}{a}+\frac{c^2+a^2}{b}+\frac{a^2+b^2}{c} \geqslant 2(a+ b+c)$. 根据正、余弦定理及 $a b c=2 R r(a+b+c)$, 并利用上式可得 $\frac{\cos A}{\sin ^2 A}+ \frac{\cos B}{\sin ^2 B}+\frac{\cos C}{\sin ^2 C}=\frac{2 R^2}{a b c}\left(\frac{a^2+b^2}{c}+\frac{b^2+c^2}{a}+\frac{c^2+a^2}{b}-a-b-c\right) \geqslant \frac{2 R^2(a+b+c)}{a b c}=\frac{R}{r}$, 当且仅当 $a=b=c$ 时等号成立.
%%PROBLEM_END%%



%%PROBLEM_BEGIN%%
%%<PROBLEM>%%
问题25 在 $\triangle A B C$ 中, 实数 $x$ 满足 $\sec ^2 x=\csc ^2 A+\csc ^2 B+\csc ^2 C$, 求证: $\cos [x+ \left.(-1)^n A\right] \cdot \cos \left[x+(-1)^n B\right] \cdot \cos \left[x+(-1)^n C\right]+\cos ^3 x=0(n \in \mathbf{N})$.
%%<SOLUTION>%%
因为 $\cot A \cot B+\cot B \cot C+\cot C \cdot \cot A=1$, 因为 $\sec ^2 x= \csc ^2 A+\csc ^2 B+\csc ^2 C$, 所以 $\tan ^2 x=\cot ^2 A+\cot ^2 B+\cot ^2 C+2=\cot ^2 A+ \cot ^2 B+\cot ^2 C+2(\cot A \cot B+\cot B \cot C+\cot C \cot A)=(\cot A+\cot B+ \cot C)^2$, 所以 $\tan x= \pm(\cot A+\cot B+\cot C)$. 当 $\tan x==\cot A+\cot B+ \cot C$ 时, $(\tan x-\cot A)(\tan x-\cot B)(\tan x-\cot C)=(\cot B+\cot C) (\cot A+\cot C)(\cot B+\cot A)=\frac{\sin (B+C)}{\sin B \sin C} \cdot \frac{\sin (A+C)}{\sin A \sin C} \cdot \frac{\sin (A+B)}{\sin A \sin B}= \frac{1}{\sin A \sin B \sin C}$, 两边乘以 $\cos ^3 x \sin A \sin B \sin C$ 得, $(\sin x \sin A-\cos A \cos x) (\sin x \sin B-\cos B \cos x)(\sin x \sin C-\cos x \cos C)=\cos ^3 x \Rightarrow \cos (x+A) \cos (x+B) \cos (x+C)+\cos ^3 x=0$, 当 $\tan x=-(\cot A+\cot B+\cot C)$ 时, $(\tan x+\cot A)(\tan x+\cot B)(\tan x+\cot C)=-(\cot B+\cot C)(\cot C+ \cot A)(\cot A+\cot B)=-\frac{\sin (B+C)}{\sin B \sin C} \cdot \frac{\sin (C+A)}{\sin C \sin A} \cdot \frac{\sin (A+B)}{\sin A \sin B}= -\frac{1}{\sin A \sin B \sin C}$, 两边乘以 $\cos ^3 x \sin A \sin B \sin C$ 得 $(\sin x \sin A+\cos A \cos x)(\sin x \sin B+\cos x \cos B)(\sin x \sin C+\cos x \cos C)=-\cos ^3 x \Rightarrow \cos (x- A) \cos (x-B) \cos (x-C)+\cos ^3 x=0$, 综上得证.
%%PROBLEM_END%%



%%PROBLEM_BEGIN%%
%%<PROBLEM>%%
问题26 设 $\alpha 、 \beta 、 \gamma 、 \varphi$ 为某四边形的四个内角, 设 $n(n \geqslant 3)$ 为奇数, 若 $\alpha 、 \beta 、 \gamma 、 \varphi$ 满足 $\sin \alpha+\sin \beta+\sin \gamma+\sin \varphi=0$. 求证: 它们之中必有两个角之和在集合 $\left\{\frac{2 \pi}{n}, \frac{4 \pi}{n}, \cdots, \frac{(n-1) \pi}{n}\right\}$ 中.
%%<SOLUTION>%%
因为 $\alpha+\beta+\gamma+\varphi=2 \pi$, 从而有 $\frac{n \alpha+n \beta}{2}+\frac{n \gamma+n \varphi}{2}=n \pi$, 又 $n$ 为奇数, 所以 $\sin \frac{n \alpha+n \beta}{2}=\sin \frac{n \gamma+n \varphi}{2}, \sin n \alpha+\sin n \beta+\sin n \gamma+\sin n \varphi=0 \Leftrightarrow 2 \sin \frac{n \alpha+n \beta}{2} \cos \frac{n \alpha-n \beta}{2}+2 \sin \frac{n \gamma+n \varphi}{2} \cos \frac{n \gamma-n \varphi}{2}=0 \Leftrightarrow 2 \sin \frac{n \alpha+n \beta}{2} \cos \frac{n \alpha-n \beta}{2}+ 2 \sin \frac{n \alpha+n \beta}{2} \cos \frac{m \gamma-n \varphi}{2}=0 \Leftrightarrow 2 \sin \frac{n \alpha+n \beta}{2}=0$ 或 $\cos \frac{n \alpha-n \beta}{2}+ \cos \frac{n \gamma-n \varphi}{2}=0$. (1) 若 $\sin \frac{n \alpha+n \beta}{2}=0$, 用例 5(1) 即 $\alpha+\beta=\frac{2 \pi}{n}, \frac{4 \pi}{n}, \cdots$, $\frac{2(n-1)}{n} \pi$. (2) 若 $\cos \frac{n \alpha-n \beta}{2}+\cos \frac{n \gamma-n \varphi}{2}=0$, 则 $\frac{n \alpha-n \beta}{2}+\frac{n \gamma-n \varphi}{2}= 2 k \pi+\pi$ 或 $\frac{n \alpha-n \beta}{2}=\frac{n \gamma-n \varphi}{2}+\pi+2 k \pi(k \in \mathbf{Z})$, 由于 $\alpha+\beta+\gamma+\varphi=2 \pi$, 故 $\alpha+\gamma=\pi+\frac{(2 k+1)}{n} \pi$ 或 $\alpha+\varphi=\pi+\frac{(2 k+1)}{n} \pi(k \in \mathbf{Z})$, 当 $\alpha+\gamma=\pi+ \frac{(2 k+1)}{n} \pi(k \in \mathbf{Z})$ 时, 由 $0<\alpha+\gamma<2 \pi$, 则有 $\alpha+\gamma=\frac{2 \pi}{n}, \frac{4 \pi}{n}, \cdots$, $\frac{2(n-1)}{n} \pi$. 类似地, $\alpha+\varphi=\frac{2 \pi}{n}, \frac{4 \pi}{n}, \cdots, \frac{2(n-1)}{n} \pi$, 综合 (1)、 (2), $\alpha, \beta 、 \gamma 、 \varphi$ 至少存在两个角之和在集合 $\left\{\frac{2 \pi}{n}, \frac{4 \pi}{n}, \cdots, \frac{2(n-1) \pi}{n}\right\}$ 中.
注意到 $n$ 为奇数, $\pi$ 不在上面的集合中, 又因为 $\frac{2 i \pi}{n}+\frac{2(n-i) \pi}{n}=2 \pi$, 结合四边形的内角之和为 $2 \pi$. 因此 $\alpha 、 \beta 、 \gamma 、 \varphi$ 至少存在两个角之和在集合 $\left\{\frac{2 \pi}{n}, \frac{4 \pi}{n}, \cdots, \frac{2(n-1) \pi}{n}\right\}$ 中, 得证.
特别地, 当 $n=3$ 时,必有两角之和为 $\frac{2 \pi}{3}$.
%%PROBLEM_END%%



%%PROBLEM_BEGIN%%
%%<PROBLEM>%%
问题27 在 $\triangle A B C$ 中, 若 $\sin A \cos ^2 \frac{C}{2}+\sin C \cos ^2 \frac{A}{2}=\frac{3}{2} \sin B$, 求证: (1) $a 、 b 、 c$ 成等差数列; $(2) 0<\sin \frac{B}{2} \leqslant \frac{1}{2}$.
%%<SOLUTION>%%
(1) $\sin A \cdot(\cos C+1)+\sin C \cdot(\cos A+1)=3 \sin B$. 即 $\sin (A+ C)+\sin A+\sin C=3 \sin B$. 所以 $\sin A+\sin C=2 \sin B$, 即 $a+c=2 b$, 从而 $a 、 b 、 c$ 成等差数列.
(2) 由 $\sin A+\sin C=2 \sin B$ 得 $2 \sin \frac{A+C}{2} \cos \frac{A-C}{2}= 4 \sin \frac{B}{2} \cos \frac{B}{2}$, 所以 $\sin \frac{B}{2}=\frac{1}{2} \cos \frac{A-C}{2} \leqslant \frac{1}{2}$, 显然 $\sin \frac{B}{2}>0$. 故 $0< \sin \frac{B}{2} \leqslant \frac{1}{2}$.
%%PROBLEM_END%%



%%PROBLEM_BEGIN%%
%%<PROBLEM>%%
问题28 在锐角三角形 $A B C$ 中, 已知 $A<B<C, B=60^{\circ}$, 又 $\sqrt{(1+\cos 2 A)(1+\cos 2 C)}= \frac{\sqrt{3}-1}{2}$, 试比较 $a+\sqrt{2} b$ 与 $2 c$ 的大小, 并证明你的结论.
%%<SOLUTION>%%
因 $\sqrt{(1+\cos 2 A)(1+\cos 2 C)}=\sqrt{2 \cos ^2 A \cdot 2 \cos ^2 C}=2 \cos A \cos C=\cos (A+C)+\cos (A-C)=-\frac{1}{2}+\cos (A-C)=\frac{\sqrt{3}-1}{2}$, 所以 $\cos (C-A)= \frac{\sqrt{3}}{2}$, 故 $C-A=30^{\circ}$. 又 $C+A=120^{\circ}$, 解得 $A=45^{\circ}, B=60^{\circ}, C=75^{\circ}$, 故 $\frac{a+\sqrt{2} b}{2 c}=\frac{\sin A+\sqrt{2} \sin B}{2 \sin C}=\frac{\sin 45^{\circ}+\sqrt{2} \sin 60^{\circ}}{2 \sin 75^{\circ}}=1$, 即 $a+\sqrt{2} b=2 c$.
%%PROBLEM_END%%



%%PROBLEM_BEGIN%%
%%<PROBLEM>%%
问题29 设三角形的三边 $a 、 b 、 c$ 满足 $a>b>c$, 在此三角形的两边上分别取点 $P 、 Q$, 使线段 $P Q$ 把 $\triangle A B C$ 分成面积相等的两部分, 求使 $P Q$ 长度为最短的点 $P 、 Q$ 的位置.
%%<SOLUTION>%%
若 $P 、 Q$ 分别取在 $B C 、 A C$ 上, 设 $C P=x, C Q=y$, 则 $\frac{1}{2} x y \sin C= \frac{1}{2} \cdot \frac{1}{2} a b \sin C, x y=\frac{1}{2} a b$. 由余弦定理, 在 $\triangle C P Q$ 中, $P Q^2=x^2+y^2- 2 x y \cos C=(x-y)^2+2 x y(1-\cos C)=(x-y)^2+a b(1-\cos C)$. 所以, 当 $x=y$ 时, $P Q$ 有最小值, 这时 $P Q=\sqrt{a b-a b \cos C}=\sqrt{a b-\frac{1}{2}\left(a^2+b^2-c^2\right)}= \sqrt{\frac{1}{2}(c+a-b)(c-a+b)}=\sqrt{2(p-a)(p-b)}$. 其中 $p=\frac{1}{2}(a+b+c)$. 若 $P 、 Q$ 分别取在 $A C 、 A B$ 边上, 则当 $A P=A Q$ 时, $P Q$ 有最小值 $\sqrt{2(p-b)(p-c)}$; 若 $P 、 Q$ 分别取在边 $A B 、 B C$ 上, 则当 $B P=B Q$ 时, $P Q$ 有最小值 $\sqrt{2(p-c)(p-a)}$. 但 $a>b>c$, 因此 $\sqrt{2(p-a)(p-b)}$ 最小, 即 $P$ 、 $Q$ 分别取在边 $B C 、 A C$ 上, 且 $C P=C Q$ 时, $P Q$ 长度最短.
因为此时 $P Q= \sqrt{a b(1-\cos C)}=\sqrt{2 a b} \sin \frac{C}{2}, \triangle C P Q$ 是等腰三角形, 所以 $C P=C Q= \frac{\frac{1}{2} P Q}{\sin \frac{C}{2}}=\frac{1}{2} \sqrt{2 a b}$.
%%PROBLEM_END%%



%%PROBLEM_BEGIN%%
%%<PROBLEM>%%
问题30. $\triangle A B C$ 中, 若 $\frac{\cos A}{\sin B}+\frac{\cos B}{\sin A}=2$, 且该三角形的周长为 12 , 求这个三角形面积的最大值.
%%<SOLUTION>%%
由 $\frac{\cos A}{\sin B}+\frac{\cos B}{\sin A}=2$ 得 $\sin A \cos A+\sin B \cos B=2 \sin A \sin B$, 即 $\sin A(\cos A-\sin B)+\sin B(\cos B-\sin A)=0 \cdots$ (1). 也可以变形为 $\frac{1}{2}(\sin 2 A+ \sin 2 B)=2 \sin A \sin B, \sin (A+B) \cos (A-B)=\cos (A-B)-\cos (A+B)$, $\cos (A+B)=\cos (A-B)[1-\sin (A+B)] \cdots$ (2). 若 $A+B<\frac{\pi}{2}$, 则 $A$ 、 $B \in\left(0, \frac{\pi}{2}\right), A<\frac{\pi}{2}-B, \cos A>\sin B, \sin A<\cos B$. 所以 (1) 不成立; 若 $A+B>\frac{\pi}{2}$, 则 $\cos (A+B)<0$, 由 (2) 得 $\cos (A-B)<0$, 且 $\cos (A+B)> \cos (A-B)$, 从而 $A+B<|A-B|$. 这不可能, 所以 (2) 不成立.
综上所述, 可知 $A+B=\frac{\pi}{2}, \triangle A B C$ 是直角三角形.
由 $a+b+c=12$ 得 $c(1+\sin A+ \cos A)=12, c=\frac{12}{1+\sin A+\cos A} . \triangle A B C$ 的面 积 $S=\frac{1}{2} a b= \frac{36\left(t^2-1\right)}{(1+t)^2}=\frac{36(t-1)}{t+1}=36\left(1+\frac{-2}{t+1}\right)$, 且由 $t=\sqrt{2} \sin \left(A+\frac{\pi}{4}\right)$ 及 $0< A<\frac{\pi}{2}$ 得 $1<t \leqslant \sqrt{2}$. 当 $t=\sqrt{2}$ 时, 即 $A=B=\frac{\pi}{4}$ 时, $S_{\text {max }}=36\left(1+\frac{-2}{\sqrt{2}+1}\right)= 36(3-2 \sqrt{2})$.
%%PROBLEM_END%%


