
%%TEXT_BEGIN%%
4.3 柯西不等式在证明分式不等式中的应用.
在各种不等式中, 分式不等式的问题由于自身的特点, 证明它们需要有较灵活的技巧和方法.
对于分式型的不等式,通常运用柯西不等式的一些变形.
%%TEXT_END%%



%%PROBLEM_BEGIN%%
%%<PROBLEM>%%
例1. 设 $a_1, a_2, \cdots, a_n$ 为正整数, 求证:
$$
\frac{a_1^2}{a_2}+\frac{a_2^2}{a_3}+\cdots+\frac{a_n^2}{a_1} \geqslant a_1+a_2+\cdots+a_n .
$$
%%<SOLUTION>%%
证明:由柯西不等式, 得
$$
\begin{aligned}
& \left(\frac{a_1^2}{a_2}+\frac{a_2^2}{a_3}+\cdots+\frac{a_n^2}{a_1}\right)\left(a_2+a_3+\cdots+a_1\right) \\
\geqslant & \left(\frac{a_1}{\sqrt{a_2}} \cdot \sqrt{a_2}+\frac{a_2}{\sqrt{a_3}} \cdot \sqrt{a_3}+\cdots+\frac{a_n}{\sqrt{a_1}} \cdot \sqrt{a_1}\right)^2 \\
= & \left(a_1+a_2+\cdots+a_n\right)^2,
\end{aligned}
$$
故
$$
\frac{a_1^2}{a_2}+\frac{a_2^2}{a_3}+\cdots+\frac{a_n^2}{a_1} \geqslant a_1+a_2+\cdots+a_n .
$$
%%PROBLEM_END%%



%%PROBLEM_BEGIN%%
%%<PROBLEM>%%
例2. 已知正数 $a_1, a_2, \cdots, a_n(n \geqslant 2)$ 满足 $\sum_{i=1}^n a_i=1$, 求证:
$$
\sum_{i=1}^n \frac{a_i}{2-a_i} \geqslant \frac{n}{2 n-1} \text {. }
$$
%%<SOLUTION>%%
证明:因为
$$
\sum_{i=1}^n \frac{a_i}{2-a_i}=\sum_{i=1}^n\left(\frac{2}{2-a_i}-1\right)=\sum_{i=1}^n \frac{2}{2-a_i}-n .
$$
由柯西不等式, 得
$$
\left(\sum_{i=1}^n \frac{1}{2-a_i}\right)\left[\sum_{i=1}^n\left(2-a_i\right)\right] \geqslant n^2 .
$$
所以
$$
\sum_{i=1}^n \frac{1}{2-a_i} \geqslant \frac{n^2}{\sum_{i=1}^n\left(2-a_i\right)}=\frac{n^2}{2 n-1},
$$
故
$$
\sum_{i=1}^n \frac{a_i}{2-a_i} \geqslant \frac{2 n^2}{2 n-1}-n=\frac{n}{2 n-1} .
$$
%%PROBLEM_END%%



%%PROBLEM_BEGIN%%
%%<PROBLEM>%%
例3. 设 $a_i, b_i, i \geqslant 1$ 为正数,满足 $\sum_{i=1}^n a_i=\sum_{i=1}^n b_i$. 求证:
$$
\sum_{i=1}^n \frac{a_i^2}{a_i+b_i} \geqslant \frac{1}{2}\left(\sum_{i=1}^n a_i\right) .
$$
%%<SOLUTION>%%
证明:由柯西不等式, 得
$$
\left(\sum_{i=1}^n \frac{a_i^2}{a_i+b_i}\right) \sum_{i=1}^n\left(a_i+b_i\right) \geqslant\left(\sum_{i=1}^n a_i\right)^2 .
$$
由于 $\sum_{i=1}^n a_i=\sum_{i=1}^n b_i$, 所以上式即为
$$
2\left(\sum_{i=1}^n a_i\right)\left(\sum_{i=1}^n \frac{a_i^2}{a_i+b_i}\right) \geqslant\left(\sum_{i=1}^n a_i\right)^2
$$
故命题成立.
%%PROBLEM_END%%



%%PROBLEM_BEGIN%%
%%<PROBLEM>%%
例4. 设 $P_1, P_2, \cdots, P_n$ 是 $1,2, \cdots, n$ 的任一排列, 求证:
$$
\frac{1}{P_1+P_2}+\frac{1}{P_2+P_3}+\cdots+\frac{1}{P_{n-2}+P_{n-1}}+\frac{1}{P_{n-1}+P_n}>\frac{n-1}{n+2} .
$$
%%<SOLUTION>%%
证明:由柯西不等式, 得
$$
\begin{gathered}
{\left[\left(P_1+P_2\right)+\left(P_2+P_3\right)+\cdots+\left(P_{n-1}+P_n\right)\right]} \\
\left(\frac{1}{P_1+P_2}+\frac{1}{P_2+P_3}+\cdots+\frac{1}{P_{n-2}+P_{n-1}}+\frac{1}{P_{n-1}+P_n}\right) \geqslant(n-1)^2,
\end{gathered}
$$
所以
$$
\begin{aligned}
& \frac{1}{P_1+P_2}+\frac{1}{P_2+P_3}+\cdots+\frac{1}{P_{n-2}+P_{n-1}}+\frac{1}{P_{n-1}+P_n} \\
\geqslant & \frac{(n-1)^2}{2\left(P_1+P_2+\cdots+P_n\right)-P_1-P_n} \\
= & \frac{(n-1)^2}{n(n+1)-P_1-P_n} \\
\geqslant & \frac{(n-1)^2}{n(n+1)-1-2} \\
= & \frac{(n-1)^2}{(n-1)(n+2)-1} \\
> & \frac{(n-1)^2}{(n-1)(n+2)}=\frac{n-1}{n+2} .
\end{aligned}
$$
%%PROBLEM_END%%



%%PROBLEM_BEGIN%%
%%<PROBLEM>%%
例5. 设正数 $x_i$ 满足 $\sum_{i=1}^n x_i==1$, 求证:
$$
\sum_{i=1}^n \frac{x_i}{\sqrt{1-x_i}} \geqslant \frac{1}{\sqrt{n-1}} \sum_{i=1}^n \sqrt{x_i} .
$$
%%<SOLUTION>%%
证明:由柯西不等式, 得
$$
\sum_{i=1}^n \frac{1}{\sqrt{1-x_i}} \cdot \sum_{i=1}^n \sqrt{1-x_i} \geqslant n^2,
$$
以及
$$
\sum_{i=1}^n \sqrt{1-x_i} \leqslant \sqrt{\sum_{i=1}^n 1 \cdot \sum_{i=1}^n\left(1-x_i\right)}=\sqrt{n(n-1)},
$$
所以
$$
\begin{aligned}
\sum_{i=1}^n \frac{x_i}{\sqrt{1-x_i}} & =\sum_{i=1}^n \frac{1}{\sqrt{1-x_i}}-\sum_{i=1}^n \sqrt{1-x_i} \\
& \geqslant \frac{n^2}{\sum_{i=1}^n \sqrt{1-x_i}}-\sum_{i=1}^n \sqrt{1-x_i} \\
& \geqslant \frac{n^2}{\sqrt{n(n-1)}}-\sqrt{n(n-1)} \\
& =\sqrt{\frac{n}{n-1}} .
\end{aligned}
$$
又由柯西不等式, 得
$$
\sum_{i=1}^n \sqrt{x_i} \leqslant \sqrt{\sum_{i=1}^n 1 \cdot \sum_{i=1}^n x_i}=\sqrt{n},
$$
故命题成立
%%PROBLEM_END%%



%%PROBLEM_BEGIN%%
%%<PROBLEM>%%
例 6 设 $a 、 b 、 c$ 是大于 -1 的实数,证明:
$$
\frac{1+a^2}{1+b+c^2}+\frac{1+b^2}{1+c+a^2}+\frac{1+c^2}{1+a+b^2} \geqslant 2 \text {. }
$$
%%<SOLUTION>%%
证明:由假设我们有 $1+a^2, 1+b^2, 1+c^2, 1+b+c^2, 1+c+a^2, 1+-a+ b^2$ 均大于零.
由柯西不等式,得
$$
\begin{aligned}
& \left(\frac{1+a^2}{1+b+c^2}+\frac{1+b^2}{1+c+a^2}+\frac{1+c^2}{1+a+b^2}\right) \cdot\left[\left(1+a^2\right)\left(1+b+c^2\right)+\right. \\
& \left.\left(1+b^2\right)\left(1+c+a^2\right)+\left(1+c^2\right)\left(1+a+b^2\right)\right] \\
\geqslant & \left(1+a^2+1+b^2+1+c^2\right)^2 .
\end{aligned}
$$
于是
$$
\begin{aligned}
& \frac{1+a^2}{1+b+c^2}+\frac{1+b^2}{1+c+a^2}+\frac{1+c^2}{1+a+b^2} \\
\geqslant & \frac{\left(a^2+b^2+c^2+3\right)^2}{\left(1+a^2\right)\left(1+b+c^2\right)+\left(1+b^2\right)\left(1+c+a^2\right)+\left(1+c^2\right)\left(1+a+b^2\right)} \\
= & \frac{a^4+b^4+c^4+9+2 a^2 b^2+2 b^2 c^2+2 c^2 a^2+6 a^2+6 b^2+6 c^2}{a^2 b^2+b^2 c^2+c^2 a^2+2\left(a^2+b^2+c^2\right)+a^2 b+b^2 c+c^2 a+a+b+c+3} \\
= & 2+\frac{a^4+b^4+c^4+3+2 a^2+2 b^2+2 c^2-2\left(a^2 b+b^2 c+c^2 a+a+b+c\right)}{a^2 b^2+b^2 c^2+c^2 a^2+2\left(a^2+b^2+c^2\right)+a^2 b+b^2 c+c^2 a+a+b+c+3}
\end{aligned}
$$
$$
\begin{aligned}
& =2+\frac{\left(a^2-b\right)^2+\left(b^2-c\right)^2+\left(c^2-a\right)^2+(a-1)^2+(b-1)^2+(c-1)^2}{a^2 b^2+b^2 c^2+c^2 a^2+2\left(a^2+b^2+c^2\right)+a^2 b+b^2 c+c^2 a+a+b+c+3} \\
& \geqslant 2
\end{aligned}
$$
当且仅当 $a=b=c=1$ 时等号成立.
%%PROBLEM_END%%



%%PROBLEM_BEGIN%%
%%<PROBLEM>%%
例7. 正数 $a, b, c$ 满足 $a b c=1, n$ 为正整数,求证:
(a) $\frac{1}{1+2 a}+\frac{1}{1+2 b}+\frac{1}{1+2 c} \geqslant 1$;
(b) $\frac{c^n}{a+b}+\frac{b^n}{c+a}+\frac{a^n}{b+c} \geqslant \frac{3}{2}$.
%%<SOLUTION>%%
证明:(a) 首先来证明
$$
\begin{aligned}
& \frac{1}{1+2 a} \geqslant \frac{a^{-\frac{2}{3}}}{a^{-\frac{2}{3}}+b^{-\frac{2}{3}}+c^{-\frac{2}{3}}} \\
\Leftrightarrow & a^{-\frac{2}{3}}+b^{-\frac{2}{3}}+c^{-\frac{2}{3}} \geqslant a^{-\frac{2}{3}}+2 a^{\frac{1}{3}} \\
\Leftrightarrow & b^{-\frac{2}{3}}+c^{-\frac{2}{3}} \geqslant 2 b c^{-\frac{1}{3}}
\end{aligned}
$$
这是显然的.
同理有 $\frac{1}{1+2 b} \geqslant \frac{b^{-\frac{2}{3}}}{a^{-\frac{2}{3}}+b^{-\frac{2}{3}}+c^{-\frac{2}{3}}}, \frac{1}{1+2 c} \geqslant \frac{c^{-\frac{2}{3}}}{a^{-\frac{2}{3}}+b^{-\frac{2}{3}}+c^{-\frac{2}{3}}}$.
所以 $\frac{1}{1+2 a}+\frac{1}{1+2 b}+\frac{1}{1+2 c} \geqslant 1$.
(b) 不妨设 $a \geqslant b \geqslant c$, 那么 $a^{n-1} \geqslant b^{n-1} \geqslant c^{n-1}, \frac{a}{b+c} \geqslant \frac{b}{c+a} \geqslant \frac{c}{a+b}$. 由排序不等式得到
$$
\begin{aligned}
& \frac{c^n}{a+b}+\frac{b^n}{c+a}+\frac{a^n}{b+c} \geqslant \frac{c a^{n-1}}{a+b}+\frac{b c^{n-1}}{c+a}+\frac{a b^{n-1}}{b+c}, \\
& \frac{c^n}{a+b}+\frac{b^n}{c+a}+\frac{a^n}{b+c} \geqslant \frac{c b^{n-1}}{a+b}+\frac{b a^{n-1}}{c+a}+\frac{a c^{n-1}}{b+c},
\end{aligned}
$$
所以
$$
\begin{aligned}
& \frac{c^n}{a+b}+\frac{b^n}{c+a}+\frac{a^n}{b+c} \\
\geqslant & \frac{1}{3}\left(\frac{c}{a+b}+\frac{b}{c+a}+\frac{a}{b+c}\right)\left(a^{n-1}+b^{n-1}+c^{n-1}\right),
\end{aligned}
$$
而显然有 $a^{n-1}+b^{n-1}+c^{n-1} \geqslant 3$, 下面来证明 $\frac{c}{a+b}+\frac{b}{c+a}+\frac{a}{b+c} \geqslant \frac{3}{2}$.
即证 $\frac{a+b+c}{a+b}+\frac{a+b+c}{c+a}+\frac{a+b+c}{b+c} \geqslant \frac{9}{2}$
$$
\Leftrightarrow(a+b+c+a+b+c)\left(\frac{1}{a+b}+\frac{1}{c+a}+\frac{1}{b+c}\right) \geqslant 9 .
$$
这由柯西不等式可知是显然的.
所以
$$
\begin{aligned}
& \frac{c^n}{a+b}+\frac{b^n}{c+a}+\frac{a^n}{b+c} \\
\geqslant & \frac{1}{3}\left(\frac{c}{a+b}+\frac{b}{c+a}+\frac{a}{b+c}\right)\left(a^{n-1}+b^{n-1}+c^{n-1}\right) \geqslant \frac{3}{2} .
\end{aligned}
$$
证毕.
%%PROBLEM_END%%



%%PROBLEM_BEGIN%%
%%<PROBLEM>%%
例8. 证明: 对任意满足 $x+y+z=0$ 的实数 $x, y, z$ 都有
$$
\frac{x(x+2)}{2 x^2+1}+\frac{y(y+2)}{2 y^2+1}+\frac{z(z+2)}{2 z^2+1} \geqslant 0 .
$$
%%<SOLUTION>%%
证明:注意到 $\frac{x(x+2)}{2 x^2+1}=\frac{(2 x+1)^2}{2\left(2 x^2+1\right)}-\frac{1}{2}$ 等式子, 所以原不等式等价于
$$
\frac{(2 x+1)^2}{2 x^2+1}+\frac{(2 y+1)^2}{2 y^2+1}+\frac{(2 z+1)^2}{2 z^2+1} \geqslant 3,
$$
由柯西不等式,我们有
$$
2 x^2=\frac{4}{3} x^2+\frac{2}{3}(y+z)^2 \leqslant \frac{4}{3} x^2+\frac{4}{3}\left(y^2+z^2\right),
$$
所以
$$
\sum \frac{(2 x+1)^2}{2 x^2+1} \geqslant 3 \sum \frac{(2 x+1)^2}{4\left(x^2+y^2+z^2\right)+3}=3 .
$$
%%PROBLEM_END%%



%%PROBLEM_BEGIN%%
%%<PROBLEM>%%
例9. 已知正数 $a_1, a_2, \cdots, a_n(n>2)$ 满足 $a_1+a_2+\cdots+a_n=1$. 证明:
$$
\frac{a_2 a_3 \cdots a_n}{a_1+n-2}+\frac{a_1 a_3 \cdots a_n}{a_2+n-2}+\cdots+\frac{a_1 a_2 \cdots a_{n-1}}{a_n+n-2} \leqslant \frac{1}{(n-1)^2} .
$$
%%<SOLUTION>%%
证明:由柯西不等式, 知对于正数 $x_1, x_2, \cdots, x_n$, 有
$$
\frac{1}{\sum_{i=1}^n x_i} \leqslant \frac{1}{n^2} \sum_{i=1}^n \frac{1}{x_i} .
$$
又 $a_1+a_2+\cdots+a_n=1(n>2)$, 则
$$
\sum_{i=1}^n \frac{1}{a_i\left(a_i+n-2\right)}
$$
$$
\begin{aligned}
& =\sum_{i=1}^n \frac{1}{a_i \sum_{\substack{j=1 \\
j \neq i}}^n\left(1-a_j\right)} \\
& \leqslant \sum_{i=1}^n \frac{1}{(n-1)^2} \sum_{\substack{j=1 \\
j \neq i}}^n \frac{1}{a_i\left(1-a_j\right)} \\
& =\frac{1}{(n-1)^2} \sum_{j=1}^n \sum_{\substack{i=1 \\
i \neq j}}^n \frac{1}{a_i\left(1-a_j\right)} .
\end{aligned}
$$
由已知得
$$
a_i \in(0,1)(i=1,2, \cdots, n) .
$$
于是,对任意的 $j \in\{1,2, \cdots, n\}$, 有
$$
\begin{array}{r}
a_i \geqslant \frac{\prod_{k=1}^n a_k}{a_{i-1} a_j}, \\
a_{j+1} \geqslant \frac{\prod_{k=1}^n a_k}{a_{j-1} a_j},
\end{array}
$$
其中, $i=1,2, \cdots, n, i \neq j, i \neq j+1, a_0=a_n$.
故 $\quad \sum_{\substack{i=1 \\ i \neq j}}^n a_i \geqslant \sum_{\substack{i=1 \\ i \neq j}}^n \frac{\prod_{k=1}^n a_k}{a_i a_j} \Rightarrow\left(1-a_j\right) \frac{a_j}{\prod_{k=1}^n a_k} \geqslant \sum_{\substack{i=1 \\ i \neq j}}^n \frac{1}{a_i}$, 即
$$
\sum_{\substack{i=1 \\ i \neq j}}^n \frac{1}{a_i\left(1-a_j\right)} \leqslant \frac{1}{\prod_{\substack{k=1 \\ k \neq j}}^n a_k}
$$
则
$$
\sum_{i=1}^n \frac{1}{a_i\left(a_i+n-2\right)} \leqslant \frac{1}{(n-1)^2} \sum_{j=1}^n \frac{1}{\prod_{\substack{k=1 \\ k \neq j}}^n a_k} .
$$
故
$$
\sum_{i=1}^n \frac{\prod_{\substack{k=1 \\ k \neq i}}^n a_k}{a_i+n-2} \leqslant \frac{1}{(n-1)^2} \sum_{j=1}^n a_j=\frac{1}{(n-1)^2} .
$$
%%PROBLEM_END%%



%%PROBLEM_BEGIN%%
%%<PROBLEM>%%
例10. 设 $n \geqslant 2, a_1, a_2, \cdots, a_n$ 是 $n$ 个正实数,满足:
$$
\left(a_1+\cdots+a_n\right)\left(\frac{1}{a_1}+\frac{1}{a_2}+\cdots+-\frac{1}{a_n}\right) \leqslant\left(n+\frac{1}{2}\right)^2 .
$$
证明: $\max \left\{a_1, \cdots, a_n\right\} \leqslant 4 \min \left\{a_1, \cdots, a_n\right\}$.
%%<SOLUTION>%%
证明:不妨设
$$
m=a_1 \leqslant a_2 \leqslant \cdots \leqslant a_n=M,
$$
要证 $M \leqslant 4 m$.
当 $n=2$ 时,条件为
$$
(m+M)\left(\frac{1}{m}+\frac{1}{M}\right) \leqslant \frac{25}{4} .
$$
等价于
$$
4(m+M)^2 \leqslant 25 m M,
$$
即
$$
(4 M-m)(M-4 m) \leqslant 0,
$$
而
$$
4 M-m \geqslant 3 M>0,
$$
故 $M \leqslant 4 m$.
当 $n \geqslant 3$ 时,利用柯西不等式可知
$$
\begin{gathered}
\left(n+\frac{1}{2}\right)^2 \geqslant\left(a_1+\cdots+a_n\right)\left(\frac{1}{a_1}+\cdots+\frac{1}{a_n}\right) \\
=\left(m+a_2+\cdots+a_{n-1}+M\right)\left(\frac{1}{M}+\frac{1}{a_2}+\cdots+\frac{1}{a_{n-1}}+\frac{1}{m}\right) \\
\geqslant(\sqrt{\frac{m}{M}}+\underbrace{1+\cdots+1}_{n-2 \uparrow}+\sqrt{\frac{M}{m}})^2 .
\end{gathered}
$$
故
$$
n+\frac{1}{2} \geqslant \sqrt{\frac{m}{M}}+\sqrt{\frac{M}{m}}+n-2,
$$
于是
$$
\sqrt{\frac{M}{m}}+\sqrt{\frac{m}{M}} \leqslant \frac{5}{2} .
$$
从而
$$
2(m+M) \leqslant 5 \sqrt{m M},
$$
同 $n=2$ 的情形可得 $M \leqslant 4 m$. 命题获证.
%%PROBLEM_END%%



%%PROBLEM_BEGIN%%
%%<PROBLEM>%%
例11. 设
$$
f(x, y, z)=\frac{x(2 y-z)}{1+x+3 y}+\frac{y(2 z-x)}{1+y+3 z}+\frac{z(2 x-y)}{1+z+3 x},
$$
其中 $x, y, z \geqslant 0$, 且 $x+y+z=1$. 求 $f(\dot{x}, y, z)$ 的最大值和最小值.
%%<SOLUTION>%%
解:先证 $f \leqslant \frac{1}{7}$, 当且仅当 $x=y=z=\frac{1}{3}$ 时等号成立.
因为
$$
f=\sum \frac{x(x+3 y-1)}{1+x+3 y}=1-2 \sum \frac{x}{1+x+3 y}, \label{(41)}
$$
由柯西不等式
$$
\sum \frac{x}{1+x+3 y} \geqslant \frac{\left(\sum x\right)^2}{\sum x(1+x+3 y)}=\frac{1}{\sum x(1+x+3 y)},
$$
因为
$$
\sum x(1+x+3 y)=\sum x(2 x+4 y+z)=2+\sum x y \leqslant \frac{7}{3} .
$$
从而
$$
\begin{aligned}
& \sum \frac{x}{1+x+3 y} \geqslant \frac{3}{7}, \\
& f \leqslant 1-2 \times \frac{3}{7}=\frac{1}{7},
\end{aligned}
$$
$f_{\text {max }}=\frac{1}{7}$, 当且仅当 $x=y=z=\frac{1}{3}$ 时等号成立.
再证 $f \geqslant 0$, 当 $x=1, y=z=0$ 时等号成立.
事实上,
$$
\begin{aligned}
f(x, y, z)= & \frac{x(2 y-z)}{1+x+3 y}+\frac{y(2 z-x)}{1+y+3 z}+\frac{z(2 x-y)}{1+z+3 x} \\
= & x y\left(\frac{2}{1+x+3 y}-\frac{1}{1+y+3 z}\right) \\
& +x z\left(\frac{2}{1+z+3 x}-\frac{1}{1+x+3 y}\right)
\end{aligned}
$$
$$
\begin{aligned}
& +y z\left(\frac{2}{1+y+3 z}-\frac{1}{1+z+3 x}\right) \\
= & \frac{7 x y z}{(1+x+3 y)(1+y+3 z)} \\
& +\frac{7 x y z}{(1+z+3 x)(1+x+3 y)} \\
& +\frac{7 x y z}{(1+y+3 z)(1+z+3 x)} \\
\geqslant & 0 .
\end{aligned}
$$
故 $f_{\min }=0$, 当 $x=1, y=z=0$ 时等号成立.
另证: 设 $z=\min \{x, y, z\}$, 若 $z=0$, 则
$$
f(x, y, 0)=\frac{2 x y}{1+x+3 y}-\frac{x y}{1+y}=\frac{2 x y}{2 x+4 y}-\frac{x y}{x+2 y}=0 .
$$
下设 $x, y \geqslant z>0$, 由 (41) 式, 要证 $f \geqslant 0$, 只要证
$$
\sum \frac{x}{1+x+3 y} \leqslant \frac{1}{2} . \label{(42)}
$$
注意到
$$
\frac{1}{2}=\frac{x}{2 x+4 y}+\frac{y}{x+2 y}
$$
于是(42)等价于
$$
\begin{aligned}
\frac{z}{1+z+3 x} \leqslant & \left(\frac{x}{2 x+4 y}-\frac{x}{1+x+3 y}\right)+\left(\frac{y}{x+2 y}-\frac{y}{1+y+3 z}\right) \\
& =\frac{z}{2 x+4 y}\left(\frac{x}{1+x+3 y}+\frac{8 y}{1+y+3 z}\right) \\
& \frac{2 x+4 y}{1+z+3 x} \leqslant \frac{x}{1+x+3 y}+\frac{8 y}{1+y+3 z} .
\end{aligned}
$$
即
$$
\frac{2 x+4 y}{1+z+3 x} \leqslant \frac{x}{1+x+3 y}+\frac{8 y}{1+y+3 z} . \label{(43)}
$$
而由柯西不等式, 可得
$$
\begin{aligned}
\frac{x}{1+x+3 y}+\frac{8 y}{1+y+3 z} & =\frac{x^2}{x(1+x+3 y)}+\frac{(2 y)^2}{y(1+y+3 z) / 2} \\
& \geqslant \frac{(x+2 y)^2}{\left(x+x^2+3 x y\right)+\left(y+y^2+3 y z\right) / 2} \\
& =\frac{2 x+4 y}{1+z+3 x},
\end{aligned}
$$
即 (43) 成立, 从而 $f \geqslant 0$, 故 $f_{\text {min }}=0$, 当 $x=1, y=z=0$ 时等号成立.
%%PROBLEM_END%%



%%PROBLEM_BEGIN%%
%%<PROBLEM>%%
例12. 设 $x, y, z \in \mathbf{R}^{+}$, 且 $x+y+z=1$, 证明:
$$
\sum_{x, y, z} \frac{x^4}{y\left(1-y^2\right)} \geqslant \frac{1}{8} \text {. }
$$
%%<SOLUTION>%%
证明:左边 $\geqslant \frac{\left(\sum x^2\right)^2}{\sum y\left(1-y^2\right)} \geqslant \frac{\left[\frac{\left(\sum x\right)^2}{3}\right]^2}{\sum x-\sum x^3}=\frac{\frac{1}{9}}{1-\sum x^3}$.
又 $\quad \sum x^3=\sum \frac{x^4}{x} \geqslant \frac{\left(\sum x^2\right)^2}{\sum x} \geqslant\left[\frac{\left(\sum x\right)^2}{3}\right]^2=\frac{1}{9}$,
所以,左边 $\geqslant \frac{\frac{1}{9}}{1-\frac{1}{9}}=\frac{1}{8}$, 故原不等式成立.
%%PROBLEM_END%%



%%PROBLEM_BEGIN%%
%%<PROBLEM>%%
例13. 设 $x, y, z, w \in \mathbf{R}^{+}$, 证明:
$$
\frac{x}{y+2 z+3 w}+\frac{y}{z+2 w+3 x}+\frac{z}{w+2 x+3 y}+\frac{w}{x+2 y+3 z} \geqslant \frac{2}{3} .
$$
%%<SOLUTION>%%
证明:左边 $=\sum \frac{x}{y+2 z+3 w}=\sum \frac{x^2}{x(y+2 z+3 w)}$
$$
\begin{gathered}
\geqslant \frac{\left(\sum x\right)^2}{\sum x(y+2 z+3 w)} \\
=\frac{\left(\sum x\right)^2}{4 \sum x y}, \\
(x-y)^2+(x-z)^2+(x-w)^2+(y-z)^2+(y-w)^2+(z-w)^2 \\
=3\left(x^2+y^2+z^2+w^2\right)-2(x y+x z+x w+y z+y w+z w) \\
=3(x+y+z+w)^2-8(x y+x z+x w+y z+y w+z w) \geqslant 0 .
\end{gathered}
$$
所以
$$
\frac{\left(\sum x\right)^2}{\sum x y} \geqslant \frac{8}{3}
$$
故原不等式成立.
%%PROBLEM_END%%



%%PROBLEM_BEGIN%%
%%<PROBLEM>%%
例14. 设 $x_1, x_2, \cdots, x_n$ 为任意实数,证明:
$$
\frac{x_1}{1+x_1^2}+\frac{x_2}{1+x_1^2+x_2^2}+\cdots+\frac{x_n}{1+x_1^2+x_2^2+\cdots+x_n^2}<\sqrt{n} .
$$
%%<SOLUTION>%%
证明:由柯西不等式,得
$$
\begin{aligned}
& \left(\frac{x_1}{1+x_1^2}+\frac{x_2}{1+x_1^2+x_2^2}+\cdots+\frac{x_n}{1+x_1^2+x_2^2+\cdots+x_n^2}\right)^2 \\
\leqslant & {\left[\left(\frac{x_1}{1+x_1^2}\right)^2+\left(\frac{x_2}{1+x_1^2+x_2^2}\right)^2+\cdots+\left(\frac{x_n}{1+x_1^2+x_2^2+\cdots+x_n^2}\right)^2\right] \cdot n . }
\end{aligned}
$$
对 $k \geqslant 2$, 有
$$
\begin{aligned}
& \left(\frac{x_k}{1+x_1^2+\cdots+x_k^2}\right)^2 \\
= & \frac{x_k^2}{\left(1+x_1^2+\cdots+x_k^2\right)^2} \\
\leqslant & \frac{x_k^2}{\left(1+x_1^2+\cdots+x_{k-1}^2\right)\left(1+x_1^2+\cdots+x_k^2\right)} \\
= & \frac{1}{1+x_1^2+\cdots+x_{k-1}^2}-\frac{1}{1+x_1^2+\cdots+x_k^2} .
\end{aligned}
$$
对于 $k=1$, 有
$$
\left(\frac{x_1}{1+x_1^2}\right)^2 \leqslant \frac{x_1^2}{1+x_1^2}=1-\frac{1}{1+x_1^2} .
$$
所以
$$
\sum_{i=1}^n\left(\frac{x_k}{1+x_1^2+\cdots+x_k^2}\right)^2 \leqslant 1-\frac{1}{1+x_1^2+\cdots+x_n^2}<1,
$$
从而
$$
\left(\frac{x_1}{1+x_1^2}+\frac{x_2^2}{1+x_1^2+x_2^2}+\cdots+\frac{x_n^2}{1+x_1^2+\cdots+x_n^2}\right)^2<n,
$$
故命题成立.
%%PROBLEM_END%%



%%PROBLEM_BEGIN%%
%%<PROBLEM>%%
例15. 已知 $x_i \in \mathbf{R}^{+}(i \geqslant 1)$ 满足 $\sum_{i=1}^n x_i=\sum_{i=1}^n \frac{1}{x_i}$, 求证:
$$
\sum_{i=1}^n \frac{1}{n-1+x_i} \leqslant 1 .
$$
%%<SOLUTION>%%
证明:令 $y_i=\frac{1}{n-1+x_i}$, 则 $x_i=\frac{1}{y_i}-(n-1), 0<y_i<\frac{1}{n-1}$.
如果 $\sum_{i=1}^n y_i>1$, 将证明 $\sum_{i=1}^n x_i<\sum_{i=1}^n \frac{1}{x_i}$, 即等价于
$$
\sum_{i=1}^n\left[\frac{1}{y_i}-(n-1)\right]<\sum_{i=1}^n \frac{y_i}{1-(n-1) y_i} .
$$
对固定 $i$, 由柯西不等式, 得
$$
\begin{aligned}
& \sum_{i \neq j} \frac{1-(n-1) y_i}{1-(n-1) y_j} \\
\geqslant & \frac{\left[1-(n-1) y_i\right](n-1)^2}{\sum_{i \neq j}\left[1-(n-1) y_i\right]} \\
> & \frac{\left[1-(n-1) y_i\right](n-1)^2}{(n-1) y_i}=\frac{(n-1)\left[1-(n-1) y_i\right]}{y_i} .
\end{aligned}
$$
对 $i$ 求和, 得
$$
\sum_{i=1}^n \sum_{i \neq j} \frac{1-(n-1) y_i}{1-(n-1) y_j} \geqslant(n-1) \sum_{i=1}\left[\frac{1}{y_i}-(n-1)\right] .
$$
由于
$$
\begin{gathered}
\sum_{i=1}^n \sum_{i \neq j} \frac{1-(n-1) y_i}{1-(n-1) y_j} \leqslant \sum_{j=1}^n \frac{(n-1) y_j}{1-(n-1) y_j}, \\
\sum_{i=1}^n \frac{y_i}{1-(n-1) y_i}>\sum_{j=1}^n\left[\frac{1}{y_j}-(n-1)\right] .
\end{gathered}
$$
%%PROBLEM_END%%


