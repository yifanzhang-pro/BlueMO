
%%TEXT_BEGIN%%
递推方法解组合问题的一般步骤是:
(1) 用枚举法求初始值;
(2) 建立递推关系;
(3) 利用递推关系求解.
当利用已建立的递推关系求解遇到困难时, 还应考虑建立新的递推关系, 而在建立递推关系遇到困难时, 则可列举简单情形寻求启示, 从中归纳得出要求的递推关系.
%%TEXT_END%%



%%PROBLEM_BEGIN%%
%%<PROBLEM>%%
例1. 整数 $1,2, \cdots, n$ 的排列满足: 每个数或者大于它前面的所有的数, 或者小于它前面的所有的数, 试问有多少个这样的排列?
%%<SOLUTION>%%
解:记所求排列的个数为 $a_n$.
$n=1$ 时, 只有数 1 , 显然 $a_1=1 . n=2$ 时, 显然只有 2 个排列 1,2 和 2 , 1 , 所以 $a_2=2$.
对于 $n>2$, 考虑 $n$ 在第 $i$ 个位置的排列, 这时 $n$ 之后的 $n-i$ 个数的位置唯一确定, 只能是 $n-i, n-i-1, \cdots, 2,1$, 而它前面的 $i-1$ 个数有 $a_{i-1}$ 种排法 (约定 $a_0=1$ ), 又 $i=1,2, \cdots, n$, 故必有
$$
a_n=1+a_1+\cdots+a_{n-1},
$$
从而
$$
a_{n-1}=1+a_1+\cdots+a_{n-2}(n \geqslant 3),
$$
两式相减得 $a_n-a_{n-1}=a_{n-1}$, 即 $a_n=2 a_{n-1}(n \geqslant 3)$, 所以 $a_n=a_2 \cdot 2^{n-2}=2^{n-1}$, 经检验 $a_1=1, a_2=2$ 满足 $a_n=2^{n-1}$, 所以 $a_n=2^{n-1}(n \geqslant 1)$, 即所求的排列个数为 $2^{n-1}$.
%%PROBLEM_END%%



%%PROBLEM_BEGIN%%
%%<PROBLEM>%%
例2. $A$ 城有 $n$ 个女孩和 $n$ 个男孩,且每个女孩都认识所有的男孩.
$B$ 城有 $n$ 个女孩 $g_1, g_2, \cdots, g_n$ 和 $2 n-1$ 个男孩 $b_1, b_2, \cdots, b_{2 n-1}$, 其中女孩 $g_i$ 认识男孩 $b_1, b_2, \cdots, b_{2 i-1}$, 但不认识其他男孩, $i=1,2, \cdots, n$. 对任意的 $r=1,2, \cdots, n$. 从 $A$ 城和 $B$ 城各选取 $r$ 个女孩与 $r$ 个男孩组成 $r$ 对舞伴, 要求每个女孩的舞伴都是自己认识的男孩,分别将 $A$ 城和 $B$ 城满足要求的舞伴的不同选法种数记为 $A(r)$ 和 $B(r)$, 求证 $A(r)=B(r), r=1,2, \cdots, n$. 
%%<SOLUTION>%%
证明:为了便于用递推方法, 将题中 $A(r)$ 和 $B(r)$ 分别记为 $A_n(r)$ 和 $B_n(r)$. 因为 $A$ 城的 $n$ 个女孩中每人都认识所有的男孩,故有
$$
A_n(r)=\left(\mathrm{C}_n^r\right)^2 \cdot r !=\frac{(n !)^2}{[(n-r) !]^2 \cdot r !}, \label{eq1}
$$
下面建立 $B_n(r)$ 的递推关系式, 设 $n \geqslant 3,2 \leqslant r \leqslant n$. 考察 $B$ 城中选取 $r$ 对舞伴且每个女孩都认识自己舞伴的所有可能选法, 分两种情况来记数.
当 $g_n$ 是选中的 $r$ 个女孩之一时, 其余 $r-1$ 对舞伴共有 $B_{n-1}(r-1)$ 种不同选法, $g_n$ 可在余下 $(2 n-1)-(r-1)=2 n-r$ 个男孩中任选 1 个男孩为舞伴, 共有 $2 n-r$ 种选法, 由乘法原理知这种情况共有 $(2 n-r) B_{n-1}(r-1)$ 种不同选法.
当 $g_n$ 未被选中时, 当然有 $r<n$, 于是 $r$ 个女孩全部选自 $g_1, g_2, \cdots$, $g_{n-1}$ 且他们的舞伴来自于 $b_1, b_2, \cdots, b_{2 n-3}$ 之中, 所以不同选法数恰为 $B_{n-1}(r)$, 这样一来, 当 $n \geqslant 3$ 时, 总有
$$
\left.\begin{array}{l}
B_n(r)=B_{n-1}(r)+(2 n-r) B_{n-1}(r-1) \\
B_n(n)=n B_{n-1}(n-1)
\end{array}\right\} \label{eq2}
$$
其中 $r=2,3, \cdots, n-1$.
因为 $A_n(1)=n^2=B_n(1)$ 和 $A_2(2)=2=B_2(2)$, 并且由式\ref{eq1}有
$$
A_{n-1}(r)=\frac{[(n-1) !]^2}{[(n-1-r) !]^2 r !}, A_{n-1}(r-1)=\frac{[(n-1) !]^2}{[(n-r) !]^2(r-1) !},
$$
由此可得
$$
\left.\begin{array}{l}
A_n(r)=A_{n-1}(r)+(2 n-r) A_{n-1}(r-1) \\
A_n(n)=n A_{n-1}(n-1)
\end{array}\right\} \label{eq3}
$$
式\ref{eq2},\ref{eq3}表明, $A_n(r)$ 与 $B_n(r)$ 有相同的递推关系并且它们有相同的初始值.
所以对任意 $n \in \mathbf{N}_{+}$和 $r=1,2, \cdots, n$, 均有 $A_n(r)=B_n(r)$, 证毕.
%%PROBLEM_END%%



%%PROBLEM_BEGIN%%
%%<PROBLEM>%%
例3. 一个由若干行数字组成的数表, 从第二行起每行中的数字均等于其肩上的两个数的和, 最后一行仅一个数.
第一行是前 100 个正整数按从小到大排成的行, 则最后一行的数是 ?(可以用指数表示). 
%%<SOLUTION>%%
解:易知:
(1) 该数表共有 100 行;
(2)每一行构成一个等差数列,且公差依次是:
$$
d_1=1, d_2=2, d_3=2^2, \cdots, d_{99}=2^{98} ;
$$
(3)设第 $n$ 行中第一个数为 $a_n$, 则 $a_{100}$ 即为所求.
依题意
$$
a_n=a_{n-1}+\left(a_{n-1}+2^{n-2}\right)=2 a_n+2^{n-2},
$$
即 $\frac{a_n}{2^n}=\frac{a_{n-1}}{2^{n-1}}+\frac{1}{4}$. 于是 $\left\{\frac{a_n}{2^n}\right\}$ 是首项为 $\frac{a_1}{2}=\frac{1}{2}$, 公差为 $\frac{1}{4}$ 的等差数列, 所以
$$
\frac{a_n}{2^n}=\frac{1}{2}+\frac{1}{4}(n-1)=\frac{1}{4}(n+1),
$$
故 $a_n=(n+1) \cdot 2^{n-2}$, 特别, $a_{100}=101 \times 2^{98}$. 也就是表中最后一行的数是 $101 \times 2^{98}$.
%%PROBLEM_END%%



%%PROBLEM_BEGIN%%
%%<PROBLEM>%%
例4. 一种密码锁的密码设置是在正 $n$ 边形 $A_1 A_2 \cdots A_n$ 的每个顶点处赋值 0 和 1 两个数中的一个, 同时, 在每个顶点处染红、蓝两种颜色之一, 使得相邻两个顶点的数字和颜色至少有一个相同,问该种密码锁共有多少种不同的密码设置?
%%<SOLUTION>%%
解:设共有 $a_n$ 种不同的密码设置, 且不妨设一个点为正 1 边形,一条线段为正 2 边形,于是 $a_1=4, a_2=4 \times 3=12$.
依次将 $A_1, A_2, \cdots, A_n$ 赋值和染色,使得 $A_{i+1}$ 与 $A_i$ 的数字与颜色中至少有一种相同, 于是 $A_1$ 有 4 种设置, 当 $A_i$ 取定后, $A_{i+1}$ 有 3 种设置 $(i=1$, $2, \cdots, n-1)$, 故共有 $4 \times 3^{n-1}$ 种设置.
设其中 $A_n$ 与 $A_1$ 的数字与颜色都不相同的有 $b_n$ 种, $A_n$ 与 $A_1$ 的数字与颜色完全相同的有 $c_n$ 种, $A_n$ 与 $A_1$ 的数字与颜色恰有一种相同的有 $d_n$ 种, 则
$$
\begin{gathered}
a_n+b_n=4 \times 3^{n-1}, \label{eq1}\\
c_n=a_{n-1}, \label{eq2}\\
a_n=c_n+d_n, \label{eq4}\\
b_n=b_{n-1}+d_{n-1},\label{eq4}
\end{gathered}
$$
由式\ref{eq1}得
$$
a_{n-1}+b_{n-1}=4 \times 3^{n-2} . \label{eq5}
$$
式\ref{eq1}-\ref{eq5}得
$$
\left(a_n-a_{n-1}\right)+\left(b_n-b_{n-1}\right)=4 \times\left(3^{n-1}-3^{n-2}\right)=8 \times 3^{n-2} . \label{eq6}
$$
而由式\ref{eq4}, \ref{eq3}, 式\ref{eq2}有
$$
b_n-b_{n-1}=d_{n-1}=a_{n-1}-c_{n-1}=a_{n-1}-a_{n-2}, \label{eq7}
$$
式\ref{eq7}代入\ref{eq6}得
$$
\left(a_n-a_{n-1}\right)+\left(a_{n-1}-a_{n-2}\right)=8 \times 3^{n-2},
$$
所以
$$
a_n-a_{n-2}=8 \times 3^{n-2}(n \geqslant 3) .
$$
于是, 当 $n=2 k$ 时,
$$
\begin{aligned}
a_n=a_{2 k} & =a_2+\left(a_4-a_2\right)+\left(a_6-a_4\right)+\cdots+\left(a_{2 k}-a_{2 k-2}\right) \\
& =12+8 \times 3^2+8 \times 3^4+\cdots+8 \times 3^{2 k-2} \\
& =12+\frac{8 \times 3^2 \times\left(3^{2 k-2}-1\right)}{3^2-1}=12+3^2 \times\left(3^{2 k-2}-1\right) \\
& =3^{2 k}+3=3^n+2+(-1)^n .
\end{aligned}
$$
当 $n=2 k+1$ 时,
$$
\begin{aligned}
a_n=a_{2 k+1} & =a_1+\left(a_3-a_1\right)+\left(a_5-a_3\right)+\cdots+\left(a_{2 k+1}-a_{2 k-1}\right) \\
& =4+8 \times 3+8 \times 3^3+\cdots+8 \times 3^{2 k-1} \\
& =4+\frac{8 \times 3 \times\left(3^{2 k}-1\right)}{3^2-1}=4+3 \times\left(3^{2 k}-1\right) \\
& =3^{2 k+1}+1=3^n+2+(-1)^n,
\end{aligned}
$$
故总有 $a_n=3^n+2+(-1)^n$.
综上得不同的密码设置共有 $3^n+2+(-1)^n$ 种.
%%PROBLEM_END%%



%%PROBLEM_BEGIN%%
%%<PROBLEM>%%
例5. 将周长为 24 的圆周等分为 24 段, 从 24 个分点中选取 8 个分点使得其中任意两点所夹的弧长不等于 3 和 8 , 问满足要求的 8 点组的不同取法有多少种?
%%<SOLUTION>%%
解:设 24 个分点依次为 $1,2, \cdots, 24$. 将这 24 个数列成下表:
\begin{tabular}{|c|c|c|c|c|c|c|c|}
\hline 1 & 4 & 7 & 10 & 13 & 16 & 19 & 22 \\
\hline 9 & 12 & 15 & 18 & 21 & 24 & 3 & 6 \\
\hline 17 & 20 & 23 & 2 & 5 & 8 & 11 & 14 \\
\hline
\end{tabular}
表中每行相邻两数所代表的点所夹的弧长等于 3(认为同一行首尾两数也相邻), 每列相邻两数所代表的点所夹的弧长等于 8 (认为同一列首尾两数也相邻), 故每列中至多只能取出 1 个数, 8 列至多取出 8 个数, 但一共要取出 8 个数, 故每列恰取出一个数且相邻两列所取的数不同行.
(认为首尾两列也相邻)
仿照例 4 ,若将每列看成正 $n$ 边形的一个顶点 (本例中 $n=8$ ), 每列中第一、三、三行看成 3 种不同的颜色,则这个问题等价于下列问题中 $n=8$ 的情形: 将正 $n$ 边形的 $n$ 个顶点染色,每个顶点任意染成红、蓝、黄三种颜色之一, 使任意相邻两顶点不同色, 则不同的染色方法共有多少种?(约定线段叫做正二边形).
假设共有 $a_n$ 种不同的染色方法, 则类似于例 4 可得 $a_2=3 \times 2, a_n+a_{n-1} =3 \times 2^{n-1}(n \geqslant 3)$ 于是
$$
\begin{aligned}
a_8= & \left(a_8+a_7\right)-\left(a_7+a_6\right)+\left(a_6+a_5\right)-\left(a_5+a_4\right)+ \\
& \left(a_4+a_3\right)-\left(a_3+a_2\right)+a_2 \\
= & 3 \times 2^7-3 \times 2^6+3 \times 2^5-3 \times 2^4+3 \times 2^3-3 \times 2^2+3 \times 2 \\
= & \frac{3 \times 2^7\left[1-\left(-\frac{1}{2}\right)^7\right]}{1-\left(-\frac{1}{2}\right)}=2^8+2=258 \text { (种). }
\end{aligned}
$$
故共有 258 种不同的取法.
%%<REMARK>%%
注1 本题的解答也可由习题 8 中第 5 题的解答内给出的公式令 $n=8$, $m=3$ 而得到.
注2 本题可推广为下列一般性结论:
设 $m, n$ 为大于 1 的正整数, 又 $d=(m, n)$ 表示 $m, n$ 的最大公约数且 $n>d$. 将周长为 $m n$ 的圆周等分为 $m n$ 段, 从 $m n$ 个分点中取 $n$ 个点,使其中任意两点所夹的弧长不等于 $m$ 和 $k n\left(k=1,2, \cdots,\left[\frac{m}{2}\right]\right)$, 记满足要求的 $n$ 点组的不同取法总数为 $A_n(m)$, 则 $A_n(m)=\left[(m-1)^{\frac{n}{d}}+(-1)^{\frac{n}{d}}\right]^d$.
%%PROBLEM_END%%



%%PROBLEM_BEGIN%%
%%<PROBLEM>%%
例6. 设 $a_1, a_2, \cdots, a_n$ 是正整数 $1,2, \cdots, n$ 的任意排列, 设 $f(n)$ 是满足下述条件的排列的个数:
(1) $a_1=1$ ;(2) $\left|a_i-a_{i+1}\right| \leqslant 2, i=1,2, \cdots, n-1$.
试问 $f(2011)$ 能否被 3 整除?
%%<SOLUTION>%%
解:我们称满足条件 (1) (2) 的排列 $a_1, a_2, \cdots, a_n$ 为好排列, 因为 $a_1=$ 1 , 故由 (2)得 $a_2=2$ 或 3 .
(1) 若 $a_2=2$, 则 $a_2, a_3, \cdots, a_n$ 的各项减去 1 后是 $n-1$ 项的好排列, 其个数为 $f(n-1)$;
(2) 若 $a_2=3, a_3=2$, 则必有 $a_3=4$, 故 $a_4, a_5, \cdots, a_n$ 的各项减去 3 以后是 $n-3$ 项的好排列, 其个数为 $f(n-3)$;
(3) 若 $a_2=3, a_3 \geqslant 4$, 设 $a_{k+1}$ 是该排列中第一个出现的偶数,则前 $k$ 个数应是 $1,3,5, \cdots,(2 k-1), a_{k+1}$ 应是 $2 k$ 或 $2 k-2$, 因此 $a_k$ 与 $a_{k+1}$ 是相邻整数.
由条件 (2) 知道, 排在 $a_{k+1}$ 后面的各数, 要么都小于它, 要么都大于它, 因为 2 在 $a_{k+1}$ 后面,故 $a_{k+2}, \cdots, a_n$ 均小于 $a_{k+1}$. 这只有一种可能,即先依递增顺序排出所有 $\leqslant n$ 的奇数再依递减次序依次排出所有 $\leqslant n$ 的正偶数.
综上所述,有递推关系
$$
f(n)=f(n-1)+f(n-3)+1(n \geqslant 4),
$$
用枚举法易算出 $f(1)=1, f(2)=1, f(3)=2$, 由上述递推关系便可算出数列 $\{f(n)\}$ 模 3 的余数依次为
$$
1,1,2,1,0,0,2,0,1,1,2,1, \cdots
$$
这个余数列以 8 为周期 (不难用数学归纳法证明这一点), 而 $2011 \equiv 251 \times 8+$ 3 , 故 $f(2011) \equiv f(3)=2(\bmod 3)$, 所以 $f(2011)$ 不能被 3 整除.
%%PROBLEM_END%%



%%PROBLEM_BEGIN%%
%%<PROBLEM>%%
例7. 是否存在无穷多对不同的正整数对 $(a, b)$ 使 $a \mid b^2+1$ 且 $b \mid a^2+1$.
%%<SOLUTION>%%
分析:先列举一些满足条件的正整数对 $(a, b)$ 如下表:
\begin{tabular}{|c|c|c|c|c|c|c|}
\hline$a$ & 1 & 2 & 5 & 13 & 34 & $\cdots$ \\
\hline$a^2+1$ & 2 & 5 & 26 & 170 & 1157 & $\cdots$ \\
\hline$b$ & 2 & 5 & 13 & 34 & 89 & $\cdots$ \\
\hline$b^2+1$ & 5 & 26 & 170 & 1157 & 7922 & $\cdots$ \\
\hline
\end{tabular}
经过观察可以看出 $a, b$ 的取值恰是下列递推数列的相邻两项: $a_1==1, a_2= 2, a_{n+2}=3 a_{n+1}-a_n\left(n \in \mathbf{N}_{+}\right)$(这个递推关系也可这样求出, 先设 $a_{n+2}= p a_{n+1}+q a_n$, 再用 $a_1=1, a_2=2, a_3=5, a_4=13$ 代入便可求出 $p=3, q= -1)$. 并且 $\left\{a_n\right\}$ 满足 $a_{n+1}^2+1=a_n a_{n+2}$.
解首先证明下列引理.
引理 1 设 $a_1=1, a_2=2, a_{n+2}=3 a_{n+1}-a_n\left(n \in \mathbf{N}_{+}\right)$, 则 $a_{n+1}^2+1=a_n a_{n+2}$.
引理 1 的证明 $n=1$ 时, $a_3=3 a_2-a_1=5, a_2^2+1=2^2+1=5=a_1 a_3$.
设 $n=k$ 时 $a_{k+1}^2+1=a_k a_{k+2}$, 那么 $n=k+1$ 时,
$$
\begin{aligned}
a_{k+2}^2+1 & =a_{k+2}\left(3 a_{k+1}-a_k\right)+1 \\
& =3 a_{k+2} a_{k+1}-\left(a_{k+2} a_k-1\right)
\end{aligned}
$$
$$
\begin{aligned}
& =3 a_{k+2} a_{k+1}-a_{k+1}^2 \\
& =a_{k+1}\left(3 a_{k+2}-a_{k+1}\right) \\
& =a_{k+1} a_{k+3} .
\end{aligned}
$$
这就完成了引理 1 的证明.
取 $a=a_{n+1}, b=a_{n+2}\left(n \in \mathbf{N}_{+}\right)$, 于是这样的 $(a, b)$ 有无穷多个不同的对, 且满足 $a^2+1=a_{n+1}^2+1=a_n a_{n+2}=a_n b$ 被 $b$ 整除并且 $b^2+1=a_{n+2}^2+1= a_{n+1} a_{n+3}=a a_{n+3}$ 被 $a$ 整除.
注:熟悉斐波那契数列的读者不难发现 $a, b$ 的取值数列恰是下列斐波那契数列的奇数项:
$$
1,1,2,3,5,8,13,21,34,55,89, \cdots
$$
由此可得本题的另一解法.
%%PROBLEM_END%%



%%PROBLEM_BEGIN%%
%%<PROBLEM>%%
例7. 是否存在无穷多对不同的正整数对 $(a, b)$ 使 $a \mid b^2+1$ 且 $b \mid a^2+1$.
%%<SOLUTION>%%
解法二先证下列引理.
引理 2 设 $f_1=1, f_2=1, f_{n+2}=f_{n+1}+f_n\left(n \in \mathbf{N}_{+}\right)$, 则
$$
f_{2 n+1}^2+1=f_{2 n-1} f_{2 n+3} \text {. }
$$
事实上, 因 $f_{2 n+1}=f_{2 n}+f_{2 n-1}=f_{2 n-1}+f_{2 n-2}+f_{2 n-1}=2 f_{2 n-1}+\left(f_{2 n-1}-\right. \left.f_{2 n-3}\right)=3 f_{2 n-1}-f_{2 n-3}$. 于是同引理 1 可证引理 2 结论成立(只要令 $a_n= f_{2 n-1}$, 由引理 1 便可推出引理 2), 再取 $a=f_{2 n+1}, b=f_{2 n+3}\left(n \in \mathbf{N}_{+}\right)$便知存在无穷多对不同的正整数 $a, b$ 满足 $a \mid b^2+1$ 且 $b \mid a^2+1$.
%%PROBLEM_END%%



%%PROBLEM_BEGIN%%
%%<PROBLEM>%%
例8. 是否存在无穷多个三角形, 其三边的长是互素的正整数, 面积是完全平方数?
%%<SOLUTION>%%
分析:与解若这种三角形存在, 设其三边长为 $x, y, z(x, y, z$ 为互素正整数), 记 $p=-\frac{1}{2}(x+y+z)$, 则三角形面积为
$$
S=\sqrt{p(p-x)(p-y)(p-z)}, \label{eq1}
$$
为了简单起见, 不妨令 $p-z=1$, 即 $z=x+y-2, p=x+y-1, p-x= y-1, p-y=x-1$, 则式\ref{eq1}化为
$$
S=\sqrt{(x-1)(y-1)(x+y-1)}, \label{eq2}
$$
再令 $y-1=(a-1) x\left(a \in \mathbf{N}_{+}\right)$则 $x+y-1=a x$, 式\ref{eq2}化为
$$
S=\sqrt{(x-1)(a-1) x \cdot a x}=x \sqrt{a(a-1)(x-1)}, \label{eq3}
$$
为了使 $S$ 为正整数,我们自然会令 $x-1=a(a-1)$, 则式\ref{eq3}化为
$$
S=x(x-1)=a(a-1)[a(a-1)+1],
$$
但这不保证 $S$ 为完全平方数, 改为令 $x-1=4 a(a-1)$, 即 $x=(2 a-1)^2$, 则 式\ref{eq3}化为
$$
S=a(2 a-2)(2 a-1)^2 .
$$
故只需要找出无穷多个正整数 $a$,使 $a$ 和 $2 a-2$ 都为完全平方数.
$a=9$ 显然是一个解,一般地只要取 $a$ 为下列递推数列的项即可:
$$
a_1=9,2 a_{n+1}-2=4 a_n\left(2 a_n-2\right) \text {, 即 } a_{n+1}=\left(2 a_n-1\right)^2 \text {. }
$$
这时对一切 $n \in \mathbf{N}_{+}, a_n$ 及 $2 a_n-2$ 皆为完全平方数.
事实上, $a_1=9$ 及 $2 a_1-2=16$ 为完全平方数设 $a_k$ 及 $2 a_k-2$ 为完全平方数, 则 $a_{k+1}=\left(2 a_k-1\right)^2$ 为完全平方数并且 $2 a_{k+1}-2=4 a_k\left(2 a_k-2\right)$ 也为完全平方数, 故对一切 $n \in \mathbf{N}_{+}, a_n$ 及 $2 a_n-2$ 都为完全平方数.
回到原题, 取三角形三边的长分别为
$$
\begin{aligned}
x & =\left(2 a_n-1\right)^2, \\
y & =\left(a_n-1\right) x+1=\left(a_n-1\right)\left(2 a_n-1\right)^2+1, \\
z & =x+y-2=\left(2 a_n-1\right)^2+\left(a_n-1\right)\left(2 a_n-1\right)^2+1-2 \\
& =a_n\left(2 a_n-1\right)^2-1 .
\end{aligned}
$$
则由 $(x, y)=1,(x, z)=1,(y, z)=1$ 知 $x, y, z$ 互素, 且
$$
\begin{aligned}
& z-y=\left(2 a_n-1\right)^2-2>0, \\
& y-x=\left(a_n-2\right)\left(2 a_n-1\right)^2+1>0,
\end{aligned}
$$
即 $z>y>x$, 又 $x+y=a_n\left(2 a_n-1\right)^2+1>a_n\left(2 a_n-1\right)^2-1=z$, 故以 $x$, $y, z$ 为边长可构成三角形, 且这个三角形的面积为
$$
S=a_n\left(2 a_n-2\right)\left(2 a_n-1\right)^2 .
$$
它是一个完全平方数, 这就证明了存在无穷多个满足题目条件的三角形.
%%<REMARK>%%
注:从例 7 和例 8 可以看出, 在组合数论和组合几何中, 常常要证明存在无穷多个数组或几何图形满足一定的条件 (这些条件通常与正整数有关). 为此, 我们可先从一些特殊情形或简单情形做起, 进行探索与归纳, 构造对应的递推数列来完成证明.
%%PROBLEM_END%%


