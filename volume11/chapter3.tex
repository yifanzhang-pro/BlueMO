
%%TEXT_BEGIN%%
一、母函数的概念.
设 $f(x)=(1+x)^n$, 由二项式定理,有
$$
f(x)=\sum_{k=0}^n \mathrm{C}_n^k x^k=\mathrm{C}_n^0+\mathrm{C}_n^1 x+\mathrm{C}_n^2 x^2+\cdots+\mathrm{C}_n^n x^n .
$$
这时, $f(x)$ 对应了一个数列 $\left\{\mathrm{C}_n^k, 0 \leqslant k \leqslant n\right\}$, 即生成数列 $\left\{\mathrm{C}_n^k\right\}$, 因此, 我们把函数 $f(x)=(1+x)^n$ 称为数列 $\left\{\mathrm{C}_n^k\right\}$ 的生成函数或母函数.
一般地说, 对于有穷数列
$$
a_0, a_1, a_2, \cdots, a_n,
$$
多项式 $f(x)=\sum_{k=0}^n a_k x^k=a_0+a_1 x+\cdots+a_k x^k+\cdots+a_n x^n$ 称为数列 $\left\{a_k\right\}$ 的母函数.
更一般地, 对于无穷数列
$$
a_0, a_1, \cdots, a_n, \cdots
$$
我们称下列形式幂级数
$$
f(x)=\sum_{n=0}^{\infty} a_n x^n=a_0+a_1 x+\cdots+a_n x^n+\cdots .
$$
为无穷数列 $\left\{a_n\right\}$ 的母函数.
关于形式幂级数我们作如下的规定: 设 $f(x)=\sum_{n=0}^{\infty} a_n x^n, g(x)= \sum_{n=0}^{\infty} b_n x^n$ 是两个形式幂级数,我们规定
(1) $f(x)=g(x)$, 当且仅当 $a_n==b_n(n=0,1,2, \cdots)$ ;
(2) $f(x) \pm g(x)=\sum_{n=0}^{\infty}\left(a_n \pm b_n\right) x^n$;
(3) $\alpha f(x)=\sum_{n=0}^{\infty}\left(\alpha a_n\right) x^n$ ( $\alpha$ 为常数);
(4) $f(x) g(x)=\sum_{n=0}^{\infty} c_n x^n$, 其中 $c_n=\sum_{k=0}^n a_k b_{n-k}, n=0,1,2, \cdots$.
二、几个重要公式在应用母函数解题时, 除了二项式定理以外, 还要用到下列几个公式: 公式 I (无穷递缩等比数列求和公式)
$$
\frac{1}{1-x}=\sum_{n=0}^{\infty} x^n=1+x+x^2+\cdots+x^n+\cdots(|x|<1) .
$$
公式 II $\quad(1-x)^{-k}=\sum_{n=0}^{\infty} \mathrm{C}_{n+k-1}^{k-1} x^n=1+\mathrm{C}_k^{k-1} x+\mathrm{C}_{k+1}^{k-1} x^2+\cdots+\mathrm{C}_{n+k-1}^{k-1} x^n+ \cdots$ ( $k$ 为正整数, $|x|<1$ ).
公式 II 可由公式 I 两边求 $k-1$ 阶导数后除以 $(k-1)$ ! 而得到.
%%TEXT_END%%



%%PROBLEM_BEGIN%%
%%<PROBLEM>%%
例1. 已知 $a_0=-1, a_1=1, a_n=2 a_{n-1}+3 a_{n-2}+3^n(n \geqslant 2)$, 求 $a_n$.
%%<SOLUTION>%%
解:设 $f(x)=a_0+a_1 x+a_2 x^2+\cdots+a_n x^n+\cdots$,
则
$$
\begin{aligned}
& -2 x f(x)=-2 a_0 x-2 a_1 x^2-\cdots-2 a_{n-1} x^n+\cdots, \\
& -3 x^2 f(x)=\quad-3 a_0 x^2-\cdots-3 a_{n-2} x^n+\cdots, \\
&
\end{aligned}
$$
将以上三式相加, 并利用 $a_0=-1, a_1=1, a_n=2 a_{n-1}+3 a_{n-2}+3^n$, 得
$$
\begin{gathered}
\left(1-2 x-3 x^2\right) f(x)=-1+3 x+3^2 x^2+\cdots+3^n x^n+\cdots \\
=-1+\frac{3 x}{1-3 x}=\frac{6 x-1}{1-3 x} . \\
f(x)=\frac{6 x-1}{(1+x)(1-3 x)^2}=\frac{A}{1+x}+\frac{B}{(1-3 x)^2}+\frac{C}{1-3 x} . \label{eq1}
\end{gathered}
$$
\ref{eq1} 式两边乘 $1+x$ 后, 令 $x=-1$ 得
$$
A=\left.\frac{6 x-1}{(1-3 x)^2}\right|_{x=-1}=-\frac{7}{16},
$$
\ref{eq1} 式两边乘 $(1-3 x)^2$ 后, 令 $x=\frac{1}{3}$ 得
$$
B=\left.\frac{6 x-1}{1+x}\right|_{x=\frac{1}{3}}=\frac{3}{4},
$$
\ref{eq1} 式两边乘 $x$ 后, 令 $x \rightarrow \infty$ 得
$$
\begin{aligned}
0 & =\lim _{x \rightarrow \infty} \frac{x(6 x-1)}{(1+x)(1-3 x)^2} \\
& =\lim _{x \rightarrow \infty}\left(\frac{A x}{1+x}+\frac{B x}{(1-3 x)^2}+\frac{C x}{1-3 x}\right) \\
& =A-\frac{1}{3} C .
\end{aligned}
$$
所以 $C=3 A=-\frac{21}{16}$. 于是
$$
\begin{aligned}
f(x) & =-\frac{7}{16(1+x)}+\frac{3}{4(1-3 x)^2}-\frac{21}{16(1-3 x)} \\
& =-\frac{7}{16} \sum_{n=0}^{\infty}(-1)^n x^n+\frac{3}{4} \sum_{n=0}^{\infty} \mathrm{C}_{n+1}^1 3^n x^n-\frac{21}{16} \sum_{n=0}^{\infty} 3^n x^n \\
& =\sum_{n=0}^{\infty}\left[\frac{(4 n-3) \cdot 3^{n+1}-7(-1)^n}{16}\right] x^n .
\end{aligned}
$$
故得
$$
a_n=\frac{1}{16}\left[(4 n-3) \cdot 3^{n+1}-7(-1)^n\right] .
$$
%%PROBLEM_END%%



%%PROBLEM_BEGIN%%
%%<PROBLEM>%%
例2. 证明: 对一切正整数 $n$, 有
$$
\sum_{i=0}^n \mathrm{C}_{2 n+1}^{2 i} \mathrm{C}_{2 i}^i 2^{2 n-2 i+1}=\mathrm{C}_{4 n+2}^{2 n+1} \text {. }
$$
%%<SOLUTION>%%
证明:一方面 $(1+x)^{4 n+2}=\sum_{i=0}^{4 n+2} \mathrm{C}_{4 n+2}^i x^i$ 中 $x^{2 n+1}$ 的系数为 $\mathrm{C}_{4 n+2}^{2 n+1}$, 另一方面
$$
\begin{aligned}
(1+x)^{4 n+2} & =\left[\left(1+x^2\right)+2 x\right]^{2 n+1} \\
& =\sum_{k=0}^{2 n+1} \mathrm{C}_{2 n+1}^k\left(1+x^2\right)^k(2 x)^{2 n+1-k} \\
& =\sum_{k=0}^{2 n+1} \mathrm{C}_{2 n+1}^k 2^{2 n+1-k} \cdot x^{2 n+1-k}\left(\sum_{j=0}^k \mathrm{C}_k^j x^{2 j}\right)
\end{aligned}
$$
中 $x^{2 n+1}$ 的系数为 $\sum_{i=0}^n \mathrm{C}_{2 n+1}^{2 i} 2^{2 n+1-2 i} \cdot \mathrm{C}_{2 i}^i$, 所以
$$
\sum_{i=0}^n \mathrm{C}_{2 n+1}^{2 i} \mathrm{C}_{2 i}^i 2^{2 n-2 i+1}=\mathrm{C}_{4 n+2}^{2 n+1} .
$$
%%PROBLEM_END%%



%%PROBLEM_BEGIN%%
%%<PROBLEM>%%
例3. 证明: $\sum_{k=0}^{\left[\frac{n-1}{2}\right]}(-1)^k \mathrm{C}_{n+1}^k \mathrm{C}_{2 n-2 k-1}^n=\frac{1}{2} n(n+1)(n \geqslant 1)$.
%%<SOLUTION>%%
证明:一方面 $(1+x)^{n+1}=\sum_{k=0}^{n+1} \mathrm{C}_{n+1}^k x^k$ 中 $x^{n-1}$ 的系数为 $\mathrm{C}_{n+1}^{n-1}=\mathrm{C}_{n+1}^2= \frac{1}{2} n(n+1)$, 另一方面, $(1+x)^{n+1}=\frac{\left(1-x^2\right)^{n+1}}{(1-x)^{n+1}}=\left(\sum_{k=0}^n(-1)^k \mathrm{C}_{n+1}^k x^{2 k}\right)\left(\sum_{j=0}^{\infty} \mathrm{C}_{n+j}^n x^j\right)$ 中 $x^{n-1}$ 的系数为 $\sum_{k=0}^{\left[\frac{n-1}{2}\right]}(-1)^k \mathrm{C}_{n+1}^k \mathrm{C}_{n+(n-1-2 k)}^n=\sum_{k=0}^{\left[\frac{n-1}{2}\right]}(-1)^k \mathrm{C}_{n+1}^k \mathrm{C}_{2 n-2 k-1}^n$, 所以 $\sum_{k=0}^{\left[\frac{n-1}{2}\right]}(-1)^k \mathrm{C}_{n+1}^k \mathrm{C}_{2 n-2 k-1}^n=\frac{1}{2} n(n+1)$.
%%<REMARK>%%
注:当恒等式中流动标号只出现在组合数 $\mathrm{C}_m^n$ 的下位 $m$ 时, 可考虑用公式 $\frac{1}{(1-x)^{n+1}}=\sum_{j=0}^{\infty} \mathrm{C}_{n+j}^n x^j$.
%%PROBLEM_END%%



%%PROBLEM_BEGIN%%
%%<PROBLEM>%%
例4. 一副三色牌, 共有纸牌 32 张.
其中红、黄、蓝每种颜色的牌各 10 张, 编号分别是 $1,2, \cdots, 10$; 另有大、小王牌各一张, 编号均为 0. 从这副牌中任取若干张牌.
然后按如下规则计算分值: 每张编号为 $k$ 的牌计为 $2^k$ 分.
若它们的分值之和为 $n$ ( $n \leqslant 2020$ 为给定的正整数), 就称这些牌为一个"好"牌组.
试求"好"牌组的个数.
%%<SOLUTION>%%
解:对 $n \in\{1,2, \cdots, 2020\}$, 用 $a_n$ 表示分值之和为 $n$ 的牌组数目, 则由题意知 $a_n$ 等于函数
$$
f(x)=\left(1+x^{2^0}\right)^2\left(1+x^{2^1}\right)^3\left(1+x^{2^2}\right)^3 \cdots\left(1+x^{2^{20}}\right)^3
$$
的展开式中 $x^n$ 的系数 (约定 $|x|<1$ ), 由于
$$
\begin{aligned}
f(x) & =\frac{1}{1+x}\left\{\left(1+x^{2^0}\right)\left(1+x^{2^1}\right)\left(1+x^{2^2}\right) \cdots\left(1+x^{2^{10}}\right)\right\}^3 \\
& =\frac{1}{(1+x)(1-x)^3}\left(1-x^{2^{11}}\right)^3,
\end{aligned}
$$
而 $n \leqslant 2020<2^{11}$, 所以 $a_n$ 等于
$$
g(x)=\frac{1}{(1+x)(1-x)^3},
$$
展开式中 $x^n$ 的系数, 由于
$$
g(x)=\frac{1}{\left(1-x^2\right)(1-x)^2}=\left(\sum_{i=0}^{\infty} x^{2 i}\right)\left(\sum_{j=0}^{\infty} \mathrm{C}_{j+1}^1 x^j\right),
$$
故当 $n=2 k$ 时, $x^{2 k}$ 的系数为
$$
\begin{aligned}
a_{2 k} & =\sum_{i=0}^k \mathrm{C}_{2 k-2 i+1}^1=\sum_{i=0}^k[2(k-i)+1] \\
& =(2 k+1)+(2 k-1)+(2 k-3)+\cdots+5+3+1 \\
& =(k+1)^2=\frac{(2 k+2)^2}{4}=\left[\frac{(n+2)^2}{4}\right] ;
\end{aligned}
$$
当 $n=2 k-1$ 时, $x^{2 k-1}$ 的系数为
$$
\begin{aligned}
a_{2 k-1} & =\sum_{i=0}^k \mathrm{C}_{(2 k-1)-2 i+1}^1=\sum_{i=0}^k 2(k-i) \\
& =2 k+(2 k-2)+(2 k-4)+\cdots+6+4+2=k(k+1) \\
& =\frac{(2 k+1)^2-1}{4}=\left[\frac{(2 k+1)^2}{4}\right]=\left[\frac{(n+2)^2}{4}\right] .
\end{aligned}
$$
从而,所求的"好"牌组的个数为
$$
a_n=\left[\frac{(n+2)^2}{4}\right] . \quad\left(n \in \mathbf{N}_{+} \text {且 } n \leqslant 2020\right)
$$
(特别,当 $n=2004$ 时, "好"牌组的个数为 $a_{2004}=1003^2$. )
%%<REMARK>%%
注:例 4 中也可用下列方法求 $g(x)=\frac{1}{(1+x)(1-x)^3}$ 的展开式中 $x^n$ 的系数:令
$$
g(x)=\frac{1}{(1+x)(1-x)^3}=\frac{A}{1+x}+\frac{B}{(1-x)^3}+\frac{C}{(1-x)^2}+\frac{D}{1-x},
$$
则 $A=\left.\frac{1}{(1-x)^3}\right|_{x=-1}=\frac{1}{8}, B=\left.\frac{1}{1+x}\right|_{x=1}=\frac{1}{2}$, 于是
$$
\begin{aligned}
\frac{C}{(1-x)^2}+\frac{D}{1-x} & =\frac{1}{(1+x)(1-x)^3}-\frac{1}{8(1+x)}-\frac{1}{2(1-x)^3} \\
& =\frac{8-(1-x)^3-4(1+x)}{8(1+x)(1-x)^3}=\frac{x^3-3 x^2-x+3}{8(1+x)(1-x)^3} \\
& =\frac{(x-3)(x+1)(x-1)}{8(1+x)(1-x)^3}=\frac{3-x}{8(1-x)^2}=\frac{2+(1-x)}{8(1-x)^2} \\
& =\frac{1}{4(1-x)^2}+\frac{1}{8(1-x)},
\end{aligned}
$$
即 $C=\frac{1}{4}, D=\frac{1}{8}$, 所以
$$
g(x)=\frac{1}{8(1+x)}+\frac{1}{2(1-x)^3}+\frac{1}{4(1-x)^2}+\frac{1}{8(1-x)}
$$
$$
\begin{aligned}
& =\frac{1}{8} \sum_{k=0}^{\infty}(-1)^k x^k+\frac{1}{2} \sum_{k=0}^{\infty} \mathrm{C}_{k+2}^2 x^k+\frac{1}{4} \sum_{k=0}^{\infty} \mathrm{C}_{k+1}^1 x^k+\frac{1}{8} \sum_{k=0}^{\infty} x^k \\
& =\frac{1}{8} \sum_{k=0}^{\infty}\left\{(-1)^k+4 \mathrm{C}_{k+2}^2+2 \mathrm{C}_{k+1}^1+1\right\} x^k,
\end{aligned}
$$
其中 $x^n$ 的系数为
$$
\begin{aligned}
a_n & =\frac{1}{8}\left\{(-1)^n+4 \mathrm{C}_{n+2}^2+2 \mathrm{C}_{n+1}^1+1\right\} \\
& =\frac{1}{8}\left\{(-1)^n+2(n+2)(n+1)+2(n+1)+1\right\} \\
& =\frac{2 n^2+8 n+7+(-1)^n}{8} \\
& =\frac{(n+2)^2}{4}-\frac{1-(-1)^n}{8}=\left[\frac{(n+2)^2}{4}\right] .
\end{aligned}
$$
%%PROBLEM_END%%


