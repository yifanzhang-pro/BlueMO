
%%PROBLEM_BEGIN%%
%%<PROBLEM>%%
问题1. 对于有限集 $A$, 存在函数 $f: \mathbf{N}_{+} \rightarrow A$, 具有以下性质: 若 $i, j \in \mathbf{N}_{+}$, 且 $|i-j|$ 为素数,则 $f(i) \neq f(j)$, 问集合 $A$ 中最少有几个元素?
%%<SOLUTION>%%
因 1,3,6,8 中任意两个之差的绝对值为素数, 由题意知 $f(1)$, $f(3), f(6), f(8)$ 为 $A$ 中两两不同的数, 从而 $|A| \geqslant 4$, 另一方面, 假设 $A= \{0,1,2,3\}$ 且对任意 $x \in \mathbf{N}_{+}, x=4 k+r(k \in \mathbf{N}, r=0,1,2,3)$, 定义 $f(x)=r$. 于是, 对任意 $x, y \in \mathbf{N}_{+}$, 若当 $|x-y|$ 为素数时有 $f(x)=f(y)$, 则 4||$x-y \mid$, 这与 $|x-y|$ 为素数矛盾.
故所作 $f: \mathbf{N}_{+} \rightarrow A$ 满足题目条件且 $|A|=4$, 所以 $A$ 中最少有 4 个元素.
%%PROBLEM_END%%



%%PROBLEM_BEGIN%%
%%<PROBLEM>%%
问题2. 设 $M$ 是有限数集, 若已知 $M$ 的任何三个元素中总存在两个数, 它们的和属于 $M$,试问 $M$ 中最多有多少个数?
%%<SOLUTION>%%
首先易验证 $M=\{-3,-2,-1,0,1,2,3\}$ 中任取 3 个数, 其中必有两个数之和属于 $M$. 另一方面, 设 $M=\left\{a_1, a_2, \cdots, a_m\right\}, a_1>a_2>\cdots> a_m$, 且 $m \geqslant 8$. 因为每个数乘 -1 不会改变 $M$ 是否满足题目条件的性质, 故可设 $a_4>0$. 于是 $a_1+a_2>a_1+a_3>a_1+a_4>a_1$, 从而 $a_1+a_2, a_1+a_3, a_1+a_4$ 都不属于 $M$, 并且 $a_2+a_3$ 与 $a_2+a_4$ 不可能都属于 $M$ (因 $a_2+a_3>a_2, a_2+ a_4>a_2$ 且 $a_2+a_3 \neq a_2+a_4$, 而 $M$ 中只有一个数 $\left.a_1>a_2\right)$, 这样 $\left(a_1, a_2, a_3\right)$ 或 $\left(a_1, a_2, a_4\right)$ 至少有一组中任何两个数之和不属于 $M$. 即 $m \geqslant 8$ 时, $M$ 不满足题目要求.
综上可知 $M$ 中最多有 7 个数.
%%PROBLEM_END%%



%%PROBLEM_BEGIN%%
%%<PROBLEM>%%
问题3. 设 $n$ 为给定的正整数, 求最大正整数 $k$, 使得存在三个由非负整数组成的集合 $A=\left\{x_1, x_2, \cdots, x_k\right\}, B=\left\{y_1, y_2, \cdots, y_k\right\}, C=\left\{z_1, z_2, \cdots, z_k\right\}$ 满足: 对任意 $1 \leqslant j \leqslant k$, 都有 $x_j+y_j+z_j=n$.
%%<SOLUTION>%%
由已知条件可知 $k n=\sum_{i=1}^k\left(x_i+y_i+z_i\right) \geqslant 3 \sum_{i=1}^k(i-1)=\frac{3 k(k-1)}{2}$, 因此, $k \leqslant\left[\frac{2 n}{3}\right]+1$. 下面给出一个 $k=\left[\frac{2 n}{3}\right]+1$ 的例子: 令 $m \in \mathbf{N}_{+}$, 若 $n= 3 m$, 则 $k=2 m+1$, 对 $1 \leqslant j \leqslant m+1$, 令 $x_j=j-1, y_j=m+j-1, z_j= 2 m-2 j+2$; 对 $m+2 \leqslant j \leqslant 2 m+1$, 令 $x_j=j-1, y_j=j-m-2, z_j=4 m- 2 j+3$ 即可; 若 $n=3 m+1$, 则 $k=2 m+1$, 对 $1 \leqslant j \leqslant m$, 令 $x_j=j-1, y_j= m+j, z_j=2 m-2 j+2$; 对 $m+1 \leqslant j \leqslant 2 m$, 令 $x_j=j+1, y_j=j-m-1$, $\boldsymbol{z}_j=4 m+1-2 j$; 而 $x_{2 m+1}=m, y_{2 m+1}=2 m+1, z_{2 m+1}=0$; 当 $n=3 m+2$ 时, $k=2 m+2$, 对 $1 \leqslant j \leqslant m+1$ 令 $x_j=j-1, y_j=m+j, z_j=2 m-2 j+3$ ; 对 $m+2 \leqslant j \leqslant 2 m+1$, 令 $x_j=j, y_j=j-m-2, z_j=4 m-2 j+4$, 而 $x_{2 m+2}=2 m+2, y_{2 m+2}=m, z_{2 m+2}=0$ 即可.
%%PROBLEM_END%%



%%PROBLEM_BEGIN%%
%%<PROBLEM>%%
问题4. 在一次由 $n$ 个是非题构成的竞赛中, 有 8 名选手参加.
已知对任意一对是非题 $(A, B)$ 而言, (称 ( $A, B)$ 为有序对), 恰有两人的答案为 (对, 对); 恰有两人的答案为 (对, 错); 恰有两人的答案为 (错, 对); 恰有两人的答案为 (错、错), 求 $n$ 的最大值,并说明理由.
%%<SOLUTION>%%
设 8 名选手为 $p_1, p_2, \cdots, p_8, n$ 道是非题为 $A_1, A_2, \cdots, A_n$. 作 $8 \times n$ 表格,其中第 $i$ 行第 $j$ 列处的数为 $x_{i j}=\left\{\begin{array}{l}1, p_i \text { 对 } A_j \text { 的答案是"对", } \\ 0, p_i \text { 对 } A_j \text { 的答案是"错" }\end{array}(i=1, 2, \cdots, 8 ; j=1,2, \cdots, n)\right.$ 于是, 第 $i$ 行各数之和为 $a_i=\sum_{i=1}^n x_{i j}$ 表示 $p_i$ 对 $A_1, A_2, \cdots, A_n$ 的答案为 "对" 的个数.
又依题意, 每列中恰有 4 个 1 和 4 个零,所以 $\sum_{i=1}^8 x_{i j}=4$. 从而 $\sum_{i=1}^8 a_i=\sum_{i=1}^8 \sum_{j=1}^n x_{i j}=\sum_{j=1}^n \sum_{i=1}^8 x_{i j}=4 n \cdots$ (1). 
注意到, 由已知条件知, 对表中任何一列, 将其中 1 全部换为 0 , 并且将 0 全部换为 1 后, 表中各数仍具有题设性质.
故不失一般性, 可设表中第一行的数全等于 1 , 从而 $a_1=n, \sum_{i=2}^8 a_i=3 n$. 如果 $p_i$ 对题目 $\left(A_j, A_k\right)(j \neq k)$ 的答案为 (对, 对), 则将 $\left(p_i, A_j, A_k\right)$ 组成三元组.
依题目条件可得这种三元组的个数既为 $\sum_{i=1}^8 \mathrm{C}_{a_i}^2$, 又为 $2 \mathrm{C}_n^2=n(n-1)$. 
于是(并利用(1)和柯西不等式) $n(n-1)=\sum_{i=1}^8 \mathrm{C}_{a_i}^2=\mathrm{C}_{a_1}^2+\sum_{i=2}^8 \mathrm{C}_{a_i}^2=\mathrm{C}_n^2+\frac{1}{2}\left(\sum_{i=2}^8 a_i^2-\sum_{i=2}^8 a_i\right) \geqslant \frac{1}{2} n(n-1)+\frac{1}{2}\left[\frac{1}{7}\left(\sum_{i=2}^8 a_i\right)^2-\sum_{i=2}^8 a_i\right]=\frac{1}{2} n(n-1)+\frac{1}{14}\left[(3 n)^2-7(3 n)\right]=\frac{2}{7} n(4 n-7)$, 解得 $n \leqslant 7$. 
如表中所示例子表明 $n$ 可以等于 7 . 故所求 $n$ 的最值为 7 .
\begin{tabular}{|c|c|c|c|c|c|c|c|}
\hline & $A_1$ & $A_2$ & $A_3$ & $A_4$ & $A_5$ & $A_6$ & $A_7$ \\
\hline$p_1$ & 1 & 1 & 1 & 1 & 1 & 1 & 1 \\
\hline$p_2$ & 1 & 0 & 0 & 0 & 0 & 1 & 1 \\
\hline$p_3$ & 1 & 0 & 0 & 1 & 1 & 0 & 0 \\
\hline$p_4$ & 1 & 1 & 1 & 0 & 0 & 0 & 0 \\
\hline$p_5$ & 0 & 1 & 0 & 1 & 0 & 1 & 0 \\
\hline$p_6$ & 0 & 1 & 0 & 0 & 1 & 0 & 1 \\
\hline$p_7$ & 0 & 0 & 1 & 1 & 0 & 0 & 1 \\
\hline$p_8$ & 0 & 0 & 1 & 0 & 1 & 1 & 0 \\
\hline
\end{tabular}
%%PROBLEM_END%%



%%PROBLEM_BEGIN%%
%%<PROBLEM>%%
问题5. 在一个圆周上给定 12 个红点, 求 $n$ 的最小值, 使得存在以红点为顶点的 $n$ 个三角形满足: 以红点为端点的每条弦, 都是其中某个三点均为红点的三角形的一条边.
%%<SOLUTION>%%
设红点集为 $A=\left\{A_1, A_2, \cdots, A_{12}\right\}$, 过 $A_1$ 的弦有 11 条, 而任意含 $A_1$ 的三角形恰含过 $A_1$ 的两条弦,故这 11 条过 $A_1$ 的弦至少要分布于 6 个含顶点 $A_1$ 的三角形中, 同理知, 过点 $A_i(i=2,3, \cdots, 12)$ 的弦, 也各要分布在 6 个含顶点 $A_i$ 的三角形中, 这样就需要 $12 \times 6=72$ (个) 三角形, 而每个三角形含有三个顶点, 故都被重复计算了三次.
因此, 至少需要 $\frac{72}{3}=24$ 个不同的三角形.
另一方面, 下面实例表明 $n=24$ 可以被取到.
不失一般性, 考虑周长为 12 的圆周, 其十二等分点为红点, 以红点为端点的弦共有 $\mathrm{C}_{12}^2=66$ (条). 如果弦所对的劣弧长为 $k$, 就称该弦的刻度为 $k$, 于是红端点的弦的刻度的弦只有 6 种, 其中刻度为 $1,2,3,4,5$ 的弦各有 12 条, 刻度为 6 的弦有 6 条.
如果刻度为 $a$, $b, c$ 的三条弦构成一个三角形的三条边, 则必须满足下列两个条件之一:或者 $a+b=c$, 或者 $a+b+c==12$. 下面是刻度组的一种搭配: 以 $(1,2,3),(1,5$, $6),(2,3,5)$ 型各 6 个, $(4,4,4)$ 型 4 个, 这时恰好得 66 条弦, 其中含刻度为 $1,2,3,4,5$ 的弦各 12 条, 刻度为 6 的弦恰有 6 条.
今构造如下: 先作刻度为 $(1,2,3) ,(1,5,6) ,(2,3,5)$ 型的三角形各 6 个, $(4,4,4)$ 型的三角形三个,再用三个 $(2,4,6)$ 型的三角形来补充.
刻度为 $(1,2,3)$ 型的 6 个, 其顶点标号为 $\{2,3,5\},\{4,5,7\},\{6,7,9\},\{8,9,11\},\{10,11,1\},\{12,1,3\}$ ; 刻度为 $(1,5,6)$ 型的 6 个, 其顶点标号为 $\{1,2,7\},\{3,4,9\},\{5,6,11\},\{7$, $8,1\},\{9,10,3\},\{11,12,5\}$; 刻度为 $(2,3,5)$ 型的 6 个, 其顶点标号为 $\{2,4$, $11\},\{4,6,1\},\{6,8,3\},\{8,10,5\},\{10,12,7\},\{12,2,9\}$; 刻度为 $(4,4,4)$型的 3 个,其顶点标号为 $\{1,5,9\},\{2,6,10\},\{3,7,11\}$; 刻度为 $(2,4,6)$ 型的 3 个, 其顶点标号为 $\{4,6,12\},\{8,10,4\},\{12,2,8\}$. (注意, 每种情况下的其余三角形都可由前一个三角形绕圆心适当旋转而得到). 这样共得 24 个三角形,且满足题目条件.
综上可知,所求 $n$ 的最小值为 24 .
%%PROBLEM_END%%



%%PROBLEM_BEGIN%%
%%<PROBLEM>%%
问题6. 设 $Z$ 是一个 56 元集合,求最小正整数 $n$,使得 $Z$ 的任意 15 个子集, 只要他们中任何 7 个的并的元素个数都不小于 $n$, 则这 15 个子集中一定存在 3 个,它们的交非空.
%%<SOLUTION>%%
设 $X=\{1,2,3, \cdots, 56\}$, 令 $A_i=\{i, i+7, i+14, i+21, i+28$, $i+35, i+42, i+49\}(i=1,2,3, \cdots, 7), B_j=\{j, j+8, j+16, j+24$, $j+32, j+40, j+48\}(i=1,2,3, \cdots, 8)$. 显然 $\left|A_i\right|=8(1 \leqslant i \leqslant 7)$, $A_i \cap A_j=\varnothing(1 \leqslant i<j \leqslant 7),\left|B_j\right|=7(1 \leqslant j \leqslant 8), B_i \cap B_j=\varnothing(1 \leqslant i<j \leqslant 8)$. $\left|A_i \cap B_j\right|=1(1 \leqslant i \leqslant 7,1 \leqslant j \leqslant 8)$ 于是从 $X$ 的 15 个子集 $A_i(1 \leqslant i \leqslant 7), B_j(1 \leqslant i \leqslant 8)$ 中任取 3 个, 必有 2 个同为 $A_i$ 或者有 2 个同为 $B_j$, 其交为空集, 并且对其中任意 7 个子集: $A_{i_1}, A_{i_2}, \cdots, A_{i_s}, B_{j_1}, B_{j_2}, \cdots$, $B_{j_t}(s+t=7)$ 有 $\left|A_{i_1} \cup A_{i_2} \cup \cdots \cup A_{i_s} \cup B_{j_1} \cup B_{j_2} \cup \cdots \cup B_{j_t}\right|=\left|A_{i_1}\right|+ \left|A_{i_2}\right|+\cdots+\left|A_{i_s}\right|+\left|B_{j_1}\right|+\left|B_{j_2}\right|+\cdots+\left|B_{j_t}\right|-s t=8 s+7 t-s t=8 s+ 7(7-s)-s(7-s)=(s-3)^2+40 \geqslant 40$, 故所求最小正整数 $n \geqslant 41$. 其次, 我们证明 $n=41$ 符合条件.
用反证法.
假设存在 $X$ 的 15 个子集, 它们中任何 7 个子集的并不少于 41 个元素, 而任何 3 个的交为空集.
于是每个元素至多属于 2 个子集.
不妨设每个元素恰属于 2 个子集(否则在一些子集中适当添加一些元素, 上述各条件仍成立). 由抽屈原理, 必有一个子集, 设为 $A$, 至少含有 $\left[\frac{2 \times 56}{15}-1\right]+1=8$ 个元素.
又设其他 14 个子集为 $A_1, A_2, \cdots, A_{14}$. 这 14 个子集中任何 7 个子集的并集都至少包含 $X$ 中 41 个元素.
这 14 个子集的所有"7 子集组"一共包含 $X$ 中 $41 \mathrm{C}_{14}^7$ 个元素.
另一方面, 对 $X$ 内任意元 $a$, 若 $a \notin A$, 则 $A_1, A_2, \cdots, A_{14}$ 中有 2 个包含 $a$, 于是 $a$ 被计算了 $\mathrm{C}_{14}^7-\mathrm{C}_{12}^7$ 次; 若 $a \in A$, 则 $A_1, A_2, \cdots, A_{14}$ 中只有 1 个含有 $a$, 于是 $a$ 被计算了 $\mathrm{C}_{14}^7-\mathrm{C}_{13}^7$ 次.
所以 $41 \mathrm{C}_{14}^7 \leqslant(56-|A|)\left(\mathrm{C}_{14}^7-\mathrm{C}_{12}^7\right)+|A|\left(\mathrm{C}_{14}^7-\mathrm{C}_{13}^7\right)=56\left(\mathrm{C}_{14}^7-\mathrm{C}_{12}^7\right)- |A|\left(C_{13}^7-C_{12}^7\right) \leqslant 56\left(C_{14}^7-C_{12}^7\right)-8\left(C_{13}^7-C_{12}^7\right)$, 由此可得 $533 \leqslant 532$, 矛盾.
故 $n=41$ 满足条件.
综上所述, 所求 $n$ 的最小值为 41 .
%%PROBLEM_END%%



%%PROBLEM_BEGIN%%
%%<PROBLEM>%%
问题7. 设平面点集 $P=\left\{P_1, P_2, \cdots, P_{1994}\right\}, P$ 中任意 3 点不共线,将 $P$ 中所有点任意分成 83 组,使每组至少 3 个点且每点恰属于一组, 然后将同一组的任意两点用线段相连,不同组的任意两点不连线段,这样得到一个图案 $G$. 不同的分组方式得到不同的图案, 将图案中以 $P$ 中点为顶点的三角形个数记为 $m(G)$. (1) 求 $m(G)$ 的最小值 $m_0$ ;(2) 设 $G^*$ 是使 $m\left(G^*\right)=m_0$ 的一个图案, 若将 $G^*$ 中线段 (指以 $P$ 中点为端点的线段) 用四种颜色染色, 每条线段恰染一色.
证明: 存在一种染色方案,使 $G^*$ 染色后, 不存在以 $P$ 中点为顶点的三边颜色相同的三角形.
%%<SOLUTION>%%
(1) 因分组方法有限, 故使 $m(G)=m_0$ 的图案 $G$ 存在.
设图案 $G$ 满足 $m(G)=m_0 . G$ 由分组 $X_1, X_2$, $\cdots, X_{83}$ 组成.
其中 $X_i$ 为第 $i$ 组点构成的集合, 并记 $\left|X_i\right|=x_i$, 于是 $x_1+ x_2+\cdots+x_{83}=1994, m_0=\sum_{i=1}^{83} \mathrm{C}_{x_i}^3$.
证明这时, 对任意 $1 \leqslant i<j \leqslant 83$, 有 $\left|x_i-x_j\right| \leqslant 1$. 又因为 $1994=83 \times 24+2=81 \times 24+ 2 \times 25$, 故使 $m(G)=m_0$ 的 $x_1, x_2, \cdots, x_{83}$ 中有 81 等于 24,2 个等于 25 ,所以 $m_0=81 \mathrm{C}_{24}^3+2 \mathrm{C}_{25}^3=168544$.
(2) 由 (1) 知 $G^*$ 可分为 83 个互相没有线段相连的子图 $G_1^*, G_2^*, \cdots$, $G_{83}^*$, 其中 $G_1^*, \cdots, G_{81}^*$ 含 24 个点, $G_{82}^*, G_{83}^*$ 含 25 个点, 设 $G_i^*$ 所含点集为 $X_i(i=1,2, \cdots, 83)$, 对于 $G_{83}^*$, 令 $X_{83}=P_1 \cup P_2 \cup \cdots \cup P_5, P_i \cap P_j= \varnothing(1 \leqslant i<j \leqslant 5)$ 且 $\left|P_i\right|=5(1 \leqslant i \leqslant 5)$. 每个 $P_i$ 用如图(<FilePath:./figures/fig-c15a7-1.png>) 所示方法染色, 而不同的 $P_i$ 与 $P_j$ 所连线段用如图(<FilePath:./figures/fig-c15a7-2.png>) 所示方法染色, 其中 $a, b, c, d$ 表示 4 种不同颜色.
这样染好色的 $G_{83}^*$ 显然不包含三边颜色相同的三角形.
对 $G_{82}^*$ 可同染 $G_{83}^*$ 的方法去染色.
而对 $G_i^*(1 \leqslant i \leqslant 81)$, 可先增加一点并与原 24 个点都连一线段, 然后按染 $G_{83}^*$ 的方法染好色后, 再去掉该点及从该点连出的线段,这样染好色的 $G_i^*$ 显然不含三边颜色相同的三角形.
综上便知结论成立.
%%PROBLEM_END%%



%%PROBLEM_BEGIN%%
%%<PROBLEM>%%
问题8. 桌上放着 2000 张互不重叠的相同的圆纸片, 某些圆纸片互相外切.
问最少要给这些纸片染上多少种颜色,才能使互相外切的圆纸片的颜色互不相同.
%%<SOLUTION>%%
如图(<FilePath:./figures/fig-c15a8.png>), 若仅用 3 色,假设 11 张圆形纸片中有 6 张圆纸片的颜色如图标号, 则 $A, B, C$ 三圆纸片只能染 1 色或 3 色.
并且 $A$ 和 $C$ 同色,但 $A 、 C$ 与 $B$ 不同色,于是 $D$ 与 $A 、 B$ 不同色, 故 $D$ 染 2 色.
同理 $E$ 与 $B 、 C$ 不同色, 从而 $E$ 也染 2 色,这与 $D$ 和 $E$ 互相外切不同色矛盾.
其次,我们证明染上 4 色就够了.
我们对 $n$ 张圆纸片来证明, $n \leqslant 4$ 时显然成立.
设对 $k \geqslant 4$ 张圆纸片, 染上 4 色就够了.
对 $k+1$ 张圆纸片, 则这 $k+1$ 张圆纸片中心的凸包为凸多边形, 取某一个顶点 $A$, 则 $A$ 是某圆纸片的中心, 易证该纸片最多与 3 张圆形纸片相切, 去掉 $A$ 纸片, 余下 $k$ 张纸片, 由归纳假设知可染上 4 色使任何两张相切的纸片不同色, 放回 $A$ 纸片, 必可将它染上同它相切的 3 张纸片不同色的第 4 种颜色.
于是对 $k+1$, 结论也成立.
故对一切 $n \in \mathbf{N}_{+}$, 染上 4 色就够了, 特别对 $n=2000$ 张纸片, 染上 4 色就够了.
综上可知, 最少要染上 4 种颜色.
%%PROBLEM_END%%


