
%%TEXT_BEGIN%%
分类和分步.
一、分类
当被研究的数学问题出现多种不同的情形时, 常常可按出现的各种情形分别进行讨论和解答.
得出各种情形下相应的结论, 综合起来就获得原问题的解答, 这就是分类的思想方法.
应用分类的思想方法解题应遵循以下原则:
(1)分类必须包含原题目中所有可能出现的各种情形,没有遗漏;
(2)任何两类之间互相排斥,没有重叠;
(3)每次分类必须使用同一标准;
(4)选择分类标准的关键在于各类情形都比原问题易于解决.
二、分步
分步就是将一个较复杂的问题变成 (或改编成)一组互相关联的"小问题". 在这一组 "小问题" 中, 后面问题的解决常常依赖于前面小题的结果, 而当最后一个小问题解出时, 便得出了原问题的结论.
%%TEXT_END%%



%%PROBLEM_BEGIN%%
%%<PROBLEM>%%
例1. 将 9 个 1,9 个 2,9 个 $3, \cdots, 9$ 个 1000 共 9000 个数填人一个 9 行、 1000 列的表格内 (每格内填人一个数), 使得同一列中任何两数之差的绝对值不超过 3. 设这个表格中每列中各数之和 (共 1000 个列和) 的最小值为 $M$, 试求 $M$ 的最大值.
%%<SOLUTION>%%
解:我们依据 9 个 1 的分布的列数的不同情形来分别求列和的最小值 $M$.
如果 9 个 1 分布在同一列,那么 $M=9$.
如果 9 个 1 分布在两列中,那么这两列中各数之和不小于 $2 M$, 同时由已知条件知两列中出现的最大数至多只能为 4 , 故这两列数之和 $\leqslant 9 \times 1+9 \times 4=45$, 即 $2 M \leqslant 45$, 所以 $M \leqslant 22$.
如果 9 个 1 分布在三列中,那么同上讨论可得 $3 M \leqslant 9 \times 1+9 \times 4+9 \times 3=72$, 所以 $M \leqslant 24$.
如果 9 个 1 分布在四列中,那么类似可得 $4 M \leqslant 9 \times 1+9 \times 4+9 \times 3+ 9 \times 2=90$, 所以 $M \leqslant 22$.
如果 9 个 1 分布的列数大于 4 , 那么其中某一列必有一个数大于 4 (因为
$2,3,4$ 一共 27 个数, 不足以填满所有出现 1 的列), 这与已知条件中任何一列中任意两个数之差的绝对值不大于 3 矛盾, 所以这种情形不出现.
综上可知, 列和的最小值 $M \leqslant 24$.
所以,所求列和的最小值的最大值为 24 .
对于较复杂的问题, 有时一次分类还不够, 还要进行第二次分类, 两次分类可以互相独立,也可能第二次是将第一次的一个子类再进行分类.
总之, 可以根据问题的需要多层次地进行分类.
%%PROBLEM_END%%



%%PROBLEM_BEGIN%%
%%<PROBLEM>%%
例2. 空间给定 $2 n(n \geqslant 2)$ 个点, 其中任意 4 点不共面, 它们之间连有 $n^2+1$ 条线段,求证这些线段至少构成 $n$ 个三角形.
%%<SOLUTION>%%
证明:$n=2$ 时,共有 4 个点 $A, B, C, D$ 它们之间连有 5 条线段,其中只有 $\mathrm{C}_4^2-5=1$ 对点之间没有连线, 不妨设 $C$ 与 $D$ 没有连线, 于是存在两个三角形: $\triangle A B C$ 和 $\triangle A B D$.
设 $n=k$ 时,结论成立, 当 $n=k+1$ 时, 我们首先证明此时至少存在一个三角形.
设已知两点 $A$ 与 $B$ 之间连有线段, 并设由 $A, B$ 向其余 $2 k$ 个点所引出的线段数分别为 $a, b$.
(1) 若 $a+b \geqslant 2 k+1$, 则存在不同于 $A, B$ 的点 $C$ 同时与 $A$ 和 $B$ 都连有线段, 即存在一个三角形 $A B C$.
(2) 若 $a+b \leqslant 2 k$, 则去掉 $A, B$ 两点以及从 $A, B$ 出发的线段, 还剩 $2 k$ 个点, 它们之间至少连有 $(k+1)^2+1-(2 k+1)=k^2+1$ 条线段, 从而由归纳假设知存在 $k$ 个三角形.
设 $\triangle A B C$ 是所连线段构成的一个三角形, $n_A, n_B, n_C$ 分别是从 $A, B, C$ 向其他 $2 k-1$ 个点所引出的线段数,下面又分两种情形.
(a) 若 $n_A+n_B+n_C \geqslant 3 k-1$, 则恰以 $A B 、 B C 、 C A$ 之一为边的三角形至少有 $k$ 个,再加上三角形 $A B C$, 则一共至少有 $k+1$ 个三角形.
(b) 若 $n_A+n_B+n_C \leqslant 3 k-2$, 即 $\left(n_A+n_B\right)+\left(n_B+n_C\right)+\left(n_C+n_A\right) \leqslant 6 k-4$, 故 $n_A+n_B, n_B+n_C, n_C+n_A$ 中必有一个不超过 $\left[\frac{6 k-4}{3}\right]=2 k-2$. 不妨设 $n_A+n_B \leqslant 2 k-2$, 于是把 $A, B$ 以及从 $A, B$ 出发的线段(包括 $\triangle A B C$ 的 3 条边) 去掉后, 还剩 $2 k$ 个点, 它们之间至少连有 $(k+1)^2+1-(2 k-2)- 3=k^2+1$ 条线段, 于是由归纳假设知存在 $k$ 个以所连线段为边的三角形, 再加上 $\triangle A B C$,一共至少有 $k+1$ 个三角形.
故 $n=k+1$ 时,结论成立, 这就完成了归纳证明.
%%<REMARK>%%
注:本题证明中, 前后两次进行了分类, 并且两次分类是独立的 (分别采用不同的分类标准).下面的例中第二次分类则是将第一次的子类再分类.
%%PROBLEM_END%%



%%PROBLEM_BEGIN%%
%%<PROBLEM>%%
例3. 8 个人参加一次聚会.
(1)如果其中任何 5 个人中都有 3 个人两两认识,求证: 可以从中找出 4 个人两两认识;
(2)试问,如果其中任何 6 个人中都有 3 个人两两认识, 那么是否一定可以找出 4 个人两两认识?
%%<SOLUTION>%%
解:(1) 分下列两种情形.
情形 I. 如果存在 3 个人两两互不认识, 那么余下的 5 人必然两两认识, 否则他们之中必有两人互不认识, 这两人与原来 3 人一起构成的 5 人组中没有 3 人两两认识,导致矛盾,所以此时题中结论成立;
情形 II. 任何 3 人中必有两人互相认识.
(a) 如果 8 人中有 1 个人 $A$ 至多认识 3 个人,那么他至少不认识 4 个人, 于是这 4 个人两两认识, 否则他们之中必有两人互不认识, 这两人与 $A$ 一起构成的 3 人组中没有两人互相认识,导致矛盾,所以此时题中结论成立.
(b) 如果 8 个人中存在 1 人 $A$ 至少认识 5 个人,那么这 5 个人中必有 3 人两两认识, 这 3 个人与 $A$ 一起构成的 4 人组中都两两认识, 从而结论也成立.
(c)如果 8 个人中任何 1 人都恰恰认识其余 4 个人.
任取其中 1 人 $A$. 如果 $A$ 所认识的 4 人两两认识, 那么题中结论成立, 否则存在两人 $B$ 和 $C$ 都与 $A$ 认识, 但他们互不认识.
因为 $A$ 恰认识 4 人, 故 $A$ 恰有 3 个不认识的人: $F 、 G 、 H$. 这 3 人中任何 2 人都与 $A$ 构成 3 人组,故 $F 、 G 、 H$ 中任何两人互相认识.
如果 $B 、 C$ 中有 1 人与 $F 、 G 、 H$ 都认识, 那么此人与 $F 、 G 、 H$ 构成的 4 人组中两两认识, 结论成立, 否则 $B 、 C$ 分别不认识 $F 、 G 、 H$ 中一个人, 并且 $B 、 C$ 不可能不认识他们中同一个人, 否则该人与 $B 、 C$ 构成的 3 人组中无 2 人互相认识, 导致矛盾.
所以 $B$ 和 $C$ 分别不认识 $F 、G 、 H$ 中两个不同的人, 不妨设 $B$ 不认识 $F, C$ 不认识 $G$. 设将 $B 、 F 、 A 、 G 、 C$ 依次排在一个圆周上, 于是任何相邻位置上的人互相不认识.
然而他们中任何 3 人中都有两个人处在圆周上的相邻位置, 故 $B 、 F 、 A 、 G 、 C$ 中找不到 3 个人两两认识,导致矛盾,即最后一种情形不存在.
综上所述,在任何情形下,都存在 4 个人两两互相认识;
(2)如果任何 6 个人中都有 3 个人两两互相认识, 那么不保证存在 4 个人,他们都两两互相认识.
例子如下:
在正八边形内连出 8 条最短的对角线, 8 个顶点代表 8 个人,如果两点间连有边或对角线, 则对应的两人互相认识, 否则互相不认识.
因为图中共有 8 个三角形,而每一顶点恰是 3 个三角形的公共顶点, 任意去掉 2 点, 至多减少 6 个三角形, 以余下 6 点为顶点的三角形至少还有 2 个, 即任意 6 人中必有 3 人两两互相认识, 但是这 8 个人中找不出 4 人两两互相认识.
%%PROBLEM_END%%



%%PROBLEM_BEGIN%%
%%<PROBLEM>%%
例4. 已知 $n(\geqslant 6)$ 个人之间互通电话问候, 满足: (1) 每个人至多与 $n- \left[\frac{n+2}{2}\right]$ 人互通电话; (2) 任何 3 个人中至少有两人互通了电话.
证明: 这 $n$ 个人总可以分成不相交的两组, 使同组的任何两人之间都互通了电话.
%%<SOLUTION>%%
证明:设已知 $n$ 个人的集合为 $S$. 由已知条件(1) 知每人至少与 $n-1- \left(n-\left[\frac{n+2}{2}\right]\right)=\left[\frac{n}{2}\right]$ 个人没有通电话.
设 $A_1$ 与 $B_1, B_2, \cdots, B\left[\frac{n}{2}\right]$ 没有通电话, $B_1$ 与 $A_1, A_2, \cdots, A_{\left[\frac{n}{2}\right]}$ 没有通电话, 并记 $S_1=\left\{A_1, A_2, \cdots, A_{\left[\frac{n}{2}\right]}\right\}$, $S_2=\left\{B_1, B_2, \cdots, B\left[\frac{n}{2}\right]\right\}$. 由已知条件 (2) 知 $S_1 \cap S_2=\varnothing$, 并且每个 $S_i(i= 1,2)$ 内任意两人互通了电话:下面分为两种情形.
(1) 若 $n=2 k$ 为偶数,则 $n$ 个人的集合 $S$ 可分为不相交的两组 $S_1$ 和 $S_2$ 使每一组内的任何两人互通了电话,结论成立.
(2) 若 $n=2 k+1 \geqslant 6$ 为奇数, 则 $k \geqslant 3$. 且已知 $n$ 个人中除了 $S_1 \cup S_2$ 内的 $2 k$ 个人外, 还有另一个人 $C$, 又分为下列 3 种情形.
(a) 若 $C$ 与 $B_1$ 没有通电话, 则记 $S_1{ }^{\prime}=S_1 \cup\{C\}$, 于是由已知条件 (2)知 $S_1{ }^{\prime}$ 和 $S_2$ 的每一个内的任意两人互通了电话, 并且 $S=S_1{ }^{\prime} \cup S_2, S_1{ }^{\prime} \cap S_2= \varnothing$, 故知结论成立.
(b) 若 $C$ 与 $A_1$ 没有通电话, 则同 (a) 可证结论成立.
(c) 若 $C$ 与 $A_1$ 和 $B_1$ 都通了电话,则由已知条件 (1) 知与 $C$ 没有通话的至少有 $\left[\frac{n}{2}\right]=k \geqslant 3$ 人, 他们不能全属于 $S_1$, 也不能全属于 $S_2$ (因为 $\left|S_1\right| \left.\left\{A_1\right\}|=| S_2 \backslash\left\{B_1\right\} \mid=k-1<k\right)$. 不妨设 $C$ 与 $S_1$ 中 $A_i(2 \leqslant i \leqslant k)$ 没有通电话, 并且 $C$ 与 $S_2$ 中 $B_j, B_t(2 \leqslant j<t \leqslant k)$ 没有通电话.
因为与 $A_i$ 没有通话的至少有 $\left[\frac{n}{2}\right]=k$ 人, 他们中除 $C$ 外, 其余 $k-1$ 人都在 $S_2$ 中, 而 $S_2$ 中除 $B_j$, $B_t$ 外只有 $k-2$ 人, 故 $A_i$ 必与 $B_j, B_t$ 中一人没有通电话, 不妨设 $A_i$ 与 $B_j$ 没有通电话.
于是 $C, A_i, B_j$ 中任何两人都没有互通电话, 这与已知条件(2) 矛盾.
证毕.
%%PROBLEM_END%%



%%PROBLEM_BEGIN%%
%%<PROBLEM>%%
例5. 求最小正整数 $n$,使得任何 $n$ 个无理数中总有 3 个数,其中每两个数之和仍为无理数.
%%<SOLUTION>%%
解:显然 $\{-\sqrt{2},-\sqrt{3}, \sqrt{2}, \sqrt{3}\}$, 这 4 个无理数中的任何三个数含有一对相反数, 它们之和为 0 , 不是无理数, 故满足要求的最小正整数 $n \geqslant 5$.
下面我们证明任意 5 个无理数中必有 3 个数, 其中每两个数之和仍为无理数, 为了表达方便, 用平面内 5 个点 $x, y, z, u, v$ 表示 5 个无理数 (以下点与对应的数用同一字母表示), 其中任意 3 点不共线.
若两数之和为有理数, 则对应点连虚线, 否则连实线, 得到图 $G$, 只须证明图中存在实线三角形即可.
第一步证明图 $G$ 中存在实线或虚线三角形.
事实上,若从某点出发的 4 条线段中有 3 条同为实线或同为虚线,则必存在实线或虚线三角形,故不妨设从每点出发有两条实线和两条虚线.
整个图 $G$ 中恰有 5 条实线和 5 条虚线.
因为每点出发有两条实线,故每点都为一条闭实折线的顶点,但每条闭实折线至少有 3 条线段.
故 5 条实线形成一条闭折线, 同理 5 条虚线形成一条闭折线.
不妨设 $x y z u v x$ 为闭虚线, 于是 $x+y, y+z$, $z+u, u+v, v+x$ 均为有理数, 从而
$$
x=\frac{1}{2}[(x+y)-(y+z)+(z+u)-(u+v)+(v+x)]
$$
为有理数,这与已知矛盾.
故 $G$ 中必存在实线或虚线三角形.
第二步证明必存在实线三角形.
若不然, 必有虚线三角形, 不妨设 $\triangle x y z$ 为虚线三角形, 于是 $x+y, y+z, z+x$ 均为有理数.
从而
$$
x=\frac{1}{2}[(x+y)+(z+x)-(y+z)]
$$
为有理数, 与已知矛盾.
可见 $G$ 中不存在虚线三角形, 于是由第一步的结论知必有实线三角形,不妨设 $\triangle x y z$ 为实线三角形,即存在 3 个数 $x, y, z$ 使其中每两个数之和均为无理数.
证毕.
综上可得,所求最小正整数 $n=5$.
%%PROBLEM_END%%



%%PROBLEM_BEGIN%%
%%<PROBLEM>%%
例6. 设有 $2^{n-1}$ 个不同的数列,每个数列有 $n$ 项,每项都等于 0 或 1 . 已知对于这些数列中任意 3 个数列, 都存在正整数 $p$, 使得这 3 个数列的第 $p$ 项都是 1 , 证明存在唯一的正整数 $k$, 使得所有 $2^{n-1}$ 个数列的第 $k$ 项都等于 1. 
%%<SOLUTION>%%
证明:记 $S=\left\{X \mid X=\left(x_1, x_2, \cdots, x_n\right), x_i=0\right.$ 或 $\left.1, i=1,2, \cdots, n\right\}$ , 而用 $S_0$ 表示已知 $2^{n-1}$ 个不同数列组成的集合.
显然 $S_0 \varsubsetneqq S$. 我们先研究 $S_0$ 的特性,再来应用这些特性去证明题中结论.
为了表达方便, 我们引人下列记号:
对任意 $X=\left(x_1, x_2, \cdots, x_n\right) \in S$, 令 $\bar{X}=\left(\bar{x}_1, \bar{x}_2, \cdots, \bar{x}_n\right)$, 其中 $\bar{x}_i= \left.y_2, \cdots, y_n\right) \in S$, 令
$$
X \cdot Y=\left(x_1 y_1, x_2 y_2, \cdots, x_n y_n\right),
$$
于是, 当 $X, Y \in S$ 时, $\bar{X}$ 及 $X \cdot Y$ 都属于 $S$.
(1)我们首先证明: 对任意 $X \in S, X$ 与 $\bar{X}$ 中有且只有一个属于 $S_0$, 并且 $0=(0,0, \cdots, 0) \notin S_0$.
事实上, 对任意 $X \in S$, 将 $X$ 与 $\bar{X}$ 配成一对, 于是将 $S$ 中 $2^n$ 个元素配成 $2^{n-1}$ 对.
若某个 $X$ 与 $\bar{X}$ 都属于 $S_0$, 则任取一个 $Y \in S_0$, 得 $X \cdot \bar{X} \cdot Y=(0,0$, $\cdots, 0)$, 这与已知条件: 存在正整数 $p$, 使 $X, \bar{X}, Y$ 的第 $p$ 项都是 1 矛盾,故对任意 $X \in S, X$ 与 $\bar{X}$ 中至多有一个属于 $S_0$. 因不同的数列对 $(X, \bar{X})$ 共 $2^{n-1}$ 个, 每对中至多只有一个属于 $S_0$, 而 $S_0$ 中恰有 $2^{n-1}$ 个数列, 故每对数列 $X$ 与 $\bar{X}$ 中恰有一个属于 $S_0$. 其次, 若 $0=(0,0, \cdots, 0) \in S_0$, 则任取 $X, Y \in S_0$ 有 $0 \cdot X \cdot Y=(0,0, \cdots, 0)$ 这与已知条件矛盾.
故 $0=(0,0, \cdots, 0)$ 不属于 $S_0$.
(2) 其次我们证明.
若 $X, Y \in S_0$, 则 $X \cdot Y \in S_0$.
事实上, 若 $Z=X \cdot Y \notin S_0$, 则由 (1) 知 $\bar{Z}=\overline{X \cdot Y} \in S_0$. 于是 $X \cdot Y \cdot \bar{Z}=(X \cdot Y) \cdot(\overline{X \cdot Y})=(0,0, \cdots, 0)$, 这与已知条件矛盾.
故 $X \cdot Y \in S_0$.
(3)最后,我们证明: 若 $S_0$ 中 $2^{n-1}$ 个不同的数列是 $X_1, X_2, \cdots, X_{2^{n-1}}$, 令 $X=X_1 \cdot X_2 \cdots \cdots X_{2^{n-1}}$, 则 $X$ 中恰有一项等于 1 , 其余 $n-1$ 项都等于 0 .
事实上, 由 (2) 和 (1) 知 $X \in S_0$ 且 $X \neq(0,0, \cdots, 0)$. 即 $X$ 至少有一项等于 1. 若 $X$ 至少有两项等于 1 , 则 $S_0$ 中每个数列的这两项的值都等于 1 . 并且每个数列的其余 $n-2$ 项只能为 0 或 1 , 这样的不同数列一共至多有 $2^{n-2}$ 个, 这与 $S_0$ 中共有 $2^{n-1}$ 个不同数列矛盾.
现回到原题, 由结论 (3), 可设 $X=X_1 \cdot X_2 \cdots \cdots X_{2^{n-1}}$ 的第 $k$ 项等于 1 , 其余各项等于零.
即知存在唯一正整数 $k$, 使 $S_0$ 中 $2^{n-1}$ 个数列的第 $k$ 项都等于 1 ,而其他任何一项不能都等于 1 . 证毕.
%%<REMARK>%%
注:本题可等价地描述为下列问题.
设 $I=\left\{a_1, a_2, \cdots, a_n\right\}$ 是 $n$ 元集合, 它的 $2^{n-1}$ 个子集构成的集合 $S_0$ 具有下列性质: $S_0$ 中的任何 3 个元素 ( $I$ 的子集)的交非空.
证明: $S_0$ 中所有元素 ( $I$ 的 $2^{n-1}$ 个子集) 的交集恰含唯一一个元素.
本题证明中引人的运算 $\bar{X}$ 和 $X \cdot Y$ 恰恰对应于集合的补集和交集运算.
故读者不难用补集、交集运算按上述证明步骤给出等价命题的证明.
这样实质上给出了原题的另一种本质相同而形式不同的证明, 并且从集合的角度来看, 本题的证明就显得非常自然了.
%%PROBLEM_END%%


