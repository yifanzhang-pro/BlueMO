
%%PROBLEM_BEGIN%%
%%<PROBLEM>%%
问题1. 以一个正 $6 n\left(n \in \mathbf{N}_{+}\right)$边形的顶点为顶点的互不全等的三角形有多少个? 
%%<SOLUTION>%%
将正 $6 n$ 边形的顶点沿顺时针方向标号为 $1,2, \cdots, 6 n$. 对以 $i, j, k$ 为顶点的三角形, 定义 $i, j$ 之间的距离为 $i, j$ 之间 (不含 $k$ 的一侧) 正 $6 n$ 边形的边数.
显然,一个三角形顶点间的 3 个距离之和为 $6 n$, 其中最小距离不大于 $2 n$. (1) 当最小距离为偶数 $2 k(k==1,2, \cdots, n)$ 时.
中间距离不小于 $2 k$ 且不大于 $\frac{1}{2}(6 n-2 k)=3 n-k$. 这时共有 $(3 n-k)-(2 k-1)=3 n-3 k+1$ 个互不全等的三角形; (2) 当最小距离为奇数 $2 k-1(k=1,2, \cdots, n)$ 时, 中间距离不小于 $2 k-1$ 且不大于 $\left[\frac{1}{2}(6 n-(2 k-1))\right]=3 n-k$, 这时共有 $3 n-k- (2 k-2)=3 n-3 k+2$ 个互不全等的三角形.
由 (1), (2) 得互不全等的三角形个数为 $m=\sum_{k=1}^n(3 n-3 k+1+3 n-3 k+2)=\sum_{k=1}^n(6 n-6 k+3)=6 n^2-6$. $\frac{n(n+1)}{2}+3 n=3 n^2$ (个).
%%PROBLEM_END%%



%%PROBLEM_BEGIN%%
%%<PROBLEM>%%
问题2. 将正三角形每边 $n$ 等分, 过分点在正三角形内作边的平行线将正三角形剖分为小正三角形, 问 (1) 其中有多少个正三角形 (包括原来的正三角形在内)?(2)有多少个菱形?
%%<SOLUTION>%%
(1) 边长为 $k$ 且 "头朝上" 的正三角形个数为 $x_k=1+2+\cdots+[n- (k-1)]=\frac{1}{2}(n-k+1)(n-k+2)=\frac{1}{6}[(n-k+1)(n-k+2)(n-k+ 3)-(n-k)(n-k+1)(n-k+2)]$, 故 "头朝上" 的正三角形个数为 $S_1= \sum_{k=1}^n x_k=\frac{1}{6} \sum_{k=1}^n[(n-k+1)(n-k+2)(n-k+3)-(n-k)(n-k+1)(n- k+2)]=\frac{1}{6} n(n+1)(n+2)$. 边长为 $l$ 且 "头朝下" 的正三角形个数为 $y_i= 1+2+3+\cdots+(n-2 l+1)=\frac{1}{2}(n-2 l+1)(n-2 l+2)\left(1 \leqslant l \leqslant\left[\frac{n}{2}\right]\right)$.
当 $n=2 m$ 为偶数时, "头朝下"的正三角形个数为 $S_2=\sum_{l=1}^{\left[\frac{n}{2}\right]} y_l=\frac{1}{2} \sum_{l=1}^m[(2 m- \left.2 l+2)^2-(2 m-2 l+2)\right]=\sum_{l=1}^m\left[2(m-l+1)^2-(m-l+1)\right]=2 \sum_{l=1}^m l^2- \sum_{l=1}^m l=2 \times \frac{1}{6} m(m+1)(2 m+1)-\frac{1}{2} m(m+1)=\frac{1}{6} m(m+1)(4 m-1)= \frac{1}{24} n(n+2)(2 n-1)$, 当 $n=2 m-1$ 时, 类似可得 $S_2=\sum_{l=1}^{\left[\frac{n}{2}\right]} y_l=\frac{1}{2} \sum_{l=1}^{m-1}(2 m- 2 l)(2 m-2 l+1)=2 \sum_{l=1}^{m-1}(m-l)^2+\sum_{l=1}^{m-1}(m-l)=\frac{1}{3}(m-1) m(2 m-1)+ \frac{1}{2}(m-1) m=\frac{1}{6}(m-1) m(4 m+1)=\frac{1}{24}(n-1)(n+1)(2 n+3)$, 故当 $n$ 为偶数时正三角形总个数为 $S=S_1+S_2=\frac{1}{6} n(n+1)(n+2)+\frac{1}{24} n(n+2) (2 n-1)=\frac{1}{8} n(n+2)(2 n+1)$; 当 $n$ 为奇数时, 正三角形总个数为 $S= \frac{1}{6} n(n+1)(n+2)+\frac{1}{24}(n-1)(n+1)(2 n+3)=\frac{1}{8}(n+1)\left(2 n^2+3 n-1\right)$.
(2)因边不平行 $B C$ 的菱形的下半部分正好是一个 "头朝下" 的正三角形,这种对应是一一对应.
故边不平行 $B C$ 的菱形个数等于 "头朝下"的正三角形个数 $S_2$. 于是由 (1) 知当 $n$ 为偶数时, 菱形的个数为 $3 S_2=\frac{3}{8} n(n+2)(2 n- 1), n$ 为奇数时,菱形的个数为 $3 S_2=\frac{3}{8}(n-1)(n+1)(2 n+3)$.
%%PROBLEM_END%%



%%PROBLEM_BEGIN%%
%%<PROBLEM>%%
问题3. 某种比赛中每个队恰好与其他队各赛一场, 每场比赛中胜者得 2 分, 负者得 0 分, 平局各得 1 分, 比赛结束后发现, 每队所得分数中恰有一半是该队同十个得分最低的队的比赛中得到的(十个得分最低的队所得分数中一半是他们彼此比赛中得到的), 问共有几个队参加比赛?
%%<SOLUTION>%%
设有 $n$ 个队, 则 $n$ 个队共得了 $2 \mathrm{C}_n^2=n(n-1)$ 分.
而 10 个得分最低的队彼此之间对局共得 $2 \mathrm{C}_{10}^2=90$ 分.
因为这是他们得分的一半, 故这 10 个队共得 180 分; 其余 $n-10$ 个队彼此之间比赛共得 $2 \mathrm{C}_{n-10}^2=(n-10)(n-11)$ 分, 这也是他们得分的一半, 所以他们共得 $2(n-10)(n-11)$ 分.
于是 $n(n- 1)=180+2(n-10)(n-11)$, 即 $(n-16)(n-25)=0$, 解得 $n_1=25, n_2=$ 16 (舍去, 因必须 $2(n-10)(n-11) \div(n-10)>180 \div-10)$.
%%PROBLEM_END%%



%%PROBLEM_BEGIN%%
%%<PROBLEM>%%
问题4. 平面内有 18 个点, 其中任意 3 点不共线, 每两点连一线段,这些线段用红蓝两色染色, 每条线段恰染一色, 其中从某点 $A$ 出发的红色线段为奇数条, 而从其他 17 个点出发的红色线段数互不相同, 求以已知点为顶点各边为红色的三角形个数以及有两边为红色一边为蓝色的三角形个数.
 
%%<SOLUTION>%%
设所有三角形中三边为红色、两边红色一边蓝色、两边蓝色一边红色、 三边为蓝色的三角形个数分别为 $m, n, p, q$. 因除 $A$ 外, 从其余各点出发的红色线段数互不相同, 它们只可能是 $0,1,2, \cdots, 16$ 或 $1,2, \cdots, 17$. 
若为前者, 设从 $A$ 出发有 $2 k-1$ 条红线, 则图中红线总数为 $\frac{1}{2}(0+1+2+\cdots+16+ 2 k-1)=17 \times 4+k-\frac{1}{2}$ 矛盾, 故只能为后者情形.
设除 $A$ 外, 其余 17 个点为 $B_1, B_2, \cdots, B_{17}$, 从 $B_i$ 出发有 $i$ 条红线 $(i=1,2, \cdots, 17)$. 
于是 $B_{17}$ 与其余 17 点连有红线, $B_1$ 仅与 $B_{17}$ 连有红线, 进一步不难得出对 $i==1,2, \cdots, 8$, $B_{17-i}$ 仅与除 $B_1, B_2, \cdots, B_i$ 外的 $17-i$ 个点连有红线, 而 $B_i$ 仅与 $B_{17}, B_{16}$, $\cdots, B_{18-i}$ 这 $i$ 个点连有红线.
从而 $A$ 仅与 $B_{17}, B_{16}, \cdots, B_9$ 这 9 个点连有红线, 其余所连线段均为蓝线.
设从一点出发的两条红色线段叫做红色角.
从一点出发的两条蓝色线段叫做蓝色角, 于是红色角的总数为 $3 m+n=\sum_{i=2}^{17} \mathrm{C}_i^2+ \mathrm{C}_9^2=\mathrm{C}_3^3+\sum_{i=3}^{17}\left(\mathrm{C}_{i+1}^3-\mathrm{C}_i^3\right)+\mathrm{C}_9^2=\mathrm{C}_{18}^3+\mathrm{C}_9^2=852 \cdots$ (1), 
蓝色角的总数为 $p+ 3 q=\sum_{k=1}^{15} \mathrm{C}_{17-k}^2+\mathrm{C}_8^2=\mathrm{C}_{17}^3+\mathrm{C}_8^2=708 \cdots$ (2), 
图中红色线段数为 $\frac{1}{16}(3 m+2 n+ p)=\frac{1}{2}(1+2+\cdots+17+9)=81 \cdots$ (3), 
蓝色线段数为 $\frac{1}{16}(n+2 p+3 q)= \frac{1}{2}(16+15+\cdots+1+8)=72 \cdots$ (4). 
联立(1), (2), (3), (4)解得 $m=204, n= 240, p=204, q=168$. 即三边为红色的三角形有 204 个, 两边红一边蓝的三角形有 240 个.
%%PROBLEM_END%%



%%PROBLEM_BEGIN%%
%%<PROBLEM>%%
问题5. 已知有三个委员会, 对任意两个来自不同委员会的工作人员, 第三个委员会中恰有 10 人都认识他们, 也恰有 10 人都不认识他们, 试求这三个委员会中工作人员的总数.
%%<SOLUTION>%%
将每个工作人员对应平面上一个点 (任何三点不共线), 将三个委员会的人员所对应的点集分别记为 $A 、 B 、 C$. 在任何不属于同一点集的两点之间连一条线段: 若两人互相认识, 就连红线; 若不认识, 就连蓝线, 这样所得到的图称为三部图.
设三部图 $A-B-C$ 的顶点数为 $(a, b, c)$. 根据题意知 $A-B$ 之间的每条红色线段上恰有 10 个红色三角形, 每条蓝色线段上恰有 10 个蓝色三角形.
由于每个同色三角形恰有一边为 $A-B$ 之间的一条线段, 故全图中同色三角形个数为 $10 a b$. 同理, 它也等于 $10 b c, 10 c a$. 因此 $a=b=c$. 这样一来, 即知同色三角形个数为 $10 a^2$. 于是异色三角形个数为 $a^3-10 a^2$. 由于每个异色三角形恰有一个同色角, 而每个同色三角形有三个同色角.
故同色角的总数为 $3 \times 10 a^2+a^3-10 a^2=a^3+20 a^2$. 另一方面, 每条线段上张有 20 个同色角, 故同色角的总数又等于 $20 \times 3 a^2=60 a^2$. 故 $a^3+20 a^2=60 a^2$, 解得 $a=$ 40 , 因此, 总人数为 $3 a=120$.
%%PROBLEM_END%%



%%PROBLEM_BEGIN%%
%%<PROBLEM>%%
问题6. 设 $4 \times 4 \times 4$ 的大正方体由 64 个单位正方体组成, 选取其中 16 个单位正方体涂成红色.
使得大正方体中每个由 4 个单位正方体组成的 $1 \times 1 \times 4$ 的小长方体中都恰有 1 个红色正方体, 问 16 个红色正方体有多少种不同取法? 说明理由.
%%<SOLUTION>%%
如图(<FilePath:./figures/fig-c13a6.png>), 以 $4 \times 4 \times 4$ 大正方体的底面为基准面, 将大正方体划分为平行基准面的 4 层 (每层为 $4 \times 4 \times 1$ 的长方体) 依次用 $1,2,3,4$ 给各层编号.
将各红色正方体投影到基准面上, 并在基准面的投影方格内填上该红色正方体所在的层号, 这样得到一个 $4 \times 4$ 的方格表.
依题意, 每格内恰填 $\{1,2,3,4\}$ 中一个数, 并且填同样数的方格既不在同一行又不在同一列, 这样的 $4 \times 4$ 的方格表被称为一个 4 阶拉丁方, 反之每一个 4 阶拉丁方唯一决定了一种符合条件的涂色法.
因此, 题目转化为确定: 总共有多少个不同的 4 阶拉丁方? 用 $a_{i j}(i$, $j==1,2,3,4)$ 表示 $4 \times 4$ 方格表内第 $i$ 行第 $j$ 列处小方格内填人的数.
对 ( 1 , $2,3,4)$ 的任意排列 $(a, b, c, d)$, 先考虑 $a_{11}=a, a_{12}=a_{21}=b, a_{13}=a_{31}= c, a_{14}=a_{41}=d$ (如图)时拉丁方的个数.
(1) $a_{22}=a$ 时, $a_{23}$ 与 $a_{32}$ 只可能为 $d, a_{24}$ 与 $a_{42}$ 只可能为 $c$. 然后 $a_{33}$ 可以等于 $a$ 或 $b$, 当 $a_{33}$ 选定后, $a_{34}, a_{43}, a_{44}$ 的值是唯一的.
因此, 这种情形下对应的拉丁方恰有 2 个; (2) $a_{22}=c$ 时, 只有 $a_{24}= a_{42}=a, a_{23}=a_{32}=d, a_{34}=a_{43}=b, a_{44}=c, a_{33}=a$, 这种情形下对应的拉丁方只有 1 个; (3) $a_{22}=d$ 时同(2)可证这时拉丁方只有 1 个, 故如图所示的拉丁方恰有 4 个.
因 $(a, b, c, d)$ 这种排列共 $4 !=24$ 个, 当 $(a, b, c, d)$ 确定后, 如图所示拉丁方通过交换第 $2,3,4$ 行一共可形成 3 ! 个不同的拉丁方.
所以 4 阶拉丁方的总数为 4 ! $\cdot 3$ ! ・ $4=576$ 个.
故题目所述 16 个红色正方体的不同取法共有 576 种.
%%<REMARK>%%
注:设 $n$ 阶拉丁方的个数为 $L_n$, 则 $L_n=n !((n-1) !) l_n$, 现已有的结论为 $l_1=l_2=l_3=1, l_4=4, l_5=56, l_6=9408, l_7=16942080, \cdots$. 当 $n \geqslant 5$ 时, 要分很多情形才能算出 $l_n$, 不可能在短时间内做到.
因此, 作为试题选择 $n=4$ 是恰当的.
%%PROBLEM_END%%


