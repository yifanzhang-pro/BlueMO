
%%PROBLEM_BEGIN%%
%%<PROBLEM>%%
问题1. 用 19 张 $1 \times 6$ 或 $1 \times 7$ 的矩形能否将一个 $11 \times 12$ 的矩形完全覆盖? 是 8 的倍数;
%%<SOLUTION>%%
如图(<FilePath:./figures/fig-c9a1.png>), 因为 $11 \times 12$ 的矩形中有 $11 \times 12=132$ 个 $1 \times 1$ 的方格, 且 $126=7 \times 18<132<7 \times 19=133$, 所以,如果用 19 张 $1 \times$ 6 或 $1 \times 7$ 的矩形能够完全覆盖 $11 \times 12$ 矩形,则要用 18 张 $1 \times 7$ 矩形和 1 张 $1 \times 6$ 矩形.
如图对 $11 \times 12$ 的矩形用黑白两种颜色染色, 使得任意两个黑色格都无法被 1 张 $1 \times 7$ 的矩形同时覆盖.
图中共有 20 个黑格, 18 张 $1 \times 7$ 的矩形只能盖住其中 18 个黑格, 还剩下两个黑格, 那张 $1 \times 6$ 的矩形无法把他们都盖住.
这就证明了用 19 张 $1 \times 6$ 或 1 $\times 7$ 的矩形不可能完全覆盖 $11 \times 12$ 的矩形.
%%PROBLEM_END%%



%%PROBLEM_BEGIN%%
%%<PROBLEM>%%
问题2. (1)如果 $m\times n$ 棋盘能用形如图(<FilePath:./figures/fig-c9p2-1.png>)或图(<FilePath:./figures/fig-c9p2-2.png>)的 $L$ 形骨牌完全覆盖,证明$m n$是8的倍数
(2) 对怎样的正整数 $m, n, m \times n$ 棋盘能用 (1) 中规定的 $L$ 形骨牌完全覆盖.
%%<SOLUTION>%%
(1) 设 $m \times n$ 棋盘能用若干张 $\mathrm{L}$ 型骨牌完全覆盖, 则显然有 $m \geqslant 2$, $n \geqslant 2$. 又因每张 $\mathrm{L}$ 型骨牌含 4 个方格, 所以 $m n$ 是 4 的倍数.
不妨设 $m$ 为偶数,将 $m \times n$ 棋盘奇数行中每行 $n$ 个小方格都染成黑色, 偶数行中每行 $n$ 个小方格染成白色.
于是每张 L 型骨牌或者盖住 3 个白格和一个黑格或者盖住 3 个黑格和一个白格.
总之盖住的黑格是奇数个, 白格也是奇数个, 如果 $m n$ 不是 8 的倍数, 那么 $m=4(2 k+1)$, 若 $m \times n$ 棋盘可用 $\mathrm{L}$ 型骨牌完全覆盖, 则必须用 $2 k+1$ 张 $\mathrm{L}$ 型骨牌, 奇数个奇数之和仍为奇数, 这表明 $m \times n$ 棋盘有奇数个黑格和奇数个白格.
但棋盘上黑格与白格各占一半都为 $2(2 k+1)$ 格, 矛盾.
因此, $m n$ 是 8 的倍数.
(2)由(1)知 $m \times n$ 矩形可用 $\mathrm{L}$ 型骨牌完全覆盖的必要条件是 $m \geqslant 2$, $n \geqslant 2$, 且 $m n$ 是 8 的倍数, 下面我们证明这一条件也是充分的.
分两种情形: 情形 $1: m$ 与 $n$ 均为偶数.
因 $m n$ 是 8 的倍数, 所以 $m, n$ 中必有一个是 4 的倍数.
不妨设 $m=2 m_1, n=4 n_1$. 于是 $m \times n$ 棋盘可分成 $m_1 n_1$ 个 $2 \times 4$ 的棋盘, 而每个 $2 \times 4$ 的棋盘可用 2 张 L 型骨牌覆盖 (如图(<FilePath:./figures/fig-c9a2-1.png>)), 故 $m \times n$ 棋盘可用 $2 m_1 n_1$ 张 $\mathrm{L}$ 型骨牌完全覆盖.
情形2: $m, n$ 中有一个为奇数, 另一个为 8 的倍数.
不妨设 $m=2 m_1+1, n=8 n_1$, 于是 $m \times n$ 棋盘可分成两部分, 一个是 $2\left(m_1-1\right) \times 8 n_1$ 棋盘, 另一个是 $3 \times 8 n_1$ 棋盘, 其中第一部分可分成 $\left(m_1-1\right) \times 2 n_1$ 个 $2 \times 4$ 的棋盘, 从而可用 $4\left(m_1-1\right) n_1$ 块 L 型骨牌覆盖.
第二部可分成 $n_1$ 个 $3 \times 8$ 的棋盘, 而 $3 \times 8$ 的棋盘可用 6 张 L 型骨牌覆盖(如图(<FilePath:./figures/fig-c9a2-2.png>)), 从而第 2 部分可用 $6 n_1$ 张 $\mathrm{L}$ 型骨牌覆盖.
因此总个 $m \times n$ 棋盘可用 $\mathrm{L}$ 型骨牌完全覆盖.
%%PROBLEM_END%%



%%PROBLEM_BEGIN%%
%%<PROBLEM>%%
问题3. 求最小正整数 $n$, 使得任意 $n$ 个人中都存在 5 个人可分成两个恰有一个公共成员的 3 人组, 并且每个 3 人组内的 3 个人都互相认识或者都互相不认识.
%%<SOLUTION>%%
如图(<FilePath:./figures/fig-c9a3.png>), 用 $n$ 点表示 $n$ 个人, 如果两人互相认识, 那么对应点连一条红线, 否则连一条蓝线, 问题化为: 求最小正整数 $n$, 使得 2 色完全图 $K_n$ 中存在两个恰有一个公共顶点的同色三角形.
如图 2 色完全图 $K_8$ 中 (图中画的实线表红线, 蓝线没有画出) 没有蓝色三角形, 共有 8 个红色三角形, 且任何两个红色三角形或者有公共边或者无公共顶点.
故所求最小正整数 $n \geqslant 9$. 设 2 色 $K_9$ 的 9 个顶点为 $A_1, A_2, \cdots, A_9$, 从 $A_i$ 出发有 $x_i$ 条红边和 $8-x_i$ 条蓝边 $(i=1,2, \cdots, 9)$. 我们称从一点出发的一条红边和一条蓝边组成的角为异色角, 于是以 $A_i$ 为顶点的异色角有 $x_i\left(8-x_i\right)$ 个.
我们称既有红边又有蓝边的三角形为异色二角形, 因每个异色三角形中有 2 个异色角,故 2 色 $K_9$ 中异色三角形的个数为 $\frac{1}{2} \sum_{i=1}^9 x_i\left(8-x_i\right) \leqslant \frac{1}{2} \sum_{i=1}^9\left(\frac{x_i+8-x_i}{2}\right)^2=\frac{1}{2} \times 9 \times16=72$, 从而 2 色 $K_9$ 中同色三角形至少有 $\mathrm{C}_9^3-72=84-72=12$ 个.
12 个三角形共 36 个顶点,但一共只有 9 点, 故由抽庶原理知其中必有 $\left[\frac{36-1}{9}\right]+1=4$ 个三角形有公共顶点, 若这 4 个三角形中有 2 个恰有 1 个公共顶点, 则结论成立, 否则这 4 个三角形有 1 条公共边, 从而这 4 个同色三角形的边都同色, 不妨设 $\triangle A_1 A_2 A_3, \triangle A_1 A_2 A_4, \triangle A_1 A_2 A_5, \triangle A_1 A_2 A_6$ 都是红色三角形.
考察以 $A_3, A_4$, $A_5, A_6$ 为顶点的 2 色 $K_4$, 若其中有 1 条红边, 不妨设为 $A_3 A_4$, 则 $\triangle A_1 A_3 A_4$ 和 $\triangle A_1 A_2 A_5$ 是两个恰有 1 个公共顶点 $A_1$ 的红色三角形.
结论成立; 若其中全是蓝色边, 则红色 $\triangle A_1 A_2 A_3$ 与蓝色 $\triangle A_3 A_4 A_5$ 是两个恰有一个公共顶点的同色三角形.
这就证明了 2 色 $K_9$ 中必存在两个恰有一个公共顶点的同色三角形.
综上知所求最小正整数 $n=9$.
%%PROBLEM_END%%



%%PROBLEM_BEGIN%%
%%<PROBLEM>%%
问题4. 致求最小正整数 $n$, 使得任意 $n$ 个人中都存在 4 个人可分成两个恰有 2 个公共成员的 3 人组, 并且每组内的 3 个人都互相认识或互相都不认识.
%%<SOLUTION>%%
如图(<FilePath:./figures/fig-c9a4.png>), 本题等价于求最小正整数 $n$,使得 2 色 $K_n$ 中必存在两个恰有一条公共边的同色三角形.
如图 $K_9$ 中 (为了清楚起见, 我们把红线 (实线)、蓝线 (虚线) 分别画在两个图中), 有 6 个红色三角形和 6 个蓝色三角形.
图中 18 条红边和 18 条蓝边恰好每边都是一个同色三角形的一条边, 任何两个同色三角形没有公共边.
可见, 所求最小正整数 $n \geqslant 10$, 同第 3 题方法可证 2 色 $K_{10}$ 中至少有 20 个同色三角形.
每个三角形 3 条边,共 60 条边, 而 $K_{10}$ 中一共只有 $\mathrm{C}_{10}^3=45$ 条边, 故至少有 $\left[\frac{60-1}{45}\right]+1=2$ 个同色三角形恰有 1 条公共边.
综上可知, 所求最小正整数 $n=10$.
%%PROBLEM_END%%



%%PROBLEM_BEGIN%%
%%<PROBLEM>%%
问题5. 9 名科学家在国际会议上相遇, 他们中任何 3 人中至少有 2 人能讲同一种语言, 而且每人最多能讲 3 种语言, 证明必存在 3 位科学家,他们能讲同一种语言.
%%<SOLUTION>%%
用 9 个点 $A_1, A_2, \cdots, A_9$ 表示 9 名科学家,若两人能同时讲第 $i$ 种语言.
则对应点的连线染第 $i$ 种颜色 $(i=1,2, \cdots)$, 否则对应点的连线不染色.
(1)若任意两点的连线都染了某种颜色.
则因为从每点出发的线段至多只有 3 种颜色, 故由抽庶原理知从一点 $A_1$ 出发至少有 $\left[\frac{8-1}{3}\right]+ 1=3$ 条线段同色,不妨设 $A_1 A_2, A_1 A_3, A_1 A_4$ 同色.
于是 $A_1, A_2, A_3$ 能讲同一种语言.
(2) 存在两点 $A_1$ 和 $A_2$, 它们之间的连线没有染色.
由已知条件知任何 3 点中必有 2 点间的连线染了某种颜色.
因此对其余 7 点中任一点 $A_i(i=3,4$, $5,6,7,8,9) A_i A_1$ 与 $A_i A_2$ 这两条线中至少有一条染了颜色.
由抽庶原理, 知 $A_1$ 和 $A_2$ 中必有一点, 例如 $A_1$, 从它出发的线段中至少有 $\left[\frac{7-1}{2}\right]+1=4$ 条染了色,但这 4 条线段至多只能染 3 种颜色, 故其中必有 $\left[\frac{4-1}{3}\right]+1=2$ 条同色.
不妨设 $A_1 A_3, A_1 A_4$ 同色, 于是 $A_1, A_3, A_4$ 能讲同一种语言.
%%PROBLEM_END%%



%%PROBLEM_BEGIN%%
%%<PROBLEM>%%
问题6. 设有两个完全相同的齿轮 $A$ 和 $B, B$ 被放在一个水平面上, $A$ 放在 $B$ 的上面,并使二者重合 (从而两轮在水平面上的投影完全重合)然后任意敲掉 4 对重合的齿.
如果两轮原各有 14 个齿, 问能否将 $A$ 轮绕两轮的公共轴旋转到一个适当的位置使两轮在水平面上的投影是一个完整的齿轮的投影?
如果两轮原各有 13 个齿, 又是怎样呢? 证明你的结论.
%%<SOLUTION>%%
两轮各有 14 个齿时, 设 $A(B)$ 轮按顺时针方向每个齿对应一个数 $a_i\left(b_i\right), i=1,2, \cdots, 14$. 对于敲掉的齿, 令 $a_i\left(b_i\right)=1$, 对于没有敲掉的齿, 令 $a_i\left(b_i\right)=0$. 旋转 $A$ 轮, 使 $A$ 轮的第 1 齿转到与 $B$ 轮的第 $i$ 齿重合, 作和 $S_i=a_1 b_i+a_2 b_{i+1}+\cdots+a_{14} b_{i+13}\left(i=1,2, \cdots, 14, b_{j+14}=b_j\right)$. 于是 $S_i=0$ 表明两轮的投影是一个完整的齿轮的投影.
若对任意 $i(1 \leqslant i \leqslant 14)$ 有 $S_i \neq 0$, 则 $S_1=4, S_i \geqslant 1(2 \leqslant i \leqslant 14)$, 于是 $\sum_{i=1}^{14} S_i \geqslant 17$. 而 $\sum_{i=1}^{14} S_i= \left(\sum_{i=1}^{14} a_i\right)\left(\sum_{i=1}^{14} b_i\right)=4 \times 4=16$, 矛盾.
故必存在 $i_0\left(2 \leqslant i_0 \leqslant 14\right)$ 使 $S_{i_0}=0$, 即 $A$ 轮的第 1 齿转到与 $B$ 轮的第 $i_0$ 齿重合时,两轮的投影是一个完整的齿轮的投影.
两轮原各有 13 齿时,结论不再成立.
事实上,依次编号后, 假设敲掉的是第 $1,2,5,7$ 对重合的齿.
并设 $A$ 轮第 $i$ 齿转到 $B$ 轮第 $j$ 齿需要转动的齿数为 $C_{i j}(i, j=1,2,5,7)$, 则可得到下表.
从表中可以看出, 无论 $A$ 轮转动多少个齿 (从 1 个到 13 个), 都会出现 $A$ 轮被敲掉的齿与 $B$ 轮某个被敲掉的齿重合的情况.
从而两轮在水平面上的投影不可能是一个完整的齿轮的投影.
\begin{tabular}{|c|c|c|c|c|}
\hline$C_{i j}$ & 1 & 2 & 5 & 7 \\
\hline 1 & 13 & 1 & 4 & 6 \\
\hline 2 & 12 & 13 & 3 & 5 \\
\hline 5 & 9 & 10 & 13 & 2 \\
\hline 7 & 7 & 8 & 11 & 13 \\
\hline
\end{tabular}
%%PROBLEM_END%%


