
%%TEXT_BEGIN%%
函数方程的解法 
7.1 代换法代换法是解函数方程的常用手段, 其基本思想是: 将函数方程中的自变量适当地以别的自变量代换 (当然在代换时应特别注意函数的定义域不能发生变化), 得到一个新的函数方程, 然后设法求得未知函数.
代换法在单变量函数方程中尤为多用.
%%TEXT_END%%



%%TEXT_BEGIN%%
7.2 赋值法所谓赋值法, 就是对自变量赋予某些特殊的数值, 从而挖掘出题中隐含的条件,并且通过这些新条件简化函数方程, 逼近最终目标.
先来看两道整数集上函数方程的例子.
%%TEXT_END%%



%%TEXT_BEGIN%%
7.3 柯西法用柯西法解函数方程的步骤是: 先求出对于自变量取所有正整数值时函数方程的解具有的形式, 然后依次证明对自变量取整数值、有理数值以及实数值时函数方程的解仍具有这种形式, 从而得到函数方程的解.
%%TEXT_END%%



%%TEXT_BEGIN%%
7.4 递归法函数方程的递归解法, 是一种借助于数列对函数方程加以研究的方法.
设 $f(n)$ 是定义在正整数集 $\mathbf{N}_{+}$上的函数,如果存在一个递推关系 $S$ 和初始条件 $f(1)=a_1$, 当知道 $f(1), f(2), \cdots, f(n)$ 的值后, 由 $S$ 可以唯一地确定 $f(n+1)$ 的值, 我们就称 $f(n)$ 为递归函数, 递归法主要解决递归函数.
%%TEXT_END%%



%%PROBLEM_BEGIN%%
%%<PROBLEM>%%
例1 求解函数方程:
$$
f(\sin x-1)=\cos ^2 x+2(-\infty<x<+\infty) .
$$
%%<SOLUTION>%%
解:令 $y=\sin x-1$, 则
$$
\sin ^2 x=(y+1)^2 .
$$
所以 $\cos ^2 x=1-\sin ^2 x=1-(y+1)^2$.
从而
$$
f(y)=1-(y+1)^2+2=-y^2-2 y+2 .
$$
所以 $f(x)=-x^2-2 x+2(-2 \leqslant x \leqslant 0)$.
经检验, $f(x)=-x^2-2 x+2(-2 \leqslant x \leqslant 0)$ 是函数方程的解.
说明这里的 “检验” 是解函数方程的一个组成部分, 因为求 $f(x)$ 时, 首先假定了函数方程存在解 $f(x)$, 这样求出的 $f(x)$ 只是满足必要性, 也就是说, 只有函数方程有解, 求出的 $f(x)$ 才是函数方程的解.
如果函数方程无解.
那么求出的 $f(x)$ 并不是原函数方程的解.
因此, 我们必须检验充分性, 有时候所得解显然满足原函数方程, 我们就省略了.
%%PROBLEM_END%%



%%PROBLEM_BEGIN%%
%%<PROBLEM>%%
例2 求解函数方程:
$$
f(x)+f\left(\frac{x-1}{x}\right)=1+x(x \neq 0,1) . \quad\quad (1)
$$
%%<SOLUTION>%%
解:.
我们希望通过适当的代换以消去 $f\left(\frac{x-1}{x}\right)$. 为此, 令 $y=\frac{x-1}{x}$, 则 $\frac{y-1}{y}=\frac{1}{1-x}$. 代入 (1), 得
$$
f\left(\frac{x-1}{x}\right)+f\left(\frac{1}{1-x}\right)=1+\frac{x-1}{x} . \quad\quad (2)
$$
从(1)和(2)中虽可消去 $f\left(\frac{x-1}{x}\right)$, 但又多出了 $f\left(\frac{1}{1-x}\right)$.
再令 $y=\frac{1}{1-x}$, 则 $\frac{y-1}{y}=x$, 代入(1), 得
$$
f\left(\frac{1}{1-x}\right)+f(x)=1+\frac{1}{1-x}. \quad\quad (3)
$$
(1) +(3)-(2)得
$$
f(x)=\frac{1+x^2-x^3}{2 x(1-x)} .
$$
%%PROBLEM_END%%



%%PROBLEM_BEGIN%%
%%<PROBLEM>%%
例3 证明: 恰有一个定义在所有非零实数上的函数 $f$, 满足条件:
(1) 对所有非零实数 $x, f(x)=x f\left(\frac{1}{x}\right)$;
(2) 对所有 $x \neq-y$ 的非零实数对 $(x, y)$, 有
$$
f(x)+f(y)=1+f(x+y) .
$$
%%<SOLUTION>%%
证 $f(x)=x+1$ 显然适合 (1)、(2).下证唯一性.
即设 $f(x)$ 满足 (1)、 (2), 那么 $f(x)=x+1$.
在(2)中, 令 $y=1$, 得
$$
f(x)+f(1)=1+f(x+1)(x \neq-1, x \neq 0) . \quad\quad <1>
$$
在 (2) 中, 以- $x$ 代换 $x, x+1$ 代换 $y$, 得
$$
f(-x)+f(x+1)=1+f(1)(x \neq-1, x \neq 0) . \quad\quad <2>
$$
综合<1>、<2>, 得
$$
f(x)+f(-x)=2(x \neq-1, x \neq 0) .  \quad\quad <3>
$$
<3>在 $x=1$ 时成立,所以在 $x=-1$ 时也成立.
由(1)及<3>, 当 $x \neq 0$ 时,
$$
\begin{aligned}
f(x) & =x f\left(\frac{1}{x}\right)=x\left[2-f\left(-\frac{1}{x}\right)\right] \\
& =2 x+\left[-x f\left(-\frac{1}{x}\right)\right] \\
& =2 x+f(-x) .
\end{aligned}
$$
所以
$$
f(x)-f(-x)=2 x .  \quad\quad <4>
$$
从<3>、<4>中消去 $f(-x)$, 得 $f(x)=x+1$.
%%PROBLEM_END%%



%%PROBLEM_BEGIN%%
%%<PROBLEM>%%
例4 求所有的函数 $f: \mathbf{R} \rightarrow \mathbf{R}$, 使得对任意实数 $x 、 y$, 都有
$$
(x-y) f(x+y)-(x+y) f(x-y)=4 x y\left(x^2-y^2\right) . \quad\quad (1)
$$
%%<SOLUTION>%%
解:令 $x=y \neq 0$, 得
$$
f(0)=0 .
$$
设 $u=x+y, v=x-y$. 那么 $u+v=2 x, u-v=2 y$, 于是 (1) 式成为
$$
v f(u)-u f(v)=u v\left(u^2-v^2\right) .
$$
若 $u v \neq 0$, 则上式为
$$
\frac{f(u)}{u}-\frac{f(v)}{v}=u^2-v^2,
$$
即对任意非零实数 $u 、 v$, 有
$$
\frac{f(u)}{u}-u^2=\frac{f(v)}{v}-v^2 .
$$
所以 $\frac{f(x)}{x}-x^2=c$ 为一常数, $x \neq 0$.
于是对 $x \in \mathbf{R}$, 所求的函数为
$$
f(x)=x^3+c x,
$$
其中 $c$ 为某个常数.
经检验, $f(x)=x^3+c x$ ( $c$ 是常数) 是欲求的函数.
%%PROBLEM_END%%



%%PROBLEM_BEGIN%%
%%<PROBLEM>%%
例5 已知函数 $f: \mathbf{R}^{+} \bigcup\{0\} \rightarrow \mathbf{R}^{+} \bigcup\{0\}$ 满足:
(1) $f(x f(y)) f(y)=f(x+y)$;
(2) $f(2)=0$;
(3) 当 $0 \leqslant x<2$ 时, $f(x) \neq 0$.
求函数 $f(x)$.
%%<SOLUTION>%%
解:.
当 $x>2$ 时, 令 $x=2+t(t>0)$, 有
$$
f(t f(2)) f(2)=f(t+2)=f(x) .
$$
因为 $f(2)=0$, 所以 $f(x)=0(x \geqslant 2)$.
当 $0 \leqslant x<2$ 时, 令 $x+t=2(t>0)$, 有
$$
0=f(2)=f(x+t)=f(t f(x)) f(x) .
$$
又因为 $f(t f(x))=0$, 故 $t f(x) \geqslant 2$, 即
$$
f(x) \geqslant \frac{2}{t}=\frac{2}{-x+2} .
$$
但 $f(x)>\frac{2}{2-x}$ 在 $x \in[0,2]$ 时不成立.
若有 $x_1 \in[0,2]$,且
$$
\begin{gathered}
f\left(x_1\right)>\frac{2}{2-x_1}, \\
f\left(x_1\right)\left(2-x_1\right)>2 .
\end{gathered}
$$
则可得这时总可找到 $y<2-x_1$, 使 $f\left(x_1\right) \cdot y \geqslant 2$, 做
$$
f\left(y f\left(x_1\right)\right)=0,
$$
也即
$$
f\left(x_1+y\right)=f\left(y f\left(x_1\right)\right) f\left(x_1\right)=0 .
$$
此式与 $x_1+y \geqslant 2$ 矛盾, 即 $f(x)=\frac{2}{2-x}$.
从而 $f(x)=\left\{\begin{array}{l}0, \text { 当 } x \geqslant 2 \text { 时; } \\ \frac{2}{2-x}, \text { 当 } 0 \leqslant x<2 \text { 时.
}\end{array}\right.$
%%PROBLEM_END%%



%%PROBLEM_BEGIN%%
%%<PROBLEM>%%
例6 求所有的函数 $f: \mathbf{R} \rightarrow \mathbf{R}$ 使得
$$
f(f(x)+y)=f\left(x^2-y\right)+4 f(x) y . \quad\quad (1)
$$
对所有 $x, y \in \mathbf{R}$ 成立.
%%<SOLUTION>%%
解:易见 $f(x) \equiv 0$ 或 $f(x)=x^2$ 皆为上述方程 (1) 的解.
我们来证明它们是唯一的解.
设对某个 $a, f(a) \neq a^2$.
在(1)中令 $y=\frac{x^2-f(x)}{2}$, 得
$$
f(x) \cdot\left(x^2-f(x)\right)=0 .\quad\quad (2)
$$
由于 $f(a) \neq a^2$, 故只能 $f(a)=0$, 并且可见 $a \neq 0$ (否则 $a^2=0=f(a)$ 与 $a$ 的定义相违).
于是我们得到, 对任何 $x$, 要么 $f(x)=0$, 要么 $f(x)=x^2$.
在(2)中令 $x=0$, 有 $f(0)=0$.
在(1)中令 $x=0$, 有
$$
f(y)=f(-y) .
$$
在(1)中令 $x=a$, 并用 $-y$ 替换 $y$, 得
$$
f\left(a^2+y\right)=f(-y)=f(y) .
$$
从上式可见 $f$ 以 $a^2$ 为周期, 进而我们有
$$
f(f(x))=f\left(f(x)+a^2\right)=f\left(x^2-a^2\right)+4 f(x) a^2 .
$$
在(1)中令 $y=0$, 有
$$
f(f(x))=f\left(x^2\right) .
$$
利用 $f(x)$ 的周期性, 得 $f(x) \cdot a^2=0$.
所以 $f(x)=0$ (因为 $a \neq 0$ ).
也就是说, 若 $f(x) \neq x^2$, 则必有 $f(x) \equiv 0$ 成立.
因此结论成立.
%%PROBLEM_END%%



%%PROBLEM_BEGIN%%
%%<PROBLEM>%%
例7 求满足下列条件的多项式 $f(x, y)$ :
(1) $f(x, y)$ 对于 $x 、 y$ 是齐次的, 即对任意实数 $t 、 x 、 y$, 若 $f$ 是 $n$ 次的, 则有
$$
f(t x, t y)==t^n f(x, y) ;
$$
(2) 对任意 $a 、 b 、 c \in \mathbf{R}$, 有
$$
f(a+b, c)+f(b+c, a)+f(c+a, b)=0 ;
$$
(3) $f(1,0)=1$.
%%<SOLUTION>%%
解:在(2) 中令 $a=b=c=x$, 得
$$
f(2 x, x)=0 .
$$
根据多项式的因式定理,有
$$
f(x, y)=(x-2 y) g(x, y) . \quad\quad <1>
$$
这里 $g(x, y)$ 是 $n-1$ 次齐次多项式.
再在 (2) 中令 $a=b=x, c=2 y$, 得
$$
f(2 x, 2 y)=-2 f(x+2 y, x) .
$$
由 $f$ 的齐次性,得
$$
2^n f(x, y)=-2 f(x+2 y, x) .
$$
将<1>式代入上式, 得
$$
2^{n-1} g(x, y)=g(x+2 y, x) . \quad\quad <2>
$$
由此利用条件(3), 在<1>中令 $x=1, y=0$, 得
$$
g(1,0)=1 \text {. }
$$
由<2>式递推,得
$$
\begin{gathered}
g(1,1)=2^{n-1}, g(3,1)=4^{n-1}, \\
g(5,3)=8^{n-1}, g(11,5)=16^{n-1}, \cdots,
\end{gathered}
$$
即存在无限多对 $(x, y)$, 使 $g(x, y)=(x+y)^{n-1}$ 成立, 由多项式恒等定理, 知
$$
g(x, y)=(x+y)^{n-1} .
$$
所以 $f(x, y)=(x-2 y)(x+y)^{n-1}$.
经检验, 可知上述的函数满足题中条件.
%%PROBLEM_END%%



%%PROBLEM_BEGIN%%
%%<PROBLEM>%%
例8 求所有满足下列条件的函数 $f: \mathbf{N}_{+} \rightarrow \mathbf{N}_{+}$, 使得
(1) $f(2)=2$;
(2) $f(m m)=f(m) \cdot f(n)$ 对所有 $m, n \in \mathbf{N}_{+}$成立;
(3) 若 $m<n$, 则 $f(m)<f(n)$.
%%<SOLUTION>%%
解:由(2), 取 $m=n=1$ 代入, 得
$$
f(1)=f(1 \cdot 1)=f(1)^2,
$$
所以 $f(1)=1$.
再取 $m=n=2$ 代入(2) 中, 又
$$
f(4)=f(2)^2=4 .
$$
现在利用 (3), 由于 $f(2)<f(3)<f(4)$, 可知 $2<f(3)<4$, 而 $f(3) \in\mathbf{N}_{+}$, 故 $f(3)=3$.
因此, $f(6)=f(2 \cdot 3)=f(2) \cdot f(3)=6$.
同样, 由 $f(4)<f(5)<f(6)$, 我们可确定 $f(5)=5$.
上述过程已经告诉了我们该如何解决问题.
我们可以通过数学归纳法来确定 $f$.
假设已经证得 $f(1)=1, f(2)=2, \cdots, f(2 k)=2 k$.
由(2), 我们有
$f(2 k+2)=f(2) \cdot f(k+1)=2(k+1)=2 k+2(k+1 \leqslant 2 k)$.
而由(3), 得
$$
2 k=f(2 k)<f(2 k+1)<f(2 k+2)=2 k+2 .
$$
从而 $f(2 k+1)=2 k+1$.
因此, 由数学归纳法即知 $f(n)=n$ 对一切 $n \in \mathbf{N}_{+}$成立.
%%PROBLEM_END%%



%%PROBLEM_BEGIN%%
%%<PROBLEM>%%
例9 求所有满足下列条件的函数 $f: \mathbf{N}_{+} \rightarrow \mathbf{N}_{+}$: 满足
(1) $f(2)=2$;
(2) $f(m n)=f(m) \cdot f(n)$ 对所有 $m, n \in \mathbf{N}_{+}$, 且 $\operatorname{gcd}(m, n)=1$ 成立 $(\operatorname{gcd}(x, y)$ 表示 $x, y$ 的最大公约数);
(3) 若 $m<n$, 则 $f(m)<f(n)$.
%%<SOLUTION>%%
解:和上题相比, 本题的条件放宽了.
同上题, 我们可以得到 $f(1)=1$. 但是, 已经不能由 (2) 得出 $f(4)=f(2)^2$ 了, 因为 (2) 的条件是 $m 、 n$ 必须互素.
如果我们能证明 $f(3)=3$, 那将是一个很重要的突破口.
比如接着我们能得到 $f(6)=f(2 \cdot 3)=f(2) \cdot f(3)=6$, 再利用 (3), 有
$$
3=f(3)<f(4)<f(5)<f(6)=6,
$$
于是立即可得 $f(4)=4, f(5)=5$.
由于对任何正整数 $k \geqslant 2, \operatorname{gcd}(k-1, k)=1$. 我们可以通过数学归纳法给出 $f(n)=n$ 的证明:
假设已证得对所有 $k \leqslant n, f(k)=k$.
利用 $(2)$, 得 $f((n-1) n)=f(n-1) f(n)=(n-1) n$.
由(3), 又
$$
\begin{aligned}
n & =f(n)<f(n+1)<f(n+2)<\cdots<f\left(n^2-n-1\right)<f\left(n^2-n\right) \\
& =f((n-1) n)=(n-1) n .
\end{aligned}
$$
从上式不难发现 $f(n+1)=n+1, f(n+2)=n+2, \cdots, f((n-1) n- 1) =(n-1) n-1, f((n-1) n)=(n-1) n$.
这样就完成了归纳步骤.
下面我们就来设法证明 $f(3)=3$ 这一关键之处.
我们有
$$
f(3) \cdot f(5)=f(15)<f(18)=f(2 \cdot 9)=f(2) \cdot f(9)=2 f(9), \quad\quad <1>
$$
以及 $f(9)<f(10)=f(2 \cdot 5)=f(2) \cdot f(5)=2 f(5)$.\quad\quad <2>
由<1>和<2>两个不等式,有
$$
f(3) f(5)<2 f(9)<4 f(5),
$$
故 $f(3)<4$, 而 $2=f(2)<f(3)<4$, 即得 $f(3)=3$.
因此, $f(n)=n$ 对一切 $n \in \mathbf{N}_{+}$成立.
说明通过赋特殊值, 我们得到了一些特定点上的函数值, 而这些值往往 “透露”了函数的信息, 帮助我们猜出函数的解析式, 进而设法 (如用数学归纳法)加以证实.
%%PROBLEM_END%%



%%PROBLEM_BEGIN%%
%%<PROBLEM>%%
例10 解函数方程: 对任意 $x, y \in \mathbf{R}$, 都有
$$
f(x+y)+f(x-y)=2 f(x) \cdot \cos y .
$$
%%<SOLUTION>%%
解:令 $x=0, y=t$, 得
$$
f(t)+f(-t)=2 f(0) \cos t . \quad\quad (1)
$$
令 $x=\frac{\pi}{2}+t, y=\frac{\pi}{2}$, 得
$$
f(\pi+t)+f(t)=0 .\quad\quad (2)
$$
令 $x=\frac{\pi}{2}, y=\frac{\pi}{2}+t$, 得
$$
f(\pi+t)+f(-t)=-2 f\left(\frac{\pi}{2}\right) \sin t .\quad\quad (3)
$$
由 $((1) +(2)-(3)) / 2$, 得
$$
f(t)=f(0) \cos t+f\left(\frac{\pi}{2}\right) \sin t .
$$
所以
$$
f(x)=a \cos x+b \sin x,
$$
其中 $a=f(0), b=f\left(\frac{\pi}{2}\right)$ 为常数.
经检验, $f(x)=a \cos x+b \sin x$ 满足题设条件.
%%PROBLEM_END%%



%%PROBLEM_BEGIN%%
%%<PROBLEM>%%
例11 求所有满足下列条件的 $f: \mathbf{N}_{+} \rightarrow \mathbf{R}$ :
$$
f(n+m)+f(n-m)=f(3 n), n, m \in \mathbf{N}_{+}, n \geqslant m .
$$
%%<SOLUTION>%%
解:令 $m=0$, 得
$$
2 f(n)=f(3 n), n \in \mathbf{N}_{+} .
$$
令 $m=n=0$, 得 $f(0)=0$.
令 $m=n$, 得
$$
f(2 n)+f(0)=f(3 n),
$$
即 $f(2 n)=f(3 n)$.
于是, 对任意 $m \in \mathbf{N}_{+}$, 有
$$
f(4 m)=f(6 m)=f(9 m) . \quad\quad (1)
$$
另一方面, 在原恒等式中令 $n=3 m$, 得
$$
f(4 m)+f(2 m)=f(9 m) .
$$
因此, 对任意 $m \in \mathbf{N}_{+}$, 都有 $f(2 m)=0$. 于是, 对任意 $n \in \mathbf{N}_{+}$, 都有
$$
f(n)=\frac{1}{2} f(3 n)=\frac{1}{2} f(2 n)=0 .
$$
故所求的 $f(n) \equiv 0$ 才能满足题意.
%%PROBLEM_END%%



%%PROBLEM_BEGIN%%
%%<PROBLEM>%%
例12 函数 $f, g: \mathbf{R} \rightarrow \mathbf{R}$ 均非常数, 且满足:
$$
\left\{\begin{array}{l}
f(x+y)=f(x) g(y)+g(x) f(y), \quad\quad (1)\\
g(x+y)=g(x) g(y)-f(x) f(y) .\quad\quad (2)
\end{array}\right.
$$
求 $f(0)$ 与 $g(0)$ 的所有可能值.
%%<SOLUTION>%%
解:自然地,令 $x=y=0$, 代入(1)、(2), 得
$$
\left\{\begin{array}{l}
f(0)=2 f(0) g(0), \quad\quad (3)\\
g(0)=g^2(0)-f^2(0) . \quad\quad (4)
\end{array}\right.
$$
若 $f(0) \neq 0$, 则由 (3), $g(0)=\frac{1}{2}$. 由 (4), 有
$$
f^2(0)=g^2(0)-g(0)=-\frac{1}{4}<0 \text {,矛盾! }
$$
所以 $f(0)=0$, 因此
$$
g(0)=g^2(0) .
$$
若 $g(0)=0$, 在 (1) 中令 $y=0$, 得
$$
f(x) \equiv 0 .
$$
与题设不符.
所以 $g(0)=1$.
综上所述, $f(0)=0, g(0)=1$.
说明我们经常会遇到函数在某个点上取值可能不确定的情况, 这就需要我们去伪存真,并意识到题目可能会有多解.
%%PROBLEM_END%%



%%PROBLEM_BEGIN%%
%%<PROBLEM>%%
例13 求所有满足 $f(1)=2$ 和 $f(x y)=f(x) f(y)-f(x+y)+1, x$, $y \in \mathbf{Q}$ 的函数 $f: \mathbf{Q} \rightarrow \mathbf{Q}$ ( $\mathbf{Q}$ 为有理数集).
%%<SOLUTION>%%
解:在原恒等式中令 $y=1$, 得
$$
f(x)=f(x) \cdot f(1)-f(x+1)+1, x \in \mathbf{Q} .
$$
即
$$
f(x+1)=f(x)+1 .
$$
因此, $f(n)=f(1)+n-1=n+1$.
另外, 在原恒等式中取 $x=\frac{1}{n}, y=n, n \in \mathbf{Z}$, 有
$$
\begin{gathered}
f\left(\frac{1}{n} \cdot n\right)=f\left(\frac{1}{n}\right) \cdot f(n)-f\left(\frac{1}{n}+n\right)+1, \\
2=f\left(\frac{1}{n}\right)(n+1)-f\left(\frac{1}{n}\right)-n+1 .
\end{gathered}
$$
即
$$
2=f\left(\frac{1}{n}\right)(n+1)-f\left(\frac{1}{n}\right)-n+1
$$
所以
$$
f\left(\frac{1}{n}\right)=1+\frac{1}{n}
$$
最后, 我们取 $x=p, y=\frac{1}{q}, p 、 q \in \mathbf{Z}, q \neq 0$, 得
$$
f\left(p, \frac{1}{q}\right)=f(p) f\left(\frac{1}{q}\right)-f\left(p+\frac{1}{q}\right)+1 .
$$
故 $\quad f\left(\frac{p}{q}\right)=(p+1)\left(\frac{1}{q}+1\right)-\frac{1}{q}-p=\frac{p}{q}+1$.
所以, 只有函数 $f(x)=x+1$ 满足条件.
说明这种“爬坡式”的推理技巧称为柯西方法, 后面我们会专门讲述.
%%PROBLEM_END%%



%%PROBLEM_BEGIN%%
%%<PROBLEM>%%
例14 设 $f: \mathbf{R} \rightarrow \mathbf{R}$ 满足如下条件:
(1)对任意实数 $x 、 y$, 有
$$
f(2 x)=f\left(\sin \left(\frac{\pi x}{2}+\frac{\pi y}{2}\right)\right)+f\left(\sin \left(\frac{\pi x}{2}-\frac{\pi y}{2}\right)\right) ;
$$
(2)对任意实数 $x 、 y$,有
$$
f\left(x^2-y^2\right)=(x+y) f(x-y)+(x-y) f(x+y) .
$$
求 $f(2012+\sqrt{2012}+\sqrt[3]{2012})$ 的值.
%%<SOLUTION>%%
解:令 $u=x+y, v=x-y$, 则
$$
\begin{gathered}
f(u+v)=f\left(\sin \frac{u \pi}{2}\right)+f\left(\sin \frac{v \pi}{2}\right), \quad\quad <1>\\
f(u v)=u f(v)+v f(u) . \quad\quad <2>
\end{gathered}
$$
在<2>中令 $u=0, v=2$, 得 $f(0)=0$.
令 $u=0$, 代入 (1), 得 $f(v)=f\left(\sin \frac{v \pi}{2}\right)$, 同理 $f(u)=f\left(\sin \frac{u \pi}{2}\right)$, 所以
$$
f(u+v)=f(u)+f(v) .  \quad\quad <3>
$$
在<2>中令 $u=v=1$, 得 $f(1)=0$.
在<3>中令 $v=1$, 得 $f(u+1)=f(u)$, 从而 $f(2012)=0$.
在<2>中令 $u=v$, 得
$$
f\left(u^2\right)=2 u f(u) .  \quad\quad <4>
$$
所以, $f(2012)=2 \cdot \sqrt{2012} f(\sqrt{2012})$, 故 $f(\sqrt{2012})=0$.
在<2>中令 $v=u^2$, 得
$$
f\left(u^3\right)=u f\left(u^2\right)+u^2 f(u)=3 u^2 f(u),
$$
所以, $f(2012)=3 \cdot 2012^{\frac{2}{3}} f(\sqrt[3]{2012})$, 于是 $f(\sqrt[3]{2012})=0$.
所以
$$
f(2012+\sqrt{2012}+\sqrt[3]{2012})=f(2012)+f(\sqrt{2012})+f(\sqrt[3]{2012})=0 .
$$
%%PROBLEM_END%%



%%PROBLEM_BEGIN%%
%%<PROBLEM>%%
例15 确定所有的函数 $f: \mathbf{R} \rightarrow \mathbf{R}$, 其中 $\mathbf{R}$ 的实数集, 使得对任意 $x, y \in\mathbf{R}$, 恒有
$$
f(x-f(y))=f(f(y))+x f(y)+f(x)-1 .  \quad\quad (1)
$$
%%<SOLUTION>%%
解:记 $I$ 为函数 $f$ 的象集.
设 $f(0)=c$.
在(1)中令 $x=y=0$, 有
$$
f(-c)=f(c)+c-1
$$
所以 $c \neq 0$.
在(1)中令 $x=f(y)$, 有
$$
f(0)=f(x)+x^2+f(x)-1,
$$
故
$$
f(x)=\frac{c+1}{2}-\frac{x^2}{2}, x \in \mathbf{R} . \quad\quad (2)
$$
在(1)中令 $y=0$, 又有
$$
f(x-c)=f(c)+c x+f(x)-1 .
$$
所以
$$
f(x-c)-f(x)=f(c)-1+c x, x \in \mathbf{R} .
$$
注意到 $c \neq 0$, 因此当 $x$ 取遍实数集的时候, $c x+f(c)-1$ 也取遍整个实数集.
即
$$
\{f(c)-1+c x \mid x \in \mathbf{R}\}=\mathbf{R} .
$$
于是
$$
\{f(x-c)-f(x) \mid x \in \mathbf{R}\}=\mathbf{R} \text {. }
$$
因此, 对任何实数 $x$, 存在 $y_1=f\left(x^{\prime}-c\right)$ 及 $y_2=f\left(x^{\prime}\right)$, 使得
$$
x=y_1-y_2 .
$$
再利用(2), 有
$$
\begin{aligned}
f(x) & =f\left(y_1-y_2\right) \\
& =f\left(y_2\right)+y_1 y_2+f\left(y_1\right)-1 \\
& =\frac{c+1}{2}-\frac{y_2^2}{2}+y_1 y_2+\frac{c+1}{2}-\frac{y_1^2}{2}-1 \\
& =c-\frac{1}{2}\left(y_1-y_2\right)^2 \\
& =c-\frac{1}{2} x^2 . \quad\quad  (3)
\end{aligned}
$$
比较(2)、(3)两式即得 $c=1$.
于是, 所求的函数为 $f(x)=1-\frac{x^2}{2}, x \in \mathbf{R}$, 不难验证其满足所有条件.
说明解法中蕴含的“算两次”的思想值得我们重视.
%%PROBLEM_END%%



%%PROBLEM_BEGIN%%
%%<PROBLEM>%%
例16 设 $f(x, y)$ 是二元多项式, 且满足下列条件:
(1) $f(1,2)=2$;
(2) $y f(x, f(x, y))=x f(f(x, y), y)^2$.
试确定所有这样的 $f(x, y)$.
%%<SOLUTION>%%
解:由 (1), $f(1,2)=2$. 再反复利用 (2), 得
$$
\begin{gathered}
f(2,2)=f(f(1,2), 2)=(f(1,2))^2=4, \\
f(4,2)=\frac{1}{2}(2 f(f(2,2), 2))=\frac{1}{2}(f(2,2))^2=8,
\end{gathered}
$$
不难用数学归纳法证明 (这里略), 对任意自然数 $n$, 有
$$
f\left(2^n, 2\right)=2^{n+1} \text {. }
$$
于是, 关于 $x$ 的多项式 $f(x, 2)-2 x$ 有无限多个根 $2^n(n=0,1,2, \cdots)$, 所以 $f(x, 2) \equiv 2 x$.
由于 $f(x, 2)$ 和 $f(x, y)$ 中对各项的 $x$ 次数不变, 所以, $f(x, y)=x g(y)$, 其中 $g(y)$ 是关于 $y$ 的一元多项式.
另外, 由 $f(2,2)=4$, 反复利用 $(2)$, 得
$$
f(2,4)=\frac{1}{2}(2 f(2, f(2,2)))=8,
$$
同样用数学归纳法易证,对任意自然数 $n$,有 $f\left(2,2^n\right)=2^{n+1}$.
类似地,得
$$
f(x, y)=y h(x)
$$
所以 $f(x, y)=x y$.
%%PROBLEM_END%%



%%PROBLEM_BEGIN%%
%%<PROBLEM>%%
例17 求所有的函数 $f: \mathbf{R} \rightarrow \mathbf{R}$, 使得对任意实数 $x 、 y 、 z$, 有
$$
\frac{1}{2} f(x y)+\frac{1}{2} f(x z)-f(x) f(y z) \geqslant \frac{1}{4} . \quad\quad (1)
$$
%%<SOLUTION>%%
解:题设所给的是一个不等式, 而不是方程, 而且变元有三个, 即 $x 、 y$ 、 $z$. 我们设法通过取一些特殊值来寻求结果.
令 $x=y=z=1$, 代入 (1), 得
$$
f(1)-(f(1))^2 \geqslant \frac{1}{4} .
$$
所以 $\left(f(1)-\frac{1}{2}\right)^2 \leqslant 0$.
故
$$
f(1)=\frac{1}{2} \text {. }  \quad\quad (2)
$$
令 $y=z=1$, 代入 (1) 并利用 (2), 得
$$
f(x)-\frac{1}{2} f(x) \geqslant \frac{1}{4} .
$$
所以
$$
f(x) \geqslant \frac{1}{2} .  \quad\quad (3)
$$
令 $x=y=z=0$, 代入 (1), 得
$$
f(0)-f^2(0) \geqslant \frac{1}{4} .
$$
所以
$$
f(0)=\frac{1}{2} . \quad\quad (4)
$$
令 $x=0$, 代入 (1) 并利用 (4), 得
$$
\frac{1}{2}-\frac{1}{2} f(y z) \geqslant \frac{1}{4} \text {. }
$$
故
$$
\begin{aligned}
& f(y z) \leqslant \frac{1}{2}, \\
& f(x) \leqslant \frac{1}{2} .  \quad\quad (5)
\end{aligned}
$$
综合(3)和(5), 即得 $f(x) \equiv \frac{1}{2}$.
%%PROBLEM_END%%



%%PROBLEM_BEGIN%%
%%<PROBLEM>%%
例18 求所有的函数 $f: \mathbf{R} \rightarrow \mathbf{R}$, 使得等式
$$
f([x] y)=f(x)[f(y)] .  \quad\quad <1>
$$
对所有 $x, y \in \mathbf{R}$ 成立.
(这里 $[z]$ 表示不超过实数 $z$ 的最大整数)
%%<SOLUTION>%%
解:答案是 $f(x)=C$ (常数), 这里 $C=0$ 或者 $1 \leqslant C<2$.
令 $x=0$ 代入 <1>, 得
$$
f(0)=f(0)[f(y)] .  \quad\quad <2>
$$
对所有 $y \in \mathbf{R}$ 成立.
于是有如下两种情形:
(1) 当 $f(0) \neq 0$ 时, 由 <2> 知, $[f(y)]=1$ 对所有 $y \in \mathbf{R}$ 成立.
所以, <1> 式为 $f([x] y)=f(x)$. 令 $y=0$, 得 $f(x)=f(0)=C \neq 0$.
由 $[f(y)]=1=[C]$, 知 $1 \leqslant C<2$.
(2) 当 $f(0)=0$ 时, 若存在 $0<\alpha<1$, 使得 $f(\alpha) \neq 0$, 令 $x=\alpha$ 代入 <1> 式, 得
$$
0=f(0)=f(\alpha)[f(y)]
$$
对所有 $y \in \mathbf{R}$ 成立, 所以 $[f(y)]=0$ 对所有 $y \in \mathbf{R}$ 成立.
令 $x=1$ 代入 <1> 式, 得 $f(y)=0$ 对所有 $y \in \mathbf{R}$ 成立,这与 $f(\alpha) \neq 0$ 矛盾.
所以, 我们有 $f(\alpha)=0,0 \leqslant \alpha<1$. 对于任意实数 $z$, 存在整数 $N$, 使得 $\alpha=\frac{z}{N} \in[0,1)$. 由 <1> 式,有
$$
f(z)=f([N] \alpha)=f(N)[f(\alpha)]=0
$$
对所有 $z \in \mathbf{R}$ 成立.
经检验, $f(x)=C$ (常数), 这里 $C=0$ 或者 $1 \leqslant C<2$ 满足题设.
%%PROBLEM_END%%



%%PROBLEM_BEGIN%%
%%<PROBLEM>%%
例19 如果非零连续函数 $f(x)$ 满足函数方程
$$
f\left(\sqrt{x^2+y^2}\right)=f(x) f(y), x, y \in \mathbf{R} .  \quad\quad (1)
$$
证明: $f(x)=(f(1))^{x^2}$.
%%<SOLUTION>%%
证易知 $f(x)$ 是偶函数,所以只需对 $x \geqslant 0$ 的情形进行证明.
当 $x=0$ 时,在 (1) 中取 $x=y=0$, 得
$$
f(0)=(f(0))^2 .
$$
因为 $f(x) \neq 0$, 故 $f(0)=1$, 从而
$$
f(0)=(f(1))^{0^2} .
$$
当 $x=n$ 是正整数时, 先证如下命题:
$$
f(\sqrt{n} y)=(f(y))^n \text {, 其中 } n \in \mathbf{N}_{+}, y \in \mathbf{R}^{+}  .  \quad\quad (2)
$$
$n=1$ 时, (2)显然成立.
设 $n=k$ 时 (2) 式成立.
当 $n=k+1$ 时,
$$
\begin{aligned}
f(\sqrt{k+1} y) & =f\left(\sqrt{k y^2+y^2}\right)=f(\sqrt{k} y) f(y) \\
& =(f(y))^k f(y)=(f(y))^{k+1},
\end{aligned}
$$
从而(2)式得证.
令 $y=1$ 及 $y=\sqrt{n}$ 代入 (2) 式, 分别得
$$
f(\sqrt{n})=(f(1))^n, f(n)=(f(\sqrt{n}))^n .
$$
所以 $f(n)=(f(1))^{n^2}$.
当 $x=\frac{p}{q}\left(p, q \in \mathbf{N}_{+}\right)$为有理数时, 由于
$$
\begin{gathered}
f(p)=(f(1))^{p^2}, \\
f(p)=f\left(\sqrt{q^2} \cdot \frac{p}{q}\right)=\left(f\left(\frac{p}{q}\right)\right)^{q^2} . \\
f\left(\frac{p}{q}\right)^{q^2}=(f(1))^{p^2}, \\
f\left(\frac{p}{q}\right)=(f(1))_{q^{\frac{p^2}{2}}}
\end{gathered}
$$
于是
$$
f{\Bigl(\frac{p}{q}\Bigr)}^{q^{2}}=(f(1))^{p^{2}},  \quad\quad (3)
$$
所以
$$
f{\Bigl(\frac{p}{q}\Bigr)}=(f(1))_{q}^{\frac{\beta^{2}}{2}}. 
$$
由 $f$ 的连续性, 知对于无理数 $x$, 也有
$$
f(x)=(f(1))^{x^2} .
$$
说明以上证明过程中尚有一点需说明,在对(3)式两边开方时必须要求 $f(x)>0$, 这是不难证的.
事实上,
$$
\begin{aligned}
f(x) & =f(|x|)=F\left(\sqrt{\frac{x^2}{2}+\frac{x^2}{2}}\right) \\
& =\left(f\left(\frac{x}{\sqrt{2}}\right)\right)^2 \geqslant 0 .
\end{aligned}
$$
又 $f(x) \neq 0$, 故 $f(x)>0$.
%%PROBLEM_END%%



%%PROBLEM_BEGIN%%
%%<PROBLEM>%%
例20 函数 $f: \mathbf{R} \rightarrow \mathbf{R}$ 满足 $f(1)=1$, 且对任意 $a, b \in \mathbf{R}$, 有
$$
f(a+b)=f(a)+f(b) ; \quad\quad (1)
$$
对任意 $x \neq 0$, 有
$$
f(x) f\left(\frac{1}{x}\right)=1 . \quad\quad (2)
$$
求证: $f(x)=x$.
%%<SOLUTION>%%
证由题设, 得
$$
1=f(1)=f(1+0)=f(1)+f(0),
$$
所以
$$
f(0)=0 \text {. }
$$
又因为
$$
f(x)+f(-x)=f(0)=0,
$$
所以 $f(x)$ 是奇函数.
下面只需证明: 当 $x>0$ 时, $f(x)=x$ 即可.
由(1)式,用数学归纳法易证得: 对任意正整数 $n$ 及正实数 $x$, 有
$$
f(n x)=n f(x) .
$$
令 $x=1$, 便有
$$
f(n)=n .
$$
令 $x=\frac{1}{n}$, 得
$$
f\left(\frac{1}{n}\right)=\frac{1}{n} f(1)=\frac{1}{n}
$$
于是对于正有理数 $\frac{m}{n}$, 有
$$
f\left(\frac{m}{n}\right)=m f\left(\frac{1}{n}\right)=\frac{m}{n} .
$$
最后, 讨论 $x$ 是无理数的情况, 为此先证 $f(x)$ 在 $x=0$ 处连续.
对于任意实数 $y>2$, 必存在正实数 $x$, 满足 $x+\frac{1}{x}=y$, 于是
$$
|f(y)|=\left|f(x)+f\left(\frac{1}{x}\right)\right| \geqslant 2 \sqrt{f(x) f\left(\frac{1}{x}\right)} .
$$
因此, 当 $y<\frac{1}{2}$ 时,
$$
|f(y)|=\left|\left(f\left(\frac{1}{y}\right)\right)^{-1}\right| \leqslant \frac{1}{2} .
$$
任给 $\varepsilon>0$, 总存在正整数 $N=\left[\frac{1}{2 \varepsilon}\right]+1$, 当 $x<\frac{1}{2} \frac{1}{N}$ 时, 有
$$
|f(x)|=\frac{1}{N}\left|f\left(N_x\right)\right| \leqslant \frac{1}{2 N}<\varepsilon .
$$
所以
$$
\lim _{x \rightarrow 0} f(x)=0=f(0) . \quad\quad (3)
$$
对于正无理数 $x$, 存在一有理数数列 $\left\{x_m\right\}, x_m \rightarrow x(m \rightarrow \infty)$, 于是
$$
f(x)=f\left(x_m\right)+f\left(x-x_m\right)=x_m+f\left(x-x_m\right) .
$$
令 $m \rightarrow \infty$, 利用(3)式, 得
$$
f(x)=\lim _{m \rightarrow \infty} x_m=x . 
$$
%%PROBLEM_END%%



%%PROBLEM_BEGIN%%
%%<PROBLEM>%%
例21 设 $f(x)$ 满足柯西方程, 且在 $\mathbf{R}$ 上连续, 求 $f(x)$.
%%<SOLUTION>%%
解:利用 $\circledast$, 用数学归纳法易得: 对任意正整数 $n$ 和实数 $x$,有
$$
f(n x)=n f(x) . \quad\quad (1)
$$
令 $x=1$, 有 $f(n)=n f(1)$.
记 $a=f(1)$, 那么对一切正整数 $n$, 有 $f(n)=a n$.
再令 $x=\frac{m}{n}$ ( $m 、 n$ 是正整数 $)$ 为正有理数, 有
$$
n f\left(\frac{m}{n}\right)=f\left(n \cdot \frac{m}{n}\right)=f(m)=f(m \cdot 1)=m f(1)=a m,
$$
所以 $f\left(\frac{m}{n}\right)=a \cdot \frac{m}{n}$.
又 $\quad f(0)=f(0+0)=f(0)+f(0)$,
所以
$$
f(0)=0=a \cdot 0 \text {. }
$$
$$
f(0)=f(x-x)=f(x)+f(-x),
$$
所以 $f(x)=-f(-x)$,
即 $f(x)$ 是奇函数.
于是从上面可知, 对一切有理数 $r$, 有 $f(r)=a r$.
所以, $g(x)=f(x)-a x$ 在有理数集上处处等于 0 . 因 $f(x)$ 连续,所以 $g(x)$ 也连续,进而 $g(x)=0$.
事实上, 若 $g\left(x_0\right) \neq 0$, 由 $g$ 的连续性知, 存在 $x_0$ 的一个邻域 $u\left(x_0, \varepsilon\right)$, 使 $g(x)$ 在此邻域内处处不等于 0 . 这与它在有理点为 0 矛盾.
所以,对一切实数 $x$, 都有: $f(x)=a x$, 其中 $a=f(1)$.
说明在本题中, 我们要求 $f(x)$ 连续, 其实这个要求太强了.
我们只要 $f(x)$ 在某个区间内有界, 或 $f(x)$ 在某区间内单调均可导出 $f(x)=a x(a=f(1))$.
%%PROBLEM_END%%



%%PROBLEM_BEGIN%%
%%<PROBLEM>%%
例22 求二元一次函数方程:
$$
f(x+y)=f(x)+f(y) . \quad\quad (1)
$$
在 $(-\infty,+\infty)$ 内的一切单调函数的解.
%%<SOLUTION>%%
解:.
同上题可证: 对一切有理数 $r$, 有 $f(r)=a r$, 其中 $a=f(1)$. 设 $\lambda$ 是任意无理数,则必存在有理数 $r_1 、 r_2$, 使得 $r_1<\lambda<r_2$.
根据题目条件, 不妨设 $f(x)$ 是单调递增的, 故
$$
f\left(r_1\right) \leqslant f(\lambda) \leqslant f\left(r_2\right),
$$
即
$$
a r_1 \leqslant f(\lambda) \leqslant a r_2 .
$$
因为可使 $r_1 、 r_2$ 无限接近于 $\lambda$, 故 $f(\lambda)=a \lambda$.
综上所述, 对任意实数 $x$, 有 $f(x)=a x$, 其中 $a=f(1)$.
我们用柯西方法来解稍复杂的函数方程.
%%PROBLEM_END%%



%%PROBLEM_BEGIN%%
%%<PROBLEM>%%
例23 设 $f: \mathbf{Q}^{+} \rightarrow \mathbf{Q}^{+}$( $\mathbf{Q}^{+}$为正有理数的全体) 满足
$$
f(x)=f(x f(y)) y . \quad\quad (1)
$$
试求其一个解.
%%<SOLUTION>%%
解:在(1)中取 $x=y=1$, 得 $f(1)=f(f(1))$.
在(1)中取 $y=f(1)$, 得
$$
\begin{aligned}
f(x) & =f(x f(f(1))) f(1)=f(x f(1)) f(1) \\
& =f(x) f(1) .
\end{aligned}
$$
于是 $f(1)=1$. 因此, 再从 (1) 中取 $x=1$, 得
$$
y f(f(y))=f(1)=1 . \quad\quad (2)
$$
另外,从(1)中取 $y=f(t)$, 得
$$
f(x)=f(t) f(x f(f(t)))=f(t) f\left(\frac{x}{t}\right) .
$$
继续取 $x=s t$, 得
$$
f(s t)=f(t) f(s) . \quad\quad (3)
$$
显然, 若 $f(x)$ 满足 (2) 和 (3), 则 $f(x)$ 也必满足(1). 事实上, $f(x f(y))=f(x) f(f(y))=\frac{f(x)}{y}$. 我们利用这一充要条件 (2) 和 (3) 来构造解.
首先, 当 $x$ 为素数时构造 $f(x)$. 记 $p_k$ 是从小到大的第 $k$ 个素数.
令
$$
f\left(p_k\right)= \begin{cases}p_{k+1}, & \text { 当 } k=2 m+1, m=0,1,2, \cdots \text { 时; } \\ \frac{1}{p_{k-1}}, & \text { 当 } k=2 m, m=1,2, \cdots \text { 时.
}\end{cases}
$$
它显然满足(2).
其次,对任意正整数 $x$ 定义 $f(x)$. 令 $x=p_1^{\alpha_1} p_2^{\alpha_2} \cdots p_s^{\alpha_s}$, 其中 $\alpha_1, \alpha_2, \cdots, \alpha_s$ 是非负整数.
由(3), 得
$$
f(x)=\left(f\left(p_1\right)\right)^{\alpha_1}\left(f\left(p_2\right)\right)^{a_2 \cdots\left(f\left(p_s\right)\right)^{a_s} .} 
$$
最后, 对任意正有理数 $x$, 即 $x=\frac{n}{m} \in \mathbf{Q}^{+}$, 其中 $m, n \in \mathbf{N}_{+}$.
由(3), 得
$$
\begin{gathered}
f(n)=f(m x)=f(m) f(x) . \\
f(x)=\frac{f(n)}{f(m)} .
\end{gathered}
$$
从而这样, 最终我们给出了原函数方程在 $\mathbf{Q}^{+}$上的一个解.
%%PROBLEM_END%%



%%PROBLEM_BEGIN%%
%%<PROBLEM>%%
例24 已知定义的正整数集 $\mathbf{N}_{+}$上的函数 $f(n)$ 满足: $f(n+2)=f(n+1)+f(n)$, 且 $f(1)=f(2)=1$. 求 $f(n)$.
%%<SOLUTION>%%
解:所给递归函数的特征方程为 $x^2=x+1$. 解方程, 得 $x_{1,2}=\frac{1 \pm \sqrt{5}}{2}$.
所以 $\quad f(n)=c_1\left(\frac{1+\sqrt{5}}{2}\right)^n+c_2\left(\frac{1-\sqrt{5}}{2}\right)^n$.
由初始条件 $f(1)=f(2)=1$, 得
$$
\begin{gathered}
\frac{1+\sqrt{5}}{2} c_1+\frac{1-\sqrt{5}}{2} c_2=1 \\
\left(\frac{1+\sqrt{5}}{2}\right)^2 c_1+\left(\frac{1-\sqrt{5}}{2}\right)^2 c_2=1 .
\end{gathered}
$$
解方程组, 得 $c_1=\frac{1}{\sqrt{5}}, c_2=-\frac{1}{\sqrt{5}}$. 所以
$$
f(n)=\frac{1}{\sqrt{5}}\left[\left(\frac{1+\sqrt{5}}{2}\right)^n-\left(\frac{1-\sqrt{5}}{2}\right)^n\right] .
$$
这就是著名的斐波那契数列.
%%PROBLEM_END%%



%%PROBLEM_BEGIN%%
%%<PROBLEM>%%
例25 定义在正整数集 $\mathbf{N}_{+}$上的函数 $f(n)$ 满足 $f(1)=a$,
$$
f(n)=r f(n-1)+q p^{n-1}, n=2,3, \cdots, \quad\quad (1)
$$
其中 $p, q, r$ 为常数, 求 $f(n)$.
%%<SOLUTION>%%
解:反复利用(1)式, 有
$$
\begin{aligned}
f(n) & =r f(n-1)+q p^{n-1} \\
& =r\left(r f(n-2)+q p^{n-2}\right)+q p^{n-1} \\
& =r^2 f(n-2)+q\left(r p^{n-2}+p^{n-1}\right) \\
& =r^2\left(r f(n-3)+q p^{n-3}\right)+q\left(r p^{n-2}+p^{n-1}\right) \\
& =r^3 f(n-3)+q\left(r^2 p^{n-3}+r p^{n-2}+p^{n-1}\right) \\
& \cdots
\end{aligned}
$$
易知
$$
f(n)=r^k f(n-k)+q\left(r^{k-1} p^{n-k}+r^{k-2} p^{n-k+1}+\cdots+p^{n-1}\right) . \quad\quad (2)
$$
上式用数学归纳法容易证明, 在(2)式中令 $k=n-1$, 便得
$$
\begin{aligned}
f(n) & =r^{n-1} f(1)+q\left(r^{n-2} p+r^{n-3} p^2+\cdots+p^{n-1}\right) \\
& = \begin{cases}r^{n-1}[a+(n-1) q], & \text { 当 } p=r \text { 时; } \\
r^{n-1}(a-q)+q \cdot \frac{p^n-r^n}{p-r}, & \text { 当 } p \neq r \text { 时.
}\end{cases}
\end{aligned}
$$
%%PROBLEM_END%%



%%PROBLEM_BEGIN%%
%%<PROBLEM>%%
例26 已知函数 $f$ 定义在正整数集 $\mathbf{N}_{+}$上, $f(1)=\frac{3}{2}$, 并且对任意正整数 $m 、 n$,均有
$$
f(m+n)=\left(1+\frac{n}{m+1}\right) f(m)+\left(1+\frac{m}{n+1}\right) f(n)+m^2 n+m n+m n^2 .
$$
求 $f$.
%%<SOLUTION>%%
解:令 $n=1$, 得
$$
f(m+1)=\left(1+\frac{1}{m+1}\right) f(m)+\left(1+\frac{m}{2}\right) \cdot \frac{3}{2}+m^2+2 m .
$$
整理, 得
$$
\frac{f(m+1)}{m+2}-\frac{f(m)}{m+1}=m+\frac{3}{4} .
$$
于是,利用累差求和的方法,有
$$
\begin{aligned}
\sum_{k=1}^{m-1}\left(\frac{f(k+1)}{k+2}-\frac{f(k)}{k+1}\right) & =\sum_{k=1}^{m-1}\left(k+\frac{3}{4}\right) \\
& =\frac{(m-1) m}{2}+\frac{3}{4}(m-1) .
\end{aligned}
$$
所以
$$
\frac{f(m)}{m+1}-\frac{f(1)}{2}=\frac{1}{4}(m-1)(2 m+3),
$$
故
$$
f(m)=\frac{1}{4} m(m+1)(2 m+1) .
$$
因此所求的函数即为
$$
f(m)=\frac{1}{4} m(m+1)(2 m+1) .
$$
%%PROBLEM_END%%



%%PROBLEM_BEGIN%%
%%<PROBLEM>%%
例27 设 $f: \mathbf{Q} \rightarrow \mathbf{Q}(\mathbf{Q}$ 为有理数集 $)$ 且
$$
f(x+y)=f(x)+f(y)+4 x y \text {, 对任意 } x, y \in \mathbf{Q} \text {. } \quad\quad (1)
$$
如果 $f(-1) f(1) \geqslant 4$, 求 $f(x)$.
%%<SOLUTION>%%
解:取 $x=y=0$, 由 (1), 得 $f(0)=0$.
在(1)中取 $x=1$ 和 $y=-1$, 得
$$
f(0)=f(1)+f(-1)-4,
$$
即 $f(1)+f(-1)=4$.
因为 $f(-1) f(1) \geqslant 4$, 所以 $f(1)$ 和 $f(-1)$ 皆为正数, 从而
$$
4=f(1)+f(-1) \geqslant 2 \sqrt{f(1) f(-1)} \geqslant 4 .
$$
上式等号成立,所以 $f(1)=f(-1)=2$.
再在(1)中取 $y=1$, 得
$$
f(x+1)=f(x)+4 x+2 . \quad\quad (2)
$$
对(2)递推,有
$$
\begin{aligned}
& f(2)=f(1+1)=f(1)+4 \cdot 1+2, \\
& f(3)=f(2+1)=f(2)+4 \cdot 2+2 \text {, } \\
& f(n)=f(n-1+1)=f(n-1)+4(n-1)+2 . \\
&
\end{aligned}
$$
因此
$$
f(n)=f(1)+4(1+2+\cdots+(n-1))+2(n-1)=2 n^2 . \quad\quad (3)
$$
又因为 $\quad f\left(\frac{k}{n}+\frac{1}{n}\right)=f\left(\frac{k}{n}\right)+f\left(\frac{1}{n}\right)+\frac{4 k}{n^2}$,
即
$$
f\left(\frac{k+1}{n}\right)-f\left(\frac{k}{n}\right)=f\left(\frac{1}{n}\right)+\frac{4 k}{n^2} .
$$
利用 $f(1)=2$, 对 $k$ 从 1 到 $n-1$ 求和,得
$$
f\left(\frac{1}{n}\right)=\frac{2}{n^2}   . \quad\quad (4)
$$
同理对 $k$ 从 1 到 $m-1$ 求和,得
$$
f\left(\frac{m}{n}\right)=2 \frac{m^2}{n^2}
$$
最后, 令 $x=\frac{m}{n}, y=-\frac{m}{n}$, 由 (1), 有
$$
f(0)=f\left(\frac{m}{n}-\frac{m}{n}\right)=f\left(\frac{m}{n}\right)+f\left(-\frac{m}{n}\right)-\frac{4 m^2}{n^2},
$$
故
$$
f\left(-\frac{m}{n}\right)=2 \cdot \frac{m^2}{n^2}
$$
因此对所有 $x \in \mathbf{Q}$, 有 $f(x)=2 x^2$.
%%PROBLEM_END%%



%%PROBLEM_BEGIN%%
%%<PROBLEM>%%
例28 已知函数 $f(x)$ 满足 $f\left(x^2\right)-f(x)=1$, 求 $f(x)$.
%%<SOLUTION>%%
解:构造数列 $\left\{x_n\right\}$ 如下:
$$
x_0, x_1=x_0^2, x_2=x_1^2, \cdots, x=x_n=x_{n-1}^2 .
$$
将它们代入所给函数方程, 得
$$
\begin{gathered}
f\left(x_1\right)-f\left(x_0\right)=1, \\
f\left(x_2\right)-f\left(x_1\right)=1, \\
\cdots \cdots \\
f\left(x_n\right)-f\left(x_{n-1}\right)=1 .
\end{gathered}
$$
将上面这些等式相加, 有 $f(x)-f\left(x_0\right)=n$.
因为 $x=x_0^{2^n}$, 所以 $\log _2 \log _2 x-\log _2 \log _2 x_0=n$. 因此
$$
f(x)=f\left(x_0\right)+\log _2 \log _2 x-\log _2 \log _2 x_0 .
$$
令 $c=f\left(x_0\right)-\log _2 \log _2 x_0$, 那么
$$
f(x)=\log _2 \log _2 x+c .
$$
容易验证, 上面的 $f(x)$ 即为所求.
%%PROBLEM_END%%



%%PROBLEM_BEGIN%%
%%<PROBLEM>%%
例29 设 $f(x)$ 是定义在 $\mathbf{R}$ 上的函数, 若 $f(0)=2008$, 且对任意 $x \in \mathbf{R}$, 满足
$$
f(x+2)-f(x) \leqslant 3 \cdot 2^x, f(x+6)-f(x) \geqslant 63 \cdot 2^x,
$$
求 $f(2008)$ 的值.
%%<SOLUTION>%%
解:法一由题设条件知
$$
\begin{aligned}
f(x+2)-f(x)= & -(f(x+4)-f(x+2))-(f(x+6)-f(x+4))+ \\
& (f(x+6)-f(x)) \\
\geqslant & -3 \cdot 2^{x+2}-3 \cdot 2^{x+4}+63 \cdot 2^x=3 \cdot 2^x,
\end{aligned}
$$
因此有 $f(x+2)-f(x)=3 \cdot 2^x$, 故
$$
\begin{aligned}
f(2008)= & f(2008)-f(2006)+f(2006)-f(2004)+\cdots+ \\
& f(2)-f(0)+f(0) \\
= & 3 \cdot\left(2^{2006}+2^{2004}+\cdots+2^2+1\right)+f(0) \\
= & 3 \cdot \frac{4^{1003+1}-1}{4-1}+f(0) \\
= & 2^{2008}+2007 .
\end{aligned}
$$
解法二令 $g(x)=f(x)-2^x$, 则
$$
\begin{aligned}
& g(x+2)-g(x)=f(x+2)-f(x)-2^{x+2}+2^x \leqslant 3 \cdot 2^x-3 \cdot 2^x=0, \\
& g(x+6)-g(x)=f(x+6)-f(x)-2^{x+6}+2^x \geqslant 63 \cdot 2^x-63 \cdot 2^x=0,
\end{aligned}
$$
即
$$
g(x+2) \leqslant g(x), g(x+6) \geqslant g(x),
$$
故 $\quad g(x) \leqslant g(x+6) \leqslant g(x+4) \leqslant g(x+2) \leqslant g(x)$,
得 $g(x)$ 是周期为 2 的周期函数,所以
$$
f(2008)=g(2008)+2^{2008}=g(0)+2^{2008}=2^{2008}+2007 .
$$
%%PROBLEM_END%%



%%PROBLEM_BEGIN%%
%%<PROBLEM>%%
例30 设 $f$ 是一个定义在整数集上取值为正整数的函数, 已知对任意两个整数 $m 、 n$, 差 $f(m)-f(n)$ 能被 $f(m-n)$ 整除.
证明: 对所有整数 $m 、 n$, 若$f(m) \leqslant f(n)$, 则 $f(n)$ 被 $f(m)$ 整除.
%%<SOLUTION>%%
证设整数 $x 、 y$ 使得 $f(x)<f(y)$ (若 $f(x)=f(y)$, 结论显然成立), 令 $m=x, n=y$, 得 $f(x-y)|| f(x)-f(y) \mid$,
即
$$
f(x-y) \mid f(y)-f(x), \quad\quad (1)
$$
所以
$$
f(x-y) \leqslant f(y)-f(x)<f(y),
$$
故差 $d=f(x)-f(x-y)$ 满足
$$
-f(y)<-f(x-y)<d<f(x)<f(y) .
$$
令 $m=x, n=x-y$, 得 $f(y) \mid d$, 故 $d=0$, 从而 $f(x)=f(x-y)$.
由(1)式,得 $f(x) \mid f(y)-f(x)$, 故 $f(x) \mid f(y)$. 从而命题得证.
%%PROBLEM_END%%



%%PROBLEM_BEGIN%%
%%<PROBLEM>%%
例31 求所有的函数 $f:(0,+\infty) \rightarrow(0,+\infty)$, 满足对所有的正实数 $w 、 x 、 y 、 z, w x=y z$, 都有
$$
\frac{(f(w))^2+(f(x))^2}{f\left(y^2\right)+f\left(z^2\right)}=\frac{w^2+x^2}{y^2+z^2} .
$$
%%<SOLUTION>%%
解:令 $w=x=y=z=1$, 得 $(f(1))^2=f(1)$, 所以 $f(1)=1$.
对任意 $t>0$, 令 $w=t, x=1, y=z=\sqrt{t}$, 得 $\frac{(f(t))^2+1}{2 f(t)}=\frac{t^2+1}{2 t}$, 去分母整理得 $(t f(t)-1)(f(t)-t)=0$, 所以, 对每个 $t>0$,
$$
f(t)=t \text {, 或者 } f(t)=\frac{1}{t} . \quad\quad \circledast
$$
若存在 $b, c \in(0,+\infty)$, 使得 $f(b) \neq b, f(c) \neq \frac{1}{c}$, 则由 $\circledast$ 式知, $b 、 c$ 都不等于 1 , 且 $f(b)=\frac{1}{b}, f(c)=c$. 令 $w=b, x=c, y=z=\sqrt{b c}$, 则所以
$$
\begin{aligned}
& \frac{\frac{1}{b^2}+c^2}{2 f(b c)}=\frac{b^2+c^2}{2 b c} \\
& f(b c)=\frac{c+b^2 c^3}{b\left(b^2+c^2\right)} .
\end{aligned}
$$
因为 $f(b c)=b c$, 或者 $f(b c)=\frac{1}{b c}$. 若 $f(b c)=b c$, 则
$$
b c=\frac{c+b^2 c^3}{b\left(b^2+c^2\right)}
$$
得 $b^4 c=c, b=1$, 矛盾! 若 $f(b c)=\frac{1}{b c}$, 则
$$
\frac{1}{b c}=\frac{c+b^2 c^3}{b\left(b^2+c^2\right)}
$$
得 $b^2 c^4=b^2, c=1$, 矛盾!
所以, 或者 $f(x)=x, x \in(0,+\infty)$, 或者 $f(x)=\frac{1}{x}, x \in(0,+\infty)$.
经检验, $f(x)==x, x \in(0,+\infty)$ 和 $f(x)=\frac{1}{x}, x \in(0,+\infty)$ 都满足要求.
%%PROBLEM_END%%



%%PROBLEM_BEGIN%%
%%<PROBLEM>%%
例32 给定整数 $n \geqslant 2$. 求所有的函数 $f: \mathbf{R} \rightarrow \mathbf{R}$, 满足对任意实数 $x 、 y$, 都有
$$
f(x-f(y))=f\left(x+y^n\right)+f\left(f(y)+y^n\right) .
$$
%%<SOLUTION>%%
解:令 $x=f(y)$, 则 $f\left(f(y)+y^n\right)=\frac{1}{2} f(0)$. 令 $z=x-f(y)$, 则
$$
x+y^n=z+f(y)+y^n .
$$
于是有
$$
f(z)=f\left(z+f(y)+y^n\right)+\frac{1}{2} f(0) . \quad\quad (1)
$$
令 $S=\left\{f(y)+y^n \mid y \in \mathbf{R}\right\}$. 下面分两种情形讨论.
情形一: $S$ 中仅含一个元素, 于是存在常数 $a$ 使得 $f(y)+y^n=a$ 对任意 $y$ 成立.
代入 (1) 式中即得
$$
f(z)=f(z+a)+\frac{1}{2} f(0),
$$
即有 $-z^n+a=-(z+a)^n+a+\frac{1}{2} a$ 对所有实数 $z$ 成立, 故 $a=0$. 易验证 $f(x)=-x^n$ 满足要求.
情形二: $S$ 含有多于一个元素.
由(1)式知对任意 $s \in S$, 都有 $f(z)=f(z+s)+\frac{1}{2} f(0)$. 于是对 $s_1, s_2 \in S$ (不妨设 $\left.s_1>s_2\right), f\left(z+s_1\right)=f\left(z+s_2\right)$ 成立, 即 $f$ 是以 $s_1-s_2$ 为周期的周期函数.
由于 $S$ 至少含有两个元素, 故存在 $\lambda>0$, 使得 $f$ 以 $\lambda$ 为周期.
于是
$$
\left(f(y+\lambda)+(y+\lambda)^n\right)-\left(f(y)+y^n\right)=(y+\lambda)^n-y^n
$$
也是 $f$ 的周期.
这是关于 $y$ 的 $n-1$ 次多项式, 因为首项系数 $n \lambda>0$, 所以该多项式可以取到充分大的所有实数, 即充分大的实数均为 $f$ 的周期, 于是所有实数均为 $f$ 的周期, 这样 $f$ 是常数函数.
由 (1) 式即知 $f(x)=0$, 易验证 $f(x)=$ 0 满足要求.
综上所述, 所有满足要求的函数为 $f(x)=-x^n$ 和 $f(x)=0$.
%%PROBLEM_END%%



%%PROBLEM_BEGIN%%
%%<PROBLEM>%%
例33 设 $f: \mathbf{R} \rightarrow \mathbf{R}$ 是一个定义在实数集上的实值函数, 满足对所有实数 $x 、 y$, 都有
$$
f(x+y) \leqslant y f(x)+f(f(x)), \quad\quad (1)
$$
证明: 对所有实数 $x \leqslant 0$, 有 $f(x)=0$.
%%<SOLUTION>%%
证令 $y=t-x$, 则 (1) 式可写为
$$
f(t) \leqslant t f(x)-x f(x)+f(f(x)) . \quad\quad (2)
$$
在(2)式中分别令 $t=f(a), x=b$ 和 $t=f(b), x=a$, 可得
$$
\begin{aligned}
& f(f(a))-f(f(b)) \leqslant f(a) f(b)-b f(b), \\
& f(f(b))-f(f(a)) \leqslant f(a) f(b)-a f(a) .
\end{aligned}
$$
把上面两式相加得
$$
2 f(a) f(b) \geqslant a f(a)+b f(b) .
$$
令 $b=2 f(a)$, 得 $2 f(a) f(b) \geqslant a f(a)+2 f(a) f(b)$, 即 $a f(a) \leqslant 0$. 所以
$$
f(a) \geqslant 0 \text {, 当 } a<0 \text { 时.
} \quad\quad (3)
$$
假设存在某个实数 $x$, 使得 $f(x)>0$. 由 (2) 式, 对每个 $t<\frac{x f(x)-f(f(x))}{f(x)}$, 有 $f(t)<0$, 这与 (3) 矛盾.
所以, 对所有实数 $x$
$$
f(x) \leqslant 0, \quad\quad (4)
$$
结合 (3)知, 对所有实数 $x<0$, 有 $f(x)=0$.
在(2)中令 $t=x<0$, 得 $f(x) \leqslant f(f(x))$, 故 $0 \leqslant f(0)$, 结合 (4), 知 $f(0)=0$.
综上可知, 对任意 $x \leqslant 0$, 都有 $f(x)=0$.
%%PROBLEM_END%%



%%PROBLEM_BEGIN%%
%%<PROBLEM>%%
例34 对每一个正整数 $n$, 求具有下述性质的最大常数 $C_n$ : 对任意 $n$ 个定义在闭区间 $[0,1]$ 上的实值函数 $f_1(x), f_2(x), \cdots, f_n(x)$, 都存在实数 $x_1, x_2, \cdots, x_n$, 满足 $0 \leqslant x_i \leqslant 1$, 且 $\mid f_1\left(x_1\right)+f_2\left(x_2\right)+\cdots+f_n\left(x_n\right)-x_1 x_2 \cdots x_n \mid \geqslant C_n$.
%%<SOLUTION>%%
解:所求的最大常数 $C_n=\frac{n-1}{2 n}$.
一方面, 取 $x_1=x_2=\cdots=x_n=1$, 得题中不等式的左式 $=\mid \sum_{i=1}^n f_i(1)-1 \mid$, 取 $x_1=x_2=\cdots=x_n=0$, 得不等式左式 $=\left|\sum_{i=1}^n f_i(0)\right|$, 取 $x_i=0, x_j=1(j \neq i)$, 得不等式左式 $=\left|\sum_{j \neq i} f_j(1)+f_i(0)\right|$. 由三角形不等式可知
$$
\begin{aligned}
& (n-1)\left|\sum_{i=1}^n f_i(1)-1\right|+\sum_{i=1}^n\left|\sum_{j \neq i} f_j(1)+f_i(0)\right|+\left|\sum_{i=1}^n f_i(0)\right| \\
\geqslant & \left|(n-1)\left(\sum_{i=1}^n f_i(1)-1\right)-\sum_{i=1}^n\left(\sum_{j \neq i} f_j(1)+f_i(0)\right)+\sum_{i=1}^n f_i(0)\right|=n-1 .
\end{aligned}
$$
故 $\left|\sum_{i=1}^n f_i(1)-1\right|,\left|\sum_{i=1}^n f_i(0)\right|,\left|\sum_{j \neq i} f_j(1)+f_i(0)\right|(i=1,2, \cdots, n)$ 中必有一个数不小于 $\frac{n-1}{2 n}$, 从而, $C_n \geqslant \frac{n-1}{2 n}$.
另一方面, 令 $f_i(x)=\frac{x}{n}-\frac{n-1}{2 n^2}, i=1,2, \cdots, n$, 我们证明: 对任意实数 $x_1, x_2, \cdots, x_n \in[0,1]$, 都有 $\mid f_1\left(x_1\right)+f_2\left(x_2\right)+\cdots+f_n\left(x_n\right)-x_1 x_2 \cdots x_n \mid \leqslant \frac{n-1}{2 n}$.
为此, 只需证明: $1-n \leqslant n x_1 \cdots x_n-\sum_{i=1}^n x_i \leqslant 0$.
左边不等式等价于 $(n-1) x_1 \cdots x_n+\left(x_1-1\right)\left(x_2 \cdots x_n-1\right)+\cdots+\left(x_{n-1}-\right.$ 1) $\left(x_n-1\right) \geqslant 0$, 此式中每一个加项都不小于 0 , 故成立.
右边不等式等价于 $\sum_{i=1}^n x_i-n x_1 \cdots x_n \geqslant 0 \Leftrightarrow \sum_{i=1}^n x_i\left(1-\frac{x_1 \cdots x_n}{x_i}\right) \geqslant 0$, 同上可知亦成立.
所以, $C_n \leqslant \frac{n-1}{2 n}$.
综上所述, 所求的最大常数 $C_n=\frac{n-1}{2 n}$.
%%PROBLEM_END%%


