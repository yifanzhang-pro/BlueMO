
%%TEXT_BEGIN%%
几个常见的初等函数相这一节我们来讨论几个常见的初等函数的性质及其解题方法.
3.1 二次函数二次函数在中学数学中占有重要地位.
它形式简单, 应用极其广泛.
$f(x)=a x^2+b x+c(a \neq 0)$ 称为二次函数.
也常常写成:
$$
f(x)=a(\dot{x}-k)^2+m(a \neq 0) \text { (顶点式) }
$$
或
$$
f(x)=a\left(x-x_1\right)\left(x-x_2\right)(a \neq 0) \text { (零点式). }
$$
二次函数 $f(x)=a x^2+b x+c(a \neq 0)$ 的性质.
(1) 对称性对任意实数 $x$, 有
$$
f\left(-\frac{b}{2 a}+x\right)=f\left(-\frac{b}{2 a}-x\right) .
$$
(2) $f(0)=c$.
(3) 若 $\Delta=b^2-4 a c \geqslant 0$, 则 $f\left(\frac{-b \pm \sqrt{\Delta}}{2 a}\right)=0$.
(4) 当 $a>0$ 时, $f(x)$ 在区间 $\left(-\infty,-\frac{b}{2 a}\right]$ 上递减,在区间 $\left[-\frac{b}{2 a},+\infty\right)$上递增;
当 $a<0$ 时, $f(x)$ 在区间 $\left(-\infty,-\frac{b}{2 a}\right]$ 上递增, 在区间 $\left[-\frac{b}{2 a},+\infty\right)$ 上递减.
(5) 当 $a>0$ 时, $f(x)$ 有最小值 $f_{\text {min }}(x)=f\left(-\frac{b}{2 a}\right)=\frac{4 a c-b^2}{4 a}$;
当 $a<0$ 时, $f(x)$ 有最大值 $f_{\text {max }}(x)=f\left(-\frac{b}{2 a}\right)=\frac{4 a c-b^2}{4 a}$.
二次函数 $f(x)=a x^2+b x+c(a \neq 0)$ 的图象.
(1) 对称轴 $f(x)$ 关于直线 $x=-\frac{b}{2 a}$ 对称.
(2) 顶点对称轴与抛物线的交点 $\left(-\frac{b}{2 a}, \frac{4 a c-b^2}{4 a}\right)$ 称为顶点.
当 $a>0$ 时, 顶点是抛物线的最低点; 当 $a<0$ 时, 顶点是抛物线的最高点.
(3) 开口当 $a>0$ 时,开口向上; 当 $a<0$ 时,开口向下.
开口大小由 $|a|$ 决定.
(4) 二次函数图象与 $x$ 轴的位置关系当 $\Delta>0$ 时, 二次函数图象与 $x$ 轴有两个不同的交点; 当 $\Delta=0$ 时,二次函数图象与 $x$ 轴相切; 当 $\Delta<0$ 时, 二次函数与 $x$ 轴无交点.
%%TEXT_END%%



%%TEXT_BEGIN%%
3.2 幕函数、指数函数和对数函数形如 $y=x^\alpha$ 的函数叫做幕函数,其中 $x$ 是自变量, $\alpha$ 是常数,在中学阶段, 我们只研究 $\alpha \in \mathbf{Q}$ 的情况.
幂函数的图象是不通过第四象限的一条曲线.
当 $\alpha>0$ 时, 图象都通过 $(0,0),(1,1)$, 在 $(0,+\infty)$ 上是增函数; 当 $\alpha<0$ 时, 图象都通过点 $(1,1)$, 在 $(0,+\infty)$ 上是减函数.
形如 $y=a^x(a>0, a \neq 1)$ 的函数叫做指数函数, 其定义域为 $\mathbf{R}$, 值域为 $(0,+\infty)$. 当 $0<a<1$ 时, $y=a^x$ 是减函数; 当 $a>1$ 时, $y=a^x$ 是增函数.
形如 $y=\log _a x(a>0, a \neq 1)$ 的函数叫做对数函数, 其定义域为 $(0 , +\infty)$, 值域为 $\mathbf{R}$. 当 $0<a<1$ 时, $y=\log _a x$ 是减函数, 当 $a>1$ 时, $y=\log _a x$ 是增函数.
%%TEXT_END%%



%%TEXT_BEGIN%%
3.3 函数 $f(x)=x+\frac{a^2}{x}$
函数 $f(x)=x+\frac{a^2}{x}(a>0)$ 具有很广泛的应用,下面我们来讨论它的一些性质.
$f(x)=x+\frac{a^2}{x}$ 的定义域为 $(-\infty, 0) \cup(0,+\infty)$.
当 $x>0$ 时, $f(x)=x+\frac{a^2}{x} \geqslant 2 a$;
当 $x=a$ 时,等号成立;
当 $x<0$ 时, $f(x)=-\left((-x)+\frac{a^2}{(-x)}\right) \leqslant-2 a$;
当 $x=-a$ 时等号成立.
所以, $f(x)=x+\frac{a^2}{x}$ 的值域为 $(-\infty,-2 a] \cup[2 a,+\infty)$.
易知 $f(x)=x+\frac{a^2}{x}$ 是奇函数.
设 $x_1, x_2 \in(0,+\infty)$, 则
$$
\begin{aligned}
f\left(x_2\right)-f\left(x_1\right) & =\left(x_2-x_1\right)+a^2\left(\frac{1}{x_2}-\frac{1}{x_1}\right) \\
& =\left(x_2-x_1\right)\left(1-\frac{a^2}{x_1 x_2}\right) .
\end{aligned}
$$
当 $0<x_1<x_2 \leqslant a$ 时, 得 $f\left(x_2\right)-f\left(x_1\right)<0$ ;
当 $a \leqslant x_1<x_2$ 时, 得 $f\left(x_2\right)-f\left(x_1\right)>0$.
所以, $f(x)=x+\frac{a^2}{x}(a>0)$ 在 $(0, a]$ 上是单调递减的, 在 $[a,+\infty)$ 上是单调递增的.
用同样的方法可知, $f(x)=x+\frac{a^2}{x}(a>0)$ 在 $(-\infty,-a]$ 上是单调递增的, 在 $[-a, 0)$ 上是单调递减的.
函数 $f(x)=x+\frac{a^2}{x}(a>0)$ 的图象如图 (<FilePath:./figures/fig-c3d3.png>) 所示.
%%TEXT_END%%



%%PROBLEM_BEGIN%%
%%<PROBLEM>%%
例1 设二次函数 $f(x)$ 满足 $f(x-2)=f(-x-2), x \in \mathbf{R}$, 且 $f(x)$ 的图象在 $y$ 轴上的截距为 1 , 在 $x$ 轴上截得的线段长为 $2 \sqrt{2}$, 求 $f(x)$ 的解析式.
%%<SOLUTION>%%
分析:由 $f(x-2)=f(-x-2)$ 知, $f(x)$ 的图象关于 $x=-2$ 对称.
解由题设知, $f(x)$ 的图象有对称轴 $x=-2$, 由 $f(x)$ 的图象在 $x$ 轴上截得的线段长为 $2 \sqrt{2}$, 可得 $f(x)$ 的图象与 $x$ 轴的交点为 $(-2-\sqrt{2}, 0),(-2+\sqrt{2}, 0)$,于是可设
$$
f(x)=a(x+2+\sqrt{2})(x+2-\sqrt{2}) .
$$
又 $f(0)=1$, 所以
$$
a(2+\sqrt{2})(2-\sqrt{2})=1 .
$$
解方程, 得 $a=\frac{1}{2}$.
所以
$$
f(x)=\frac{1}{2}(x+2+\sqrt{2})(x+2-\sqrt{2}),
$$
即
$$
f(x)=\frac{1}{2} x^2+2 x+1 .
$$
%%PROBLEM_END%%



%%PROBLEM_BEGIN%%
%%<PROBLEM>%%
例2 若函数 $f(x)=-\frac{1}{2} x^2+\frac{13}{2}$ 在区间 $[a, b]$ 上的最小值为 $2 a$, 最大值为 $2 b$, 求 $[a, b]$.
%%<SOLUTION>%%
解:分如下三种情形来讨论区间 $[a, b]$.
(1) 当 $a<b \leqslant 0$ 时, $f(x)$ 在区间 $[a, b]$ 上单调递增, 所以 $f(a)=2 a$, $f(b)=2 b$, 即
$$
\left\{\begin{array}{l}
-\frac{1}{2} a^2+\frac{13}{2}=2 a, \\
-\frac{1}{2} b^2+\frac{13}{2}=2 b .
\end{array}\right.
$$
所以 $a 、 b$ 是方程 $-\frac{1}{2} x^2-2 x+\frac{13}{2}=0$ 的两个不同实根, 而方程 $-\frac{1}{2} x^2-2 x+\frac{13}{2}=0$ 的两根异号, 不可能.
(2) 当 $a<0<b$ 时, $f(x)$ 在 $[a, 0]$ 上递增, 在 $[0, b]$ 上递减.
故 $f(0)=2 b$, 且 $\min \{f(a), f(b)\}=2 a$.
由 $f(0)=2 b$, 得 $b=\frac{13}{4}$. 于是
$$
f(b)=-\frac{1}{2}\left(\frac{13}{4}\right)^2+\frac{13}{2}=\frac{39}{32}>0 .
$$
而 $a<0$, 故 $f(b) \neq 2 a$, 所以 $f(a)=2 a$, 即
$$
-\frac{1}{2} a^2+\frac{13}{2}=2 a \text {. }
$$
解方程, 得 $a=-2-\sqrt{17}$. 此时 $[a, b]=\left[-2-\sqrt{17}, \frac{13}{4}\right]$.
(3). 当 $0 \leqslant a<b$ 时, $f(x)$ 在 $[a, b]$ 上递减, 于是 $f(a)=2 b, f(b)=2 a$, 即
$$
\left\{\begin{array}{l}
-\frac{1}{2} a^2+\frac{13}{2}=2 b, \\
-\frac{1}{2} b^2+\frac{13}{2}=2 a .
\end{array}\right.
$$
解方程组, 得 $a=1, b=3$, 此时 $[a, b]==[1,3]$.
综上所述,所求的区间 $[a, b]$ 为 $[1,3]$ 或 $\left[-2-\sqrt{17}, \frac{13}{4}\right]$.
%%PROBLEM_END%%



%%PROBLEM_BEGIN%%
%%<PROBLEM>%%
例3 设 $f(x)=a x^2+b x+c(a>0)$, 方程 $f(x)=x$ 的两个根是 $x_1$ 和 $x_2$, 且 $x_1>0, x_2-x_1>\frac{1}{a}$. 又若 $0<t<x_1$, 试比较 $f(t)$ 与 $x_1$ 的大小.
%%<SOLUTION>%%
解:因为 $x_1 、 x_2$ 是方程
$$
a x^2+b x+c=x
$$
的两个根, 所以
$$
\begin{gathered}
x_1+x_2=-\frac{b-1}{a}, x_1 x_2=\frac{c}{a}, \\
a x_1^2+b x_1+c=x_1 .
\end{gathered}
$$
因此
$$
\begin{aligned}
f(t)-x_1 & =\left(a t^2+b t+c\right)-\left(a x_1^2+b x_1+c\right) \\
& =a\left(t+x_1\right)\left(t-x_1\right)+b\left(t-x_1\right) \\
& =a\left(t-x_1\right)\left(t+x_1+\frac{b}{a}\right) .
\end{aligned}
$$
由
$$
\begin{aligned}
t+x_1+\frac{b}{a} & =t+\left(\frac{1}{a}-x_2\right) \\
& =\left(t+\frac{1}{a}\right)-x_2<\left(x_1+\frac{1}{a}\right)-x_2<0,
\end{aligned}
$$
及 $a>0, t-x_1<0$, 得 $f(t)-x_1>0$.
所以,当 $0<t<x_1$ 时,有 $f(t)>x_1$.
%%PROBLEM_END%%



%%PROBLEM_BEGIN%%
%%<PROBLEM>%%
例4 若关于 $x$ 的方程 $3 x^2-5 x+a=0$ 的一根大于 -2 而小于 0 , 另一根大于 1 小于 3 , 求 $a$ 的取值范围.
%%<SOLUTION>%%
分析:此题由于抛物线开口向上, 故只要画出一个草图就能发现: 当 $f(-2)>0, f(0)<0, f(1)<0, f(3)>0$ 时就能使方程的根满足题设条件, 然而这些条件是必需的吗? 对这个问题的分析就要用到二次函数的对称轴了.
此外本题也可采用更为自然的方法, 即利用求根公式将两根用关于 $a$ 的代数式表示出来, 再利用两根的分布条件及判别式非负转化为关于 $a$ 的不等式.
解法一由于二次函数 $y=3 x^2-5 x+a=0$ 的开口向上, 而二次方程 $y=3 x^2-5 x+a=0$ 如有两个不相等的实根, 必位于二次函数 $y=3 x^2-5 x+a=0$ 的对称轴的两侧.
对称轴左侧, 函数递减; 对称轴右侧, 函数递增.
所以必须有
$$
\left\{\begin{array}{l}
f(-2)>0, \\
f(0)<0, \\
f(1)<0, \\
f(3)>0 .
\end{array}\right.
$$
反过来在这种情况下, 此方程必有两满足题设的相异实根.
解此不等式组得 $-12<a<0$.
解法二此方程的两根为 $x_{1,2}=\frac{5 \pm \sqrt{25-12 a}}{6}$. 由题设, 得
$$
\left\{\begin{array}{l}
25-12 a \geqslant 0, \\
-2<\frac{5-\sqrt{25-12 a}}{6}<0 \\
1<\frac{5+\sqrt{25-12 a}}{6}<3 .
\end{array}\right.
$$
解不等式组, 得 $-12<a<0$.
说明本题属于二次方程的根的分布问题, 涉及二次方程所对应的二次函数及二次不等式.
解决这类问题的常用工具是求根公式、韦达定理及二次函数的图象特性.
一般需要借助草图帮助思考, 从几何思考人手, 用代数方法解决.
%%PROBLEM_END%%



%%PROBLEM_BEGIN%%
%%<PROBLEM>%%
例5 已知函数 $f(x)=3 a x^2+2 b x+c(a \neq 0)$, 当 $0 \leqslant x \leqslant 1$ 时, $|f(x)| \leqslant$ 1 , 试求 $a$ 的最大值.
%%<SOLUTION>%%
解:由 $\left\{\begin{array}{l}f(0)=c, \\ f\left(\frac{1}{2}\right)=\frac{3}{4} a+b+c, \text { 得 } \\ f(1)=3 a+2 b+c\end{array}\right.$
$$
3 a=2 f(0)+2 f(1)-4 f\left(\frac{1}{2}\right) \text {. }
$$
所以
$$
\begin{aligned}
3|a| & =\left|2 f(0)+2 f(1)-4 f\left(\frac{1}{2}\right)\right| \\
& \leqslant 2|f(0)|+2|f(1)|+4\left|f\left(\frac{1}{2}\right)\right| \leqslant 8,
\end{aligned}
$$
故 $a \leqslant \frac{8}{3}$.
又易知当 $f(x)=8 x^2-8 x+1$ 时满足题设条件, 所以 $a$ 的最大值为 $\frac{8}{3}$.
%%PROBLEM_END%%



%%PROBLEM_BEGIN%%
%%<PROBLEM>%%
例6 设二次函数 $f(x)=a x^2+b x+c(a, b, c \in \mathbf{R} ; a \neq 0)$ 满足条件:
(1) 当 $x \in \mathbf{R}$ 时, $f(x-4)=f(2-x)$, 且 $f(x) \geqslant x$;
(2) 当 $x \in(0,2)$ 时, $f(x) \leqslant\left(\frac{x+1}{2}\right)^2$;
(3) $f(x)$ 在 $\mathbf{R}$ 上的最小值为 0 .
求最大的 $m(m>1)$, 使得存在 $t \in \mathbf{R}$, 只要 $x \in[1, m]$, 就有 $f(x+t) \leqslant x$.
%%<SOLUTION>%%
分析:先根据题设条件 (1)、(2)、(3), 把 $f(x)$ 的解析式求出来, 进而再确定 $m$ 的最大值.
解由 $f(x-4)=f(2-x), x \in \mathbf{R}$, 可知二次函数 $f(x)$ 的对称轴为 $x=-1$. 又由 (3) 知, 二次函数 $f(x)$ 的开口向上, 即 $a>0$, 故可设
$$
f(x)=a(x+1)^2,(a>0) .
$$
由 (1) 知, $f(1) \geqslant 1$, 由 (2) 知, $f(1) \leqslant\left(\frac{1+1}{2}\right)^2=1$.
所以 $f(1)=1$, 故
$$
1=a(1+1)^2, a=\frac{1}{4} .
$$
所以
$$
f(x)=\frac{1}{4}(x+1)^2 .
$$
因为 $f(x)=\frac{1}{4}(x+1)^2$ 的图象开口向上, 而 $y=f(x+t)$ 的图象是由 $y=f(x)$ 的图象平移 $|t|$ 个单位得到.
要在区间 $[1, m]$ 上, 使得 $y=f(x+t)$ 的图象在 $y=x$ 的图象的下方, 且 $m$ 最大, 则 1 和 $m$ 应当是关于 $x$ 的方程
$$
\frac{1}{4}(x+t+1)^2=x
$$
的两个根.
令 $x=1$ 代入方程(1), 得 $t=0$ 或 $t=-4$.
当 $t=0$ 时,方程(1)的解为 $x_1=x_2=1$ (这与 $m>1$ 矛盾!);
当 $t=-4$ 时,方程(1)的解为 $x_1=1, x_2=9$, 所以 $m=9$.
又当 $t=-4$ 时,对任意 $x \in[1,9]$, 恒有
$$
\begin{aligned}
& (x-1)(x-9) \leqslant 0 \\
\Leftrightarrow & \frac{1}{4}(x-4+1)^2 \leqslant x,
\end{aligned}
$$
即 $f(x-4) \leqslant x$.
所以, $m$ 的最大值为 9 .
说明我们由 $f(x-4)=f(2-x), x \in \mathbf{R}$ 导出 $f(x)$ 的图象关于 $x=$ -1 对称.
一般地, 若 $f(x-a)=f(b-x), x \in \mathbf{R}$, 则
$$
f\left(x+\frac{b-a}{2}\right)=f\left(x+\frac{b+a}{2}-a\right)=f\left(b-x-\frac{b+a}{2}\right)
$$
$$
=f\left(\frac{b-a}{2}-x\right)
$$
故 $f(x)$ 的图象关于 $x=\frac{b-a}{2}$ 对称.
这个性质在解题中常常用到.
%%PROBLEM_END%%



%%PROBLEM_BEGIN%%
%%<PROBLEM>%%
例7 函数 $f(x)=\log _{\frac{1}{2}}\left(x^2-2 x-3\right)$ 的单调递增区间是 $(\quad)$.
A. $(-\infty,-1)$
B. $(-\infty, 1)$
C. $(1,+\infty)$
D. $(3,+\infty)$
%%<SOLUTION>%%
解:由 $x^2-2 x-3>0$, 得函数的定义域为 $(-\infty,-1) \cup(3,+\infty)$.
而 $u=x^2-2 x-3=(x-1)^2-4$ 在 $(-\infty,-1)$ 上单调递减, 在 $(3,+\infty)$ 上单调递增, 所以, $f(x)=\log _{\frac{1}{2}}\left(x^2-2 x-3\right)$ 在 $(-\infty,-1)$ 上单调递增, 在 $(3,+\infty)$ 上单调递减, 故应选 $\mathrm{A}$.
%%PROBLEM_END%%



%%PROBLEM_BEGIN%%
%%<PROBLEM>%%
例8 已知 $f(x)=\left(\frac{1}{2^x-1}+\frac{1}{2}\right) x$.
(1) 请判断 $f(x)$ 的奇偶性, 并证明 $f(x)>0$;
(2) 设 $F(x)=f(x+t)-f(x-t)(t \neq 0)$, 判断 $F(x)$ 的奇偶性.
%%<SOLUTION>%%
分析:第(1) 小题的解决可通过尝试的办法, 根据 $f(x)$ 的解析式的结构特点, 计算 $f(x)+f(-x)$ 及 $f(x)-f(-x)$ 可能比直接判断 $f(x)$ 与 $f(-x)$ 的关系来得更加方便.
解决第 (2) 小题的关键在于理解 $t$ 是一个与 $x$ 无关的常数.
解 (1) 首先容易确定 $f(x)$ 的定义域为 $\{x \mid x \in \mathbf{R}$, 且 $x \neq 0\}$, 关于原点对称, 而
$$
f(x)-f(-x)=\left(\frac{1}{2^x-1}+\frac{1}{2}\right) x-\left(\frac{1}{2^{-x}-1}+\frac{1}{2}\right)(-x)=0 .
$$
于是 $f(x)=f(-x)$. 故 $f(x)$ 是偶函数.
当 $x>0$ 时, $2^x-1>0 \Rightarrow f(x)>0$, 而当 $x<0$ 时, 由于 $f(x)$ 是偶函数及 $f(x)>0$, 所以 $f(x)$ 在其定义域上恒大于 0 .
(2) 由 $\left\{\begin{array}{l}x+t \neq 0, \\ x-t \neq 0,\end{array}\right.$ 知 $F(x)$ 的定义域为 $\{x \mid x \in \mathbf{R}$, 且 $x \neq \pm t\}$.
因为
$$
\begin{aligned}
F(-x) & =f(-x+t)-f(-x-t) \\
& =f(-(x-t))-f(-(x+t)) \\
& =f(x-t)-f(x+t)=-F(x) .
\end{aligned}
$$
所以 $F(x)$ 为奇函数.
说明本题在两次判断函数的奇偶性时采用了不同的方法.
这是因为一般来说如果函数的解析式中含有分式、指数式、对数式时, 判断奇偶性用作差法或作商法可能更加简洁自然.
%%PROBLEM_END%%



%%PROBLEM_BEGIN%%
%%<PROBLEM>%%
例9 若函数 $g(x)$ 的图象与函数 $f(x)=\frac{1-2^x}{1+2^x}$ 的图象关于直线 $y=x$ 对称, 求 $g\left(\frac{3}{5}\right)$ 的值.
%%<SOLUTION>%%
解:法一先求出 $g(x)$ 的解析式.
因为
$$
\begin{gathered}
f(x)=\frac{1-2^x}{1+2^x}, \\
2^x=\frac{1-f(x)}{1+f(x)}, \\
x=\log _2 \frac{1-f(x)}{1+f(x)} .
\end{gathered}
$$
所以
$$
2^x=\frac{1-f(x)}{1+f(x)}
$$
因此 $g(x)=f^{-1}(x)=\log _2 \frac{1-x}{1+x},(-1<x<1)$,
所以 $g\left(\frac{3}{5}\right)=\log _2 \frac{1-\frac{3}{5}}{1+\frac{3}{5}}=\log _2 \frac{1}{4}=-2$.
解法二由题意知, 函数 $g(x)$ 与 $f(x)$ 互为反函数, 所以 $g\left(\frac{3}{5}\right)$ 的值应是方程 $\frac{1-2^x}{1+2^x}=\frac{3}{5}$ 的解.
解方程可得 $x=-2$, 故 $g\left(\frac{3}{5}\right)=-2$.
说明解法一是常规的方法, 在求反函数时注意定义域和值域.
解法二巧妙地利用了反函数的概念, 这是一种非常简便有效的方法.
%%PROBLEM_END%%



%%PROBLEM_BEGIN%%
%%<PROBLEM>%%
例10 已知 $0<a<1, x^2+y=0$, 求证:
$$
\log _a\left(a^x+a^y\right) \leqslant \log _a 2+\frac{1}{8} .
$$
%%<SOLUTION>%%
证因为 $0<a<1$, 所以 $a^x>0, a^y>0$, 由平均不等式
$$
a^x+a^y \geqslant 2 \sqrt{a^x \cdot a^y}=2 a^{\frac{x+y}{2}} .
$$
故
$$
\begin{aligned}
\log _a\left(a^x+a^y\right) & \leqslant \log _a\left(2 a^{\frac{x+y}{2}}\right) \\
& =\log _a 2+\frac{1}{2}(x+y) \\
& =\log _a 2+\frac{1}{2}\left(x-x^2\right) \\
& =\log _a 2-\frac{1}{2}\left(x-\frac{1}{2}\right)^2+\frac{1}{8} \\
& \leqslant \log _a 2+\frac{1}{8} .
\end{aligned}
$$
%%PROBLEM_END%%



%%PROBLEM_BEGIN%%
%%<PROBLEM>%%
例11 已知 $f(x)=\log _2 x, g(x)=x^2-2 a x+5 a-1$.
(1) 若函数 $f[g(x)]$ 在区间 $[2,4]$ 上是单调函数,求 $a$ 的取值范围;
(2) 求函数 $g[f(x)]$ 在区间 $[2,4]$ 上的最小值 $h(a)$.
%%<SOLUTION>%%
解:(1) $f[g(x)]=\log _2\left(x^2-2 a x+5 a-1\right)$.
由 $f[g(x)]$ 在区间 $[2,4]$ 上是单调函数知, $g(x)=x^2-2 a x+5 a-1$ 在区间 $[2,4]$ 上是单调函数, 且 $g(x)>0$.
所以 $\left\{\begin{array}{l}a \leqslant 2, \\ g(2)=4-4 a+5 a-1>0\end{array}\right.$ 或 $\left\{\begin{array}{l}a \geqslant 4, \\ g(4)=16-8 a+5 a-1>0 .\end{array}\right.$
解得 $-3<a \leqslant 2$ 或 $4 \leqslant a<5$.
故 $a$ 的取值范围是 $(-3,2] \cup[4,5)$.
(2) $g[f(x)]=\left(\log _2 x\right)^2-2 a \log _2 x+5 a-1$.
设 $\log _2 x=t$, 则 $1 \leqslant t \leqslant 2, g[f(x)]=t^2-2 a t+5 a-1=(t-a)^2- a^2+5 a-1$.
若 $a<1$, 则当 $t=1$ 时, $g[f(x)]$ 取得最小值 $3 a$;
若 $1 \leqslant a \leqslant 2$, 则当 $t=a$ 时, $g[f(x)]$ 取得最小值 $-a^2+5 a-1$;
若 $a>2$, 则当 $t=2$ 时, $g[f(x)]$ 取得最小值 $a+3$.
%%PROBLEM_END%%



%%PROBLEM_BEGIN%%
%%<PROBLEM>%%
例12 设 $f(x)=\lg \frac{1+2^x+4^x a}{3}$, 其中 $a \in \mathbf{R}$. 当 $x \in(-\infty, 1]$ 时, $f(x)$有意义,求 $a$ 的取值范围.
%%<SOLUTION>%%
分析:首先需理解题意, 当 $x \in(-\infty, 1]$ 时, $f(x)$ 有意义, 说明当 $x \leqslant 1$ 时, 恒有 $\frac{1+2^x+4^x a}{3}>0$, 即
$$
a>-\left[\left(\frac{1}{4}\right)^x+\left(\frac{1}{2}\right)^x\right]
$$
对 $x \leqslant 1$ 恒成立, 故 $a$ 就大于 $-\left[\left(\frac{1}{4}\right)^x+\left(\frac{1}{2}\right)^x\right](x \in(-\infty, 1])$ 的最大值.
解由题意知, 当 $x \leqslant 1$ 时, 恒有
$$
\frac{1+2^x+4^x a}{3}>0 .
$$
故
$$
a>-\left[\left(\frac{1}{4}\right)^x+\left(\frac{1}{2}\right)^x\right], x \leqslant 1 .
$$
令 $u=-\left[\left(\frac{1}{4}\right)^x+\left(\frac{1}{2}\right)^x\right], x \leqslant 1$, 则
$$
u=-\left[\left(\frac{1}{2}\right)^x+\frac{1}{2}\right]^2+\frac{1}{4}
$$
在 $(-\infty, 1]$ 上单调递增,所以,当 $x=1$ 时,有
$$
u_{\max }=-\frac{3}{4} \text {. }
$$
所以 $a$ 的取值范围为 $\left(-\frac{3}{4},+\infty\right)$.
%%PROBLEM_END%%



%%PROBLEM_BEGIN%%
%%<PROBLEM>%%
例13 求函数 $f(x)=\frac{x-1}{x^2-2 x+5}, \frac{3}{2} \leqslant x \leqslant 2$ 的最大值和最小值.
%%<SOLUTION>%%
解:因为 $x \neq 1$, 所以
$$
f(x)=\frac{x-1}{(x-1)^2+4}=\frac{1}{x-1+\frac{4}{x-1}}, x \in\left[\frac{3}{2}, 2\right] .
$$
令 $g(t)=t+\frac{4}{t}, t \in\left[\frac{1}{2}, 1\right]$, 则 $g(t)$ 在 $\left[\frac{1}{2}, 1\right]$ 上是减函数, 所以
$$
\begin{gathered}
g_{\min }(t)=g(1)=5, \\
g_{\max }(t)=g\left(\frac{1}{2}\right)=\frac{17}{2} .
\end{gathered}
$$
所以, $f(x)$ 的最大值为 $f\left(\frac{3}{2}\right)=\frac{1}{5}, f(x)$ 的最小值为 $f(2)=\frac{2}{17}$.
%%PROBLEM_END%%



%%PROBLEM_BEGIN%%
%%<PROBLEM>%%
例14 设 $x, y \in \mathbf{R}^{+}, x+y=c, c$ 为常数且 $c \in(0,2]$, 求 $u=\left(x+\frac{1}{x}\right)\left(y+\frac{1}{y}\right)$ 的最小值.
%%<SOLUTION>%%
解:$u=\left(x+\frac{1}{x}\right)\left(y+\frac{1}{y}\right)=x y+\frac{1}{x y}+\frac{x}{y}+\frac{y}{x} \geqslant x y+\frac{1}{x y}+2$.
令 $x y=t$, 则 $0<t=x y \leqslant \frac{(x+y)^2}{4}=\frac{c^2}{4}$. 设
$$
f(t)=t+\frac{1}{t}, 0<t \leqslant \frac{c^2}{4} .
$$
由于 $0<c \leqslant 2$, 所以 $\frac{c^2}{4} \leqslant 1$. 于是函数 $f(t)$ 在 $\left(0, \frac{c^2}{4}\right]$ 上是单调递减的, 所以
$$
\begin{gathered}
f_{\min }(t)=f\left(\frac{c^2}{4}\right)=\frac{c^2}{4}+\frac{4}{c^2} . \\
u \geqslant \frac{c^2}{4}+\frac{4}{c^2}+2 .
\end{gathered}
$$
故当 $x=y=\frac{c}{2}$ 时, 等号成立.
所以 $u$ 的最小值为 $\frac{c^2}{4}+\frac{4}{c^2}+2$.
%%PROBLEM_END%%



%%PROBLEM_BEGIN%%
%%<PROBLEM>%%
例15 设 $x, y, z \in \mathbf{R}^{+}$, 且 $x+y+z=1$. 求
$$
u=\frac{3 x^2-x}{1+x^2}+\frac{3 y^2-y}{1+y^2}+\frac{3 z^2-z}{1+z^2}
$$
的最小值.
%%<SOLUTION>%%
分析:对于递增函数 $f(x), x \in D$, 若 $x_1, x_2 \in D$, 则
$$
\left(x_1-x_2\right)\left[f\left(x_1\right)-f\left(x_2\right)\right] \geqslant 0 .
$$
对于递减函数 $f(x), x \in D$, 若 $x_1, x_2 \in D$, 则
$$
\left(x_1-x_2\right)\left[f\left(x_2\right)-f\left(x_1\right)\right] \leqslant 0 .
$$
解令 $f(t)=\frac{t}{1+t^2}, t \in(0,+\infty)$, 则
$$
f(t)=\frac{1}{\frac{1}{t}+t} .
$$
由于 $t+\frac{1}{t}$ 在 $(0,1)$ 上递减, 所以 $f(t)$ 在 $(0,1)$ 上递增.
设 $x \in(0,1)$, 则
$$
\begin{gathered}
\left(x-\frac{1}{3}\right)\left[f(x)-f\left(\frac{1}{3}\right)\right] \geqslant 0, \\
\left(x-\frac{1}{3}\right)\left(\frac{x}{1+x^2}-\frac{3}{10}\right) \geqslant 0 .
\end{gathered}
$$
即
$$
\left(x-\frac{1}{3}\right)\left(\frac{x}{1+x^2}-\frac{3}{10}\right) \geqslant 0 \text {. }
$$
整理, 得
$$
\frac{3 x^2-x}{1+x^2} \geqslant \frac{9}{10}\left(x-\frac{1}{3}\right) \text {. }
$$
同理,得
$$
\frac{3 y^2-y}{1+y^2} \geqslant \frac{9}{10}\left(y-\frac{1}{3}\right), \frac{3 z^2-z}{1+z^2} \geqslant \frac{9}{10}\left(z-\frac{1}{3}\right) .
$$
所以
$$
u \geqslant \frac{9}{10}(x+y+z-1)=0 \text {, }
$$
当 $x==y=z=\frac{1}{3}$. 时等号成立.
故 $u$ 的最小值为 0 .
%%PROBLEM_END%%



%%PROBLEM_BEGIN%%
%%<PROBLEM>%%
例16 已知不等式
$$
\sqrt{2}(2 a+3) \cos \left(\theta-\frac{\pi}{4}\right)+\frac{6}{\sin \theta+\cos \theta}-2 \sin 2 \theta<3 a+6
$$
对于 $\theta \in\left[0, \frac{\pi}{2}\right]$ 恒成立,求实数 $a$ 的取值范围.
%%<SOLUTION>%%
分析:首先需把 $a$ 解出来, 可得 $a>g(\theta)$ (或 $a<g(\theta)$ ), 于是 $a>g_{\max }(\theta)$ (或 $a<g_{\min }(\theta)$ ). 为了方便起见, 令 $x=\sin \theta+\cos \theta$, 可以化简.
解令 $x=\sin \theta+\cos \theta$, 由于 $\theta \in\left[0, \frac{\pi}{2}\right]$, 则 $x \in[1, \sqrt{2}]$, 且
$$
\begin{aligned}
\cos \left(\theta-\frac{\pi}{4}\right) & =\cos \theta \cos \frac{\pi}{4}+\sin \theta \sin \frac{\pi}{4}=\frac{\sqrt{2}}{2} x, \\
& \sin 2 \theta=2 \sin \theta \cos \theta=(\sin \theta+\cos \theta)^2-1=x^2-1 .
\end{aligned}
$$
从而原不等式可化为
$$
\begin{gathered}
(2 a+3) x+\frac{6}{x}-2\left(x^2-1\right)<3 a+6, \\
2 x^2-2 a x-3 x-\frac{6}{x}+3 a+4>0,
\end{gathered}
$$
$$
2 x\left(x+\frac{2}{x}-a\right)-3\left(x+\frac{2}{x}-a\right)>0,
$$
即
$$
(2 x-3)\left(x+\frac{2}{x}-a\right)>0 \text {. }
$$
因为 $x \in[1, \sqrt{2}]$, 所以 $2 x-3<0$, 故
$$
x+\frac{2}{x}-a<0 \text {. }
$$
所以 $a>x+\frac{2}{x}, x \in[1, \sqrt{2}]$ 恒成立.
令 $f(x)=x+\frac{2}{x}, x \in[1, \sqrt{2}]$, 由于 $f(x)$ 在 $[1, \sqrt{2}]$ 上单调递减, 所以
$$
\begin{gathered}
f_{\max }(x)=f(1)=3, \\
a>f_{\max }(x)=3 .
\end{gathered}
$$
所以 $a$ 的取值范围为 $(3,+\infty)$.
%%PROBLEM_END%%


